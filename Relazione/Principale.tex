% !TEX program = pdflatex % Usa il compilatore «PDFLaTeX»

\documentclass[
    a4paper,
    twoside,
    openright,
    titlepage,
    headinclude,
    footinclude,
    BCOR5mm,
    numbers=noenddot,
    cleardoublepage=empty,
    tablecaptionabove
]{scrreprt}

\usepackage[T1]{fontenc}
\usepackage[utf8]{inputenc} % if pdfLaTeX
\usepackage[italian]{babel}
\usepackage{microtype}
\usepackage{subfig}

\usepackage[
    eulerchapternumbers,
    subfig,
    beramono,
    eulermath,
    pdfspacing
]{classicthesis}
\usepackage{arsclassica}

% Pacchetti per vari comandi matematici
\usepackage{amsmath,
            amsfonts,
            amsthm,
            amssymb}

\newcommand\derS[3][d]{\frac{\mathrm{#1}#2}{\mathrm{#1}#3}} % Derivata sostanziale
% \usepackage[hidelinks]{hyperref}

\begin{document}
    Si consideri la distribuzione totale
    %
    \begin{equation}
        % \label{eqDistribuzioneTotale}
        f(x,v,t)=\frac1N\sum_{i\in\mathcal I}f_i(v,t)\otimes\delta(x-i),
    \end{equation}
    %
    se le particelle sono \textit{indistinguibili} (1° ipotesi) allora vale
    %
    \begin{equation}
        f_i(v,t)=f_j(v,t)=f(v,t)\quad\forall i,j\in\mathcal I,
    \end{equation}
    %
    da cui la distribuzione marginale
    %
    \begin{equation}
        \begin{aligned}
            \int_{\mathcal I}f(x,v,t)dx&=\frac1N\sum_{i\in\mathcal I}\int_{\mathcal I}f_i(v,t)\otimes\delta(x-i)dx\\
            &=\frac1Nf(v,t)\otimes\sum_{i\in\mathcal I}\int_{\mathcal I}\delta(x-i)dx\\
            &=\frac1Nf(v,t)\otimes\sum_{i\in\mathcal I}1\\
            &=\frac1Nf(v,t)N=f(v,t),
        \end{aligned}
    \end{equation}
    %
    e ciò è anche coerente col fatto che in un gas il numero di particelle è molto elevato.

    Considerando una \textit{matrice d'adiacenza unitaria} (2° ipotesi) si può scrivere la (2.10) come
    %
    \begin{equation}
        \begin{aligned}
            \derS{}t\int_{\mathcal O}\varphi(v)f(v,t)dv&=
            \frac1{2N}\sum_{j\in\mathcal I}\int_{\mathcal O^{2}}\langle\varphi(v^{'})-\varphi(v)\rangle f(v,t)f(v_*,t)dvdv_*\\
            &+\frac1{2N}\sum_{j\in\mathcal I}\int_{\mathcal O^{2}}\langle\varphi(v_*^{'})-\varphi(v_*)\rangle f(v,t)f(v_*,t)dvdv_*\\
            &=\frac1{2N}N\int_{\mathcal O^{2}}\langle\varphi(v^{'})-\varphi(v)\rangle f(v,t)f(v_*,t)dvdv_*\\
            &+\frac1{2N}N\int_{\mathcal O^{2}}\langle\varphi(v_*^{'})-\varphi(v_*)\rangle f(v,t)f(v_*,t)dvdv_*\\
            &=\frac12\int_{\mathcal O^{2}}\langle\varphi(v^{'})-\varphi(v)\rangle f(v,t)f(v_*,t)dvdv_*\\
            &+\frac12\int_{\mathcal O^{2}}\langle\varphi(v_*^{'})-\varphi(v_*)\rangle f(v,t)f(v_*,t)dvdv_*,
        \end{aligned}
    \end{equation}
    
    che unita all'ipotesi d'\textit{interazioni simmetriche} (3° ipotesi), ossia tali che 
    %
    \begin{equation}
        v^{'}=\Psi(v,v_*)=\Psi_*(v_*,v)\quad\text{ove }v_*=\Psi_*(v,v_*),
    \end{equation}
    %
    porta all'equivalenza (mediante il cambio di variabili \(v_*=v\) e \(v=v_*\))
    %
    \begin{equation}
        \int_{\mathcal O^{2}}\langle\varphi(v_*^{'})-\varphi(v_*)\rangle f(v,t)f(v_*,t)dvdv_*
        =\int_{\mathcal O^{2}}\langle\varphi(v^{'})-\varphi(v)\rangle f(v,t)f(v_*,t)dvdv_*,
    \end{equation}
    %
    e quindi a
    %
    \begin{equation}
        \derS{}t\int_{\mathcal O}\varphi(v)f(v,t)dv=
        \int_{\mathcal O^{2}}\langle\varphi(v^{'})-\varphi(v)\rangle f(v,t)f(v_*,t)dvdv_*,
    \end{equation}
    %
    che equivale alla formula classica di Boltzmann.
\end{document}