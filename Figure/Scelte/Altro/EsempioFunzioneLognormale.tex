
% !TEX root = ../../../Esperimenti/.tex/MEF.tex
% LTeX: language=it

\begin{tikzpicture}%
    % La lognormale è «f(x)=(1/(x*sigma*sqrt(2*pi)))*exp(-(ln(x)-mu)^2/(2*sigma^2))»

    \newcommand\muPrm{0}    % Media della normale sottostante
    \newcommand\sigmaPrm{1} % Deviazione standard della normale sottostante

    % Definizione della funzione matematica pgf
    \pgfmathdeclarefunction{lognormalpdf}{1}{%
        \pgfmathparse{1/(#1*\sigmaPrm*sqrt(2*pi))*exp(-((ln(#1)-\muPrm)^2)/(2*\sigmaPrm*\sigmaPrm))}%
    }

    \newcommand\captionSize{12}
    \newcommand\labelSize{10}
    \newcommand\tickSize{10}

    \pgfmathsetmacro\hsep{1cm}
    % \pgfmathsetmacro\widthPlot{(\linewidth-\hsep)/2}
    \pgfmathsetmacro\widthPlot{7cm}
    \pgfmathsetmacro\heightPlot{.75*\widthPlot}

    \begin{groupplot}[
        group style={
            group name=lgnrml,
            group size=2 by 1,
            horizontal sep=\hsep,
            xlabels at=edge bottom,
            ylabels at=edge left,
            x descriptions at=edge bottom,
            y descriptions at=edge left
        },
        width=\widthPlot,
        height=\heightPlot,
        axis x line*=bottom,
        axis y line*=left,
        xlabel={$x$},
        ylabel={$f_X(x)$},
        xlabel style={font=\color{white!15!black}\dimTesto{\labelSize}},
        ylabel style={font=\color{white!15!black}\dimTesto{\labelSize}},
        xticklabel style={font=\dimTesto{\tickSize}},
        yticklabel style={font=\dimTesto{\tickSize}},
        grid=major,
        grid style={dashed},
        every axis plot/.append style={
            % ultra thick,
            thick,
            smooth
        },
        ymin=0,
    ]
        % Scala lineare
        \nextgroupplot[
            xmin=0,
            xmax=10, %5
            minor tick num=1,
        ]
        \addplot[
            % color=blue,
            domain=0.01:20,
            samples=400
        ] {lognormalpdf(x)};

        % Scala semilogaritmica
        \nextgroupplot[
            xmode=log,
            xmin=0.01,
            xmax=20,
            % log ticks with fixed point, % Mostra i numeri (0.1,1,10) invece di (10^-1,10^0,10^1)
        ]
        % Nota: il dominio è molto più ampio qui per mostrare le code
        \addplot[
            % color=red,
            domain=0.005:25,
            samples=500
        ] {lognormalpdf(x)};
    \end{groupplot}

    \tikzset{SubCaption/.style={
        text width=\widthPlot,
        anchor=north,
        align=center,
        yshift=-2.5em
    }}
    \node[SubCaption] at (lgnrml c1r1.south)
        {\dimTesto{\captionSize}\sottoDidascalia{Scala lineare.\label{figLognormaleScalaLineare}}};
    \node[SubCaption] at (lgnrml c2r1.south)
        {\dimTesto{\captionSize}\sottoDidascalia{Scala semilogaritmica.\label{figLognormaleScalaSemiLogaritmica}}};

    \redefineTikZbounds{lgnrml c1r1}{lgnrml c2r1}
\end{tikzpicture}%