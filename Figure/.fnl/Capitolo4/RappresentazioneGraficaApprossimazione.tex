
% !TEX root = ../../../Esperimenti/.tex/MEF.tex
% LTeX: language=it

%\begingroup Impostazioni
    %\begingroup Parametri
        \pgfmathsetmacro\xPlotSep{4}
        \pgfmathsetmacro\yPlotSep{4}
        \pgfmathsetmacro\plotScale{.8}

        \pgfmathsetmacro\titleSize{11}
        \pgfmathsetmacro\commentSize{10}

        \pgfmathsetmacro\nodeSize{20pt} % Diametro del nodo
        \pgfmathsetmacro\starSize{2.25*\nodeSize}
        \pgfmathsetmacro\numberSize{.53*\nodeSize}
    %\endgroup

    %\begingroup Colori
        \definecolor{cRed}{HTML}{D68490}
        \definecolor{cPurple}{HTML}{7E6EAC}
        \definecolor{cTeal}{HTML}{87BDB1}
        \definecolor{cOrange}{HTML}{F2D0A4}
        \definecolor{cEdgeLight}{HTML}{EEE8D5} 
        \definecolor{cEdgeDark}{HTML}{998888}
        \definecolor{cStar}{HTML}{FDF6E3}
        %
        \definecolor{color1}{HTML}{800000} % Deep Red
        \definecolor{color2}{HTML}{4682B4} % Steel Blue
        \definecolor{color3}{HTML}{6B8E23} % Olive Green
        \definecolor{color4}{HTML}{708090} % Slate Gray
        \definecolor{color5}{HTML}{BC8F8F} % Rosy Brown
        \definecolor{color6}{HTML}{008B8B} % Dark Cyan
        \definecolor{color7}{HTML}{B8860B} % Dark Goldenrod
    %\endgroup

    %\begingroup Frecce
    \tikzset{
        transformingArrow/.style={
            % >=Latex,
            >=Stealth,
            ->,
            dashed,
            line width=.8pt,
            black!90,
            shorten >=5pt,
            shorten <=10pt
        },
        nodeArrowExact/.style={
            % ->,
            % -{Stealth[scale=1.2]},
            -{Triangle[scale=.75]},
            % draw=cEdgeLight,
            % draw=red!50,
            draw=black!95,
            line width=.5pt,
            % shorten >=2pt,
            % shorten <=2pt
        },
        nodeArrowApprox/.style={
            -,
            draw=black!95,
            line width=.5pt,
            dashed,
        },
        transparentNodeArrow/.style={
            nodeArrowApprox,
            opacity=.4,
        },
    }
    %\endgroup

    %\begingroup Nodi
        \tikzset{
            nodeStyle/.style={
                circle,
                minimum size=\nodeSize,
                % draw=black!80,
                draw=black,
                line width=.5pt,
                inner sep=0pt,
            },
            starStyle/.style={
                star,
                star points=5,
                star point ratio=2,
                % fill=cStar,
                fill=black!50,
                fill opacity=.25,
                minimum size=\starSize,
                draw=none
            },
        }
    %\endgroup

    %\begingroup Testo
    \tikzset{
        titleStyle/.style={
            font=\dimTesto{\titleSize},
            align=center,
            anchor=south,
        },
        commentStyle/.style={font=\dimTesto{\commentSize}},
        arrowCommentStyle/.style={
            midway,
            commentStyle,
            left,
            sloped,
            below
        },
        numberStyle/.style={font=\dimTesto{\numberSize}}
    }
    %\endgroup

    %\begingroup Altro
        % %\begingroup Selezione «recto»/«verso»
        % \Ifthispageodd{%
        %     \pgfplotsset{/pgfplots/group/yDescPos/.style={y descriptions at=edge right}}
        % }{%
        %     \pgfplotsset{/pgfplots/group/yDescPos/.style={y descriptions at=edge left}}
        % }
        % %
        % \Ifthispageodd{%
        %     \pgfplotsset{
        %         yLineAxis/.style={axis y line*=right},
        %         /tikz/yLabelShift/.style={
        %             yshift=-2.5em,
        %             at={(axis description cs:1,0.5)},
        %             anchor=south,
        %             % rotate=180,
        %         },
        %         /tikz/yScaleLabelShift/.style={
        %             yshift=.25em,
        %             at={(1,1)},
        %             anchor=south east,
        %         },
        %         /tikz/yTickLabelShift/.style={xshift=-0.15em},
        %     }
        % }{%
        %     \pgfplotsset{
        %         yLineAxis/.style={axis y line*=left},
        %         /tikz/yLabelShift/.style={
        %             yshift=1.25em,
        %             at={(axis description cs:0,0.5)},
        %             anchor=south,
        %             % rotate=180,
        %         },
        %         /tikz/yScaleLabelShift/.style={
        %             yshift=.25em,
        %             at={(0,1)},
        %             anchor=south west,
        %         },
        %         /tikz/yTickLabelShift/.style={xshift=+0.15em},
        %     }
        % }
        % %\endgroup
        % %
        % %\begingroup Gruppi
        % \pgfplotsset{
        %     plotRow/.style={
        %         scale only axis,
        %         width=\plotWidth cm,height=\plotHeight cm,
        %         xlabel style={font=\color{white!15!black}\dimTesto{\labelSize}},
        %         ylabel style={font=\color{white!15!black}\dimTesto{\labelSize},yLabelShift},
        %         xticklabel style={font=\dimTesto{\tickSize}},
        %         yticklabel style={font=\dimTesto{\tickSize},yTickLabelShift},
        %         xlabel={\(s\)},ylabel={\(\underhat{\bar f}_N(s)\)},
        %         % axis x line*=bottom,yLineAxis,
        %         enlarge x limits=0.1,enlarge y limits=0.1,
        %         xmajorgrids,ymajorgrids,
        %         xminorticks=true,yminorticks=true,
        %         axis background/.style={fill=white},
        %         grid style={dashed}
        %     },
        % }
        % %\endgroup
        % %
        % %\begingroup Estetica
        % \newcommand\smallminus{\scalebox{0.4}[1]{$-$}\hspace{-.125em}}
        % \pgfplotsset{
        %     /pgfplots/log number format basis/.code 2 args={%
        %         \pgfmathsetmacro\exponent{#2}%
        %         \pgfmathparse{sign(\exponent)}%
        %         \ifnum\pgfmathresult<0%
        %             \pgfmathparse{abs(\exponent)}%
        %             $#1^{\smash{\smallminus\pgfmathprintnumber{\pgfmathresult}}}$%
        %         \else%
        %             $#1^{\smash{\pgfmathprintnumber{\exponent}}}$%
        %         \fi%
        %     },
        %     ytick scale label code/.code={%
        %         \pgfmathsetmacro\exponent{#1}%
        %         \pgfmathparse{sign(\exponent)}%
        %         \ifnum\pgfmathresult<0
        %             \pgfmathparse{abs(\exponent)}%
        %             $\cdot 10^{\smash{\smallminus\pgfmathprintnumber{\pgfmathresult}}}$%
        %         \else
        %             $\cdot 10^{\smash{\pgfmathprintnumber{\exponent}}}$%
        %         \fi
        %     },
        % }
        % %\endgroup
    %\endgroup
%\endgroup

\begin{tikzpicture}%
    %\begingroup Coordinate dei vertici
        \coordinate (O) at (0,0);

        \newcommand\vertexCoordinates{
            \coordinate (v1) at (0,0);
            \coordinate (v2) at (2,2);
            \coordinate (v3) at (-1,2.5);
            \coordinate (v4) at (1.5,4.0);
            \coordinate (v5) at (3.0,-0.5);
            \coordinate (v6) at (4,3);
            \coordinate (v7) at (5.0,0.5);
        }

        % \newcommand\vertexCoordinates{
        %     \coordinate (v1) at (0,0);
        %     \coordinate (v2) at (2.5,1.5);
        %     \coordinate (v3) at (-1,2.5);
        %     \coordinate (v4) at (1.5,4.0);
        %     \coordinate (v5) at (3.0,-0.5);
        %     \coordinate (v6) at (4.5,2.8);
        %     \coordinate (v7) at (5.0,0.5);
        % }

        \newcommand\drawEmptyNodes[1]{\vertexCoordinates
            \foreach \i in {1,2,...,7}
                \node[
                    nodeStyle,
                    numberStyle,
                    fill=white
                ] (#1\i) at (v\i) {\i};

            \foreach \i in {1,2}
                \node[starStyle,rotate=\i*45] at (v\i) {};
        }

        \newcommand\drawUniformedNodes[2]{\vertexCoordinates
            \foreach \i in {1,2,...,7}
                \node[nodeStyle,fill=#2] (#1\i) at (v\i) {};

            \begin{scope}[on background layer]
                \foreach \i in {1,2}
                    \node[
                        starStyle,
                        % fill=cStar,
                        % fill opacity=0.9,
                        rotate=\i*45
                    ] at (v\i) {};
            \end{scope}
        }

        \newcommand\drawColoredNodes[1]{\vertexCoordinates
            \foreach \i in {1,2,...,7}
                \node[nodeStyle,fill=color\i] (#1\i) at (v\i) {};

            \begin{scope}[on background layer]
                \foreach \i in {1,2}
                    \node[
                        starStyle,
                        fill=cStar,
                        % fill opacity=0.9,
                        rotate=\i*45
                    ] at (v\i) {};
            \end{scope}
        }

        % \newcommand\drawColoredNodes[1]{\vertexCoordinates
        %     \node[nodeStyle,fill=cTeal] (#11) at (v1) {};
        %     \node[nodeStyle,fill=cPurple] (#12) at (v2) {};
        %     \node[nodeStyle,fill=cRed] (#13) at (v3) {};
        %     \node[nodeStyle,fill=cOrange] (#14) at (v4) {};
        %     \node[nodeStyle,fill=cRed] (#15) at (v5) {};
        %     \node[nodeStyle,fill=cTeal] (#16) at (v6) {};
        %     \node[nodeStyle,fill=cPurple] (#17) at (v7) {};
        % }
    %\endgroup

    %\begingroup Lati del grafo completo
        \newcommand\drawCompleteEdges[1]{
            \begin{scope}[on background layer]
                \foreach \i [evaluate=\j as \s using int(\i+1)] in {1,...,6}
                    \foreach \j in {\s,...,7}{
                        \ifcase\i % Caso 0
                        \or % Caso 1
                            \ifnum\j=2\def\b{15}
                            \else\ifnum\j=5\def\b{-20}
                            \else\ifnum\j=6\def\b{-10}
                            \else\ifnum\j=7\def\b{-10}
                            \else\def\b{20}
                            \fi\fi\fi\fi
                        \or % Caso 2
                            \ifnum\j=3\def\b{-10}
                            \else\ifnum\j=4\def\b{0}
                            \else\ifnum\j=5\def\b{-15}
                            \else\ifnum\j=6\def\b{0}
                            \else\ifnum\j=7\def\b{10}
                            \else\def\b{20}
                            \fi\fi\fi\fi\fi
                        \or % Caso 3
                            \ifnum\j=5\def\b{-20}
                            \else\ifnum\j=7\def\b{-10}
                            \else\def\b{20}
                            \fi\fi
                        \or % Caso 4
                            \def\b{20}
                        \or % Caso 5
                            \def\b{-20}
                        \or % Caso 6
                            \def\b{20}
                        \else % Caso 7
                            \def\b{20}
                        \fi
                        \ifnum\i=1
                            \draw[nodeArrowApprox] (#1\i) to[bend left=\b] (#1\j);
                        \else
                            \draw[nodeArrowApprox,transparentNodeArrow] (#1\i) to[bend left=\b] (#1\j);
                        \fi
                    }
            \end{scope}
        }
    %\endgroup

    \begin{scope}[
        local bounding box=tikZFigure,
        scale=\plotScale,
        transform shape,
    ]
        %\begingroup Grafo esatto
            \tikzbox{{(0,0)}}{ExactGraph}{
                % \drawColoredNodes{e}
                % \drawUniformedNodes{e}{none}
                \drawEmptyNodes{e}

                \begin{scope}[on background layer]
                    \foreach \i/\j/\bi/\bo/\t in {
                        1/2/20/20/0,
                        2/4/20/20/1,
                        2/5/20/20/1,
                        2/6/20/20/1,
                        3/4/20/20/1,
                        % 3/6/20/15/1,
                        5/7/20/20/1,
                        6/7/20/20/1
                    }{
                        \ifnum\t=0
                            \draw[nodeArrowExact] (e\i) to[bend left=\bi] (e\j);
                            \draw[nodeArrowExact] (e\j) to[bend left=\bo] (e\i);
                        \else
                            \draw[nodeArrowExact,transparentNodeArrow] (e\i) to[bend left=\bi] (e\j);
                            \draw[nodeArrowExact,transparentNodeArrow] (e\j) to[bend left=\bo] (e\i);
                        \fi
                    }
                \end{scope}
            }
        %\endgroup

        %\begingroup Grafo approssimato
            \tikzbox[north]{{
                ($(ExactGraph.south east)
                +(.5*\xPlotSep em,-\yPlotSep em)$)
            }}{ApproxGraph}{
                % \drawColoredNodes{a}
                % \drawUniformedNodes{a}{none}
                \drawEmptyNodes{a}
                \drawCompleteEdges{a}
            }
        %\endgroup

        %\begingroup Grafo rilassato
            \tikzbox{{
                ($(ExactGraph.south east)
                +(\xPlotSep em,0)$)
            }}{RelaxedGraph}{
                % \drawColoredNodes{r}
                \drawUniformedNodes{r}{none}
                % \drawEmptyNodes{r}
                \drawCompleteEdges{r}
            }
        %\endgroup

        % %\begingroup Testo
            \node[titleStyle] at (ExactGraph.north) {Topologia esatta};
            \node[titleStyle] at (ApproxGraph.north) {Topologia approssimata};
            \node[titleStyle] at (RelaxedGraph.north) {Topologia rilassata};
            
            \node[commentStyle,below=0.5em of e1] {$(i_1,s_1)$};
            \node[commentStyle,below=0.5em of a1] {$(i_1,s_1)$};
            \node[commentStyle,below=0.5em of r1] {$(k,s)$};
        % %\endgroup

        %\begingroup Frecce
            \draw[transformingArrow] 
                ($(ExactGraph.south west)!0.2!(ExactGraph.south east)$) 
                to[bend right=40] node[arrowCommentStyle] {\cref{defMatriceGradi}} (ApproxGraph.west);

            \draw[transformingArrow]
                (ApproxGraph.east) to[bend right=40] node[arrowCommentStyle] {\cref{teoRilassamentoTopologia}}
                ($(RelaxedGraph.south west)!0.8!(RelaxedGraph.south east)$);

            % \draw[transformingArrow] (ExactGraph.south) to[bend right=40] (ApproxGraph.west);
            % \draw[transformingArrow] (ApproxGraph.east) to[bend right=40] (RelaxedGraph.south);

            % \draw[transformingArrow] (ExactGraph.south) |- (ApproxGraph.west);
            % \draw[transformingArrow] (ApproxGraph.east) -| (RelaxedGraph.south);
        %\endgroup

        %\begingroup Scatole di delimitazione
            % \draw[red,dashed] (ExactGraph.south west) rectangle (ExactGraph.north east);
            % \draw[blue,dashed] (ApproxGraph.south west) rectangle (ApproxGraph.north east);
            % \draw[green,dashed] (RelaxedGraph.south west) rectangle (RelaxedGraph.north east);
        %\endgroup
    \end{scope}

    %\begingroup Riferimento
        % \begin{pgfonlayer}{foreground}
        %     \node[commentStyle] (origine) at (0,0) {origine};
    
        %     \node[commentStyle] (NW) at (ExactGraph.north west) {NW};
        %     \node[commentStyle] (NE) at (ExactGraph.north east) {NE};
        %     \node[commentStyle] (SW) at (ExactGraph.south west) {SW};
        %     \node[commentStyle] (SE) at (ExactGraph.south east) {SE};
        % \end{pgfonlayer}
    
        % \draw[style=help lines] (tikZFigure.north west) grid[step=1em] (tikZFigure.south east);
    %\endgroup
\end{tikzpicture}
