
% !TEX root = ../../../Esperimenti/.tex/MEF.tex
% LTeX: language=it

\begin{tikzpicture}%
    %\begingroup Comandi
        \definecolor{colore}{rgb}{0.50196,0.50196,0.50196}%
    
        % \renewcommand\subCaptionSize{10}
        % \renewcommand\labelSize{10}
        % \renewcommand\tickSize{9}
        % \renewcommand\legendSize{9}

        \pgfmathsetmacro\plotWidth{7}
        \pgfmathsetmacro\plotHeight{.618*\plotWidth}
        \pgfmathsetmacro\halfHeightPlot{\plotHeight/2}
        \pgfmathsetmacro\lineWidth{.5pt}
        \pgfmathsetmacro\lineMidWidth{.75pt}
        \pgfmathsetmacro\lineThickWidth{1.2pt}
        
        \pgfmathsetmacro\xPlotSep{2.5}
        \pgfmathsetmacro\yPlotSep{5.5}
        \pgfmathsetmacro\yPlotSepPartial{.8*\yPlotSep}
        \pgfmathsetmacro\xLegendSep{.25cm}
    %\endgroup

    %\begingroup Impostazioni pgfplots - Prima colonna
        \pgfplotsset{
            /pgfplots/group/plotGroup/.style={
                group name=Col1,
                group size=1 by 5,
                vertical sep=\yPlotSep em,
                % x descriptions at=edge bottom,
                % y descriptions at=edge left,
            },
            plotColumn/.style={
                width=\plotWidth cm,
                height=\plotHeight cm,
                scale only axis,
                xlabel style={
                    font=\color{white!15!black}\dimTesto{\labelSize},
                },
                ylabel style={
                    font=\color{white!15!black}\dimTesto{\labelSize},
                },
                xticklabel style={font=\dimTesto{\tickSize}},
                yticklabel style={font=\dimTesto{\tickSize}},
                % axis background/.style={fill=white},
                axis x line*=bottom,
                axis y line*=left,
                xmajorgrids,
                ymajorgrids,
                grid style={dashed}
            },
            plotLegend/.style={
                font=\dimTesto{\legendSize},
                legend cell align=left,
                align=left,
                legend pos=north east,
                xshift=\xLegendSep,
                draw=white!15!black,
            },
        }
    %\endgroup

    \begin{groupplot}[group style=plotGroup,plotColumn]
        % This file was created by matlab2tikz.
%
\nextgroupplot[
    % title style={font=\dimTesto{\titleSize}},
    % title={Prova},
    % width=4.521in,
    % height=3.566in,
    % at={(0.758in,0.481in)},
    scale only axis,
    % xmin=-3.65000000000001,
    xmin=0,
    xmax=300,
    xlabel style={font=\color{white!15!black}},
    xlabel={$k$},
    ymin=0,
    ymax=0.03,
    ylabel style={font=\color{white!15!black}},
    ylabel={$P(k)$},
    axis background/.style={fill=white},
    % axis x line*=bottom,
    % axis y line*=left,
    xmajorgrids,
    ymajorgrids,
    grid style={dashed},
    legend style=plotLegend,
    % legend style={
    %     legend cell align=left,
    %     align=left,
    %     draw=white!15!black
    % },
]
    \addplot[
        ybar interval,
        fill=colore,
        area legend,
        draw=none
    ] table[
        row sep=crcr,
        x=Lower,
        y=Count
    ] {%
    Lower	Upper	Count\\
    10	20.92	0.00927960927960928\\
    20.92	31.84	0.0212454212454212\\
    31.84	42.76	0.0251526251526252\\
    42.76	53.68	0.0151404151404151\\
    53.68	64.6	0.00537240537240538\\
    64.6	75.52	0.00390720390720391\\
    75.52	86.44	0.00268620268620269\\
    86.44	97.36	0.00317460317460317\\
    97.36	108.28	0.00146520146520146\\
    108.28	119.2	0.000976800976800977\\
    119.2	130.12	0.000976800976800977\\
    130.12	141.04	0.000732600732600733\\
    141.04	151.96	0.000244200244200244\\
    151.96	162.88	0.000244200244200244\\
    162.88	173.8	0\\
    173.8	184.72	0.000244200244200244\\
    184.72	195.64	0\\
    195.64	206.56	0.000488400488400488\\
    206.56	217.48	0\\
    217.48	228.4	0\\
    228.4	239.32	0\\
    239.32	250.24	0\\
    250.24	261.16	0\\
    261.16	272.08	0\\
    272.08	283	0.000244200244200244\\
    283	283	0.000244200244200244\\
    };
    \addlegendentry{Istogramma}

    \addplot [color=black]
    table[row sep=crcr]{%
    10	0.00293376181378972\\
    10.5470941883768	0.00355750249447913\\
    11.0941883767535	0.00423360895998364\\
    11.6412825651303	0.00495554107665914\\
    12.188376753507	0.0057161768479886\\
    12.7354709418838	0.00650807002096127\\
    13.2825651302605	0.00732367401393397\\
    13.8296593186373	0.00815553005334939\\
    14.376753507014	0.0089964199162941\\
    14.9238476953908	0.00983948533947425\\
    15.4709418837675	0.0106783171626301\\
    16.0180360721443	0.0115070177933485\\
    16.565130260521	0.0123202407500482\\
    17.1122244488978	0.0131132109715274\\
    17.6593186372745	0.0138817293600731\\
    18.2064128256513	0.0146221647141141\\
    18.7535070140281	0.0153314358514356\\
    19.3006012024048	0.0160069863568422\\
    19.8476953907816	0.0166467540300424\\
    20.3947895791583	0.017249136773729\\
    20.9418837675351	0.017812956355913\\
    21.4889779559118	0.0183374212080772\\
    22.0360721442886	0.018822089182422\\
    22.5831663326653	0.019266830986348\\
    23.1302605210421	0.019671794838215\\
    23.6773547094188	0.0200373727425699\\
    24.2244488977956	0.0203641686624064\\
    24.7715430861723	0.0206529687675336\\
    25.3186372745491	0.0209047138588067\\
    25.8657314629259	0.0211204740050476\\
    26.4128256513026	0.0213014253804247\\
    26.9599198396794	0.0214488292526203\\
    27.5070140280561	0.0215640130442982\\
    28.0541082164329	0.0216483533704807\\
    28.6012024048096	0.0217032609409554\\
    29.1482965931864	0.0217301672085133\\
    29.6953907815631	0.0217305126395896\\
    30.2424849699399	0.021705736482854\\
    30.7895791583166	0.0216572679127372\\
    31.3366733466934	0.0215865184281694\\
    31.8837675350701	0.0214948753914493\\
    32.4308617234469	0.0213836965977531\\
    32.9779559118236	0.0212543057720069\\
    33.5250501002004	0.021107988896423\\
    34.0721442885771	0.0209459912787428\\
    34.6192384769539	0.0207695152779714\\
    35.1663326653307	0.0205797186110216\\
    35.7134268537074	0.020377713170112\\
    36.2605210420842	0.0201645642869268\\
    36.8076152304609	0.019941290385392\\
    37.3547094188377	0.0197088629704415\\
    37.9018036072144	0.0194682069053028\\
    38.4488977955912	0.019220200934636\\
    38.9959919839679	0.018965678415304\\
    39.5430861723447	0.0187054282206517\\
    40.0901803607214	0.0184401957879286\\
    40.6372745490982	0.0181706842819329\\
    41.1843687374749	0.0178975558510833\\
    41.7314629258517	0.0176214329549776\\
    42.2785571142285	0.0173428997450696\\
    42.8256513026052	0.0170625034824291\\
    43.372745490982	0.0167807559786421\\
    43.9198396793587	0.0164981350477976\\
    44.4669338677355	0.0162150859591888\\
    45.0140280561122	0.0159320228818702\\
    45.561122244489	0.015649330313552\\
    46.1082164328657	0.0153673644875122\\
    46.6553106212425	0.0150864547522614\\
    47.2024048096192	0.0148069049196376\\
    47.749498997996	0.0145289945778295\\
    48.2965931863727	0.0142529803665555\\
    48.8436873747495	0.0139790972122597\\
    49.3907815631263	0.0137075595217438\\
    49.937875751503	0.0134385623331319\\
    50.4849699398798	0.0131722824234889\\
    51.0320641282565	0.0129088793727688\\
    51.5791583166333	0.0126484965840788\\
    52.12625250501	0.0123912622605108\\
    52.6733466933868	0.0121372903390099\\
    53.2204408817635	0.0118866813819393\\
    53.7675350701403	0.0116395234271544\\
    54.314629258517	0.0113958927975256\\
    54.8617234468938	0.0111558548709518\\
    55.4088176352705	0.0109194648119883\\
    55.9559118236473	0.0106867682662725\\
    56.503006012024	0.0104578020189804\\
    57.0501002004008	0.0102325946185732\\
    57.5971943887776	0.0100111669671184\\
    58.1442885771543	0.00979353287847373\\
    58.6913827655311	0.00957969960562572\\
    59.2384769539078	0.00936966833846407\\
    59.7855711422846	0.00916343467326103\\
    60.3326653306613	0.00896098905510394\\
    60.8797595190381	0.00876231719450554\\
    61.4268537074148	0.00856740045938896\\
    61.9739478957916	0.00837621624361331\\
    62.5210420841683	0.00818873831317316\\
    63.0681362725451	0.00800493713117022\\
    63.6152304609218	0.00782478016261966\\
    64.1623246492986	0.00764823216011622\\
    64.7094188376753	0.00747525543134844\\
    65.2565130260521	0.00730581008941083\\
    65.8036072144289	0.00713985428682687\\
    66.3507014028056	0.00697734443415777\\
    66.8977955911824	0.0068182354040352\\
    67.4448897795591	0.00666248072141974\\
    67.9919839679359	0.00651003274085098\\
    68.5390781563126	0.0063608428114204\\
    69.0861723446894	0.00621486143016416\\
    69.6332665330661	0.00607203838453997\\
    70.1803607214429	0.00593232288461991\\
    70.7274549098196	0.00579566368560042\\
    71.2745490981964	0.00566200920120061\\
    71.8216432865731	0.00553130760849127\\
    72.3687374749499	0.00540350694466921\\
    72.9158316633267	0.00527855519626493\\
    73.4629258517034	0.00515640038124609\\
    74.0100200400802	0.00503699062445489\\
    74.5571142284569	0.00492027422679375\\
    75.1042084168337	0.00480619972855155\\
    75.6513026052104	0.0046947159672414\\
    76.1983967935872	0.00458577213030005\\
    76.7454909819639	0.0044793178029802\\
    77.2925851703407	0.00437530301174793\\
    77.8396793587174	0.00427367826348035\\
    78.3867735470942	0.00417439458074129\\
    78.9338677354709	0.00407740353339739\\
    79.4809619238477	0.00398265726682154\\
    80.0280561122244	0.00389010852691643\\
    80.5751503006012	0.00379971068217719\\
    81.122244488978	0.00371141774299945\\
    81.6693386773547	0.00362518437842671\\
    82.2164328657315	0.0035409659305193\\
    82.7635270541082	0.00345871842651662\\
    83.310621242485	0.0033783985889534\\
    83.8577154308617	0.0032999638438815\\
    84.4048096192385	0.00322337232733906\\
    84.9519038076152	0.0031485828902004\\
    85.498997995992	0.00307555510153141\\
    86.0460921843687	0.00300424925056784\\
    86.5931863727455	0.00293462634742606\\
    87.1402805611222	0.00286664812264927\\
    87.687374749499	0.00280027702568531\\
    88.2344689378757	0.00273547622238632\\
    88.7815631262525	0.00267220959161437\\
    89.3286573146293	0.00261044172103195\\
    89.875751503006	0.00255013790215083\\
    90.4228456913828	0.00249126412470817\\
    90.9699398797595	0.00243378707043392\\
    91.5170340681363	0.0023776741062694\\
    92.064128256513	0.00232289327709295\\
    92.6112224448898	0.00226941329800448\\
    93.1583166332665	0.00221720354621745\\
    93.7054108216433	0.00216623405260333\\
    94.25250501002	0.00211647549293023\\
    94.7995991983968	0.00206789917883495\\
    95.3466933867736	0.00202047704856411\\
    95.8937875751503	0.00197418165751833\\
    96.4408817635271	0.00192898616863018\\
    96.9879759519038	0.0018848643426047\\
    97.5350701402806	0.00184179052804921\\
    98.0821643286573	0.00179973965151682\\
    98.6292585170341	0.00175868720748637\\
    99.1763527054108	0.00171860924829966\\
    99.7234468937876	0.00167948237407524\\
    100.270541082164	0.00164128372261633\\
    100.817635270541	0.00160399095932926\\
    101.364729458918	0.00156758226716705\\
    101.911823647295	0.00153203633661196\\
    102.458917835671	0.00149733235570921\\
    103.006012024048	0.00146345000016324\\
    103.553106212425	0.00143036942350678\\
    104.100200400802	0.00139807124735194\\
    104.647294589178	0.00136653655173182\\
    105.194388777555	0.00133574686554003\\
    105.741482965932	0.00130568415707505\\
    106.288577154309	0.00127633082469529\\
    106.835671342685	0.00124766968759043\\
    107.382765531062	0.0012196839766735\\
    107.929859719439	0.00119235732559813\\
    108.476953907816	0.00116567376190434\\
    109.024048096192	0.0011396176982961\\
    109.571142284569	0.0011141739240533\\
    110.118236472946	0.0010893275965802\\
    110.665330661323	0.00106506423309242\\
    111.212424849699	0.00104136970244356\\
    111.759519038076	0.00101823021709293\\
    112.306613226453	0.000995632325214829\\
    112.85370741483	0.000973562902950099\\
    113.400801603206	0.000952009146800089\\
    113.947895791583	0.000930958566163025\\
    114.49498997996	0.000910398976012537\\
    115.042084168337	0.000890318489717939\\
    115.589178356713	0.000870705512005592\\
    116.13627254509	0.000851548732060575\\
    116.683366733467	0.0008328371167677\\
    117.230460921844	0.000814559904090769\\
    117.77755511022	0.000796706596588887\\
    118.324649298597	0.00077926695506844\\
    118.871743486974	0.000762230992369359\\
    119.418837675351	0.000745588967284099\\
    119.965931863727	0.000729331378607736\\
    120.513026052104	0.000713448959317502\\
    121.060120240481	0.000697932670880018\\
    121.607214428858	0.000682773697684388\\
    122.154308617234	0.00066796344159936\\
    122.701402805611	0.000653493516652608\\
    123.248496993988	0.000639355743830246\\
    123.795591182365	0.000625542145994597\\
    124.342685370741	0.000612044942918267\\
    124.889779559118	0.000598856546432485\\
    125.436873747495	0.000585969555687752\\
    125.983967935872	0.000573376752524738\\
    126.531062124248	0.000561071096953431\\
    127.078156312625	0.000549045722738498\\
    127.625250501002	0.000537293933088847\\
    128.172344689379	0.000525809196449363\\
    128.719438877756	0.000514585142392817\\
    129.266533066132	0.00050361555760994\\
    129.813627254509	0.000492894381995682\\
    130.360721442886	0.000482415704829682\\
    130.907815631263	0.000472173761048995\\
    131.454909819639	0.00046216292761113\\
    132.002004008016	0.000452377719945501\\
    132.549098196393	0.000442812788491371\\
    133.09619238477	0.000433462915320439\\
    133.643286573146	0.000424323010842202\\
    134.190380761523	0.000415388110590281\\
    134.7374749499	0.000406653372087901\\
    135.284569138277	0.000398114071790753\\
    135.831663326653	0.000389765602105505\\
    136.37875751503	0.000381603468482218\\
    136.925851703407	0.000373623286578998\\
    137.472945891784	0.000365820779497205\\
    138.02004008016	0.000358191775085606\\
    138.567134268537	0.000350732203311824\\
    139.114228456914	0.000343438093699558\\
    139.661322645291	0.000336305572829986\\
    140.208416833667	0.00032933086190584\\
    140.755511022044	0.000322510274376683\\
    141.302605210421	0.000315840213623893\\
    141.849699398798	0.000309317170703947\\
    142.396793587174	0.000302937722148588\\
    142.943887775551	0.000296698527820502\\
    143.490981963928	0.000290596328823159\\
    144.038076152305	0.000284627945463493\\
    144.585170340681	0.000278790275266137\\
    145.132264529058	0.000273080291037942\\
    145.679358717435	0.000267495038981544\\
    146.226452905812	0.000262031636856769\\
    146.773547094188	0.000256687272188698\\
    147.320641282565	0.000251459200521222\\
    147.867735470942	0.000246344743714972\\
    148.414829659319	0.000241341288288495\\
    148.961923847695	0.000236446283801623\\
    149.509018036072	0.000231657241279952\\
    150.056112224449	0.00022697173167942\\
    150.603206412826	0.000222387384389961\\
    151.150300601202	0.000217901885777253\\
    151.697394789579	0.000213512977761614\\
    152.244488977956	0.00020921845643308\\
    152.791583166333	0.000205016170701768\\
    153.338677354709	0.00020090402098263\\
    153.885771543086	0.000196879957913709\\
    154.432865731463	0.000192941981107062\\
    154.97995991984	0.0001890881379315\\
    155.527054108216	0.000185316522326361\\
    156.074148296593	0.000181625273645488\\
    156.62124248497	0.000178012575530672\\
    157.168336673347	0.00017447665481379\\
    157.715430861723	0.000171015780446899\\
    158.2625250501	0.00016762826245959\\
    158.809619238477	0.00016431245094288\\
    159.356713426854	0.000161066735058969\\
    159.90380761523	0.000157889542076208\\
    160.450901803607	0.000154779336428612\\
    160.997995991984	0.000151734618799299\\
    161.545090180361	0.000148753925227237\\
    162.092184368737	0.00014583582623669\\
    162.639278557114	0.000142978925988795\\
    163.186372745491	0.000140181861454683\\
    163.733466933868	0.000137443301609595\\
    164.280561122244	0.000134761946647461\\
    164.827655310621	0.000132136527215388\\
    165.374749498998	0.000129565803667576\\
    165.921843687375	0.000127048565338137\\
    166.468937875751	0.00012458362983233\\
    167.016032064128	0.000122169842335758\\
    167.563126252505	0.000119806074941028\\
    168.110220440882	0.000117491225991461\\
    168.657314629259	0.000115224219441375\\
    169.204408817635	0.000113004004232545\\
    169.751503006012	0.000110829553686398\\
    170.298597194389	0.000108699864911553\\
    170.845691382766	0.000106613958226293\\
    171.392785571142	0.000104570876595604\\
    171.939879759519	0.000102569685082384\\
    172.486973947896	0.00010060947031247\\
    173.034068136273	9.8689339953122e-05\\
    173.581162324649	9.68084222046131e-05\\
    174.128256513026	9.49658653045932e-05\\
    174.675350701403	9.31608370448921e-05\\
    175.22244488978	9.1392524300444e-05\\
    175.769539078156	8.96601325700183e-05\\
    176.316633266533	8.79628855284539e-05\\
    176.86372745491	8.63000245901014e-05\\
    177.410821643287	8.46708084831796e-05\\
    177.957915831663	8.30745128347726e-05\\
    178.50501002004	8.1510429766186e-05\\
    179.052104208417	7.9977867498401e-05\\
    179.599198396794	7.84761499673634e-05\\
    180.14629258517	7.70046164488558e-05\\
    180.693386773547	7.55626211927073e-05\\
    181.240480961924	7.4149533066099e-05\\
    181.787575150301	7.27647352057321e-05\\
    182.334669338677	7.14076246786335e-05\\
    182.881763527054	7.00776121513734e-05\\
    183.428857715431	6.87741215674826e-05\\
    183.975951903808	6.74965898328565e-05\\
    184.523046092184	6.62444665089436e-05\\
    185.070140280561	6.50172135135176e-05\\
    185.617234468938	6.38143048288404e-05\\
    186.164328657315	6.26352262170251e-05\\
    186.711422845691	6.14794749424164e-05\\
    187.258517034068	6.03465595008102e-05\\
    187.805611222445	5.92359993553344e-05\\
    188.352705410822	5.81473246788244e-05\\
    188.899799599198	5.70800761025265e-05\\
    189.446893787575	5.60338044709674e-05\\
    189.993987975952	5.50080706028331e-05\\
    190.541082164329	5.40024450577062e-05\\
    191.088176352705	5.30165079085085e-05\\
    191.635270541082	5.2049848519509e-05\\
    192.182364729459	5.11020653297518e-05\\
    192.729458917836	5.01727656417694e-05\\
    193.276553106212	4.9261565415445e-05\\
    193.823647294589	4.83680890668955e-05\\
    194.370741482966	4.7491969272247e-05\\
    194.917835671343	4.66328467761786e-05\\
    195.464929859719	4.57903702051161e-05\\
    196.012024048096	4.49641958849545e-05\\
    196.559118236473	4.41539876631995e-05\\
    197.10621242485	4.33594167354138e-05\\
    197.653306613226	4.25801614758589e-05\\
    198.200400801603	4.18159072722302e-05\\
    198.74749498998	4.10663463643792e-05\\
    199.294589178357	4.03311776869248e-05\\
    199.841683366733	3.96101067156543e-05\\
    200.38877755511	3.89028453176202e-05\\
    200.935871743487	3.82091116048404e-05\\
    201.482965931864	3.75286297915103e-05\\
    202.03006012024	3.68611300546401e-05\\
    202.577154308617	3.62063483980305e-05\\
    203.124248496994	3.55640265195056e-05\\
    203.671342685371	3.49339116813185e-05\\
    204.218436873747	3.43157565836536e-05\\
    204.765531062124	3.37093192411463e-05\\
    205.312625250501	3.31143628623462e-05\\
    205.859719438878	3.25306557320499e-05\\
    206.406813627255	3.19579710964329e-05\\
    206.953907815631	3.13960870509101e-05\\
    207.501002004008	3.08447864306583e-05\\
    208.048096192385	3.03038567037335e-05\\
    208.595190380762	2.97730898667206e-05\\
    209.142284569138	2.92522823428505e-05\\
    209.689378757515	2.87412348825258e-05\\
    210.236472945892	2.82397524661946e-05\\
    210.783567134269	2.77476442095146e-05\\
    211.330661322645	2.7264723270751e-05\\
    211.877755511022	2.67908067603536e-05\\
    212.424849699399	2.63257156526592e-05\\
    212.971943887776	2.58692746996668e-05\\
    213.519038076152	2.54213123468363e-05\\
    214.066132264529	2.49816606508582e-05\\
    214.613226452906	2.45501551993491e-05\\
    215.160320641283	2.41266350324239e-05\\
    215.707414829659	2.37109425660995e-05\\
    216.254509018036	2.3302923517485e-05\\
    216.801603206413	2.29024268317159e-05\\
    217.34869739479	2.25093046105876e-05\\
    217.895791583166	2.21234120428488e-05\\
    218.442885771543	2.17446073361142e-05\\
    218.98997995992	2.13727516503559e-05\\
    219.537074148297	2.10077090329364e-05\\
    220.084168336673	2.06493463551454e-05\\
    220.63126252505	2.02975332502038e-05\\
    221.178356713427	1.99521420526996e-05\\
    221.725450901804	1.96130477394215e-05\\
    222.27254509018	1.92801278715549e-05\\
    222.819639278557	1.89532625382099e-05\\
    223.366733466934	1.86323343012466e-05\\
    223.913827655311	1.83172281413683e-05\\
    224.460921843687	1.80078314054514e-05\\
    225.008016032064	1.77040337550821e-05\\
    225.555110220441	1.74057271162717e-05\\
    226.102204408818	1.71128056303208e-05\\
    226.649298597194	1.68251656058069e-05\\
    227.196392785571	1.65427054716667e-05\\
    227.743486973948	1.62653257313476e-05\\
    228.290581162325	1.59929289180035e-05\\
    228.837675350701	1.57254195507086e-05\\
    229.384769539078	1.54627040916667e-05\\
    229.931863727455	1.52046909043903e-05\\
    230.478957915832	1.49512902128282e-05\\
    231.026052104208	1.47024140614181e-05\\
    231.573146292585	1.44579762760424e-05\\
    232.120240480962	1.42178924258663e-05\\
    232.667334669339	1.39820797860361e-05\\
    233.214428857715	1.37504573012191e-05\\
    233.761523046092	1.35229455499632e-05\\
    234.308617234469	1.32994667098591e-05\\
    234.855711422846	1.30799445234837e-05\\
    235.402805611222	1.28643042651087e-05\\
    235.949899799599	1.26524727081544e-05\\
    236.496993987976	1.24443780933725e-05\\
    237.044088176353	1.22399500977403e-05\\
    237.591182364729	1.20391198040495e-05\\
    238.138276553106	1.18418196711734e-05\\
    238.685370741483	1.16479835049973e-05\\
    239.23246492986	1.14575464299948e-05\\
    239.779559118236	1.12704448614371e-05\\
    240.326653306613	1.10866164782194e-05\\
    240.87374749499	1.09060001962896e-05\\
    241.420841683367	1.07285361426665e-05\\
    241.967935871743	1.05541656300324e-05\\
    242.51503006012	1.0382831131888e-05\\
    243.062124248497	1.02144762582559e-05\\
    243.609218436874	1.004904573192e-05\\
    244.156312625251	9.88648536518828e-06\\
    244.703406813627	9.72674203716783e-06\\
    245.250501002004	9.56976367153836e-06\\
    245.797595190381	9.41549921481488e-06\\
    246.344689378758	9.26389861508668e-06\\
    246.891783567134	9.1149128012223e-06\\
    247.438877755511	8.96849366252981e-06\\
    247.985971943888	8.82459402886151e-06\\
    248.533066132265	8.68316765115346e-06\\
    249.080160320641	8.54416918238922e-06\\
    249.627254509018	8.40755415897854e-06\\
    250.174348697395	8.2732789825414e-06\\
    250.721442885772	8.14130090208799e-06\\
    251.268537074148	8.0115779965857e-06\\
    251.815631262525	7.88406915790437e-06\\
    252.362725450902	7.75873407413096e-06\\
    252.909819639279	7.63553321324549e-06\\
    253.456913827655	7.51442780714971e-06\\
    254.004008016032	7.39537983604102e-06\\
    254.551102204409	7.2783520131231e-06\\
    255.098196392786	7.16330776964624e-06\\
    255.645290581162	7.05021124026954e-06\\
    256.192384769539	6.9390272487379e-06\\
    256.739478957916	6.82972129386651e-06\\
    257.286573146293	6.72225953582612e-06\\
    257.833667334669	6.61660878272198e-06\\
    258.380761523046	6.51273647746038e-06\\
    258.927855711423	6.41061068489563e-06\\
    259.4749498998	6.31020007925172e-06\\
    260.022044088176	6.21147393181235e-06\\
    260.569138276553	6.11440209887314e-06\\
    261.11623246493	6.01895500995032e-06\\
    261.663326653307	5.92510365624029e-06\\
    262.210420841683	5.83281957932395e-06\\
    262.75751503006	5.74207486011095e-06\\
    263.304609218437	5.65284210801813e-06\\
    263.851703406814	5.56509445037698e-06\\
    264.39879759519	5.47880552206508e-06\\
    264.945891783567	5.39394945535662e-06\\
    265.492985971944	5.31050086998696e-06\\
    266.040080160321	5.22843486342655e-06\\
    266.587174348697	5.14772700135971e-06\\
    267.134268537074	5.06835330836344e-06\\
    267.681362725451	4.99029025878226e-06\\
    268.228456913828	4.91351476779421e-06\\
    268.775551102204	4.83800418266434e-06\\
    269.322645290581	4.76373627418118e-06\\
    269.869739478958	4.69068922827231e-06\\
    270.416833667335	4.61884163779509e-06\\
    270.963927855711	4.54817249449867e-06\\
    271.511022044088	4.47866118115355e-06\\
    272.058116232465	4.41028746384497e-06\\
    272.605210420842	4.34303148442662e-06\\
    273.152304609218	4.27687375313108e-06\\
    273.699398797595	4.21179514133364e-06\\
    274.246492985972	4.14777687446611e-06\\
    274.793587174349	4.08480052507733e-06\\
    275.340681362725	4.02284800603722e-06\\
    275.887775551102	3.96190156388136e-06\\
    276.434869739479	3.90194377229263e-06\\
    276.981963927856	3.84295752571765e-06\\
    277.529058116232	3.78492603311429e-06\\
    278.076152304609	3.72783281182805e-06\\
    278.623246492986	3.67166168159414e-06\\
    279.170340681363	3.61639675866277e-06\\
    279.717434869739	3.56202245004482e-06\\
    280.264529058116	3.50852344787538e-06\\
    280.811623246493	3.45588472389267e-06\\
    281.35871743487	3.4040915240297e-06\\
    281.905811623247	3.35312936311645e-06\\
    282.452905811623	3.30298401968994e-06\\
    283	3.25364153091023e-06\\
    };
    \addlegendentry{Fittagio lognormale}
        % This file was created by matlab2tikz.
%
\definecolor{mycolor1}{rgb}{0.00000,0.44700,0.74100}%
%
\nextgroupplot[
    % width=4.521in,
    % height=3.566in,
    % at={(0.758in,0.481in)},
    scale only axis,
    xmin=0,
    xmax=300,
    xlabel style={font=\color{white!15!black}},
    xlabel={$k$},
    ymin=40,
    ymax=95,
    ylabel style={font=\color{white!15!black}},
    ylabel={$k_{nn}(k)$},
    axis background/.style={fill=white},
    % axis x line*=bottom,
    % axis y line*=left,
    xmajorgrids,
    ymajorgrids,
    grid style={dashed},
    legend style=plotLegend
]
    \addplot[
        only marks,
        mark=*,
        mark options={},
        mark size=1.5000pt,
        color=black,
        fill=black,
        forget plot
    ] table[row sep=crcr]{%
        x	y\\
        10	89.3\\
        11	87\\
        12	89.75\\
        13	76.0512820512821\\
        14	75.3285714285714\\
        15	71.3666666666667\\
        16	80.9\\
        17	73.3921568627451\\
        18	74.75\\
        19	69.2842105263158\\
        20	70.07\\
        21	66.7619047619048\\
        22	60.5454545454545\\
        23	73.3188405797101\\
        24	68.65\\
        25	74\\
        26	65.2211538461538\\
        27	67.0940170940171\\
        28	70.0178571428571\\
        29	68.367816091954\\
        30	72.2388888888889\\
        31	66.6589861751152\\
        32	64.1741071428571\\
        33	65.7212121212121\\
        34	70.1617647058823\\
        35	68.7285714285714\\
        36	71.9027777777778\\
        37	70.7837837837838\\
        38	66.4368421052632\\
        39	67.2582417582418\\
        40	74.4583333333333\\
        41	72.8024390243902\\
        42	69.7047619047619\\
        43	71.4249471458774\\
        44	68.2318181818182\\
        45	74.2395061728395\\
        46	73.2217391304348\\
        47	72.5471124620061\\
        48	68.2625\\
        49	69.0979591836735\\
        50	73.9366666666667\\
        51	67.7303921568627\\
        52	75.3205128205128\\
        53	70.2924528301887\\
        54	66.7453703703704\\
        55	67.6\\
        56	75.8154761904762\\
        57	69.7280701754386\\
        58	50.6896551724138\\
        59	54.7966101694915\\
        61	76.1803278688525\\
        62	74.4596774193548\\
        63	64.5714285714286\\
        64	69.76953125\\
        65	71.2307692307692\\
        67	72.1940298507463\\
        68	75.6911764705882\\
        69	50.7971014492754\\
        70	63.0714285714286\\
        71	65.6056338028169\\
        72	73.5972222222222\\
        73	73.1872146118722\\
        76	51.5131578947368\\
        77	63.6883116883117\\
        80	64.85\\
        82	61.1341463414634\\
        83	69.2048192771084\\
        84	57.6547619047619\\
        85	75.5058823529412\\
        86	52.1860465116279\\
        87	62.5632183908046\\
        88	72.0397727272727\\
        89	62.9101123595506\\
        90	60.9\\
        91	62.2747252747253\\
        93	66.5161290322581\\
        94	71.6063829787234\\
        95	58.1684210526316\\
        96	60.0347222222222\\
        97	44.7216494845361\\
        98	68.7908163265306\\
        99	67.4949494949495\\
        101	68.2772277227723\\
        103	66.3300970873786\\
        108	63.75\\
        109	57.0917431192661\\
        111	66.045045045045\\
        112	65.7589285714286\\
        117	63.8034188034188\\
        120	63.125\\
        127	53.8740157480315\\
        129	62.2286821705426\\
        132	63.0606060606061\\
        135	48.3925925925926\\
        140	61.7071428571429\\
        149	62.1744966442953\\
        154	47.7207792207792\\
        182	55.7967032967033\\
        200	54.56\\
        206	54.3155339805825\\
        283	50.3462897526502\\
    };
        % This file was created by matlab2tikz.
%
\nextgroupplot[
    % width=4.521in,
    % height=3.566in,
    % at={(0.758in,0.481in)},
    scale only axis,
    xmode=log,
    xmin=0.1,
    xmax=100000,
    xminorticks=true,
    xlabel style={font=\color{white!15!black}},
    xlabel={$w$},
    ymode=log,
    ymin=1e-08,
    ymax=1,
    yminorticks=true,
    ytickten={-8,-6,-4,-2,0},
    ylabel style={font=\color{white!15!black}},
    ylabel={$P(w)$},
    axis background/.style={fill=white},
    % axis x line*=bottom,
    % axis y line*=left,
    xmajorgrids,
    ymajorgrids,
    grid style={dashed},
    % legend style={
    %     legend cell align=left,
    %     align=left,
    %     draw=white!15!black
    % },
    legend style=plotLegend,
    log origin=infty
]
    \addplot[
        ybar interval,
        fill=colore,
        area legend,
        draw=none
    ] table[
        row sep=crcr,
        x=Lower,
        y=Count
    ] {%
        Lower	Upper	Count\\
        1	1.65709146891364	0.645588797757302\\
        1.65709146891364	2.74595213634636	0.134535967904517\\
        2.74595213634636	4.55029385918474	0.0705509116756377\\
        4.55029385918474	7.54025313510515	0.0254058774758624\\
        7.54025313510515	12.4948891436321	0.0118545769628516\\
        12.4948891436321	20.7051742049343	0.00525272541319737\\
        20.7051742049343	34.3103675373674	0.00279592349281125\\
        34.3103675373674	56.8554173414631	0.00121543363316469\\
        56.8554173414631	94.214627038063	0.000529215472189377\\
        94.214627038063	156.122254711654	0.000235320892045568\\
        156.122254711654	258.708856390245	9.46722601860312e-05\\
        258.708856390245	428.704238856679	4.1488412073954e-05\\
        428.704238856679	710.402136896518	1.19027824068337e-05\\
        710.402136896518	1177.20132054924	4.95374914842227e-06\\
        1177.20132054924	1950.73026547601	7.47356021280746e-07\\
        1950.73026547601	3232.53848107194	6.31406563500675e-07\\
        3232.53848107194	5356.61193991936	1.63299864856744e-07\\
        5356.61193991936	8876.39594792132	3.28486926079384e-08\\
        8876.39594792132	14709	1.98231016357072e-08\\
        14709	14709	1.98231016357072e-08\\
    };
    \addlegendentry{Istogramma}

    \addplot [color=blue]
    table[row sep=crcr]{%
        1.28728064885387	0.62520801005048\\
        2.13314178131336	0.245418093491598\\
        3.53481104779761	0.0963360028099894\\
        5.85750523152711	0.0378155714005007\\
        9.70642194828058	0.0148440603578591\\
        16.0844290041719	0.00582686231484118\\
        26.6533700851603	0.00228726666542692\\
        44.1670721859171	0.000897839783419662\\
        73.1888785261761	0.000352436508115071\\
        121.280666225083	0.000138344830053366\\
        200.973157345748	5.43056453057517e-05\\
        333.030904518277	2.13170460431121e-05\\
        551.862670761829	8.36775715389635e-06\\
        914.486923731325	3.28466522260985e-06\\
        1515.38847974825	1.28935692398755e-06\\
        2511.13732188084	5.06121983449424e-07\\
        4161.18423335939	1.98672266278731e-07\\
        6895.46289367779	7.79864749587012e-08\\
        11426.412735324	3.06126788122075e-08\\
    };
    \addlegendentry{Regressione}
        % This file was created by matlab2tikz.
%
\definecolor{mycolor1}{rgb}{0.00000,0.44700,0.74100}%
%
\nextgroupplot[%
    % width=4.521in,
    % height=3.566in,
    % at={(0.758in,0.481in)},
    height=\halfHeightPlot cm,
    scale only axis,
    xmode=log,
    xmin=10,
    xmax=283,
    % xminorticks=true,
    xtickten={1,2,3},
    xlabel style={font=\color{white!15!black}},
    xticklabels=\empty,
    % xlabel={$k$},
    % ymode=log,
    ymin=0.2,
    ymax=1,
    ytick={0.5,1},
    ylabel style={font=\color{white!15!black}},
    ylabel={$C^w(k)$},
    axis background/.style={fill=white},
    % axis x line*=bottom,
    % axis y line*=left,
    xmajorgrids,
    ymajorgrids,
    grid style={dashed}
]
    \addplot[
        only marks,
        mark=*,
        mark options={},
        mark size=1.5000pt,
        color=black,
        fill=black,
        forget plot
    ] table[row sep=crcr]{%
        x	y\\
        10	0.894327894327894\\
        11	0.896875\\
        12	0.830909090909091\\
        13	0.800453303527074\\
        14	0.815850170144056\\
        15	0.813130799972905\\
        16	0.834878759270644\\
        17	0.775269541778976\\
        18	0.757456305734169\\
        19	0.770403781959471\\
        20	0.83479343860568\\
        21	0.816423991450708\\
        22	0.847839502331492\\
        23	0.729939824999801\\
        24	0.810997209910532\\
        25	0.78459333853098\\
        26	0.801518547409401\\
        27	0.776575852318383\\
        28	0.734995863363864\\
        29	0.749257653419528\\
        30	0.775776268651412\\
        31	0.762293497616241\\
        32	0.787049482034579\\
        33	0.778817140903613\\
        34	0.784012737306352\\
        35	0.721792410868428\\
        36	0.750934488976285\\
        37	0.731722001441343\\
        38	0.773717163822543\\
        39	0.747765922770739\\
        40	0.748744286264932\\
        41	0.741600052528491\\
        42	0.749792400268859\\
        43	0.768847661947213\\
        44	0.765081107651386\\
        45	0.766256093035422\\
        46	0.722138909086021\\
        47	0.751204822423547\\
        48	0.6547134119275\\
        49	0.780532499792838\\
        50	0.775736585299164\\
        51	0.697561305096584\\
        52	0.819252193888351\\
        53	0.722607000952232\\
        54	0.729430834357906\\
        55	0.722550842209151\\
        56	0.843210995241948\\
        57	0.805505769117992\\
        58	0.88051919923265\\
        59	0.533710344827586\\
        61	0.771145313366611\\
        62	0.632787499133979\\
        63	0.456271962727524\\
        64	0.529447076518452\\
        65	0.766292869922502\\
        67	0.742633028853125\\
        68	0.835663014131182\\
        69	0.574723798148701\\
        70	0.713903911612884\\
        71	0.708923132055187\\
        72	0.756661579712847\\
        73	0.805549028209804\\
        76	0.493916159082766\\
        77	0.501117659012396\\
        80	0.611005383300957\\
        82	0.467625344862132\\
        83	0.822259065390164\\
        84	0.43300425421362\\
        85	0.811411555130148\\
        86	0.504297798688477\\
        87	0.485174637010117\\
        88	0.732616441115305\\
        89	0.398602675163464\\
        90	0.481058796652059\\
        91	0.403352490421456\\
        93	0.546489872775994\\
        94	0.694512907930786\\
        95	0.613204482702615\\
        96	0.782246320183604\\
        97	0.746396568491248\\
        98	0.734958906667555\\
        99	0.917936104201023\\
        101	0.871418955304254\\
        103	0.553052938347056\\
        108	0.880113487920137\\
        109	0.469670557324036\\
        111	0.870154491465592\\
        112	0.563468856504336\\
        117	0.57590193342635\\
        120	0.641274333680985\\
        127	0.436843340963785\\
        129	0.602637264165786\\
        132	0.62281237907623\\
        135	0.33888893378263\\
        140	0.780628406299617\\
        149	0.879238140141167\\
        154	0.377818602505703\\
        182	0.29470083659524\\
        200	0.3072496314974\\
        206	0.27787132519668\\
        283	0.369798951169732\\
    };
        % This file was created by matlab2tikz.
%
\definecolor{mycolor1}{rgb}{0.00000,0.44700,0.74100}%
%
\nextgroupplot[%
    % width=4.521in,
    % height=3.566in,
    % at={(0.758in,0.481in)},
    height=\halfHeightPlot cm,
    yshift=\yPlotSepPartial em,
    scale only axis,
    xmode=log,
    xmin=10,
    xmax=283,
    xminorticks=true,
    xtickten={1,2,3},
    xlabel style={font=\color{white!15!black}},
    xlabel={$k$},
    ymode=log,
    % ymin=0.128303426794971,
    % ymax=1.9673950820926,
    ymin=1,
    ymax=1000,
    ytickten={0,1,2,3},
    yminorticks=true,
    ylabel style={font=\color{white!15!black}},
    ylabel={$C^w_{\text{rel}}(k) [\%]$},
    axis background/.style={fill=white},
    % axis x line*=bottom,
    % axis y line*=left,
    xmajorgrids,
    ymajorgrids,
    grid style={dashed}
]
    \addplot[
        only marks,
        mark=*,
        mark options={},
        mark size=1.5000pt,
        color=black,
        fill=black,
        forget plot
    ] table[
        row sep=crcr,
        y expr=\thisrow{y}*100
    ]{%
        x	y\\
        10	0.166514644775514\\
        11	0.147165697674419\\
        12	0.246363636363636\\
        13	0.142110201373996\\
        14	0.128303426794971\\
        15	0.130844158902716\\
        16	0.175885576437527\\
        17	0.207289973457336\\
        18	0.15123325937743\\
        19	0.224340582853806\\
        20	0.330627125294288\\
        21	0.293955005318103\\
        22	0.222540106358143\\
        23	0.314411215124197\\
        24	0.273237940473873\\
        25	0.282325674354508\\
        26	0.368018527307074\\
        27	0.365516614307815\\
        28	0.300086272117644\\
        29	0.379585520581988\\
        30	0.32251865524506\\
        31	0.407410853511551\\
        32	0.432950079509207\\
        33	0.373465098186732\\
        34	0.372328067484753\\
        35	0.3496746840563\\
        36	0.453979955605254\\
        37	0.448026845639999\\
        38	0.470857669462541\\
        39	0.407862737354564\\
        40	0.415235565314976\\
        41	0.439318445144054\\
        42	0.538539696452544\\
        43	0.433364081479291\\
        44	0.464521909830457\\
        45	0.445245933307707\\
        46	0.50870765220838\\
        47	0.50419870105292\\
        48	0.608968907743398\\
        49	0.467710616815442\\
        50	0.516803379076578\\
        51	0.522277559260838\\
        52	0.515102383676365\\
        53	0.746934118091535\\
        54	0.687656465587977\\
        55	0.497192559089194\\
        56	0.597881377365792\\
        57	0.765916493835598\\
        58	1.05288890878924\\
        59	0.381510438729198\\
        61	0.681997524983192\\
        62	0.434773574175485\\
        63	0.291448033633122\\
        64	0.511848875724079\\
        65	0.961709747001604\\
        67	0.631357801087192\\
        68	0.601043184348892\\
        69	0.608952303647796\\
        70	0.932100799266099\\
        71	0.438101210740523\\
        72	0.514508220631196\\
        73	0.656481100262413\\
        76	0.339353999415683\\
        77	0.64380075142407\\
        80	0.900371074046283\\
        82	0.507751233288487\\
        83	1.05594974248547\\
        84	0.331087151841869\\
        85	0.776051043417921\\
        86	0.530904031732876\\
        87	0.570102350393465\\
        88	0.73489374363711\\
        89	0.247744265339827\\
        90	0.605533733826248\\
        91	0.316118285478774\\
        93	0.30316815815814\\
        94	0.659768135902387\\
        95	1.10450270197323\\
        96	1.0425901240908\\
        97	1.49119886945896\\
        98	0.828931771408844\\
        99	1.39403658144041\\
        101	1.33580983242382\\
        103	0.517861590980713\\
        108	1.47098918037053\\
        109	0.775517598207628\\
        111	1.26827206250958\\
        112	0.555294143885859\\
        117	0.708072779821335\\
        120	0.926251048583185\\
        127	0.910980629333647\\
        129	1.077400105617\\
        132	0.931433224351897\\
        135	0.423050327791963\\
        140	1.52594426115573\\
        149	1.9673950820926\\
        154	0.813067599234089\\
        182	0.515459718876113\\
        200	0.450253241650442\\
        206	0.52515025514112\\
        283	1.29523838054531\\
    };
    \end{groupplot}

    %\begingroup Impostazioni pgfplots - Seconda colonna
        \pgfplotsset{
            /pgfplots/group/plotGroup/.append style={
                group name=Col2,
                y descriptions at=edge right,
            },
            plotColumn/.append style={
                xmax=50,
                % xlabel={},
                axis y line*=right,
                y tick scale label style={
                    xshift=.4cm,
                    at={(1,1)},
                    anchor=south west,
                },
                % ytick distance=0.01,
            },
            % plotLegend/.append style={
            %     at={(1,.5)},
            %     anchor=east,
            %     xshift=-2*\xLegendSep, % Le traslazioni vengono sommate; quindi se prima si aveva \xLegendSep ora si ha «\xLegendSep-2*\xLegendSepcm=-\xLegendSep», come desiderato
            % },
        }
    %\endgroup

    \begin{groupplot}[group style=plotGroup,plotColumn,clip=false]
        % This file was created by matlab2tikz.
%
\definecolor{mycolor1}{rgb}{0.00000,0.44700,0.74100}%
%
\nextgroupplot[%
    at={($(Col1 c1r1.south east)+(\xPlotSep em,0)$)},
    % at={(0,0)},
    anchor=south west,
    % title style={font=\dimTesto{\titleSize}},
    % title={Prova},
    %
    % width=4.521in,
    % height=3.566in,
    % at={(0.758in,0.481in)},
    scale only axis,
    xmin=0,
    xmax=300,
    xlabel style={font=\color{white!15!black}},
    xlabel={$k$},
    ymin=0.1,
    ymax=0.8,
    ylabel style={font=\color{white!15!black}},
    ylabel={$C(k)$},
    axis background/.style={fill=white},
    % axis x line*=bottom,
    % axis y line*=right,
    xmajorgrids,
    ymajorgrids,
    grid style={dashed}
]
    \addplot[
        only marks,
        mark=*,
        mark options={},
        mark size=1.5000pt,
        color=black,
        fill=black,
        forget plot
    ] table[row sep=crcr]{%
        x	y\\
        10	0.766666666666667\\
        11	0.781818181818182\\
        12	0.666666666666667\\
        13	0.700854700854701\\
        14	0.723076923076923\\
        15	0.719047619047619\\
        16	0.71\\
        17	0.642156862745098\\
        18	0.657952069716776\\
        19	0.629239766081871\\
        20	0.627368421052632\\
        21	0.630952380952381\\
        22	0.693506493506494\\
        23	0.555335968379447\\
        24	0.63695652173913\\
        25	0.611851851851852\\
        26	0.585897435897436\\
        27	0.56870479947403\\
        28	0.565343915343915\\
        29	0.543103448275862\\
        30	0.586590038314176\\
        31	0.541628264208909\\
        32	0.549251152073733\\
        33	0.567045454545455\\
        34	0.571301247771836\\
        35	0.534789915966387\\
        36	0.516468253968254\\
        37	0.505323505323505\\
        38	0.526031294452347\\
        39	0.531135531135531\\
        40	0.529059829059829\\
        41	0.515243902439024\\
        42	0.487340301974448\\
        43	0.53639383871942\\
        44	0.522410147991543\\
        45	0.530190796857464\\
        46	0.478647342995169\\
        47	0.499405312541298\\
        48	0.406914893617021\\
        49	0.531802721088435\\
        50	0.511428571428571\\
        51	0.458235294117647\\
        52	0.540723981900452\\
        53	0.413642960812772\\
        54	0.432215234102027\\
        55	0.482603815937149\\
        56	0.527705627705628\\
        57	0.456140350877193\\
        58	0.428917120387175\\
        59	0.386323787258913\\
        61	0.458469945355191\\
        62	0.441036488630354\\
        63	0.353302611367127\\
        64	0.350198412698413\\
        65	0.390625\\
        67	0.455223880597015\\
        68	0.521949078138718\\
        69	0.357203751065644\\
        70	0.369496204278813\\
        71	0.492957746478873\\
        72	0.499608763693271\\
        73	0.486301369863014\\
        76	0.368771929824561\\
        77	0.304853041695147\\
        80	0.321518987341772\\
        82	0.310147545919904\\
        83	0.399941228327946\\
        84	0.325301204819277\\
        85	0.456862745098039\\
        86	0.329411764705882\\
        87	0.309008286554397\\
        88	0.422283176593521\\
        89	0.319458631256384\\
        90	0.299625468164794\\
        91	0.306471306471306\\
        93	0.419354838709677\\
        94	0.418439716312057\\
        95	0.291377379619261\\
        96	0.38296783625731\\
        97	0.299613402061856\\
        98	0.401851462234378\\
        99	0.383426097711812\\
        101	0.373069306930693\\
        103	0.364363221016562\\
        108	0.356178608515057\\
        109	0.264525993883792\\
        111	0.383619983619984\\
        112	0.362290862290862\\
        117	0.337164750957854\\
        120	0.332913165266106\\
        127	0.228596425446819\\
        129	0.290092054263566\\
        132	0.322461253758964\\
        135	0.238142620232172\\
        140	0.309044193216855\\
        149	0.296299655360058\\
        154	0.208386384856973\\
        182	0.194462995567968\\
        200	0.211859296482412\\
        206	0.182192753966375\\
        283	0.161115705586046\\
    };
        % This file was created by matlab2tikz.
%
\definecolor{mycolor1}{rgb}{0.00000,0.44700,0.74100}%
%
\nextgroupplot[%
    % width=4.521in,
    % height=3.566in,
    % at={(0.758in,0.481in)},
    scale only axis,
    xmode=log,
    xmin=10,
    xmax=283,
    xminorticks=true,
    xtickten={1,2,3},
    xlabel style={font=\color{white!15!black}},
    xlabel={$k$},
    ymode=log,
    ymin=0.1,
    ymax=100000,
    yminorticks=true,
    ylabel style={font=\color{white!15!black}},
    ylabel={$g(i)$},
    axis background/.style={fill=white},
    % axis x line*=bottom,
    % axis y line*=left,
    xmajorgrids,
    ymajorgrids,
    grid style={dashed}
]
    \addplot[
        only marks,
        mark=*,
        mark options={},
        mark size=1.5000pt,
        color=black,
        fill=black,
        forget plot
    ] table[row sep=crcr]{%
        x	y\\
        34	24.4077595983695\\
        29	24.0548838676952\\
        96	910.816660309013\\
        31	25.2996229232657\\
        27	13.9840093551101\\
        70	554.608345889032\\
        24	14.8482406741793\\
        37	40.2826018794221\\
        42	246.671576922462\\
        16	1.32168783345254\\
        29	29.4747553886748\\
        59	195.572145500989\\
        69	270.874479765742\\
        28	18.7726637358274\\
        11	3.37575686673779\\
        20	5.29951330297145\\
        45	81.287143222938\\
        27	20.2192077351399\\
        22	5.53956779308469\\
        27	10.8248169060855\\
        44	67.7241071465653\\
        16	8.09628111142582\\
        39	35.7214069499101\\
        19	4.41152544194396\\
        33	18.8581472042029\\
        37	149.092921666421\\
        33	39.684293605326\\
        18	4.87634613863792\\
        29	32.2821736541342\\
        22	8.93167057828336\\
        33	44.623208877497\\
        22	9.68002451218361\\
        49	93.8802330926963\\
        25	5.47032516352949\\
        43	149.160960658111\\
        26	11.4727075628203\\
        31	45.4979602836858\\
        26	94.8645864265104\\
        18	17.9909172382737\\
        13	1.16726108273929\\
        32	30.8667101637254\\
        38	43.719369594331\\
        32	55.6734563269646\\
        29	28.6298860183576\\
        35	48.4885214667306\\
        35	24.2038859532622\\
        135	2523.77533612426\\
        27	25.7647196168196\\
        42	61.5692480434759\\
        39	32.4530959423884\\
        51	98.7921520282781\\
        86	799.134065872299\\
        15	6.62365182766655\\
        39	51.6490650231878\\
        42	52.964020421646\\
        43	34.5077581608904\\
        58	186.898874230204\\
        97	891.35557363391\\
        50	124.195428418057\\
        20	6.17627374991726\\
        28	32.8537941255416\\
        22	4.9634136481218\\
        47	107.151810331161\\
        154	4102.09786983822\\
        30	23.9189212447411\\
        10	3.67655526782783\\
        37	67.8857830398076\\
        31	90.6606211853056\\
        54	292.203711602299\\
        76	584.256949076862\\
        55	202.948768295517\\
        26	15.5371884724382\\
        33	29.0888330131769\\
        34	36.9250648178486\\
        25	8.20116141300345\\
        31	87.8908619235439\\
        34	45.3020561970372\\
        26	27.7785523028594\\
        39	64.0148005251858\\
        27	12.7663630780348\\
        32	13.7207539004658\\
        26	7.91316099030401\\
        29	236.506588680164\\
        23	16.2512132708547\\
        26	12.786842760483\\
        14	5.67042257379387\\
        33	40.3118703371952\\
        24	25.6318730925871\\
        20	4.78292497766301\\
        51	145.08881801201\\
        35	54.2674991608661\\
        29	25.6468745604937\\
        46	73.8796585934957\\
        44	117.753574596781\\
        45	122.239044968079\\
        29	25.2761563957573\\
        28	28.800768897232\\
        77	553.786673567684\\
        70	233.169294565903\\
        70	305.858498707502\\
        40	134.109563749234\\
        90	552.938195550776\\
        53	216.086799826985\\
        62	245.435720195646\\
        47	78.7214824837711\\
        34	69.0789759577431\\
        14	0.689398918075389\\
        47	99.6366900263505\\
        43	134.256195850689\\
        20	5.51092800048796\\
        28	28.8896516270878\\
        80	576.507141396327\\
        25	10.543199036633\\
        36	45.9067819055063\\
        27	17.6588538860079\\
        82	430.283635099357\\
        31	25.7159892924415\\
        39	71.3563966359726\\
        23	9.98588642384381\\
        27	19.8025243042393\\
        37	73.9726704345446\\
        129	1161.47704583538\\
        63	226.175921872939\\
        64	346.339609070301\\
        91	523.055435625984\\
        27	25.5013993028184\\
        26	20.9404570310581\\
        17	8.74641658422946\\
        33	54.234936330109\\
        30	22.2482078041211\\
        33	53.8873400217068\\
        182	4849.33892917923\\
        26	37.6618254049445\\
        48	102.731744079356\\
        46	194.956390934928\\
        10	0.640807656395892\\
        25	33.2825476098488\\
        26	22.4177475617209\\
        206	5474.49939142422\\
        37	88.5311567063907\\
        43	55.7579270228685\\
        51	181.727873021351\\
        57	185.607597733343\\
        36	70.3632175138736\\
        38	61.3889586894383\\
        28	28.7156195342204\\
        19	10.2781895725273\\
        36	28.5866769223062\\
        45	137.953976107289\\
        53	125.345111019604\\
        48	85.5688453009621\\
        64	263.899912886093\\
        65	277.529004087262\\
        45	137.945953194533\\
        54	210.791057031158\\
        16	11.1174192026491\\
        24	22.1801333980861\\
        127	1535.78662133912\\
        48	108.291963026947\\
        64	240.243409190148\\
        42	102.933931123422\\
        42	64.3778371990855\\
        17	14.26170366487\\
        48	165.657946273305\\
        42	76.2141116197296\\
        20	10.2620776202165\\
        34	26.9103761127203\\
        47	56.2109594941505\\
        42	49.9744088269208\\
        50	151.019622709178\\
        87	560.979858655156\\
        64	236.218660446638\\
        32	37.5871782528139\\
        39	79.0270632587659\\
        35	99.9860073120595\\
        18	4.75274834573202\\
        28	18.6411194641975\\
        27	43.6029363491292\\
        37	49.4242313582483\\
        30	93.3958683325663\\
        89	502.962905812225\\
        22	6.84463605137342\\
        28	30.8436118137574\\
        25	13.6441680736215\\
        15	3.04232528064302\\
        41	78.2646177322482\\
        32	46.0425989008112\\
        34	35.1536316868496\\
        16	15.9773860926892\\
        88	366.425107836358\\
        39	51.1132869888248\\
        140	1104.30349177925\\
        39	38.3647357762183\\
        49	24.0678484897534\\
        72	148.30540362676\\
        37	17.1166748784462\\
        41	27.0856456061036\\
        283	10694.3669159702\\
        31	22.184716658744\\
        99	398.400067111576\\
        103	620.533173276366\\
        39	24.9600766207824\\
        35	23.6287318728375\\
        98	299.755019783444\\
        45	28.42260984773\\
        73	193.469054674577\\
        34	17.2463209717962\\
        56	48.9711786684488\\
        39	17.9267216934751\\
        37	31.3709034109002\\
        55	46.9290721312557\\
        25	10.6891832911646\\
        29	11.1210569127706\\
        42	57.8470412126995\\
        35	13.0588815772823\\
        26	8.75749615207151\\
        45	65.3911643929573\\
        56	60.244837870085\\
        39	11.9391742844899\\
        62	86.8097755661464\\
        94	272.407258219325\\
        112	654.92039746093\\
        25	7.64124645574208\\
        57	72.974182335944\\
        50	65.9173891830578\\
        47	97.6263436667008\\
        96	244.272398585917\\
        54	120.971005579926\\
        28	4.90039069187048\\
        38	13.4288226015146\\
        47	21.6920288022092\\
        35	16.0800042344566\\
        45	26.1211356264126\\
        50	28.0813027165556\\
        43	38.5213576543652\\
        29	22.2122156877576\\
        42	15.1864424930664\\
        71	88.9198022966985\\
        68	94.4818574789457\\
        149	1036.46247873234\\
        67	124.772689033249\\
        51	21.735938738428\\
        34	12.1224815492461\\
        120	595.834430302694\\
        43	29.1233744717856\\
        117	719.031481968108\\
        39	23.9393254811616\\
        72	77.8621887415518\\
        40	23.171367308371\\
        43	36.71688770357\\
        32	10.9318195658633\\
        49	107.652890376015\\
        41	74.3921372703194\\
        98	374.207992108966\\
        96	228.83979973588\\
        45	18.4819221847899\\
        129	807.609002869321\\
        43	14.9034003743462\\
        93	251.62431099296\\
        41	13.3592819657656\\
        88	349.419702261447\\
        73	131.694504438992\\
        108	436.50120231456\\
        52	53.006432846204\\
        18	6.38694965822758\\
        36	22.4163744097497\\
        50	96.6718932249648\\
        37	17.2054923949993\\
        83	188.809920049495\\
        40	30.9956451862742\\
        26	5.96268343702886\\
        40	53.190362004783\\
        49	55.6267185366156\\
        27	36.5237292245088\\
        41	44.2278979726284\\
        36	37.9515457793217\\
        52	34.4759336267436\\
        37	25.8927310519444\\
        68	135.588943829043\\
        43	37.2681119412443\\
        132	742.281665109493\\
        73	116.283195237917\\
        46	68.4397666376253\\
        47	74.4679059428968\\
        44	38.0646810375424\\
        41	34.9696703174062\\
        30	15.4734135468493\\
        25	7.29853437929289\\
        24	2.70935242453087\\
        85	356.220437102301\\
        44	19.7586897127959\\
        29	8.70484721509163\\
        34	10.3995326130367\\
        101	449.731434113423\\
        18	2.48120402291706\\
        21	3.15093837703114\\
        111	397.902079005116\\
        95	455.542060611277\\
        23	8.83517979995395\\
        31	33.76769608053\\
        84	270.458388401051\\
        23	23.1147829453057\\
        55	83.5762879569899\\
        32	37.5458931078993\\
        27	19.3157856930691\\
        48	86.8689480907807\\
        18	4.23863553931953\\
        28	13.948787065896\\
        40	38.2364604747105\\
        30	18.6822344176352\\
        13	3.67083565344704\\
        33	24.4293991543148\\
        14	7.36327809129481\\
        52	105.753770341713\\
        65	123.367074026716\\
        49	98.9285902033178\\
        35	24.9689466322811\\
        109	697.750975293326\\
        26	12.7344395275036\\
        41	52.7850002477668\\
        39	29.5426837260549\\
        61	120.367420881037\\
        40	36.5211666004454\\
        36	34.1516678713418\\
        21	11.5050456284073\\
        80	217.807477965195\\
        36	27.0134092145702\\
        36	30.5561456009748\\
        38	48.9779446108512\\
        42	65.3986151883739\\
        27	19.9446489737459\\
        46	59.7218389608714\\
        23	11.8081415469247\\
        34	23.4918250127662\\
        200	2892.58375117005\\
        37	29.0624500327908\\
        16	3.91776965369167\\
        50	63.9900173007261\\
        19	5.4616470126747\\
        41	33.0292560090971\\
        25	12.9936442154897\\
        39	61.3637445566803\\
        45	37.6788090221551\\
        83	258.780256886106\\
        27	18.0547299202938\\
        67	163.784228182039\\
        46	49.5574913034486\\
        35	57.6704215975226\\
        44	61.850023991591\\
        30	21.8430252023888\\
        29	28.9195568516962\\
        23	12.4133892981056\\
        43	55.745805795836\\
        28	22.1483646199189\\
        24	12.03401314593\\
        56	122.028257659249\\
        33	34.0181633009693\\
        13	3.52899452344483\\
        41	37.8438854466842\\
        34	41.7679009572519\\
        14	2.69923144260421\\
        83	248.062575837943\\
        33	16.8326173996442\\
        35	63.2582441272547\\
        29	18.3768100152577\\
        54	72.8450547804046\\
        41	56.7771829502471\\
        19	4.77886246913444\\
        34	26.4497727315071\\
        19	11.118254822446\\
        38	32.8774733594555\\
        43	67.842890904133\\
        14	1.2763225550872\\
        17	5.36823748040343\\
        12	2.60795038370574\\
    };
        % This file was created by matlab2tikz.
%
\nextgroupplot[
    % width=4.521in,
    % height=3.566in,
    % at={(0.758in,0.481in)},
    scale only axis,
    xmode=log,
    xmin=10,
    xmax=100000,
    xminorticks=true,
    xlabel style={font=\color{white!15!black}},
    xlabel={$s$},
    ymode=log,
    ymin=1e-07,
    ymax=0.01,
    yminorticks=true,
    ylabel style={font=\color{white!15!black}},
    ylabel={$P(s)$},
    axis background/.style={fill=white},
    % axis x line*=bottom,
    % axis y line*=left,
    xmajorgrids,
    ymajorgrids,
    grid style={dashed},
    % legend style={
    %     legend cell align=left,
    %     align=left,
    %     draw=white!15!black
    % },
    legend style=plotLegend,
    log origin=infty
]
    \addplot[
        ybar interval,
        fill=colore,
        % fill=gray,
        % fill={rgb,1:red,0.50196;green,0.50196;blue,0.50196},
        area legend,
        draw=none
    ] table[
        row sep=crcr,
        x=Lower,
        y expr=\thisrow{Count}
        % y expr={(\thisrow{Count}==0) ? 1e-15 : \thisrow{Count}}
    ] {%
        Lower	Upper	Count\\
        32	46.9815074680496	0.000894758138715566\\
        46.9815074680496	68.9769388740751	0.00134076100066422\\
        68.9769388740751	101.270017776111	0.00149435650903862\\
        101.270017776111	148.681815513684	0.00203566938850381\\
        148.681815513684	218.290494559979	0.00192574057156727\\
        218.290494559979	320.488015636684	0.00141659128751084\\
        320.488015636684	470.531565626727	0.000964867320001916\\
        470.531565626727	690.82132076391	0.000657189517834413\\
        690.82132076391	1014.24459501742	0.000223812151887666\\
        1014.24459501742	1489.08562547625	8.46904008546565e-05\\
        1489.08562547625	2186.23398230868	7.30667114142712e-05\\
        2186.23398230868	3209.7677552106	2.09545808330449e-05\\
        3209.7677552106	4712.49149256661	1.78407053858758e-05\\
        4712.49149256661	6918.74857034808	4.86065776187599e-06\\
        6918.74857034808	10157.9136758552	3.3106866246429e-06\\
        10157.9136758552	14913.5655381873	1.12748587367721e-06\\
        14913.5655381873	21895.6809596123	3.83976056773051e-07\\
        21895.6809596123	32146.6280788143	2.6153340918432e-07\\
        32146.6280788143	47196.7827236696	0\\
        47196.7827236696	69293	0\\
        69293	69293	0\\
    };
    \addlegendentry{Istogramma}

    \addplot [color=blue]
    table[row sep=crcr]{%
        388.329406254705	0.00114686983396336\\
        570.134403125584	0.000533495034511199\\
        837.055428694895	0.000248168487320418\\
        1228.94143357502	0.000115441745686222\\
        1804.29753560314	5.37005999068522e-05\\
        2649.01931698507	2.49801699828243e-05\\
        3889.21627574818	1.16201475114466e-05\\
        5710.0388594976	5.4054006950561e-06\\
        8383.31979126063	2.51445660610857e-06\\
        12308.15629313	1.16966204370117e-06\\
        18070.4917751128	5.44097397883717e-07\\
        26530.5920088684	2.53100440403337e-07\\
    };
    \addlegendentry{Regressione}
        % This file was created by matlab2tikz.
%
\definecolor{mycolor1}{rgb}{0.00000,0.44700,0.74100}%
%
\nextgroupplot[%
    % width=4.521in,
    % height=3.566in,
    % at={(0.758in,0.481in)},
    height=\halfHeightPlot cm,
    scale only axis,
    xmode=log,
    xmin=10,
    xmax=283,
    % xminorticks=true,
    xtickten={1,2,3},
    xlabel style={font=\color{white!15!black}},
    xticklabels=\empty,
    % xlabel={$k$},
    ymode=log,
    % ymin=51.7058402006449,
    % ymax=239.173547251027,
    ymin=10,
    ymax=1000,
    yminorticks=true,
    ylabel style={font=\color{white!15!black}},
    ylabel={$k_{nn}^w(k)$},
    axis background/.style={fill=white},
    % axis x line*=bottom,
    % axis y line*=left,
    xmajorgrids,
    ymajorgrids,
    grid style={dashed}
]
    \addplot[
        only marks,
        mark=*,
        mark options={},
        mark size=1.5000pt,
        color=black,
        fill=black,
        forget plot
    ] table[row sep=crcr]{%
        x	y\\
        10	90.7659574468085\\
        11	123.125\\
        12	103.16\\
        13	97.0215827338129\\
        14	87.472972972973\\
        15	83.6382978723404\\
        16	101.102521008403\\
        17	89.434456928839\\
        18	81.0640243902439\\
        19	99.7971014492754\\
        20	112.584337349398\\
        21	89.0989010989011\\
        22	78.7355072463768\\
        23	103.068376068376\\
        24	126.060240963855\\
        25	118.303880597015\\
        26	108.964928615767\\
        27	105.287221905305\\
        28	103.751041184637\\
        29	99.5552387740556\\
        30	101.399585921325\\
        31	94.8460921843687\\
        32	103.071215510813\\
        33	108.202391118702\\
        34	116.456153846154\\
        35	92.389226519337\\
        36	120.750543478261\\
        37	120.831153388823\\
        38	108.214364640884\\
        39	114.794029352902\\
        40	119.897660818713\\
        41	127.982328686432\\
        42	111.902226524685\\
        43	110.648392821929\\
        44	105.732180663373\\
        45	121.52722513089\\
        46	129.118884316415\\
        47	167.834926375614\\
        48	115.055004955401\\
        49	109.794285714286\\
        50	143.99156626506\\
        51	129.527538403897\\
        52	205.90963060686\\
        53	131.846755277561\\
        54	124.982682512733\\
        55	110.335412026726\\
        56	135.259582863585\\
        57	168.340934371524\\
        58	126.459435626102\\
        59	85.1824\\
        61	147.031613976706\\
        62	127.914043583535\\
        63	68.3402692778458\\
        64	93.5602379771939\\
        65	168.014519906323\\
        67	158.128321451717\\
        68	181.765517241379\\
        69	93.7681895093063\\
        70	111.147068906411\\
        71	84.9118437555993\\
        72	185.472670250896\\
        73	186.912613122172\\
        76	51.7058402006449\\
        77	100.899613899614\\
        80	127.672243346008\\
        82	85.605493133583\\
        83	190.114580389941\\
        84	75.9946091644205\\
        85	175.970965940815\\
        86	67.1205967675093\\
        87	86.8542449286251\\
        88	162.978809869376\\
        89	55.5067542898868\\
        90	94.8354898336414\\
        91	54.9750453720508\\
        93	99.1987784564131\\
        94	125.994614607876\\
        95	114.96401308615\\
        96	146.162619737135\\
        97	112.637260350097\\
        98	155.530179084678\\
        99	239.173547251027\\
        101	217.956919763059\\
        103	87.030303030303\\
        108	224.160127513487\\
        109	88.3833560709413\\
        111	212.979899497487\\
        112	102.97824499927\\
        117	113.29633911368\\
        120	129.969928922909\\
        127	90.932382461701\\
        129	209.865220759101\\
        132	127.887096774194\\
        135	53.8669915177766\\
        140	195.225798615329\\
        149	237.026167265264\\
        154	62.2550379481811\\
        182	71.588102621934\\
        200	70.003553099092\\
        206	67.7991107518189\\
        283	106.299813834009\\
    };
        % This file was created by matlab2tikz.
%
\definecolor{mycolor1}{rgb}{0.00000,0.44700,0.74100}%
%
\nextgroupplot[%
    % width=4.521in,
    % height=3.566in,
    % at={(0.758in,0.481in)},
    height=\halfHeightPlot cm,
    yshift=\yPlotSepPartial em,
    scale only axis,
    xmode=log,
    xmin=10,
    xmax=283,
    xminorticks=true,
    xtickten={1,2,3},
    xlabel style={font=\color{white!15!black}},
    xlabel={$k$},
    ymode=log,
    % ymin=0.001,
    % ymax=10,
    ymin=1,
    ymax=1000,
    ytickten={0,1,2,3},
    yminorticks=true,
    ylabel style={font=\color{white!15!black}},
    ylabel={$k_{nn,rel}^w(k)[\%]$},
    axis background/.style={fill=white},
    % axis x line*=bottom,
    % axis y line*=left,
    xmajorgrids,
    ymajorgrids,
    grid style={dashed}
]
    \addplot[
        only marks,
        mark=*,
        mark options={},
        mark size=1.5000pt,
        color=black,
        fill=black,
        forget plot
    ] table[
        row sep=crcr,
        y expr=\thisrow{y}*100
    ]{%
        x	y\\
        10	0.0164160968287628\\
        11	0.415229885057471\\
        12	0.14941504178273\\
        13	0.275738950309745\\
        14	0.161219060896664\\
        15	0.171951861826349\\
        16	0.249722138546395\\
        17	0.218583303064597\\
        18	0.0844685537156375\\
        19	0.440401798493035\\
        20	0.606740935484481\\
        21	0.334576977943597\\
        22	0.300436305871088\\
        23	0.405755672804497\\
        24	0.836274449582745\\
        25	0.59870108914885\\
        26	0.670699185616944\\
        27	0.569249039862511\\
        28	0.481779726176909\\
        29	0.456171111859925\\
        30	0.403670342677729\\
        31	0.422855306187903\\
        32	0.606118419090231\\
        33	0.646384593746387\\
        34	0.65982361382068\\
        35	0.344262285669007\\
        36	0.679358533983926\\
        37	0.70704569506928\\
        38	0.628830648955712\\
        39	0.706765243217723\\
        40	0.610265170480763\\
        41	0.757940124005403\\
        42	0.605374202089352\\
        43	0.549156103622195\\
        44	0.549602274728008\\
        45	0.636961658230301\\
        46	0.763395486774864\\
        47	1.31346115206875\\
        48	0.685478922620786\\
        49	0.588965680193751\\
        50	0.947498754768409\\
        51	0.912399061619375\\
        52	1.73377892550385\\
        53	0.875688640373295\\
        54	0.872529612454077\\
        55	0.63218065128293\\
        56	0.784062959965635\\
        57	1.41424915314552\\
        58	1.49477798173943\\
        59	0.554519517476029\\
        61	0.930047009377886\\
        62	0.717896827072279\\
        63	0.058367002090532\\
        64	0.340989917818802\\
        65	1.35873516067192\\
        67	1.19032407220696\\
        68	1.40140959246431\\
        69	0.845935827715302\\
        70	0.762241183114099\\
        71	0.294276708168217\\
        72	1.52010421929883\\
        73	1.55389707223332\\
        76	0.00374044833946731\\
        77	0.584272077950708\\
        80	0.968731585906054\\
        82	0.400289335119451\\
        83	1.74712920828083\\
        84	0.318097701798745\\
        85	1.33055969226695\\
        86	0.286178993316801\\
        87	0.388263698105894\\
        88	1.26234486450116\\
        89	-0.11768152673693\\
        90	0.55723300219444\\
        91	-0.117217376238464\\
        93	0.491349239645396\\
        94	0.759544461913587\\
        95	0.976399066808595\\
        96	1.43463473015008\\
        97	1.5186293808113\\
        98	1.26091486320529\\
        99	2.54357694969346\\
        101	2.19223446868749\\
        103	0.312078631750763\\
        108	2.51623729432921\\
        109	0.548093493770304\\
        111	2.22476726834281\\
        112	0.565996393743141\\
        117	0.77570953466853\\
        120	1.05892956709558\\
        127	0.687870881706523\\
        129	2.37248377177503\\
        132	1.02800297623661\\
        135	0.113124729052478\\
        140	2.16374717052275\\
        149	2.81227319975436\\
        154	0.304568763644018\\
        182	0.283016708733825\\
        200	0.283056325129985\\
        206	0.248245313689758\\
        283	1.11137333766315\\
    };
    \end{groupplot}

    %\begingroup Didascalie
        \tikzset{SubCaption/.style={
            text width=\plotWidth cm,
            anchor=north,
            align=center,
            yshift=-3em
        }}
        \node[SubCaption] at (Col1 c1r1.south)
            {\dimTesto{\subCaptionSize}\sottoDidascalia{Distribuzione di probabilità dei gradi.\label{figDeMontisPk}}};
        \node[SubCaption] at (Col2 c1r1.south)
            {\dimTesto{\subCaptionSize}\sottoDidascalia{Coefficiente d'aggregrazione \(C(k)\).\label{figDeMontisCk}}};

        \node[SubCaption] at (Col1 c1r2.south)
            {\dimTesto{\subCaptionSize}\sottoDidascalia{Assortatività \(k_{nn}(k)\).\label{figDeMontisKnnk}}};
        \node[SubCaption] at (Col2 c1r2.south)
            {\dimTesto{\subCaptionSize}\sottoDidascalia{Centralità per intermediazione \(g(i)\).\label{figDeMontisGi}}};
    
        \node[SubCaption] at (Col1 c1r3.south)
            {\dimTesto{\subCaptionSize}\sottoDidascalia{Distribuzione di probabilità dei pesi.\label{figDeMontisWk}}};
        \node[SubCaption] at (Col2 c1r3.south)
            {\dimTesto{\subCaptionSize}\sottoDidascalia{Distribuzione di probabilità delle forze.\label{figDeMontisSk}}};

        
        \pgfmathparse{1.5*\plotWidth}\node[SubCaption,text width=\pgfmathresult cm] at (Col1 c1r5.south)
            {\dimTesto{\subCaptionSize}\sottoDidascalia{Coefficiente [relativo] d'aggregazione pesato \(C^w(k)\).\label{figDeMontisCwk}}};
        \node[SubCaption] at (Col2 c1r5.south)
            {\dimTesto{\subCaptionSize}\sottoDidascalia{Assortatività [relativa] pesata \(k_{nn}^w(k)\).\label{figDeMontisKnnwk}}};
    
        % https://tex.stackexchange.com/questions/632045/get-subcaption-for-every-plot-using-groupplot-pgf-tikz-after-2020-update
    %\endgroup

    \redefineTikZbounds{Col1 c1r1}{Col2 c1r1}
\end{tikzpicture}%
