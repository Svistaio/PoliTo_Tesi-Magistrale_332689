
% !TEX root = ../../../Esperimenti/.tex/MEF.tex
% LTeX: language=it

\begin{tikzpicture}[%
    node distance=5em,
    main node/.style={
        circle,
        draw,
        thick,
        fill=white,
        inner sep=0pt,
        minimum size=1.25em
    }, % Ponendo la separazione interna nulla la dimensione del cerchio sarà data dalla chiave «minimum size», oltre che dalla dimensione del testo stesso
    edge/.style={thick}
]
    % \renewcommand\subCaptionSize{10}

    \pgfmathsetmacro\radius{1.5}
    \newcommand\hSepPlot{10em}
    \newcommand\hSepLabel{1.5em}

    % Grafo Indiretto a forma di triangolo equilatero
    \node[main node] (A) at (90:\radius) {1};
    \node[main node] (B) at (330:\radius) {2};
    \node[main node] (C) at (210:\radius) {3};
    % Coordinate polari: (angolo:raggio)

    \draw[edge] (A) -- (B);
    \draw[edge] (B) -- (C);
    \draw[edge] (C) -- (A);

    \node[left=\hSepLabel of A] {\dimTesto{\subCaptionSize}\sottoDidascalia[]{}};
    % \node[below=\hSepLabel of 1] {\textbf{Grafo Indiretto}};
    
    % Grado diretto a forma di triangolo equilatero
    \begin{scope}[xshift=\hSepPlot]
        \node[main node] (A) at (90:\radius) {1};
        \node[main node] (B) at (330:\radius) {2};
        \node[main node] (C) at (210:\radius) {3};

        \draw[edge, -{Stealth}] (A) edge[bend left=15] (B);
        \draw[edge, -{Stealth}] (B) edge[bend left=15] (A);

        \draw[edge, -{Stealth}] (B) edge[bend left=15] (C);
        \draw[edge, -{Stealth}] (C) edge[bend left=15] (B);

        \draw[edge, -{Stealth}] (C) edge[bend left=15] (A);
        \draw[edge, -{Stealth}] (A) edge[bend left=15] (C);

        \node[left=\hSepLabel of A] {\dimTesto{\subCaptionSize}\sottoDidascalia[]{}};
        % \node[below=\hSepLabel of A] {\textbf{Grafo Diretto Simmetrico}};
    \end{scope}

    % Matrice d'adiacenza
    \pgfmathparse{1.9*\hSepPlot}
    \begin{scope}[xshift=\pgfmathresult]
        \node[
            inner sep=0pt,
            outer sep=0pt
        ] (M) {
            \(
                A = 
                \begin{bmatrix}
                    0 & 1 & 1 \\
                    1 & 0 & 1 \\
                    1 & 1 & 0 
                \end{bmatrix}%
            \)%
        };%

        \node at (90:\radius) {\dimTesto{\subCaptionSize}\sottoDidascalia[]{}};
        % \pgfmathparse{\radius/sqrt(3)}
        % \node[above=\pgfmathresult em of M] {\sottoDidascalia[]{}};
    \end{scope}%
\end{tikzpicture}%