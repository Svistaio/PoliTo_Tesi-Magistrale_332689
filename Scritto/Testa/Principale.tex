
% !LW recipe=pdfLaTeX ➞ biber ➞ pdfLaTeX
% !TEX root = Principale.tex
% LTeX: language=it

\documentclass[
    a4paper,
    twoside,
    openright,
    titlepage,
    headinclude,
    footinclude,
    BCOR5mm,
    numbers=noenddot,
    cleardoublepage=empty,
    tablecaptionabove,
]{scrreprt}


% !TEX root = ../Testa/Principale.tex

\newcommand\bozza{true}
% \newcommand\bozza{false}

\usepackage[
    pass,
    showframe=\bozza
]{geometry}

\usepackage[T1]{fontenc}
% \usepackage[utf8]{inputenc}
\usepackage[utf8]{inputenx}
\usepackage{microtype}

\usepackage[italian]{babel}

\usepackage{csquotes}
\usepackage[
    backend=biber,
    style=numeric-comp
]{biblatex}
\DeclareDelimFormat{finalnamedelim}{\addspace\&\space}
\addbibresource{../Corpo/Bibliografia.bib}
\nocite{*}

\usepackage[
    drafting=\bozza,
    eulerchapternumbers,
    % subfig,
    beramono,
    eulermath,
    pdfspacing,
    dottedtoc=true
]{classicthesis}
\usepackage{arsclassica}
\usepackage{scrhack}

\usepackage{tikz}
\usetikzlibrary{
    % external,
    positioning,
    shapes.geometric,
    plotmarks,
    arrows.meta,
    calc,
    decorations.pathreplacing,
    backgrounds,
    patterns.meta
} 
% \immediate\write18{if not exist "../../Figure/.fnl/TikZpdf" mkdir "../../Figure/.fnl/TikZpdf"}
% \tikzexternalize[prefix={../../Figure/.fnl/TikZpdf/}]
\usepackage{pgfplots}
\pgfplotsset{compat=newest}
\usepgfplotslibrary{
    patchplots,
    groupplots,
    colorbrewer
}
\usepackage{grffile}
\usepackage{nicematrix}

\usepackage{zref-savepos}
\newcounter{figposcount} % Crea un contatore unico per tracciare la posizione
\usepackage{xstring}
    
\usepackage{accents}
\usepackage{
    amsmath,
    amsfonts,
    amsthm,
    % amssymb
    bbold
}

\usepackage{
    graphicx,
    caption,
    subcaption,
    float,
    stfloats
}

\usepackage[
    italiano,
    italianokw,
    ruled,
    vlined,
    linesnumbered
]{algorithm2e}
% \usepackage{listings}

\usepackage[left]{lineno}
\usepackage[
    section,
    newfloat
]{minted}
\usepackage{newunicodechar}

\usepackage{enumitem}
\usepackage{ragged2e} 

% \usepackage{hyperref} % Già caricato da «classicthesis» colle impostazioni
\PassOptionsToPackage{hyphens}{url}
\hypersetup{
    colorlinks=true,
    linktocpage=true,
    breaklinks=true,
    urlcolor=CTurl,
    linkcolor=CTlink,
    citecolor=CTcitation,
    % destlabel=true % Migliora la precisione dei nessi ipertestuali
} % Si v. la linea 732 di «classicthesis.sty»
% \hypersetup{
%     colorlinks=false,
%     linkbordercolor=CTlink,
%     citebordercolor=CTcitation,
%     urlbordercolor=CTurl
%     linktocpage=true,
%     breaklinks=true,
%     pdfborder={1 1 1},
% }
% \usepackage[all]{hypcap}
\usepackage{xurl}

\usepackage[italian]{cleveref}

\usepackage{siunitx}
\sisetup{
    exponent-mode=threshold,
    exponent-thresholds=-3:3,
    round-mode=figures,
    round-precision=3,
    round-pad=false,
    separate-uncertainty=true,
    print-unity-mantissa=false,
    tight-spacing=true,
}

\usepackage[normalem]{ulem}
\usepackage{circledsteps}


% !TEX root = ../Testa/Principale.tex

%\begingroup Tipografia
    \newcommand\dimTesto[1]{%
        \pgfmathparse{#1*1.2}%
        \fontsize{#1 pt}{\pgfmathresult pt}\selectfont%
    }

    \newcommand\daRivedere{{\color[HTML]{DC143C}\{§\}}}
    \newcommand\dubbio{{\color[HTML]{DC143C}\{?\}}}

    \newcommand\tref[2]{\hyperref[#1]{#2}}
%\endgroup

%\begingroup Citazioni
    \newcommand\etal{\textit{et al.}}

    % Comando per definire citazioni con note separate da un punto e virgola: [citazione1, nota1; citazione2, nota2; ecc.]
    \DeclareMultiCiteCommand{\bracketcitesemi}[\mkbibbrackets]{\cite}{\addsemicolon\space}
 
    \crefname{equation}{}{}
    \crefrangeformat{equation}{(#3#1#4--#5#2#6)}
    \crefmultiformat{equation}{(#2#1#3}{,~#2#1#3)}{,~#2#1#3}{,~#2#1#3)}
    \crefrangemultiformat{equation}{(#3#1#4--#5#2#6}{,~#3#1#4--#5#2#6)}{,~#3#1#4--#5#2#6}{,~#3#1#4--#5#2#6)}
    
    \crefname{chapter}{Cap.}{Capp.}
    \crefrangeformat{chapter}{Capp.~#3#1#4--#5#2#6}
    \crefmultiformat{chapter}{Capp.~#2#1#3}{ e~#2#1#3}{, #2#1#3}{ e~#2#1#3}
    \crefrangemultiformat{chapter}{Capp.~#3#1#4--#5#2#6}{ e~#3#1#4--#5#2#6}{, #3#1#4--#5#2#6}{ e~#3#1#4--#5#2#6}
    
    \crefname{section}{§}{§§}
    \crefrangeformat{section}{§§~#3#1#4--#5#2#6}
    \crefmultiformat{section}{§§~#2#1#3}{ e~#2#1#3}{, #2#1#3}{ e~#2#1#3}
    \crefrangemultiformat{section}{§§~#3#1#4--#5#2#6}{ e~#3#1#4--#5#2#6}{, #3#1#4--#5#2#6}{ e~#3#1#4--#5#2#6}
    
    \crefname{figure}{Fig.}{Figg.}
    \crefrangeformat{figure}{Figg.~#3#1#4--#5#2#6}
    \crefmultiformat{figure}{Figg.~#2#1#3}{ e~#2#1#3}{, #2#1#3}{ e~#2#1#3}
    \crefrangemultiformat{figure}{Figg.~#3#1#4--#5#2#6}{ e~#3#1#4--#5#2#6}{, #3#1#4--#5#2#6}{ e~#3#1#4--#5#2#6}
    
    \crefname{table}{Tab.}{Tabb.}
    \crefrangeformat{table}{Tabb.~#3#1#4--#5#2#6}
    \crefmultiformat{table}{Tabb.~#2#1#3}{ e~#2#1#3}{, #2#1#3}{ e~#2#1#3}
    \crefrangemultiformat{table}{Tabb.~#3#1#4--#5#2#6}{ e~#3#1#4--#5#2#6}{, #3#1#4--#5#2#6}{ e~#3#1#4--#5#2#6}

    \crefname{algorithm}{Alg.}{Algg.}
    \crefrangeformat{figure}{Algg.~#3#1#4--#5#2#6}
    \crefmultiformat{figure}{Algg.~#2#1#3}{ e~#2#1#3}{, #2#1#3}{ e~#2#1#3}
    \crefrangemultiformat{figure}{Algg.~#3#1#4--#5#2#6}{ e~#3#1#4--#5#2#6}{, #3#1#4--#5#2#6}{ e~#3#1#4--#5#2#6}
    
    \crefname{prp}{Prop.}{Propp.}
    \crefrangeformat{prp}{Propp.~#3#1#4--#5#2#6}
    \crefmultiformat{prp}{Propp.~#2#1#3}{ e~#2#1#3}{, #2#1#3}{ e~#2#1#3}
    \crefrangemultiformat{prp}{Propp.~#3#1#4--#5#2#6}{ e~#3#1#4--#5#2#6}{, #3#1#4--#5#2#6}{ e~#3#1#4--#5#2#6}

    \crefname{dfn}{Def.}{Deff.}
    \crefrangeformat{dfn}{Deff.~#3#1#4--#5#2#6}
    \crefmultiformat{dfn}{Deff.~#2#1#3}{ e~#2#1#3}{, #2#1#3}{ e~#2#1#3}
    \crefrangemultiformat{dfn}{Deff.~#3#1#4--#5#2#6}{ e~#3#1#4--#5#2#6}{, #3#1#4--#5#2#6}{ e~#3#1#4--#5#2#6}

    \crefname{oss}{Oss.}{Oss.ni}
    \crefrangeformat{oss}{Oss.ni~#3#1#4--#5#2#6}
    \crefmultiformat{oss}{Oss.ni~#2#1#3}{ e~#2#1#3}{, #2#1#3}{ e~#2#1#3}
    \crefrangemultiformat{oss}{Oss.ni~#3#1#4--#5#2#6}{ e~#3#1#4--#5#2#6}{, #3#1#4--#5#2#6}{ e~#3#1#4--#5#2#6}
    % https://www.treccani.it/magazine/lingua_italiana/domande_e_risposte/grammatica/grammatica_123.html

    \crefname{teo}{Teo.}{Teo.ri}
    \crefrangeformat{teo}{Teo.ri~#3#1#4--#5#2#6}
    \crefmultiformat{teo}{Teo.ri~#2#1#3}{ e~#2#1#3}{, #2#1#3}{ e~#2#1#3}
    \crefrangemultiformat{teo}{Teo.ri~#3#1#4--#5#2#6}{ e~#3#1#4--#5#2#6}{, #3#1#4--#5#2#6}{ e~#3#1#4--#5#2#6}

    \crefname{ipo}{Ip.}{Ipp.}
    \crefrangeformat{ipo}{Ipp.~#3#1#4--#5#2#6}
    \crefmultiformat{ipo}{Ipp.~#2#1#3}{ e~#2#1#3}{, #2#1#3}{ e~#2#1#3}
    \crefrangemultiformat{ipo}{Ipp.~#3#1#4--#5#2#6}{ e~#3#1#4--#5#2#6}{, #3#1#4--#5#2#6}{ e~#3#1#4--#5#2#6}
%\endgroup

%\begingroup TikZ
    \newcommand\didascalia[2]{%
        \captionsetup{width=#2} % aboveskip=-.2cm,
        \caption{#1}%
    }
    
    \newcommand\sottoDidascalia[2][ ]{%
        \refstepcounter{subfigure}%
        (\thesubfigure)#1#2%
    }
    
    \makeatletter
        \newcommand\tikzfigure{\@ifstar{\tikzfigure@star}{\tikzfigure@nostar}}
        
        \newcommand\tikzfigure@nostar[6]{%
            \begin{figure}[#1]%
                \centering%
                \IfSubStr{#1}{p}{% Se p è all'interno delle opzioni della figura, allora questa è centrata rispetto alla pagina e non al corpo del testo definito da ArsClassica
                    \leavevmode% Forza la modalità orizzontale cosí da far leggere a «zref» la corretta linea d'inizio
                    \stepcounter{figposcount}%
                    \zsaveposx{figpos-\thefigposcount}%
                    \makebox[0pt][l]{%
                        % Muove indietro il cursore del testo fino al lato sinistro del foglio
                        \hspace*{-\dimexpr\zposx{figpos-\thefigposcount}sp\relax}%
                        % Muove avanti il cursore del testo fino al centro del foglio
                        \hspace*{0.5\paperwidth}%
                        % Centra la figura TikZ riseptto a questo punto
                        \makebox[0pt][c]{%
                            \begin{minipage}{#2}%
                                \centering
                                \addtocounter{figure}{1}%
                                \setcounter{subfigure}{0}%
                                \resizebox{\linewidth}{!}{#3}%
                                \addtocounter{figure}{-1}%
                                \didascalia{#4}{#5}%
                                \label{#6}%
                            \end{minipage}%
                        }%
                    }%
                }{% Altrimenti centra la figura rispetto al corpo del testo
                    \makebox[\textwidth][c]{%
                        \begin{minipage}{#2}%
                            \centering
                            \addtocounter{figure}{1}%
                            \setcounter{subfigure}{0}%
                            \resizebox{\linewidth}{!}{#3}%
                            \addtocounter{figure}{-1}%
                            \didascalia{#4}{#5}%
                            \label{#6}%
                        \end{minipage}%
                    }%
                }% Cosa che non richiede l'uso del pacchetto «zref»
            \end{figure}%
        }
        \newcommand\tikzfigure@star[5]{%
            \begin{figure}[#1]%
                \centering%
                \IfSubStr{#1}{p}{% Se p è all'interno delle opzioni della figura, allora questa è centrata rispetto alla pagina e non al corpo del testo definito da ArsClassica
                    \leavevmode% Forza la modalità orizzontale cosí da far leggere a «zref» la corretta linea d'inizio
                    \stepcounter{figposcount}%
                    \zsaveposx{figpos-\thefigposcount}%
                    \makebox[0pt][l]{%
                        % Muove indietro il cursore del testo fino al lato sinistro del foglio
                        \hspace*{-\dimexpr\zposx{figpos-\thefigposcount}sp\relax}%
                        % Muove avanti il cursore del testo fino al centro del foglio
                        \hspace*{0.5\paperwidth}%
                        % Centra la figura TikZ riseptto a questo punto
                        \makebox[0pt][c]{%
                            \begin{minipage}{\linewidth}%
                                \centering
                                \addtocounter{figure}{1}%
                                \setcounter{subfigure}{0}%
                                #2%
                                \addtocounter{figure}{-1}%
                                \didascalia{#3}{#4}%
                                \label{#5}%
                            \end{minipage}%
                        }%
                    }%
                }{% Altrimenti centra la figura rispetto al corpo del testo
                    \makebox[\textwidth][c]{%
                        \begin{minipage}{\linewidth}%
                            \centering
                            \addtocounter{figure}{1}%
                            \setcounter{subfigure}{0}%
                            #2%
                            \addtocounter{figure}{-1}%
                            \didascalia{#3}{#4}%
                            \label{#5}%
                        \end{minipage}%
                    }%
                }% Cosa che non richiede l'uso del pacchetto «zref»
            \end{figure}%
        }
    \makeatother

    \newcommand\redefineTikZbounds[2]{
        % Salva le coordinate della cima e del fondo della figura corrente (cosí da mantenere i titoli e le didascalie)
        \coordinate (Cima) at (current bounding box.north);
        \coordinate (Fondo) at (current bounding box.south);

        \pgfresetboundingbox % Resetta la scatola dei limiti

        \useasboundingbox % Definisce la nuova scatola come un rettangolo
            (#1.west |- Cima)   % con ascissa sinistra data da «Col1 c1r1.west», 
            rectangle           % ascissa destra data da «Col2 c1r1.east»
            (#2.east |- Fondo); % ordinate date dalla cima e fondo originali
            % L'operatore «|-» in «(A |- B)» prende come ascissa quella del punto A mentre come coordinata quella del punto B
    }
%\endgroup

%\begingroup Liste
    \setlist[enumerate]{
        topsep=0.5em,
        parsep=0em,
        itemsep=0.25em,
        leftmargin=2em,
        rightmargin=1.5em,
        % \leftmargin + \itemindent = \labelindent + \labelwidth + \labelsep
        %itemindent=!,
        %labelindent=30pt,
        %labelwidth=!,
        %labelsep=!,
    }

    \setlist[itemize]{
        topsep=0.5em,
        parsep=0em,
        itemsep=0.25em,
        leftmargin=2em,
        rightmargin=1.5em,
        % \leftmargin + \itemindent = \labelindent + \labelwidth + \labelsep
        %itemindent=!,
        labelindent=30pt,
        %labelwidth=!,
        %labelsep=!,
    }
%\endgroup

%\begingroup Algoritmi
    \SetKwIF{Sea}{AltSe}{Altrimenti}{se}{allora}{altrimenti se}{altrimenti}{fine se}
    \SetKwInput{KwDati}{Dati}
    \SetKwFor{Per}{per}{fai}{fine per}
    \SetKwRepeat{Ripeti}{ripeti}{finché}

    \newcommand\pythoncommfont[1]{\footnotesize\ttfamily\textcolor{gray}{#1}}
    \SetCommentSty{pythoncommfont}
    \SetKwComment{tcp}{\# }{}
%\endgroup

%\begingroup Matematica
    \numberwithin{equation}{chapter}
    \numberwithin{figure}{chapter}
    \numberwithin{table}{chapter}

    \newcommand\SpazMate[3]{ % Spaz[iatura] Mate[matica]
        \thinmuskip = #1mu\relax
        \medmuskip = #2mu\relax
        \thickmuskip = #3mu\relax
    }
    \SpazMate123

    \newcommand\de[1][]{\mathrm{d}#1}
    \newcommand\derS[3][d]{\frac{\mathrm{#1}#2}{\mathrm{#1}#3}}
    \newcommand\derP[2]{\frac{\partial#1}{\partial#2}}
    
    \DeclareMathOperator\DstrLognormale{Lognormale}
    \DeclareMathOperator\DstrBernoulli{Bernoulli}
    \DeclareMathOperator\DstrEsponenziale{Exp}
    \DeclareMathOperator\DstrGamma{Gamma}

    % \newcommand\independent{\perp\!\!\!\!\!\!\!\!\!\!\perp}
    \newcommand\independent{\perp\hspace{-.5em}\perp}

    % \newcommand\sepEnv{0.5em}
    % \mdfdefinestyle{ars_style}{
    %     linecolor=gray!40,
    %     linewidth=0.5pt,
    %     roundcorner=0pt,
    %     innertopmargin=\sepEnv,
    %     innerbottommargin=\sepEnv,
    %     innerleftmargin=0pt,
    %     innerrightmargin=0pt,
    %     backgroundcolor=gray!5,
    %     skipabove=\baselineskip,
    %     skipbelow=\baselineskip
    % }

    % \newtheoremstyle{compact}
    %     {0pt}{0pt}{\itshape}{0pt}{\bfseries}
    %     {.}{5pt plus 5pt minus 5pt}{}
    % \theoremstyle{compact}

    % \newmdtheoremenv[style=ars_style]{dfn}{Definizione}[chapter]
    % \newmdtheoremenv[style=ars_style]{ipo}{Ipotesi}[chapter]
    % \newmdtheoremenv[style=ars_style]{oss}{Osservazione}[chapter]
    % \newmdtheoremenv[style=ars_style]{prt}{Proprietà}[chapter]

    % \newmdtheoremenv[style=ars_style]{prp}{Proposizione}[chapter]
    % \newmdtheoremenv[style=ars_style]{lmm}{Lemma}[chapter]
    % \newmdtheoremenv[style=ars_style]{teo}{Teorema}[chapter]
    % \newmdtheoremenv[style=ars_style]{crl}{Corollario}[chapter]

    \theoremstyle{definition}
    \newtheorem{dfn}{Definizione}[chapter]
    \newtheorem{ipo}{Ipotesi}[chapter]
    \newtheorem{oss}{Osservazione}[chapter]
    \newtheorem{prt}{Proprietà}[chapter]

    \theoremstyle{plain}
    \newtheorem{prp}{Proposizione}[chapter]
    \newtheorem{lmm}{Lemma}[chapter]
    \newtheorem{teo}{Teorema}[chapter]
    \newtheorem{crl}{Corollario}[chapter]

    \newcommand\RPlus{\mathbb R_+}
    % \newcommand\RPlusStar{\mathbb R^*_+}
    \newcommand\NPlus{\mathbb N_+}

    \makeatletter
        \newcommand\eqtag{\@ifstar{\eqtag@star}{\eqtag@nostar}}
        
        \newcounter{eqtag}
        \newcommand\eqtag@nostar[2]{%
            \begin{equation*}%
                \refstepcounter{eqtag}%
                % \def\@currentlabeltype{eqtag}%
                \protected@edef\@currentlabel{#1}%
                #2%
                \tag*{#1}%
            \end{equation*}%
        }
        \newcommand\eqtag@star[2]{%
            \begin{equation*}%
                \refstepcounter{eqtag}%
                % \def\@currentlabeltype{eqtag}%
                \protected@edef\@currentlabel{#1}%
                #2%
                \notag%
            \end{equation*}%
        }
    \makeatother

    \DeclareMathAlphabet\pazocal{OMS}{zplm}{m}{n}

    % \newcommand\underdot[1]{\underaccent{\scalebox{0.3}{\(\bullet\)}}{#1}}
    % \newcommand\underhat[1]{\underaccent{\check}{#1}}
    \newcommand\underhat[1]{\underaccent{\hat}{#1}} % Circonflesso sottoscritto
    
%\endgroup

%\begingroup Frontespizio
    \newcommand\lineaPuntallata[1][4cm]{\makebox[#1]{\dotfill}}
    \newcommand\Frontespizio{
        \begin{titlepage}
            \newgeometry{
                left=1cm,
                right=1cm,
                top=3cm,
                bottom=3.5cm
            }
            \centering
            %
            {\Huge\scshape Politecnico di Torino}\\[1cm]
            {\Large Corso di Laurea\\in Ingegneria Matematica}
            %
            \vfill
            %
            {\Large Tesi di Laurea Magistrale}\\[0.5cm]
            \parbox{.9\textwidth}{
                \centering\huge\bfseries%
                Modellizzazione della distribuzione della popolazione tra città su reti spaziali mediante la teoria cinetica dei sistemi multiagente
            }\\[1.5cm]
            \includegraphics[width=0.25\textwidth]{../../Figure/Scelte/Altro/LogoPoliTO.jpg}
            %
            \vfill
            %
            \begin{minipage}[t]{0.85\textwidth}
                \begin{flushleft}\large
                    \textbf{Relatori} \hfill \textbf{Candidato}\\
                    prof. Andrea Tosin \hfill Valerio Taralli\\
                    prof. Nome Cognome\\[0.5cm]
                    \textit{firma dei relatori} \hfill \textit{firma del candidato}\\[0.35cm]
                    \lineaPuntallata\ \hfill \\
                    \lineaPuntallata\ \hfill \lineaPuntallata
                \end{flushleft}
            \end{minipage}
            %
            \vfill
            %
            {\large Anno Accademico 2025-2026\par}
            %
            \restoregeometry
        \end{titlepage}
    }
%\endgroup

%\begingroup Dedica
    \newcommand\Dedica{
        \thispagestyle{empty}
        \phantomsection
        \pdfbookmark[1]{Dedica}{Dedica}
    
        \vspace*{3cm}
    
        \begin{center}
            \Large
            Ai miei genitori, \\
            \emph{Elisabetta} e \emph{Marco}
        \end{center}
    }
%\endgroup

%\begingroup Altro
    \newcommand\TCSMA{\hyperlink{acrTeorieCineticheSistemiMultiAgenti}{TCSMA}}
%\endgroup


\begin{document}

    \Frontespizio
    \Dedica

    %%%
    %%%
    %%%

    \tableofcontents
    \linenumbers

    %%%
    %%%
    %%%

    % \part{Introduzione}
    \chapter{Introduzone}

    %%%
    %%%
    %%%

    % \part{Teoria}
    \chapter{\texorpdfstring
        {Teorie cinetiche dei\newline sistemi multiagente}
        {Teorie cinetiche dei sistemi multiagente}%
    }\label{secTCSM}
        \section{Descrizione cinetica classica}

        \section{Descrizione cinetica retale}\label{secDescrizioneCinetica} %  d'interazioni mediante da un grafo
            
            \subsection{Impostazione}

                La popolazione degli agenti evolve a causa delle interazioni con altri agenti connessi. Seguendo la teoria cinetica collisionale, l'ipotesi fondamentale ipotizzata è che solo le interazioni binarie siano rilevanti: le interazioni fra tre o piú agenti possono essere trascurate.

                Dopodiché, sia \(X\in\mathcal I\) la posizione di un agente sul grafo \(\mathcal G=(\mathcal I,\mathcal E)\), ove \(\mathcal I\) è l'insieme dei vertici mentre \(\mathcal E\) dei lati di \(\mathcal G\). Si assume che il grafico sia statico, ovvero che le connessioni tra agenti non varia nel tempo.

                Si consideri, allora, un generico agente rappresentativo, il cui stato microscopico è descritto dal processo stocastico \((X,S_t)_{t\geq0}\); la funzione \(S_t:\Omega\to\mathcal P\) è una variabile aleatoria da uno spazio astratto \(\Omega\) allo spazio delle popolazioni \(\mathcal P\) e indica la popolazione dell'agente al tempo \(t\geq0\). Tale variabile aleatoria evolve nel tempo per le interazioni binarie con altri agenti mediate dalle connessioni descritte da \(\mathcal E\), definendo cosí un processo stocastico \(\{S_t,t\in[0,+\infty)\}\).

                Nel complesso di descrive statisticamente lo stato microscopico \(X,S_t\) dell'agente mediante una probabilità di misura \(f=f(x,s,t)\), discreta in \(x\in\mathcal I\) e continua in \(s\in\mathcal P\). Pertanto si può dare alla \(f\) la seguente forma
                %
                \begin{equation}
                    \label{eqDistribuzioneTotale}
                    f(x,s,t)=\frac1N\sum_{i\in\mathcal I}f_i(s,t)\otimes\delta(x-i),
                \end{equation}
                %
                ove \(N\equiv\vert\mathcal I\vert\) è il numero totale d'agenti/vertici del grafo mentre \(\delta(\cdot)\) denota la delta di Dirac centrata all'origine; d'altra parte \[f_i=f_i(s,t):\mathcal P\times[0,+\infty)\rightarrow\mathbb R_+\] è la densità di probabilità della taglia \(S_t\) dell'agente \(X=i\).
                
                Logicamente si richiede
                %
                \begin{equation*}
                    % \label{eqUnitaryIntegrationfi}
                    \int_{\mathcal P}f_i(s,t)\de s=1,
                    \qquad\forall t\geq0,\;\forall i\in\mathcal I,
                \end{equation*}
                %
                che implica coerentemente
                %
                \begin{equation*}
                    % \label{eqUnitaryIntegrationf}
                    \int_{\mathcal I}\int_{\mathcal P}
                    f(x,s,t)\de s\de x=1
                    \qquad\forall t\geq0
                \end{equation*}

            \subsection{Algoritmi d'interazione}

                Un algoritmo d'interazione è una regola che descrive come gli agenti interagiscono a coppie e modificano di conseguenza il loro stato nel tempo; nel dettaglio, in un dato passo temporale \(\Delta t>0\) si assume che un agente \((X,S_t)\in\mathcal I\times\mathcal P\) cambi la sua popolazione a \(S_{t+\Delta t}\in\mathcal P\) a seguito di un interazione con un altro agente \((X^*,S^*_t)\in\mathcal I\times\mathcal P\) secondo il successivo schema
                %
                \begin{equation}
                    \label{eqSchemaInterazione}
                    S_{t+\Delta t}=(1-\Theta)S_t+\Theta S'_t,
                \end{equation}
                %
                ove \(\Theta\in{0,1}\) è una variabile aleatoria che tiene in considerazione qualora l'interazione tra i due agenti effettivamente si manifesti \((\Theta=1)\) o no \((\Theta=0)\); d'altro canto \(S'_t\in\mathcal P\) è la nuova popolazione ottenuta dall'agente \((X,V_t)\) in seguito a un'interazione avvenuta.

                Con maggiore dettaglio si pone
                %
                \begin{equation}
                    \label{eqVariabileAleatoriaInterazione}
                    \Theta\sim\Bernoulli(A(X,X^*),\Delta t),
                \end{equation}
                %
                il che significa che la probabilità che un'interazione avvenga è proporzionale al passo temporale d'interazione \(\Delta t\) mediante un nucleo d'interazione \(A(X,X^*)=1\), che contiene le informazioni sui lati del grafo, e quindi alle connessioni tra gli agenti, ponendo
                %
                \begin{equation*}
                    %\label{}
                    A(X,X^*)=
                    \left\{\begin{aligned}
                        1&\quad\text{se }(X,X^*)\in\mathcal E,\\
                        0&\quad\text{se }(X,X^*)\notin\mathcal E,
                    \end{aligned}\right.
                \end{equation*}
                %
                dove la coppia ordinata \((X,X^*)\) denota il lato dal vertice \(X\) al vertice \(X^*\); per coerenza è necessario imporre \(\Delta t\leq1\) che impone un limite superiore al massimo passo temporale ammissibile, seppure tale condizione sia molto facile da verificare nella pratica.

                La popolazione postinterazione è una variabile aleatoria \(S'_t:\Omega\to\mathcal P\) dipendente in generale dagli stati preinterazione \(V_t\), \(V^*_t\) degli agenti integranti:
                %
                \begin{equation}
                    \label{eqPopolazionePostInterazione}
                    V'_t(\omega)=\Psi(S_t(\omega),S^*_t(\omega),\omega),
                    \quad\omega\in\Omega,
                \end{equation}
                %
                in cui \(\Psi:\mathcal P^2\times\Omega\to\mathcal P\) è funzione nota potenzialmente stocastica. 
                
                [...]

                %
                \begin{equation}
                    \label{eqBoltzmannConGrafo}
                    \begin{aligned}
                        \derS{}t\int_{\mathcal P}\varphi f_i\de s&=
                        \frac1{2N}\sum_{j\in\mathcal I}A(i,j)
                        \int_{\mathcal P^2}%\int_{\mathcal P}
                        \langle\varphi'-\varphi\rangle
                        f_if^*_j\de s\de s_*\\&
                        +\frac1{2N}\sum_{j\in\mathcal I}A(j,i)
                        \int_{\mathcal P^2}%\int_{\mathcal P}
                        \langle\varphi'_*-\varphi_*\rangle
                        f_jf^*_i\de s\de s_*,%\\&
                        \quad\forall i\in\mathcal I,
                    \end{aligned}
                \end{equation}
                %
                ove gli argomenti di tutte le funzioni sono stati sottintesi, vale a dire
                \[
                    \newcommand\spazOr{\;\;\;}
                    f_i\equiv f_i(s,t)\spazOr\forall i\in\mathcal I,
                    \spazOr\varphi\equiv\varphi(s),
                    \spazOr\varphi_*\equiv\varphi(s_*)
                    \spazOr\text{e}\spazOr
                    \varphi'\equiv\varphi(s').
                \]

        \section{Nesso discreto-continuo}\label{secNessoDC}

            È d'interesse esplorare il legame presente tra la \eqref{eqBoltzmannConGrafo} coll'equazione classica di Boltzmann \daRivedere{}.


            \subsection{Ipotesi semplificative}

                A questo scopo si possono fare tre principali ipotesi semplificative da applicare alla \eqref{eqBoltzmannConGrafo}:

                \begin{enumerate}[
                    label=\arabic*\(^\circ\) IS,
                    topsep=7.5pt,
                    parsep=5pt,
                    itemsep=0pt,
                    leftmargin=1cm,
                    rightmargin=0pt,
                    % \leftmargin + \itemindent = \labelindent + \labelwidth + \labelsep
                ]
                    \item\label{ipMatriceUnitaria} Si presuppone che il grafo sia completamente connesso e quindi che la matrice d'adiacenza sia unitaria \(A\equiv I\).
                    %
                    \item\label{ipAgentiIndistinguibili} Si assume che gli agenti siano indistinguibili:
                    %
                    \begin{equation}
                        f_i(s,t)=f_j(s,t)=f(s,t)
                        \quad\forall i,j\in\mathcal I,
                    \end{equation}
                    %
                    \item\label{ipInterazioniSimmetriche} S'ipotizza che le interazioni siano simmetriche:% \(\psi(s,s_*)=\psi_*(s_*,s).\)
                    %
                    \begin{equation}
                        s'=\Psi(s,s_*)=\Psi_*(s_*,s),
                        \;\;\text{ove }s_*=\Psi_*(s,s_*).
                    \end{equation}
                \end{enumerate}


            \newcommand\titoloSez{
                Analisi della %
                \texorpdfstring{%
                % \tref{ipMatriceUnitaria}{\(1^\circ\)}
                \ref{ipMatriceUnitaria}}{1° IS}%
            }
            \subsection{\titoloSez}

                Con tal'ipotesi la \eqref{eqBoltzmannConGrafo} diventa
                %
                \begin{equation*}
	                % \begin{aligned}
                    %     \derS{}t\int_{\mathcal P}\varphi f_i\de s=
                    %     \frac1{2N}\left[\vphantom{\int_{\mathcal P^2}}\right.&
                    %     \smash{\sum_{j\in\mathcal I}}\int_{\mathcal P^2}
                    %     \langle\varphi'-\varphi\rangle f_if^*_j\de s\de s_*\\
                    %     +&\smash{\sum_{j\in\mathcal I}}\int_{\mathcal P^2}
                    %     \langle\varphi'_*-\varphi_*\rangle f_jf^*_i\de s\de s_*
                    %     \left.\vphantom{\int_{\mathcal P^2}}\right],
                    % \end{aligned}
                    %
                    \derS{}t\int_{\mathcal P}\varphi f_i\de s=
                    \frac1{2N}\left[\vphantom{\int_{\mathcal P^2}}\right.
                    \smash{\sum_{j\in\mathcal I}}\int_{\mathcal P^2}
                    \langle\varphi'-\varphi\rangle f_if^*_j\de s\de s_*
                    +\smash{\sum_{j\in\mathcal I}}\int_{\mathcal P^2}
                    \langle\varphi'_*-\varphi_*\rangle f_jf^*_i\de s\de s_*
                    \left.\vphantom{\int_{\mathcal P^2}}\right],
                \end{equation*}
                %
                e valutando la distribuzione marginale della \eqref{eqDistribuzioneTotale} rispetto agl'indici
                %
                \begin{equation}
                    \label{eqDistribuzioneMarginaleIndici}
                    F(s,t)\equiv\int_{\mathcal I}f(x,s,t)\de x=
                    \frac1N\sum_{i\in\mathcal I}f_i(s,t)
                    \otimes\int_{\mathcal I}\delta(x-i)\de x=
                    \frac1N\sum_{i\in\mathcal I}f_i(s,t),
                \end{equation}
                %
                che corrisponde a una media tra le distribuzione dei singoli agenti, si ricava
                %
                \begin{equation*}
                    % \label{}
	                % \begin{aligned}
                    %     \derS{}t\int_{\mathcal P}\varphi f_i\de s=
                    %     \frac1{2N}\left[\vphantom{\int_{\mathcal P^2}}\right.&
                    %     \int_{\mathcal P^2}
                    %     \langle\varphi'-\varphi\rangle f_iF^*\de s\de s_*\\
                    %     +&\int_{\mathcal P^2}
                    %     \langle\varphi'_*-\varphi_*\rangle F_jf^*_i\de s\de s_*
                    %     \left.\vphantom{\int_{\mathcal P^2}}\right],
                    % \end{aligned}
                    %
                    \derS{}t\int_{\mathcal P}\varphi f_i\de s=
                    \frac12\left[\vphantom{\int_{\mathcal P^2}}\right.
                    \int_{\mathcal P^2}
                    \langle\varphi'-\varphi\rangle f_iF^*\de s\de s_*
                    +\int_{\mathcal P^2}
                    \langle\varphi'_*-\varphi_*\rangle Ff^*_i\de s\de s_*
                    \left.\vphantom{\int_{\mathcal P^2}}\right];
                \end{equation*}
                %
                mediando ora rispetto a tutti gli agenti, si ha
                %
                \begin{equation}
                    \label{eqBoltzmannConGrafoUnitario}
                    % \begin{aligned}
                    %     \derS{}t\int_{\mathcal P}\varphi F\de s&=
                    %     \frac1{2N}\left[\vphantom{\int_{\mathcal P^2}}\right.
                    %     \int_{\mathcal P^2}
                    %     \langle\varphi'-\varphi\rangle FF^*\de s\de s_*
                    %     +\int_{\mathcal P^2}
                    %     \langle\varphi'_*-\varphi_*\rangle FF^*\de s\de s_*
                    %     \left.\vphantom{\int_{\mathcal P^2}}\right]\\&=
                    %     \frac1{2N}\int_{\mathcal P^2}
                    %     \langle\varphi'-\varphi+\varphi'_*-\varphi_*\rangle FF^*\de s\de s_*
                    % \end{aligned}
                    \derS{}t\int_{\mathcal P}\varphi F\de s=
                    \frac12\int_{\mathcal P^2}
                    \langle\varphi'-\varphi+\varphi'_*-\varphi_*\rangle
                    FF^*\de s\de s_*,
                \end{equation}
                %
                la quale è formalmente analoga a quella classica di Boltzmann \daRivedere{}. Ciò significa che con tal'ipotesi semplificativa, nonostante gli agenti siano distinti, questi si possono vedere come indistinguibili purché si consideri la distribuzione media \eqref{eqDistribuzioneMarginaleIndici}.

                Tale risultato è anche confermato a livello pratico nell'algoritmo \ref{algMonteCarlo} ove una matrice d'adiacenza unitaria porta ad avere un algoritmo del tutto analogo a quello classico; pertanto l'unica distribuzione che può calcolare \ref{algMonteCarlo} è proprio quella media \(F\).
    
                \newcommand\ave[1]{\left\langle#1\right\rangle}
                \newcommand\bM{\mathbf M}
                \newcommand\cO{\mathcal O}
                \begin{algorithm}[!t]
                    \caption{Algoritmo di Monte Carlo per equazioni di tipo su un grafo}
                    \label{algMonteCarlo}
                    \begin{algorithmic}[1]
                        \Require adjacency matrix \(\bM\); initial state \(V_0\in\cO^N\); time step \(\Delta{t}>0\); final time \(T>0\)
                        \State \(\tilde{V}\gets V_0\)
                        \State \(t\gets 1\)
                        \For{\(t<T\)}
                        \State \(\ave{\varphi}(t)\gets\frac{1}{N}\sum_{i=1}^N\varphi(\tilde{V}(i))\)
                        \State \(V\gets\tilde{V}\)
                        \State \(P\gets \text{random permutation of } \left\{1,\,\dots,\,N\right\}\)
                        \State \(p_1\gets (P(1),\,\dots,\,P(N/2))\)
                        \State \(p_2\gets (P(N/2 + 1),\,\dots,\,P(N))\)
                        \State \(i\gets 1\)
                        \For{\(i<N/2\)}
                        \State \(\Theta\sim\operatorname{Bernoulli}\!\left(B(p_1(i),p_2(i))\Delta{t}\right)\)
                        \State \(\tilde{V}(p_1(i))\gets V(p_1(i))(1-\Theta)+\Psi(V(p_1(i)),V(p_2(i)))\Theta\)
                        \State \(\tilde{V}(p_2(i))\gets V(p_2(i))(1-\Theta)+\Psi_\ast(V(p_2(i)),V(p_1(i)))\Theta\)
                        \State \(i\gets i+1\)
                        \EndFor
                        \State \(t\gets t+\Delta t\)
                        \EndFor
                    \end{algorithmic}
                \end{algorithm}


            \renewcommand\titoloSez{
                Analisi della %
                \texorpdfstring{%
                % \tref{ipAgentiIndistinguibili}{\(2^\circ\)}
                \ref{ipAgentiIndistinguibili}}{2° IS}%
            }
            \subsection{\titoloSez}

                La previa discussione suggerisce di studiare anche il caso in cui gli agenti siano effettivamente indistinguibili; tuttavia, prima di affrontarlo assieme alla prima ipotesi risulta interessante analizzare tale ipotesi isolatamente. Pertanto la \eqref{eqBoltzmannConGrafo} diventa
                %
                \begin{equation*}
                    \begin{aligned}
                        \derS{}t\int_{\mathcal P}\varphi f\de s&=
                        \frac1{2N}\sum_{j\in\mathcal I}A(i,j)
                        \int_{\mathcal P^2}
                        \langle\varphi'-\varphi\rangle
                        ff^*\de s\de s_*\\&
                        +\frac1{2N}\sum_{j\in\mathcal I}A(j,i)
                        \int_{\mathcal P^2}
                        \langle\varphi'_*-\varphi_*\rangle
                        ff^*\de s\de s_*,
                    \end{aligned}
                \end{equation*}
                %
                che sommata su tutti gl'indici porta a
                %
                \begin{equation*}
                    \derS{}t\int_{\mathcal P}\varphi f\de s=
                    \frac1{2N^2}\sum_{i,j\in\mathcal I}A(i,j)
                    \int_{\mathcal P^2}
                    \langle\varphi'-\varphi+\varphi'_*-\varphi_*\rangle
                    ff^*\de s\de s_*,
                \end{equation*}
                %
                e definendo \(L\equiv\sum_{i,j\in\mathcal I}A(i,j)\) si arriva a
                %
                \begin{equation}
                    \label{eqBoltzmannConAgentiIndistinguibili}
                    \derS{}t\int_{\mathcal P}\varphi f\de s=
                    \frac L{N^2}\left[\frac12
                    \smash{\sum_{i,j\in\mathcal I}}A(i,j)
                    \int_{\mathcal P^2}
                    \langle\varphi'-\varphi+\varphi'_*-\varphi_*\rangle
                    ff^*\de s\de s_*\right].
                \end{equation}
                %
                In questo contesto il rapporto \(L/N^2\in[0,1]\) rappresenta topologicamente simile è la rete a una completamente connessa\footnotemark{}; d'altra parte l'equazione è analoga a quella classica di Boltzmann \daRivedere{}.
               
                Dunque l'indistinguibilità degli agenti ha una notevole conseguenza sulla \eqref{eqBoltzmannConGrafo}, riassumendo l'effetto complessivo del grafo al solo coefficiente \(L/N^2\) che quindi ne rappresenta gli ultimi bagliori prima di una sua totale scomparsa per la \ref{ipMatriceUnitaria}.

                \footnotetext{Difatti \(L\) è interpretabile come il numero di lati presenti in un grafo diretto e che ha come limite superiore prorio \(N^2\), ossia il numero totale di coppie [e quindi lati] dati \(N\) nodi.}


            \renewcommand\titoloSez{
                Analisi della %
                \texorpdfstring{%
                \tref{ipMatriceUnitaria}{\(1^\circ\)}, %
                \tref{ipAgentiIndistinguibili}{\(2^\circ\)} e %
                % \tref{ipInterazioniSimmetriche}{\(3^\circ\)}%
                \ref{ipInterazioniSimmetriche}}{1°, 2° e 3° IS}%
            }
            \subsection{\titoloSez}

                Visto che vale \ref{ipMatriceUnitaria} si può partire dalla \eqref{eqBoltzmannConGrafoUnitario} nella quale la distribuzione media \eqref{eqDistribuzioneMarginaleIndici} diventa per \ref{ipAgentiIndistinguibili}                %
                \begin{equation*}
                        F(s,t)=\frac1N\sum_{i\in\mathcal I}f_i(s,t)
                        \overset{\tref{ipAgentiIndistinguibili}{2^\circ}}=
                        \frac1N\sum_{i\in\mathcal I}f(s,t)=f(s,t),
                \end{equation*}
                %
                ossia la \(F\) coincide con quella di tutti gli agenti\footnotemark, essendo questi, appunto, indistinguibili.
                \footnotetext{Si noti anche che la perdita della dipendenza della distribuzione media dal numero di nodi \(N\) è coerente colla situazione in cui \(N\to\infty\), condizione fondamentale analoga a casi classici come lo studio del gas nel quale \(N\gg1\) ben approssima il limite.}

                In tal modo la \eqref{eqBoltzmannConGrafoUnitario} diventa
                %
                \begin{equation*}
                    \derS{}t\int_{\mathcal P}\varphi f\de s=
                    \frac12\int_{\mathcal P^2}
                    \langle\varphi'-\varphi+\varphi'_*-\varphi_*\rangle
                    ff^*\de s\de s_*,
                \end{equation*}
                %
                che unita all \ref{ipInterazioniSimmetriche} porta all'equivalenza (mediante il cambio di variabili \(s_*=s\) e \(s=s_*\))
                %
                \begin{equation*}
                    \int_{\mathcal P^2}
                    \langle\varphi_*'-\varphi_*\rangle ff_*dsds_*
                    =\int_{\mathcal P^2}
                    \langle\varphi'-\varphi\rangle ff^*dsds_*,
                \end{equation*}
                %
                e quindi a
                %
                \begin{equation*}
                    \derS{}t\int_{\mathcal P}\varphi fds=
                    \int_{\mathcal P^2}
                    \langle\varphi'-\varphi\rangle ff^*dsds_*,
                \end{equation*}
                %
                che equivale alla formula classica di Boltzmann con interazioni simmetriche \daRivedere{} usata classicamente per modellizzare la distribuzione dell'energia cinetica tra una popolazione di particelle di un gas.

    %%%
    %%%
    %%%

    % \part{Risultati}
    \chapter{Simulazioni}\label{secSimulazioni}

    %%%
    %%%
    %%%

    % \part{Conclusioni}
    \chapter{Conclusioni}\label{secConclusioni}

    %%%
    %%%
    %%%

    \printbibliography

\end{document}