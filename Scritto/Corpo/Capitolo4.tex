
% !TEX root = ../Testa/Principale.tex
% LTeX: language=it

\chapter{Simulazioni}\label{secSimulazioni}

    In questo capitolo si applicheranno tutta la teoria affrontata in quello precedente. Come consuetudine, è prima necessario un paragrafo propedeutico per approfondire l'algoritmo con cui sono stati svolte le simulazioni; questo è seguíto dalle definizioni dei grafici da analizzare e da una rapida spiegazione del perché le fluttuazioni possono [anzi devono] essere trascurate in questa trattazione. Per i risultati, invece, si propongono varie leggi d'emigrazione che si studiano tramite dei studi parametrici concludendo cercando d'interpretare cosa queste dicono sul fenomeno della migrazione in sé. Infine si mostrano brevemente alcuni risultati per l'Italia, mostrando che tutta la teoria scala bene col numero d'agenti.

    \section{Premesse}

        \subsection{Metodo Monte Carlo}\label{secMetodoMonteCarlo}

            Nel contesto delle \TCSMA{} l'obbiettivo, com'è chiaro dal \cref{secTCSM}, è quello di ricavare la densità \(f(\mathbf x,t)\) nella \ref{eqFormaDeboleTipoBoltzmannOmogeneoAssimmetrico} al variare del tempo. Si potrebbe allora pensare di discretizzare quest'ultima equazione mediante il Metodo delle Differenze Finite, degli Elementi Finiti o dei Volumi Finiti; tuttavia, vi sono due principali problemi:
           
            \begin{enumerate}[
                label=\arabic*.,
                % topsep=0.5em,
                % parsep=0em,
                % itemsep=0.25em,
                % leftmargin=2em,
                % rightmargin=1.5em,
                % \leftmargin + \itemindent = \labelindent + \labelwidth + \labelsep
                %itemindent=!,
                %labelindent=3em,
                %labelwidth=!,
                %labelsep=!,
            ]
                \item l'impossibilità, in generale, di ricavare la forma forte della \ref{eqFormaDeboleTipoBoltzmannOmogeneoAssimmetrico}, specie nel caso di \ref{eqRegoleInterazione} non lineari;
                \item anche ipotizzando di trovare la forma forte a cui applicare i precedenti metodi, la sua natura integro differenziale la rende complessa da manipolare dato che l'operatore collisionale\footnotemark{}, come%
                \footnotetext{Vale a dire la parte integrale della forma forte dell'equazione di tipo Boltzmann, in genere scritta a secondo membro.}%
                \[
                    Q(f,f)(v,t)\equiv
                    \frac1{4\pi}\int_{\mathbb R^3}\int_{\mathbb S^2}
                    B(\mathbf v,\mathbf v_*,n)
                    \left(f(v',t)f(v'_*,t)-f(v,t)f(v_*,t)\right)
                    \de n\de\mathbf v_*
                \]
                nella \cref{eqFormaForteBoltzmannDisomogeneoSimmetrico}, dipende dalla densità medesima.
            \end{enumerate}

            Per tali ragioni si procede in maniera piú diretta: conoscendo l'algoritmo d'interazione \ref{eqAlgoritmoInterazione}, che governa le interazioni binarie tra agenti, si possono quindi direttamente simulare tutte le molteplici collisioni mediante un metodo di Monte Carlo di tipo Nanbu-Babovsky, descritto nel dettaglio nell'\cref{algMonteCarlo}.
            
            \begin{algorithm}[htb]
                \caption{Algoritmo \ref{eqAlgoritmoInterazioneUrbano} di tipo Nanbu-Babovsky}
                \label{algMonteCarlo}
                \KwDati{%
                    \justifying
                    \(N\in\NPlus\) (numero d'agenti); %
                    \(\Delta t\leq 1\) (passo temporale); %
                    \(\sigma\) (fluttuazione); %
                    \(T>0\) (tempo finale); %
                    \(\bar q_0(s)\) (densità iniziale); %
                    \(\mathbf A\) (matrice d'adiacenza); %
                    \(\mathbf B\) (matrice d'adiacenza approssimata);%
                }

                \(\pazocal S^0\gets\{s_1^0,s_2^0,\dots,s_N^0\}\subseteq\RPlus\) campionati da \(\bar q_0\)\;
                \Per{\(n=0,1,2,\dots,\left\lfloor T/\Delta t\right\rfloor-1\)}{

                    \(P\gets\) permutazione indipendente di \(\{1,2,\ldots,N\}\)\;

                    \Per{\(i=1,2,\dots,\left\lfloor N/2\right\rfloor\)}{

                        \(j\gets\left\lfloor N/2\right\rfloor+i\), \(s_i^n\gets P(i)\) e \(s_j^n\gets P(j)\)\;

                        % Si campiona \(\Theta\sim\DstrBernoulli(A(i,j)(\Delta t))\)\;
                        \uSea{esatto}{
                            \(\Theta\gets\DstrBernoulli(A(i,j)(\Delta t))\)\;
                        }
                        \Altrimenti(\tcp*[h]{è approssimato}){
                            \(\Theta\gets\DstrBernoulli(B(i,j)(\Delta t))\)\;
                        }

                        \uSea{\(\Theta=1\)}{
                            \(E\gets E(s_i^n,i,s_j^n,j)\)\;
                            \(\gamma\gets\DstrGamma((1-E)^2/\sigma^2,\sigma^2/(1-E))\)\;

                            \(s_i^{n+1}\gets s_i^n(1-E+\gamma)\)\tcp*[r]{città interangente}
                            \(s_j^{n+1}\gets s_j^n+s_i^nE\)\tcp*[r]{città ricevente}
                        }
                        \Altrimenti{
                            \(s_i^{n+1}\gets s_i^n\) e \(s_j^{n+1}\gets s_j^n\)\;
                        }

                        % \(s_i^{n+1}\gets s_i^n(1-\Theta)+\Theta\psi(s_i^n,i,s_j^n,j,\gamma)\)\;
                        % \(s_j^{n+1}\gets s_j^n(1-\Theta)+\Theta\psi_*(s_i^n,i,s_j^n,j)\)\;
                    }

                    \(\pazocal S^{n+1}\gets\{s_1^{n+1},s_2^{n+1},\dots,s_N^{n+1}\}\)\;
                    \(\bar q(s,(n+1)\Delta{t})\gets\) istogramma di \(\pazocal S^{n+1}\)\;

                }
            \end{algorithm}

        \subsection{Definizione dei grafici}
        
        \subsection{Sulle fluttuazioni \texorpdfstring{\(\gamma\)}{γ}}

    \section{Regole d'emigrazione}

        \(s\) e \(s_*\) vanno intese come la città interagente e ricevente rispettivamente
               
        La scelta di \(E(s,s_*)\) dipende da come si vuole modellizzare il fenomeno dell'immigrazione.

        Si vuole evitare lo spopolamento delle città a causa delle fluttuazioni nella \cref{eqVincoloPositivitàFluttuazioni} dato che nella \eqref{eqLeggeEmigrazione} compare il rapporto tra popolazioni delle città interagenti

        Si avvisa anche sono stati ottenuti dei risultati analoghi a quelli della Sardegna per tutte le regioni italiane, Italia inclusa; questa è la ragione per cui in questo paragrafo ci si concentrerà solo su quella regione, mostrandone anche alcuni per l'Italia.

        \subsection{Popolazione}

            Una prima possibilità consiste nella
            %
            \begin{equation}
                \label{eqLeggeEmigrazione}
                E(s,s_*)\equiv\lambda\frac{(s_*/s)^\alpha}{1+(s_*/s)^\alpha},
            \end{equation}
            %
            che in essenza è la \cite[(2.2), § 2, p. 223]{Gualandi2019SDC} modificata mediante la \cite[(4.5), § 4, p. 228]{Gualandi2019SDC}, ossia è una funzione di Hill di ordine \(\alpha\), in cui v'è un tasso di emigrazione maggiore verso città con popolazione relativa, data dal rapporto \(s_*/s\), maggiore; gli unici due parametri presenti, invece, presentano il seguente significato:

            \begin{itemize}[label=\(\diamond\)]
                \item \(\lambda\in(0,1)\) rappresenta l'attrattività dei poli, ossia la frazione che le città piú popolose riescono al massimo ad attrarre in un'interazione;
                \item \(\alpha\in\mathbb R^+\) indica la rapidità d'emigrazione e influenza quanto rapidamente il rapporto \(s_*/s\) raggiunge la massima attrattività \(\lambda\).
            \end{itemize}

            In poche parole la \eqref{eqLeggeEmigrazione} descrive la tendenza degl'individui di aggregarsi per i piú svariati motivi: lavoro, sicurezza, famiglia, eccetera.

            In questo caso si hanno quindi interazioni non simmetriche, poiché dalla \eqref{eqLegameLineareInterazioneCittà} \(p\neq q\) e \(q\neq p\), e non lineari, a causa della \eqref{eqLeggeEmigrazione}.
            
            % Mediante la conservazione di popolazione ho rifatto molto velocemente i conti della formula prima del Teorema 5.1 a p. 16 di [1], confermando che d⟨s⟩/dt=0, come ci si aspetta da questo tipo d'interazione.

        \subsection{Popolazione-Connettività}

        \subsection{Popolazione-Connettività frazionata}

        \subsection{Popolazione-Forza}

        \subsection{Interpretazioni}
        
    \section{Il caso dell'Italia}
