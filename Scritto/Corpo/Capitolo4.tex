
% !TEX root = ../Testa/Principale.tex
% LTeX: language=it

\chapter{Simulazioni}\label{secSimulazioni}

    In questo capitolo si applica tutta la teoria affrontata in quello precedente. Innanzitutto si descrive l'algoritmo con cui sono stati svolte le simulazioni; quindi si spiegano le formule che definiscono i grafici per analizzare i risultati per poi motivare rapidamente il perché le fluttuazioni possono [anzi devono] essere trascurate. Per le simulazioni si propongono varie leggi d'emigrazione che studiate tramite dei studi parametrici, cercando successivamente d'interpretare cosa queste dicono sul fenomeno della migrazione in sé. Infine si mostra brevemente una simulazione per l'Italia, confermando che i risultati migliorano col numero d'agenti.

    \section{Metodo Monte Carlo}\label{secMetodoMonteCarlo}

        Nel contesto delle \TCSMA{} l'obbiettivo, com'è chiaro dal \cref{secTCSM}, è quello di ricavare la densità \(f(\mathbf x,t)\) nella \ref{eqFormaDeboleTipoBoltzmannOmogeneoAssimmetrico} al variare del tempo. Si potrebbe allora pensare di discretizzare quest'ultima equazione mediante il Metodo delle Differenze Finite, degli Elementi Finiti o dei Volumi Finiti; tuttavia, vi sono due principali problemi:
        
        \begin{enumerate}[
            label=\arabic*.,
            % topsep=0.5em,
            % parsep=0em,
            % itemsep=0.25em,
            % leftmargin=2em,
            % rightmargin=1.5em,
            % \leftmargin + \itemindent = \labelindent + \labelwidth + \labelsep
            %itemindent=!,
            %labelindent=3em,
            %labelwidth=!,
            %labelsep=!,
        ]
            \item l'impossibilità, in generale, di ricavare la forma forte della \ref{eqFormaDeboleTipoBoltzmannOmogeneoAssimmetrico}, specie nel caso di \ref{eqRegoleInterazione} non lineari;
            \item anche ipotizzando di trovare la forma forte a cui applicare i precedenti metodi, la sua natura integro differenziale la rende complessa da manipolare dato che l'operatore collisionale\footnotemark{}, come%
            \footnotetext{Vale a dire la parte integrale della forma forte dell'equazione di tipo Boltzmann, in genere scritta a secondo membro.}%
            \[
                Q(f,f)(v,t)\equiv
                \frac1{4\pi}\int_{\mathbb R^3}\int_{\mathbb S^2}
                B(\mathbf v,\mathbf v_*,n)
                \left(f(v',t)f(v'_*,t)-f(v,t)f(v_*,t)\right)
                \de n\de\mathbf v_*
            \]
            nella \cref{eqFormaForteBoltzmannDisomogeneoSimmetrico}, dipende dalla densità medesima.
        \end{enumerate}

        Per tali ragioni si procede in maniera piú semplice: conoscendo \ref{eqAlgoritmoInterazione}, che governa le interazioni binarie tra agenti, si possono quindi direttamente simulare tutte le molteplici collisioni mediante un metodo di Monte Carlo di tipo Nanbu-Babovsky, descritto nel dettaglio nell'\cref{algMonteCarlo} nel quale \(T>0\) è il tempo finale di simulazione, mentre \(P\) è la popolazione totale.
        
        \begin{algorithm}[htb]
            \caption{Algoritmo \ref{eqAlgoritmoInterazioneUrbano} di tipo Nanbu-Babovsky}
            \label{algMonteCarlo}
            \KwDati{%
                \justifying
                \(N\in\NPlus\), %(numero d'agenti), %
                \(\Delta t\leq 1\), %(passo temporale), %
                \(\sigma\), %(fluttuazione), %
                \(T>0\), %(tempo finale), %
                \(P\), %(Popolazione totale), %
                \(\mathbf A\) %(matrice d'adiacenza) %
                e \(\mathbf B\) %(matrice d'adiacenza approssimata)%
                ;%
            }%
            \(\pazocal S^0\gets(s_1^0,s_2^0,\dots,s_N^0)\equiv(P/N)\mathbf 1\in\RPlus^N\)\;\label{algMonteCarloDistribuzioneIniziale}
            \Per{\(n=0,1,2,\dots,\left\lfloor T/\Delta t\right\rfloor-1\)}{
                %
                \(P\gets\) permutazione indipendente di \(\{1,2,\ldots,N\}\)\;
                %
                \Per{\(i=1,2,\dots,\left\lfloor N/2\right\rfloor\)}{
                    %
                    \(j\gets\left\lfloor N/2\right\rfloor+i\), \(s_i^n\gets P(i)\) e \(s_j^n\gets P(j)\)\;
                    %
                    % Si campiona \(\Theta\sim\DstrBernoulli(A(i,j)(\Delta t))\)\;
                    \uSea{esatto}{
                        \(\Theta\gets\DstrBernoulli(A(i,j)(\Delta t))\)\;
                    }
                    \Altrimenti(\tcp*[h]{è approssimato}){
                        \(\Theta\gets\DstrBernoulli(B(i,j)(\Delta t))\)\;
                    }
                    %
                    \uSea{\(\Theta=1\)}{
                        \(E\gets E(s_i^n,i,s_j^n,j)\)\;
                        \(\gamma\gets\DstrGamma((1-E)^2/\sigma^2,\sigma^2/(1-E))\)\;
                        %
                        \(s_i^{n+1}\gets s_i^n(1-E+\gamma)\)\tcp*[r]{città interangente}
                        \(s_j^{n+1}\gets s_j^n+s_i^nE\)\tcp*[r]{città ricevente}
                    }
                    \Altrimenti{
                        \(s_i^{n+1}\gets s_i^n\) e \(s_j^{n+1}\gets s_j^n\)\;
                    }
                    %
                    % \(s_i^{n+1}\gets s_i^n(1-\Theta)+\Theta\psi(s_i^n,i,s_j^n,j,\gamma)\)\;
                    % \(s_j^{n+1}\gets s_j^n(1-\Theta)+\Theta\psi_*(s_i^n,i,s_j^n,j)\)\;
                }
                %
                \(\pazocal S^{n+1}\gets(s_1^{n+1},s_2^{n+1},\dots,s_N^{n+1})\)\;
                \(\bar l(s,(n+1)\Delta{t})\gets\) istogramma di \(\pazocal S^{n+1}\)\;\label{algMonteCarloDistribuzioneNesima}
            }
        \end{algorithm}

        \begin{oss}
            Nella linea \ref{algMonteCarloDistribuzioneNesima}, \(\bar l\) è una densità arbitraria: va intesa come \(\bar f\) se l'approccio è esatto e come \(\bar g\) se è approssimato.
        \end{oss}

        \begin{oss}
            Piú in generale, nella linea \ref{algMonteCarloDistribuzioneIniziale}, \(\pazocal S_0\) si può campionare da una densità \(\bar l_0\) iniziale; tuttavia non conoscendone alcuna per la distribuzione della popolazione, si è deciso di definire \(\pazocal S^0\) come un vettore uniforme rispetto a una popolazione totale \(P\) iniziale.
        \end{oss}

    \section{Rappresentazione dei risultati}

        Per analizzare i risultati delle simulazioni è opportuno descrivere nel dettaglio le funzioni usate per descriverli; per farlo, però, si devono prima approfondire le conseguenze della natura stocastica dell'\cref{algMonteCarlo} e la struttura dei risultati. Verso la fine, si motiva anche l'omissione delle fluttuazioni \(\gamma\).
        
        \subsection{Intervalli di confidenza}

            Innanzitutto simulando la \ref{eqFormaDeboleTipoBoltzmannOmogeneoAssimmetricoUrbano} mediante \cref{algMonteCarlo}, e quindi l'algoritmo \ref{eqAlgoritmoInterazione}, si sta introducendo nei risultati un rumore di natura stocastica: piú simulazioni daranno risultati diversi per cui la singola non ha rilevanza statistica; bisogna cioè calcolarne molteplici e valutare gl'intervalli di confidenza per conoscere l'incertezza sulla stima della media.

            Sia \(R\in\NPlus\) il numero di simulazioni eseguite e \(\mathbf S\in\RPlus^R\) il vettore aleatorio della popolazione di una città relativa a ciascuna simulazione.

            \begin{ipo}[Numero di simulazioni]
                In questo elaborato si assume \(R=100\) per avere un numero statisticamente significativo di dati.
            \end{ipo}
           
            Per l'\cref{algMonteCarlo} le componenti \((S_1,S_2,\ldots,S_R)\) di \(\mathbf S\) sono indipendenti e identicamente distribuite dalla densità \(q\), proprietà che permette di definire
            %
            \begin{equation*}
                \bar S\equiv\frac1R\sum_{r=1}^RS_r
                \hspace{1em}\text{e}\hspace{1em}
                V\equiv\frac1{R-1}\sum_{r=1}^R(S_r-\bar S)^2,
            \end{equation*}
            %
            rispettivamente la media e la varianza campionarie. Allora, data \(\mu\) la \textit{vera}\footnotemark{} media della densità \(l\), la distribuzione \(T\) di Student con parametro \(R-1\) si scrive%
            \footnotetext{Vale a dire \(\mu\) \textit{non} è una variabile aleatoria ma l'esatto parametro della media di \(l\).}%
            %
            \begin{equation*}
                T=\frac{\bar S-\mu}{\sqrt{V/R}}\sim\DstrStudent(R-1).
            \end{equation*}
            %
            Da questa si può definire l'intervallo di confidenza [simmetrico] con livello di confidenza \(\alpha\) ponendo
            %
            \begin{equation}
                \label{eqCondizioneIntervalloConfindenza}
                \mathbb P\left(-t_{R-1}^{\alpha/2}\leq T\leq t_{R-1}^{\alpha/2}\right)=1-\alpha
            \end{equation}
            %
            ove \(t_{R-1}^{\alpha/2}\in\mathbb R\) è quel valore reale tale che
            %
            \begin{equation*}
                \mathbb P\left(T<t_{R-1}^{\alpha/2}\right)=1-\frac\alpha2.
            \end{equation*}
            %
            Esplicitando la \(T\) nella \cref{eqCondizioneIntervalloConfindenza} e isolando la media \(\mu\) si ha
            %
            \begin{equation*}
                \mathbb P\left(
                    \bar S-t_{R-1}^{\alpha/2}\sqrt{\frac VR}
                    \leq\mu\leq
                    \bar S+t_{R-1}^{\alpha/2}\sqrt{\frac VR}
                \right)=1-\alpha,
            \end{equation*}
            %
            da cui
            %
            \begin{equation*}
                \IC^\alpha_R(\mathbf S)\equiv\left[
                    \bar S-t_{R-1}^{\alpha/2}\sqrt{\frac VR},
                    \bar S+t_{R-1}^{\alpha/2}\sqrt{\frac VR}
                \right]
            \end{equation*}
            %
            è la stima intervallare che definisce l'intervallo di confidenza ricercato, esprimibile anche piú compattamente come
            %
            \begin{equation}
                \label{eqIntervalloConfidenza}
                \IC^\alpha_R(\mathbf S)\equiv\bar S\pm t_{R-1}^{\alpha/2}\sqrt{V/R}.
            \end{equation}
            %
            \begin{ipo}[livello di confidenza]
                Si sceglie \(\alpha=0.05\) cosí d'avere un intervallo con livello di confidenza \(0.95\).   
            \end{ipo}
            %
            Il significato dell'intervallo di confidenza in essenza è l'errore statistico commesso: esso valuta quanto è probabile che la stima intervallare \cref{eqIntervalloConfidenza} contenga il parametro \(\mu\); in altre parole \(\IC^\alpha_R\) misura l'incertezza sulla stima della media: preso un campione \(\mathbf s\), piú l'intervallo di confidenza è esteso piú la media campionaria \(\bar s\) è una stima incerta dell'effettiva media \(\mu\); viceversa piú è stretto, piú la stima \(\bar s\) è precisa nel senso che \(\mu\) si trova in un intorno piccolo della media campionaria. Sotto questo punto di vista è concettualmente analogo alla precisione di uno strumento di misura.
            %
            \begin{oss}
                La stima intervallare \cref{eqIntervalloConfidenza} appena ricavata vale tanto per il vettore aleatorio \(\mathbf S\) che per le sue realizzazioni \(\mathbf s\), le quali sono, appunto, il risultato delle \(R\) simulazioni.
            \end{oss}

        \subsection{Struttura dei dati}

            I dati presentano una struttura di un tensore del quart'ordine \(\mathbf s\in\RPlus^{N_f\times2\times N\times R}\) i cui indici hanno il seguente significato
            %
            \begin{equation*}
                s^{n,a}_{i,r}
                \left\{\begin{aligned}
                    n\ &=\text{ istante temporale},&\ 
                    a\ &=\text{ tipo di simulazione},\\
                    i\ &=\text{ indice della città},&\ 
                    r\ &=\text{ numero della simulazione}.
                \end{aligned}\right.
            \end{equation*}
            % 
            L'istante temporale è definito tramite tre parametri in \(\NPlus\):

            \begin{itemize}[
                label=\(\diamond\),
                topsep=0.5em,
                parsep=0em,
                itemsep=0.25em,
                leftmargin=2em,
                rightmargin=1.5em,
                % \leftmargin + \itemindent = \labelindent + \labelwidth + \labelsep
                %itemindent=!,
                labelindent=30pt,
                %labelwidth=!,
                %labelsep=!,
            ]
                \item \(N_t\equiv\left\lfloor T/\Delta t\right\rfloor\) è il numero totale di tempi simulati;
                \item \(N_s\ll N_t\) è il numero di catture dai tempi simulati;
                \item \(N_f<N_s\) è il numero di tempi ridotti dalle catture.
            \end{itemize}

            Sia le catture che la riduzione sono campinate in intervalli equispaziati con passi
            %
            \begin{equation*}
                \Delta s=N_t/N_s
                \quad\text{e}\quad
                \Delta f=N_s/N_f
            \end{equation*}
            %
            ove si suppone, per semplicità, che \(N_s\) e \(N_f\) siano divisori ordinatamente di \(N_t\) e \(N_s\), da cui
            %
            \begin{equation*}
                N_t/N_s-\left\lfloor N_t/N_s\right\rfloor=0
                \quad\text{e}\quad
                N_s/N_f-\left\lfloor N_s/N_f\right\rfloor=0,
            \end{equation*}
            %
            e ugualmente si assume per \(\Delta t\) e \(T\).
            
            Tramite l'\cref{algMonteCarlo} si simulano in totale \(N_t\) tempi con passo \(\Delta t\) di cui \(N_s\) sono salvati nel tensore \(\undertilde{\mathbf s}\in\RPlus^{N_s\times2\times N\times R}\) delle catture:
            %
            \begin{equation*}
                \undertilde s^{n,a}_{\cdot,r}\equiv\pazocal S^{n\Delta s,a}_{\cdot,r}
                \quad\forall n,\ \forall a,\ \forall r,
            \end{equation*}
            %
            dove \(\pazocal S^{n\Delta s,a}_{\cdot,r}\) va intesa come il vettore delle popolazioni predette nell'\(r\)-esima simulazione esatta, se \(a=1\), o approssimata, se \(a=2\); successivamente si convolve \(\undertilde{\mathbf s}\) rispetto all'indice del tempo:
            %
            \begin{equation*}
                s^{n,a}_{i,r}\equiv\frac1{N_f}\sum_{j=1}^{N_f}\undertilde s^{(n-1)N_f+j,a}_ir{,}
                \quad\forall n,\ \forall a,\ \forall i,\ \forall r,
            \end{equation*}
            %
            vale a dire si mediano ogni \(\Delta f\) elementi delle \(N_s\) catture. Tale mollificazione è necessaria per rendere \(\mathbf s\) meno rumoroso rispetto a \(\undertilde{\mathbf s}\) e quindi piú leggibile una volta raffigurato.

            Per quanto riguarda gl'istanti temporali considerati, si hanno tre forme a seconda di come s'intende l'indice \(n\):
            %
            \begin{equation*}
                \newcommand\hsep{\hspace{1em}}
                \begin{aligned}
                    t_n&=n\Delta t,&\hsep&\forall n\in\{1,2,\ldots,N_t\},\\
                    t^s_n&=t_n\Delta s,&\hsep&\forall n\in\{1,2,\ldots,N_s\},\\
                    t^f_n&=t^s_n\Delta f,&\hsep&\forall n\in\{1,2,\ldots,N_f\},
                \end{aligned}
            \end{equation*}
            %
            rispettivamente per i tempi discretizzati, campionanti e ridotti; a prescindere vale comunque \(t_{N_t}=t^s_{N_s}=t^f_{N_f}=T\).

            \begin{oss}
                I parametri temporali \(N_t\), \(N_s\) ed \(N_f\) non contano il tempo iniziale perché fa riferimento alla distribuzione iniziale \(\mathbf s_0\in\RPlus^N\).
            \end{oss}

            \begin{ipo}
                In questa trattazione si considerano \(\Delta t=0.01\), \(N_s=1000\), \(N_f=50\) mentre \(N_t\) viene scelto per essere poco superiore al tempo di convergenza della simulazione, se i risultati convergono, ma sicuramente è un multiplo di \(10\) per garantire che \(N_s\) sia un suo divisore.
            \end{ipo}

            \begin{oss}
                Per la maggior parte dei grafici si considera esclusivamente la distribuzione al tempo finale \(T\) \textit{senza} convoluzione, cosicché,  per leggerezza di notazione, si può impropriamente denotare con \(\mathbf s^T\in\RPlus^{2\times N\times R}\) il tensore del terz'ordine tale che \(\mathbf s^T\equiv\undertilde{\mathbf s}^{N_s}\).
            \end{oss}

        \subsection{Definizione dei grafici}\label{secDefinizioneGrafici}

            \subsubsection{Istogrammi}

                I primi grafici considerano gl'istogrammi [normalizzati] della distribuzione al tempo finale \(T\).
                
                Si inizia considerando il massimo e il minimo elemento del tensore \(\mathbf s^T\),
                %
                \begin{equation*}
                    s_{\max}\equiv\max_{a,i,r}\ s^{T,a}_{i,r}
                    \quad\text{e}\quad
                    s_{\min}\equiv\min_{a,i,r}\ s^{T,a}_{i,r},
                \end{equation*}
                %
                coi quali si può definire una griglia comune equispaziata su cui costruire gl'istogrammi. Siano \(N_c\) il numero di intervalli della griglia, ossia il numero di classi, allora definita \(\Histogram\colon\mathbb R^N\to\mathbb R^{N_c}\) come la funzione che restituisce i valori [normalizzati] delle classi dell'istogramma, il tensore \(\mathbf h\in\RPlus^{2\times N_c\times R}\) si scrive
                %
                \begin{equation*}
                    \mathbf h^a_{\cdot,r}\equiv\Histogram(\mathbf s^{T,a}_{\cdot,r}),
                    \ \ \forall a\text{ e }\forall r,
                \end{equation*}
                %
                Valutati gl'intervalli di confidenza
                %
                \begin{equation*}
                    \IC_R^{0.05}(\mathbf h^a_c)=
                    \bar h^a_c\pm t^{0.025}_{R-1}\sqrt{\frac{v^a_c}R},
                    \ \ \forall a\text{ e }\forall c,
                \end{equation*}
                %
                si possono rappresentare con \(\bar{\mathbf h}^a\) l'istogramma medio della simulazione esatta e approssimata, e con \(\pm t^{0.025}_{R-1}\sqrt{v^a_c/R}\) l'errore stocastico sulle stime dei valori medi delle classi.

            \subsubsection{Lognormale bimodale}

                Per quanto riguarda i fittaggi lognormali bimodali la logica è simile agl'istogrammi: sia
                %
                \begin{equation*}
                    \mathcal L(x;\mathbf s^{T,a}_{\cdot,r})\colon\RPlus\to\RPlus,
                    \ \ \forall a\text{ e }\forall r,
                \end{equation*}
                %
                la densità di una distribuzione lognormale bimodale, con densità \cref{eqDistribuzioneLognormaleBimodale}, fittata dal vettore \(\mathbf s^{T,a}_{\cdot,r}\) e sia
                %
                \begin{equation*}
                    \mathcal L(x;\mathbf s^{T,a})\colon\RPlus\to\RPlus^R
                \end{equation*}
                %
                la funzione vettoriale tale che
                %
                \begin{equation*}
                    \mathcal L(x;\mathbf s^{T,a})\equiv
                    \begin{bmatrix}
                        \mathcal L(x;\mathbf s^{T,a}_{\cdot,1})&
                        \mathcal L(x;\mathbf s^{T,a}_{\cdot,2})&
                        \cdots&
                        \mathcal L(x;\mathbf s^{T,a}_{\cdot,R})
                    \end{bmatrix}^\top,
                \end{equation*}
                %
                che in essenza raccoglie puntualmente tutt'i fittaggi in un unico vettore. Allora gl'intervalli di confidenza hanno forma
                %
                \begin{equation*}
                    \IC_R^{0.05}(\mathcal L(x;\mathbf s^{T,a}))=
                    \bar{\mathcal L}^a(x)\pm t^{0.025}_{R-1}\sqrt{\frac{v^a(x)}R},
                    \ \ \forall x\in[s_{\min},s_{\max}]\text{ e }\forall a,
                \end{equation*}
                %
                in cui, come prima, \(\bar{\mathcal L}^a(x)\) è la funzione media mentre \(\pm t^{0.025}_{R-1}\sqrt{v^a(x)/R}\) la sua incertezza stocastica.

                \begin{oss}
                    Essendo \(x\) continuo, l'insieme degl'intervalli di confidenza forma per le funzioni un fascio di confidenza.
                \end{oss}

                Si nota, in conclusione, che i tre grafici sono in scala semilogaritmica per far emergere la distribuzione normale dalla lognormale bimodale.

            \subsubsection{Funzione di ripartizione empirica}\label{secFRCEmpirica}

                La distribuzione di Pareto è fittata a partire dalla funzione di ripartizione complementare (FRC) empirica che quindi va prima analizzata, ma solo nella coda della distribuzione della popolazione.
                
                Il dominio in questo caso viene definito mediante l'ultimo quartile, ossia quel valore \(s_\TreQuarti\) tale che \(\mathbb P(S\leq s_\TreQuarti)=3/4\), data una qualunque variabile aleatoria \(S\); ciò corrisponde nella pratica a trovare quell'elemento di \(\mathbf s^{T,a}_{\cdot,r}\) tale che
                %
                \begin{equation*}
                    \left\vert\left\{
                        s^{T,a}_{i,r}\geq s_\TreQuarti\ |\ i\in\{1,2,\ldots,N\}
                    \right\}\right\vert=\left\lceil\frac N4\right\rceil,
                    \ \ \forall a\text{ e }\forall r,
                \end{equation*}
                %
                che a parole vuol dire che \textit{almeno} \(1/4\) di tutti gli elementi di \(\mathbf s^{T,a}_{\cdot,r}\) sono superiori a \(s_\TreQuarti\). % Infatti se si considerasse la parte intera si troverebbe un valore soglia minore, e quindi un insieme con cardinalità piú piccola di N/4; ecco allora il perché dell'«almeno» che copre perfettamente i casi in cui N non sia divisibile da 4 nei quali, non potendo scegliere perfettamente 1/4 degli elementi, si può solo scegliere per eccesso (piú di N/4) o per difetto (meno di N/4)
                Allora l'intervallo \([s_\TreQuarti,s_{\max}]\) caratterizza la coda.

                La FRC empirica, che approssima la FRC \(\mathbb P(S>x)\), si scrive
                %
                \begin{equation}
                    \label{eqFunzioneRipartizioneComplementare}
                    \mathcal F(x;\mathbf s^a_{\cdot,r})\equiv
                    1-\frac1N\sum_{i=1}^N\chi_{[0,x]}
                    \left(s^{T,a}_{i,r}\right),
                    \ \ \forall a\text{ e }\forall r,
                \end{equation}
                %
                dove \(\chi_{[0,x]}\colon\mathbb R\to\{0,1\}\) è la funzione indicatrice dell'insieme \([0,x]\):
                %
                \begin{equation*}
                    \newcommand\hsep{\hspace{0.5em}}
                    \chi_{[0,x]}(w)\equiv
                    \left\{\begin{aligned}
                        1&\hsep\text{se }w\in[0,x],\\
                        0&\hsep\text{altrimenti}.
                    \end{aligned}\right.
                \end{equation*}
                %
                Tuttavia, siccome la \cref{eqFunzioneRipartizioneComplementare} dev'essere valutata in \(x\in[s_\TreQuarti,s_{\max}]\), e in particolare nel tensore \(\mathbf p\in\mathbb R^{2\times N_\UnQuarto\times R}\) di \(\mathbf s\) ristretto alla coda
                %
                \begin{equation*}
                    p^a_{i,r}\equiv s^{T,a}_{N-N_\UnQuarto+i,r}\ \ \forall a,\ \forall i\in\{1,2,\ldots,N_\UnQuarto\}\text{ e }\forall r
                \end{equation*}
                %
                nel quale \(N_\UnQuarto\equiv\left\lceil N/4\right\rceil\), vale il seguente riscalamento:
                %
                \begin{equation*}
                    \begin{aligned}
                        \mathcal F(x;\mathbf s^{T,a}_{\cdot,r})&=1-\frac{N-N_\UnQuarto}N-\frac1N\sum_{i=1}^{N_\UnQuarto}
                        \chi_{[0,x]}\left(s^{T,a}_{N-N_\UnQuarto+i,r}\right)\\&
                        % =\frac1N\left(
                        %     N_\UnQuarto-\sum_{i=1}^{N_\UnQuarto}
                        %     \chi_{[0,x]}\left(p^{T,a}_{i,r}\right)
                        % \right)%\\&
                        =\frac{N_\UnQuarto}N\left[
                            1-\frac1{N_\UnQuarto}\smash{\sum_{i=1}^{N_\UnQuarto}}
                            \chi_{[0,x]}\left(p^{T,a}_{i,r}\right)
                        \right]%\\&
                        =\frac{N_\UnQuarto}N\mathcal F(x;\mathbf p^a_{\cdot,r}),
                        \ \ \forall a\text{ e }\forall r,
                    \end{aligned}
                \end{equation*}
                %
                Vale a dire che \(\mathcal F(x;\mathbf p^a_{\cdot,r})\) è uguale a \(\mathcal F(x;\mathbf s^{T,a}_{\cdot,r})\) a meno di un riscalamento \(N_\UnQuarto/N\). Tuttavia l'indice di Pareto è invariante rispetto ai riscalamenti:
                
                \begin{oss}
                    In scala logaritmica la \cref{eqFunzioneRipartizioneComplementarePareto} si scrive
                    %
                    \begin{equation*}
                        \log(R)=\log(c)-\beta\log(s),
                    \end{equation*}
                    %
                    che equivale a una retta traslata di \(\log(c)\); indicando con \(R'\) il rapporto \(R/c\) segue
                    %
                    \begin{equation*}
                        \log(R')\equiv\log(R)-\log(c)=-\beta\log(s),
                    \end{equation*}
                    %
                    per la quale il coefficiente di Pareto \(\beta\) è invariato.
                \end{oss}

                Da quest'osservazione si può pertanto scegliere \(\mathcal F(x;\mathbf s^{T,a}_{\cdot,r})\) in scala logaritmica per una maggiore leggibilità dei valori i quali altrimenti, con \(\mathcal F(x;\mathbf p^a_{\cdot,r})\), sarebbero bassi. L'uso di tale scala introduce però un lieve problema: l'ultimo elemento \(p^a_{N_\UnQuarto,r}\) ha ordinata nulla ed è dunque non visibile; per ovviare il problema si può traslare la FRC empirica in scala lineare
                %
                \begin{equation}
                    \label{eqFRCEmpiricaFinale}
                    \mathcal F(x;\mathbf p^a_{\cdot,r})+\frac1{2N_\UnQuarto},
                    \ \ \forall a\text{ e }\forall r,
                \end{equation}
                %
                ciò corrisponde in scala logaritmica a una traslazione non uniforme dei valori che quindi vengono falsati, seppure di poco essendo \(1/2N_\UnQuarto\) una quantità piccola; per tale ragione tale modifica è valida \textit{solo graficamente} e non per il calcolo dell'indice di Pareto.

                Di conseguenza, nel complesso, la FRC empirica è raffigurata in scala logaritmica dal diagramma a dispersione delle coppie
                %
                \begin{equation*}
                    \left(
                        \frac N{N_\UnQuarto}\mathcal F(\bar p^a_i;\mathbf s^{T,a}_{\cdot,r})
                        +\frac1{2N_\UnQuarto},\bar p^a_i
                    \right)
                    =\left(
                        \mathcal F(\bar p^a_i;\mathbf p^a_{\cdot,r})
                        +\frac1{2N_\UnQuarto},\bar p^a_i
                    \right)
                    \ \ \forall a\text{ e }\forall i,
                \end{equation*}
                %
                ove \(\bar p^a_i\) viene dagl'intervalli di confidenza
                %
                \begin{equation*}
                    \IC_R^{0.05}(\mathbf p^a_i)=
                    \bar p^a_i\pm t^{0.025}_{R-1}\sqrt{\frac{v^a_i}R},
                    \ \ \forall a\text{ e }\forall i.
                \end{equation*}
                %
                \begin{oss}
                    Si noti che, contrariamente ai casi precedenti, l'intervallo di confidenza è orizzontale e non verticale; difatti l'immagine della FRC empirica è invariata rispetto alle simulazioni che modificano solo le popolazioni dei centri maggiori, ossia l'ascissa della FRC empirica.
                \end{oss}

            \subsubsection{Pareto e relativo indice}

                Per quanto riguarda il fittaggio della distribuzione di Pareto il ragionamento è analogo a quello visto per la lognormale bimodale: sia
                %
                \begin{equation*}
                    \mathcal P(x;\mathbf p^a_{\cdot,r})\colon\RPlus\to\RPlus,
                    \ \ \forall a\text{ e }\forall r,
                \end{equation*}
                %
                la densità di distribuzione di Pareto, con FRC \cref{eqFunzioneRipartizioneComplementarePareto}, fittata dal vettore \(\mathbf p^a_{\cdot,r}\) e sia
                %
                \begin{equation*}
                    \mathcal P(x;\mathbf p^a)\colon\RPlus\to\RPlus^R
                \end{equation*}
                %
                la funzione vettoriale tale che
                %
                \begin{equation*}
                    \mathcal P(x;\mathbf p^a)\equiv
                    \begin{bmatrix}
                        \mathcal P(x;\mathbf p^a_{\cdot,1})&
                        \mathcal P(x;\mathbf p^a_{\cdot,2})&
                        \cdots&
                        \mathcal P(x;\mathbf p^a_{\cdot,R})
                    \end{bmatrix}^\top.
                \end{equation*}
                %
                Allora gl'intervalli di confidenza hanno forma
                %
                \begin{equation*}
                    \IC_R^{0.05}\left(\frac N{N_\UnQuarto}\mathcal P(x;\mathbf p^a)\right)
                    =\bar{\mathcal P}^a(x)\pm t^{0.025}_{R-1}\sqrt{\frac{v^a(x)}R},
                    \ \ \forall x\in[s_\TreQuarti,s_{\max}]\text{ e }\forall a,
                \end{equation*}
                %
                ove \(N/N_\UnQuarto\) è il fattore di riscalamento descritto nel \cref{secFRCEmpirica} necessario per confrontare l'adattamento colla FRC empirica.

                \begin{oss}
                    È fittando i dati \(\mathbf p\) alla distribuzione di Pareto che si ricava il relativo indice \(\beta\), non dalla FRC empirica: ecco perché,  essendo puramente grafiche, le modifiche nella \cref{eqFRCEmpiricaFinale} sono tollerabili.
                \end{oss}
        
            \subsubsection{Evoluzione della taglia media}

                Si esprima con \(\bar{\mathbf s}\in\RPlus^{N_f\times2\times R}\) il tensore delle taglie medie rispetto ai nodi
                %
                \begin{equation*}
                    \bar s^{n,a}_r\equiv
                    \frac1N\sum_{i=1}^Ns^{n,a}_{i,r}
                    ,\ \ \forall n,\ \forall a\text{ e }\forall r,
                \end{equation*}
                %
                allora gl'intervallo di confidenza di \(\bar{\mathbf s}\) sono
                %
                \begin{equation*}
                    \IC_R^{0.05}(\bar{\mathbf s}^{n,a})=
                    \bar{\bar s}^{n,a}\pm t^{0.025}_{R-1}\sqrt{\frac{v^{n,a}}R},
                    \ \ \forall n\text{ e }\forall a.
                \end{equation*}
                %
                L'andamento della taglia media totale è raffigurato interpolando linearmente le coppie
                %
                \begin{equation*}
                    \left(t^f_n,\bar{\bar s}^{n,a}\right),
                    \ \ \forall n\text{ e }\forall a,
                \end{equation*}
                %
                a istanti temporali contingui; lo stesso vale per gli estremi degl'intervalli di confidenza \(\pm t^{0.025}_{R-1}\sqrt{v^{n,a}/R}\), che quindi diventano un fascio di confidenza.
        
            \subsubsection{Taglie medie \textit{vs} gradi}
            
                S'indichi ora con \(\bar{\mathbf s}\in\RPlus^{2\times R}\) il tensore delle taglie medie rispetto alle simulazioni
                %
                \begin{equation*}
                    \bar s^a_i\equiv
                    \frac1R\sum_{r=1}^Rs^{T,a}_{i,r}
                    ,\ \ \forall a\text{ e }\forall i,
                \end{equation*}
                %
                allora dalle coppie
                %
                \begin{equation*}
                    \left(k_i,\bar s^a_i\right),
                    \ \ \forall a\text{ e }\forall i,
                \end{equation*}
                %
                si può definire il grafico a dispersione che lega la taglia media tra tutte le simulazioni di un nodo al suo grado.

                \begin{oss}
                    \label{ossMancanzaICDensità}
                    In questo caso non si disegnano gl'intervalli di confidenza erché il grafico è già molto denso e aggiungere ulteriori barre verticali/orizzontali lo renderebbe difficilmente leggibile.
                \end{oss}
        
            \subsubsection{Evoluzioni delle taglie medie delle classi dei gradi}

                Sia \(\pazocal K\subset\{1,2,\ldots,N\}\) l'insieme dei gradi distinti di \(\mathbf k\), si denoti con \(\bar s\in\mathbb R^{N_f\times2\times\vert\pazocal K\vert}\) il tensore
                %
                \begin{equation*}
                    \bar s^{n,a}_k\equiv
                    \frac1{\vert\mathcal I_k\vert}\sum_{i\in\mathcal I_k}
                    \frac1R\sum_{r=1}^Rs^{n,a}_{i,r},
                    \ \ \forall n,\ \forall a\text{ e }\forall k,
                \end{equation*}
                %
                ove \(\mathcal I_k\) viene dalla \cref{eqInsiemeNodiGradok} e rappresenta l'insieme degl'indici con grado \(k\). Pertanto interpolando linearmente le coppie
                %
                \begin{equation*}
                    \left(t^f_n,\bar s^{n,a}_k\right),
                    \ \ \forall n,\ \forall a\text{ e }\forall k,
                \end{equation*}
                %
                a istanti temporali contigui, si può rappresentare l'andamento temporale della popolazione media della classe \(k\)-esima tra tutte le simulazioni.
               
                Si conclude notando che non si disegnano gl'intervalli di confidenza per la medesima ragione chiarita nell'\cref{ossMancanzaICDensità}.
        
        \subsection{Sulle fluttuazioni \texorpdfstring{\(\gamma\)}{γ}}

            Nelle \ref{eqRegoleInterazioneUrbaneEsplicite} è stato introdotto per completezza il termine \(\gamma\) legato alle fluttuazioni, ossia a quei fenomeni di morte e nascita che caratterizzano la naturale variazione di una popolazione intorno a una media. Si argomenta adesso che è possibile trascurarle per due ragioni:

            \begin{enumerate}[
                label=\arabic*.,
                % topsep=0.5em,
                % parsep=0em,
                % itemsep=0.25em,
                % leftmargin=2em,
                % rightmargin=1.5em,
                % \leftmargin + \itemindent = \labelindent + \labelwidth + \labelsep
                %itemindent=!,
                %labelindent=3em,
                %labelwidth=!,
                %labelsep=!,
            ]
                \item si è interessati solo alla predizione della distribuzione stazionaria da confrontare con quella reale del 1991 e
                \item dal vincolo \cref{eqVincoloSimmetricoTraParametri} le uniche perturbazioni simmetriche sono per loro natura piccole, essendo concentrate in un intorno dell'origine.
            \end{enumerate}

            In effetti la \cref{figConfrontoConESenzaFluttuazioni} conferma che \(\gamma\) introduce solo una lieve incertezza sui risultati, simulati coi parametri nella \cref{tabParametriConEsenzaGamma} e colla regola d'interazione
            %
            \begin{equation}
                E(s,s_*,i,i_*)\equiv
                (1-\zeta)\lambda\frac{\frac{(s_*/s)(w_{i_*}/w_i)}\alpha}{1+\frac{(s_*/s)(w_{i_*}/w_i)}\alpha}
                +\zeta\lambda\frac{\frac{(s/s_*)(w_i/w_{i_*})}\alpha}{1+\frac{(s/s_*)(w_i/w_{i_*})}\alpha},
                \label{eqVarianteRegolaTagliaForzaFrazionata}
            \end{equation}
            %
            che è spiegata implicitamente nel corso del \cref{secRegoleEmigrazione} ed è brevemente discussa piú nel dettaglio nel \cref{secVariantiRegoleInterazione}.

            \begin{table}[htb]
                \centering
                \begin{tabular}{cccccc}
                    \hline
                    &\(\lambda\)&\(\alpha\)&\(\zeta\)&\(\sigma\)&\(N_t\)\\
                    \hline
                    Senza \(\gamma\)&\(0.1\)&\(1\)&\(0.1\)&--&\(10^6\)\\
                    Con \(\gamma\)&\(0.1\)&\(1\)&\(0.1\)&\(0.05\)&\(10^6\)\\
                    \hline
                \end{tabular}
                \caption{Parametri della \cref{figConfrontoConESenzaFluttuazioni}.}
                \label{tabParametriConEsenzaGamma}
            \end{table}

            Dunque \(\gamma\) introduce solo del lieve rumore che, sebbene abbia il nobile scopo di rendere piú realistico il modello, non aiuta a interpretare i risultati rispetto a quelli reali.

            \tikzfigure{p}
                {
% !TEX root = ../../../../Esperimenti/.tex/MEF.tex
% LTeX: language=it

\begin{tikzpicture}%
    %\begingroup Comandi
        \definecolor{colore}{rgb}{0.50196,0.50196,0.50196}%
    
        % \renewcommand\titleSize{12}
        \renewcommand\subCaptionSize{9}
        \renewcommand\labelSize{8}
        \renewcommand\tickSize{7}
        \renewcommand\legendSize{6}

        \pgfmathsetmacro\plotWidth{7}
        \pgfmathsetmacro\plotHeight{.618*\plotWidth}
        \pgfmathsetmacro\halfHeightPlot{\plotHeight/2}
        \pgfmathsetmacro\lineWidth{.5pt}
        \pgfmathsetmacro\lineMidWidth{.75pt}
        \pgfmathsetmacro\lineThickWidth{1.2pt}
        
        \pgfmathsetmacro\xPlotSep{2.25}
        \pgfmathsetmacro\yPlotSepI{1.5}
        \pgfmathsetmacro\yPlotSepII{3}
        \pgfmathsetmacro\xLegendSep{.25cm}
    %\endgroup

    %\begingroup Impostazioni pgfplots
        %\begingroup Gruppi
        \pgfplotsset{
            /pgfplots/group/plotGroup/.style={
                horizontal sep=\xPlotSep em,
                vertical sep=\yPlotSepI em,
                % x descriptions at=edge bottom,
            },
            /pgfplots/group/plotGroup1/.style={
                plotGroup,
                group name=Group1,
                group size=2 by 4,
            },
            /pgfplots/group/plotGroup2/.style={
                plotGroup,
                group name=Group2,
                group size=2 by 1,
            }
        }
        %\endgroup

        %\begingroup Colonne
        \pgfplotsset{
            plotColumn/.style={
                width=\plotWidth cm,
                height=\plotHeight cm,
                scale only axis,
                xlabel style={font=\color{white!15!black}\dimTesto{\labelSize}},
                ylabel style={font=\color{white!15!black}\dimTesto{\labelSize}},
                xticklabel style={font=\dimTesto{\tickSize}},
                yticklabel style={font=\dimTesto{\tickSize}},
                axis x line*=bottom,
                xmajorgrids,ymajorgrids,
                xminorticks=true,yminorticks=true,
                axis background/.style={fill=white},
                grid style={dashed}
            },
            plotColumn1/.style={
                plotColumn,
                axis y line*=left,
                ylabel={\(\underhat{\bar f}_N(s)\)},
                y tick scale label style={
                    xshift=-.4cm,
                    at={(0,1)},
                    anchor=south east,
                },
            },
            plotColumn2/.style={
                plotColumn,
                axis y line*=right,
                ylabel={\(\underhat{\bar f}_N(s)\)},
                y tick scale label style={
                    xshift=.4cm,
                    at={(1,1)},
                    anchor=south west,
                },
            },
        }
        %\endgroup

        %\begingroup Legende
        \pgfplotsset{
            plotLegend/.style={
                font=\dimTesto{\legendSize},
                legend cell align=left,
                align=left,
                % draw=white!15!black,
                draw=none, % Bordo invisibile
                fill=white,
                fill opacity=0.5,
                text opacity=1,
            },
            plotLegend1/.style={legend style={plotLegend},legend pos=north east},
            plotLegend2/.style={legend style={plotLegend},legend pos=south west},
            plotLegend3/.style={legend style={plotLegend},legend pos=north west},
        }
        %\endgroup
    %\endgroup

    \begin{groupplot}[group style=plotGroup1]
        \nextgroupplot[%
            plotColumn1,plotLegend1,xmode=log,
            xmin=100,xmax=1000000,
            ymin=0,ymax=0.0004,
        ] 
% !TEX root = ../../../../Esperimenti/.tex/MEF.tex

% This file was created by matlab2tikz.
%
\definecolor{mycolor1}{rgb}{0.00000,0.44706,0.69804}%

\addplot[ybar interval, fill=mycolor1, fill opacity=0.35, area legend, draw=none] table[row sep=crcr, x=Lower, y=Count] {%
Lower	Upper	Count\\
102.379054928703	140.510222226835	2.09802126891402e-06\\
140.510222226835	192.843375668819	6.62422659186734e-06\\
192.843375668819	264.668057241483	1.11382324638733e-05\\
264.668057241483	363.243903406255	3.19212747249152e-05\\
363.243903406255	498.534408485211	7.4506337263781e-05\\
498.534408485211	684.21397885303	0.000148068703944493\\
684.21397885303	939.050065331212	0.000237329025240615\\
939.050065331212	1288.80007198445	0.000300481290447917\\
1288.80007198445	1768.81477023409	0.000319379109068321\\
1768.81477023409	2427.61135680322	0.000251286466933288\\
2427.61135680322	3331.77730017487	0.000159793676215266\\
3331.77730017487	4572.70062889246	8.40664347150339e-05\\
4572.70062889246	6275.80692154186	4.02088820266589e-05\\
6275.80692154186	8613.23662161811	1.92690287106948e-05\\
8613.23662161811	11821.244029247	1.00332270399765e-05\\
11821.244029247	16224.0765623776	5.74781495236032e-06\\
16224.0765623776	22266.7478693998	3.51720824341956e-06\\
22266.7478693998	30560.0173158182	1.8199497111301e-06\\
30560.0173158182	41942.1221195282	8.48114957629527e-07\\
41942.1221195282	57563.5016731115	3.46533628145e-07\\
57563.5016731115	79003.077513036	8.45787876995481e-08\\
79003.077513036	108427.841863662	1.72190560518759e-08\\
108427.841863662	148811.885072089	1.98098044782464e-09\\
148811.885072089	204237	0\\
204237	204237	0\\
};
\addlegendentry{Istogramma esatto}

\addplot [color=mycolor1, only marks, every error bar/.append style={opacity=0.45}, mark=*, mark size=0pt, draw=none, forget plot]
plot [error bars/.cd, y dir=both, y explicit, error bar style={line width=1pt, color=mycolor1}, error mark options={mark=none,mark size=0pt}]
table[row sep=crcr, y error plus index=2, y error minus index=3]{%
119.938750032695	2.09802126891402e-06	2.37906706401472e-06	2.09802126891402e-06\\
164.610040915489	6.62422659186734e-06	3.97604216234847e-06	3.97604216234847e-06\\
225.919192611331	1.11382324638733e-05	4.25327586005448e-06	4.25327586005448e-06\\
310.062990728249	3.19212747249152e-05	7.29306948821726e-06	7.29306948821726e-06\\
425.546219017977	7.4506337263781e-05	1.10840948860289e-05	1.10840948860289e-05\\
584.041275274966	0.000148068703944493	1.80305748299039e-05	1.80305748299039e-05\\
801.56795191828	0.000237329025240615	2.23528160541064e-05	2.23528160541064e-05\\
1100.11262686866	0.000300481290447917	1.9395704657882e-05	1.9395704657882e-05\\
1509.85052346411	0.000319379109068321	1.24510062274221e-05	1.24510062274221e-05\\
2072.19565299745	0.000251286466933288	6.49863316385546e-06	6.49863316385546e-06\\
2843.98671098226	0.000159793676215266	6.50191448775475e-06	6.50191448775474e-06\\
3903.23201409284	8.40664347150339e-05	4.89901996623344e-06	4.89901996623344e-06\\
5356.9941438219	4.02088820266589e-05	2.73278910825682e-06	2.73278910825682e-06\\
7352.21123246801	1.92690287106948e-05	1.24608353929241e-06	1.24608353929241e-06\\
10090.5486464213	1.00332270399765e-05	6.20808867037839e-07	6.20808867037839e-07\\
13848.7821916965	5.74781495236032e-06	4.22395785808937e-07	4.22395785808937e-07\\
19006.7730619456	3.51720824341956e-06	2.32660119291265e-07	2.32660119291265e-07\\
26085.8620799815	1.8199497111301e-06	1.2927446113576e-07	1.2927446113576e-07\\
35801.5639076695	8.48114957629527e-07	1.01672901232989e-07	1.01672901232989e-07\\
49135.8872597342	3.46533628145e-07	5.7461208056497e-08	5.7461208056497e-08\\
67436.5908435666	8.45787876995481e-08	2.69167734993982e-08	2.69167734993983e-08\\
92553.4072594094	1.72190560518759e-08	9.45852275954512e-09	9.45852275954512e-09\\
127025.003609643	1.98098044782464e-09	2.24634774094364e-09	1.98098044782464e-09\\
174335.575748234	0	0	0\\
};
\addplot [color=mycolor1]
table[row sep=crcr]{%
102.379054928703	2.39319287028901e-06\\
103.160714811762	2.43085681373844e-06\\
103.948342635952	2.46910741852549e-06\\
104.741983966257	2.50795572265851e-06\\
105.54168471555	2.54741305061813e-06\\
106.347491147248	2.58749102170338e-06\\
107.159449877983	2.62820155858502e-06\\
107.977607880307	2.66955689606912e-06\\
108.802012485405	2.71156959007391e-06\\
109.632711385831	2.75425252682304e-06\\
110.469752638272	2.79761893225806e-06\\
111.313184666327	2.84168238167323e-06\\
112.163056263304	2.88645680957526e-06\\
113.019416595047	2.93195651977115e-06\\
113.882315202781	2.97819619568639e-06\\
114.751802005974	3.0251909109167e-06\\
115.627927305229	3.07295614001547e-06\\
116.51074178519	3.12150776951955e-06\\
117.400296517478	3.17086210921597e-06\\
118.296642963642	3.22103590365155e-06\\
119.199832978139	3.27204634388788e-06\\
120.109918811333	3.32391107950372e-06\\
121.026953112515	3.37664823084659e-06\\
121.950988932956	3.43027640153579e-06\\
122.882079728965	3.48481469121813e-06\\
123.820279364994	3.54028270857823e-06\\
124.765642116742	3.59670058460482e-06\\
125.718222674305	3.65408898611409e-06\\
126.678076145334	3.71246912953141e-06\\
127.645258058225	3.77186279493231e-06\\
128.619824365331	3.83229234034339e-06\\
129.601831446199	3.89378071630369e-06\\
130.59133611083	3.95635148068711e-06\\
131.58839560297	4.02002881378574e-06\\
132.593067603415	4.08483753365412e-06\\
133.605410233356	4.15080311171422e-06\\
134.625482057733	4.21795168862047e-06\\
135.653342088628	4.28631009038406e-06\\
136.689049788679	4.35590584475547e-06\\
137.732665074517	4.42676719786386e-06\\
138.784248320236	4.49892313111179e-06\\
139.843860360883	4.57240337832316e-06\\
140.911562495976	4.6472384431423e-06\\
141.987416493056	4.72345961668166e-06\\
143.071484591254	4.80109899541509e-06\\
144.163829504894	4.88018949931356e-06\\
145.264514427124	4.96076489021983e-06\\
146.373603033566	5.04285979045788e-06\\
147.491159486005	5.12650970167292e-06\\
148.617248436097	5.21175102389739e-06\\
149.751935029113	5.29862107483733e-06\\
150.895284907701	5.38715810937408e-06\\
152.047364215693	5.47740133927496e-06\\
153.208239601924	5.56939095310657e-06\\
154.37797822409	5.66316813634387e-06\\
155.556647752633	5.7587750916674e-06\\
156.744316374658	5.85625505944097e-06\\
157.941052797874	5.95565233836133e-06\\
159.146926254571	6.05701230627082e-06\\
160.362006505625	6.16038144112356e-06\\
161.586363844532	6.26580734209522e-06\\
162.820069101478	6.37333875082565e-06\\
164.063193647433	6.48302557278325e-06\\
165.315809398282	6.59491889873936e-06\\
166.577988818986	6.70907102634017e-06\\
167.84980492777	6.82553548176304e-06\\
169.131331300355	6.94436704144394e-06\\
170.422642074203	7.06562175386118e-06\\
171.723811952819	7.18935696136083e-06\\
173.034916210062	7.31563132200795e-06\\
174.356030694505	7.44450483144717e-06\\
175.687231833822	7.57603884475581e-06\\
177.02859663921	7.71029609827136e-06\\
178.380202709841	7.84734073137483e-06\\
179.742128237356	7.98723830821092e-06\\
181.114452010385	8.13005583932435e-06\\
182.497253419104	8.27586180319186e-06\\
183.890612459833	8.4247261676282e-06\\
185.294609739659	8.57672041104314e-06\\
186.7093264811	8.73191754352686e-06\\
188.134844526807	8.89039212773877e-06\\
189.571246344294	9.05222029957555e-06\\
191.018615030712	9.21747978859171e-06\\
192.477034317655	9.38624993814682e-06\\
193.946588576005	9.55861172525097e-06\\
195.427362820812	9.73464778008055e-06\\
196.919442716211	9.91444240513479e-06\\
198.422914580381	1.00980815940025e-05\\
199.937865390536	1.02856530497087e-05\\
201.464382787958	1.04772462026076e-05\\
203.002555083066	1.06729522277907e-05\\
204.552471260524	1.08728640619746e-05\\
206.114220984395	1.10770764198346e-05\\
207.687894603317	1.12856858097479e-05\\
209.273583155741	1.14987905489103e-05\\
210.871378375191	1.17164907777879e-05\\
212.481372695573	1.19388884738667e-05\\
214.103659256522	1.2166087464659e-05\\
215.73833190879	1.2398193439928e-05\\
217.385485219676	1.26353139630887e-05\\
219.045214478497	1.28775584817426e-05\\
220.717615702098	1.31250383373044e-05\\
222.402785640412	1.33778667736758e-05\\
224.100821782051	1.36361589449228e-05\\
225.811822359948	1.39000319219105e-05\\
227.535886357044	1.41696046978488e-05\\
229.273113512007	1.44449981927036e-05\\
231.023604325008	1.4726335256423e-05\\
232.787460063531	1.50137406709333e-05\\
234.564782768236	1.53073411508525e-05\\
236.355675258854	1.56072653428737e-05\\
238.160241140146	1.5913643823766e-05\\
239.978584807887	1.62266090969433e-05\\
241.810811454912	1.65462955875483e-05\\
243.657027077199	1.68728396359994e-05\\
245.517338479998	1.72063794899482e-05\\
247.391853284017	1.75470552945939e-05\\
249.28067993164	1.78950090813013e-05\\
251.183927693206	1.82503847544672e-05\\
253.101706673328	1.86133280765828e-05\\
255.034127817263	1.8983986651436e-05\\
256.981302917332	1.93625099053986e-05\\
258.943344619384	1.9749049066745e-05\\
260.920366429315	2.01437571429451e-05\\
262.912482719633	2.05467888958786e-05\\
264.919808736078	2.09583008149136e-05\\
266.942460604284	2.13784510877965e-05\\
268.980555336501	2.18073995692965e-05\\
271.03421083836	2.22453077475525e-05\\
273.103545915701	2.26923387080652e-05\\
275.188680281438	2.3148657095285e-05\\
277.28973456249	2.36144290717377e-05\\
279.406830306757	2.40898222746393e-05\\
281.540089990152	2.4575005769946e-05\\
283.689637023688	2.50701500037886e-05\\
285.855595760614	2.55754267512415e-05\\
288.038091503612	2.60910090623772e-05\\
290.237250512045	2.66170712055563e-05\\
292.453200009261	2.71537886079089e-05\\
294.686068189952	2.77013377929591e-05\\
296.935984227572	2.82598963153495e-05\\
299.203078281811	2.88296426926225e-05\\
301.487481506119	2.94107563340174e-05\\
303.789326055299	3.00034174662424e-05\\
306.108745093152	3.06078070561866e-05\\
308.445872800177	3.12241067305328e-05\\
310.800844381337	3.18524986922395e-05\\
313.173796073877	3.24931656338605e-05\\
315.564865155213	3.31462906476716e-05\\
317.974189950861	3.38120571325785e-05\\
320.401909842454	3.44906486977818e-05\\
322.848165275792	3.51822490631764e-05\\
325.313097768977	3.58870419564673e-05\\
327.796849920595	3.66052110069861e-05\\
330.299565417964	3.7336939636194e-05\\
332.821389045452	3.80824109448624e-05\\
335.362466692849	3.88418075969258e-05\\
337.922945363805	3.96153117000007e-05\\
340.502973184339	4.04031046825746e-05\\
343.102699411407	4.12053671678668e-05\\
345.722274441534	4.20222788443692e-05\\
348.361849819517	4.28540183330804e-05\\
351.02157824719	4.37007630514457e-05\\
353.701613592262	4.45626890740272e-05\\
356.402110897212	4.54399709899236e-05\\
359.123226388267	4.63327817569716e-05\\
361.865117484429	4.72412925527586e-05\\
364.627942806591	4.81656726224886e-05\\
367.41186218671	4.91060891237379e-05\\
370.217036677053	5.00627069681516e-05\\
373.043628559514	5.10356886601309e-05\\
375.891801355004	5.2025194132568e-05\\
378.761719832906	5.30313805796893e-05\\
381.653550020615	5.40544022870762e-05\\
384.567459213134	5.50944104589315e-05\\
387.503615982761	5.61515530426732e-05\\
390.462190188833	5.72259745509335e-05\\
393.443352987558	5.83178158810535e-05\\
396.447276841914	5.94272141321669e-05\\
399.474135531625	6.05543024199698e-05\\
402.524104163219	6.16992096892824e-05\\
405.597359180154	6.28620605245117e-05\\
408.694078373027	6.40429749581293e-05\\
411.814440889858	6.52420682772877e-05\\
414.958627246454	6.64594508286998e-05\\
418.126819336855	6.7695227821914e-05\\
421.319200443854	6.8949499131123e-05\\
424.535955249599	7.02223590956493e-05\\
427.77726984628	7.15138963192552e-05\\
431.043331746893	7.2824193468435e-05\\
434.334329896089	7.41533270698457e-05\\
437.6504546811	7.55013673070411e-05\\
440.991897942762	7.68683778166862e-05\\
444.358852986605	7.82544154844189e-05\\
447.751514594037	7.96595302405474e-05\\
451.170079033618	8.1083764855767e-05\\
454.614744072408	8.25271547370869e-05\\
458.08570898741	8.39897277241663e-05\\
461.5831745771	8.54715038862575e-05\\
465.107343173043	8.69724953199656e-05\\
468.658418651595	8.84927059480319e-05\\
472.236606445701	9.00321313193601e-05\\
475.842113556776	9.15907584105e-05\\
479.475148566685	9.31685654288166e-05\\
483.135921649807	9.47655216175683e-05\\
486.824644585191	9.6381587063127e-05\\
490.541530768814	9.80167125045765e-05\\
494.286795225918	9.96708391459218e-05\\
498.060654623459	0.000101343898471158\\
501.863327282632	0.000103035812062435\\
505.695033191508	0.00010474649142157\\
509.555994017757	0.000106475837795159\\
513.446433121472	0.000108223742003525\\
517.366575568092	0.000109990084273778\\
521.316648141422	0.000111774734077219\\
525.296879356751	0.000113577549971361\\
529.307499474071	0.000115398379446815\\
533.348740511404	0.000117237058779303\\
537.420836258218	0.000119093412887051\\
541.524022288951	0.000120967255193828\\
545.658535976647	0.000122858387497881\\
549.824616506682	0.000124766599847033\\
554.0225048906	0.000126691670420191\\
558.25244398006	0.000128633365415523\\
562.514678480885	0.000130591438945562\\
566.809454967215	0.000132565632939484\\
571.137021895773	0.000134555677052808\\
575.497629620241	0.000136561288584775\\
579.891530405739	0.00013858217240363\\
584.318978443423	0.000140618020880074\\
588.780229865185	0.000142668513829096\\
593.275542758476	0.00014473331846044\\
597.805177181235	0.000146812089337915\\
602.369395176928	0.000148904468347783\\
606.968460789717	0.000151010084676442\\
611.602640079728	0.000153128554797607\\
616.272201138446	0.000155259482469204\\
620.977414104223	0.000157402458740169\\
625.718551177906	0.000159557061967347\\
630.495886638585	0.000161722857842672\\
635.30969685946	0.000163899399430812\\
640.160260323826	0.000166086227217438\\
645.04785764119	0.000168282869168282\\
649.972771563501	0.00017048884079914\\
654.935287001506	0.000172703645256947\\
659.935691041233	0.000174926773412073\\
664.974272960603	0.000177157703961953\\
670.051324246158	0.000179395903546161\\
675.167138609933	0.000181640826873039\\
680.322012006438	0.00018389191685796\\
685.516242649784	0.000186148604773306\\
690.750131030938	0.000188410310410237\\
696.023979935099	0.000190676442252294\\
701.33809445922	0.000192946397660887\\
706.692782029658	0.000195219563072687\\
712.088352419958	0.000197495314208963\\
717.525117768769	0.000199773016296829\\
723.00339259791	0.000202052024302436\\
728.52349383056	0.000204331683176051\\
734.085740809593	0.000206611328109004\\
739.690455316055	0.000208890284802443\\
745.337961587772	0.000211167869747844\\
751.028586338118	0.000213443390519183\\
756.762658774907	0.000215716146076675\\
762.540510619441	0.000217985427081993\\
768.362476125704	0.00022025051622481\\
774.228892099692	0.000222510688560557\\
780.140097918901	0.000224765211859214\\
786.096435551963	0.000227013346964998\\
792.098249578423	0.000229254348166728\\
798.145887208678	0.000231487463578701\\
804.239698304065	0.000233711935531838\\
810.380035397094	0.000235927000974898\\
816.567253711847	0.00023813189188548\\
822.801711184529	0.0002403258356906\\
829.083768484172	0.000242508055696519\\
835.413789033501	0.000244677771527582\\
841.792139029961	0.000246834199573721\\
848.219187466894	0.000248976553446337\\
854.695306154899	0.000251104044442216\\
861.220869743325	0.000253215882015131\\
867.79625574196	0.000255311274254773\\
874.421844542861	0.000257389428372646\\
881.098019442364	0.000259449551194528\\
887.825166663254	0.000261490849659104\\
894.603675377116	0.000263512531322367\\
901.433937726837	0.000265513804867343\\
908.316348849305	0.000267493880618739\\
915.251306898262	0.000269451971062027\\
922.239213067332	0.000271387291366541\\
929.280471613241	0.0002732990599121\\
936.375489879197	0.000275186498818685\\
943.524678318455	0.000277048834478683\\
950.728450518066	0.000278885298091203\\
957.987223222801	0.000280695126197957\\
965.301416359258	0.000282477561220199\\
972.671453060162	0.000284231851996194\\
980.097759688832	0.000285957254318704\\
987.580765863859	0.000287653031471947\\
995.120904483953	0.000289318454767499\\
1002.71861175299	0.0002909528040786\\
1010.37432720523	0.000292555368372304\\
1018.08849373079	0.000294125446238952\\
1025.86155760121	0.000295662346418381\\
1033.69396849529	0.000297165388322349\\
1041.58617952513	0.000298633902552603\\
1049.53864726231	0.000300067231414036\\
1057.55183176433	0.00030146472942239\\
1065.62619660117	0.000302825763805944\\
1073.7622088822	0.000304149715000633\\
1081.96033928312	0.000305435977138061\\
1090.22106207322	0.000306683958525851\\
1098.54485514283	0.000307893082119801\\
1106.93220003095	0.000309062785987306\\
1115.38358195309	0.000310192523761516\\
1123.89948982939	0.00031128176508571\\
1132.48041631286	0.000312329996047368\\
1141.1268578179	0.000313336719601437\\
1149.839314549	0.000314301455982281\\
1158.61829052972	0.000315223743103852\\
1167.46429363178	0.000316103136947565\\
1176.37783560452	0.000316939211937454\\
1185.35943210443	0.000317731561302111\\
1194.40960272504	0.000318479797423004\\
1203.52887102695	0.000319183552168713\\
1212.71776456812	0.000319842477214689\\
1221.97681493439	0.000320456244348124\\
1231.30655777025	0.000321024545757561\\
1240.70753280979	0.000321547094306855\\
1250.18028390797	0.000322023623793148\\
1259.72535907206	0.000322453889188518\\
1269.34331049332	0.000322837666864982\\
1279.03469457899	0.000323174754802559\\
1288.80007198445	0.000323464972780107\\
1298.64000764567	0.000323708162548677\\
1308.55507081185	0.000323904187987141\\
1318.54583507842	0.000324052935239873\\
1328.61287842018	0.000324154312836287\\
1338.75678322474	0.00032420825179204\\
1348.9781363262	0.000324214705691767\\
1359.27752903913	0.000324173650753178\\
1369.65555719278	0.000324085085872434\\
1380.1128211655	0.000323949032650694\\
1390.64992591951	0.000323765535401765\\
1401.26748103591	0.000323534661140808\\
1411.96610074992	0.000323256499554087\\
1422.7464039864	0.00032293116294974\\
1433.6090143957	0.000322558786189608\\
1444.5545603897	0.000322139526602171\\
1455.5836751782	0.000321673563876639\\
1466.69699680551	0.000321161099938308\\
1477.89516818739	0.000320602358805286\\
1489.17883714823	0.000319997586426728\\
1500.54865645854	0.000319347050502733\\
1512.0052838727	0.000318651040286097\\
1523.54938216702	0.000317909866366107\\
1535.18161917809	0.000317123860434623\\
1546.90266784138	0.000316293375034678\\
1558.71320623023	0.000315418783291871\\
1570.613917595	0.000314500478628842\\
1582.60549040267	0.000313538874463145\\
1594.68861837664	0.000312534403888834\\
1606.86400053683	0.000311487519342128\\
1619.13234124017	0.000310398692251515\\
1631.49435022133	0.000309268412672685\\
1643.95074263376	0.000308097188908695\\
1656.50223909109	0.000306885547115797\\
1669.1495657088	0.000305634030895357\\
1681.89345414622	0.000304343200872332\\
1694.73464164889	0.000303013634260772\\
1707.67387109119	0.000301645924416836\\
1720.71189101928	0.000300240680379816\\
1733.84945569449	0.000298798526401693\\
1747.08732513687	0.00029732010146575\\
1760.42626516921	0.000295806058794773\\
1773.8670474613	0.000294257065349396\\
1787.41044957462	0.000292673801317153\\
1801.05725500729	0.000291056959592802\\
1814.80825323942	0.000289407245250496\\
1828.66423977876	0.000287725375008398\\
1842.62601620672	0.000286012076686322\\
1856.69439022475	0.000284268088657001\\
1870.87017570109	0.000282494159291592\\
1885.15419271781	0.000280691046400021\\
1899.54726761828	0.000278859516666775\\
1914.05023305497	0.000277000345082766\\
1928.66392803762	0.000275114314373877\\
1943.38919798175	0.000273202214426797\\
1958.22689475763	0.000271264841712781\\
1973.17787673949	0.000269302998709928\\
1988.24300885525	0.000267317493324604\\
2003.42316263648	0.000265309138312617\\
2018.71921626889	0.000263278750700743\\
2034.13205464308	0.000261227151209216\\
2049.66256940575	0.000259155163675765\\
2065.31165901132	0.000257063614481798\\
2081.08022877384	0.000254953331981313\\
2096.9691909194	0.000252825145933107\\
2112.9794646389	0.000250679886936847\\
2129.11197614124	0.000248518385873575\\
2145.36765870687	0.000246341473351169\\
2161.74745274181	0.000244149979155321\\
2178.25230583203	0.000241944731706529\\
2194.88317279829	0.000239726557523647\\
2211.64101575135	0.000237496280694454\\
2228.52680414767	0.000235254722353762\\
2245.54151484545	0.000233002700169513\\
2262.68613216118	0.000230741027837325\\
2279.96164792655	0.000228470514583934\\
2297.36906154586	0.000226191964679952\\
2314.9093800538	0.000223906176962355\\
2332.58361817375	0.000221613944367084\\
2350.39279837646	0.000219316053472151\\
2368.33795093917	0.000217013284051592\\
2386.42011400529	0.000214706408640612\\
2404.64033364438	0.000212396192112258\\
2422.99966391271	0.000210083391265899\\
2441.49916691423	0.000207768754427822\\
2460.13991286201	0.000205453021064198\\
2478.92298014014	0.000203136921406669\\
2497.84945536614	0.000200821176090784\\
2516.9204334538	0.000198506495807498\\
2536.13701767655	0.000196193580967914\\
2555.50031973125	0.000193883121381456\\
2575.01145980252	0.000191575795947606\\
2594.67156662757	0.000189272272361349\\
2614.48177756143	0.000186973206832437\\
2634.44323864282	0.000184679243818568\\
2654.55710466042	0.000182391015772547\\
2674.82453921964	0.0001801091429035\\
2695.24671481002	0.000177834232952162\\
2715.82481287296	0.000175566880980272\\
2736.56002387015	0.000173307669174069\\
2757.45354735239	0.000171057166661879\\
2778.50659202901	0.000168815929345743\\
2799.72037583779	0.000166584499747062\\
2821.09612601541	0.000164363406866161\\
2842.63507916843	0.00016215316605571\\
2864.33848134488	0.000159954278907881\\
2886.20758810631	0.000157767233155152\\
2908.24366460042	0.00015559250258459\\
2930.44798563427	0.000153430546965503\\
2952.82183574802	0.000151281811990268\\
2975.36650928923	0.000149146729228187\\
2998.08331048777	0.000147025716092159\\
3020.97355353122	0.000144919175817994\\
3044.03856264098	0.00014282749745612\\
3067.27967214876	0.000140751055875497\\
3090.6982265739	0.000138690211779478\\
3114.29558070104	0.000136645311733368\\
3138.07309965858	0.000134616688203444\\
3162.03215899761	0.000132604659607152\\
3186.17414477149	0.000130609530374215\\
3210.50045361603	0.000128631591018364\\
3235.01249283033	0.00012667111821941\\
3259.71168045813	0.000124728374915345\\
3284.59944536989	0.000122803610404185\\
3309.67722734543	0.000120897060455228\\
3334.94647715724	0.000119008947429425\\
3360.4086566544	0.000117139480408541\\
3386.06523884712	0.000115288855332772\\
3411.91770799201	0.000113457255146508\\
3437.96755967793	0.000111644849951895\\
3464.21630091246	0.000109851797169873\\
3490.66545020915	0.000108078241708355\\
3517.31653767532	0.000106324316137205\\
3544.17110510062	0.000104590140869681\\
3571.23070604618	0.000102875824350013\\
3598.49690593453	0.000101181463246767\\
3625.97128214009	9.95071426516708e-05\\
3653.65542408053	9.7852936283558e-05\\
3681.55093330861	9.62189066971033e-05\\
3709.65942360489	9.46051054960156e-05\\
3737.9825210711	9.30115735503622e-05\\
3766.52186422418	9.1438341217698e-05\\
3795.27910409107	8.9885428567679e-05\\
3824.25590430422	8.83528456098426e-05\\
3853.45394119789	8.68405925242389e-05\\
3882.87490390504	8.5348659894608e-05\\
3912.52049445511	8.38770289437966e-05\\
3942.39242787247	8.24256717711147e-05\\
3972.49243227562	8.09945515913419e-05\\
4002.82224897718	7.95836229750923e-05\\
4033.38363258461	7.81928320902589e-05\\
4064.17835110174	7.68221169442628e-05\\
4095.20818603102	7.54714076268359e-05\\
4126.47493247662	7.41406265530796e-05\\
4157.98039924825	7.28296887065389e-05\\
4189.72640896578	7.15385018820506e-05\\
4221.71479816475	7.02669669281194e-05\\
4253.94741740255	6.90149779885934e-05\\
4286.42613136551	6.77824227434092e-05\\
4319.15281897675	6.65691826481938e-05\\
4352.12937350489	6.53751331725078e-05\\
4385.3577026736	6.42001440365307e-05\\
4418.83972877192	6.30440794459941e-05\\
4452.57738876551	6.19067983251739e-05\\
4486.57263440865	6.07881545477657e-05\\
4520.82743235723	5.9687997165469e-05\\
4555.34376428243	5.86061706341218e-05\\
4590.12362698543	5.75425150372266e-05\\
4625.16903251291	5.64968663067234e-05\\
4660.48200827344	5.54690564408698e-05\\
4696.06459715477	5.4458913719097e-05\\
4731.918857642	5.34662629137168e-05\\
4768.0468639367	5.24909254983661e-05\\
4804.45070607688	5.15327198530782e-05\\
4841.1324900579	5.05914614658812e-05\\
4878.09433795431	4.96669631308294e-05\\
4915.33838804261	4.87590351423827e-05\\
4952.86679492495	4.78674854860523e-05\\
4990.68172965378	4.69921200252424e-05\\
5028.78537985746	4.61327426842235e-05\\
5067.1799498668	4.52891556271758e-05\\
5105.86766084255	4.44611594332537e-05\\
5144.85075090399	4.36485532676237e-05\\
5184.13147525829	4.28511350484379e-05\\
5223.71210633108	4.20687016097087e-05\\
5263.59493389784	4.13010488600585e-05\\
5303.78226521641	4.05479719373219e-05\\
5344.27642516044	3.98092653589856e-05\\
5385.07975635387	3.90847231684542e-05\\
5426.19461930653	3.83741390771375e-05\\
5467.6233925506	3.76773066023588e-05\\
5509.36847277828	3.69940192010889e-05\\
5551.43227498041	3.63240703995156e-05\\
5593.81723258619	3.566725391846e-05\\
5636.52579760393	3.50233637946626e-05\\
5679.56044076296	3.43921944979555e-05\\
5722.92365165649	3.37735410443523e-05\\
5766.61793888571	3.31671991050815e-05\\
5810.64583020485	3.25729651116e-05\\
5855.00987266743	3.19906363566218e-05\\
5899.71263277365	3.14200110912017e-05\\
5944.75669661882	3.086088861792e-05\\
5990.14467004298	3.03130693802108e-05\\
6035.87917878165	2.97763550478851e-05\\
6081.96286861774	2.9250548598899e-05\\
6128.39840553459	2.87354543974214e-05\\
6175.18847587023	2.8230878268255e-05\\
6222.33578647276	2.77366275676731e-05\\
6269.84306485697	2.7252511250726e-05\\
6317.71305936208	2.67783399350843e-05\\
6365.9485393108	2.63139259614776e-05\\
6414.55229516949	2.58590834507971e-05\\
6463.52713870962	2.54136283579245e-05\\
6512.8759031704	2.49773785223573e-05\\
6562.60144342272	2.45501537156958e-05\\
6612.70663613428	2.41317756860636e-05\\
6663.19437993603	2.37220681995288e-05\\
6714.06759558984	2.33208570785972e-05\\
6765.32922615749	2.2927970237849e-05\\
6816.9822371709	2.25432377167899e-05\\
6869.02961680372	2.21664917099871e-05\\
6921.47437604418	2.17975665945632e-05\\
6974.31954886926	2.143629895512e-05\\
7027.56819242026	2.10825276061624e-05\\
7081.22338717961	2.07360936120957e-05\\
7135.2882371491	2.0396840304867e-05\\
7189.76587002947	2.0064613299322e-05\\
7244.65943740126	1.97392605063478e-05\\
7299.97211490729	1.9420632143872e-05\\
7355.7071024362	1.91085807457887e-05\\
7411.86762430771	1.88029611688788e-05\\
7468.45692945906	1.8503630597795e-05\\
7525.47829163299	1.82104485481777e-05\\
7582.93500956713	1.79232768679694e-05\\
7640.83040718485	1.76419797369935e-05\\
7699.16783378753	1.7366423664863e-05\\
7757.95066424834	1.70964774872814e-05\\
7817.18229920744	1.68320123608021e-05\\
7876.86616526879	1.65729017561047e-05\\
7937.00571519826	1.63190214498522e-05\\
7997.6044281235	1.60702495151862e-05\\
8058.66580973513	1.58264663109214e-05\\
8120.19339248961	1.55875544694937e-05\\
8182.19073581352	1.53533988837211e-05\\
8244.66142630952	1.51238866924296e-05\\
8307.60907796385	1.48989072649996e-05\\
8371.03733235538	1.46783521848842e-05\\
8434.9498588663	1.44621152321501e-05\\
8499.35035489435	1.4250092365092e-05\\
8564.24254606679	1.40421817009672e-05\\
8629.63018645584	1.38382834958993e-05\\
8695.51705879595	1.36383001239944e-05\\
8761.90697470259	1.34421360557166e-05\\
8828.80377489274	1.32496978355644e-05\\
8896.2113294071	1.30608940590905e-05\\
8964.13353783398	1.28756353493042e-05\\
9032.57432953488	1.26938343324982e-05\\
9101.53766387181	1.25154056135348e-05\\
9171.02753043637	1.23402657506313e-05\\
9241.0479492805	1.21683332296773e-05\\
9311.60297114908	1.19995284381212e-05\\
9382.69667771427	1.18337736384566e-05\\
9454.33318181161	1.16709929413429e-05\\
9526.51662767799	1.15111122783905e-05\\
9599.25119119138	1.13540593746405e-05\\
9672.54108011239	1.11997637207687e-05\\
9746.39053432772	1.10481565450422e-05\\
9820.80382609543	1.08991707850553e-05\\
9895.78526029209	1.07527410592724e-05\\
9971.33917466183	1.06088036384028e-05\\
10047.4699400673	1.04672964166328e-05\\
10124.1819607425	1.03281588827405e-05\\
10201.4796745474	1.01913320911144e-05\\
10279.3675532253	1.00567586327025e-05\\
10357.8501026605	9.92438260591162e-06\\
10436.9318631401	9.79414958748108e-06\\
10516.6174096155	9.66600660335104e-06\\
10596.9113519683	9.53990209954758e-06\\
10677.818335276	9.41578591310581e-06\\
10759.343040081	9.29360924305047e-06\\
10841.4901826617	9.17332462145578e-06\\
10924.2645153049	9.05488588460373e-06\\
11007.670826581	8.93824814426089e-06\\
11091.713941621	8.82336775909419e-06\\
11176.3987223955	8.71020230624406e-06\\
11261.730067996	8.59871055307549e-06\\
11347.7129149183	8.48885242912567e-06\\
11434.3522373485	8.38058899826764e-06\\
11521.65304745	8.27388243110921e-06\\
11609.6203956541	8.16869597764543e-06\\
11698.2593709518	8.06499394018484e-06\\
11787.5751011885	7.96274164656652e-06\\
11877.5727533603	7.86190542368797e-06\\
11968.2575339131	7.76245257136153e-06\\
12059.6346890441	7.6643513365174e-06\\
12151.7095050046	7.56757088777211e-06\\
12244.4873084065	7.4720812903789e-06\\
12337.9734665302	7.37785348157849e-06\\
12432.1733876346	7.28485924636672e-06\\
12527.0925212711	7.1930711936954e-06\\
12622.7363585976	7.10246273312345e-06\\
12719.1104326973	7.01300805193242e-06\\
12816.2203188977	6.92468209272274e-06\\
12914.0716350943	6.83746053150398e-06\\
13012.6700420745	6.75131975629304e-06\\
13112.0212438457	6.66623684623295e-06\\
13212.1309879655	6.58218955124348e-06\\
13313.0050658734	6.4991562722154e-06\\
13414.6493132266	6.41711604175682e-06\\
13517.0696102371	6.3360485055012e-06\\
13620.2718820123	6.25593390398392e-06\\
13724.2620988974	6.17675305509352e-06\\
13829.046276821	6.09848733710268e-06\\
13934.630477643	6.02111867228188e-06\\
14041.0208095055	5.94462951109799e-06\\
14148.2234271857	5.86900281699815e-06\\
14256.2445324528	5.79422205177734e-06\\
14365.0903744256	5.72027116152783e-06\\
14474.7672499352	5.64713456316501e-06\\
14585.2815038886	5.5747971315245e-06\\
14696.6395296357	5.50324418702241e-06\\
14808.8477693396	5.43246148386951e-06\\
14921.9127143489	5.36243519882847e-06\\
15035.8409055736	5.29315192050117e-06\\
15150.638933863	5.22459863913258e-06\\
15266.3134403874	5.15676273691449e-06\\
15382.8711170223	5.08963197877249e-06\\
15500.318706735	5.02319450361723e-06\\
15618.6630039756	4.95743881603935e-06\\
15737.910855069	4.89235377842749e-06\\
15858.069158612	4.82792860348565e-06\\
15979.1448658717	4.76415284712674e-06\\
16101.1449811876	4.70101640171738e-06\\
16224.0765623776	4.63850948964767e-06\\
16347.9467211451	4.57662265720046e-06\\
16472.7626234916	4.51534676869166e-06\\
16598.5314901305	4.45467300085507e-06\\
16725.260596905	4.39459283744307e-06\\
16852.9572752091	4.33509806401519e-06\\
16981.6289124118	4.2761807628863e-06\\
17111.2829522843	4.21783330820622e-06\\
17241.9268954305	4.1600483611429e-06\\
17373.5682997215	4.10281886514182e-06\\
17506.2147807321	4.04613804123445e-06\\
17639.8740121818	3.98999938337011e-06\\
17774.5537263786	3.93439665374525e-06\\
17910.2617146665	3.87932387810685e-06\\
18047.005827876	3.82477534100595e-06\\
18184.7939767782	3.77074558098025e-06\\
18323.6341325429	3.71722938564519e-06\\
18463.5343271993	3.66422178667448e-06\\
18604.5026541008	3.61171805465368e-06\\
18746.5472683933	3.55971369379058e-06\\
18889.6763874869	3.50820443646944e-06\\
19033.8982915311	3.4571862376374e-06\\
19179.2213238944	3.40665526901294e-06\\
19325.6538916461	3.35660791310956e-06\\
19473.2044660434	3.30704075706809e-06\\
19621.8815830212	3.25795058629497e-06\\
19771.6938436859	3.20933437790444e-06\\
19922.649914813	3.16118929396548e-06\\
20074.7585293483	3.11351267455615e-06\\
20228.0284869136	3.06630203062961e-06\\
20382.4686543151	3.01955503669902e-06\\
20538.087966057	2.97326952334893e-06\\
20694.8954248579	2.92744346958421e-06\\
20852.9001021718	2.8820749950279e-06\\
21012.1111387131	2.83716235198178e-06\\
21172.5377449848	2.79270391736487e-06\\
21334.1892018119	2.748698184546e-06\\
21497.0748608783	2.70514375508846e-06\\
21661.2041452672	2.66203933042544e-06\\
21826.5865500072	2.61938370348598e-06\\
21993.2316426205	2.57717575029247e-06\\
22161.1490636774	2.53541442155043e-06\\
22330.3485273531	2.49409873425295e-06\\
22500.8398219907	2.45322776332176e-06\\
22672.6328106664	2.41280063330742e-06\\
22845.7374317606	2.37281651017116e-06\\
23020.1636995334	2.33327459317062e-06\\
23195.9217047026	2.29417410687202e-06\\
23373.0216150288	2.25551429331001e-06\\
23551.4736759026	2.2172944043169e-06\\
23731.2882109381	2.17951369404163e-06\\
23912.4756225696	2.14217141167814e-06\\
24095.0463926535	2.1052667944225e-06\\
24279.0110830748	2.06879906067591e-06\\
24464.3803363582	2.03276740351106e-06\\
24651.1648762833	1.99717098441681e-06\\
24839.3755085058	1.96200892733566e-06\\
25029.0231211817	1.92728031300693e-06\\
25220.1186855978	1.89298417362681e-06\\
25412.6732568062	1.85911948783556e-06\\
25606.697974264	1.82568517604012e-06\\
25802.2040624775	1.79268009607915e-06\\
25999.2028316513	1.76010303923583e-06\\
26197.7056783437	1.72795272660219e-06\\
26397.7240861246	1.6962278057976e-06\\
26599.2696262408	1.66492684804169e-06\\
26802.353958285	1.63404834558133e-06\\
27006.9888308703	1.60359070946955e-06\\
27213.1860823103	1.57355226769235e-06\\
27420.9576413033	1.54393126363921e-06\\
27630.3155276229	1.51472585491024e-06\\
27841.2718528132	1.48593411245321e-06\\
28053.8388208892	1.45755402002148e-06\\
28268.0287290435	1.42958347394316e-06\\
28483.8539683568	1.40202028319112e-06\\
28701.3270245155	1.37486216974197e-06\\
28920.4604785335	1.34810676921182e-06\\
29141.2670074804	1.32175163175578e-06\\
29363.7593852146	1.29579422321742e-06\\
29587.9504831223	1.27023192651409e-06\\
29813.8532708625	1.24506204324325e-06\\
30041.4808171167	1.22028179549517e-06\\
30270.8462903454	1.19588832785648e-06\\
30501.9629595499	1.1718787095894e-06\\
30734.8441950392	1.14824993697113e-06\\
30969.503469205	1.12499893577798e-06\\
31205.9543572992	1.10212256389909e-06\\
31444.2105382208	1.07961761406439e-06\\
31684.2857953065	1.05748081667199e-06\\
31926.194017128	1.03570884270062e-06\\
32169.949198296	1.0142983066926e-06\\
32415.5654402693	9.93245769793848e-07\\
32663.0569521706	9.72547742837493e-07\\
32912.4380516091	9.52200689458448e-07\\
33163.7231655079	9.32201029226706e-07\\
33416.9268309394	9.12545140787713e-07\\
33672.0636959658	8.93229364999058e-07\\
33929.1485204865	8.74250008052936e-07\\
34188.1961770926	8.55603344574955e-07\\
34449.2216519263	8.37285620690209e-07\\
34712.2400455487	8.19293057048413e-07\\
34977.2665738129	8.0162185180053e-07\\
35244.3165687446	7.84268183519919e-07\\
35513.4054794286	7.67228214061978e-07\\
35784.5488729031	7.50498091356518e-07\\
36057.76243506	7.34073952128204e-07\\
36333.0619715519	7.17951924540712e-07\\
36610.4634087077	7.02128130760965e-07\\
36889.9827944524	6.86598689440577e-07\\
37171.6362992366	6.71359718111845e-07\\
37455.4402169718	6.56407335496573e-07\\
37741.4109659722	6.41737663726252e-07\\
38029.5650899061	6.27346830472654e-07\\
38319.9192587512	6.13230970988534e-07\\
38612.4902697601	5.99386230058246e-07\\
38907.2950484319	5.85808763858754e-07\\
39204.350649491	5.72494741731721e-07\\
39503.674257874	5.5944034786772e-07\\
39805.2831897235	5.46641782903934e-07\\
40109.1948933907	5.34095265436883e-07\\
40415.4269504438	5.21797033452142e-07\\
40723.9970766856	5.09743345672932e-07\\
41034.9231231787	4.97930482829914e-07\\
41348.223077277	4.86354748854495e-07\\
41663.9150636682	4.75012471998089e-07\\
41982.0173454204	4.63900005880009e-07\\
42302.5483250399	4.53013730466477e-07\\
42625.5265455352	4.42350052983583e-07\\
42950.9706914898	4.3190540876684e-07\\
43278.8995901437	4.21676262050037e-07\\
43609.3322124815	4.11659106696218e-07\\
43942.2876743309	4.01850466873353e-07\\
44277.7852374682	3.92246897677458e-07\\
44615.8443107321	3.82844985705748e-07\\
44956.4844511476	3.73641349582356e-07\\
45299.7253650563	3.64632640439147e-07\\
45645.5869092572	3.55815542353974e-07\\
45994.0890921549	3.47186772748766e-07\\
46345.2520749175	3.38743082749626e-07\\
46699.096172643	3.30481257511112e-07\\
47055.6418555338	3.22398116506742e-07\\
47414.9097500821	3.14490513787626e-07\\
47776.920640262	3.06755338211158e-07\\
48141.6954687325	2.9918951364141e-07\\
48509.2553380493	2.91789999122945e-07\\
48879.6215118846	2.84553789029581e-07\\
49252.8154162586	2.77477913189515e-07\\
49628.8586407777	2.70559436988244e-07\\
50007.7729398843	2.63795461450451e-07\\
50389.5802341153	2.57183123302117e-07\\
50774.3026113693	2.50719595013908e-07\\
51161.962328186	2.44402084826834e-07\\
51552.5818110322	2.38227836761184e-07\\
51946.1836576	2.32194130609514e-07\\
52342.7906381142	2.26298281914548e-07\\
52742.4256966491	2.20537641932696e-07\\
53145.1119524562	2.14909597583869e-07\\
53550.8727013012	2.09411571388216e-07\\
53959.7314168126	2.04041021390322e-07\\
54371.7117518386	1.98795441071454e-07\\
54786.8375398161	1.93672359250248e-07\\
55205.1327961494	1.88669339972362e-07\\
55626.6217195991	1.83783982389453e-07\\
56051.3286936829	1.7901392062786e-07\\
56479.2782880851	1.74356823647384e-07\\
56910.4952600788	1.69810395090429e-07\\
57345.0045559579	1.65372373121864e-07\\
57782.83131248	1.6104053025987e-07\\
58224.0008583212	1.5681267319803e-07\\
58668.5387155406	1.5268664261897e-07\\
59116.4706010574	1.48660312999733e-07\\
59567.8224281384	1.44731592409179e-07\\
60022.620307897	1.40898422297621e-07\\
60480.8905508042	1.3715877727892e-07\\
60942.6596682098	1.33510664905271e-07\\
61407.9543738774	1.29952125434887e-07\\
61876.8015855284	1.26481231592817e-07\\
62349.2284264003	1.23096088325089e-07\\
62825.2622268155	1.19794832546415e-07\\
63304.9305257616	1.16575632881666e-07\\
63788.2610724863	1.1343668940132e-07\\
64275.2818281007	1.10376233351136e-07\\
64766.0209671984	1.07392526876225e-07\\
65260.5068794848	1.04483862739785e-07\\
65758.7681714191	1.01648564036688e-07\\
66260.8336678705	9.8884983902162e-08\\
66766.7324137839	9.61915052157886e-08\\
67276.4936758618	9.3566540301036e-08\\
67790.1469442567	9.10085306205684e-08\\
68307.721934277	8.85159464675502e-08\\
68829.2485881065	8.60872866531805e-08\\
69354.7570765359	8.37210781906903e-08\\
69884.2778007095	8.14158759760203e-08\\
70417.8413938822	7.91702624654361e-08\\
70955.4787231929	7.69828473502797e-08\\
71497.22089145	7.48522672291115e-08\\
72043.0992389301	7.27771852774632e-08\\
72593.1453451923	7.07562909154187e-08\\
73147.3910309034	6.87882994732728e-08\\
73705.8683596805	6.68719518554623e-08\\
74268.6096399447	6.50060142030127e-08\\
74835.6474267902	6.318927755471e-08\\
75407.0145238692	6.14205575072074e-08\\
75982.743985287	5.96986938742965e-08\\
76562.8691175166	5.8022550345528e-08\\
77147.4234813242	5.63910141444074e-08\\
77736.4408937111	5.48029956863541e-08\\
78329.9554298704	5.32574282366247e-08\\
78928.001425157	5.1753267568394e-08\\
79530.6134770757	5.02894916211752e-08\\
80137.8264472814	4.88651001597697e-08\\
80749.6754635966	4.74791144339191e-08\\
81366.195922043	4.61305768388334e-08\\
81987.4234888893	4.48185505767658e-08\\
82613.3941027154	4.35421193197876e-08\\
83244.1439764901	4.23003868739372e-08\\
83879.709599667	4.10924768448741e-08\\
84520.1277402954	3.99175323052014e-08\\
85165.4354471463	3.87747154635903e-08\\
85815.670051858	3.76632073358384e-08\\
86470.8691710928	3.65822074180046e-08\\
87131.0707087156	3.55309333617291e-08\\
87796.3128579857	3.45086206518698e-08\\
88466.6341037656	3.35145222865661e-08\\
89142.0732247495	3.25479084598314e-08\\
89822.669295704	3.16080662467912e-08\\
90508.4616897307	3.069429929165e-08\\
91199.4900805429	2.98059274984897e-08\\
91895.7944447613	2.89422867249835e-08\\
92597.4150642266	2.81027284791074e-08\\
93304.3925283287	2.72866196189287e-08\\
94016.7677363574	2.64933420555399e-08\\
94734.5818998656	2.57222924592139e-08\\
95457.8765450552	2.49728819688339e-08\\
96186.6935151793	2.42445359046632e-08\\
96921.0749729615	2.35366934845085e-08\\
97661.0634030374	2.28488075433208e-08\\
98406.70161441	2.21803442562881e-08\\
99158.0327429277	2.15307828654523e-08\\
99915.1002537795	2.08996154098931e-08\\
100677.947944009	2.02863464595106e-08\\
101446.619945048	1.96904928524328e-08\\
102221.160725271	1.911158343608e-08\\
103001.615092567	1.85491588119003e-08\\
103788.028196929	1.80027710838009e-08\\
104580.445533071	1.74719836102858e-08\\
105378.912943057	1.69563707603141e-08\\
106183.47661895	1.6455517672887e-08\\
106994.183105493	1.5969020020366e-08\\
107811.079302791	1.54964837755304e-08\\
108634.212469033	1.50375249823675e-08\\
109463.63022322	1.45917695305963e-08\\
110299.380547924	1.41588529339179e-08\\
111141.51179206	1.37384201119826e-08\\
111990.072673687	1.33301251760666e-08\\
112845.112282822	1.29336312184405e-08\\
113706.680084285	1.25486101054179e-08\\
114574.825920556	1.21747422740654e-08\\
115449.600014662	1.18117165325535e-08\\
116331.052973077	1.14592298641296e-08\\
117219.235788658	1.1116987234686e-08\\
118114.199843588	1.07847014039022e-08\\
119015.99691235	1.04620927399314e-08\\
119924.679164726	1.01488890376045e-08\\
120840.299168808	9.84482534012292e-09\\
121762.909894048	9.54964376420588e-09\\
122692.564714314	9.26309332866438e-09\\
123629.317410982	8.98492978636378e-09\\
124573.222176048	8.71491545954354e-09\\
125524.33361526	8.45281907845696e-09\\
126482.706751281	8.19841562329334e-09\\
127448.397026865	7.95148616934749e-09\\
128421.460308075	7.71181773539464e-09\\
129401.952887506	7.47920313523429e-09\\
130389.931487544	7.25344083236167e-09\\
131385.45326365	7.03433479772546e-09\\
132388.575807663	6.821694370533e-09\\
133399.357151135	6.61533412205791e-09\\
134417.855768684	6.41507372241071e-09\\
135444.130581383	6.22073781022821e-09\\
136478.240960161	6.03215586523924e-09\\
137520.246729245	5.8491620836636e-09\\
138570.208169615	5.67159525639995e-09\\
139628.186022495	5.49929864996e-09\\
140694.241492865	5.33211989010423e-09\\
141768.436253005	5.16991084813573e-09\\
142850.832446056	5.01252752980798e-09\\
143941.492689623	4.85982996680191e-09\\
145040.480079392	4.71168211072986e-09\\
146147.858192783	4.56795172962027e-09\\
147263.691092627	4.42851030684108e-09\\
148388.04333087	4.2932329424173e-09\\
149520.979952313	4.16199825669877e-09\\
150662.566498367	4.03468829633612e-09\\
151812.869010854	3.91118844252e-09\\
152971.954035819	3.79138732144207e-09\\
154139.888627384	3.67517671693468e-09\\
155316.740351627	3.56245148524635e-09\\
156502.577290491	3.45310947191235e-09\\
157697.468045719	3.34705143067723e-09\\
158901.481742829	3.24418094442932e-09\\
160114.688035108	3.14440434810544e-09\\
161337.157107642	3.04763065352587e-09\\
162568.959681378	2.95377147611936e-09\\
163810.167017216	2.86274096349827e-09\\
165060.850920127	2.77445572584552e-09\\
166321.083743312	2.68883476807362e-09\\
167590.938392387	2.60579942371833e-09\\
168870.488329596	2.52527329052912e-09\\
170159.807578068	2.44718216771872e-09\\
171458.970726091	2.37145399483594e-09\\
172768.052931436	2.29801879222469e-09\\
174087.129925698	2.2268086030341e-09\\
175416.27801868	2.15775743674455e-09\\
176755.574102809	2.09080121417464e-09\\
178105.095657579	2.02587771393582e-09\\
179464.920754039	1.96292652030017e-09\\
180835.128059308	1.90188897244936e-09\\
182215.796841124	1.84270811507172e-09\\
183607.00697243	1.78532865027585e-09\\
185008.838935997	1.72969689078955e-09\\
186421.373829081	1.67576071441291e-09\\
187844.693368106	1.62346951969609e-09\\
189278.879893403	1.57277418281133e-09\\
190724.016373967	1.52362701559079e-09\\
192180.186412254	1.47598172470129e-09\\
193647.474249029	1.429793371928e-09\\
195125.964768223	1.38501833553995e-09\\
196615.743501859	1.34161427270994e-09\\
198116.896634991	1.29954008296289e-09\\
199629.511010691	1.2587558726267e-09\\
201153.674135077	1.21922292026013e-09\\
202689.47418237	1.18090364303326e-09\\
204237	1.14376156403585e-09\\
};
\addlegendentry{Adatt. BLN esatto (\textit{ML})}


\addplot[area legend, draw=none, fill=mycolor1, fill opacity=0.15, forget plot]
table[row sep=crcr] {%
x	y\\
102.379054928703	3.36896786327755e-06\\
103.160714811762	3.4188368377037e-06\\
103.948342635952	3.46939587295543e-06\\
104.741983966257	3.52065604562075e-06\\
105.54168471555	3.57262870804131e-06\\
106.347491147248	3.62532549714757e-06\\
107.159449877983	3.67875834355657e-06\\
107.977607880307	3.73293948093844e-06\\
108.802012485405	3.78788145565764e-06\\
109.632711385831	3.84359713669527e-06\\
110.469752638272	3.90009972585883e-06\\
111.313184666327	3.95740276828559e-06\\
112.163056263304	4.01552016324625e-06\\
113.019416595047	4.07446617525508e-06\\
113.882315202781	4.1342554454933e-06\\
114.751802005974	4.19490300355219e-06\\
115.627927305229	4.25642427950246e-06\\
116.51074178519	4.31883511629656e-06\\
117.400296517478	4.38215178251064e-06\\
118.296642963642	4.44639098543273e-06\\
119.199832978139	4.51156988450379e-06\\
120.109918811333	4.57770610511846e-06\\
121.026953112515	4.64481775279182e-06\\
121.950988932956	4.71292342769924e-06\\
122.882079728965	4.78204223959536e-06\\
123.820279364994	4.85219382311911e-06\\
124.765642116742	4.92339835349109e-06\\
125.718222674305	4.99567656260953e-06\\
126.678076145334	5.06904975555125e-06\\
127.645258058225	5.14353982748371e-06\\
128.619824365331	5.21916928099404e-06\\
129.601831446199	5.29596124384086e-06\\
130.59133611083	5.37393948713474e-06\\
131.58839560297	5.45312844395234e-06\\
132.593067603415	5.53355322838966e-06\\
133.605410233356	5.61523965505919e-06\\
134.625482057733	5.69821425903558e-06\\
135.653342088628	5.78250431625398e-06\\
136.689049788679	5.86813786436513e-06\\
137.732665074517	5.95514372405034e-06\\
138.784248320236	6.04355152079977e-06\\
139.843860360883	6.13339170715618e-06\\
140.911562495976	6.22469558542625e-06\\
141.987416493056	6.31749533086101e-06\\
143.071484591254	6.41182401530584e-06\\
144.163829504894	6.50771563132024e-06\\
145.264514427124	6.60520511676678e-06\\
146.373603033566	6.70432837986745e-06\\
147.491159486005	6.80512232472513e-06\\
148.617248436097	6.90762487730708e-06\\
149.751935029113	7.01187501188551e-06\\
150.895284907701	7.1179127779304e-06\\
152.047364215693	7.22577932744757e-06\\
153.208239601924	7.33551694275421e-06\\
154.37797822409	7.44716906468279e-06\\
155.556647752633	7.56078032120256e-06\\
156.744316374658	7.67639655644659e-06\\
157.941052797874	7.79406486013082e-06\\
159.146926254571	7.91383359734927e-06\\
160.362006505625	8.03575243872867e-06\\
161.586363844532	8.15987239092323e-06\\
162.820069101478	8.28624582742842e-06\\
164.063193647433	8.41492651969071e-06\\
165.315809398282	8.54596966848809e-06\\
166.577988818986	8.67943193555347e-06\\
167.84980492777	8.81537147541118e-06\\
169.131331300355	8.95384796739426e-06\\
170.422642074203	9.09492264780731e-06\\
171.723811952819	9.23865834219743e-06\\
173.034916210062	9.38511949769293e-06\\
174.356030694505	9.5343722153666e-06\\
175.687231833822	9.6864842825777e-06\\
177.02859663921	9.8415252052437e-06\\
178.380202709841	9.99956623998987e-06\\
179.742128237356	1.01606804261223e-05\\
181.114452010385	1.03249426173662e-05\\
182.497253419104	1.04924295133086e-05\\
183.890612459833	1.06632196904825e-05\\
185.294609739659	1.08373936330252e-05\\
186.7093264811	1.10150337628409e-05\\
188.134844526807	1.11962244691971e-05\\
189.571246344294	1.13810521376775e-05\\
191.018615030712	1.1569605178416e-05\\
192.477034317655	1.17619740535314e-05\\
193.946588576005	1.19582513036801e-05\\
195.427362820812	1.21585315736447e-05\\
196.919442716211	1.23629116368714e-05\\
198.422914580381	1.25714904188706e-05\\
199.937865390536	1.27843690193921e-05\\
201.464382787958	1.30016507332864e-05\\
203.002555083066	1.32234410699639e-05\\
204.552471260524	1.34498477713626e-05\\
206.114220984395	1.36809808283356e-05\\
207.687894603317	1.3916952495372e-05\\
209.273583155741	1.41578773035648e-05\\
210.871378375191	1.4403872071742e-05\\
212.481372695573	1.46550559156784e-05\\
214.103659256522	1.49115502553112e-05\\
215.73833190879	1.51734788198817e-05\\
217.385485219676	1.54409676509341e-05\\
219.045214478497	1.5714145103102e-05\\
220.717615702098	1.5993141842622e-05\\
222.402785640412	1.6278090843516e-05\\
224.100821782051	1.65691273813916e-05\\
225.811822359948	1.68663890248141e-05\\
227.535886357044	1.71700156242111e-05\\
229.273113512007	1.74801492982773e-05\\
231.023604325008	1.77969344178508e-05\\
232.787460063531	1.81205175872452e-05\\
234.564782768236	1.84510476230198e-05\\
236.355675258854	1.87886755301854e-05\\
238.160241140146	1.91335544758441e-05\\
239.978584807887	1.94858397602699e-05\\
241.810811454912	1.98456887854432e-05\\
243.657027077199	2.02132610210572e-05\\
245.517338479998	2.0588717968019e-05\\
247.391853284017	2.09722231194759e-05\\
249.28067993164	2.13639419193966e-05\\
251.183927693206	2.17640417187447e-05\\
253.101706673328	2.21726917292839e-05\\
255.034127817263	2.25900629750546e-05\\
256.981302917332	2.30163282415641e-05\\
258.943344619384	2.34516620227352e-05\\
260.920366429315	2.3896240465654e-05\\
262.912482719633	2.435024131316e-05\\
264.919808736078	2.48138438443186e-05\\
266.942460604284	2.5287228812814e-05\\
268.980555336501	2.57705783832976e-05\\
271.03421083836	2.62640760657236e-05\\
273.103545915701	2.67679066476987e-05\\
275.188680281438	2.72822561248692e-05\\
277.28973456249	2.78073116293631e-05\\
279.406830306757	2.83432613562986e-05\\
281.540089990152	2.88902944883685e-05\\
283.689637023688	2.94486011184976e-05\\
285.855595760614	3.00183721705725e-05\\
288.038091503612	3.05997993182298e-05\\
290.237250512045	3.11930749016895e-05\\
292.453200009261	3.17983918426088e-05\\
294.686068189952	3.24159435569325e-05\\
296.935984227572	3.30459238657055e-05\\
299.203078281811	3.36885269038124e-05\\
301.487481506119	3.43439470266026e-05\\
303.789326055299	3.50123787143577e-05\\
306.108745093152	3.56940164745534e-05\\
308.445872800177	3.63890547418652e-05\\
310.800844381337	3.70976877758671e-05\\
313.173796073877	3.78201095563697e-05\\
315.564865155213	3.8556513676343e-05\\
317.974189950861	3.93070932323707e-05\\
320.401909842454	4.00720407125808e-05\\
322.848165275792	4.08515478820009e-05\\
325.313097768977	4.1645805665286e-05\\
327.796849920595	4.245500402677e-05\\
330.299565417964	4.32793318477941e-05\\
332.821389045452	4.41189768012685e-05\\
335.362466692849	4.49741252234292e-05\\
337.922945363805	4.58449619827503e-05\\
340.502973184339	4.67316703459856e-05\\
343.102699411407	4.76344318413088e-05\\
345.722274441534	4.85534261185322e-05\\
348.361849819517	4.94888308063901e-05\\
351.02157824719	5.04408213668732e-05\\
353.701613592262	5.14095709466128e-05\\
356.402110897212	5.2395250225316e-05\\
359.123226388267	5.33980272612599e-05\\
361.865117484429	5.44180673338603e-05\\
364.627942806591	5.54555327833386e-05\\
367.41186218671	5.65105828475114e-05\\
370.217036677053	5.75833734957435e-05\\
373.043628559514	5.86740572601029e-05\\
375.891801355004	5.97827830637705e-05\\
378.761719832906	6.09096960467593e-05\\
381.653550020615	6.20549373890088e-05\\
384.567459213134	6.3218644130925e-05\\
387.503615982761	6.44009489914452e-05\\
390.462190188833	6.56019801837128e-05\\
393.443352987558	6.68218612284542e-05\\
396.447276841914	6.80607107651609e-05\\
399.474135531625	6.93186423611801e-05\\
402.524104163219	7.05957643188309e-05\\
405.597359180154	7.1892179480667e-05\\
408.694078373027	7.32079850330131e-05\\
411.814440889858	7.45432723079113e-05\\
414.958627246454	7.58981265836207e-05\\
418.126819336855	7.72726268838173e-05\\
421.319200443854	7.8666845775652e-05\\
424.535955249599	8.00808491668277e-05\\
427.77726984628	8.15146961018644e-05\\
431.043331746893	8.29684385577316e-05\\
434.334329896089	8.44421212390241e-05\\
437.6504546811	8.59357813728732e-05\\
440.991897942762	8.74494485037892e-05\\
444.358852986605	8.89831442886306e-05\\
447.751514594037	9.05368822919102e-05\\
451.170079033618	9.21106677816503e-05\\
454.614744072408	9.37044975260019e-05\\
458.08570898741	9.53183595908539e-05\\
461.5831745771	9.69522331386588e-05\\
465.107343173043	9.86060882287092e-05\\
468.658418651595	0.000100279885619102\\
472.236606445701	0.000101973576570639\\
475.842113556776	0.000103687102652898\\
479.475148566685	0.000105420395552746\\
483.135921649807	0.000107173376885528\\
486.824644585191	0.00010894595800921\\
490.541530768814	0.000110738039841729\\
494.286795225918	0.00011254951268181\\
498.060654623459	0.000114380256033541\\
501.863327282632	0.000116230138434951\\
505.695033191508	0.000118099017290879\\
509.555994017757	0.000119986738710414\\
513.446433121472	0.000121893137349171\\
517.366575568092	0.000123818036256687\\
521.316648141422	0.000125761246729223\\
525.296879356751	0.00012772256816824\\
529.307499474071	0.000129701787944837\\
533.348740511404	0.000131698681270432\\
537.420836258218	0.000133713011073963\\
541.524022288951	0.000135744527885878\\
545.658535976647	0.000137792969729216\\
549.824616506682	0.000139858062018021\\
554.0225048906	0.000141939517463392\\
558.25244398006	0.000144037035987418\\
562.514678480885	0.000146150304645275\\
566.809454967215	0.000148278997555746\\
571.137021895773	0.00015042277584042\\
575.497629620241	0.00015258128757183\\
579.891530405739	0.000154754167730773\\
584.318978443423	0.000156941038173067\\
588.780229865185	0.000159141507605963\\
593.275542758476	0.000161355171574468\\
597.805177181235	0.000163581612457788\\
602.369395176928	0.000165820399476108\\
606.968460789717	0.000168071088707933\\
611.602640079728	0.000170333223118172\\
616.272201138446	0.00017260633259718\\
620.977414104223	0.000174889934010913\\
625.718551177906	0.000177183531262409\\
630.495886638585	0.000179486615364718\\
635.30969685946	0.000181798664525466\\
640.160260323826	0.000184119144243187\\
645.04785764119	0.000186447507415542\\
649.972771563501	0.000188783194459576\\
654.935287001506	0.000191125633444083\\
659.935691041233	0.000193474240234213\\
664.974272960603	0.000195828418648378\\
670.051324246158	0.000198187560627543\\
675.167138609933	0.000200551046416956\\
680.322012006438	0.000202918244760359\\
685.516242649784	0.000205288513106714\\
690.750131030938	0.00020766119782946\\
696.023979935099	0.000210035634458302\\
701.33809445922	0.000212411147923528\\
706.692782029658	0.000214787052812809\\
712.088352419958	0.000217162653640467\\
717.525117768769	0.000219537245129129\\
723.00339259791	0.000221910112503714\\
728.52349383056	0.000224280531797656\\
734.085740809593	0.000226647770171265\\
739.690455316055	0.000229011086242112\\
745.337961587772	0.00023136973042729\\
751.028586338118	0.000233722945297432\\
756.762658774907	0.000236069965942283\\
762.540510619441	0.000238410020347687\\
768.362476125704	0.000240742329783758\\
774.228892099692	0.000243066109204058\\
780.140097918901	0.000245380567655525\\
786.096435551963	0.00024768490869894\\
792.098249578423	0.000249978330839651\\
798.145887208678	0.000252260027968304\\
804.239698304065	0.000254529189811283\\
810.380035397094	0.000256785002390564\\
816.567253711847	0.000259026648492667\\
822.801711184529	0.000261253308146381\\
829.083768484172	0.000263464159108912\\
835.413789033501	0.000265658377360108\\
841.792139029961	0.000267835137604381\\
848.219187466894	0.000269993613779951\\
854.695306154899	0.000272132979575017\\
861.220869743325	0.000274252408950435\\
867.79625574196	0.000276351076668511\\
874.421844542861	0.000278428158827436\\
881.098019442364	0.000280482833400967\\
887.825166663254	0.00028251428078286\\
894.603675377116	0.000284521684335619\\
901.433937726837	0.000286504230943073\\
908.316348849305	0.000288461111566303\\
915.251306898262	0.000290391521802424\\
922.239213067332	0.000292294662445725\\
929.280471613241	0.000294169740050654\\
936.375489879197	0.000296015967496136\\
943.524678318455	0.000297832564550696\\
950.728450518066	0.000299618758437866\\
957.987223222801	0.000301373784401329\\
965.301416359258	0.000303096886269276\\
972.671453060162	0.000304787317017418\\
980.097759688832	0.000306444339330106\\
987.580765863859	0.000308067226159026\\
995.120904483953	0.000309655261278886\\
1002.71861175299	0.000311207739839565\\
1010.37432720523	0.00031272396891416\\
1018.08849373079	0.000314203268042369\\
1025.86155760121	0.000315644969768665\\
1033.69396849529	0.000317048420174697\\
1041.58617952513	0.000318412979405379\\
1049.53864726231	0.000319738022188108\\
1057.55183176433	0.000321022938344567\\
1065.62619660117	0.000322267133294582\\
1073.7622088822	0.0003234700285515\\
1081.96033928312	0.000324631062208543\\
1090.22106207322	0.000325749689415647\\
1098.54485514283	0.000326825382846257\\
1106.93220003095	0.000327857633153579\\
1115.38358195309	0.000328845949415815\\
1123.89948982939	0.000329789859569887\\
1132.48041631286	0.000330688910833194\\
1141.1268578179	0.000331542670112956\\
1149.839314549	0.000332350724402703\\
1158.61829052972	0.000333112681165502\\
1167.46429363178	0.000333828168703509\\
1176.37783560452	0.000334496836513485\\
1185.35943210443	0.0003351183556279\\
1194.40960272504	0.000335692418941294\\
1203.52887102695	0.000336218741521588\\
1212.71776456812	0.000336697060906046\\
1221.97681493439	0.000337127137381625\\
1231.30655777025	0.000337508754249497\\
1240.70753280979	0.000337841718073503\\
1250.18028390797	0.000338125858912408\\
1259.72535907206	0.000338361030535775\\
1269.34331049332	0.00033854711062338\\
1279.03469457899	0.000338684000948083\\
1288.80007198445	0.000338771627542122\\
1298.64000764567	0.000338809940846851\\
1308.55507081185	0.000338798915845952\\
1318.54583507842	0.000338738552182228\\
1328.61287842018	0.000338628874258108\\
1338.75678322474	0.000338469931320052\\
1348.9781363262	0.000338261797527098\\
1359.27752903913	0.00033800457200382\\
1369.65555719278	0.000337698378878059\\
1380.1128211655	0.000337343367303803\\
1390.64992591951	0.000336939711469672\\
1401.26748103591	0.000336487610593525\\
1411.96610074992	0.000335987288903731\\
1422.7464039864	0.00033543899560775\\
1433.6090143957	0.000334843004848682\\
1444.5545603897	0.000334199615650526\\
1455.5836751782	0.000333509151852936\\
1466.69699680551	0.000332771962036291\\
1477.89516818739	0.000331988419437967\\
1489.17883714823	0.000331158921860683\\
1500.54865645854	0.000330283891573857\\
1512.0052838727	0.000329363775208873\\
1523.54938216702	0.000328399043649139\\
1535.18161917809	0.00032739019191576\\
1546.90266784138	0.000326337739049572\\
1558.71320623023	0.000325242227990104\\
1570.613917595	0.000324104225451846\\
1582.60549040267	0.000322924321797933\\
1594.68861837664	0.000321703130910934\\
1606.86400053683	0.000320441290059986\\
1619.13234124017	0.000319139459762823\\
1631.49435022133	0.0003177983236405\\
1643.95074263376	0.000316418588261536\\
1656.50223909109	0.000315000982971007\\
1669.1495657088	0.000313546259698535\\
1681.89345414622	0.000312055192737315\\
1694.73464164889	0.000310528578484082\\
1707.67387109119	0.00030896723512736\\
1720.71189101928	0.000307372002268308\\
1733.84945569449	0.000305743740455102\\
1747.08732513687	0.000304083330608027\\
1760.42626516921	0.000302391673308372\\
1773.8670474613	0.000300669687920089\\
1787.41044957462	0.000298918311509051\\
1801.05725500729	0.000297138497521145\\
1814.80825323942	0.0002953312141777\\
1828.66423977876	0.000293497442545621\\
1842.62601620672	0.000291638174240709\\
1856.69439022475	0.000289754408726979\\
1870.87017570109	0.000287847150183229\\
1885.15419271781	0.000285917403921581\\
1899.54726761828	0.000283966172361856\\
1914.05023305497	0.000281994450590874\\
1928.66392803762	0.000280003221566556\\
1943.38919798175	0.000277993451061991\\
1958.22689475763	0.000275966082482004\\
1973.17787673949	0.000273922031720883\\
1988.24300885525	0.000271862182260434\\
2003.42316263648	0.000269787380727636\\
2018.71921626889	0.00026769843313621\\
2034.13205464308	0.0002655961020229\\
2049.66256940575	0.000263481104655569\\
2065.31165901132	0.000261354112437562\\
2081.08022877384	0.000259215751565054\\
2096.9691909194	0.000257066604918014\\
2112.9794646389	0.000254907215089018\\
2129.11197614124	0.000252738088385998\\
2145.36765870687	0.000250559699592334\\
2161.74745274181	0.0002483724972354\\
2178.25230583203	0.000246176909104599\\
2194.88317279829	0.000243973347770924\\
2211.64101575135	0.000241762215888331\\
2228.52680414767	0.000239543911097622\\
2245.54151484545	0.000237318830400139\\
2262.68613216118	0.000235087373916145\\
2279.96164792655	0.000232849947986964\\
2297.36906154586	0.000230606967617771\\
2314.9093800538	0.000228358858287858\\
2332.58361817375	0.000226106057176902\\
2350.39279837646	0.000223849013869598\\
2368.33795093917	0.000221588190608275\\
2386.42011400529	0.00021932406216489\\
2404.64033364438	0.000217057115401687\\
2422.99966391271	0.000214787848584895\\
2441.49916691423	0.000212516770509369\\
2460.13991286201	0.000210244399484753\\
2478.92298014014	0.000207971262226323\\
2497.84945536614	0.000205697892686509\\
2516.9204334538	0.000203424830856469\\
2536.13701767655	0.00020115262156127\\
2555.50031973125	0.00019888181326703\\
2575.01145980252	0.00019661295691411\\
2594.67156662757	0.000194346604786755\\
2614.48177756143	0.000192083309426642\\
2634.44323864282	0.000189823622595382\\
2654.55710466042	0.000187568094289114\\
2674.82453921964	0.000185317271806848\\
2695.24671481002	0.000183071698873058\\
2715.82481287296	0.000180831914814179\\
2736.56002387015	0.000178598453788022\\
2757.45354735239	0.000176371844064667\\
2778.50659202901	0.000174152607357107\\
2799.72037583779	0.00017194125819969\\
2821.09612601541	0.00016973830337233\\
2842.63507916843	0.000167544241368385\\
2864.33848134488	0.000165359561904111\\
2886.20758810631	0.000163184745467654\\
2908.24366460042	0.000161020262905566\\
2930.44798563427	0.000158866575044952\\
2952.82183574802	0.000156724132349385\\
2975.36650928923	0.000154593374606889\\
2998.08331048777	0.0001524747306483\\
3020.97355353122	0.000150368618094481\\
3044.03856264098	0.000148275443130914\\
3067.27967214876	0.000146195600308275\\
3090.6982265739	0.000144129472367722\\
3114.29558070104	0.000142077430089634\\
3138.07309965858	0.000140039832164691\\
3162.03215899761	0.000138017025086168\\
3186.17414477149	0.000136009343062453\\
3210.50045361603	0.000134017107948791\\
3235.01249283033	0.000132040629197358\\
3259.71168045813	0.000130080203824784\\
3284.59944536989	0.000128136116396299\\
3309.67722734543	0.000126208639025717\\
3334.94647715724	0.000124298031390505\\
3360.4086566544	0.000122404540761219\\
3386.06523884712	0.000120528402044624\\
3411.91770799201	0.000118669837839828\\
3437.96755967793	0.000116829058506802\\
3464.21630091246	0.000115006262246686\\
3490.66545020915	0.000113201635193262\\
3517.31653767532	0.000111415351515063\\
3544.17110510062	0.000109647573527534\\
3571.23070604618	0.000107898451814737\\
3598.49690593453	0.000106168125360069\\
3625.97128214009	0.000104456721685499\\
3653.65542408053	0.000102764356998832\\
3681.55093330861	0.000101091136348537\\
3709.65942360489	9.94371537856656e-05\\
3737.9825210711	9.78024925324279e-05\\
3766.52186422418	9.618722515698e-05\\
3795.27910409107	9.45914137540076e-05\\
3824.25590430422	9.30151101306873e-05\\
3853.45394119789	9.14583559976283e-05\\
3882.87490390504	8.99211831644036e-05\\
3912.52049445511	8.84036137392886e-05\\
3942.39242787247	8.69056603328388e-05\\
3972.49243227562	8.54273262649469e-05\\
4002.82224897718	8.39686057750274e-05\\
4033.38363258461	8.252948423499e-05\\
4064.17835110174	8.11099383646714e-05\\
4095.20818603102	7.9709936449403e-05\\
4126.47493247662	7.83294385594046e-05\\
4157.98039924825	7.69683967706979e-05\\
4189.72640896578	7.56267553872535e-05\\
4221.71479816475	7.43044511640795e-05\\
4253.94741740255	7.30014135309836e-05\\
4286.42613136551	7.1717564816738e-05\\
4319.15281897675	7.04528204733955e-05\\
4352.12937350489	6.92070893005032e-05\\
4385.3577026736	6.79802736689786e-05\\
4418.83972877192	6.67722697444183e-05\\
4452.57738876551	6.55829677096153e-05\\
4486.57263440865	6.44122519860748e-05\\
4520.82743235723	6.32600014543214e-05\\
4555.34376428243	6.21260896728054e-05\\
4590.12362698543	6.1010385095218e-05\\
4625.16903251291	5.99127512860374e-05\\
4660.48200827344	5.88330471341363e-05\\
4696.06459715477	5.77711270642866e-05\\
4731.918857642	5.67268412464063e-05\\
4768.0468639367	5.57000358024035e-05\\
4804.45070607688	5.46905530104784e-05\\
4841.1324900579	5.36982315067504e-05\\
4878.09433795431	5.27229064840891e-05\\
4915.33838804261	5.17644098880312e-05\\
4952.86679492495	5.08225706096758e-05\\
4990.68172965378	4.98972146754545e-05\\
5028.78537985746	4.89881654336851e-05\\
5067.1799498668	4.80952437378175e-05\\
5105.86766084255	4.7218268126295e-05\\
5144.85075090399	4.63570549989517e-05\\
5184.13147525829	4.55114187898844e-05\\
5223.71210633108	4.46811721367321e-05\\
5263.59493389784	4.38661260463122e-05\\
5303.78226521641	4.30660900565615e-05\\
5344.27642516044	4.22808723947412e-05\\
5385.07975635387	4.1510280131866e-05\\
5426.19461930653	4.07541193333276e-05\\
5467.6233925506	4.0012195205685e-05\\
5509.36847277828	3.92843122396016e-05\\
5551.43227498041	3.85702743489121e-05\\
5593.81723258619	3.78698850058077e-05\\
5636.52579760393	3.71829473721359e-05\\
5679.56044076296	3.65092644268103e-05\\
5722.92365165649	3.58486390893353e-05\\
5766.61793888571	3.52008743394524e-05\\
5810.64583020485	3.45657733329201e-05\\
5855.00987266743	3.39431395134415e-05\\
5899.71263277365	3.33327767207604e-05\\
5944.75669661882	3.27344892949494e-05\\
5990.14467004298	3.2148082176915e-05\\
6035.87917878165	3.15733610051524e-05\\
6081.96286861774	3.10101322087815e-05\\
6128.39840553459	3.04582030969021e-05\\
6175.18847587023	2.99173819443083e-05\\
6222.33578647276	2.93874780736047e-05\\
6269.84306485697	2.88683019337699e-05\\
6317.71305936208	2.83596651752176e-05\\
6365.9485393108	2.78613807214033e-05\\
6414.55229516949	2.73732628370339e-05\\
6463.52713870962	2.68951271929329e-05\\
6512.8759031704	2.64267909276233e-05\\
6562.60144342272	2.59680727056851e-05\\
6612.70663613428	2.55187927729535e-05\\
6663.19437993603	2.50787730086201e-05\\
6714.06759558984	2.46478369743062e-05\\
6765.32922615749	2.42258099601746e-05\\
6816.9822371709	2.38125190281525e-05\\
6869.02961680372	2.34077930523363e-05\\
6921.47437604418	2.30114627566519e-05\\
6974.31954886926	2.2623360749847e-05\\
7027.56819242026	2.22433215578903e-05\\
7081.22338717961	2.18711816538566e-05\\
7135.2882371491	2.15067794853763e-05\\
7189.76587002947	2.11499554997295e-05\\
7244.65943740126	2.08005521666643e-05\\
7299.97211490729	2.0458413999023e-05\\
7355.7071024362	2.01233875712554e-05\\
7411.86762430771	1.97953215359047e-05\\
7468.45692945906	1.94740666381446e-05\\
7525.47829163299	1.9159475728454e-05\\
7582.93500956713	1.88514037735074e-05\\
7640.83040718485	1.85497078653654e-05\\
7699.16783378753	1.82542472290438e-05\\
7757.95066424834	1.79648832285417e-05\\
7817.18229920744	1.76814793714061e-05\\
7876.86616526879	1.7403901311908e-05\\
7937.00571519826	1.7132016852907e-05\\
7997.6044281235	1.68656959464715e-05\\
8058.66580973513	1.66048106933281e-05\\
8120.19339248961	1.63492353412019e-05\\
8182.19073581352	1.60988462821129e-05\\
8244.66142630952	1.58535220486868e-05\\
8307.60907796385	1.56131433095338e-05\\
8371.03733235538	1.53775928637485e-05\\
8434.9498588663	1.51467556345762e-05\\
8499.35035489435	1.49205186622883e-05\\
8564.24254606679	1.46987710963044e-05\\
8629.63018645584	1.44814041865956e-05\\
8695.51705879595	1.42683112743955e-05\\
8761.90697470259	1.40593877822463e-05\\
8828.80377489274	1.38545312033971e-05\\
8896.2113294071	1.36536410905736e-05\\
8964.13353783398	1.34566190441289e-05\\
9032.57432953488	1.32633686995863e-05\\
9101.53766387181	1.30737957145787e-05\\
9171.02753043637	1.28878077551884e-05\\
9241.0479492805	1.27053144816887e-05\\
9311.60297114908	1.25262275336857e-05\\
9382.69667771427	1.23504605146597e-05\\
9454.33318181161	1.21779289759022e-05\\
9526.51662767799	1.20085503998486e-05\\
9599.25119119138	1.18422441828015e-05\\
9672.54108011239	1.16789316170452e-05\\
9746.39053432772	1.15185358723516e-05\\
9820.80382609543	1.13609819768784e-05\\
9895.78526029209	1.12061967974641e-05\\
9971.33917466183	1.10541090193268e-05\\
10047.4699400673	1.09046491251741e-05\\
10124.1819607425	1.0757749373739e-05\\
10201.4796745474	1.06133437777529e-05\\
10279.3675532253	1.04713680813759e-05\\
10357.8501026605	1.03317597371049e-05\\
10436.9318631401	1.01944578821812e-05\\
10516.6174096155	1.00594033145246e-05\\
10596.9113519683	9.92653846822237e-06\\
10677.818335276	9.79580738860238e-06\\
10759.343040081	9.66715570692232e-06\\
10841.4901826617	9.54053061470844e-06\\
10924.2645153049	9.41588083777705e-06\\
11007.670826581	9.29315660997242e-06\\
11091.713941621	9.17230964665602e-06\\
11176.3987223955	9.05329311797857e-06\\
11261.730067996	8.93606162196792e-06\\
11347.7129149183	8.82057115746207e-06\\
11434.3522373485	8.70677909691525e-06\\
11521.65304745	8.59464415910197e-06\\
11609.6203956541	8.48412638174011e-06\\
11698.2593709518	8.37518709405179e-06\\
11787.5751011885	8.26778888927426e-06\\
11877.5727533603	8.16189559713068e-06\\
11968.2575339131	8.05747225626411e-06\\
12059.6346890441	7.95448508663275e-06\\
12151.7095050046	7.85290146186028e-06\\
12244.4873084065	7.75268988152647e-06\\
12337.9734665302	7.65381994338022e-06\\
12432.1733876346	7.55626231544904e-06\\
12527.0925212711	7.45998870801296e-06\\
12622.7363585976	7.36497184540692e-06\\
12719.1104326973	7.27118543760669e-06\\
12816.2203188977	7.1786041515516e-06\\
12914.0716350943	7.08720358215045e-06\\
13012.6700420745	6.99696022291415e-06\\
13112.0212438457	6.90785143615653e-06\\
13212.1309879655	6.81985542270207e-06\\
13313.0050658734	6.7329511910422e-06\\
13414.6493132266	6.64711852588155e-06\\
13517.0696102371	6.56233795602305e-06\\
13620.2718820123	6.47859072154689e-06\\
13724.2620988974	6.39585874024956e-06\\
13829.046276821	6.31412457332404e-06\\
13934.630477643	6.23337139027918e-06\\
14041.0208095055	6.15358293311909e-06\\
14148.2234271857	6.07474347982914e-06\\
14256.2445324528	5.99683780724428e-06\\
14365.0903744256	5.91985115341082e-06\\
14474.7672499352	5.8437691795874e-06\\
14585.2815038886	5.76857793207233e-06\\
14696.6395296357	5.69426380408407e-06\\
14808.8477693396	5.62081349796327e-06\\
14921.9127143489	5.54821398800452e-06\\
15035.8409055736	5.47645248426156e-06\\
15150.638933863	5.40551639770264e-06\\
15266.3134403874	5.33539330711402e-06\\
15382.8711170223	5.26607092816622e-06\\
15500.318706735	5.19753708505917e-06\\
15618.6630039756	5.12977968515224e-06\\
15737.910855069	5.06278669696217e-06\\
15858.069158612	4.99654613187098e-06\\
15979.1448658717	4.93104602983475e-06\\
16101.1449811876	4.8662744493162e-06\\
16224.0765623776	4.80221946158602e-06\\
16347.9467211451	4.738869149452e-06\\
16472.7626234916	4.67621161038055e-06\\
16598.5314901305	4.61423496388228e-06\\
16725.260596905	4.55292736293969e-06\\
16852.9572752091	4.49227700916827e-06\\
16981.6289124118	4.43227217132505e-06\\
17111.2829522843	4.37290120671283e-06\\
17241.9268954305	4.31415258497838e-06\\
17373.5682997215	4.256014913768e-06\\
17506.2147807321	4.19847696568609e-06\\
17639.8740121818	4.14152770600123e-06\\
17774.5537263786	4.08515632055752e-06\\
17910.2617146665	4.02935224337727e-06\\
18047.005827876	3.97410518347865e-06\\
18184.7939767782	3.91940515048056e-06\\
18323.6341325429	3.86524247862001e-06\\
18463.5343271993	3.81160784886457e-06\\
18604.5026541008	3.75849230886199e-06\\
18746.5472683933	3.70588729052549e-06\\
18889.6763874869	3.65378462510994e-06\\
19033.8982915311	3.60217655568485e-06\\
19179.2213238944	3.55105574695671e-06\\
19325.6538916461	3.50041529243574e-06\\
19473.2044660434	3.45024871897629e-06\\
19621.8815830212	3.40054998875195e-06\\
19771.6938436859	3.35131349875048e-06\\
19922.649914813	3.30253407789387e-06\\
20074.7585293483	3.25420698190449e-06\\
20228.0284869136	3.20632788604953e-06\\
20382.4686543151	3.15889287590467e-06\\
20538.087966057	3.11189843628244e-06\\
20694.8954248579	3.0653414384746e-06\\
20852.9001021718	3.01921912595855e-06\\
21012.1111387131	2.97352909871744e-06\\
21172.5377449848	2.92826929632299e-06\\
21334.1892018119	2.88343797992677e-06\\
21497.0748608783	2.83903371330431e-06\\
21661.2041452672	2.7950553430923e-06\\
21826.5865500072	2.75150197835633e-06\\
21993.2316426205	2.70837296962362e-06\\
22161.1490636774	2.66566788751062e-06\\
22330.3485273531	2.62338650107318e-06\\
22500.8398219907	2.58152875600227e-06\\
22672.6328106664	2.54009475278548e-06\\
22845.7374317606	2.49908472495038e-06\\
23020.1636995334	2.45849901750188e-06\\
23195.9217047026	2.4183380656625e-06\\
23373.0216150288	2.37860237401905e-06\\
23551.4736759026	2.3392924961759e-06\\
23731.2882109381	2.30040901500944e-06\\
23912.4756225696	2.26195252361312e-06\\
24095.0463926535	2.22392360701763e-06\\
24279.0110830748	2.18632282476337e-06\\
24464.3803363582	2.1491506943971e-06\\
24651.1648762833	2.11240767595675e-06\\
24839.3755085058	2.07609415750115e-06\\
25029.0231211817	2.04021044173368e-06\\
25220.1186855978	2.00475673375978e-06\\
25412.6732568062	1.96973313001024e-06\\
25606.697974264	1.93513960835265e-06\\
25802.2040624775	1.90097601940404e-06\\
25999.2028316513	1.86724207904858e-06\\
26197.7056783437	1.8339373621545e-06\\
26397.7240861246	1.80106129747521e-06\\
26599.2696262408	1.76861316371027e-06\\
26802.353958285	1.73659208669324e-06\\
27006.9888308703	1.70499703766501e-06\\
27213.1860823103	1.67382683258361e-06\\
27420.9576413033	1.64308013241444e-06\\
27630.3155276229	1.6127554443386e-06\\
27841.2718528132	1.58285112381201e-06\\
28053.8388208892	1.5533653774034e-06\\
28268.0287290435	1.52429626633614e-06\\
28483.8539683568	1.49564171065683e-06\\
28701.3270245155	1.46739949395193e-06\\
28920.4604785335	1.43956726853418e-06\\
29141.2670074804	1.41214256102102e-06\\
29363.7593852146	1.38512277822913e-06\\
29587.9504831223	1.358505213312e-06\\
29813.8532708625	1.33228705207064e-06\\
30041.4808171167	1.30646537937186e-06\\
30270.8462903454	1.28103718561306e-06\\
30501.9629595499	1.25599937317751e-06\\
30734.8441950392	1.23134876282965e-06\\
30969.503469205	1.20708210000515e-06\\
31205.9543572992	1.18319606095681e-06\\
31444.2105382208	1.15968725872214e-06\\
31684.2857953065	1.1365522488848e-06\\
31926.194017128	1.11378753510703e-06\\
32169.949198296	1.09138957441526e-06\\
32415.5654402693	1.06935478222626e-06\\
32663.0569521706	1.04767953710495e-06\\
32912.4380516091	1.02636018524985e-06\\
33163.7231655079	1.0053930447049e-06\\
33416.9268309394	9.8477440930028e-07\\
33672.0636959658	9.64500552327177e-07\\
33929.1485204865	9.44567729953926e-07\\
34188.1961770926	9.24972184392977e-07\\
34449.2216519263	9.05710146829547e-07\\
34712.2400455487	8.86777840124103e-07\\
34977.2665738129	8.68171481301724e-07\\
35244.3165687446	8.49887283841784e-07\\
35513.4054794286	8.31921459782054e-07\\
35784.5488729031	8.1427022165091e-07\\
36057.76243506	7.96929784241685e-07\\
36333.0619715519	7.79896366242582e-07\\
36610.4634087077	7.63166191735178e-07\\
36889.9827944524	7.46735491574179e-07\\
37171.6362992366	7.30600504660082e-07\\
37455.4402169718	7.14757479116066e-07\\
37741.4109659722	6.99202673379401e-07\\
38029.5650899061	6.8393235721693e-07\\
38319.9192587512	6.68942812673548e-07\\
38612.4902697601	6.54230334961496e-07\\
38907.2950484319	6.39791233297805e-07\\
39204.350649491	6.25621831696265e-07\\
39503.674257874	6.1171846971966e-07\\
39805.2831897235	5.98077503197302e-07\\
40109.1948933907	5.84695304912162e-07\\
40415.4269504438	5.71568265261527e-07\\
40723.9970766856	5.58692792894201e-07\\
41034.9231231787	5.46065315327114e-07\\
41348.223077277	5.33682279543521e-07\\
41663.9150636682	5.21540152574573e-07\\
41982.0173454204	5.09635422065878e-07\\
42302.5483250399	4.97964596829993e-07\\
42625.5265455352	4.86524207385885e-07\\
42950.9706914898	4.75310806485889e-07\\
43278.8995901437	4.64320969630555e-07\\
43609.3322124815	4.53551295571737e-07\\
43942.2876743309	4.4299840680388e-07\\
44277.7852374682	4.32658950043586e-07\\
44615.8443107321	4.22529596697327e-07\\
44956.4844511476	4.12607043317097e-07\\
45299.7253650563	4.02888012043832e-07\\
45645.5869092572	3.93369251038278e-07\\
45994.0890921549	3.84047534899076e-07\\
46345.2520749175	3.7491966506775e-07\\
46699.096172643	3.65982470220324e-07\\
47055.6418555338	3.57232806645292e-07\\
47414.9097500821	3.48667558607646e-07\\
47776.920640262	3.40283638698816e-07\\
48141.6954687325	3.32077988172175e-07\\
48509.2553380493	3.24047577264042e-07\\
48879.6215118846	3.1618940549995e-07\\
49252.8154162586	3.08500501986053e-07\\
49628.8586407777	3.00977925685625e-07\\
50007.7729398843	2.93618765680481e-07\\
50389.5802341153	2.86420141417355e-07\\
50774.3026113693	2.79379202939177e-07\\
51161.962328186	2.72493131101217e-07\\
51552.5818110322	2.65759137772203e-07\\
51946.1836576	2.59174466020353e-07\\
52342.7906381142	2.52736390284465e-07\\
52742.4256966491	2.46442216530102e-07\\
53145.1119524562	2.40289282390975e-07\\
53550.8727013012	2.34274957295646e-07\\
53959.7314168126	2.28396642579628e-07\\
54371.7117518386	2.22651771583096e-07\\
54786.8375398161	2.17037809734253e-07\\
55205.1327961494	2.11552254618578e-07\\
55626.6217195991	2.06192636034085e-07\\
56051.3286936829	2.00956516032745e-07\\
56479.2782880851	1.95841488948293e-07\\
56910.4952600788	1.90845181410543e-07\\
57345.0045559579	1.85965252346438e-07\\
57782.83131248	1.81199392968015e-07\\
58224.0008583212	1.76545326747465e-07\\
58668.5387155406	1.72000809379535e-07\\
59116.4706010574	1.67563628731422e-07\\
59567.8224281384	1.63231604780421e-07\\
60022.620307897	1.59002589539524e-07\\
60480.8905508042	1.54874466971193e-07\\
60942.6596682098	1.50845152889547e-07\\
61407.9543738774	1.46912594851179e-07\\
61876.8015855284	1.43074772034872e-07\\
62349.2284264003	1.39329695110427e-07\\
62825.2622268155	1.35675406096871e-07\\
63304.9305257616	1.32109978210315e-07\\
63788.2610724863	1.28631515701668e-07\\
64275.2818281007	1.2523815368456e-07\\
64766.0209671984	1.21928057953655e-07\\
65260.5068794848	1.18699424793688e-07\\
65758.7681714191	1.15550480779493e-07\\
66260.8336678705	1.12479482567287e-07\\
66766.7324137839	1.0948471667753e-07\\
67276.4936758618	1.06564499269627e-07\\
67790.1469442567	1.03717175908779e-07\\
68307.721934277	1.00941121325286e-07\\
68829.2485881065	9.82347391665929e-08\\
69354.7570765359	9.55964617423925e-08\\
69884.2778007095	9.30247497630775e-08\\
70417.8413938822	9.0518092071873e-08\\
70955.4787231929	8.80750053709254e-08\\
71497.22089145	8.56940339416831e-08\\
72043.0992389301	8.3373749359869e-08\\
72593.1453451923	8.11127502053402e-08\\
73147.3910309034	7.89096617671719e-08\\
73705.8683596805	7.67631357442367e-08\\
74268.6096399447	7.46718499416117e-08\\
74835.6474267902	7.26345079631024e-08\\
75407.0145238692	7.06498389001796e-08\\
75982.743985287	6.87165970176433e-08\\
76562.8691175166	6.68335614362821e-08\\
77147.4234813242	6.49995358128387e-08\\
77736.4408937111	6.32133480175546e-08\\
78329.9554298704	6.14738498095771e-08\\
78928.001425157	5.97799165105081e-08\\
79530.6134770757	5.81304466763551e-08\\
80137.8264472814	5.65243617681614e-08\\
80749.6754635966	5.49606058215649e-08\\
81366.195922043	5.34381451155426e-08\\
81987.4234888893	5.19559678405889e-08\\
82613.3941027154	5.05130837665585e-08\\
83244.1439764901	4.91085239104249e-08\\
83879.709599667	4.77413402041606e-08\\
84520.1277402954	4.64106051629748e-08\\
85165.4354471463	4.51154115541144e-08\\
85815.670051858	4.38548720664306e-08\\
86470.8691710928	4.26281189809205e-08\\
87131.0707087156	4.14343038424209e-08\\
87796.3128579857	4.02725971326477e-08\\
88466.6341037656	3.91421879447564e-08\\
89142.0732247495	3.80422836595843e-08\\
89822.669295704	3.69721096237506e-08\\
90508.4616897307	3.59309088297536e-08\\
91199.4900805429	3.49179415982231e-08\\
91895.7944447613	3.39324852624633e-08\\
92597.4150642266	3.29738338554192e-08\\
93304.3925283287	3.20412977991956e-08\\
94016.7677363574	3.11342035972402e-08\\
94734.5818998656	3.02518935293149e-08\\
95457.8765450552	2.93937253493476e-08\\
96186.6935151793	2.85590719862732e-08\\
96921.0749729615	2.77473212479524e-08\\
97661.0634030374	2.6957875528252e-08\\
98406.70161441	2.61901515173735e-08\\
99158.0327429277	2.54435799154959e-08\\
99915.1002537795	2.47176051498069e-08\\
100677.947944009	2.40116850949826e-08\\
101446.619945048	2.33252907971698e-08\\
102221.160725271	2.26579062015298e-08\\
103001.615092567	2.20090278833802e-08\\
103788.028196929	2.13781647829843e-08\\
104580.445533071	2.07648379440191e-08\\
105378.912943057	2.01685802557559e-08\\
106183.47661895	1.958893619898e-08\\
106994.183105493	1.90254615956706e-08\\
107811.079302791	1.84777233624631e-08\\
108634.212469033	1.79452992679027e-08\\
109463.63022322	1.74277776935048e-08\\
110299.380547924	1.69247573986267e-08\\
111141.51179206	1.64358472891515e-08\\
111990.072673687	1.59606661899899e-08\\
112845.112282822	1.54988426213891e-08\\
113706.680084285	1.50500145790476e-08\\
114574.825920556	1.46138293180247e-08\\
115449.600014662	1.41899431404292e-08\\
116331.052973077	1.37780211868787e-08\\
117219.235788658	1.33777372317048e-08\\
118114.199843588	1.2988773481888e-08\\
119015.99691235	1.26108203797004e-08\\
119924.679164726	1.2243576409029e-08\\
120840.299168808	1.18867479053558e-08\\
121762.909894048	1.15400488693632e-08\\
122692.564714314	1.12032007841389e-08\\
123629.317410982	1.08759324359438e-08\\
124573.222176048	1.05579797385127e-08\\
125524.33361526	1.02490855608514e-08\\
126482.706751281	9.94899955849389e-09\\
127448.397026865	9.65747800818449e-09\\
128421.460308075	9.37428364594228e-09\\
129401.952887506	9.099185508472e-09\\
130389.931487544	8.83195877787912e-09\\
131385.45326365	8.57238462964653e-09\\
132388.575807663	8.32025008383385e-09\\
133399.357151135	8.07534785945231e-09\\
134417.855768684	7.83747623197475e-09\\
135444.130581383	7.60643889393508e-09\\
136478.240960161	7.38204481857263e-09\\
137520.246729245	7.16410812647695e-09\\
138570.208169615	6.95244795518646e-09\\
139628.186022495	6.7468883316964e-09\\
140694.241492865	6.54725804782939e-09\\
141768.436253005	6.35339053842251e-09\\
142850.832446056	6.16512376228496e-09\\
143941.492689623	5.98230008587894e-09\\
145040.480079392	5.80476616967921e-09\\
146147.858192783	5.63237285716265e-09\\
147263.691092627	5.46497506638332e-09\\
148388.04333087	5.30243168408616e-09\\
149520.979952313	5.14460546231262e-09\\
150662.566498367	4.99136291745403e-09\\
151812.869010854	4.84257423170496e-09\\
152971.954035819	4.69811315687281e-09\\
154139.888627384	4.55785692049793e-09\\
155316.740351627	4.42168613423879e-09\\
156502.577290491	4.28948470447935e-09\\
157697.468045719	4.16113974511259e-09\\
158901.481742829	4.0365414924581e-09\\
160114.688035108	3.9155832222694e-09\\
161337.157107642	3.79816116878858e-09\\
162568.959681378	3.68417444580595e-09\\
163810.167017216	3.573524969682e-09\\
165060.850920127	3.4661173842915e-09\\
166321.083743312	3.36185898784738e-09\\
167590.938392387	3.26065966156494e-09\\
168870.488329596	3.16243180012618e-09\\
170159.807578068	3.06709024390441e-09\\
171458.970726091	2.9745522129114e-09\\
172768.052931436	2.88473724242741e-09\\
174087.129925698	2.79756712027761e-09\\
175416.27801868	2.71296582571707e-09\\
176755.574102809	2.63085946988775e-09\\
178105.095657579	2.55117623781215e-09\\
179464.920754039	2.47384633188725e-09\\
180835.128059308	2.39880191684489e-09\\
182215.796841124	2.32597706614368e-09\\
183607.00697243	2.2553077097592e-09\\
185008.838935997	2.18673158333945e-09\\
186421.373829081	2.12018817869272e-09\\
187844.693368106	2.05561869557678e-09\\
189278.879893403	1.99296599475731e-09\\
190724.016373967	1.93217455230551e-09\\
192180.186412254	1.87319041510466e-09\\
193647.474249029	1.8159611575359e-09\\
195125.964768223	1.76043583931479e-09\\
196615.743501859	1.70656496444981e-09\\
198116.896634991	1.65430044129546e-09\\
199629.511010691	1.60359554367256e-09\\
201153.674135077	1.55440487302908e-09\\
202689.47418237	1.50668432161572e-09\\
204237	1.4603910366503e-09\\
204237	8.27132091421398e-10\\
202689.47418237	8.55122964450807e-10\\
201153.674135077	8.84040967491189e-10\\
199629.511010691	9.13916201580843e-10\\
198116.896634991	9.44779724630318e-10\\
196615.743501859	9.76663580970072e-10\\
195125.964768223	1.00960083176512e-09\\
193647.474249029	1.04362558632009e-09\\
192180.186412254	1.07877303429791e-09\\
190724.016373967	1.11507947887607e-09\\
189278.879893403	1.15258237086535e-09\\
187844.693368106	1.19132034381539e-09\\
186421.373829081	1.2313332501331e-09\\
185008.838935997	1.27266219823965e-09\\
183607.00697243	1.3153495907925e-09\\
182215.796841124	1.35943916399975e-09\\
180835.128059308	1.40497602805382e-09\\
179464.920754039	1.45200670871308e-09\\
178105.095657579	1.50057919005949e-09\\
176755.574102809	1.55074295846154e-09\\
175416.27801868	1.60254904777202e-09\\
174087.129925698	1.65605008579059e-09\\
172768.052931436	1.71130034202197e-09\\
171458.970726091	1.76835577676049e-09\\
170159.807578068	1.82727409153302e-09\\
168870.488329596	1.88811478093206e-09\\
167590.938392387	1.95093918587171e-09\\
166321.083743312	2.01581054829985e-09\\
165060.850920127	2.08279406739954e-09\\
163810.167017216	2.15195695731453e-09\\
162568.959681378	2.22336850643276e-09\\
161337.157107642	2.29710013826315e-09\\
160114.688035108	2.37322547394148e-09\\
158901.481742829	2.45182039640053e-09\\
157697.468045719	2.53296311624187e-09\\
156502.577290491	2.61673423934535e-09\\
155316.740351627	2.70321683625391e-09\\
154139.888627384	2.79249651337143e-09\\
152971.954035819	2.88466148601132e-09\\
151812.869010854	2.97980265333504e-09\\
150662.566498367	3.07801367521822e-09\\
149520.979952313	3.17939105108492e-09\\
148388.04333087	3.28403420074843e-09\\
147263.691092627	3.39204554729883e-09\\
146147.858192783	3.50353060207788e-09\\
145040.480079392	3.6185980517805e-09\\
143941.492689623	3.73735984772488e-09\\
142850.832446056	3.85993129733099e-09\\
141768.436253005	3.98643115784895e-09\\
140694.241492865	4.11698173237907e-09\\
139628.186022495	4.2517089682236e-09\\
138570.208169615	4.39074255761345e-09\\
137520.246729245	4.53421604085024e-09\\
136478.240960161	4.68226691190584e-09\\
135444.130581383	4.83503672652133e-09\\
134417.855768684	4.99267121284666e-09\\
133399.357151135	5.15532038466352e-09\\
132388.575807663	5.32313865723215e-09\\
131385.45326365	5.4962849658044e-09\\
130389.931487544	5.67492288684422e-09\\
129401.952887506	5.85922076199658e-09\\
128421.460308075	6.04935182484701e-09\\
127448.397026865	6.24549433051048e-09\\
126482.706751281	6.44783168809278e-09\\
125524.33361526	6.65655259606253e-09\\
124573.222176048	6.87185118057436e-09\\
123629.317410982	7.09392713678371e-09\\
122692.564714314	7.32298587318988e-09\\
121762.909894048	7.55923865904861e-09\\
120840.299168808	7.80290277489002e-09\\
119924.679164726	8.05420166617992e-09\\
119015.99691235	8.31336510016232e-09\\
118114.199843588	8.5806293259165e-09\\
117219.235788658	8.8562372376673e-09\\
116331.052973077	9.14043854138044e-09\\
115449.600014662	9.43348992467793e-09\\
114574.825920556	9.73565523010621e-09\\
113706.680084285	1.00472056317882e-08\\
112845.112282822	1.0368419815492e-08\\
111990.072673687	1.06995841621433e-08\\
111141.51179206	1.10409929348137e-08\\
110299.380547924	1.13929484692091e-08\\
109463.63022322	1.17557613676878e-08\\
108634.212469033	1.21297506968324e-08\\
107811.079302791	1.25152441885977e-08\\
106994.183105493	1.29125784450613e-08\\
106183.47661895	1.3322099146794e-08\\
105378.912943057	1.37441612648724e-08\\
104580.445533071	1.41791292765525e-08\\
103788.028196929	1.46273773846176e-08\\
103001.615092567	1.50892897404205e-08\\
102221.160725271	1.55652606706301e-08\\
101446.619945048	1.60556949076957e-08\\
100677.947944009	1.65610078240386e-08\\
99915.1002537795	1.70816256699793e-08\\
99158.0327429277	1.76179858154087e-08\\
98406.70161441	1.81705369952027e-08\\
97661.0634030374	1.87397395583896e-08\\
96921.0749729615	1.93260657210646e-08\\
96186.6935151793	1.99299998230533e-08\\
95457.8765450552	2.05520385883202e-08\\
94734.5818998656	2.1192691389113e-08\\
94016.7677363574	2.18524805138395e-08\\
93304.3925283287	2.25319414386617e-08\\
92597.4150642266	2.32316231027955e-08\\
91895.7944447613	2.39520881875038e-08\\
91199.4900805429	2.46939133987563e-08\\
90508.4616897307	2.54576897535464e-08\\
89822.669295704	2.62440228698319e-08\\
89142.0732247495	2.70535332600785e-08\\
88466.6341037656	2.78868566283757e-08\\
87796.3128579857	2.87446441710919e-08\\
87131.0707087156	2.96275628810372e-08\\
86470.8691710928	3.05362958550887e-08\\
85815.670051858	3.14715426052463e-08\\
85165.4354471463	3.24340193730661e-08\\
84520.1277402954	3.34244594474281e-08\\
83879.709599667	3.44436134855876e-08\\
83244.1439764901	3.54922498374495e-08\\
82613.3941027154	3.65711548730168e-08\\
81987.4234888893	3.76811333129427e-08\\
81366.195922043	3.88230085621243e-08\\
80749.6754635966	3.99976230462734e-08\\
80137.8264472814	4.12058385513781e-08\\
79530.6134770757	4.24485365659953e-08\\
78928.001425157	4.372661862628e-08\\
78329.9554298704	4.50410066636722e-08\\
77736.4408937111	4.63926433551536e-08\\
77147.4234813242	4.7782492475976e-08\\
76562.8691175166	4.92115392547738e-08\\
75982.743985287	5.06807907309496e-08\\
75407.0145238692	5.21912761142352e-08\\
74835.6474267902	5.37440471463175e-08\\
74268.6096399447	5.53401784644137e-08\\
73705.8683596805	5.69807679666879e-08\\
73147.3910309034	5.86669371793738e-08\\
72593.1453451923	6.03998316254972e-08\\
72043.0992389301	6.21806211950574e-08\\
71497.22089145	6.40105005165399e-08\\
70955.4787231929	6.5890689329634e-08\\
70417.8413938822	6.78224328589991e-08\\
69884.2778007095	6.98070021889631e-08\\
69354.7570765359	7.18456946389881e-08\\
68829.2485881065	7.39398341397681e-08\\
68307.721934277	7.60907716098142e-08\\
67790.1469442567	7.82998853323577e-08\\
67276.4936758618	8.05685813324451e-08\\
66766.7324137839	8.2898293754047e-08\\
66260.8336678705	8.52904852370373e-08\\
65758.7681714191	8.77466472938834e-08\\
65260.5068794848	9.02683006858813e-08\\
64766.0209671984	9.28569957987957e-08\\
64275.2818281007	9.55143130177114e-08\\
63788.2610724863	9.82418631009723e-08\\
63304.9305257616	1.01041287553017e-07\\
62825.2622268155	1.03914258995958e-07\\
62349.2284264003	1.06862481539751e-07\\
61876.8015855284	1.09887691150761e-07\\
61407.9543738774	1.12991656018595e-07\\
60942.6596682098	1.16176176920995e-07\\
60480.8905508042	1.19443087586647e-07\\
60022.620307897	1.22794255055718e-07\\
59567.8224281384	1.26231580037937e-07\\
59116.4706010574	1.29756997268044e-07\\
58668.5387155406	1.33372475858405e-07\\
58224.0008583212	1.37080019648595e-07\\
57782.83131248	1.40881667551725e-07\\
57345.0045559579	1.4477949389729e-07\\
56910.4952600788	1.48775608770315e-07\\
56479.2782880851	1.52872158346474e-07\\
56051.3286936829	1.57071325222975e-07\\
55626.6217195991	1.6137532874482e-07\\
55205.1327961494	1.65786425326146e-07\\
54786.8375398161	1.70306908766243e-07\\
54371.7117518386	1.74939110559811e-07\\
53959.7314168126	1.79685400201017e-07\\
53550.8727013012	1.84548185480786e-07\\
53145.1119524562	1.89529912776762e-07\\
52742.4256966491	1.94633067335289e-07\\
52342.7906381142	1.9986017354463e-07\\
51946.1836576	2.05213795198675e-07\\
51552.5818110322	2.10696535750165e-07\\
51161.962328186	2.1631103855245e-07\\
50774.3026113693	2.22059987088639e-07\\
50389.5802341153	2.2794610518688e-07\\
50007.7729398843	2.33972157220421e-07\\
49628.8586407777	2.40140948290862e-07\\
49252.8154162586	2.46455324392976e-07\\
48879.6215118846	2.52918172559212e-07\\
48509.2553380493	2.59532420981847e-07\\
48141.6954687325	2.66301039110644e-07\\
47776.920640262	2.73227037723501e-07\\
47414.9097500821	2.80313468967606e-07\\
47055.6418555338	2.87563426368193e-07\\
46699.096172643	2.949800448019e-07\\
46345.2520749175	3.02566500431501e-07\\
45994.0890921549	3.10326010598456e-07\\
45645.5869092572	3.1826183366967e-07\\
45299.7253650563	3.26377268834461e-07\\
44956.4844511476	3.34675655847615e-07\\
44615.8443107321	3.43160374714169e-07\\
44277.7852374682	3.5183484531133e-07\\
43942.2876743309	3.60702526942826e-07\\
43609.3322124815	3.69766917820699e-07\\
43278.8995901437	3.79031554469519e-07\\
42950.9706914898	3.88500011047792e-07\\
42625.5265455352	3.98175898581281e-07\\
42302.5483250399	4.08062864102962e-07\\
41982.0173454204	4.18164589694141e-07\\
41663.9150636682	4.28484791421605e-07\\
41348.223077277	4.39027218165469e-07\\
41034.9231231787	4.49795650332713e-07\\
40723.9970766856	4.60793898451664e-07\\
40415.4269504438	4.72025801642757e-07\\
40109.1948933907	4.83495225961603e-07\\
39805.2831897235	4.95206062610567e-07\\
39503.674257874	5.07162226015781e-07\\
39204.350649491	5.19367651767178e-07\\
38907.2950484319	5.31826294419702e-07\\
38612.4902697601	5.44542125154995e-07\\
38319.9192587512	5.5751912930352e-07\\
38029.5650899061	5.70761303728377e-07\\
37741.4109659722	5.84272654073103e-07\\
37455.4402169718	5.98057191877081e-07\\
37171.6362992366	6.12118931563608e-07\\
36889.9827944524	6.26461887306975e-07\\
36610.4634087077	6.41090069786752e-07\\
36333.0619715519	6.56007482838841e-07\\
36057.76243506	6.71218120014723e-07\\
35784.5488729031	6.86725961062126e-07\\
35513.4054794286	7.02534968341901e-07\\
35244.3165687446	7.18649083198055e-07\\
34977.2665738129	7.35072222299337e-07\\
34712.2400455487	7.51808273972723e-07\\
34449.2216519263	7.68861094550872e-07\\
34188.1961770926	7.86234504756933e-07\\
33929.1485204865	8.03932286151947e-07\\
33672.0636959658	8.21958177670939e-07\\
33416.9268309394	8.40315872275146e-07\\
33163.7231655079	8.59009013748512e-07\\
32912.4380516091	8.78041193667049e-07\\
32663.0569521706	8.97415948570031e-07\\
32415.5654402693	9.17136757361438e-07\\
32169.949198296	9.37207038969942e-07\\
31926.194017128	9.5763015029422e-07\\
31684.2857953065	9.78409384459171e-07\\
31444.2105382208	9.99547969406631e-07\\
31205.9543572992	1.02104906684138e-06\\
30969.503469205	1.04291577155081e-06\\
30734.8441950392	1.06515111111261e-06\\
30501.9629595499	1.08775804600129e-06\\
30270.8462903454	1.1107394700999e-06\\
30041.4808171167	1.13409821161847e-06\\
29813.8532708625	1.15783703441586e-06\\
29587.9504831223	1.18195863971617e-06\\
29363.7593852146	1.20646566820571e-06\\
29141.2670074804	1.23136070249053e-06\\
28920.4604785335	1.25664626988946e-06\\
28701.3270245155	1.28232484553201e-06\\
28483.8539683568	1.30839885572542e-06\\
28268.0287290435	1.33487068155017e-06\\
28053.8388208892	1.36174266263955e-06\\
27841.2718528132	1.3890171010944e-06\\
27630.3155276229	1.41669626548187e-06\\
27420.9576413033	1.44478239486397e-06\\
27213.1860823103	1.47327770280109e-06\\
27006.9888308703	1.50218438127408e-06\\
26802.353958285	1.53150460446943e-06\\
26599.2696262408	1.5612405323731e-06\\
26397.7240861246	1.59139431412e-06\\
26197.7056783437	1.62196809104988e-06\\
25999.2028316513	1.65296399942307e-06\\
25802.2040624775	1.68438417275427e-06\\
25606.697974264	1.71623074372759e-06\\
25412.6732568062	1.74850584566087e-06\\
25220.1186855978	1.78121161349384e-06\\
25029.0231211817	1.81435018428017e-06\\
24839.3755085058	1.84792369717017e-06\\
24651.1648762833	1.88193429287687e-06\\
24464.3803363582	1.91638411262501e-06\\
24279.0110830748	1.95127529658845e-06\\
24095.0463926535	1.98660998182736e-06\\
23912.4756225696	2.02239029974317e-06\\
23731.2882109381	2.05861837307382e-06\\
23551.4736759026	2.0952963124579e-06\\
23373.0216150288	2.13242621260098e-06\\
23195.9217047026	2.17001014808155e-06\\
23020.1636995334	2.20805016883936e-06\\
22845.7374317606	2.24654829539194e-06\\
22672.6328106664	2.28550651382936e-06\\
22500.8398219907	2.32492677064126e-06\\
22330.3485273531	2.36481096743272e-06\\
22161.1490636774	2.40516095559024e-06\\
21993.2316426205	2.44597853096133e-06\\
21826.5865500072	2.48726542861563e-06\\
21661.2041452672	2.52902331775857e-06\\
21497.0748608783	2.57125379687261e-06\\
21334.1892018119	2.61395838916524e-06\\
21172.5377449848	2.65713853840676e-06\\
21012.1111387131	2.70079560524612e-06\\
20852.9001021718	2.74493086409726e-06\\
20694.8954248579	2.78954550069382e-06\\
20538.087966057	2.83464061041543e-06\\
20382.4686543151	2.88021719749336e-06\\
20228.0284869136	2.9262761752097e-06\\
20074.7585293483	2.9728183672078e-06\\
19922.649914813	3.01984451003709e-06\\
19771.6938436859	3.0673552570584e-06\\
19621.8815830212	3.115351183838e-06\\
19473.2044660434	3.1638327951599e-06\\
19325.6538916461	3.21280053378338e-06\\
19179.2213238944	3.26225479106918e-06\\
19033.8982915311	3.31219591958995e-06\\
18889.6763874869	3.36262424782894e-06\\
18746.5472683933	3.41354009705567e-06\\
18604.5026541008	3.46494380044538e-06\\
18463.5343271993	3.51683572448438e-06\\
18323.6341325429	3.56921629267036e-06\\
18184.7939767782	3.62208601147994e-06\\
18047.005827876	3.67544549853325e-06\\
17910.2617146665	3.72929551283643e-06\\
17774.5537263786	3.78363698693298e-06\\
17639.8740121818	3.838471060739e-06\\
17506.2147807321	3.89379911678282e-06\\
17373.5682997215	3.94962281651564e-06\\
17241.9268954305	4.00594413730743e-06\\
17111.2829522843	4.06276540969961e-06\\
16981.6289124118	4.12008935444756e-06\\
16852.9572752091	4.17791911886211e-06\\
16725.260596905	4.23625831194646e-06\\
16598.5314901305	4.29511103782786e-06\\
16472.7626234916	4.35448192700276e-06\\
16347.9467211451	4.41437616494891e-06\\
16224.0765623776	4.47479951770933e-06\\
16101.1449811876	4.53575835411855e-06\\
15979.1448658717	4.59725966441873e-06\\
15858.069158612	4.65931107510032e-06\\
15737.910855069	4.72192085989281e-06\\
15618.6630039756	4.78509794692645e-06\\
15500.318706735	4.84885192217528e-06\\
15382.8711170223	4.91319302937876e-06\\
15266.3134403874	4.97813216671496e-06\\
15150.638933863	5.04368088056253e-06\\
15035.8409055736	5.10985135674078e-06\\
14921.9127143489	5.17665640965243e-06\\
14808.8477693396	5.24410946977575e-06\\
14696.6395296357	5.31222456996076e-06\\
14585.2815038886	5.38101633097667e-06\\
14474.7672499352	5.45049994674261e-06\\
14365.0903744256	5.52069116964484e-06\\
14256.2445324528	5.59160629631041e-06\\
14148.2234271857	5.66326215416717e-06\\
14041.0208095055	5.7356760890769e-06\\
13934.630477643	5.80886595428459e-06\\
13829.046276821	5.88285010088132e-06\\
13724.2620988974	5.95764736993748e-06\\
13620.2718820123	6.03327708642095e-06\\
13517.0696102371	6.10975905497934e-06\\
13414.6493132266	6.18711355763209e-06\\
13313.0050658734	6.26536135338861e-06\\
13212.1309879655	6.34452367978489e-06\\
13112.0212438457	6.42462225630937e-06\\
13012.6700420745	6.50567928967193e-06\\
12914.0716350943	6.58771748085751e-06\\
12816.2203188977	6.67076003389387e-06\\
12719.1104326973	6.75483066625816e-06\\
12622.7363585976	6.83995362083998e-06\\
12527.0925212711	6.92615367937785e-06\\
12432.1733876346	7.0134561772844e-06\\
12337.9734665302	7.10188701977675e-06\\
12244.4873084065	7.19147269923134e-06\\
12151.7095050046	7.28224031368394e-06\\
12059.6346890441	7.37421758640205e-06\\
11968.2575339131	7.46743288645895e-06\\
11877.5727533603	7.56191525024525e-06\\
11787.5751011885	7.65769440385878e-06\\
11698.2593709518	7.75480078631789e-06\\
11609.6203956541	7.85326557355076e-06\\
11521.65304745	7.95312070311644e-06\\
11434.3522373485	8.05439889962004e-06\\
11347.7129149183	8.15713370078927e-06\\
11261.730067996	8.26135948418306e-06\\
11176.3987223955	8.36711149450954e-06\\
11091.713941621	8.47442587153235e-06\\
11007.670826581	8.58333967854936e-06\\
10924.2645153049	8.69389093143042e-06\\
10841.4901826617	8.80611862820312e-06\\
10759.343040081	8.92006277917863e-06\\
10677.818335276	9.03576443760923e-06\\
10596.9113519683	9.15326573087279e-06\\
10516.6174096155	9.27260989217748e-06\\
10436.9318631401	9.39384129278098e-06\\
10357.8501026605	9.5170054747183e-06\\
10279.3675532253	9.64214918402904e-06\\
10201.4796745474	9.76932040447597e-06\\
10124.1819607425	9.89856839174201e-06\\
10047.4699400673	1.00299437080916e-05\\
9971.33917466183	1.01634982574788e-05\\
9895.78526029209	1.02992853210806e-05\\
9820.80382609543	1.04373595932321e-05\\
9746.39053432772	1.05777772177327e-05\\
9672.54108011239	1.07205958244922e-05\\
9599.25119119138	1.08658745664794e-05\\
9526.51662767799	1.10136741569324e-05\\
9454.33318181161	1.11640569067837e-05\\
9382.69667771427	1.13170867622536e-05\\
9311.60297114908	1.14728293425567e-05\\
9241.0479492805	1.1631351977666e-05\\
9171.02753043637	1.17927237460741e-05\\
9101.53766387181	1.19570155124909e-05\\
9032.57432953488	1.212429996541e-05\\
8964.13353783398	1.22946516544795e-05\\
8896.2113294071	1.24681470276074e-05\\
8828.80377489274	1.26448644677318e-05\\
8761.90697470259	1.28248843291868e-05\\
8695.51705879595	1.30082889735932e-05\\
8629.63018645584	1.3195162805203e-05\\
8564.24254606679	1.33855923056301e-05\\
8499.35035489435	1.35796660678957e-05\\
8434.9498588663	1.37774748297239e-05\\
8371.03733235538	1.39791115060198e-05\\
8307.60907796385	1.41846712204654e-05\\
8244.66142630952	1.43942513361724e-05\\
8182.19073581352	1.46079514853293e-05\\
8120.19339248961	1.48258735977856e-05\\
8058.66580973513	1.50481219285147e-05\\
7997.6044281235	1.52748030839009e-05\\
7937.00571519826	1.55060260467973e-05\\
7876.86616526879	1.57419022003014e-05\\
7817.18229920744	1.59825453501981e-05\\
7757.95066424834	1.62280717460211e-05\\
7699.16783378753	1.64786001006822e-05\\
7640.83040718485	1.67342516086217e-05\\
7582.93500956713	1.69951499624313e-05\\
7525.47829163299	1.72614213679013e-05\\
7468.45692945906	1.75331945574454e-05\\
7411.86762430771	1.7810600801853e-05\\
7355.7071024362	1.8093773920322e-05\\
7299.97211490729	1.83828502887211e-05\\
7244.65943740126	1.86779688460312e-05\\
7189.76587002947	1.89792710989145e-05\\
7135.2882371491	1.92869011243577e-05\\
7081.22338717961	1.96010055703348e-05\\
7027.56819242026	1.99217336544344e-05\\
6974.31954886926	2.02492371603929e-05\\
6921.47437604418	2.05836704324744e-05\\
6869.02961680372	2.09251903676378e-05\\
6816.9822371709	2.12739564054273e-05\\
6765.32922615749	2.16301305155234e-05\\
6714.06759558984	2.19938771828881e-05\\
6663.19437993603	2.23653633904375e-05\\
6612.70663613428	2.27447585991738e-05\\
6562.60144342272	2.31322347257065e-05\\
6512.8759031704	2.35279661170912e-05\\
6463.52713870962	2.39321295229162e-05\\
6414.55229516949	2.43449040645603e-05\\
6365.9485393108	2.47664712015519e-05\\
6317.71305936208	2.5197014694951e-05\\
6269.84306485697	2.56367205676822e-05\\
6222.33578647276	2.60857770617416e-05\\
6175.18847587023	2.65443745922018e-05\\
6128.39840553459	2.70127056979406e-05\\
6081.96286861774	2.74909649890166e-05\\
6035.87917878165	2.79793490906178e-05\\
5990.14467004298	2.84780565835066e-05\\
5944.75669661882	2.89872879408906e-05\\
5899.71263277365	2.95072454616429e-05\\
5855.00987266743	3.0038133199802e-05\\
5810.64583020485	3.05801568902799e-05\\
5766.61793888571	3.11335238707106e-05\\
5722.92365165649	3.16984429993694e-05\\
5679.56044076296	3.22751245691008e-05\\
5636.52579760393	3.28637802171893e-05\\
5593.81723258619	3.34646228311123e-05\\
5551.43227498041	3.4077866450119e-05\\
5509.36847277828	3.47037261625762e-05\\
5467.6233925506	3.53424179990326e-05\\
5426.19461930653	3.59941588209474e-05\\
5385.07975635387	3.66591662050425e-05\\
5344.27642516044	3.73376583232301e-05\\
5303.78226521641	3.80298538180824e-05\\
5263.59493389784	3.87359716738049e-05\\
5223.71210633108	3.94562310826854e-05\\
5184.13147525829	4.01908513069914e-05\\
5144.85075090399	4.09400515362957e-05\\
5105.86766084255	4.17040507402124e-05\\
5067.1799498668	4.2483067516534e-05\\
5028.78537985746	4.3277319934762e-05\\
4990.68172965378	4.40870253750303e-05\\
4952.86679492495	4.49124003624287e-05\\
4915.33838804261	4.57536603967342e-05\\
4878.09433795431	4.66110197775698e-05\\
4841.1324900579	4.74846914250119e-05\\
4804.45070607688	4.83748866956781e-05\\
4768.0468639367	4.92818151943287e-05\\
4731.918857642	5.02056845810273e-05\\
4696.06459715477	5.11467003739073e-05\\
4660.48200827344	5.21050657476033e-05\\
4625.16903251291	5.30809813274095e-05\\
4590.12362698543	5.40746449792353e-05\\
4555.34376428243	5.50862515954382e-05\\
4520.82743235723	5.61159928766166e-05\\
4486.57263440865	5.71640571094565e-05\\
4452.57738876551	5.82306289407325e-05\\
4418.83972877192	5.93158891475698e-05\\
4385.3577026736	6.04200144040828e-05\\
4352.12937350489	6.15431770445124e-05\\
4319.15281897675	6.2685544822992e-05\\
4286.42613136551	6.38472806700803e-05\\
4253.94741740255	6.50285424462032e-05\\
4221.71479816475	6.62294826921593e-05\\
4189.72640896578	6.74502483768476e-05\\
4157.98039924825	6.86909806423798e-05\\
4126.47493247662	6.99518145467547e-05\\
4095.20818603102	7.12328788042688e-05\\
4064.17835110174	7.25342955238542e-05\\
4033.38363258461	7.38561799455279e-05\\
4002.82224897718	7.51986401751572e-05\\
3972.49243227562	7.65617769177369e-05\\
3942.39242787247	7.79456832093906e-05\\
3912.52049445511	7.93504441483045e-05\\
3882.87490390504	8.07761366248124e-05\\
3853.45394119789	8.22228290508494e-05\\
3824.25590430422	8.36905810889979e-05\\
3795.27910409107	8.51794433813504e-05\\
3766.52186422418	8.66894572784159e-05\\
3737.9825210711	8.82206545682966e-05\\
3709.65942360489	8.97730572063655e-05\\
3681.55093330861	9.13466770456696e-05\\
3653.65542408053	9.29415155682837e-05\\
3625.97128214009	9.45575636178425e-05\\
3598.49690593453	9.61948011334642e-05\\
3571.23070604618	9.78531968852883e-05\\
3544.17110510062	9.95327082118277e-05\\
3517.31653767532	0.000101233280759347\\
3490.66545020915	0.000102954848223448\\
3464.21630091246	0.00010469733209306\\
3437.96755967793	0.000106460641396987\\
3411.91770799201	0.000108244672453188\\
3386.06523884712	0.000110049308620919\\
3360.4086566544	0.000111874420055862\\
3334.94647715724	0.000113719863468345\\
3309.67722734543	0.000115585481884738\\
3284.59944536989	0.000117471104412071\\
3259.71168045813	0.000119376546005906\\
3235.01249283033	0.000121301607241461\\
3210.50045361603	0.000123246074087936\\
3186.17414477149	0.000125209717685976\\
3162.03215899761	0.000127192294128135\\
3138.07309965858	0.000129193544242198\\
3114.29558070104	0.000131213193377103\\
3090.6982265739	0.000133250951191235\\
3067.27967214876	0.00013530651144272\\
3044.03856264098	0.000137379551781325\\
3020.97355353122	0.000139469733541506\\
2998.08331048777	0.000141576701536019\\
2975.36650928923	0.000143700083849484\\
2952.82183574802	0.000145839491631151\\
2930.44798563427	0.000147994518886055\\
2908.24366460042	0.000150164742263615\\
2886.20758810631	0.000152349720842651\\
2864.33848134488	0.000154548995911652\\
2842.63507916843	0.000156762090743034\\
2821.09612601541	0.000158988510359992\\
2799.72037583779	0.000161227741294434\\
2778.50659202901	0.000163479251334379\\
2757.45354735239	0.00016574248925909\\
2736.56002387015	0.000168016884560117\\
2715.82481287296	0.000170301847146364\\
2695.24671481002	0.000172596767031266\\
2674.82453921964	0.000174901014000152\\
2654.55710466042	0.00017721393725598\\
2634.44323864282	0.000179534865041753\\
2614.48177756143	0.000181863104238232\\
2594.67156662757	0.000184197939935943\\
2575.01145980252	0.000186538634981103\\
2555.50031973125	0.000188884429495883\\
2536.13701767655	0.000191234540374559\\
2516.9204334538	0.000193588160758527\\
2497.84945536614	0.00019594445949506\\
2478.92298014014	0.000198302580587015\\
2460.13991286201	0.000200661642643644\\
2441.49916691423	0.000203020738346275\\
2422.99966391271	0.000205378933946902\\
2404.64033364438	0.000207735268822829\\
2386.42011400529	0.000210088755116334\\
2368.33795093917	0.000212438377494908\\
2350.39279837646	0.000214783093074704\\
2332.58361817375	0.000217121831557266\\
2314.9093800538	0.000219453495636851\\
2297.36906154586	0.000221776961742133\\
2279.96164792655	0.000224091081180903\\
2262.68613216118	0.000226394681758505\\
2245.54151484545	0.000228686569938887\\
2228.52680414767	0.000230965533609902\\
2211.64101575135	0.000233230345500577\\
2194.88317279829	0.00023547976727637\\
2178.25230583203	0.000237712554308459\\
2161.74745274181	0.000239927461075242\\
2145.36765870687	0.000242123247110005\\
2129.11197614124	0.000244298683361152\\
2112.9794646389	0.000246452558784676\\
2096.9691909194	0.000248583686948199\\
2081.08022877384	0.000250690912397572\\
2065.31165901132	0.000252773116526033\\
2049.66256940575	0.00025482922269596\\
2034.13205464308	0.000256858200395532\\
2018.71921626889	0.000258859068265276\\
2003.42316263648	0.000260830895897598\\
1988.24300885525	0.000262772804388775\\
1973.17787673949	0.000264683965698974\\
1958.22689475763	0.000266563600943559\\
1943.38919798175	0.000268410977791604\\
1928.66392803762	0.000270225407181197\\
1914.05023305497	0.000272006239574658\\
1899.54726761828	0.000273752860971694\\
1885.15419271781	0.000275464688878461\\
1870.87017570109	0.000277141168399955\\
1856.69439022475	0.000278781768587023\\
1842.62601620672	0.000280385979131935\\
1828.66423977876	0.000281953307471175\\
1814.80825323942	0.000283483276323292\\
1801.05725500729	0.000284975421664458\\
1787.41044957462	0.000286429291125255\\
1773.8670474613	0.000287844442778704\\
1760.42626516921	0.000289220444281174\\
1747.08732513687	0.000290556872323474\\
1733.84945569449	0.000291853312348284\\
1720.71189101928	0.000293109358491324\\
1707.67387109119	0.000294324613706312\\
1694.73464164889	0.000295498690037462\\
1681.89345414622	0.000296631209007349\\
1669.1495657088	0.000297721802092178\\
1656.50223909109	0.000298770111260587\\
1643.95074263376	0.000299775789555854\\
1631.49435022133	0.00030073850170487\\
1619.13234124017	0.000301657924740207\\
1606.86400053683	0.00030253374862427\\
1594.68861837664	0.000303365676866734\\
1582.60549040267	0.000304153427128358\\
1570.613917595	0.000304896731805839\\
1558.71320623023	0.000305595338593638\\
1546.90266784138	0.000306249011019784\\
1535.18161917809	0.000306857528953487\\
1523.54938216702	0.000307420689083075\\
1512.0052838727	0.00030793830536332\\
1500.54865645854	0.00030841020943161\\
1489.17883714823	0.000308836250992773\\
1477.89516818739	0.000309216298172605\\
1466.69699680551	0.000309550237840325\\
1455.5836751782	0.000309837975900343\\
1444.5545603897	0.000310079437553817\\
1433.6090143957	0.000310274567530534\\
1422.7464039864	0.000310423330291729\\
1411.96610074992	0.000310525710204444\\
1401.26748103591	0.000310581711688092\\
1390.64992591951	0.000310591359333858\\
1380.1128211655	0.000310554697997586\\
1369.65555719278	0.000310471792866808\\
1359.27752903913	0.000310342729502535\\
1348.9781363262	0.000310167613856436\\
1338.75678322474	0.000309946572264029\\
1328.61287842018	0.000309679751414466\\
1318.54583507842	0.000309367318297519\\
1308.55507081185	0.00030900946012833\\
1298.64000764567	0.000308606384250503\\
1288.80007198445	0.000308158318018092\\
1279.03469457899	0.000307665508657034\\
1269.34331049332	0.000307128223106583\\
1259.72535907206	0.00030654674784126\\
1250.18028390797	0.000305921388673888\\
1240.70753280979	0.000305252470540207\\
1231.30655777025	0.000304540337265626\\
1221.97681493439	0.000303785351314622\\
1212.71776456812	0.000302987893523331\\
1203.52887102695	0.000302148362815838\\
1194.40960272504	0.000301267175904715\\
1185.35943210443	0.000300344766976322\\
1176.37783560452	0.000299381587361422\\
1167.46429363178	0.000298378105191621\\
1158.61829052972	0.000297334805042201\\
1149.839314549	0.000296252187561859\\
1141.1268578179	0.000295130769089917\\
1132.48041631286	0.000293971081261542\\
1123.89948982939	0.000292773670601532\\
1115.38358195309	0.000291539098107216\\
1106.93220003095	0.000290267938821033\\
1098.54485514283	0.000288960781393345\\
1090.22106207322	0.000287618227636055\\
1081.96033928312	0.00028624089206758\\
1073.7622088822	0.000284829401449767\\
1065.62619660117	0.000283384394317306\\
1057.55183176433	0.000281906520500214\\
1049.53864726231	0.000280396440639963\\
1041.58617952513	0.000278854825699826\\
1033.69396849529	0.000277282356470001\\
1025.86155760121	0.000275679723068096\\
1018.08849373079	0.000274047624435534\\
1010.37432720523	0.000272386767830448\\
1002.71861175299	0.000270697868317634\\
995.120904483953	0.000268981648256113\\
987.580765863859	0.000267238836784868\\
980.097759688832	0.000265470169307302\\
972.671453060162	0.000263676386974971\\
965.301416359258	0.000261858236171122\\
957.987223222801	0.000260016467994586\\
950.728450518066	0.00025815183774454\\
943.524678318455	0.00025626510440667\\
936.375489879197	0.000254357030141234\\
929.280471613241	0.000252428379773546\\
922.239213067332	0.000250479920287357\\
915.251306898262	0.00024851242032163\\
908.316348849305	0.000246526649671175\\
901.433937726837	0.000244523378791613\\
894.603675377116	0.000242503378309114\\
887.825166663254	0.000240467418535349\\
881.098019442364	0.000238416268988088\\
874.421844542861	0.000236350697917855\\
867.79625574196	0.000234271471841034\\
861.220869743325	0.000232179355079826\\
854.695306154899	0.000230075109309416\\
848.219187466894	0.000227959493112723\\
841.792139029961	0.000225833261543061\\
835.413789033501	0.000223697165695056\\
829.083768484172	0.000221551952284127\\
822.801711184529	0.000219398363234819\\
816.567253711847	0.000217237135278294\\
810.380035397094	0.000215068999559232\\
804.239698304065	0.000212894681252394\\
798.145887208678	0.000210714899189097\\
792.098249578423	0.000208530365493805\\
786.096435551963	0.000206341785231056\\
780.140097918901	0.000204149856062904\\
774.228892099692	0.000201955267917056\\
768.362476125704	0.000199758702665863\\
762.540510619441	0.000197560833816299\\
756.762658774907	0.000195362326211067\\
751.028586338118	0.000193163835740934\\
745.337961587772	0.000190966009068399\\
739.690455316055	0.000188769483362774\\
734.085740809593	0.000186574886046742\\
728.52349383056	0.000184382834554447\\
723.00339259791	0.000182193936101159\\
717.525117768769	0.000180008787464529\\
712.088352419958	0.000177827974777459\\
706.692782029658	0.000175652073332566\\
701.33809445922	0.000173481647398245\\
696.023979935099	0.000171317250046286\\
690.750131030938	0.000169159422991014\\
685.516242649784	0.000167008696439898\\
680.322012006438	0.000164865588955561\\
675.167138609933	0.000162730607329123\\
670.051324246158	0.000160604246464779\\
664.974272960603	0.000158486989275528\\
659.935691041233	0.000156379306589933\\
654.935287001506	0.000154281657069811\\
649.972771563501	0.000152194487138704\\
645.04785764119	0.000150118230921022\\
640.160260323826	0.000148053310191689\\
635.30969685946	0.000146000134336158\\
630.495886638585	0.000143959100320626\\
625.718551177906	0.000141930592672285\\
620.977414104223	0.000139914983469425\\
616.272201138446	0.000137912632341229\\
611.602640079728	0.000135923886477042\\
606.968460789717	0.000133949080644951\\
602.369395176928	0.000131988537219458\\
597.805177181235	0.000130042566218042\\
593.275542758476	0.000128111465346412\\
588.780229865185	0.000126195520052229\\
584.318978443423	0.00012429500358708\\
579.891530405739	0.000122410177076486\\
575.497629620241	0.00012054128959772\\
571.137021895773	0.000118688578265197\\
566.809454967215	0.000116852268323221\\
562.514678480885	0.000115032573245848\\
558.25244398006	0.000113229694843628\\
554.0225048906	0.00011144382337699\\
549.824616506682	0.000109675137676046\\
545.658535976647	0.000107923805266546\\
541.524022288951	0.000106189982501778\\
537.420836258218	0.00010447381470014\\
533.348740511404	0.000102775436288174\\
529.307499474071	0.000101094970948793\\
525.296879356751	9.94325317744813e-05\\
521.316648141422	9.77882214252139e-05\\
517.366575568092	9.61621322908689e-05\\
513.446433121472	9.455434665788e-05\\
509.555994017757	9.29649368799042e-05\\
505.695033191508	9.13939655522619e-05\\
501.863327282632	8.98414856899183e-05\\
498.060654623459	8.83075409087743e-05\\
494.286795225918	8.67921656100336e-05\\
490.541530768814	8.52953851674243e-05\\
486.824644585191	8.38172161170437e-05\\
483.135921649807	8.23576663496086e-05\\
479.475148566685	8.09167353048872e-05\\
475.842113556776	7.94944141681017e-05\\
472.236606445701	7.80906860680811e-05\\
468.658418651595	7.67055262769614e-05\\
465.107343173043	7.53389024112221e-05\\
461.5831745771	7.39907746338563e-05\\
458.08570898741	7.26610958574786e-05\\
454.614744072408	7.13498119481719e-05\\
451.170079033618	7.00568619298836e-05\\
447.751514594037	6.87821781891845e-05\\
444.358852986605	6.75256866802071e-05\\
440.991897942762	6.62873071295831e-05\\
437.6504546811	6.5066953241209e-05\\
434.334329896089	6.38645329006672e-05\\
431.043331746893	6.26799483791385e-05\\
427.77726984628	6.15130965366459e-05\\
424.535955249599	6.03638690244709e-05\\
421.319200443854	5.92321524865941e-05\\
418.126819336855	5.81178287600108e-05\\
414.958627246454	5.7020775073779e-05\\
411.814440889858	5.5940864246664e-05\\
408.694078373027	5.48779648832454e-05\\
405.597359180154	5.38319415683563e-05\\
402.524104163219	5.2802655059734e-05\\
399.474135531625	5.17899624787595e-05\\
396.447276841914	5.07937174991728e-05\\
393.443352987558	4.98137705336528e-05\\
390.462190188833	4.88499689181542e-05\\
387.503615982761	4.79021570939011e-05\\
384.567459213134	4.6970176786938e-05\\
381.653550020615	4.60538671851435e-05\\
378.761719832906	4.51530651126194e-05\\
375.891801355004	4.42676052013656e-05\\
373.043628559514	4.3397320060159e-05\\
370.217036677053	4.25420404405597e-05\\
367.41186218671	4.17015953999643e-05\\
364.627942806591	4.08758124616386e-05\\
361.865117484429	4.00645177716568e-05\\
359.123226388267	3.92675362526832e-05\\
356.402110897212	3.84846917545313e-05\\
353.701613592262	3.77158072014416e-05\\
351.02157824719	3.69607047360182e-05\\
348.361849819517	3.62192058597706e-05\\
345.722274441534	3.54911315702063e-05\\
343.102699411407	3.47763024944249e-05\\
340.502973184339	3.40745390191636e-05\\
337.922945363805	3.33856614172511e-05\\
335.362466692849	3.27094899704225e-05\\
332.821389045452	3.20458450884564e-05\\
330.299565417964	3.1394547424594e-05\\
327.796849920595	3.07554179872023e-05\\
325.313097768977	3.01282782476486e-05\\
322.848165275792	2.95129502443518e-05\\
320.401909842454	2.89092566829828e-05\\
317.974189950861	2.83170210327863e-05\\
315.564865155213	2.77360676190001e-05\\
313.173796073877	2.71662217113513e-05\\
310.800844381337	2.66073096086119e-05\\
308.445872800177	2.60591587192004e-05\\
306.108745093152	2.55215976378198e-05\\
303.789326055299	2.49944562181271e-05\\
301.487481506119	2.44775656414322e-05\\
299.203078281811	2.39707584814327e-05\\
296.935984227572	2.34738687649934e-05\\
294.686068189952	2.29867320289856e-05\\
292.453200009261	2.2509185373209e-05\\
290.237250512045	2.20410675094231e-05\\
288.038091503612	2.15822188065245e-05\\
285.855595760614	2.11324813319106e-05\\
283.689637023688	2.06916988890795e-05\\
281.540089990152	2.02597170515235e-05\\
279.406830306757	1.98363831929801e-05\\
277.28973456249	1.94215465141124e-05\\
275.188680281438	1.90150580657007e-05\\
273.103545915701	1.86167707684317e-05\\
271.03421083836	1.82265394293813e-05\\
268.980555336501	1.78442207552955e-05\\
266.942460604284	1.7469673362779e-05\\
264.919808736078	1.71027577855086e-05\\
262.912482719633	1.67433364785972e-05\\
260.920366429315	1.63912738202363e-05\\
258.943344619384	1.60464361107549e-05\\
256.981302917332	1.57086915692332e-05\\
255.034127817263	1.53779103278173e-05\\
253.101706673328	1.50539644238816e-05\\
251.183927693206	1.47367277901897e-05\\
249.28067993164	1.44260762432061e-05\\
247.391853284017	1.4121887469712e-05\\
245.517338479998	1.38240410118773e-05\\
243.657027077199	1.35324182509416e-05\\
241.810811454912	1.32469023896533e-05\\
239.978584807887	1.29673784336167e-05\\
238.160241140146	1.2693733171688e-05\\
236.355675258854	1.2425855155562e-05\\
234.564782768236	1.21636346786853e-05\\
232.787460063531	1.19069637546213e-05\\
231.023604325008	1.16557360949951e-05\\
229.273113512007	1.14098470871299e-05\\
227.535886357044	1.11691937714865e-05\\
225.811822359948	1.09336748190069e-05\\
224.100821782051	1.0703190508454e-05\\
222.402785640412	1.04776427038355e-05\\
220.717615702098	1.02569348319868e-05\\
219.045214478497	1.00409718603832e-05\\
217.385485219676	9.82966027524329e-06\\
215.73833190879	9.62290805997432e-06\\
214.103659256522	9.4206246740068e-06\\
212.481372695573	9.22272103205499e-06\\
210.871378375191	9.02910948383385e-06\\
209.273583155741	8.83970379425568e-06\\
207.687894603317	8.65441912412374e-06\\
206.114220984395	8.47317201133351e-06\\
204.552471260524	8.29588035258656e-06\\
203.002555083066	8.12246338561752e-06\\
201.464382787958	7.95284167192891e-06\\
199.937865390536	7.78693708002534e-06\\
198.422914580381	7.62467276913443e-06\\
196.919442716211	7.46597317339815e-06\\
195.427362820812	7.31076398651641e-06\\
193.946588576005	7.15897214682182e-06\\
192.477034317655	7.01052582276222e-06\\
191.018615030712	6.86535439876737e-06\\
189.571246344294	6.72338846147359e-06\\
188.134844526807	6.58455978628044e-06\\
186.7093264811	6.44880132421279e-06\\
185.294609739659	6.31604718906112e-06\\
183.890612459833	6.18623264477387e-06\\
182.497253419104	6.05929409307515e-06\\
181.114452010385	5.93516906128251e-06\\
179.742128237356	5.81379619029953e-06\\
178.380202709841	5.6951152227598e-06\\
177.02859663921	5.57906699129902e-06\\
175.687231833822	5.46559340693391e-06\\
174.356030694505	5.35463744752773e-06\\
173.034916210062	5.24614314632296e-06\\
171.723811952819	5.14005558052424e-06\\
170.422642074203	5.03632085991505e-06\\
169.131331300355	4.93488611549362e-06\\
167.84980492777	4.83569948811491e-06\\
166.577988818986	4.73871011712687e-06\\
165.315809398282	4.64386812899064e-06\\
164.063193647433	4.55112462587578e-06\\
162.820069101478	4.46043167422287e-06\\
161.586363844532	4.3717422932672e-06\\
160.362006505625	4.28501044351845e-06\\
159.146926254571	4.20019101519237e-06\\
157.941052797874	4.11723981659184e-06\\
156.744316374658	4.03611356243534e-06\\
155.556647752633	3.95676986213224e-06\\
154.37797822409	3.87916720800495e-06\\
153.208239601924	3.80326496345893e-06\\
152.047364215693	3.72902335110235e-06\\
150.895284907701	3.65640344081776e-06\\
149.751935029113	3.58536713778915e-06\\
148.617248436097	3.5158771704877e-06\\
147.491159486005	3.44789707862071e-06\\
146.373603033566	3.3813912010483e-06\\
145.264514427124	3.31632466367288e-06\\
144.163829504894	3.25266336730688e-06\\
143.071484591254	3.19037397552433e-06\\
141.987416493056	3.1294239025023e-06\\
140.911562495976	3.06978130085835e-06\\
139.843860360883	3.01141504949014e-06\\
138.784248320236	2.95429474142381e-06\\
137.732665074517	2.89839067167738e-06\\
136.689049788679	2.84367382514581e-06\\
135.653342088628	2.79011586451414e-06\\
134.625482057733	2.73768911820537e-06\\
133.605410233356	2.68636656836925e-06\\
132.593067603415	2.63612183891858e-06\\
131.58839560297	2.58692918361915e-06\\
130.59133611083	2.53876347423949e-06\\
129.601831446199	2.49160018876653e-06\\
128.619824365331	2.44541539969275e-06\\
127.645258058225	2.40018576238091e-06\\
126.678076145334	2.35588850351156e-06\\
125.718222674305	2.31250140961864e-06\\
124.765642116742	2.27000281571855e-06\\
123.820279364994	2.22837159403735e-06\\
122.882079728965	2.1875871428409e-06\\
121.950988932956	2.14762937537235e-06\\
121.026953112515	2.10847870890137e-06\\
120.109918811333	2.07011605388898e-06\\
119.199832978139	2.03252280327197e-06\\
118.296642963642	1.99568082187037e-06\\
117.400296517478	1.9595724359213e-06\\
116.51074178519	1.92418042274254e-06\\
115.627927305229	1.88948800052847e-06\\
114.751802005974	1.85547881828122e-06\\
113.882315202781	1.82213694587948e-06\\
113.019416595047	1.78944686428722e-06\\
112.163056263304	1.75739345590428e-06\\
111.313184666327	1.72596199506086e-06\\
110.469752638272	1.6951381386573e-06\\
109.632711385831	1.6649079169508e-06\\
108.802012485405	1.63525772449018e-06\\
107.977607880307	1.60617431119979e-06\\
107.159449877983	1.57764477361347e-06\\
106.347491147248	1.54965654625919e-06\\
105.54168471555	1.52219739319494e-06\\
104.741983966257	1.49525539969627e-06\\
103.948342635952	1.46881896409555e-06\\
103.160714811762	1.44287678977317e-06\\
102.379054928703	1.41741787730047e-06\\
}--cycle;
\addplot[ybar interval, fill=black, fill opacity=0.35, area legend, draw=none] table[row sep=crcr, x=Lower, y=Count] {%
Lower	Upper	Count\\
102.379054928703	140.510222226835	0.000139868084594268\\
140.510222226835	192.843375668819	0.000152866767504631\\
192.843375668819	264.668057241483	0.00025989209082371\\
264.668057241483	363.243903406255	0.000216415421863832\\
363.243903406255	498.534408485211	0.000197106712338045\\
498.534408485211	684.21397885303	0.000387764792095182\\
684.21397885303	939.050065331212	0.000334855767535259\\
939.050065331212	1288.80007198445	0.000396473664128183\\
1288.80007198445	1768.81477023409	0.000244436959453925\\
1768.81477023409	2427.61135680322	0.000165959167916637\\
2427.61135680322	3331.77730017487	0.000132719000178792\\
3331.77730017487	4572.70062889246	7.09149372596145e-05\\
4572.70062889246	6275.80692154186	3.28810951152584e-05\\
6275.80692154186	8613.23662161811	2.05353769563355e-05\\
8613.23662161811	11821.244029247	9.14378603477561e-06\\
11821.244029247	16224.0765623776	4.84536560789068e-06\\
16224.0765623776	22266.7478693998	1.32391778296847e-06\\
22266.7478693998	30560.0173158182	6.43091770717349e-07\\
30560.0173158182	41942.1221195282	1.1714294994883e-06\\
41942.1221195282	57563.5016731115	0\\
57563.5016731115	79003.077513036	1.24380570146394e-07\\
79003.077513036	108427.841863662	0\\
108427.841863662	148811.885072089	6.60326815941545e-08\\
148811.885072089	204237	4.81129659385478e-08\\
204237	204237	4.81129659385478e-08\\
};
\addlegendentry{Istogramma reale}

\addplot [color=black]
table[row sep=crcr]{%
102.379054928703	4.49026988990107e-05\\
103.160714811762	4.56521826004552e-05\\
103.948342635952	4.64111338483195e-05\\
104.741983966257	4.71796096837295e-05\\
105.54168471555	4.7957666204809e-05\\
106.347491147248	4.87453585362334e-05\\
107.159449877983	4.95427407986033e-05\\
107.977607880307	5.03498660776466e-05\\
108.802012485405	5.11667863932574e-05\\
109.632711385831	5.19935526683833e-05\\
110.469752638272	5.28302146977698e-05\\
111.313184666327	5.36768211165714e-05\\
112.163056263304	5.45334193688418e-05\\
113.019416595047	5.54000556759109e-05\\
113.882315202781	5.62767750046614e-05\\
114.751802005974	5.71636210357158e-05\\
115.627927305229	5.80606361315445e-05\\
116.51074178519	5.89678613045048e-05\\
117.400296517478	5.98853361848276e-05\\
118.296642963642	6.0813098988557e-05\\
119.199832978139	6.17511864854591e-05\\
120.109918811333	6.26996339669119e-05\\
121.026953112515	6.36584752137868e-05\\
121.950988932956	6.46277424643367e-05\\
122.882079728965	6.56074663821011e-05\\
123.820279364994	6.65976760238437e-05\\
124.765642116742	6.75983988075334e-05\\
125.718222674305	6.86096604803844e-05\\
126.678076145334	6.96314850869662e-05\\
127.645258058225	7.06638949374003e-05\\
128.619824365331	7.1706910575655e-05\\
129.601831446199	7.27605507479529e-05\\
130.59133611083	7.38248323713073e-05\\
131.58839560297	7.48997705021989e-05\\
132.593067603415	7.59853783054088e-05\\
133.605410233356	7.70816670230228e-05\\
134.625482057733	7.81886459436217e-05\\
135.653342088628	7.93063223716701e-05\\
136.689049788679	8.04347015971241e-05\\
137.732665074517	8.1573786865266e-05\\
138.784248320236	8.27235793467883e-05\\
139.843860360883	8.38840781081366e-05\\
140.911562495976	8.50552800821303e-05\\
141.987416493056	8.62371800388761e-05\\
143.071484591254	8.74297705569887e-05\\
144.163829504894	8.86330419951351e-05\\
145.264514427124	8.98469824639193e-05\\
146.373603033566	9.10715777981219e-05\\
147.491159486005	9.23068115293098e-05\\
148.617248436097	9.35526648588349e-05\\
149.751935029113	9.48091166312355e-05\\
150.895284907701	9.60761433080552e-05\\
152.047364215693	9.73537189420993e-05\\
153.208239601924	9.86418151521405e-05\\
154.37797822409	9.9940401098094e-05\\
155.556647752633	0.000101249443456674\\
156.744316374658	0.000102568906397548\\
157.941052797874	0.000103898751560015\\
159.146926254571	0.000105238938030202\\
160.362006505625	0.000106589422318818\\
161.586363844532	0.000107950158339468\\
162.820069101478	0.000109321097387547\\
164.063193647433	0.000110702188119726\\
165.315809398282	0.000112093376534051\\
166.577988818986	0.000113494605950668\\
167.84980492777	0.000114905816993193\\
169.131331300355	0.00011632694757073\\
170.422642074203	0.000117757932860573\\
171.723811952819	0.000119198705291587\\
173.034916210062	0.000120649194528297\\
174.356030694505	0.000122109327455693\\
175.687231833822	0.000123579028164765\\
177.02859663921	0.000125058217938793\\
178.380202709841	0.00012654681524039\\
179.742128237356	0.000128044735699323\\
181.114452010385	0.000129551892101131\\
182.497253419104	0.000131068194376538\\
183.890612459833	0.00013259354959169\\
185.294609739659	0.000134127861939215\\
186.7093264811	0.000135671032730142\\
188.134844526807	0.000137222960386661\\
189.571246344294	0.000138783540435765\\
191.018615030712	0.000140352665503769\\
192.477034317655	0.000141930225311725\\
193.946588576005	0.000143516106671747\\
195.427362820812	0.000145110193484249\\
196.919442716211	0.000146712366736121\\
198.422914580381	0.000148322504499836\\
199.937865390536	0.000149940481933518\\
201.464382787958	0.000151566171281971\\
203.002555083066	0.000153199441878671\\
204.552471260524	0.000154840160148757\\
206.114220984395	0.000156488189613001\\
207.687894603317	0.000158143390892779\\
209.273583155741	0.000159805621716062\\
210.871378375191	0.000161474736924414\\
212.481372695573	0.000163150588481024\\
214.103659256522	0.000164833025479765\\
215.73833190879	0.000166521894155296\\
217.385485219676	0.000168217037894218\\
219.045214478497	0.000169918297247272\\
220.717615702098	0.000171625509942607\\
222.402785640412	0.000173338510900108\\
224.100821782051	0.000175057132246798\\
225.811822359948	0.000176781203333309\\
227.535886357044	0.000178510550751442\\
229.273113512007	0.0001802449983528\\
231.023604325008	0.000181984367268515\\
232.787460063531	0.000183728475930058\\
234.564782768236	0.000185477140091145\\
236.355675258854	0.000187230172850735\\
238.160241140146	0.000188987384677119\\
239.978584807887	0.000190748583433114\\
241.810811454912	0.000192513574402336\\
243.657027077199	0.000194282160316592\\
245.517338479998	0.000196054141384342\\
247.391853284017	0.000197829315320275\\
249.28067993164	0.000199607477375965\\
251.183927693206	0.000201388420371622\\
253.101706673328	0.000203171934728931\\
255.034127817263	0.000204957808504972\\
256.981302917332	0.000206745827427219\\
258.943344619384	0.000208535774929625\\
260.920366429315	0.000210327432189761\\
262.912482719633	0.000212120578167038\\
264.919808736078	0.000213914989641977\\
266.942460604284	0.000215710441256539\\
268.980555336501	0.000217506705555496\\
271.03421083836	0.000219303553028847\\
273.103545915701	0.000221100752155257\\
275.188680281438	0.000222898069446528\\
277.28973456249	0.000224695269493071\\
279.406830306757	0.000226492115010392\\
281.540089990152	0.000228288366886564\\
283.689637023688	0.000230083784230687\\
285.855595760614	0.000231878124422309\\
288.038091503612	0.000233671143161819\\
290.237250512045	0.000235462594521775\\
292.453200009261	0.000237252230999172\\
294.686068189952	0.000239039803568628\\
296.935984227572	0.000240825061736479\\
299.203078281811	0.000242607753595757\\
301.487481506119	0.000244387625882059\\
303.789326055299	0.000246164424030258\\
306.108745093152	0.000247937892232078\\
308.445872800177	0.000249707773494486\\
310.800844381337	0.000251473809698895\\
313.173796073877	0.000253235741661166\\
315.564865155213	0.000254993309192381\\
317.974189950861	0.000256746251160377\\
320.401909842454	0.000258494305552014\\
322.848165275792	0.00026023720953616\\
325.313097768977	0.000261974699527383\\
327.796849920595	0.000263706511250311\\
330.299565417964	0.000265432379804656\\
332.821389045452	0.000267152039730867\\
335.362466692849	0.000268865225076405\\
337.922945363805	0.000270571669462602\\
340.502973184339	0.000272271106152096\\
343.102699411407	0.000273963268116806\\
345.722274441534	0.000275647888106434\\
348.361849819517	0.000277324698717464\\
351.02157824719	0.000278993432462634\\
353.701613592262	0.000280653821840864\\
356.402110897212	0.000282305599407601\\
359.123226388267	0.000283948497845573\\
361.865117484429	0.000285582250035906\\
364.627942806591	0.000287206589129602\\
367.41186218671	0.000288821248619331\\
370.217036677053	0.000290425962411521\\
373.043628559514	0.000292020464898716\\
375.891801355004	0.000293604491032177\\
378.761719832906	0.000295177776394697\\
381.653550020615	0.000296740057273604\\
384.567459213134	0.00029829107073392\\
387.503615982761	0.000299830554691654\\
390.462190188833	0.000301358247987193\\
393.443352987558	0.000302873890458778\\
396.447276841914	0.00030437722301601\\
399.474135531625	0.000305867987713384\\
402.524104163219	0.00030734592782381\\
405.597359180154	0.000308810787912086\\
408.694078373027	0.00031026231390831\\
411.814440889858	0.00031170025318118\\
414.958627246454	0.000313124354611176\\
418.126819336855	0.00031453436866358\\
421.319200443854	0.000315930047461301\\
424.535955249599	0.000317311144857493\\
427.77726984628	0.000318677416507915\\
431.043331746893	0.000320028619943021\\
434.334329896089	0.000321364514639739\\
437.6504546811	0.000322684862092915\\
440.991897942762	0.000323989425886389\\
444.358852986605	0.000325277971763679\\
447.751514594037	0.000326550267698233\\
451.170079033618	0.000327806083963235\\
454.614744072408	0.000329045193200916\\
458.08570898741	0.000330267370491362\\
461.5831745771	0.000331472393420773\\
465.107343173043	0.000332660042149152\\
468.658418651595	0.000333830099477394\\
472.236606445701	0.000334982350913746\\
475.842113556776	0.000336116584739608\\
479.475148566685	0.000337232592074651\\
483.135921649807	0.00033833016694122\\
486.824644585191	0.000339409106327993\\
490.541530768814	0.000340469210252874\\
494.286795225918	0.000341510281825087\\
498.060654623459	0.00034253212730645\\
501.863327282632	0.000343534556171794\\
505.695033191508	0.00034451738116851\\
509.555994017757	0.000345480418375189\\
513.446433121472	0.000346423487259336\\
517.366575568092	0.000347346410734131\\
521.316648141422	0.00034824901521421\\
525.296879356751	0.000349131130670439\\
529.307499474071	0.000349992590683665\\
533.348740511404	0.000350833232497416\\
537.420836258218	0.000351652897069521\\
541.524022288951	0.000352451429122634\\
545.658535976647	0.000353228677193639\\
549.824616506682	0.000353984493681915\\
554.0225048906	0.000354718734896424\\
558.25244398006	0.000355431261101634\\
562.514678480885	0.000356121936562214\\
566.809454967215	0.000356790629586525\\
571.137021895773	0.000357437212568843\\
575.497629620241	0.000358061562030339\\
579.891530405739	0.000358663558658757\\
584.318978443423	0.000359243087346808\\
588.780229865185	0.000359800037229229\\
593.275542758476	0.000360334301718521\\
597.805177181235	0.00036084577853933\\
602.369395176928	0.000361334369761458\\
606.968460789717	0.000361799981831495\\
611.602640079728	0.000362242525603061\\
616.272201138446	0.000362661916365622\\
620.977414104223	0.000363058073871904\\
625.718551177906	0.00036343092236385\\
630.495886638585	0.000363780390597148\\
635.30969685946	0.000364106411864287\\
640.160260323826	0.000364408924016152\\
645.04785764119	0.000364687869482135\\
649.972771563501	0.000364943195288764\\
654.935287001506	0.000365174853076826\\
659.935691041233	0.000365382799116995\\
664.974272960603	0.000365566994323945\\
670.051324246158	0.000365727404268938\\
675.167138609933	0.000365863999190905\\
680.322012006438	0.000365976754005978\\
685.516242649784	0.000366065648315503\\
690.750131030938	0.000366130666412514\\
696.023979935099	0.00036617179728666\\
701.33809445922	0.000366189034627598\\
706.692782029658	0.000366182376826839\\
712.088352419958	0.000366151826978052\\
717.525117768769	0.000366097392875819\\
723.00339259791	0.000366019087012848\\
728.52349383056	0.00036591692657565\\
734.085740809593	0.000365790933438665\\
739.690455316055	0.000365641134156857\\
745.337961587772	0.000365467559956774\\
751.028586338118	0.000365270246726076\\
756.762658774907	0.000365049235001536\\
762.540510619441	0.000364804569955529\\
768.362476125704	0.000364536301381\\
774.228892099692	0.000364244483674927\\
780.140097918901	0.000363929175820286\\
786.096435551963	0.000363590441366526\\
792.098249578423	0.000363228348408563\\
798.145887208678	0.000362842969564296\\
804.239698304065	0.000362434381950673\\
810.380035397094	0.000362002667158291\\
816.567253711847	0.000361547911224572\\
822.801711184529	0.000361070204605503\\
829.083768484172	0.000360569642145965\\
835.413789033501	0.000360046323048659\\
841.792139029961	0.000359500350841653\\
848.219187466894	0.000358931833344549\\
854.695306154899	0.000358340882633302\\
861.220869743325	0.000357727615003691\\
867.79625574196	0.000357092150933476\\
874.421844542861	0.000356434615043241\\
881.098019442364	0.000355755136055951\\
887.825166663254	0.000355053846755247\\
894.603675377116	0.000354330883942473\\
901.433937726837	0.000353586388392489\\
908.316348849305	0.00035282050480826\\
915.251306898262	0.00035203338177426\\
922.239213067332	0.000351225171708704\\
929.280471613241	0.00035039603081463\\
936.375489879197	0.000349546119029854\\
943.524678318455	0.000348675599975825\\
950.728450518066	0.000347784640905389\\
957.987223222801	0.000346873412649503\\
965.301416359258	0.000345942089562912\\
972.671453060162	0.000344990849468812\\
980.097759688832	0.000344019873602531\\
987.580765863859	0.000343029346554246\\
995.120904483953	0.000342019456210765\\
1002.71861175299	0.000340990393696401\\
1010.37432720523	0.000339942353312957\\
1018.08849373079	0.000338875532478856\\
1025.86155760121	0.000337790131667437\\
1033.69396849529	0.000336686354344455\\
1041.58617952513	0.00033556440690479\\
1049.53864726231	0.000334424498608415\\
1057.55183176433	0.000333266841515645\\
1065.62619660117	0.00033209165042168\\
1073.7622088822	0.000330899142790497\\
1081.96033928312	0.000329689538688089\\
1090.22106207322	0.000328463060715109\\
1098.54485514283	0.000327219933938925\\
1106.93220003095	0.000325960385825126\\
1115.38358195309	0.000324684646168506\\
1123.89948982939	0.000323392947023559\\
1132.48041631286	0.000322085522634499\\
1141.1268578179	0.000320762609364862\\
1149.839314549	0.000319424445626688\\
1158.61829052972	0.000318071271809335\\
1167.46429363178	0.00031670333020795\\
1176.37783560452	0.000315320864951617\\
1185.35943210443	0.000313924121931221\\
1194.40960272504	0.000312513348727062\\
1203.52887102695	0.000311088794536229\\
1212.71776456812	0.00030965071009979\\
1221.97681493439	0.000308199347629801\\
1231.30655777025	0.000306734960736191\\
1240.70753280979	0.000305257804353525\\
1250.18028390797	0.000303768134667692\\
1259.72535907206	0.000302266209042548\\
1269.34331049332	0.000300752285946529\\
1279.03469457899	0.000299226624879278\\
1288.80007198445	0.000297689486298301\\
1298.64000764567	0.000296141131545701\\
1308.55507081185	0.000294581822774992\\
1318.54583507842	0.000293011822878044\\
1328.61287842018	0.00029143139541217\\
1338.75678322474	0.000289840804527403\\
1348.9781363262	0.000288240314893965\\
1359.27752903913	0.000286630191629979\\
1369.65555719278	0.000285010700229436\\
1380.1128211655	0.000283382106490452\\
1390.64992591951	0.000281744676443831\\
1401.26748103591	0.000280098676281982\\
1411.96610074992	0.000278444372288188\\
1422.7464039864	0.000276782030766274\\
1433.6090143957	0.000275111917970687\\
1444.5545603897	0.000273434300037026\\
1455.5836751782	0.000271749442913025\\
1466.69699680551	0.000270057612290043\\
1477.89516818739	0.000268359073535052\\
1489.17883714823	0.000266654091623179\\
1500.54865645854	0.000264942931070793\\
1512.0052838727	0.000263225855869191\\
1523.54938216702	0.000261503129418879\\
1535.18161917809	0.000259775014464485\\
1546.90266784138	0.000258041773030326\\
1558.71320623023	0.000256303666356637\\
1570.613917595	0.000254560954836493\\
1582.60549040267	0.000252813897953451\\
1594.68861837664	0.000251062754219908\\
1606.86400053683	0.000249307781116222\\
1619.13234124017	0.000247549235030594\\
1631.49435022133	0.000245787371199737\\
1643.95074263376	0.000244022443650352\\
1656.50223909109	0.000242254705141428\\
1669.1495657088	0.00024048440710737\\
1681.89345414622	0.000238711799601995\\
1694.73464164889	0.000236937131243387\\
1707.67387109119	0.000235160649159645\\
1720.71189101928	0.000233382598935524\\
1733.84945569449	0.000231603224559998\\
1747.08732513687	0.000229822768374743\\
1760.42626516921	0.000228041471023568\\
1773.8670474613	0.000226259571402794\\
1787.41044957462	0.000224477306612597\\
1801.05725500729	0.000222694911909337\\
1814.80825323942	0.000220912620658863\\
1828.66423977876	0.000219130664290824\\
1842.62601620672	0.000217349272253985\\
1856.69439022475	0.000215568671972564\\
1870.87017570109	0.000213789088803585\\
1885.15419271781	0.000212010745995277\\
1899.54726761828	0.000210233864646509\\
1914.05023305497	0.000208458663667273\\
1928.66392803762	0.000206685359740229\\
1943.38919798175	0.0002049141672833\\
1958.22689475763	0.000203145298413347\\
1973.17787673949	0.000201378962910909\\
1988.24300885525	0.000199615368186019\\
2003.42316263648	0.000197854719245106\\
2018.71921626889	0.000196097218658973\\
2034.13205464308	0.000194343066531869\\
2049.66256940575	0.000192592460471646\\
2065.31165901132	0.000190845595561008\\
2081.08022877384	0.000189102664329852\\
2096.9691909194	0.000187363856728701\\
2112.9794646389	0.000185629360103234\\
2129.11197614124	0.000183899359169901\\
2145.36765870687	0.000182174035992639\\
2161.74745274181	0.000180453569960674\\
2178.25230583203	0.000178738137767414\\
2194.88317279829	0.000177027913390428\\
2211.64101575135	0.000175323068072507\\
2228.52680414767	0.000173623770303816\\
2245.54151484545	0.000171930185805109\\
2262.68613216118	0.00017024247751203\\
2279.96164792655	0.000168560805560478\\
2297.36906154586	0.000166885327273034\\
2314.9093800538	0.000165216197146453\\
2332.58361817375	0.000163553566840199\\
2350.39279837646	0.000161897585166039\\
2368.33795093917	0.000160248398078666\\
2386.42011400529	0.000158606148667362\\
2404.64033364438	0.000156970977148682\\
2422.99966391271	0.000155343020860157\\
2441.49916691423	0.000153722414255007\\
2460.13991286201	0.000152109288897855\\
2478.92298014014	0.000150503773461429\\
2497.84945536614	0.000148905993724254\\
2516.9204334538	0.000147316072569304\\
2536.13701767655	0.00014573412998363\\
2555.50031973125	0.000144160283058932\\
2575.01145980252	0.000142594645993074\\
2594.67156662757	0.000141037330092531\\
2614.48177756143	0.000139488443775754\\
2634.44323864282	0.000137948092577443\\
2654.55710466042	0.000136416379153714\\
2674.82453921964	0.000134893403288151\\
2695.24671481002	0.00013337926189873\\
2715.82481287296	0.000131874049045599\\
2736.56002387015	0.000130377855939706\\
2757.45354735239	0.000128890770952254\\
2778.50659202901	0.000127412879624982\\
2799.72037583779	0.000125944264681248\\
2821.09612601541	0.000124485006037898\\
2842.63507916843	0.000123035180817923\\
2864.33848134488	0.000121594863363871\\
2886.20758810631	0.000120164125252008\\
2908.24366460042	0.000118743035307218\\
2930.44798563427	0.00011733165961862\\
2952.82183574802	0.000115930061555883\\
2975.36650928923	0.000114538301786245\\
2998.08331048777	0.000113156438292192\\
3020.97355353122	0.0001117845263898\\
3044.03856264098	0.000110422618747729\\
3067.27967214876	0.000109070765406833\\
3090.6982265739	0.00010772901380039\\
3114.29558070104	0.000106397408774926\\
3138.07309965858	0.000105075992611623\\
3162.03215899761	0.000103764805048285\\
3186.17414477149	0.000102463883301866\\
3210.50045361603	0.000101173262091518\\
3235.01249283033	9.98929736621686e-05\\
3259.71168045813	9.86230478085865e-05\\
3284.59944536989	9.73635118999479e-05\\
3309.67722734543	9.61143909048606e-05\\
3334.94647715724	9.48757074168477e-05\\
3360.4086566544	9.36474816802662e-05\\
3386.06523884712	9.24297316166483e-05\\
3411.91770799201	9.12224728514472e-05\\
3437.96755967793	9.00257187411737e-05\\
3464.21630091246	8.88394804009052e-05\\
3490.66545020915	8.76637667321533e-05\\
3517.31653767532	8.64985844510729e-05\\
3544.17110510062	8.53439381169968e-05\\
3571.23070604618	8.41998301612824e-05\\
3598.49690593453	8.30662609164514e-05\\
3625.97128214009	8.19432286456104e-05\\
3653.65542408053	8.08307295721336e-05\\
3681.55093330861	7.9728757909596e-05\\
3709.65942360489	7.86373058919374e-05\\
3737.9825210711	7.75563638038468e-05\\
3766.52186422418	7.6485920011348e-05\\
3795.27910409107	7.54259609925752e-05\\
3824.25590430422	7.43764713687195e-05\\
3853.45394119789	7.33374339351364e-05\\
3882.87490390504	7.23088296925959e-05\\
3912.52049445511	7.12906378786634e-05\\
3942.39242787247	7.02828359991957e-05\\
3972.49243227562	6.92853998599386e-05\\
4002.82224897718	6.82983035982125e-05\\
4033.38363258461	6.73215197146707e-05\\
4064.17835110174	6.63550191051201e-05\\
4095.20818603102	6.53987710923866e-05\\
4126.47493247662	6.44527434582162e-05\\
4157.98039924825	6.35169024751953e-05\\
4189.72640896578	6.25912129386799e-05\\
4221.71479816475	6.16756381987192e-05\\
4253.94741740255	6.07701401919625e-05\\
4286.42613136551	5.98746794735366e-05\\
4319.15281897675	5.89892152488816e-05\\
4352.12937350489	5.81137054055331e-05\\
4385.3577026736	5.72481065448392e-05\\
4418.83972877192	5.63923740136017e-05\\
4452.57738876551	5.55464619356289e-05\\
4486.57263440865	5.47103232431908e-05\\
4520.82743235723	5.38839097083626e-05\\
4555.34376428243	5.30671719742513e-05\\
4590.12362698543	5.2260059586089e-05\\
4625.16903251291	5.14625210221876e-05\\
4660.48200827344	5.06745037247417e-05\\
4696.06459715477	4.98959541304734e-05\\
4731.918857642	4.91268177011057e-05\\
4768.0468639367	4.83670389536577e-05\\
4804.45070607688	4.76165614905529e-05\\
4841.1324900579	4.68753280295304e-05\\
4878.09433795431	4.61432804333515e-05\\
4915.33838804261	4.54203597392928e-05\\
4952.86679492495	4.47065061884189e-05\\
4990.68172965378	4.40016592546247e-05\\
5028.78537985746	4.33057576734423e-05\\
5067.1799498668	4.26187394706038e-05\\
5105.86766084255	4.19405419903528e-05\\
5144.85075090399	4.12711019234975e-05\\
5184.13147525829	4.0610355335201e-05\\
5223.71210633108	3.99582376924985e-05\\
5263.59493389784	3.9314683891539e-05\\
5303.78226521641	3.86796282845433e-05\\
5344.27642516044	3.80530047064738e-05\\
5385.07975635387	3.74347465014094e-05\\
5426.19461930653	3.68247865486212e-05\\
5467.6233925506	3.62230572883443e-05\\
5509.36847277828	3.56294907472395e-05\\
5551.43227498041	3.50440185635424e-05\\
5593.81723258619	3.44665720118925e-05\\
5636.52579760393	3.38970820278423e-05\\
5679.56044076296	3.33354792320382e-05\\
5722.92365165649	3.2781693954073e-05\\
5766.61793888571	3.22356562560042e-05\\
5810.64583020485	3.16972959555367e-05\\
5855.00987266743	3.11665426488646e-05\\
5899.71263277365	3.06433257331721e-05\\
5944.75669661882	3.01275744287878e-05\\
5990.14467004298	2.9619217800993e-05\\
6035.87917878165	2.91181847814796e-05\\
6081.96286861774	2.8624404189456e-05\\
6128.39840553459	2.81378047524009e-05\\
6175.18847587023	2.76583151264599e-05\\
6222.33578647276	2.71858639164875e-05\\
6269.84306485697	2.67203796957306e-05\\
6317.71305936208	2.6261791025153e-05\\
6365.9485393108	2.58100264724e-05\\
6414.55229516949	2.5365014630404e-05\\
6463.52713870962	2.49266841356274e-05\\
6512.8759031704	2.44949636859455e-05\\
6562.60144342272	2.40697820581674e-05\\
6612.70663613428	2.36510681251966e-05\\
6663.19437993603	2.32387508728288e-05\\
6714.06759558984	2.283275941619e-05\\
6765.32922615749	2.24330230158137e-05\\
6816.9822371709	2.20394710933591e-05\\
6869.02961680372	2.16520332469692e-05\\
6921.47437604418	2.1270639266272e-05\\
6974.31954886926	2.08952191470243e-05\\
7027.56819242026	2.05257031053998e-05\\
7081.22338717961	2.01620215919232e-05\\
7135.2882371491	1.98041053050512e-05\\
7189.76587002947	1.94518852044028e-05\\
7244.65943740126	1.91052925236399e-05\\
7299.97211490729	1.87642587830002e-05\\
7355.7071024362	1.84287158014855e-05\\
7411.86762430771	1.80985957087056e-05\\
7468.45692945906	1.77738309563817e-05\\
7525.47829163299	1.74543543295115e-05\\
7582.93500956713	1.71400989571968e-05\\
7640.83040718485	1.68309983231387e-05\\
7699.16783378753	1.65269862758007e-05\\
7757.95066424834	1.62279970382445e-05\\
7817.18229920744	1.59339652176397e-05\\
7876.86616526879	1.5644825814451e-05\\
7937.00571519826	1.5360514231307e-05\\
7997.6044281235	1.50809662815508e-05\\
8058.66580973513	1.48061181974796e-05\\
8120.19339248961	1.45359066382731e-05\\
8182.19073581352	1.42702686976158e-05\\
8244.66142630952	1.40091419110164e-05\\
8307.60907796385	1.37524642628271e-05\\
8371.03733235538	1.35001741929677e-05\\
8434.9498588663	1.32522106033568e-05\\
8499.35035489435	1.30085128640539e-05\\
8564.24254606679	1.27690208191175e-05\\
8629.63018645584	1.25336747921807e-05\\
8695.51705879595	1.23024155917506e-05\\
8761.90697470259	1.20751845162331e-05\\
8828.80377489274	1.18519233586889e-05\\
8896.2113294071	1.16325744113234e-05\\
8964.13353783398	1.14170804697144e-05\\
9032.57432953488	1.12053848367837e-05\\
9101.53766387181	1.09974313265128e-05\\
9171.02753043637	1.07931642674111e-05\\
9241.0479492805	1.05925285057376e-05\\
9311.60297114908	1.03954694084813e-05\\
9382.69667771427	1.02019328661046e-05\\
9454.33318181161	1.00118652950539e-05\\
9526.51662767799	9.8252136400409e-06\\
9599.25119119138	9.6419253760994e-06\\
9672.54108011239	9.46194851042197e-06\\
9746.39053432772	9.28523158397989e-06\\
9820.80382609543	9.11172367293136e-06\\
9895.78526029209	8.94137438982172e-06\\
9971.33917466183	8.77413388458023e-06\\
10047.4699400673	8.60995284531728e-06\\
10124.1819607425	8.44878249892665e-06\\
10201.4796745474	8.29057461149643e-06\\
10279.3675532253	8.13528148853349e-06\\
10357.8501026605	7.9828559750052e-06\\
10436.9318631401	7.83325145520268e-06\\
10516.6174096155	7.68642185243003e-06\\
10596.9113519683	7.54232162852284e-06\\
10677.818335276	7.40090578320164e-06\\
10759.343040081	7.26212985326241e-06\\
10841.4901826617	7.12594991161008e-06\\
10924.2645153049	6.99232256613798e-06\\
11007.670826581	6.86120495845729e-06\\
11091.713941621	6.73255476248155e-06\\
11176.3987223955	6.60633018286859e-06\\
11261.730067996	6.4824899533253e-06\\
11347.7129149183	6.36099333477839e-06\\
11434.3522373485	6.2418001134151e-06\\
11521.65304745	6.12487059859813e-06\\
11609.6203956541	6.01016562065782e-06\\
11698.2593709518	5.89764652856661e-06\\
11787.5751011885	5.78727518749806e-06\\
11877.5727533603	5.67901397627536e-06\\
11968.2575339131	5.5728257847125e-06\\
12059.6346890441	5.46867401085146e-06\\
12151.7095050046	5.36652255809992e-06\\
12244.4873084065	5.26633583227185e-06\\
12337.9734665302	5.16807873853559e-06\\
12432.1733876346	5.07171667827232e-06\\
12527.0925212711	4.97721554584814e-06\\
12622.7363585976	4.8845417253041e-06\\
12719.1104326973	4.79366208696625e-06\\
12816.2203188977	4.70454398398004e-06\\
12914.0716350943	4.61715524877179e-06\\
13012.6700420745	4.53146418944054e-06\\
13112.0212438457	4.44743958608367e-06\\
13212.1309879655	4.3650506870587e-06\\
13313.0050658734	4.28426720518552e-06\\
13414.6493132266	4.20505931389072e-06\\
13517.0696102371	4.12739764329793e-06\\
13620.2718820123	4.05125327626685e-06\\
13724.2620988974	3.97659774438338e-06\\
13829.046276821	3.90340302390453e-06\\
13934.630477643	3.83164153165993e-06\\
14041.0208095055	3.76128612091337e-06\\
14148.2234271857	3.69231007718671e-06\\
14256.2445324528	3.62468711404864e-06\\
14365.0903744256	3.55839136887139e-06\\
14474.7672499352	3.49339739855712e-06\\
14585.2815038886	3.4296801752372e-06\\
14696.6395296357	3.36721508194627e-06\\
14808.8477693396	3.30597790827367e-06\\
14921.9127143489	3.24594484599462e-06\\
15035.8409055736	3.18709248468292e-06\\
15150.638933863	3.12939780730833e-06\\
15266.3134403874	3.07283818581959e-06\\
15382.8711170223	3.01739137671619e-06\\
15500.318706735	2.96303551661048e-06\\
15618.6630039756	2.90974911778194e-06\\
15737.910855069	2.85751106372632e-06\\
15858.069158612	2.80630060470052e-06\\
15979.1448658717	2.75609735326602e-06\\
16101.1449811876	2.70688127983207e-06\\
16224.0765623776	2.65863270820057e-06\\
16347.9467211451	2.61133231111459e-06\\
16472.7626234916	2.56496110581174e-06\\
16598.5314901305	2.51950044958443e-06\\
16725.260596905	2.47493203534834e-06\\
16852.9572752091	2.43123788722078e-06\\
16981.6289124118	2.38840035611034e-06\\
17111.2829522843	2.34640211531924e-06\\
17241.9268954305	2.30522615615987e-06\\
17373.5682997215	2.26485578358681e-06\\
17506.2147807321	2.22527461184564e-06\\
17639.8740121818	2.18646656013985e-06\\
17774.5537263786	2.14841584831685e-06\\
17910.2617146665	2.11110699257472e-06\\
18047.005827876	2.07452480119029e-06\\
18184.7939767782	2.03865437027002e-06\\
18323.6341325429	2.00348107952463e-06\\
18463.5343271993	1.96899058806836e-06\\
18604.5026541008	1.93516883024413e-06\\
18746.5472683933	1.90200201147519e-06\\
18889.6763874869	1.86947660414442e-06\\
19033.8982915311	1.83757934350205e-06\\
19179.2213238944	1.80629722360247e-06\\
19325.6538916461	1.77561749327133e-06\\
19473.2044660434	1.74552765210315e-06\\
19621.8815830212	1.71601544649053e-06\\
19771.6938436859	1.68706886568555e-06\\
19922.649914813	1.65867613789394e-06\\
20074.7585293483	1.63082572640268e-06\\
20228.0284869136	1.6035063257416e-06\\
20382.4686543151	1.57670685787956e-06\\
20538.087966057	1.55041646845567e-06\\
20694.8954248579	1.52462452304607e-06\\
20852.9001021718	1.49932060346677e-06\\
21012.1111387131	1.4744945041128e-06\\
21172.5377449848	1.45013622833437e-06\\
21334.1892018119	1.42623598485007e-06\\
21497.0748608783	1.40278418419768e-06\\
21661.2041452672	1.3797714352229e-06\\
21826.5865500072	1.35718854160603e-06\\
21993.2316426205	1.33502649842731e-06\\
22161.1490636774	1.31327648877067e-06\\
22330.3485273531	1.29192988036648e-06\\
22500.8398219907	1.27097822227323e-06\\
22672.6328106664	1.2504132415985e-06\\
22845.7374317606	1.23022684025926e-06\\
23020.1636995334	1.21041109178152e-06\\
23195.9217047026	1.1909582381397e-06\\
23373.0216150288	1.1718606866355e-06\\
23551.4736759026	1.15311100681654e-06\\
23731.2882109381	1.13470192743472e-06\\
23912.4756225696	1.11662633344421e-06\\
24095.0463926535	1.09887726303936e-06\\
24279.0110830748	1.08144790473215e-06\\
24464.3803363582	1.06433159446944e-06\\
24651.1648762833	1.04752181278982e-06\\
24839.3755085058	1.03101218201989e-06\\
25029.0231211817	1.01479646351024e-06\\
25220.1186855978	9.98868554910518e-07\\
25412.6732568062	9.83222487483957e-07\\
25606.697974264	9.67852423460875e-07\\
25802.2040624775	9.52752653431189e-07\\
25999.2028316513	9.37917593775735e-07\\
26197.7056783437	9.23341784136152e-07\\
26397.7240861246	9.09019884923301e-07\\
26599.2696262408	8.94946674863833e-07\\
26802.353958285	8.81117048584837e-07\\
27006.9888308703	8.67526014236313e-07\\
27213.1860823103	8.54168691151151e-07\\
27420.9576413033	8.41040307542558e-07\\
27630.3155276229	8.28136198238464e-07\\
27841.2718528132	8.15451802452822e-07\\
28053.8388208892	8.02982661593459e-07\\
28268.0287290435	7.90724417106151e-07\\
28483.8539683568	7.78672808354787e-07\\
28701.3270245155	7.66823670537136e-07\\
28920.4604785335	7.55172932636076e-07\\
29141.2670074804	7.43716615405893e-07\\
29363.7593852146	7.3245082939335e-07\\
29587.9504831223	7.21371772993242e-07\\
29813.8532708625	7.10475730538039e-07\\
30041.4808171167	6.99759070421364e-07\\
30270.8462903454	6.8921824325492e-07\\
30501.9629595499	6.78849780058538e-07\\
30734.8441950392	6.68650290483014e-07\\
30969.503469205	6.58616461065331e-07\\
31205.9543572992	6.48745053516026e-07\\
31444.2105382208	6.39032903038195e-07\\
31684.2857953065	6.29476916677902e-07\\
31926.194017128	6.20074071705574e-07\\
32169.949198296	6.10821414027989e-07\\
32415.5654402693	6.01716056630579e-07\\
32663.0569521706	5.92755178049567e-07\\
32912.4380516091	5.83936020873648e-07\\
33163.7231655079	5.75255890274816e-07\\
33416.9268309394	5.66712152567931e-07\\
33672.0636959658	5.58302233798719e-07\\
33929.1485204865	5.50023618359748e-07\\
34188.1961770926	5.41873847634064e-07\\
34449.2216519263	5.33850518666085e-07\\
34712.2400455487	5.25951282859365e-07\\
34977.2665738129	5.18173844700888e-07\\
35244.3165687446	5.10515960511443e-07\\
35513.4054794286	5.0297543722181e-07\\
35784.5488729031	4.95550131174262e-07\\
36057.76243506	4.88237946949092e-07\\
36333.0619715519	4.81036836215767e-07\\
36610.4634087077	4.73944796608299e-07\\
36889.9827944524	4.66959870624534e-07\\
37171.6362992366	4.60080144548905e-07\\
37455.4402169718	4.53303747398346e-07\\
37741.4109659722	4.46628849890961e-07\\
38029.5650899061	4.40053663437083e-07\\
38319.9192587512	4.33576439152403e-07\\
38612.4902697601	4.2719546689273e-07\\
38907.2950484319	4.20909074310094e-07\\
39204.350649491	4.14715625929796e-07\\
39503.674257874	4.08613522248058e-07\\
39805.2831897235	4.02601198849931e-07\\
40109.1948933907	3.96677125547076e-07\\
40415.4269504438	3.90839805535141e-07\\
40723.9970766856	3.85087774570297e-07\\
41034.9231231787	3.79419600164671e-07\\
41348.223077277	3.73833880800304e-07\\
41663.9150636682	3.68329245161274e-07\\
41982.0173454204	3.62904351383719e-07\\
42302.5483250399	3.57557886323358e-07\\
42625.5265455352	3.52288564840227e-07\\
42950.9706914898	3.47095129100307e-07\\
43278.8995901437	3.41976347893689e-07\\
43609.3322124815	3.36931015969019e-07\\
43942.2876743309	3.31957953383851e-07\\
44277.7852374682	3.27056004870634e-07\\
44615.8443107321	3.22224039218024e-07\\
44956.4844511476	3.17460948667211e-07\\
45299.7253650563	3.12765648322968e-07\\
45645.5869092572	3.08137075579124e-07\\
45994.0890921549	3.03574189558173e-07\\
46345.2520749175	2.99075970564735e-07\\
46699.096172643	2.94641419552582e-07\\
47055.6418555338	2.90269557604948e-07\\
47414.9097500821	2.85959425427842e-07\\
47776.920640262	2.8171008285612e-07\\
48141.6954687325	2.77520608372004e-07\\
48509.2553380493	2.73390098635816e-07\\
48879.6215118846	2.69317668028664e-07\\
49252.8154162586	2.65302448206797e-07\\
49628.8586407777	2.61343587667421e-07\\
50007.7729398843	2.57440251325672e-07\\
50389.5802341153	2.5359162010255e-07\\
50774.3026113693	2.49796890523546e-07\\
51161.962328186	2.46055274327718e-07\\
51552.5818110322	2.42365998087015e-07\\
51946.1836576	2.38728302835576e-07\\
52342.7906381142	2.35141443708807e-07\\
52742.4256966491	2.31604689592004e-07\\
53145.1119524562	2.28117322778303e-07\\
53550.8727013012	2.2467863863574e-07\\
53959.7314168126	2.21287945283203e-07\\
54371.7117518386	2.17944563275093e-07\\
54786.8375398161	2.14647825294449e-07\\
55205.1327961494	2.11397075854376e-07\\
55626.6217195991	2.08191671007551e-07\\
56051.3286936829	2.05030978063617e-07\\
56479.2782880851	2.01914375314303e-07\\
56910.4952600788	1.98841251766031e-07\\
57345.0045559579	1.95811006879875e-07\\
57782.83131248	1.92823050318667e-07\\
58224.0008583212	1.89876801701073e-07\\
58668.5387155406	1.86971690362487e-07\\
59116.4706010574	1.84107155122536e-07\\
59567.8224281384	1.81282644059067e-07\\
60022.620307897	1.78497614288432e-07\\
60480.8905508042	1.75751531751912e-07\\
60942.6596682098	1.7304387100813e-07\\
61407.9543738774	1.70374115031289e-07\\
61876.8015855284	1.67741755015101e-07\\
62349.2284264003	1.65146290182237e-07\\
62825.2622268155	1.62587227599171e-07\\
63304.9305257616	1.60064081996276e-07\\
63788.2610724863	1.57576375593012e-07\\
64275.2818281007	1.55123637928104e-07\\
64766.0209671984	1.52705405694548e-07\\
65260.5068794848	1.50321222579328e-07\\
65758.7681714191	1.47970639107724e-07\\
66260.8336678705	1.45653212492057e-07\\
66766.7324137839	1.43368506484801e-07\\
67276.4936758618	1.41116091235881e-07\\
67790.1469442567	1.38895543154092e-07\\
68307.721934277	1.36706444772499e-07\\
68829.2485881065	1.34548384617711e-07\\
69354.7570765359	1.3242095708293e-07\\
69884.2778007095	1.30323762304647e-07\\
70417.8413938822	1.28256406042916e-07\\
70955.4787231929	1.26218499565066e-07\\
71497.22089145	1.24209659532783e-07\\
72043.0992389301	1.22229507892447e-07\\
72593.1453451923	1.20277671768638e-07\\
73147.3910309034	1.18353783360721e-07\\
73705.8683596805	1.16457479842408e-07\\
74268.6096399447	1.14588403264227e-07\\
74835.6474267902	1.127462004588e-07\\
75407.0145238692	1.10930522948847e-07\\
75982.743985287	1.09141026857847e-07\\
76562.8691175166	1.07377372823256e-07\\
77147.4234813242	1.0563922591223e-07\\
77736.4408937111	1.03926255539753e-07\\
78329.9554298704	1.02238135389116e-07\\
78928.001425157	1.00574543334668e-07\\
79530.6134770757	9.89351613667618e-08\\
80137.8264472814	9.73196755188436e-08\\
80749.6754635966	9.57277757966054e-08\\
81366.195922043	9.41591561091441e-08\\
81987.4234888893	9.26135142020641e-08\\
82613.3941027154	9.10905515924541e-08\\
83244.1439764901	8.95899735056963e-08\\
83879.709599667	8.81114888140304e-08\\
84520.1277402954	8.66548099768296e-08\\
85165.4354471463	8.52196529825325e-08\\
85815.670051858	8.38057372921685e-08\\
86470.8691710928	8.24127857844421e-08\\
87131.0707087156	8.1040524702307e-08\\
87796.3128579857	7.9688683600995e-08\\
88466.6341037656	7.835699529745e-08\\
89142.0732247495	7.70451958211108e-08\\
89822.669295704	7.57530243660176e-08\\
90508.4616897307	7.44802232441773e-08\\
91199.4900805429	7.32265378401614e-08\\
91895.7944447613	7.19917165668886e-08\\
92597.4150642266	7.0775510822553e-08\\
93304.3925283287	6.95776749486626e-08\\
94016.7677363574	6.83979661891425e-08\\
94734.5818998656	6.72361446504818e-08\\
95457.8765450552	6.609197326287e-08\\
96186.6935151793	6.49652177423049e-08\\
96921.0749729615	6.38556465536312e-08\\
97661.0634030374	6.27630308744757e-08\\
98406.70161441	6.16871445600585e-08\\
99158.0327429277	6.06277641088361e-08\\
99915.1002537795	5.95846686289589e-08\\
100677.947944009	5.85576398055103e-08\\
101446.619945048	5.75464618684961e-08\\
102221.160725271	5.65509215615692e-08\\
103001.615092567	5.55708081114492e-08\\
103788.028196929	5.46059131980255e-08\\
104580.445533071	5.36560309251114e-08\\
105378.912943057	5.27209577918294e-08\\
106183.47661895	5.18004926646064e-08\\
106994.183105493	5.08944367497521e-08\\
107811.079302791	5.00025935666084e-08\\
108634.212469033	4.91247689212395e-08\\
109463.63022322	4.82607708806509e-08\\
110299.380547924	4.74104097475162e-08\\
111141.51179206	4.65734980353895e-08\\
111990.072673687	4.5749850444395e-08\\
112845.112282822	4.49392838373656e-08\\
113706.680084285	4.41416172164232e-08\\
114574.825920556	4.33566716999816e-08\\
115449.600014662	4.25842705001535e-08\\
116331.052973077	4.18242389005556e-08\\
117219.235788658	4.10764042344865e-08\\
118114.199843588	4.03405958634737e-08\\
119015.99691235	3.96166451561709e-08\\
119924.679164726	3.89043854675944e-08\\
120840.299168808	3.82036521186877e-08\\
121762.909894048	3.75142823761994e-08\\
122692.564714314	3.68361154328687e-08\\
123629.317410982	3.61689923879004e-08\\
124573.222176048	3.55127562277261e-08\\
125524.33361526	3.48672518070368e-08\\
126482.706751281	3.42323258300785e-08\\
127448.397026865	3.36078268322042e-08\\
128421.460308075	3.29936051616683e-08\\
129401.952887506	3.238951296166e-08\\
130389.931487544	3.17954041525656e-08\\
131385.45326365	3.12111344144488e-08\\
132388.575807663	3.06365611697486e-08\\
133399.357151135	3.007154356618e-08\\
134417.855768684	2.95159424598357e-08\\
135444.130581383	2.89696203984801e-08\\
136478.240960161	2.843244160503e-08\\
137520.246729245	2.79042719612158e-08\\
138570.208169615	2.73849789914163e-08\\
139628.186022495	2.68744318466643e-08\\
140694.241492865	2.63725012888148e-08\\
141768.436253005	2.58790596748724e-08\\
142850.832446056	2.53939809414737e-08\\
143941.492689623	2.49171405895164e-08\\
145040.480079392	2.44484156689377e-08\\
146147.858192783	2.39876847636293e-08\\
147263.691092627	2.35348279764922e-08\\
148388.04333087	2.30897269146239e-08\\
149520.979952313	2.2652264674634e-08\\
150662.566498367	2.22223258280886e-08\\
151812.869010854	2.17997964070744e-08\\
152971.954035819	2.1384563889885e-08\\
154139.888627384	2.09765171868236e-08\\
155316.740351627	2.05755466261178e-08\\
156502.577290491	2.01815439399499e-08\\
157697.468045719	1.97944022505916e-08\\
158901.481742829	1.94140160566487e-08\\
160114.688035108	1.90402812194087e-08\\
161337.157107642	1.86730949492903e-08\\
162568.959681378	1.83123557923948e-08\\
163810.167017216	1.79579636171536e-08\\
165060.850920127	1.76098196010744e-08\\
166321.083743312	1.72678262175807e-08\\
167590.938392387	1.6931887222946e-08\\
168870.488329596	1.66019076433188e-08\\
170159.807578068	1.62777937618382e-08\\
171458.970726091	1.595945310584e-08\\
172768.052931436	1.56467944341484e-08\\
174087.129925698	1.53397277244568e-08\\
175416.27801868	1.50381641607931e-08\\
176755.574102809	1.47420161210694e-08\\
178105.095657579	1.44511971647173e-08\\
179464.920754039	1.41656220204038e-08\\
180835.128059308	1.38852065738314e-08\\
182215.796841124	1.3609867855618e-08\\
183607.00697243	1.33395240292587e-08\\
185008.838935997	1.30740943791674e-08\\
186421.373829081	1.28134992987964e-08\\
187844.693368106	1.25576602788381e-08\\
189278.879893403	1.23064998955017e-08\\
190724.016373967	1.20599417988701e-08\\
192180.186412254	1.18179107013337e-08\\
193647.474249029	1.15803323661007e-08\\
195125.964768223	1.1347133595786e-08\\
196615.743501859	1.11182422210745e-08\\
198116.896634991	1.0893587089462e-08\\
199629.511010691	1.06730980540719e-08\\
201153.674135077	1.0456705962547e-08\\
202689.47418237	1.02443426460178e-08\\
204237	1.00359409081449e-08\\
};
\addlegendentry{Adatt. BLN reale (\textit{ML})}


        \nextgroupplot[%
            plotColumn2,plotLegend1,xmode=log,
            xmin=100,xmax=1000000,
            ymin=0,ymax=0.00045,
            ytick={0,0.0001,0.0002,0.0003,0.0004},
        ] 
% !TEX root = ../../../../Esperimenti/.tex/MEF.tex

% This file was created by matlab2tikz.
%
\definecolor{mycolor1}{rgb}{0.00000,0.44706,0.69804}%


\addplot[ybar interval, fill=mycolor1, fill opacity=0.35, area legend, draw=none] table[row sep=crcr, x=Lower, y=Count] {%
Lower	Upper	Count\\
119	166.945095217458	0\\
166.945095217458	234.207267371144	3.17166880733337e-06\\
234.207267371144	328.569365982322	1.10213742096321e-05\\
328.569365982322	460.949950331585	3.18274266127819e-05\\
460.949950331585	646.666666794865	8.09835554193296e-05\\
646.666666794865	907.208640857353	0.000180239672202442\\
907.208640857353	1272.72296579858	0.000286573355732386\\
1272.72296579858	1785.50299756883	0.000343383105992112\\
1785.50299756883	2504.88208353096	0.00027775805167604\\
2504.88208353096	3514.09897431581	0.000155922875882135\\
3514.09897431581	4929.92930983802	6.88406401680306e-05\\
4929.92930983802	6916.19763064073	3.02476756895162e-05\\
6916.19763064073	9702.73337806785	1.42973224143223e-05\\
9702.73337806785	13611.9642661441	7.31260636302151e-06\\
13611.9642661441	19096.2241219165	4.46854585872192e-06\\
19096.2241219165	26790.0920531703	2.83862423882819e-06\\
26790.0920531703	37583.8190647142	1.42551934565426e-06\\
37583.8190647142	52726.3382554153	4.40261397902708e-07\\
52726.3382554153	73969.7778194808	7.53173701073576e-09\\
73969.7778194808	103772.198330147	0\\
103772.198330147	145582.012866817	0\\
145582.012866817	204237	0\\
204237	204237	0\\
};
\addlegendentry{Istogramma esatto}

\addplot [color=mycolor1, only marks, every error bar/.append style={opacity=0.45}, mark=*, mark size=0pt, draw=none, forget plot]
plot [error bars/.cd, y dir=both, y explicit, error bar style={line width=1pt, color=mycolor1}, error mark options={mark=none,mark size=0pt}]
table[row sep=crcr, y error plus index=2, y error minus index=3]{%
140.948452743822	0	0	0\\
197.736578689671	3.17166880733337e-06	2.1449062241398e-06	2.1449062241398e-06\\
277.404638296819	1.10213742096321e-05	3.1775175906533e-06	3.1775175906533e-06\\
389.170955917874	3.18274266127819e-05	4.37037270335533e-06	4.37037270335533e-06\\
545.967918416627	8.09835554193296e-05	6.52363581049669e-06	6.52363581049669e-06\\
765.938370804547	0.000180239672202442	7.10519944668116e-06	7.10519944668116e-06\\
1074.53490961907	0.000286573355732386	8.26890316052978e-06	8.26890316052979e-06\\
1507.46498151966	0.000343383105992112	6.19029878782422e-06	6.19029878782423e-06\\
2114.82256198977	0.00027775805167604	5.11436792956057e-06	5.11436792956059e-06\\
2966.88448722194	0.000155922875882135	3.09906320195729e-06	3.09906320195729e-06\\
4162.24212788627	6.88406401680306e-05	1.8715247402272e-06	1.8715247402272e-06\\
5839.20931393352	3.02476756895162e-05	1.08519869048488e-06	1.08519869048488e-06\\
8191.82651184286	1.42973224143223e-05	6.03992438287251e-07	6.0399243828725e-07\\
11492.3130842395	7.31260636302151e-06	3.19654954295681e-07	3.19654954295681e-07\\
16122.5655640101	4.46854585872192e-06	2.03719926583672e-07	2.03719926583673e-07\\
22618.3465817931	2.83862423882819e-06	1.61416954355465e-07	1.61416954355465e-07\\
31731.2775105792	1.42551934565426e-06	9.08077732564802e-08	9.08077732564803e-08\\
44515.8079443296	4.40261397902708e-07	3.27568237609491e-08	3.27568237609491e-08\\
62451.2251760353	7.53173701073576e-09	5.94504309356008e-09	5.94504309356008e-09\\
87612.8212895809	0	0	0\\
122912.023466044	0	0	0\\
172433.272780749	0	0	0\\
};
\addplot [color=mycolor1]
table[row sep=crcr]{%
119	1.30775318032512e-06\\
119.890504214077	1.32958770906898e-06\\
120.787672274837	1.35176727888127e-06\\
121.691554049369	1.37429811343094e-06\\
122.602199777928	1.39718660441989e-06\\
123.519660076729	1.42043931757573e-06\\
124.44398594076	1.44406299885082e-06\\
125.375228746615	1.46806458083331e-06\\
126.313440255352	1.49245118937606e-06\\
127.258672615369	1.517230150449e-06\\
128.2109783653	1.54240899722129e-06\\
129.17041043694	1.56799547737892e-06\\
130.137022158185	1.59399756068408e-06\\
131.110867255994	1.62042344678242e-06\\
132.091999859379	1.64728157326444e-06\\
133.080474502409	1.67458062398736e-06\\
134.076346127247	1.70232953766386e-06\\
135.0796700872	1.73053751672413e-06\\
136.090502149795	1.75921403645778e-06\\
137.108898499881	1.78836885444214e-06\\
138.134915742751	1.81801202026365e-06\\
139.168610907289	1.84815388553887e-06\\
140.21004144914	1.87880511424194e-06\\
141.259265253899	1.90997669334522e-06\\
142.316340640336	1.94167994377976e-06\\
143.381326363632	1.97392653172253e-06\\
144.454281618647	2.00672848021709e-06\\
145.535266043209	2.04009818113463e-06\\
146.624339721429	2.07404840748202e-06\\
147.721563187044	2.10859232606383e-06\\
148.826997426776	2.14374351050496e-06\\
149.940703883725	2.17951595464069e-06\\
151.062744460785	2.2159240862809e-06\\
152.193181524082	2.25298278135499e-06\\
153.332077906443	2.2907073784442e-06\\
154.479496910888	2.32911369370788e-06\\
155.635502314145	2.36821803621004e-06\\
156.800158370201	2.40803722365274e-06\\
157.97352981387	2.44858859852227e-06\\
159.15568186439	2.48989004465463e-06\\
160.34668022905	2.53196000422599e-06\\
161.546591106842	2.57481749517412e-06\\
162.755481192139	2.61848212905652e-06\\
163.973417678405	2.66297412935071e-06\\
165.200468261928	2.70831435020202e-06\\
166.436701145581	2.75452429562393e-06\\
167.682185042617	2.80162613915605e-06\\
168.936989180483	2.84964274398413e-06\\
170.201183304674	2.89859768352671e-06\\
171.474837682604	2.94851526249233e-06\\
172.758023107516	2.99942053841132e-06\\
174.050810902413	3.05133934364539e-06\\
175.353272924028	3.10429830787843e-06\\
176.665481566809	3.15832488109118e-06\\
177.987509766954	3.21344735702209e-06\\
179.319431006454	3.26969489711663e-06\\
180.661319317186	3.32709755496626e-06\\
182.013249285024	3.38568630123854e-06\\
183.375296053983	3.44549304909875e-06\\
184.7475353304	3.50655068012331e-06\\
186.13004338714	3.56889307070453e-06\\
187.522897067834	3.63255511894585e-06\\
188.926173791152	3.69757277204592e-06\\
190.339951555105	3.76398305416948e-06\\
191.764308941382	3.83182409480232e-06\\
193.199325119717	3.90113515758678e-06\\
194.645079852288	3.97195666963382e-06\\
196.101653498152	4.04433025130666e-06\\
197.569127017711	4.11829874647062e-06\\
199.047581977213	4.19390625320261e-06\\
200.537100553285	4.27119815495325e-06\\
202.037765537499	4.35022115215343e-06\\
203.549660340977	4.43102329425673e-06\\
205.072868999023	4.51365401220751e-06\\
206.6074761758	4.59816415132433e-06\\
208.15356716903	4.68460600458679e-06\\
209.711227914737	4.7730333463131e-06\\
211.280544992026	4.86350146621486e-06\\
212.86160562789	4.95606720381391e-06\\
214.454497702065	5.05078898320577e-06\\
216.059309751909	5.14772684815225e-06\\
217.676130977326	5.24694249748542e-06\\
219.305051245723	5.3484993208032e-06\\
220.946161097006	5.45246243443623e-06\\
222.599551748611	5.55889871766399e-06\\
224.265315100576	5.66787684915692e-06\\
225.943543740646	5.77946734362038e-06\\
227.634330949423	5.89374258861398e-06\\
229.33777070555	6.01077688151969e-06\\
231.05395769093	6.13064646662942e-06\\
232.782987295997	6.25342957232234e-06\\
234.524955625009	6.37920644830009e-06\\
236.279959501398	6.50805940284675e-06\\
238.048096473146	6.64007284007902e-06\\
239.829464818207	6.77533329714985e-06\\
241.624163549975	6.91392948136849e-06\\
243.432292422782	7.05595230719632e-06\\
245.253951937446	7.20149493307848e-06\\
247.089243346851	7.35065279806731e-06\\
248.938268661586	7.50352365819413e-06\\
250.801130655604	7.66020762254203e-06\\
252.677932871941	7.82080718897268e-06\\
254.568779628468	7.98542727945635e-06\\
256.473776023691	8.15417527495467e-06\\
258.393027942593	8.32716104980159e-06\\
260.326642062517	8.50449700552857e-06\\
262.274725859099	8.68629810407602e-06\\
264.237387612236	8.87268190033272e-06\\
266.214736412113	9.06376857394262e-06\\
268.206882165259	9.25968096031602e-06\\
270.213935600659	9.46054458078154e-06\\
272.236008275909	9.66648767181193e-06\\
274.273212583414	9.87764121325646e-06\\
276.325661756641	1.00941389555093e-05\\
278.393469876404	1.03161174455431e-05\\
280.476751877215	1.05437160517331e-05\\
282.575623553662	1.07770769873982e-05\\
284.690201566855	1.10163453329796e-05\\
286.820603450903	1.12616690567809e-05\\
288.96694761945	1.1513199034187e-05\\
291.129353372257	1.17710890652801e-05\\
293.307940901833	1.20354958907698e-05\\
295.502831300113	1.23065792061505e-05\\
297.714146565192	1.25845016739999e-05\\
299.942009608104	1.28694289343291e-05\\
302.186544259657	1.31615296128946e-05\\
304.447875277308	1.34609753273816e-05\\
306.726128352108	1.37679406913623e-05\\
309.02143011568	1.40826033159397e-05\\
311.333908147261	1.44051438089764e-05\\
313.663690980792	1.47357457718159e-05\\
316.010908112063	1.50745957933948e-05\\
318.375690005912	1.54218834416504e-05\\
320.758168103475	1.5777801252121e-05\\
323.158474829489	1.61425447136401e-05\\
325.57674359966	1.65163122510226e-05\\
328.013108828071	1.68993052046406e-05\\
330.467705934659	1.72917278067865e-05\\
332.940671352735	1.76937871547202e-05\\
335.432142536577	1.81056931802981e-05\\
337.942257969062	1.85276586160782e-05\\
340.471157169366	1.89598989578012e-05\\
343.018980700718	1.94026324231402e-05\\
345.585870178217	1.98560799066197e-05\\
348.171968276697	2.03204649305976e-05\\
350.777418738662	2.07960135922116e-05\\
353.402366382274	2.12829545061853e-05\\
356.046957109401	2.17815187433945e-05\\
358.711337913731	2.22919397650956e-05\\
361.395656888935	2.28144533527147e-05\\
364.100063236907	2.33492975331034e-05\\
366.824707276051	2.38967124991633e-05\\
369.569740449639	2.4456940525748e-05\\
372.335315334224	2.50302258807485e-05\\
375.12158564813	2.56168147312749e-05\\
377.928706259986	2.6216955044845e-05\\
380.75683319734	2.68308964854968e-05\\
383.60612365533	2.74588903047431e-05\\
386.476736005421	2.81011892272889e-05\\
389.368829804207	2.87580473314366e-05\\
392.282565802281	2.94297199241062e-05\\
395.21810595317	3.01164634104047e-05\\
398.175613422337	3.08185351576771e-05\\
401.155252596246	3.15361933539821e-05\\
404.157189091507	3.22696968609345e-05\\
407.181589764074	3.30193050608649e-05\\
410.228622718523	3.37852776982499e-05\\
413.298457317395	3.45678747153731e-05\\
416.391264190611	3.536735608218e-05\\
419.507215244952	3.61839816202981e-05\\
422.646483673618	3.70180108211992e-05\\
425.809243965854	3.78697026584846e-05\\
428.995671916648	3.87393153942839e-05\\
432.205944636502	3.96271063797623e-05\\
435.440240561274	4.05333318497388e-05\\
438.698739462102	4.14582467114263e-05\\
441.981622455389	4.24021043273096e-05\\
445.289072012877	4.33651562921881e-05\\
448.621271971783	4.43476522044132e-05\\
451.978407545023	4.5349839431367e-05\\
455.360665331498	4.63719628692251e-05\\
458.768233326478	4.74142646970694e-05\\
462.201300932039	4.84769841254112e-05\\
465.660058967601	4.95603571392077e-05\\
469.144699680525	5.06646162354515e-05\\
472.655416756805	5.17899901554349e-05\\
476.192405331832	5.29367036117882e-05\\
479.755862001239	5.41049770104122e-05\\
483.345984831829	5.52950261674279e-05\\
486.962973372584	5.65070620212785e-05\\
490.607028665758	5.77412903401339e-05\\
494.278353258049	5.89979114247482e-05\\
497.977151211859	6.02771198069461e-05\\
501.703628116633	6.15791039439089e-05\\
505.457991100293	6.29040459084577e-05\\
509.240448840744	6.42521210755283e-05\\
513.051211577476	6.56234978050547e-05\\
516.890491123249	6.70183371214827e-05\\
520.758500875867	6.84367923901519e-05\\
524.655455830038	6.98790089907874e-05\\
528.581572589324	7.13451239883655e-05\\
532.537069378183	7.28352658016207e-05\\
536.522166054094	7.43495538694733e-05\\
540.537084119782	7.58880983156766e-05\\
544.582046735526	7.74509996119811e-05\\
548.657278731565	7.9038348240137e-05\\
552.763006620593	8.06502243530599e-05\\
556.899458610353	8.22866974355005e-05\\
561.066864616317	8.39478259645642e-05\\
565.265456274466	8.56336570704451e-05\\
569.495466954168	8.73442261977464e-05\\
573.757131771146	8.90795567677642e-05\\
578.050687600549	9.08396598421363e-05\\
582.376373090114	9.26245337882491e-05\\
586.734428673439	9.4434163946825e-05\\
591.125096583335	9.62685223021051e-05\\
595.548620865302	9.81275671550654e-05\\
600.005247391086	0.000100011242800104\\
604.495223872346	0.000101919479205647\\
609.018799874428	0.000103852191699144\\
613.576226830228	0.000105809280656895\\
618.167758054176	0.000107790631199209\\
622.793648756308	0.000109796112891358\\
627.454156056458	0.000111825579450816\\
632.149538998544	0.000113878868461278\\
636.880058564972	0.00011595580109396\\
641.645977691138	0.000118056181836678\\
646.447561280041	0.000120179798231215\\
651.285076217013	0.000122326420619488\\
656.158791384547	0.00012449580189903\\
661.06897767725	0.000126687677288303\\
666.015908016889	0.000128901764102356\\
670.999857367572	0.000131137761539353\\
676.021102751025	0.00013339535047849\\
681.079923261988	0.00013567419328981\\
686.176600083736	0.00013797393365645\\
691.311416503699	0.000140294196409817\\
696.484657929211	0.000142634587378215\\
701.696611903378	0.000144994693249436\\
706.947568121055	0.0001473740814478\\
712.237818444947	0.000149772300026162\\
717.56765692184	0.000152188877573364\\
722.937379798933	0.000154623323137616\\
728.347285540317	0.000157075126166292\\
733.797674843554	0.00015954375646259\\
739.288850656394	0.000162028664159523\\
744.821118193618	0.000164529279711678\\
750.394784953994	0.000167045013905181\\
756.01016073738	0.000169575257886281\\
761.667557661931	0.000172119383208947\\
767.367290181458	0.000174676741901894\\
773.109675102898	0.000177246666555372\\
778.89503160393	0.000179828470428104\\
784.72368125071	0.000182421447574689\\
790.59594801575	0.000185024872993797\\
796.512158295919	0.000187638002797439\\
802.472640930591	0.000190260074401605\\
808.47772721992	0.000192890306738497\\
814.527750943253	0.000195527900490611\\
820.623048377686	0.000198172038346844\\
826.763958316753	0.000200821885280833\\
832.950822089257	0.000203476588851653\\
839.183983578243	0.000206135279527011\\
845.463789240111	0.000208797071029029\\
851.790588123873	0.000211461060702685\\
858.164731890556	0.000214126329906948\\
864.586574832748	0.000216791944428614\\
871.056473894285	0.000219456954918817\\
877.574788690099	0.000222120397352161\\
884.141881526201	0.000224781293508382\\
890.758117419823	0.000227438651476419\\
897.423864119701	0.000230091466180728\\
904.139492126523	0.000232738719929667\\
910.905374713515	0.000235379382985704\\
917.721887947192	0.000238012414157194\\
924.589410708265	0.000240636761411435\\
931.508324712691	0.000243251362508656\\
938.479014532895	0.000245855145656572\\
945.501867619149	0.000248447030185102\\
952.577274321103	0.000251025927240797\\
959.705627909479	0.000253590740500503\\
966.88732459794	0.000256140366903741\\
974.122763565099	0.000258673697403242\\
981.412346976721	0.000261189617733056\\
988.756480008064	0.000263687009193596\\
996.155570866409	0.000266164749452957\\
1003.61003081374	0.000268621713363813\\
1011.12027418962	0.00027105677379515\\
1018.6867184342	0.000273468802478066\\
1026.30978411142	0.000275856670864833\\
1033.98989493243	0.000278219251000386\\
1041.72747777907	0.000280555416405367\\
1049.52296272766	0.00028286404296981\\
1057.37678307286	0.000285144009856545\\
1065.2893753518	0.000287394200413362\\
1073.26117936829	0.000289613503092916\\
1081.2926382173	0.000291800812379385\\
1089.38419830959	0.000293955029720806\\
1097.53630939651	0.000296075064466037\\
1105.749424595	0.000298159834805237\\
1114.02400041277	0.000300208268712749\\
1122.3604967737	0.000302219304891255\\
1130.75937704337	0.00030419189371603\\
1139.22110805483	0.000306124998178139\\
1147.74616013456	0.000308017594825368\\
1156.3350071286	0.000309868674699709\\
1164.98812642888	0.000311677244270159\\
1173.70599899975	0.000313442326359623\\
1182.48910940477	0.000315162961064693\\
1191.33794583355	0.000316838206667053\\
1200.25300012896	0.000318467140535273\\
1209.23476781446	0.000320048860015758\\
1218.28374812157	0.000321582483311602\\
1227.40044401774	0.000323067150348116\\
1236.58536223419	0.000324502023623784\\
1245.83901329415	0.000325886289045449\\
1255.16191154121	0.000327219156746489\\
1264.5545751679	0.00032849986188679\\
1274.01752624451	0.000329727665433345\\
1283.55129074812	0.000330901854920281\\
1293.15639859177	0.000332021745187187\\
1302.83338365401	0.000333086679094613\\
1312.5827838085	0.000334096028215621\\
1322.40514095393	0.00033504919350233\\
1332.30100104415	0.000335945605926396\\
1342.27091411851	0.00033678472709241\\
1352.31543433242	0.000337566049823236\\
1362.43511998817	0.000338289098716341\\
1372.63053356595	0.000338953430670188\\
1382.90224175512	0.00033955863537985\\
1393.2508154857	0.000340104335800992\\
1403.67682996012	0.000340590188581446\\
1414.18086468518	0.000341015884459625\\
1424.76350350425	0.0003413811486291\\
1435.42533462974	0.000341685741068682\\
1446.16695067579	0.000341929456837409\\
1456.98894869122	0.000342112126333911\\
1467.89193019267	0.000342233615519642\\
1478.8765011981	0.000342293826105563\\
1489.94327226042	0.000342292695701874\\
1501.09285850146	0.000342230197930485\\
1512.32587964614	0.000342106342499958\\
1523.64296005691	0.000341921175242705\\
1535.0447287685	0.00034167477811429\\
1546.53181952283	0.00034136726915476\\
1558.10487080425	0.000340998802411954\\
1569.76452587505	0.000340569567826839\\
1581.51143281119	0.000340079791080941\\
1593.34624453832	0.000339529733406052\\
1605.26961886811	0.000338919691356389\\
1617.28221853476	0.000338249996543505\\
1629.38471123188	0.000337521015334265\\
1641.57776964957	0.000336733148512284\\
1653.86207151182	0.000335886830903274\\
1666.2382996142	0.000334982530964806\\
1678.70714186178	0.00033402075034105\\
1691.26929130741	0.00033300202338311\\
1703.92544619016	0.000331926916635635\\
1716.67630997424	0.000330796028290419\\
1729.522591388	0.00032960998760778\\
1742.4650044634	0.000328369454306553\\
1755.50426857564	0.00032707511792355\\
1768.64110848317	0.000325727697143446\\
1781.876254368	0.000324327939100033\\
1795.21044187622	0.000322876618649872\\
1808.64441215895	0.000321374537619395\\
1822.17891191353	0.000319822524026547\\
1835.81469342496	0.000318221431278102\\
1849.55251460781	0.000316572137343825\\
1863.39313904828	0.000314875543908665\\
1877.33733604663	0.00031313257550422\\
1891.38588066002	0.00031134417862072\\
1905.53955374551	0.000309511320800817\\
1919.7991420035	0.000307634989716477\\
1934.16543802145	0.000305716192230299\\
1948.63924031791	0.000303755953442602\\
1963.22135338698	0.000301755315725636\\
1977.91258774292	0.000299715337746271\\
1992.71375996528	0.000297637093478552\\
2007.62569274426	0.0002955216712075\\
2022.64921492642	0.000293370172525522\\
2037.7851615608	0.000291183711322853\\
2053.03437394529	0.000288963412773373\\
2068.39769967338	0.000286710412317213\\
2083.87599268133	0.000284425854641497\\
2099.47011329558	0.000282110892660593\\
2115.18092828061	0.000279766686497229\\
2131.00931088707	0.000277394402465798\\
2146.95614090037	0.000274995212059184\\
2163.02230468953	0.000272570290940394\\
2179.20869525649	0.000270120817940292\\
2195.51621228573	0.000267647974062674\\
2211.94576219425	0.000265152941497928\\
2228.49825818201	0.00026263690264646\\
2245.17462028264	0.000260101039153081\\
2261.97577541457	0.00025754653095348\\
2278.90265743261	0.000254974555333882\\
2295.95620717979	0.000252386286004987\\
2313.1373725397	0.0002497828921912\\
2330.44710848916	0.000247165537736152\\
2347.88637715129	0.000244535380225491\\
2365.45614784899	0.00024189357012782\\
2383.15739715885	0.00023924124995469\\
2400.9911089654	0.000236579553440467\\
2418.95827451578	0.000233909604742849\\
2437.05989247487	0.000231232517664793\\
2455.29696898081	0.00022854939489854\\
2473.67051770087	0.000225861327292387\\
2492.18155988785	0.000223169393140816\\
2510.8311244368	0.000220474657498537\\
2529.62024794223	0.000217778171518951\\
2548.54997475574	0.000215080971817502\\
2567.62135704402	0.000212384079860331\\
2586.83545484739	0.000209688501378606\\
2606.1933361387	0.000206995225808842\\
2625.69607688265	0.000204305225759497\\
2645.34476109567	0.000201619456504063\\
2665.14048090611	0.000198938855500846\\
2685.08433661496	0.000196264341939568\\
2705.17743675704	0.000193596816314875\\
2725.42089816257	0.000190937160026826\\
2745.81584601927	0.00018828623500834\\
2766.3634139349	0.000185644883379591\\
2787.06474400025	0.000183013927129251\\
2807.92098685266	0.000180394167822489\\
2828.93330173995	0.000177786386335543\\
2850.10285658484	0.000175191342616695\\
2871.4308280499	0.000172609775473402\\
2892.91840160291	0.000170042402385323\\
2914.56677158281	0.000167489919342939\\
2936.37714126603	0.000164953000711436\\
2958.3507229334	0.000162432299119483\\
2980.48873793751	0.000159928445372507\\
3002.79241677064	0.000157442048390047\\
3025.2629991331	0.000154973695166735\\
3047.90173400216	0.000152523950756423\\
3070.7098797015	0.00015009335827896\\
3093.68870397109	0.000147682438949092\\
3116.83948403772	0.00014529169212694\\
3140.16350668593	0.000142921595389502\\
3163.66206832958	0.000140572604622579\\
3187.33647508389	0.000138245154132541\\
3211.18804283803	0.000135939656777308\\
3235.21809732829	0.000133656504115936\\
3259.42797421173	0.000131396066576144\\
3283.81901914043	0.000129158693639156\\
3308.39258783631	0.000126944714041176\\
3333.15004616647	0.000124754435990855\\
3358.09277021909	0.000122588147402053\\
3383.22214637995	0.000120446116141232\\
3408.53957140944	0.000118328590288798\\
3434.04645252026	0.000116235798413699\\
3459.7442074556	0.000114167949860604\\
3485.63426456793	0.000112125235048977\\
3511.71806289842	0.000110107825783359\\
3537.99705225692	0.000108115875574178\\
3564.47269330252	0.000106149519968417\\
3591.14645762477	0.000104208876889461\\
3618.01982782546	0.000102294046985456\\
3645.09429760103	0.000100405113985537\\
3672.37137182558	9.85421450632415e-05\\
3699.85256663454	9.67051912065059e-05\\
3727.53940950892	9.48942875935767e-05\\
3755.43343936023	9.31094539742346e-05\\
3783.53620661599	9.13506950557106e-05\\
3811.84927330594	8.96180008926968e-05\\
3840.37421314884	8.79113472808646e-05\\
3869.11261163994	8.62306961533112e-05\\
3898.06606613913	8.45759959793799e-05\\
3927.23618595969	8.29471821653e-05\\
3956.62459245778	8.13441774561184e-05\\
3986.23291912252	7.97668923384038e-05\\
4016.06281166681	7.82152254432192e-05\\
4046.1159281188	7.66890639488778e-05\\
4076.39393891404	7.51882839830112e-05\\
4106.89852698833	7.37127510234954e-05\\
4137.63138787127	7.22623202977941e-05\\
4168.59422978048	7.08368371802999e-05\\
4199.78877371658	6.94361375872681e-05\\
4231.21675355883	6.80600483689533e-05\\
4262.8799161615	6.6708387698579e-05\\
4294.78002145095	6.53809654577848e-05\\
4326.91884252352	6.40775836182124e-05\\
4359.29816574399	6.2798036618912e-05\\
4391.91979084494	6.15421117392632e-05\\
4424.78553102675	6.03095894671226e-05\\
4457.89721305839	5.91002438619272e-05\\
4491.25667737898	5.79138429125007e-05\\
4524.86577820005	5.67501488893206e-05\\
4558.72638360862	5.5608918691028e-05\\
4592.84037567103	5.44899041849683e-05\\
4627.20965053756	5.33928525415747e-05\\
4661.83611854782	5.23175065624185e-05\\
4696.72170433693	5.12636050017628e-05\\
4731.86834694246	5.02308828814748e-05\\
4767.27799991228	4.92190717991625e-05\\
4802.95263141311	4.82279002294205e-05\\
4838.89422433987	4.7257093818078e-05\\
4875.10477642598	4.63063756693571e-05\\
4911.58630035433	4.5375466625865e-05\\
4948.34082386919	4.44640855413534e-05\\
4985.37038988889	4.357194954619e-05\\
5022.6770566194	4.26987743055033e-05\\
5060.2628976687	4.18442742699666e-05\\
5098.13000216207	4.10081629192038e-05\\
5136.28047485818	4.01901529978082e-05\\
5174.7164362661	3.93899567439754e-05\\
5213.44002276314	3.86072861107595e-05\\
5252.45338671363	3.78418529799754e-05\\
5291.7586965885	3.7093369368777e-05\\
5331.3581370859	3.63615476289463e-05\\
5371.25390925253	3.56461006389427e-05\\
5411.44823060603	3.49467419887655e-05\\
5451.94333525824	3.42631861576894e-05\\
5492.74147403939	3.35951486849446e-05\\
5533.84491462315	3.2942346333414e-05\\
5575.25594165272	3.23044972464286e-05\\
5616.97685686784	3.16813210977502e-05\\
5659.00997923266	3.10725392348321e-05\\
5701.35764506468	3.04778748154563e-05\\
5744.02220816459	2.98970529378494e-05\\
5787.00603994713	2.93298007643836e-05\\
5830.31152957285	2.87758476389747e-05\\
5873.94108408097	2.82349251982905e-05\\
5917.8971285231	2.77067674768886e-05\\
5962.18210609809	2.71911110064044e-05\\
6006.79847828778	2.66876949089122e-05\\
6051.74872499389	2.61962609845873e-05\\
6097.03534467575	2.57165537937967e-05\\
6142.66085448928	2.52483207337489e-05\\
6188.62779042682	2.47913121098345e-05\\
6234.93870745815	2.43452812017919e-05\\
6281.59617967246	2.39099843248319e-05\\
6328.60280042144	2.34851808858562e-05\\
6375.9611824634	2.30706334349068e-05\\
6423.67395810857	2.26661077119818e-05\\
6471.74377936531	2.22713726893545e-05\\
6520.17331808759	2.18862006095321e-05\\
6568.96526612346	2.15103670189907e-05\\
6618.12233546469	2.11436507978214e-05\\
6667.64725839753	2.07858341854247e-05\\
6717.54278765452	2.04367028023853e-05\\
6767.81169656754	2.00960456686629e-05\\
6818.45677922192	1.97636552182303e-05\\
6869.48085061182	1.94393273102899e-05\\
6920.88674679662	1.91228612371994e-05\\
6972.67732505857	1.8814059729235e-05\\
7024.85546406162	1.85127289563173e-05\\
7077.42406401143	1.8218678526827e-05\\
7130.38604681656	1.79317214836319e-05\\
7183.7443562509	1.76516742974479e-05\\
7237.50195811723	1.73783568576521e-05\\
7291.66184041212	1.71115924606654e-05\\
7346.22701349204	1.68512077960202e-05\\
7401.20051024062	1.65970329302249e-05\\
7456.58538623724	1.63489012885367e-05\\
7512.38471992689	1.61066496347498e-05\\
7568.60161279127	1.5870118049105e-05\\
7625.23918952119	1.56391499044243e-05\\
7682.30059819021	1.54135918405705e-05\\
7739.78901042966	1.51932937373305e-05\\
7797.70762160489	1.49781086858177e-05\\
7856.05965099295	1.47678929584864e-05\\
7914.84834196143	1.45625059778497e-05\\
7974.0769621488	1.43618102839871e-05\\
8033.748803646	1.41656715009289e-05\\
8093.86718317946	1.39739583019987e-05\\
8154.43544229543	1.37865423741949e-05\\
8215.4569475457	1.36032983816892e-05\\
8276.93509067475	1.34241039285157e-05\\
8338.87328880825	1.32488395205236e-05\\
8401.27498464301	1.30773885266646e-05\\
8464.14364663834	1.29096371396807e-05\\
8527.48276920879	1.27454743362577e-05\\
8591.29587291844	1.25847918367073e-05\\
8655.58650467655	1.24274840642377e-05\\
8720.35823793472	1.22734481038694e-05\\
8785.61467288549	1.21225836610529e-05\\
8851.35943666247	1.19747930200397e-05\\
8917.59618354194	1.18299810020581e-05\\
8984.32859514597	1.16880549233419e-05\\
9051.56038064706	1.15489245530582e-05\\
9119.29527697427	1.14125020711781e-05\\
9187.53704902097	1.12787020263329e-05\\
9256.28948985408	1.1147441293694e-05\\
9325.55642092493	1.10186390329169e-05\\
9395.34169228163	1.08922166461827e-05\\
9465.64918278304	1.07680977363719e-05\\
9536.48280031448	1.06462080654041e-05\\
9607.84648200483	1.0526475512771e-05\\
9679.74419444542	1.04088300342941e-05\\
9752.17993391047	1.02932036211323e-05\\
9825.15772657923	1.01795302590648e-05\\
9898.68162875981	1.00677458880738e-05\\
9972.75572711457	9.95778836224812e-06\\
10047.3841388873	9.84959741002904e-06\\
10122.5710121321	9.74311459481723e-06\\
10198.3205259438	9.63828327595872e-06\\
10274.6368906905	9.53504857012637e-06\\
10351.5243482474	9.43335731311225e-06\\
10428.9871722326	9.33315802204476e-06\\
10507.0296682446	9.23440085804338e-06\\
10585.6561741017	9.13703758932292e-06\\
10664.8710600833	9.0410215547579e-06\\
10744.6787291723	8.94630762791653e-06\\
10825.0836173003	8.85285218157334e-06\\
10906.0901935939	8.76061305270767e-06\\
10987.7029606233	8.66954950799541e-06\\
11069.9264546524	8.57962220979976e-06\\
11152.765245891	8.49079318266594e-06\\
11236.2239387488	8.40302578032488e-06\\
11320.3071720913	8.31628465320905e-06\\
11405.019619498	8.2305357164837e-06\\
11490.3659895215	8.14574611859594e-06\\
11576.3510259497	8.06188421034303e-06\\
11662.9795080693	7.97891951446145e-06\\
11750.2562509318	7.89682269573708e-06\\
11838.1861056203	7.81556553163655e-06\\
11926.7739595202	7.73512088345947e-06\\
12016.0247365899	7.65546266801017e-06\\
12105.9433976352	7.57656582978818e-06\\
12196.5349405845	7.49840631369512e-06\\
12287.8044007671	7.42096103825573e-06\\
12379.7568511926	7.34420786935104e-06\\
12472.3974028332	7.26812559445976e-06\\
12565.7312049077	7.19269389740519e-06\\
12659.7634451676	7.11789333360385e-06\\
12754.4993501855	7.04370530581125e-06\\
12849.9441856459	6.97011204036153e-06\\
12946.1032566372	6.89709656389508e-06\\
13042.9819079474	6.82464268057021e-06\\
13140.5855243605	6.75273494975325e-06\\
13238.9195309561	6.68135866418144e-06\\
13337.9893934111	6.61049982859342e-06\\
13437.8006183031	6.54014513882101e-06\\
13538.3587534167	6.47028196133586e-06\\
13639.669388052	6.40089831324539e-06\\
13741.738153335	6.33198284273038e-06\\
13844.5707225307	6.26352480991811e-06\\
13948.1728113584	6.19551406818375e-06\\
14052.5501783096	6.12794104587262e-06\\
14157.7086249677	6.06079672843639e-06\\
14263.6539963308	5.99407264097501e-06\\
14370.3921811364	5.92776083117716e-06\\
14477.9291121888	5.86185385265109e-06\\
14586.2707666889	5.79634474863773e-06\\
14695.4231665662	5.73122703609805e-06\\
14805.3923788139	5.66649469016638e-06\\
14916.1845158256	5.60214212896105e-06\\
15027.8057357356	5.53816419874411e-06\\
15140.2622427609	5.47455615942127e-06\\
15253.5602875458	5.41131367037351e-06\\
15367.70616751	5.34843277661179e-06\\
15482.7062271979	5.28590989524541e-06\\
15598.5668586318	5.22374180225619e-06\\
15715.2945016669	5.1619256195686e-06\\
15832.8956443492	5.10045880240771e-06\\
15951.3768232763	5.0393391269358e-06\\
16070.7446239608	4.97856467815879e-06\\
16191.0056811961	4.91813383809405e-06\\
16312.166679425	4.85804527419039e-06\\
16434.234353112	4.79829792799215e-06\\
16557.2154871169	4.7388910040386e-06\\
16681.116917072	4.67982395899018e-06\\
16805.9455297624	4.62109649097372e-06\\
16931.7082635086	4.56270852913831e-06\\
17058.4121085521	4.50466022341381e-06\\
17186.0641074439	4.44695193446481e-06\\
17314.6713554361	4.38958422383195e-06\\
17444.2410008763	4.33255784425395e-06\\
17574.7802456044	4.27587373016304e-06\\
17706.2963453539	4.21953298834712e-06\\
17838.7966101542	4.16353688877263e-06\\
17972.2884047374	4.10788685556141e-06\\
18106.7791489477	4.05258445811639e-06\\
18242.2763181536	3.99763140239051e-06\\
18378.7874436634	3.94302952229346e-06\\
18516.3201131441	3.88878077123217e-06\\
18654.8819710428	3.83488721378021e-06\\
18794.480719012	3.78135101747239e-06\\
18935.1241163369	3.72817444472106e-06\\
19076.8199803677	3.67535984485081e-06\\
19219.5761869535	3.6229096462489e-06\\
19363.4006708799	3.57082634862936e-06\\
19508.3014263109	3.51911251540818e-06\\
19654.286507232	3.46777076618907e-06\\
19801.3640278989	3.41680376935766e-06\\
19949.5421632881	3.36621423478426e-06\\
20098.8291495512	3.31600490663457e-06\\
20249.233284473	3.26617855628822e-06\\
20400.7629279322	3.21673797536634e-06\\
20553.4265023667	3.16768596886814e-06\\
20707.2324932414	3.11902534841853e-06\\
20862.1894495195	3.07075892562783e-06\\
21018.3059841386	3.02288950556575e-06\\
21175.5907744885	2.9754198803518e-06\\
21334.0525628939	2.92835282286471e-06\\
21493.7001571006	2.88169108057346e-06\\
21654.5424307645	2.83543736949342e-06\\
21816.5883239452	2.78959436827036e-06\\
21979.8468436028	2.7441647123962e-06\\
22144.3270640985	2.69915098856019e-06\\
22310.0381276992	2.65455572913894e-06\\
22476.9892450851	2.61038140683005e-06\\
22645.1896958625	2.5666304294325e-06\\
22814.6488290788	2.52330513477876e-06\\
22985.3760637425	2.48040778582232e-06\\
23157.3808893468	2.43794056588508e-06\\
23330.6728663967	2.39590557406876e-06\\
23505.261626941	2.35430482083459e-06\\
23681.1568751072	2.31314022375522e-06\\
23858.3683876408	2.27241360344317e-06\\
24036.9060144492	2.23212667965937e-06\\
24216.7796791487	2.19228106760586e-06\\
24397.9993796163	2.1528782744062e-06\\
24580.5751885457	2.11391969577667e-06\\
24764.5172540065	2.07540661289203e-06\\
24949.8358000088	2.0373401894482e-06\\
25136.5411270713	1.99972146892486e-06\\
25324.6436127937	1.96255137205036e-06\\
25514.153712434	1.92583069447083e-06\\
25705.0819594888	1.8895601046256e-06\\
25897.4389662797	1.85374014182995e-06\\
26091.2354245425	1.81837121456672e-06\\
26286.4821060218	1.78345359898723e-06\\
26483.1898630694	1.7489874376219e-06\\
26681.3696292481	1.71497273830076e-06\\
26881.0324199387	1.68140937328324e-06\\
27082.189332953	1.64829707859649e-06\\
27284.8515491498	1.61563545358128e-06\\
27489.0303330572	1.58342396064351e-06\\
27694.7370334982	1.55166192520976e-06\\
27901.9830842215	1.52034853588434e-06\\
28110.7800045375	1.48948284480501e-06\\
28321.1393999579	1.4590637681946e-06\\
28533.0729628413	1.4290900871046e-06\\
28746.5924730428	1.3995604483472e-06\\
28961.7097985688	1.37047336561151e-06\\
29178.4368962368	1.34182722075925e-06\\
29396.78581234	1.31362026529528e-06\\
29616.7686833165	1.2858506220077e-06\\
29838.3977364244	1.25851628677191e-06\\
30061.6852904209	1.23161513051319e-06\\
30286.6437562476	1.20514490132143e-06\\
30513.2856377198	1.17910322671207e-06\\
30741.6235322216	1.1534876160266e-06\\
30971.6701314066	1.12829546296592e-06\\
31203.4382219025	1.10352404825002e-06\\
31436.9406860226	1.07917054239646e-06\\
31672.1905024814	1.05523200861097e-06\\
31909.2007471158	1.03170540578273e-06\\
32147.9845936127	1.00858759157686e-06\\
32388.5553142404	9.85875325616963e-07\\
32630.9262805866	9.63565272750046e-07\\
32875.1109643017	9.41654006386391e-07\\
33121.1229378475	9.2013801190693e-07\\
33368.9758752518	8.99013690130526e-07\\
33618.6835528682	8.78277360833837e-07\\
33870.2598501416	8.5792526631633e-07\\
34123.7187503806	8.37953575003032e-07\\
34379.0743415334	8.18358385078041e-07\\
34636.340816972	7.99135728141402e-07\\
34895.5324762807	7.80281572882597e-07\\
35156.6637260502	7.61791828763744e-07\\
35419.7490806799	7.4366234970575e-07\\
35684.8031631832	7.25888937771133e-07\\
35951.840706001	7.08467346836968e-07\\
36220.8765518205	6.91393286252035e-07\\
36491.9256543999	6.74662424472216e-07\\
36765.0030794002	6.58270392668397e-07\\
37040.1240052216	6.42212788301619e-07\\
37317.3037238483	6.26485178660056e-07\\
37596.5576416978	6.11083104353041e-07\\
37877.9012804769	5.96002082757361e-07\\
38161.3502780454	5.81237611411348e-07\\
38446.9203892846	5.66785171352645e-07\\
38734.6274869731	5.52640230395693e-07\\
39024.4875626694	5.38798246345207e-07\\
39316.5167276	5.252546701424e-07\\
39610.7312135559	5.12004948940636e-07\\
39907.1473737941	4.99044529107769e-07\\
40205.7816839466	4.8636885915253e-07\\
40506.6507429366	4.73973392572522e-07\\
40809.7712739007	4.61853590621893e-07\\
41115.1601251185	4.50004924996643e-07\\
41422.8342709494	4.38422880436104e-07\\
41732.8108127753	4.27102957239173e-07\\
42045.1069799522	4.16040673694139e-07\\
42359.7401307671	4.05231568421254e-07\\
42676.7277534027	3.94671202627325e-07\\
42996.0874669105	3.84355162271827e-07\\
43317.8370221886	3.7427906014436e-07\\
43641.9943029698	3.64438537853237e-07\\
43968.5773268145	3.54829267725429e-07\\
44297.6042461126	3.45446954618106e-07\\
44629.0933490932	3.36287337642207e-07\\
44963.0630608394	3.27346191798742e-07\\
45299.531944314	3.18619329528483e-07\\
45638.5187013907	3.10102602176053e-07\\
45980.0421738933	3.0179190136941e-07\\
46324.1213446437	2.93683160315877e-07\\
46670.775338516	2.85772355016084e-07\\
47020.0234235008	2.78055505397096e-07\\
47371.885011775	2.70528676366309e-07\\
47726.3796607813	2.63187978787647e-07\\
48083.5270743154	2.56029570381713e-07\\
48443.347103621	2.49049656551639e-07\\
48805.8597484928	2.42244491136419e-07\\
49171.0851583894	2.35610377093549e-07\\
49539.0436335514	2.29143667112926e-07\\
49909.7556261313	2.22840764163865e-07\\
50283.2417413299	2.16698121977272e-07\\
50659.5227385407	2.10712245464948e-07\\
51038.6195325055	2.04879691078001e-07\\
51420.5531944749	1.99197067106502e-07\\
51805.3449533811	1.93661033922301e-07\\
52193.0161970173	1.88268304167162e-07\\
52583.5884732259	1.83015642888199e-07\\
52977.0834910972	1.77899867622664e-07\\
53373.5231221757	1.72917848434146e-07\\
53772.929401675	1.68066507902165e-07\\
54175.3245297041	1.63342821067157e-07\\
54580.7308724997	1.58743815332869e-07\\
54989.1709636708	1.54266570328049e-07\\
55400.6675054502	1.49908217729411e-07\\
55815.2433699566	1.45665941047732e-07\\
56232.9216004666	1.4153697537894e-07\\
56653.725412694	1.37518607122047e-07\\
57077.6781960819	1.33608173665658e-07\\
57504.8035151016	1.2980306304485e-07\\
57935.1251105625	1.26100713570102e-07\\
58368.6669009326	1.22498613429927e-07\\
58805.4529836665	1.18994300268868e-07\\
59245.5076365459	1.15585360742376e-07\\
59688.8553190289	1.12269430050146e-07\\
60135.5206736089	1.09044191449381e-07\\
60585.528527185	1.05907375749411e-07\\
61038.9038924417	1.02856760789084e-07\\
61495.6719692387	9.98901708982678e-08\\
61955.8581460127	9.70054763447465e-08\\
62419.4880011873	9.4200592767814e-08\\
62886.5873045955	9.1473480599722e-08\\
63357.182018912	8.88221444761901e-08\\
63831.2983010957	8.62446326370847e-08\\
64308.9625038448	8.37390363183351e-08\\
64790.2011770598	8.13034891361503e-08\\
65275.0410693208	7.8936166464493e-08\\
65763.5091293736	7.66352848067902e-08\\
66255.6325076273	7.43991011627767e-08\\
66751.438557664	7.22259123913375e-08\\
67250.9548377589	7.01140545701887e-08\\
67754.209112412	6.80619023531796e-08\\
68261.2293538915	6.60678683259729e-08\\
68772.0437437882	6.41304023608197e-08\\
69286.6806745825	6.22479909711063e-08\\
69805.1687512222	6.04191566663251e-08\\
70327.5367927121	5.86424573080802e-08\\
70853.8138337169	5.69164854676983e-08\\
71384.0291261735	5.5239867786013e-08\\
71918.2121409184	5.36112643358153e-08\\
72456.392569325	5.20293679874757e-08\\
72998.6003249533	5.04929037781905e-08\\
73544.8655452147	4.9000628285276e-08\\
74095.2185930443	4.75513290039299e-08\\
74649.690058591	4.61438237298183e-08\\
75208.3107609164	4.47769599468569e-08\\
75771.111749708	4.34496142205075e-08\\
76338.1243070055	4.21606915968924e-08\\
76909.3799489393	4.0909125008016e-08\\
77484.9104274818	3.96938746833528e-08\\
78064.7477322134	3.85139275680356e-08\\
78648.9240920989	3.73682967478784e-08\\
79237.4719772807	3.62560208814204e-08\\
79830.4241008822	3.51761636391857e-08\\
80427.8134208264	3.41278131503217e-08\\
81029.6731416689	3.31100814567591e-08\\
81636.0367164413	3.21221039750405e-08\\
82246.937848513	3.11630389659191e-08\\
82862.4104934629	3.02320670118442e-08\\
83482.488860967	2.93283905024168e-08\\
84107.2074167008	2.84512331278893e-08\\
84736.6008842538	2.75998393807799e-08\\
85370.7042470603	2.6773474065645e-08\\
86009.5527503438	2.5971421817056e-08\\
86653.1819030753	2.51929866258054e-08\\
87301.627479948	2.44374913733613e-08\\
87954.9255233655	2.37042773745818e-08\\
88613.1123454441	2.29927039286863e-08\\
89276.2245300332	2.23021478784727e-08\\
89944.2989347464	2.16320031777705e-08\\
90617.3726930118	2.09816804670936e-08\\
91295.4832161355	2.03506066574694e-08\\
91978.6681953803	1.97382245223999e-08\\
92666.9656040625	1.91439922979074e-08\\
93360.4136996602	1.85673832906187e-08\\
94059.0510259418	1.80078854938213e-08\\
94762.9164151073	1.74650012114346e-08\\
95472.0489899465	1.6938246689826e-08\\
96186.4881660146	1.64271517573969e-08\\
96906.2736538223	1.59312594718643e-08\\
97631.4454610423	1.54501257751553e-08\\
98362.0438947356	1.49833191558281e-08\\
99098.1095635883	1.45304203189359e-08\\
99839.6833801717	1.40910218632365e-08\\
100586.806563215	1.36647279656576e-08\\
101339.520639896	1.32511540729209e-08\\
102097.867448152	1.28499266002228e-08\\
102861.889138999	1.24606826368761e-08\\
103631.628178883	1.20830696588046e-08\\
104407.127352034	1.17167452477904e-08\\
105188.429762846	1.13613768173657e-08\\
105975.578838274	1.10166413452436e-08\\
106768.618330246	1.068222511218e-08\\
107567.592318097	1.03578234471564e-08\\
108372.545211017	1.00431404787775e-08\\
109183.521750518	9.73788889277073e-09\\
110000.567012926	9.44178969547948e-09\\
110823.726411883	9.15457198323946e-09\\
111653.04570087	8.87597271752857e-09\\
112488.570975754	8.6057365057784e-09\\
113330.348677345	8.34361538774089e-09\\
114178.425593984	8.08936862729677e-09\\
115032.848864136	7.84276250959983e-09\\
115893.665979016	7.60357014344764e-09\\
116760.924785228	7.37157126876963e-09\\
117634.673487419	7.14655206912852e-09\\
118514.960650966	6.92830498912554e-09\\
119401.835204671	6.71662855660673e-09\\
120295.34644348	6.51132720956498e-09\\
121195.544031226	6.31221112763442e-09\\
122102.478003387	6.11909606807569e-09\\
123016.198769868	5.93180320615105e-09\\
123936.757117803	5.75015897978854e-09\\
124864.204214378	5.57399493843899e-09\\
125798.591609674	5.40314759602604e-09\\
126739.971239535	5.23745828789557e-09\\
127688.395428448	5.07677303166916e-09\\
128643.916892463	4.92094239190809e-09\\
129606.588742111	4.76982134849811e-09\\
130576.464485363	4.62326916866243e-09\\
131553.598030603	4.48114928251648e-09\\
132538.043689621	4.34332916207619e-09\\
133529.856180639	4.20968020363411e-09\\
134529.090631344	4.08007761341998e-09\\
135535.802581959	3.95440029646265e-09\\
136550.047988325	3.83253074857229e-09\\
137571.883225014	3.71435495136392e-09\\
138601.365088463	3.59976227024346e-09\\
139638.550800127	3.48864535528071e-09\\
140683.498009666	3.38090004489404e-09\\
141736.264798142	3.27642527227297e-09\\
142796.909681254	3.17512297446789e-09\\
143865.491612584	3.07689800407531e-09\\
144942.069986881	2.98165804345082e-09\\
146026.704643354	2.88931352138201e-09\\
147119.455869007	2.79977753215514e-09\\
148220.384401982	2.71296575695225e-09\\
149329.55143494	2.62879638751436e-09\\
150447.018618462	2.54719005201035e-09\\
151572.84806447	2.46806974305104e-09\\
152707.102349689	2.39136074778952e-09\\
153849.844519117	2.31699058005091e-09\\
155001.138089531	2.24488891443533e-09\\
156161.047053023	2.17498752233928e-09\\
157329.635880547	2.10722020984276e-09\\
158506.969525512	2.04152275740939e-09\\
159693.113427386	1.97783286134937e-09\\
160888.133515336	1.91609007699578e-09\\
162092.096211894	1.85623576354555e-09\\
163305.068436644	1.79821303051896e-09\\
164527.117609946	1.74196668579093e-09\\
165758.311656683	1.68744318515025e-09\\
166998.719010032	1.63459058334291e-09\\
168248.408615274	1.5833584865571e-09\\
169507.449933624	1.53369800630904e-09\\
170775.912946089	1.48556171468909e-09\\
172053.868157361	1.43890360092925e-09\\
173341.386599734	1.39367902925406e-09\\
174638.539837053	1.34984469797757e-09\\
175945.399968693	1.30735859981072e-09\\
177262.039633563	1.26617998334388e-09\\
178588.532014148	1.22626931567022e-09\\
179924.950840571	1.18758824611733e-09\\
181271.370394698	1.15009957105405e-09\\
182627.865514261	1.11376719974175e-09\\
183994.51159702	1.07855612119925e-09\\
185371.384604954	1.04443237205164e-09\\
186758.561068483	1.01136300533446e-09\\
188156.118090722	9.79316060224816e-10\\
189564.133351766	9.48260532672517e-10\\
190982.685113006	9.18166346904527e-10\\
192411.852221483	8.89004327777053e-10\\
193851.71411427	8.60746173950311e-10\\
195302.350822881	8.33364431861672e-10\\
196763.842977729	8.06832470473399e-10\\
198236.2718126	7.81124456772415e-10\\
199719.719169172	7.56215331999405e-10\\
201214.267501562	7.32080788585946e-10\\
202719.999880911	7.08697247778536e-10\\
204237	6.86041837929025e-10\\
};
\addlegendentry{Adatt. BLN esatto (\textit{ML})}


\addplot[area legend, draw=none, fill=mycolor1, fill opacity=0.15, forget plot]
table[row sep=crcr] {%
x	y\\
119	1.70976723735557e-06\\
119.890504214077	1.73745941812425e-06\\
120.787672274837	1.76555901748268e-06\\
121.691554049369	1.79407237051189e-06\\
122.602199777928	1.82300596282724e-06\\
123.519660076729	1.85236643597324e-06\\
124.44398594076	1.88216059301453e-06\\
125.375228746615	1.91239540432911e-06\\
126.313440255352	1.94307801360966e-06\\
127.258672615369	1.97421574407924e-06\\
128.2109783653	2.00581610492735e-06\\
129.17041043694	2.03788679797302e-06\\
130.137022158185	2.07043572456105e-06\\
131.110867255994	2.10347099269826e-06\\
132.091999859379	2.13700092443624e-06\\
133.080474502409	2.17103406350753e-06\\
134.076346127247	2.20557918322207e-06\\
135.0796700872	2.24064529463101e-06\\
136.090502149795	2.27624165496485e-06\\
137.108898499881	2.3123777763534e-06\\
138.134915742751	2.34906343483458e-06\\
139.168610907289	2.38630867965982e-06\\
140.21004144914	2.4241238429033e-06\\
141.259265253899	2.46251954938281e-06\\
142.316340640336	2.50150672689989e-06\\
143.381326363632	2.5410966168071e-06\\
144.454281618647	2.5813007849102e-06\\
145.535266043209	2.62213113271323e-06\\
146.624339721429	2.66359990901456e-06\\
147.721563187044	2.70571972186194e-06\\
148.826997426776	2.74850355087471e-06\\
149.940703883725	2.7919647599414e-06\\
151.062744460785	2.83611711030093e-06\\
152.193181524082	2.88097477401574e-06\\
153.332077906443	2.92655234784516e-06\\
154.479496910888	2.9728648675274e-06\\
155.635502314145	3.01992782247846e-06\\
156.800158370201	3.06775717091649e-06\\
157.97352981387	3.1163693554198e-06\\
159.15568186439	3.1657813189271e-06\\
160.34668022905	3.21601052118813e-06\\
161.546591106842	3.26707495567328e-06\\
162.755481192139	3.31899316695017e-06\\
163.973417678405	3.37178426853577e-06\\
165.200468261928	3.42546796123207e-06\\
166.436701145581	3.48006455195341e-06\\
167.682185042617	3.53559497305368e-06\\
168.936989180483	3.59208080216115e-06\\
170.201183304674	3.64954428252888e-06\\
171.474837682604	3.70800834390833e-06\\
172.758023107516	3.76749662395392e-06\\
174.050810902413	3.82803349016568e-06\\
175.353272924028	3.8896440623776e-06\\
176.665481566809	3.95235423579837e-06\\
177.987509766954	4.01619070461169e-06\\
179.319431006454	4.08118098614259e-06\\
180.661319317186	4.14735344559632e-06\\
182.013249285024	4.21473732137587e-06\\
183.375296053983	4.28336275098398e-06\\
184.7475353304	4.35326079751538e-06\\
186.13004338714	4.42446347674434e-06\\
187.522897067834	4.49700378481261e-06\\
188.926173791152	4.57091572652229e-06\\
190.339951555105	4.64623434423775e-06\\
191.764308941382	4.72299574740059e-06\\
193.199325119717	4.80123714266076e-06\\
194.645079852288	4.880996864627e-06\\
196.101653498152	4.96231440723886e-06\\
197.569127017711	5.04523045576233e-06\\
199.047581977213	5.12978691941045e-06\\
200.537100553285	5.21602696458975e-06\\
202.037765537499	5.30399504877275e-06\\
203.549660340977	5.3937369549962e-06\\
205.072868999023	5.48529982698393e-06\\
206.6074761758	5.57873220489269e-06\\
208.15356716903	5.6740840616785e-06\\
209.711227914737	5.77140684008018e-06\\
211.280544992026	5.87075349021641e-06\\
212.86160562789	5.97217850779085e-06\\
214.454497702065	6.07573797290032e-06\\
216.059309751909	6.18148958943864e-06\\
217.676130977326	6.28949272508904e-06\\
219.305051245723	6.39980845189605e-06\\
220.946161097006	6.51249958740732e-06\\
222.599551748611	6.62763073637432e-06\\
224.265315100576	6.74526833299964e-06\\
225.943543740646	6.86548068371785e-06\\
227.634330949423	6.98833801049448e-06\\
229.33777070555	7.11391249462769e-06\\
231.05395769093	7.24227832103392e-06\\
232.782987295997	7.37351172299911e-06\\
234.524955625009	7.50769102737356e-06\\
236.279959501398	7.64489670018814e-06\\
238.048096473146	7.78521139266657e-06\\
239.829464818207	7.92871998760698e-06\\
241.624163549975	8.07550964610381e-06\\
243.432292422782	8.22566985457797e-06\\
245.253951937446	8.37929247208217e-06\\
247.089243346851	8.53647177784344e-06\\
248.938268661586	8.69730451900458e-06\\
250.801130655604	8.86188995852053e-06\\
252.677932871941	9.03032992316464e-06\\
254.568779628468	9.2027288515943e-06\\
256.473776023691	9.37919384242332e-06\\
258.393027942593	9.55983470224239e-06\\
260.326642062517	9.74476399352667e-06\\
262.274725859099	9.93409708236269e-06\\
264.237387612236	1.01279521859231e-05\\
266.214736412113	1.03264504196117e-05\\
268.206882165259	1.05297158437953e-05\\
270.213935600659	1.07378755100328e-05\\
272.236008275909	1.09510595067045e-05\\
274.273212583414	1.11694010039386e-05\\
276.325661756641	1.13930362977229e-05\\
278.393469876404	1.16221048530827e-05\\
280.476751877215	1.18567493461943e-05\\
282.575623553662	1.20971157053001e-05\\
284.690201566855	1.23433531502737e-05\\
286.820603450903	1.25956142306802e-05\\
288.96694761945	1.28540548621623e-05\\
291.129353372257	1.31188343609708e-05\\
293.307940901833	1.33901154764514e-05\\
295.502831300113	1.36680644212809e-05\\
297.714146565192	1.395285089924e-05\\
299.942009608104	1.42446481302924e-05\\
302.186544259657	1.45436328727315e-05\\
304.447875277308	1.48499854421404e-05\\
306.726128352108	1.51638897269005e-05\\
309.02143011568	1.54855331999733e-05\\
311.333908147261	1.58151069266646e-05\\
313.663690980792	1.61528055680775e-05\\
316.010908112063	1.64988273799437e-05\\
318.375690005912	1.68533742065216e-05\\
320.758168103475	1.72166514692419e-05\\
323.158474829489	1.75888681497771e-05\\
325.57674359966	1.79702367672119e-05\\
328.013108828071	1.83609733489959e-05\\
330.467705934659	1.87612973953603e-05\\
332.940671352735	1.91714318368945e-05\\
335.432142536577	1.95916029849915e-05\\
337.942257969062	2.00220404748834e-05\\
340.471157169366	2.04629772010184e-05\\
343.018980700718	2.09146492445507e-05\\
345.585870178217	2.13772957927497e-05\\
348.171968276697	2.18511590501698e-05\\
350.777418738662	2.23364841414634e-05\\
353.402366382274	2.28335190057644e-05\\
356.046957109401	2.33425142826175e-05\\
358.711337913731	2.38637231894812e-05\\
361.395656888935	2.43974013908853e-05\\
364.100063236907	2.4943806859382e-05\\
366.824707276051	2.5503199728481e-05\\
369.569740449639	2.60758421378199e-05\\
372.335315334224	2.66619980708698e-05\\
375.12158564813	2.72619331855296e-05\\
377.928706259986	2.78759146380065e-05\\
380.75683319734	2.85042109004201e-05\\
383.60612365533	2.91470915726025e-05\\
386.476736005421	2.98048271885903e-05\\
389.368829804207	3.04776890183204e-05\\
392.282565802281	3.1165948865052e-05\\
395.21810595317	3.18698788590308e-05\\
398.175613422337	3.25897512479019e-05\\
401.155252596246	3.3325838184355e-05\\
404.157189091507	3.40784115114537e-05\\
407.181589764074	3.4847742546064e-05\\
410.228622718523	3.56341018607486e-05\\
413.298457317395	3.64377590644414e-05\\
416.391264190611	3.72589825821606e-05\\
419.507215244952	3.80980394339575e-05\\
422.646483673618	3.89551950132393e-05\\
425.809243965854	3.98307128645379e-05\\
428.995671916648	4.07248544607446e-05\\
432.205944636502	4.16378789797666e-05\\
435.440240561274	4.2570043080515e-05\\
438.698739462102	4.35216006780853e-05\\
441.981622455389	4.44928027179524e-05\\
445.289072012877	4.54838969489674e-05\\
448.621271971783	4.64951276949211e-05\\
451.978407545023	4.75267356244193e-05\\
455.360665331498	4.85789575188047e-05\\
458.768233326478	4.96520260378605e-05\\
462.201300932039	5.07461694830271e-05\\
465.660058967601	5.1861611557887e-05\\
469.144699680525	5.29985711256726e-05\\
472.655416756805	5.41572619635894e-05\\
476.192405331832	5.53378925137593e-05\\
479.755862001239	5.65406656306259e-05\\
483.345984831829	5.77657783246926e-05\\
486.962973372584	5.90134215024969e-05\\
490.607028665758	6.02837797027634e-05\\
494.278353258049	6.15770308287071e-05\\
497.977151211859	6.28933458765034e-05\\
501.703628116633	6.4232888659968e-05\\
505.457991100293	6.5595815531532e-05\\
509.240448840744	6.69822750996262e-05\\
513.051211577476	6.83924079426241e-05\\
516.890491123249	6.98263463195242e-05\\
520.758500875867	7.12842138775813e-05\\
524.655455830038	7.27661253571219e-05\\
528.581572589324	7.42721862938084e-05\\
532.537069378183	7.58024927186425e-05\\
536.522166054094	7.73571308560115e-05\\
540.537084119782	7.89361768201169e-05\\
544.582046735526	8.05396963101295e-05\\
548.657278731565	8.21677443044422e-05\\
552.763006620593	8.38203647544058e-05\\
556.899458610353	8.54975902779488e-05\\
561.066864616317	8.71994418534915e-05\\
565.265456274466	8.89259285145847e-05\\
569.495466954168	9.06770470457129e-05\\
573.757131771146	9.24527816797051e-05\\
578.050687600549	9.42531037972209e-05\\
582.376373090114	9.60779716287724e-05\\
586.734428673439	9.79273299597663e-05\\
591.125096583335	9.98011098390419e-05\\
595.548620865302	0.000101699228291406\\
600.005247391086	0.000103621588034655\\
604.495223872346	0.000105568077201591\\
609.018799874428	0.000107538569067542\\
613.576226830228	0.000109532921783897\\
618.167758054176	0.000111550978118178\\
622.793648756308	0.000113592565201164\\
627.454156056458	0.000115657494281602\\
632.149538998544	0.000117745560489023\\
636.880058564972	0.000119856542605202\\
641.645977691138	0.000121990202844781\\
646.447561280041	0.000124146286645606\\
651.285076217013	0.000126324522469292\\
656.158791384547	0.000128524621612572\\
661.06897767725	0.000130746278029941\\
666.015908016889	0.000132989168168143\\
670.999857367572	0.00013525295081303\\
676.021102751025	0.000137537266949314\\
681.079923261988	0.000139841739633754\\
686.176600083736	0.000142165973882284\\
691.311416503699	0.000144509556571616\\
696.484657929211	0.000146872056355825\\
701.696611903378	0.000149253023598429\\
706.947568121055	0.000151651990320475\\
712.237818444947	0.000154068470165112\\
717.56765692184	0.000156501958379165\\
722.937379798933	0.000158951931812179\\
728.347285540317	0.000161417848933413\\
733.797674843554	0.000163899149867254\\
739.288850656394	0.000166395256447505\\
744.821118193618	0.000168905572290999\\
750.394784953994	0.000171429482890973\\
756.01016073738	0.000173966355730632\\
761.667557661931	0.00017651554041731\\
767.367290181458	0.000179076368837647\\
773.109675102898	0.000181648155334142\\
778.89503160393	0.000184230196903491\\
784.72368125071	0.000186821773417038\\
790.59594801575	0.000189422147863714\\
796.512158295919	0.00019203056661576\\
802.472640930591	0.00019464625971758\\
808.47772721992	0.000197268441197984\\
814.527750943253	0.000199896309406115\\
820.623048377686	0.000202529047371307\\
826.763958316753	0.000205165823187091\\
832.950822089257	0.000207805790419568\\
839.183983578243	0.000210448088540301\\
845.463789240111	0.000213091843383893\\
851.790588123873	0.000215736167630333\\
858.164731890556	0.000218380161312186\\
864.586574832748	0.000221022912346643\\
871.056473894285	0.00022366349709239\\
877.574788690099	0.000226300980931194\\
884.141881526201	0.000228934418874038\\
890.758117419823	0.000231562856191524\\
897.423864119701	0.000234185329068193\\
904.139492126523	0.000236800865280261\\
910.905374713515	0.000239408484896134\\
917.721887947192	0.000242007200998904\\
924.589410708265	0.000244596020429796\\
931.508324712691	0.00024717394455134\\
938.479014532895	0.000249739970028701\\
945.501867619149	0.000252293089627342\\
952.577274321103	0.000254832293024724\\
959.705627909479	0.000257356567633382\\
966.88732459794	0.000259864899432137\\
974.122763565099	0.000262356273801641\\
981.412346976721	0.000264829676359787\\
988.756480008064	0.00026728409379174\\
996.155570866409	0.000269718514668547\\
1003.61003081374	0.000272131930247373\\
1011.12027418962	0.000274523335245434\\
1018.6867184342	0.000276891728578757\\
1026.30978411142	0.000279236114055844\\
1033.98989493243	0.000281555501015441\\
1041.72747777907	0.000283848904896762\\
1049.52296272766	0.000286115347729898\\
1057.37678307286	0.000288353858533872\\
1065.2893753518	0.000290563473609892\\
1073.26117936829	0.000292743236718084\\
1081.2926382173	0.000294892199127414\\
1089.38419830959	0.000297009419530791\\
1097.53630939651	0.000299093963820665\\
1105.749424595	0.000301144904724802\\
1114.02400041277	0.000303161321307418\\
1122.3604967737	0.000305142298347379\\
1130.75937704337	0.000307086925612559\\
1139.22110805483	0.000308994297057345\\
1147.74616013456	0.000310863509978227\\
1156.3350071286	0.000312693664169839\\
1164.98812642888	0.000314483861129951\\
1173.70599899975	0.000316233203366178\\
1182.48910940477	0.000317940793858704\\
1191.33794583355	0.000319605735731799\\
1200.25300012896	0.000321227132181907\\
1209.23476781446	0.000322804086701721\\
1218.28374812157	0.000324335703628304\\
1227.40044401774	0.000325821089029717\\
1236.58536223419	0.000327259351929699\\
1245.83901329415	0.000328649605855041\\
1255.16191154121	0.000329990970676297\\
1264.5545751679	0.000331282574700795\\
1274.01752624451	0.000332523556967967\\
1283.55129074812	0.000333713069691547\\
1293.15639859177	0.000334850280791221\\
1302.83338365401	0.000335934376457516\\
1312.5827838085	0.000336964563697814\\
1322.40514095393	0.000337940072817465\\
1332.30100104415	0.000338860159797556\\
1342.27091411851	0.000339724108539056\\
1352.31543433242	0.00034053123295123\\
1362.43511998817	0.000341280878870018\\
1372.63053356595	0.000341972425798863\\
1382.90224175512	0.000342605288470352\\
1393.2508154857	0.000343178918231685\\
1403.67682996012	0.000343692804260515\\
1414.18086468518	0.000344146474620176\\
1424.76350350425	0.000344539497164911\\
1435.42533462974	0.00034487148030647\\
1446.16695067579	0.000345142073653649\\
1456.98894869122	0.000345350968536075\\
1467.89193019267	0.000345497898422936\\
1478.8765011981	0.00034558263924653\\
1489.94327226042	0.000345605009639619\\
1501.09285850146	0.000345564871094531\\
1512.32587964614	0.000345462128051045\\
1523.64296005691	0.000345296727919105\\
1535.0447287685	0.000345068661041595\\
1546.53181952283	0.000344777960601579\\
1558.10487080425	0.000344424702477753\\
1569.76452587505	0.000344009005051229\\
1581.51143281119	0.000343531028966279\\
1593.34624453832	0.000342990976847208\\
1605.26961886811	0.000342389092973191\\
1617.28221853476	0.00034172566291261\\
1629.38471123188	0.000341001013118197\\
1641.57776964957	0.000340215510484121\\
1653.86207151182	0.00033936956186602\\
1666.2382996142	0.000338463613564889\\
1678.70714186178	0.000337498150775687\\
1691.26929130741	0.000336473697001472\\
1703.92544619016	0.000335390813433894\\
1716.67630997424	0.000334250098300854\\
1729.522591388	0.000333052186182199\\
1742.4650044634	0.000331797747294325\\
1755.50426857564	0.000330487486744612\\
1768.64110848317	0.000329122143756706\\
1781.876254368	0.000327702490867648\\
1795.21044187622	0.000326229333097977\\
1808.64441215895	0.000324703507095946\\
1822.17891191353	0.00032312588025709\\
1835.81469342496	0.000321497349820385\\
1849.55251460781	0.000319818841942358\\
1863.39313904828	0.000318091310750501\\
1877.33733604663	0.000316315737377409\\
1891.38588066002	0.000314493128977095\\
1905.53955374551	0.000312624517724943\\
1919.7991420035	0.000310710959802789\\
1934.16543802145	0.000308753534370603\\
1948.63924031791	0.000306753342526221\\
1963.22135338698	0.00030471150625454\\
1977.91258774292	0.000302629167367526\\
1992.71375996528	0.00030050748643625\\
2007.62569274426	0.000298347641716084\\
2022.64921492642	0.000296150828065969\\
2037.7851615608	0.000293918255862455\\
2053.03437394529	0.000291651149908939\\
2068.39769967338	0.000289350748340152\\
2083.87599268133	0.000287018301521553\\
2099.47011329558	0.000284655070942732\\
2115.18092828061	0.000282262328103383\\
2131.00931088707	0.00027984135338963\\
2146.95614090037	0.000277393434937744\\
2163.02230468953	0.000274919867481321\\
2179.20869525649	0.000272421951177037\\
2195.51621228573	0.000269900990402943\\
2211.94576219425	0.000267358292522189\\
2228.49825818201	0.000264795166603906\\
2245.17462028264	0.000262212922091957\\
2261.97577541457	0.000259612867411428\\
2278.90265743261	0.000256996308502196\\
2295.95620717979	0.000254364547268958\\
2313.1373725397	0.00025171887993777\\
2330.44710848916	0.000249060595310844\\
2347.88637715129	0.000246390972914176\\
2365.45614784899	0.000243711281036818\\
2383.15739715885	0.000241022774666361\\
2400.9911089654	0.000238326693332586\\
2418.95827451578	0.000235624258879952\\
2437.05989247487	0.000232916673199506\\
2455.29696898081	0.000230205115961007\\
2473.67051770087	0.000227490742395949\\
2492.18155988785	0.00022477468119036\\
2510.8311244368	0.000222058032551699\\
2529.62024794223	0.000219341866515626\\
2548.54997475574	0.000216627221554877\\
2567.62135704402	0.000213915103543664\\
2586.83545484739	0.000211206485116886\\
2606.1933361387	0.000208502305445127\\
2625.69607688265	0.000205803470425292\\
2645.34476109567	0.000203110853264979\\
2665.14048090611	0.0002004252954184\\
2685.08433661496	0.000197747607814918\\
2705.17743675704	0.000195078572309589\\
2725.42089816257	0.000192418943279296\\
2745.81584601927	0.000189769449288194\\
2766.3634139349	0.000187130794751582\\
2787.06474400025	0.000184503661536828\\
2807.92098685266	0.000181888710452135\\
2828.93330173995	0.000179286582587299\\
2850.10285658484	0.000176697900483781\\
2871.4308280499	0.000174123269123543\\
2892.91840160291	0.00017156327673629\\
2914.56677158281	0.000169018495432899\\
2936.37714126603	0.000166489481678682\\
2958.3507229334	0.000163976776623972\\
2980.48873793751	0.000161480906311524\\
3002.79241677064	0.000159002381780897\\
3025.2629991331	0.000156541699089451\\
3047.90173400216	0.000154099339268373\\
3070.7098797015	0.000151675768230461\\
3093.68870397109	0.000149271436644382\\
3116.83948403772	0.000146886779788135\\
3140.16350668593	0.000144522217392401\\
3163.66206832958	0.000142178153482628\\
3187.33647508389	0.000139854976226973\\
3211.18804283803	0.000137553057795737\\
3235.21809732829	0.000135272754236604\\
3259.42797421173	0.000133014405368927\\
3283.81901914043	0.000130778334699314\\
3308.39258783631	0.000128564849360066\\
3333.15004616647	0.000126374240071357\\
3358.09277021909	0.000124206781127555\\
3383.22214637995	0.000122062730407698\\
3408.53957140944	0.00011994232940981\\
3434.04645252026	0.000117845803308537\\
3459.7442074556	0.000115773361035379\\
3485.63426456793	0.000113725195380671\\
3511.71806289842	0.000111701483116402\\
3537.99705225692	0.000109702385138847\\
3564.47269330252	0.000107728046630024\\
3591.14645762477	0.000105778597236891\\
3618.01982782546	0.000103854151267269\\
3645.09429760103	0.000101954807901431\\
3672.37137182558	0.000100080651418339\\
3699.85256663454	9.82317514355301e-05\\
3727.53940950892	9.6408163161679e-05\\
3755.43343936023	9.46099276608813e-05\\
3783.53620661599	9.2837072127754e-05\\
3811.84927330594	9.10896101724661e-05\\
3840.37421314884	8.9367542114852e-05\\
3869.11261163994	8.76708552867875e-05\\
3898.06606613913	8.59995243420513e-05\\
3927.23618595969	8.43535115729134e-05\\
3956.62459245778	8.27327672327353e-05\\
3986.23291912252	8.11372298638911e-05\\
4016.06281166681	7.95668266303489e-05\\
4046.1159281188	7.80214736542819e-05\\
4076.39393891404	7.65010763561094e-05\\
4106.89852698833	7.50055297973874e-05\\
4137.63138787127	7.35347190260016e-05\\
4168.59422978048	7.20885194231343e-05\\
4199.78877371658	7.06667970515057e-05\\
4231.21675355883	6.92694090044053e-05\\
4262.8799161615	6.7896203755056e-05\\
4294.78002145095	6.65470215058645e-05\\
4326.91884252352	6.5221694537134e-05\\
4359.29816574399	6.39200475548277e-05\\
4391.91979084494	6.264189803698e-05\\
4424.78553102675	6.13870565783631e-05\\
4457.89721305839	6.0155327233019e-05\\
4491.25667737898	5.89465078542705e-05\\
4524.86577820005	5.77603904318128e-05\\
4558.72638360862	5.65967614254839e-05\\
4592.84037567103	5.54554020952856e-05\\
4627.20965053756	5.4336088827207e-05\\
4661.83611854782	5.32385934543624e-05\\
4696.72170433693	5.21626835729118e-05\\
4731.86834694246	5.11081228521676e-05\\
4767.27799991228	5.00746713382265e-05\\
4802.95263141311	4.90620857503728e-05\\
4838.89422433987	4.80701197693976e-05\\
4875.10477642598	4.70985243168583e-05\\
4911.58630035433	4.61470478241706e-05\\
4948.34082386919	4.52154364902663e-05\\
4985.37038988889	4.43034345263979e-05\\
5022.6770566194	4.34107843864905e-05\\
5060.2628976687	4.25372269812773e-05\\
5098.13000216207	4.16825018742937e-05\\
5136.28047485818	4.0846347457674e-05\\
5174.7164362661	4.00285011056083e-05\\
5213.44002276314	3.92286993033033e-05\\
5252.45338671363	3.84466777493806e-05\\
5291.7586965885	3.76821714298749e-05\\
5331.3581370859	3.69349146623918e-05\\
5371.25390925253	3.62046411096044e-05\\
5411.44823060603	3.54910837621242e-05\\
5451.94333525824	3.47939748919038e-05\\
5492.74147403939	3.41130459787208e-05\\
5533.84491462315	3.34480276139089e-05\\
5575.25594165272	3.27986493873043e-05\\
5616.97685686784	3.21646397652275e-05\\
5659.00997923266	3.15457259690967e-05\\
5701.35764506468	3.09416338657651e-05\\
5744.02220816459	3.03520878816876e-05\\
5787.00603994713	2.97768109533528e-05\\
5830.31152957285	2.9215524525886e-05\\
5873.94108408097	2.86679486102448e-05\\
5917.8971285231	2.81338019069919e-05\\
5962.18210609809	2.76128020013529e-05\\
6006.79847828778	2.7104665630405e-05\\
6051.74872499389	2.66091090191018e-05\\
6097.03534467575	2.61258482778338e-05\\
6142.66085448928	2.56545998507123e-05\\
6188.62779042682	2.51950810010922e-05\\
6234.93870745815	2.47470103192149e-05\\
6281.59617967246	2.43101082363633e-05\\
6328.60280042144	2.38840975305172e-05\\
6375.9611824634	2.34687038100195e-05\\
6423.67395810857	2.30636559639787e-05\\
6471.74377936531	2.26686865707525e-05\\
6520.17331808759	2.22835322586212e-05\\
6568.96526612346	2.190793401542e-05\\
6618.12233546469	2.1541637446291e-05\\
6667.64725839753	2.11843929807117e-05\\
6717.54278765452	2.08359560315114e-05\\
6767.81169656754	2.04960871096906e-05\\
6818.45677922192	2.01645518995546e-05\\
6869.48085061182	1.98411212990036e-05\\
6920.88674679662	1.95255714298791e-05\\
6972.67732505857	1.92176836230931e-05\\
7024.85546406162	1.89172443829551e-05\\
7077.42406401143	1.86240453347003e-05\\
7130.38604681656	1.83378831587646e-05\\
7183.7443562509	1.80585595148802e-05\\
7237.50195811723	1.77858809586108e-05\\
7291.66184041212	1.75196588525127e-05\\
7346.22701349204	1.72597092737228e-05\\
7401.20051024062	1.70058529194302e-05\\
7456.58538623724	1.67579150113868e-05\\
7512.38471992689	1.65157252003612e-05\\
7568.60161279127	1.6279117471226e-05\\
7625.23918952119	1.60479300491914e-05\\
7682.30059819021	1.58220053075569e-05\\
7739.78901042966	1.56011896772345e-05\\
7797.70762160489	1.5385333558211e-05\\
7856.05965099295	1.51742912330381e-05\\
7914.84834196143	1.49679207823941e-05\\
7974.0769621488	1.47660840027117e-05\\
8033.748803646	1.45686463258425e-05\\
8093.86718317946	1.43754767407047e-05\\
8154.43544229543	1.4186447716848e-05\\
8215.4569475457	1.4001435129861e-05\\
8276.93509067475	1.38203181885404e-05\\
8338.87328880825	1.36429793637418e-05\\
8401.27498464301	1.34693043188322e-05\\
8464.14364663834	1.32991818416662e-05\\
8527.48276920879	1.31325037780131e-05\\
8591.29587291844	1.29691649663648e-05\\
8655.58650467655	1.28090631740605e-05\\
8720.35823793472	1.26520990346704e-05\\
8785.61467288549	1.24981759865822e-05\\
8851.35943666247	1.23472002127443e-05\\
8917.59618354194	1.21990805815191e-05\\
8984.32859514597	1.20537285886099e-05\\
9051.56038064706	1.19110583000246e-05\\
9119.29527697427	1.1770986296046e-05\\
9187.53704902097	1.16334316161829e-05\\
9256.28948985408	1.14983157050763e-05\\
9325.55642092493	1.1365562359342e-05\\
9395.34169228163	1.12350976753322e-05\\
9465.64918278304	1.11068499977991e-05\\
9536.48280031448	1.09807498694499e-05\\
9607.84648200483	1.08567299813814e-05\\
9679.74419444542	1.0734725124385e-05\\
9752.17993391047	1.06146721411155e-05\\
9825.15772657923	1.04965098791162e-05\\
9898.68162875981	1.03801791446969e-05\\
9972.75572711457	1.02656226576583e-05\\
10047.3841388873	1.01527850068624e-05\\
10122.5710121321	1.00416126066428e-05\\
10198.3205259438	9.93205365405385e-06\\
10274.6368906905	9.8240580869564e-06\\
10351.5243482474	9.71757754293609e-06\\
10428.9871722326	9.61256531905231e-06\\
10507.0296682446	9.50897633241357e-06\\
10585.6561741017	9.40676708157518e-06\\
10664.8710600833	9.30589560875329e-06\\
10744.6787291723	9.20632146284836e-06\\
10825.0836173003	9.10800566326907e-06\\
10906.0901935939	9.01091066454497e-06\\
10987.7029606233	8.91500032171408e-06\\
11069.9264546524	8.82023985646785e-06\\
11152.765245891	8.72659582403126e-06\\
11236.2239387488	8.63403608075241e-06\\
11320.3071720913	8.54252975236898e-06\\
11405.019619498	8.45204720291295e-06\\
11490.3659895215	8.36256000420773e-06\\
11576.3510259497	8.27404090590197e-06\\
11662.9795080693	8.18646380597573e-06\\
11750.2562509318	8.09980372164223e-06\\
11838.1861056203	8.01403676055628e-06\\
11926.7739595202	7.92914009222623e-06\\
12016.0247365899	7.84509191951068e-06\\
12105.9433976352	7.7618714500649e-06\\
12196.5349405845	7.67945886758387e-06\\
12287.8044007671	7.5978353026703e-06\\
12379.7568511926	7.51698280313891e-06\\
12472.3974028332	7.43688430354873e-06\\
12565.7312049077	7.35752359374122e-06\\
12659.7634451676	7.27888528614842e-06\\
12754.4993501855	7.20095478162739e-06\\
12849.9441856459	7.12371823357778e-06\\
12946.1032566372	7.04716251010582e-06\\
13042.9819079474	6.97127515402029e-06\\
13140.5855243605	6.89604434048059e-06\\
13238.9195309561	6.82145883216942e-06\\
13337.9893934111	6.7475079319365e-06\\
13437.8006183031	6.67418143295383e-06\\
13538.3587534167	6.60146956654007e-06\\
13639.669388052	6.5293629479523e-06\\
13741.738153335	6.4578525205987e-06\\
13844.5707225307	6.38692949929929e-06\\
13948.1728113584	6.31658531339611e-06\\
14052.5501783096	6.24681155068434e-06\\
14157.7086249677	6.17759990328641e-06\\
14263.6539963308	6.10894211670514e-06\\
14370.3921811364	6.04082994335763e-06\\
14477.9291121888	5.97325510189232e-06\\
14586.2707666889	5.90620924351685e-06\\
14695.4231665662	5.83968392641229e-06\\
14805.3923788139	5.77367059907672e-06\\
14916.1845158256	5.70816059314185e-06\\
15027.8057357356	5.6431451258554e-06\\
15140.2622427609	5.57861531204229e-06\\
15253.5602875458	5.51456218497839e-06\\
15367.70616751	5.4509767252576e-06\\
15482.7062271979	5.38784989643373e-06\\
15598.5668586318	5.32517268599437e-06\\
15715.2945016669	5.26293615008488e-06\\
15832.8956443492	5.20113146035591e-06\\
15951.3768232763	5.13974995134919e-06\\
16070.7446239608	5.0787831669561e-06\\
16191.0056811961	5.01822290466415e-06\\
16312.166679425	4.95806125652707e-06\\
16434.234353112	4.89829064603945e-06\\
16557.2154871169	4.83890386034308e-06\\
16681.116917072	4.77989407742771e-06\\
16805.9455297624	4.72125488820142e-06\\
16931.7082635086	4.6629803134864e-06\\
17058.4121085521	4.60506481614173e-06\\
17186.0641074439	4.54750330862631e-06\\
17314.6713554361	4.49029115639035e-06\\
17444.2410008763	4.43342417753193e-06\\
17574.7802456044	4.3768986391752e-06\\
17706.2963453539	4.32071125102687e-06\\
17838.7966101542	4.26485915655294e-06\\
17972.2884047374	4.20933992218901e-06\\
18106.7791489477	4.15415152496419e-06\\
18242.2763181536	4.09929233887884e-06\\
18378.7874436634	4.04476112033609e-06\\
18516.3201131441	3.99055699288694e-06\\
18654.8819710428	3.93667943150981e-06\\
18794.480719012	3.88312824660974e-06\\
18935.1241163369	3.82990356788992e-06\\
19076.8199803677	3.77700582821865e-06\\
19219.5761869535	3.72443574758976e-06\\
19363.4006708799	3.6721943172525e-06\\
19508.3014263109	3.62028278406748e-06\\
19654.286507232	3.56870263513097e-06\\
19801.3640278989	3.51745558269518e-06\\
19949.5421632881	3.46654354940318e-06\\
20098.8291495512	3.41596865384806e-06\\
20249.233284473	3.36573319645899e-06\\
20400.7629279322	3.31583964571325e-06\\
20553.4265023667	3.26629062466782e-06\\
20707.2324932414	3.21708889780335e-06\\
20862.1894495195	3.16823735817057e-06\\
21018.3059841386	3.11973901482864e-06\\
21175.5907744885	3.07159698056504e-06\\
21334.0525628939	3.02381445988632e-06\\
21493.7001571006	2.97639473726958e-06\\
21654.5424307645	2.92934116566621e-06\\
21816.5883239452	2.88265715524916e-06\\
21979.8468436028	2.83634616239747e-06\\
22144.3270640985	2.79041167891228e-06\\
22310.0381276992	2.74485722146004e-06\\
22476.9892450851	2.69968632124043e-06\\
22645.1896958625	2.6549025138769e-06\\
22814.6488290788	2.61050932952996e-06\\
22985.3760637425	2.56651028323408e-06\\
23157.3808893468	2.52290886546005e-06\\
23330.6728663967	2.47970853290622e-06\\
23505.261626941	2.43691269952256e-06\\
23681.1568751072	2.39452472777241e-06\\
23858.3683876408	2.35254792013787e-06\\
24036.9060144492	2.31098551087477e-06\\
24216.7796791487	2.26984065802449e-06\\
24397.9993796163	2.22911643568987e-06\\
24580.5751885457	2.1888158265825e-06\\
24764.5172540065	2.14894171484997e-06\\
24949.8358000088	2.10949687919031e-06\\
25136.5411270713	2.07048398626234e-06\\
25324.6436127937	2.03190558439958e-06\\
25514.153712434	1.99376409763564e-06\\
25705.0819594888	1.95606182004899e-06\\
25897.4389662797	1.91880091043415e-06\\
26091.2354245425	1.88198338730672e-06\\
26286.4821060218	1.8456111242487e-06\\
26483.1898630694	1.80968584560032e-06\\
26681.3696292481	1.77420912250419e-06\\
26881.0324199387	1.73918236930704e-06\\
27082.189332953	1.7046068403234e-06\\
27284.8515491498	1.67048362696578e-06\\
27489.0303330572	1.63681365524451e-06\\
27694.7370334982	1.60359768364036e-06\\
27901.9830842215	1.5708363013525e-06\\
28110.7800045375	1.53852992692335e-06\\
28321.1393999579	1.50667880724226e-06\\
28533.0729628413	1.47528301692846e-06\\
28746.5924730428	1.44434245809426e-06\\
28961.7097985688	1.41385686048874e-06\\
29178.4368962368	1.38382578202179e-06\\
29396.78581234	1.35424860966874e-06\\
29616.7686833165	1.32512456075541e-06\\
29838.3977364244	1.29645268462357e-06\\
30061.6852904209	1.26823186467734e-06\\
30286.6437562476	1.240460820811e-06\\
30513.2856377198	1.21313811221976e-06\\
30741.6235322216	1.18626214059552e-06\\
30971.6701314066	1.15983115371026e-06\\
31203.4382219025	1.13384324939153e-06\\
31436.9406860226	1.10829637989467e-06\\
31672.1905024814	1.08318835667855e-06\\
31909.2007471158	1.05851685559202e-06\\
32147.9845936127	1.03427942247972e-06\\
32388.5553142404	1.01047347921644e-06\\
32630.9262805866	9.87096330178585e-07\\
32875.1109643017	9.64145169160641e-07\\
33121.1229378475	9.41617086740526e-07\\
33368.9758752518	9.19509078091792e-07\\
33618.6835528682	8.97818051230189e-07\\
33870.2598501416	8.76540835665392e-07\\
34123.7187503806	8.55674191403513e-07\\
34379.0743415334	8.35214818209421e-07\\
34636.340816972	8.1515936498461e-07\\
34895.5324762807	7.95504439043332e-07\\
35156.6637260502	7.76246614969459e-07\\
35419.7490806799	7.57382442604644e-07\\
35684.8031631832	7.38908453550355e-07\\
35951.840706001	7.20821165360475e-07\\
36220.8765518205	7.03117082365874e-07\\
36491.9256543999	6.85792691826181e-07\\
36765.0030794002	6.6884445388881e-07\\
37040.1240052216	6.52268783721031e-07\\
37317.3037238483	6.36062024270276e-07\\
37596.5576416978	6.20220408535331e-07\\
37877.9012804769	6.04740011134357e-07\\
38161.3502780454	5.89616690429479e-07\\
38446.9203892846	5.74846024480911e-07\\
38734.6274869731	5.60423246411156e-07\\
39024.4875626694	5.46343186854005e-07\\
39316.5167276	5.32600232331369e-07\\
39610.7312135559	5.19188307936578e-07\\
39907.1473737941	5.06100890237147e-07\\
40205.7816839466	4.93331052101693e-07\\
40506.6507429366	4.80871536149876e-07\\
40809.7712739007	4.687148490564e-07\\
41115.1601251185	4.56853366205325e-07\\
41422.8342709494	4.45279435765163e-07\\
41732.8108127753	4.33985472927604e-07\\
42045.1069799522	4.22964038033664e-07\\
42359.7401307671	4.12207895634281e-07\\
42676.7277534027	4.01710054417705e-07\\
42996.0874669105	3.91463789966053e-07\\
43317.8370221886	3.81462653407537e-07\\
43641.9943029698	3.7170046936429e-07\\
43968.5773268145	3.62171326405479e-07\\
44297.6042461126	3.52869562739404e-07\\
44629.0933490932	3.43789749300002e-07\\
44963.0630608394	3.34926671819638e-07\\
45299.531944314	3.26275312991666e-07\\
45638.5187013907	3.17830835435885e-07\\
45980.0421738933	3.09588565885527e-07\\
46324.1213446437	3.01543980803774e-07\\
46670.775338516	2.9369269349432e-07\\
47020.0234235008	2.8603044267762e-07\\
47371.885011775	2.78553082448903e-07\\
47726.3796607813	2.71256573503812e-07\\
48083.5270743154	2.64136975504795e-07\\
48443.347103621	2.57190440459716e-07\\
48805.8597484928	2.5041320698911e-07\\
49171.0851583894	2.43801595367368e-07\\
49539.0436335514	2.37352003233827e-07\\
49909.7556261313	2.31060901880761e-07\\
50283.2417413299	2.24924833036418e-07\\
50659.5227385407	2.189404060714e-07\\
51038.6195325055	2.13104295566131e-07\\
51420.5531944749	2.07413239185669e-07\\
51805.3449533811	2.01864035815459e-07\\
52193.0161970173	1.96453543918272e-07\\
52583.5884732259	1.91178680078222e-07\\
52977.0834910972	1.86036417702664e-07\\
53373.5231221757	1.81023785857058e-07\\
53772.929401675	1.7613786821153e-07\\
54175.3245297041	1.71375802080928e-07\\
54580.7308724997	1.66734777543001e-07\\
54989.1709636708	1.62212036621456e-07\\
55400.6675054502	1.57804872522763e-07\\
55815.2433699566	1.53510628917191e-07\\
56232.9216004666	1.49326699256002e-07\\
56653.725412694	1.45250526118037e-07\\
57077.6781960819	1.41279600579872e-07\\
57504.8035151016	1.37411461604763e-07\\
57935.1251105625	1.33643695446271e-07\\
58368.6669009326	1.29973935063181e-07\\
58805.4529836665	1.26399859542909e-07\\
59245.5076365459	1.22919193531032e-07\\
59688.8553190289	1.19529706665065e-07\\
60135.5206736089	1.1622921301091e-07\\
60585.528527185	1.13015570500748e-07\\
61038.9038924417	1.09886680371408e-07\\
61495.6719692387	1.06840486602449e-07\\
61955.8581460127	1.03874975353404e-07\\
62419.4880011873	1.00988174399812e-07\\
62886.5873045955	9.81781525677559e-08\\
63357.182018912	9.54430191667997e-08\\
63831.2983010957	9.27809234212757e-08\\
64308.9625038448	9.01900538999643e-08\\
64790.2011770598	8.76686379443062e-08\\
65275.0410693208	8.52149410952858e-08\\
65763.5091293736	8.28272665192339e-08\\
66255.6325076273	8.05039544327908e-08\\
66751.438557664	7.82433815273127e-08\\
67250.9548377589	7.60439603930381e-08\\
67754.209112412	7.39041389433275e-08\\
68261.2293538915	7.18223998393156e-08\\
68772.0437437882	6.97972599153167e-08\\
69286.6806745825	6.78272696053229e-08\\
69805.1687512222	6.5911012370946e-08\\
70327.5367927121	6.40471041311425e-08\\
70853.8138337169	6.22341926940495e-08\\
71384.0291261735	6.04709571912811e-08\\
71918.2121409184	5.8756107514982e-08\\
72456.392569325	5.70883837579694e-08\\
72998.6003249533	5.54665556572568e-08\\
73544.8655452147	5.3889422041243e-08\\
74095.2185930443	5.23558102808595e-08\\
74649.690058591	5.08645757449207e-08\\
75208.3107609164	4.94146012599452e-08\\
75771.111749708	4.8004796574679e-08\\
76338.1243070055	4.66340978295427e-08\\
76909.3799489393	4.53014670312221e-08\\
77484.9104274818	4.40058915325962e-08\\
78064.7477322134	4.2746383518183e-08\\
78648.9240920989	4.15219794952895e-08\\
79237.4719772807	4.03317397910064e-08\\
79830.4241008822	3.91747480552111e-08\\
80427.8134208264	3.80501107697054e-08\\
81029.6731416689	3.69569567636085e-08\\
81636.0367164413	3.58944367351279e-08\\
82246.937848513	3.4861722779793e-08\\
82862.4104934629	3.38580079252542e-08\\
83482.488860967	3.28825056727196e-08\\
84107.2074167008	3.19344495450967e-08\\
84736.6008842538	3.10130926419049e-08\\
85370.7042470603	3.01177072010019e-08\\
86009.5527503438	2.92475841671683e-08\\
86653.1819030753	2.84020327675834e-08\\
87301.627479948	2.75803800942122e-08\\
87954.9255233655	2.67819706931251e-08\\
88613.1123454441	2.60061661607571e-08\\
89276.2245300332	2.52523447471046e-08\\
89944.2989347464	2.45199009658636e-08\\
90617.3726930118	2.38082452114874e-08\\
91295.4832161355	2.31168033831546e-08\\
91978.6681953803	2.24450165156211e-08\\
92666.9656040625	2.17923404169242e-08\\
93360.4136996602	2.11582453129131e-08\\
94059.0510259418	2.05422154985574e-08\\
94762.9164151073	1.99437489959977e-08\\
95472.0489899465	1.93623572192877e-08\\
96186.4881660146	1.87975646457755e-08\\
96906.2736538223	1.82489084940707e-08\\
97631.4454610423	1.77159384085377e-08\\
98362.0438947356	1.71982161502505e-08\\
99098.1095635883	1.66953152943502e-08\\
99839.6833801717	1.62068209337282e-08\\
100586.806563215	1.57323293889728e-08\\
101339.520639896	1.52714479245011e-08\\
102097.867448152	1.48237944708011e-08\\
102861.889138999	1.43889973527116e-08\\
103631.628178883	1.39666950236531e-08\\
104407.127352034	1.35565358057357e-08\\
105188.429762846	1.31581776356588e-08\\
105975.578838274	1.27712878163184e-08\\
106768.618330246	1.239554277404e-08\\
107567.592318097	1.20306278213486e-08\\
108372.545211017	1.16762369251914e-08\\
109183.521750518	1.13320724805253e-08\\
110000.567012926	1.09978450891807e-08\\
110823.726411883	1.0673273343914e-08\\
111653.04570087	1.03580836175617e-08\\
112488.570975754	1.0052009857203e-08\\
113330.348677345	9.75479338324794e-09\\
114178.425593984	9.46618269335631e-09\\
115032.848864136	9.18593327110277e-09\\
115893.665979016	8.91380739929765e-09\\
116760.924785228	8.64957397787414e-09\\
117634.673487419	8.39300834625689e-09\\
118514.960650966	8.14389211012057e-09\\
119401.835204671	7.90201297245418e-09\\
120295.34644348	7.66716456884344e-09\\
121195.544031226	7.4391463068848e-09\\
122102.478003387	7.21776320964689e-09\\
123016.198769868	7.00282576309463e-09\\
123936.757117803	6.79414976739099e-09\\
124864.204214378	6.59155619199633e-09\\
125798.591609674	6.39487103448005e-09\\
126739.971239535	6.20392518296616e-09\\
127688.395428448	6.01855428213152e-09\\
128643.916892463	5.83859860267724e-09\\
129606.588742111	5.66390291419705e-09\\
130576.464485363	5.49431636136318e-09\\
131553.598030603	5.32969234335606e-09\\
132538.043689621	5.16988839646197e-09\\
133529.856180639	5.01476607976496e-09\\
134529.090631344	4.86419086386118e-09\\
135535.802581959	4.71803202252382e-09\\
136550.047988325	4.5761625272485e-09\\
137571.883225014	4.43845894461064e-09\\
138601.365088463	4.30480133636617e-09\\
139638.550800127	4.17507316222988e-09\\
140683.498009666	4.04916118526572e-09\\
141736.264798142	3.92695537982427e-09\\
142796.909681254	3.80834884196589e-09\\
143865.491612584	3.6932377023059e-09\\
144942.069986881	3.58152104122293e-09\\
146026.704643354	3.47310080637008e-09\\
147119.455869007	3.36788173243081e-09\\
148220.384401982	3.26577126306336e-09\\
149329.55143494	3.16667947497665e-09\\
150447.018618462	3.07051900408418e-09\\
151572.84806447	2.97720497368188e-09\\
152707.102349689	2.88665492459762e-09\\
153849.844519117	2.79878874726138e-09\\
155001.138089531	2.71352861564614e-09\\
156161.047053023	2.63079892302998e-09\\
157329.635880547	2.55052621953256e-09\\
158506.969525512	2.47263915137814e-09\\
159693.113427386	2.39706840184032e-09\\
160888.133515336	2.3237466338237e-09\\
162092.096211894	2.25260843403834e-09\\
163305.068436644	2.18359025872564e-09\\
164527.117609946	2.11663038089291e-09\\
165758.311656683	2.05166883901689e-09\\
166998.719010032	1.98864738717657e-09\\
168248.408615274	1.92750944657642e-09\\
169507.449933624	1.86820005842277e-09\\
170775.912946089	1.81066583811648e-09\\
172053.868157361	1.75485493072616e-09\\
173341.386599734	1.70071696770704e-09\\
174638.539837053	1.64820302483143e-09\\
175945.399968693	1.59726558129777e-09\\
177262.039633563	1.54785847998592e-09\\
178588.532014148	1.49993688882697e-09\\
179924.950840571	1.45345726325748e-09\\
181271.370394698	1.40837730972757e-09\\
182627.865514261	1.36465595023435e-09\\
183994.51159702	1.32225328785209e-09\\
185371.384604954	1.28113057323143e-09\\
186758.561068483	1.24125017204118e-09\\
188156.118090722	1.20257553332605e-09\\
189564.133351766	1.16507115875519e-09\\
190982.685113006	1.12870257273673e-09\\
192411.852221483	1.09343629337394e-09\\
193851.71411427	1.0592398042399e-09\\
195302.350822881	1.02608152694766e-09\\
196763.842977729	9.9393079449345e-10\\
198236.2718126	9.62757825351874e-10\\
199719.719169172	9.3253369830142e-10\\
201214.267501562	9.03230327960275e-10\\
202719.999880911	8.7482044101236e-10\\
204237	8.47277553104144e-10\\
204237	5.24806122753905e-10\\
202719.999880911	5.42574054544712e-10\\
201214.267501562	5.60931249211618e-10\\
199719.719169172	5.79896965697389e-10\\
198236.2718126	5.99491088192957e-10\\
196763.842977729	6.19734146453349e-10\\
195302.350822881	6.40647336775683e-10\\
193851.71411427	6.62252543660718e-10\\
192411.852221483	6.84572362180167e-10\\
190982.685113006	7.07630121072325e-10\\
189564.133351766	7.31449906589838e-10\\
188156.118090722	7.56056587123581e-10\\
186758.561068483	7.81475838627728e-10\\
185371.384604954	8.07734170871848e-10\\
183994.51159702	8.34858954546413e-10\\
182627.865514261	8.62878449249145e-10\\
181271.370394698	8.91821832380524e-10\\
179924.950840571	9.21719228977175e-10\\
178588.532014148	9.52601742513466e-10\\
177262.039633563	9.84501486701829e-10\\
175945.399968693	1.01745161832367e-09\\
174638.539837053	1.0514863711237e-09\\
173341.386599734	1.08664109080108e-09\\
172053.868157361	1.12295227113234e-09\\
170775.912946089	1.16045759126169e-09\\
169507.449933624	1.19919595419532e-09\\
168248.408615274	1.23920752653778e-09\\
166998.719010032	1.28053377950924e-09\\
165758.311656683	1.32321753128361e-09\\
164527.117609946	1.36730299068894e-09\\
163305.068436644	1.41283580231227e-09\\
162092.096211894	1.45986309305277e-09\\
160888.133515336	1.50843352016787e-09\\
159693.113427386	1.55859732085842e-09\\
158506.969525512	1.61040636344064e-09\\
157329.635880547	1.66391420015297e-09\\
156161.047053023	1.71917612164858e-09\\
155001.138089531	1.77624921322453e-09\\
153849.844519117	1.83519241284044e-09\\
152707.102349689	1.89606657098143e-09\\
151572.84806447	1.9589345124202e-09\\
150447.018618462	2.02386109993652e-09\\
149329.55143494	2.09091330005206e-09\\
148220.384401982	2.16016025084114e-09\\
147119.455869007	2.23167333187946e-09\\
146026.704643354	2.30552623639393e-09\\
144942.069986881	2.38179504567872e-09\\
143865.491612584	2.46055830584472e-09\\
142796.909681254	2.54189710696989e-09\\
141736.264798142	2.62589516472166e-09\\
140683.498009666	2.71263890452237e-09\\
139638.550800127	2.80221754833153e-09\\
138601.365088463	2.89472320412074e-09\\
137571.883225014	2.9902509581172e-09\\
136550.047988325	3.08889896989609e-09\\
135535.802581959	3.19076857040148e-09\\
134529.090631344	3.29596436297878e-09\\
133529.856180639	3.40459432750325e-09\\
132538.043689621	3.51676992769041e-09\\
131553.598030603	3.63260622167691e-09\\
130576.464485363	3.75222197596168e-09\\
129606.588742111	3.87573978279916e-09\\
128643.916892463	4.00328618113894e-09\\
127688.395428448	4.1349917812068e-09\\
126739.971239535	4.27099139282499e-09\\
125798.591609674	4.41142415757204e-09\\
124864.204214378	4.55643368488165e-09\\
123936.757117803	4.70616819218609e-09\\
123016.198769868	4.86078064920748e-09\\
122102.478003387	5.0204289265045e-09\\
121195.544031226	5.18527594838405e-09\\
120295.34644348	5.35548985028652e-09\\
119401.835204671	5.53124414075928e-09\\
118514.960650966	5.71271786813052e-09\\
117634.673487419	5.90009579200014e-09\\
116760.924785228	6.09356855966511e-09\\
115893.665979016	6.29333288759763e-09\\
115032.848864136	6.49959174809689e-09\\
114178.425593984	6.71255456123723e-09\\
113330.348677345	6.93243739223384e-09\\
112488.570975754	7.15946315435377e-09\\
111653.04570087	7.39386181749542e-09\\
110823.726411883	7.6358706225649e-09\\
110000.567012926	7.88573430177832e-09\\
109183.521750518	8.14370530501615e-09\\
108372.545211017	8.4100440323636e-09\\
107567.592318097	8.68501907296426e-09\\
106768.618330246	8.96890745031997e-09\\
105975.578838274	9.26199487416889e-09\\
105188.429762846	9.56457599907255e-09\\
104407.127352034	9.876954689845e-09\\
103631.628178883	1.01994442939562e-08\\
102861.889138999	1.05323679210405e-08\\
102097.867448152	1.08760587296444e-08\\
101339.520639896	1.12308602213407e-08\\
100586.806563215	1.15971265423423e-08\\
99839.6833801717	1.19752227927447e-08\\
99098.1095635883	1.23655253435216e-08\\
98362.0438947356	1.27684221614056e-08\\
97631.4454610423	1.31843131417729e-08\\
96906.2736538223	1.36136104496579e-08\\
96186.4881660146	1.40567388690183e-08\\
95472.0489899465	1.45141361603643e-08\\
94762.9164151073	1.49862534268715e-08\\
94059.0510259418	1.54735554890852e-08\\
93360.4136996602	1.59765212683244e-08\\
92666.9656040625	1.64956441788905e-08\\
91978.6681953803	1.70314325291786e-08\\
91295.4832161355	1.75844099317841e-08\\
90617.3726930118	1.81551157226998e-08\\
89944.2989347464	1.87441053896773e-08\\
89276.2245300332	1.93519510098407e-08\\
88613.1123454441	1.99792416966154e-08\\
87954.9255233655	2.06265840560386e-08\\
87301.627479948	2.12946026525105e-08\\
86653.1819030753	2.19839404840273e-08\\
86009.5527503438	2.26952594669437e-08\\
85370.7042470603	2.34292409302881e-08\\
84736.6008842538	2.41865861196548e-08\\
84107.2074167008	2.4968016710682e-08\\
83482.488860967	2.5774275332114e-08\\
82862.4104934629	2.66061260984343e-08\\
82246.937848513	2.74643551520452e-08\\
81636.0367164413	2.83497712149531e-08\\
81029.6731416689	2.92632061499097e-08\\
80427.8134208264	3.0205515530938e-08\\
79830.4241008822	3.11775792231604e-08\\
79237.4719772807	3.21803019718343e-08\\
78648.9240920989	3.32146140004673e-08\\
78064.7477322134	3.42814716178881e-08\\
77484.9104274818	3.53818578341095e-08\\
76909.3799489393	3.65167829848099e-08\\
76338.1243070055	3.7687285364242e-08\\
75771.111749708	3.88944318663361e-08\\
75208.3107609164	4.01393186337686e-08\\
74649.690058591	4.14230717147158e-08\\
74095.2185930443	4.27468477270002e-08\\
73544.8655452147	4.4111834529309e-08\\
72998.6003249533	4.55192518991242e-08\\
72456.392569325	4.6970352216982e-08\\
71918.2121409184	4.84664211566486e-08\\
71384.0291261735	5.0008778380745e-08\\
70853.8138337169	5.15987782413471e-08\\
70327.5367927121	5.3237810485018e-08\\
69805.1687512222	5.49273009617043e-08\\
69286.6806745825	5.66687123368897e-08\\
68772.0437437882	5.84635448063226e-08\\
68261.2293538915	6.03133368126301e-08\\
67754.209112412	6.22196657630317e-08\\
67250.9548377589	6.41841487473392e-08\\
66751.438557664	6.62084432553623e-08\\
66255.6325076273	6.82942478927627e-08\\
65763.5091293736	7.04433030943465e-08\\
65275.0410693208	7.26573918337002e-08\\
64790.2011770598	7.49383403279944e-08\\
64308.9625038448	7.72880187367059e-08\\
63831.2983010957	7.97083418528937e-08\\
63357.182018912	8.22012697855805e-08\\
62886.5873045955	8.4768808631688e-08\\
62419.4880011873	8.74130111358162e-08\\
61955.8581460127	9.01359773360891e-08\\
61495.6719692387	9.29398551940865e-08\\
61038.9038924417	9.58268412067608e-08\\
60585.528527185	9.87991809980741e-08\\
60135.5206736089	1.01859169887853e-07\\
59688.8553190289	1.05009153435227e-07\\
59245.5076365459	1.0825152795372e-07\\
58805.4529836665	1.11588740994828e-07\\
58368.6669009326	1.15023291796673e-07\\
57935.1251105625	1.18557731693932e-07\\
57504.8035151016	1.22194664484937e-07\\
57077.6781960819	1.25936746751444e-07\\
56653.725412694	1.29786688126057e-07\\
56232.9216004666	1.33747251501877e-07\\
55815.2433699566	1.37821253178272e-07\\
55400.6675054502	1.42011562936059e-07\\
54989.1709636708	1.46321104034643e-07\\
54580.7308724997	1.50752853122737e-07\\
54175.3245297041	1.55309840053387e-07\\
53772.929401675	1.599951475928e-07\\
53373.5231221757	1.64811911011234e-07\\
52977.0834910972	1.69763317542665e-07\\
52583.5884732259	1.74852605698175e-07\\
52193.0161970173	1.80083064416053e-07\\
51805.3449533811	1.85458032029144e-07\\
51420.5531944749	1.90980895027334e-07\\
51038.6195325055	1.96655086589872e-07\\
50659.5227385407	2.02484084858495e-07\\
50283.2417413299	2.08471410918127e-07\\
49909.7556261313	2.14620626446969e-07\\
49539.0436335514	2.20935330992026e-07\\
49171.0851583894	2.2741915881973e-07\\
48805.8597484928	2.34075775283729e-07\\
48443.347103621	2.40908872643561e-07\\
48083.5270743154	2.47922165258631e-07\\
47726.3796607813	2.55119384071482e-07\\
47371.885011775	2.62504270283714e-07\\
47020.0234235008	2.70080568116573e-07\\
46670.775338516	2.77852016537849e-07\\
46324.1213446437	2.85822339827979e-07\\
45980.0421738933	2.93995236853292e-07\\
45638.5187013907	3.02374368916221e-07\\
45299.531944314	3.10963346065301e-07\\
44963.0630608394	3.19765711777846e-07\\
44629.0933490932	3.28784925984412e-07\\
44297.6042461126	3.38024346496808e-07\\
43968.5773268145	3.47487209045378e-07\\
43641.9943029698	3.57176606342185e-07\\
43317.8370221886	3.67095466881183e-07\\
42996.0874669105	3.772465345776e-07\\
42676.7277534027	3.87632350836944e-07\\
42359.7401307671	3.98255241208226e-07\\
42045.1069799522	4.09117309354615e-07\\
41732.8108127753	4.20220441550742e-07\\
41422.8342709494	4.31566325107044e-07\\
41115.1601251185	4.43156483787961e-07\\
40809.7712739007	4.54992332187386e-07\\
40506.6507429366	4.67075248995169e-07\\
40205.7816839466	4.79406666203367e-07\\
39907.1473737941	4.91988167978392e-07\\
39610.7312135559	5.04821589944693e-07\\
39316.5167276	5.17909107953431e-07\\
39024.4875626694	5.31253305836408e-07\\
38734.6274869731	5.4485721438023e-07\\
38446.9203892846	5.5872431822438e-07\\
38161.3502780454	5.72858532393218e-07\\
37877.9012804769	5.87264154380365e-07\\
37596.5576416978	6.01945800170751e-07\\
37317.3037238483	6.16908333049836e-07\\
37040.1240052216	6.32156792882206e-07\\
36765.0030794002	6.47696331447984e-07\\
36491.9256543999	6.63532157118251e-07\\
36220.8765518205	6.79669490138197e-07\\
35951.840706001	6.96113528313461e-07\\
35684.8031631832	7.12869421991911e-07\\
35419.7490806799	7.29942256806856e-07\\
35156.6637260502	7.47337042558029e-07\\
34895.5324762807	7.65058706721861e-07\\
34636.340816972	7.83112091298194e-07\\
34379.0743415334	8.01501951946661e-07\\
34123.7187503806	8.2023295860255e-07\\
33870.2598501416	8.39309696967268e-07\\
33618.6835528682	8.58736670437486e-07\\
33368.9758752518	8.78518302169259e-07\\
33121.1229378475	8.98658937073334e-07\\
32875.1109643017	9.19162843612141e-07\\
32630.9262805866	9.40034215321508e-07\\
32388.5553142404	9.61277172017487e-07\\
32147.9845936127	9.82895760673993e-07\\
31909.2007471158	1.00489395597345e-06\\
31672.1905024814	1.0272756605434e-06\\
31436.9406860226	1.05004470489824e-06\\
31203.4382219025	1.07320484710851e-06\\
30971.6701314066	1.09675977222158e-06\\
30741.6235322216	1.12071309145767e-06\\
30513.2856377198	1.14506834120439e-06\\
30286.6437562476	1.16982898183187e-06\\
30061.6852904209	1.19499839634903e-06\\
29838.3977364244	1.22057988892024e-06\\
29616.7686833165	1.24657668325998e-06\\
29396.78581234	1.27299192092183e-06\\
29178.4368962368	1.29982865949671e-06\\
28961.7097985688	1.32708987073428e-06\\
28746.5924730428	1.35477843860014e-06\\
28533.0729628413	1.38289715728074e-06\\
28321.1393999579	1.41144872914694e-06\\
28110.7800045375	1.44043576268667e-06\\
27901.9830842215	1.46986077041618e-06\\
27694.7370334982	1.49972616677916e-06\\
27489.0303330572	1.53003426604252e-06\\
27284.8515491498	1.56078728019678e-06\\
27082.189332953	1.59198731686959e-06\\
26881.0324199387	1.62363637725943e-06\\
26681.3696292481	1.65573635409732e-06\\
26483.1898630694	1.68828902964348e-06\\
26286.4821060218	1.72129607372575e-06\\
26091.2354245425	1.75475904182672e-06\\
25897.4389662797	1.78867937322575e-06\\
25705.0819594888	1.82305838920221e-06\\
25514.153712434	1.85789729130602e-06\\
25324.6436127937	1.89319715970114e-06\\
25136.5411270713	1.92895895158738e-06\\
24949.8358000088	1.96518349970609e-06\\
24764.5172540065	2.00187151093409e-06\\
24580.5751885457	2.03902356497083e-06\\
24397.9993796163	2.07664011312253e-06\\
24216.7796791487	2.11472147718724e-06\\
24036.9060144492	2.15326784844398e-06\\
23858.3683876408	2.19227928674848e-06\\
23681.1568751072	2.23175571973803e-06\\
23505.261626941	2.27169694214662e-06\\
23330.6728663967	2.3121026152313e-06\\
23157.3808893468	2.3529722663101e-06\\
22985.3760637425	2.39430528841056e-06\\
22814.6488290788	2.43610094002755e-06\\
22645.1896958625	2.47835834498811e-06\\
22476.9892450851	2.52107649241967e-06\\
22310.0381276992	2.56425423681785e-06\\
22144.3270640985	2.6078902982081e-06\\
21979.8468436028	2.65198326239494e-06\\
21816.5883239452	2.69653158129156e-06\\
21654.5424307645	2.74153357332064e-06\\
21493.7001571006	2.78698742387734e-06\\
21334.0525628939	2.83289118584311e-06\\
21175.5907744885	2.87924278013856e-06\\
21018.3059841386	2.92603999630287e-06\\
20862.1894495195	2.9732804930851e-06\\
20707.2324932414	3.02096179903371e-06\\
20553.4265023667	3.06908131306847e-06\\
20400.7629279322	3.11763630501943e-06\\
20249.233284473	3.16662391611745e-06\\
20098.8291495512	3.21604115942108e-06\\
19949.5421632881	3.26588492016534e-06\\
19801.3640278989	3.31615195602014e-06\\
19654.286507232	3.36683889724716e-06\\
19508.3014263109	3.41794224674888e-06\\
19363.4006708799	3.46945838000622e-06\\
19219.5761869535	3.52138354490805e-06\\
19076.8199803677	3.57371386148297e-06\\
18935.1241163369	3.6264453215522e-06\\
18794.480719012	3.67957378833504e-06\\
18654.8819710428	3.73309499605061e-06\\
18516.3201131441	3.7870045495774e-06\\
18378.7874436634	3.84129792425083e-06\\
18242.2763181536	3.89597046590218e-06\\
18106.7791489477	3.9510173912686e-06\\
17972.2884047374	4.0064337889338e-06\\
17838.7966101542	4.06221462099232e-06\\
17706.2963453539	4.11835472566738e-06\\
17574.7802456044	4.17484882115088e-06\\
17444.2410008763	4.23169151097596e-06\\
17314.6713554361	4.28887729127356e-06\\
17186.0641074439	4.34640056030331e-06\\
17058.4121085521	4.40425563068589e-06\\
16931.7082635086	4.46243674479022e-06\\
16805.9455297624	4.52093809374601e-06\\
16681.116917072	4.57975384055264e-06\\
16557.2154871169	4.63887814773412e-06\\
16434.234353112	4.69830520994486e-06\\
16312.166679425	4.75802929185372e-06\\
16191.0056811961	4.81804477152394e-06\\
16070.7446239608	4.87834618936148e-06\\
15951.3768232763	4.93892830252242e-06\\
15832.8956443492	4.99978614445951e-06\\
15715.2945016669	5.06091508905232e-06\\
15598.5668586318	5.12231091851801e-06\\
15482.7062271979	5.18396989405709e-06\\
15367.70616751	5.24588882796597e-06\\
15253.5602875458	5.30806515576864e-06\\
15140.2622427609	5.37049700680024e-06\\
15027.8057357356	5.43318327163283e-06\\
14916.1845158256	5.49612366478025e-06\\
14805.3923788139	5.55931878125604e-06\\
14695.4231665662	5.62277014578381e-06\\
14586.2707666889	5.6864802537586e-06\\
14477.9291121888	5.75045260340986e-06\\
14370.3921811364	5.81469171899668e-06\\
14263.6539963308	5.87920316524489e-06\\
14157.7086249677	5.94399355358636e-06\\
14052.5501783096	6.0090705410609e-06\\
13948.1728113584	6.07444282297139e-06\\
13844.5707225307	6.14012012053693e-06\\
13741.738153335	6.20611316486206e-06\\
13639.669388052	6.27243367853848e-06\\
13538.3587534167	6.33909435613165e-06\\
13437.8006183031	6.4061088446882e-06\\
13337.9893934111	6.47349172525034e-06\\
13238.9195309561	6.54125849619345e-06\\
13140.5855243605	6.6094255590259e-06\\
13042.9819079474	6.67801020712013e-06\\
12946.1032566372	6.74703061768434e-06\\
12849.9441856459	6.81650584714527e-06\\
12754.4993501855	6.88645582999512e-06\\
12659.7634451676	6.95690138105927e-06\\
12565.7312049077	7.02786420106916e-06\\
12472.3974028332	7.09936688537078e-06\\
12379.7568511926	7.17143293556316e-06\\
12287.8044007671	7.24408677384115e-06\\
12196.5349405845	7.31735375980637e-06\\
12105.9433976352	7.39126020951147e-06\\
12016.0247365899	7.46583341650967e-06\\
11926.7739595202	7.54110167469271e-06\\
11838.1861056203	7.61709430271682e-06\\
11750.2562509318	7.69384166983192e-06\\
11662.9795080693	7.77137522294717e-06\\
11576.3510259497	7.8497275147841e-06\\
11490.3659895215	7.92893223298416e-06\\
11405.019619498	8.00902423005445e-06\\
11320.3071720913	8.09003955404912e-06\\
11236.2239387488	8.17201547989734e-06\\
11152.765245891	8.25499054130062e-06\\
11069.9264546524	8.33900456313166e-06\\
10987.7029606233	8.42409869427673e-06\\
10906.0901935939	8.51031544087037e-06\\
10825.0836173003	8.5976986998776e-06\\
10744.6787291723	8.6862937929847e-06\\
10664.8710600833	8.77614750076251e-06\\
10585.6561741017	8.86730809707067e-06\\
10507.0296682446	8.95982538367318e-06\\
10428.9871722326	9.05375072503722e-06\\
10351.5243482474	9.14913708328841e-06\\
10274.6368906905	9.24603905329633e-06\\
10198.3205259438	9.34451289786358e-06\\
10122.5710121321	9.44461658299168e-06\\
10047.3841388873	9.54640981319565e-06\\
9972.75572711457	9.64995406683794e-06\\
9898.68162875981	9.75531263145077e-06\\
9825.15772657923	9.8625506390134e-06\\
9752.17993391047	9.97173510114915e-06\\
9679.74419444542	1.00829349442033e-05\\
9607.84648200483	1.01962210441606e-05\\
9536.48280031448	1.03116662613583e-05\\
9465.64918278304	1.04293454749447e-05\\
9395.34169228163	1.05493356170331e-05\\
9325.55642092493	1.06717157064919e-05\\
9256.28948985408	1.07965668823117e-05\\
9187.53704902097	1.09239724364828e-05\\
9119.29527697427	1.10540178463102e-05\\
9051.56038064706	1.11867908060918e-05\\
8984.32859514597	1.13223812580739e-05\\
8917.59618354194	1.14608814225972e-05\\
8851.35943666247	1.16023858273352e-05\\
8785.61467288549	1.17469913355236e-05\\
8720.35823793472	1.18947971730684e-05\\
8655.58650467655	1.20459049544148e-05\\
8591.29587291844	1.22004187070498e-05\\
8527.48276920879	1.23584448945022e-05\\
8464.14364663834	1.25200924376952e-05\\
8401.27498464301	1.2685472734497e-05\\
8338.87328880825	1.28546996773053e-05\\
8276.93509067475	1.30278896684909e-05\\
8215.4569475457	1.32051616335175e-05\\
8154.43544229543	1.33866370315418e-05\\
8093.86718317946	1.35724398632926e-05\\
8033.748803646	1.37626966760153e-05\\
7974.0769621488	1.39575365652625e-05\\
7914.84834196143	1.41570911733054e-05\\
7856.05965099295	1.43614946839348e-05\\
7797.70762160489	1.45708838134244e-05\\
7739.78901042966	1.47853977974265e-05\\
7682.30059819021	1.50051783735842e-05\\
7625.23918952119	1.52303697596571e-05\\
7568.60161279127	1.5461118626984e-05\\
7512.38471992689	1.56975740691384e-05\\
7456.58538623724	1.59398875656866e-05\\
7401.20051024062	1.61882129410196e-05\\
7346.22701349204	1.64427063183176e-05\\
7291.66184041212	1.67035260688182e-05\\
7237.50195811723	1.69708327566934e-05\\
7183.7443562509	1.72447890800157e-05\\
7130.38604681656	1.75255598084993e-05\\
7077.42406401143	1.78133117189537e-05\\
7024.85546406162	1.81082135296795e-05\\
6972.67732505857	1.84104358353768e-05\\
6920.88674679662	1.87201510445197e-05\\
6869.48085061182	1.90375333215762e-05\\
6818.45677922192	1.93627585369061e-05\\
6767.81169656754	1.96960042276352e-05\\
6717.54278765452	2.00374495732592e-05\\
6667.64725839753	2.03872753901376e-05\\
6618.12233546469	2.07456641493518e-05\\
6568.96526612346	2.11128000225613e-05\\
6520.17331808759	2.14888689604431e-05\\
6471.74377936531	2.18740588079566e-05\\
6423.67395810857	2.2268559459985e-05\\
6375.9611824634	2.26725630597941e-05\\
6328.60280042144	2.30862642411951e-05\\
6281.59617967246	2.35098604133005e-05\\
6234.93870745815	2.3943552084369e-05\\
6188.62779042682	2.43875432185767e-05\\
6142.66085448928	2.48420416167855e-05\\
6097.03534467575	2.53072593097597e-05\\
6051.74872499389	2.57834129500727e-05\\
6006.79847828778	2.62707241874194e-05\\
5962.18210609809	2.67694200114559e-05\\
5917.8971285231	2.72797330467853e-05\\
5873.94108408097	2.78019017863361e-05\\
5830.31152957285	2.83361707520635e-05\\
5787.00603994713	2.88827905754144e-05\\
5744.02220816459	2.94420179940111e-05\\
5701.35764506468	3.00141157651475e-05\\
5659.00997923266	3.05993525005675e-05\\
5616.97685686784	3.11980024302728e-05\\
5575.25594165272	3.18103451055528e-05\\
5533.84491462315	3.2436665052919e-05\\
5492.74147403939	3.30772513911685e-05\\
5451.94333525824	3.37323974234749e-05\\
5411.44823060603	3.44024002154068e-05\\
5371.25390925253	3.5087560168281e-05\\
5331.3581370859	3.57881805955007e-05\\
5291.7586965885	3.65045673076791e-05\\
5252.45338671363	3.72370282105702e-05\\
5213.44002276314	3.79858729182156e-05\\
5174.7164362661	3.87514123823425e-05\\
5136.28047485818	3.95339585379425e-05\\
5098.13000216207	4.03338239641138e-05\\
5060.2628976687	4.11513215586558e-05\\
5022.6770566194	4.19867642245162e-05\\
4985.37038988889	4.28404645659822e-05\\
4948.34082386919	4.37127345924405e-05\\
4911.58630035433	4.46038854275595e-05\\
4875.10477642598	4.55142270218558e-05\\
4838.89422433987	4.64440678667585e-05\\
4802.95263141311	4.73937147084682e-05\\
4767.27799991228	4.83634722600984e-05\\
4731.86834694246	4.9353642910782e-05\\
4696.72170433693	5.03645264306139e-05\\
4661.83611854782	5.13964196704746e-05\\
4627.20965053756	5.24496162559425e-05\\
4592.84037567103	5.35244062746509e-05\\
4558.72638360862	5.46210759565722e-05\\
4524.86577820005	5.57399073468285e-05\\
4491.25667737898	5.68811779707309e-05\\
4457.89721305839	5.80451604908355e-05\\
4424.78553102675	5.92321223558822e-05\\
4391.91979084494	6.04423254415464e-05\\
4359.29816574399	6.16760256829963e-05\\
4326.91884252352	6.29334726992909e-05\\
4294.78002145095	6.42149094097051e-05\\
4262.8799161615	6.55205716421019e-05\\
4231.21675355883	6.68506877335012e-05\\
4199.78877371658	6.82054781230306e-05\\
4168.59422978048	6.95851549374655e-05\\
4137.63138787127	7.09899215695865e-05\\
4106.89852698833	7.24199722496034e-05\\
4076.39393891404	7.38754916099131e-05\\
4046.1159281188	7.53566542434736e-05\\
4016.06281166681	7.68636242560895e-05\\
3986.23291912252	7.83965548129164e-05\\
3956.62459245778	7.99555876795015e-05\\
3927.23618595969	8.15408527576865e-05\\
3898.06606613913	8.31524676167085e-05\\
3869.11261163994	8.4790537019835e-05\\
3840.37421314884	8.64551524468773e-05\\
3811.84927330594	8.81463916129276e-05\\
3783.53620661599	8.98643179836673e-05\\
3755.43343936023	9.1608980287588e-05\\
3727.53940950892	9.33804120254744e-05\\
3699.85256663454	9.51786309774816e-05\\
3672.37137182558	9.70036387081443e-05\\
3645.09429760103	9.88554200696423e-05\\
3618.01982782546	0.000100733942703644\\
3591.14645762477	0.00010263915654203\\
3564.47269330252	0.000104570993306811\\
3537.99705225692	0.000106529366009509\\
3511.71806289842	0.000108514168450316\\
3485.63426456793	0.000110525274717283\\
3459.7442074556	0.000112562538685829\\
3434.04645252026	0.00011462579351886\\
3408.53957140944	0.000116714851167787\\
3383.22214637995	0.000118829501874766\\
3358.09277021909	0.00012096951367655\\
3333.15004616647	0.000123134631910354\\
3308.39258783631	0.000125324578722287\\
3283.81901914043	0.000127539052578998\\
3259.42797421173	0.000129777727783362\\
3235.21809732829	0.000132040253995267\\
3211.18804283803	0.000134326255758879\\
3187.33647508389	0.000136635332038108\\
3163.66206832958	0.000138967055762531\\
3140.16350668593	0.000141320973386604\\
3116.83948403772	0.000143696604465745\\
3093.68870397109	0.000146093441253801\\
3070.7098797015	0.00014851094832746\\
3047.90173400216	0.000150948562244473\\
3025.2629991331	0.000153405691244019\\
3002.79241677064	0.000155881714999197\\
2980.48873793751	0.00015837598443349\\
2958.3507229334	0.000160887821614995\\
2936.37714126603	0.00016341651974419\\
2914.56677158281	0.000165961343252979\\
2892.91840160291	0.000168521528034355\\
2871.4308280499	0.00017109628182326\\
2850.10285658484	0.000173684784749609\\
2828.93330173995	0.000176286190083788\\
2807.92098685266	0.000178899625192843\\
2787.06474400025	0.000181524192721675\\
2766.3634139349	0.0001841589720076\\
2745.81584601927	0.000186803020728487\\
2725.42089816257	0.000189455376774357\\
2705.17743675704	0.000192115060320161\\
2685.08433661496	0.000194781076064217\\
2665.14048090611	0.000197452415583293\\
2645.34476109567	0.000200128059743147\\
2625.69607688265	0.000202806981093702\\
2606.1933361387	0.000205488146172557\\
2586.83545484739	0.000208170517640326\\
2567.62135704402	0.000210853056176999\\
2548.54997475574	0.000213534722080127\\
2529.62024794223	0.000216214476522276\\
2510.8311244368	0.000218891282445374\\
2492.18155988785	0.000221564105091272\\
2473.67051770087	0.000224231912188825\\
2455.29696898081	0.000226893673836074\\
2437.05989247487	0.00022954836213008\\
2418.95827451578	0.000232194950605745\\
2400.9911089654	0.000234832413548348\\
2383.15739715885	0.000237459725243019\\
2365.45614784899	0.000240075859218821\\
2347.88637715129	0.000242679787536805\\
2330.44710848916	0.000245270480161461\\
2313.1373725397	0.000247846904444629\\
2295.95620717979	0.000250408024741016\\
2278.90265743261	0.000252952802165567\\
2261.97577541457	0.000255480194495531\\
2245.17462028264	0.000257989156214205\\
2228.49825818201	0.000260478638689013\\
2211.94576219425	0.000262947590473666\\
2195.51621228573	0.000265394957722405\\
2179.20869525649	0.000267819684703546\\
2163.02230468953	0.000270220714399467\\
2146.95614090037	0.000272596989180625\\
2131.00931088707	0.000274947451541966\\
2115.18092828061	0.000277271044891075\\
2099.47011329558	0.000279566714378454\\
2083.87599268133	0.000281833407761442\\
2068.39769967338	0.000284070076294274\\
2053.03437394529	0.000286275675637807\\
2037.7851615608	0.00028844916678325\\
2022.64921492642	0.000290589516985076\\
2007.62569274426	0.000292695700698916\\
1992.71375996528	0.000294766700520855\\
1977.91258774292	0.000296801508125015\\
1963.22135338698	0.000298799125196732\\
1948.63924031791	0.000300758564358984\\
1934.16543802145	0.000302678850089994\\
1919.7991420035	0.000304559019630165\\
1905.53955374551	0.000306398123876691\\
1891.38588066002	0.000308195228264345\\
1877.33733604663	0.00030994941363103\\
1863.39313904828	0.000311659777066829\\
1849.55251460781	0.000313325432745291\\
1835.81469342496	0.000314945512735819\\
1822.17891191353	0.000316519167796003\\
1808.64441215895	0.000318045568142844\\
1795.21044187622	0.000319523904201768\\
1781.876254368	0.000320953387332418\\
1768.64110848317	0.000322333250530186\\
1755.50426857564	0.000323662749102488\\
1742.4650044634	0.000324941161318781\\
1729.522591388	0.000326167789033361\\
1716.67630997424	0.000327341958279984\\
1703.92544619016	0.000328463019837377\\
1691.26929130741	0.000329530349764748\\
1678.70714186178	0.000330543349906412\\
1666.2382996142	0.000331501448364722\\
1653.86207151182	0.000332404099940528\\
1641.57776964957	0.000333250786540447\\
1629.38471123188	0.000334041017550333\\
1617.28221853476	0.000334774330174401\\
1605.26961886811	0.000335450289739587\\
1593.34624453832	0.000336068489964896\\
1581.51143281119	0.000336628553195603\\
1569.76452587505	0.000337130130602448\\
1558.10487080425	0.000337572902346156\\
1546.53181952283	0.00033795657770794\\
1535.0447287685	0.000338280895186985\\
1523.64296005691	0.000338545622566306\\
1512.32587964614	0.000338750556948872\\
1501.09285850146	0.000338895524766438\\
1489.94327226042	0.000338980381764129\\
1478.8765011981	0.000339005012964597\\
1467.89193019267	0.000338969332616349\\
1456.98894869122	0.000338873284131746\\
1446.16695067579	0.00033871684002117\\
1435.42533462974	0.000338500001830894\\
1424.76350350425	0.00033822280009329\\
1414.18086468518	0.000337885294299074\\
1403.67682996012	0.000337487572902377\\
1393.2508154857	0.0003370297533703\\
1382.90224175512	0.000336511982289348\\
1372.63053356595	0.000335934435541512\\
1362.43511998817	0.000335297318562663\\
1352.31543433242	0.000334600866695243\\
1342.27091411851	0.000333845345645764\\
1332.30100104415	0.000333031052055235\\
1322.40514095393	0.000332158314187196\\
1312.5827838085	0.000331227492733427\\
1302.83338365401	0.000330238981731709\\
1293.15639859177	0.000329193209583154\\
1283.55129074812	0.000328090640149014\\
1274.01752624451	0.000326931773898724\\
1264.5545751679	0.000325717149072786\\
1255.16191154121	0.000324447342816681\\
1245.83901329415	0.000323122972235858\\
1236.58536223419	0.000321744695317868\\
1227.40044401774	0.000320313211666515\\
1218.28374812157	0.0003188292629949\\
1209.23476781446	0.000317293633329795\\
1200.25300012896	0.000315707148888638\\
1191.33794583355	0.000314070677602307\\
1182.48910940477	0.000312385128270682\\
1173.70599899975	0.000310651449353067\\
1164.98812642888	0.000308870627410366\\
1156.3350071286	0.00030704368522958\\
1147.74616013456	0.000305171679672508\\
1139.22110805483	0.000303255699298932\\
1130.75937704337	0.000301296861819501\\
1122.3604967737	0.000299296311435131\\
1114.02400041277	0.00029725521611808\\
1105.749424595	0.000295174764885671\\
1097.53630939651	0.00029305616511141\\
1089.38419830959	0.000290900639910822\\
1081.2926382173	0.000288709425631356\\
1073.26117936829	0.000286483769467749\\
1065.2893753518	0.000284224927216832\\
1057.37678307286	0.000281934161179218\\
1049.52296272766	0.000279612738209721\\
1041.72747777907	0.000277261927913973\\
1033.98989493243	0.000274883000985332\\
1026.30978411142	0.000272477227673822\\
1018.6867184342	0.000270045876377375\\
1011.12027418962	0.000267590212344866\\
1003.61003081374	0.000265111496480254\\
996.155570866409	0.000262610984237366\\
988.756480008064	0.000260089924595452\\
981.412346976721	0.000257549559106326\\
974.122763565099	0.000254991121004844\\
966.88732459794	0.000252415834375346\\
959.705627909479	0.000249824913367625\\
952.577274321103	0.00024721956145687\\
945.501867619149	0.000244600970742862\\
938.479014532895	0.000241970321284442\\
931.508324712691	0.000239328780465972\\
924.589410708265	0.000236677502393074\\
917.721887947192	0.000234017627315484\\
910.905374713515	0.000231350281075274\\
904.139492126523	0.000228676574579074\\
897.423864119701	0.000225997603293262\\
890.758117419823	0.000223314446761313\\
884.141881526201	0.000220628168142727\\
877.574788690099	0.000217939813773129\\
871.056473894285	0.000215250412745245\\
864.586574832748	0.000212560976510585\\
858.164731890556	0.00020987249850171\\
851.790588123873	0.000207185953775037\\
845.463789240111	0.000204502298674165\\
839.183983578243	0.000201822470513722\\
832.950822089257	0.000199147387283739\\
826.763958316753	0.000196477947374575\\
820.623048377686	0.000193815029322381\\
814.527750943253	0.000191159491575107\\
808.47772721992	0.000188512172279011\\
802.472640930591	0.000185873889085629\\
796.512158295919	0.000183245438979118\\
790.59594801575	0.000180627598123879\\
784.72368125071	0.000178021121732339\\
778.89503160393	0.000175426743952716\\
773.109675102898	0.000172845177776601\\
767.367290181458	0.000170277114966141\\
761.667557661931	0.000167723226000584\\
756.01016073738	0.000165184160041929\\
750.394784953994	0.000162660544919389\\
744.821118193618	0.000160152987132356\\
739.288850656394	0.000157662071871541\\
733.797674843554	0.000155188363057927\\
728.347285540317	0.000152732403399172\\
722.937379798933	0.000150294714463053\\
717.56765692184	0.000147875796767563\\
712.237818444947	0.000145476129887213\\
706.947568121055	0.000143096172575125\\
701.696611903378	0.000140736362900443\\
696.484657929211	0.000138397118400606\\
691.311416503699	0.000136078836248017\\
686.176600083736	0.000133781893430616\\
681.079923261988	0.000131506646945866\\
676.021102751025	0.000129253434007665\\
670.999857367572	0.000127022572265676\\
666.015908016889	0.000124814360036568\\
661.06897767725	0.000122629076546665\\
656.158791384547	0.000120466982185488\\
651.285076217013	0.000118328318769684\\
646.447561280041	0.000116213309816825\\
641.645977691138	0.000114122160828575\\
636.880058564972	0.000112055059582718\\
632.149538998544	0.000110012176433533\\
627.454156056458	0.000107993664620031\\
622.793648756308	0.000105999660581553\\
618.167758054176	0.00010403028428024\\
613.576226830228	0.000102085639529892\\
609.018799874428	0.000100165814330745\\
604.495223872346	9.8270881209703e-05\\
600.005247391086	9.64008975655518e-05\\
595.548620865302	9.45559060187247e-05\\
591.125096583335	9.27359347651683e-05\\
586.734428673439	9.09409979338837e-05\\
582.376373090114	8.91710959477257e-05\\
578.050687600549	8.74262158870518e-05\\
573.757131771146	8.57063318558234e-05\\
569.495466954168	8.40114053497799e-05\\
565.265456274466	8.23413856263056e-05\\
561.066864616317	8.06962100756369e-05\\
556.899458610353	7.90758045930522e-05\\
552.763006620593	7.7480083951714e-05\\
548.657278731565	7.59089521758318e-05\\
544.582046735526	7.43623029138326e-05\\
540.537084119782	7.28400198112362e-05\\
536.522166054094	7.13419768829352e-05\\
532.537069378183	6.98680388845988e-05\\
528.581572589324	6.84180616829226e-05\\
524.655455830038	6.69918926244529e-05\\
520.758500875867	6.55893709027224e-05\\
516.890491123249	6.42103279234413e-05\\
513.051211577476	6.28545876674853e-05\\
509.240448840744	6.15219670514305e-05\\
505.457991100293	6.02122762853833e-05\\
501.703628116633	5.89253192278498e-05\\
497.977151211859	5.76608937373887e-05\\
494.278353258049	5.64187920207894e-05\\
490.607028665758	5.51988009775043e-05\\
486.962973372584	5.40007025400601e-05\\
483.345984831829	5.28242740101631e-05\\
479.755862001239	5.16692883901985e-05\\
476.192405331832	5.05355147098171e-05\\
472.655416756805	4.94227183472804e-05\\
469.144699680525	4.83306613452305e-05\\
465.660058967601	4.72591027205283e-05\\
462.201300932039	4.62077987677953e-05\\
458.768233326478	4.51765033562783e-05\\
455.360665331498	4.41649682196454e-05\\
451.978407545023	4.31729432383148e-05\\
448.621271971783	4.22001767139054e-05\\
445.289072012877	4.12464156354087e-05\\
441.981622455389	4.03114059366668e-05\\
438.698739462102	3.93948927447672e-05\\
435.440240561274	3.84966206189627e-05\\
432.205944636502	3.7616333779758e-05\\
428.995671916648	3.67537763278232e-05\\
425.809243965854	3.59086924524313e-05\\
422.646483673618	3.50808266291592e-05\\
419.507215244952	3.42699238066386e-05\\
416.391264190611	3.34757295821994e-05\\
413.298457317395	3.26979903663048e-05\\
410.228622718523	3.19364535357512e-05\\
407.181589764074	3.11908675756658e-05\\
404.157189091507	3.04609822104154e-05\\
401.155252596246	2.97465485236092e-05\\
398.175613422337	2.90473190674522e-05\\
395.21810595317	2.83630479617785e-05\\
392.282565802281	2.76934909831604e-05\\
389.368829804207	2.70384056445527e-05\\
386.476736005421	2.63975512659876e-05\\
383.60612365533	2.57706890368836e-05\\
380.75683319734	2.51575820705734e-05\\
377.928706259986	2.45579954516834e-05\\
375.12158564813	2.39716962770202e-05\\
372.335315334224	2.33984536906273e-05\\
369.569740449639	2.28380389136761e-05\\
366.824707276051	2.22902252698456e-05\\
364.100063236907	2.17547882068248e-05\\
361.395656888935	2.12315053145442e-05\\
358.711337913731	2.07201563407101e-05\\
356.046957109401	2.02205232041715e-05\\
353.402366382274	1.97323900066062e-05\\
350.777418738662	1.92555430429599e-05\\
348.171968276697	1.87897708110253e-05\\
345.585870178217	1.83348640204896e-05\\
343.018980700718	1.78906156017298e-05\\
340.471157169366	1.74568207145841e-05\\
337.942257969062	1.70332767572731e-05\\
335.432142536577	1.66197833756046e-05\\
332.940671352735	1.62161424725459e-05\\
330.467705934659	1.58221582182128e-05\\
328.013108828071	1.54376370602853e-05\\
325.57674359966	1.50623877348333e-05\\
323.158474829489	1.46962212775031e-05\\
320.758168103475	1.4338951035e-05\\
318.375690005912	1.39903926767792e-05\\
316.010908112063	1.3650364206846e-05\\
313.663690980792	1.33186859755542e-05\\
311.333908147261	1.29951806912882e-05\\
309.02143011568	1.26796734319061e-05\\
306.726128352108	1.23719916558241e-05\\
304.447875277308	1.20719652126228e-05\\
302.186544259657	1.17794263530577e-05\\
299.942009608104	1.14942097383657e-05\\
297.714146565192	1.12161524487598e-05\\
295.502831300113	1.09450939910201e-05\\
293.307940901833	1.06808763050882e-05\\
291.129353372257	1.04233437695893e-05\\
288.96694761945	1.01723432062117e-05\\
286.820603450903	9.92772388288159e-06\\
284.690201566855	9.68933751568552e-06\\
282.575623553662	9.45703826949626e-06\\
280.476751877215	9.23068275727198e-06\\
278.393469876404	9.0101300380035e-06\\
276.325661756641	8.79524161329569e-06\\
274.273212583414	8.58588142257436e-06\\
272.236008275909	8.38191583691936e-06\\
270.213935600659	8.18321365153023e-06\\
268.206882165259	7.98964607683669e-06\\
266.214736412113	7.80108672827354e-06\\
264.237387612236	7.61741161474237e-06\\
262.274725859099	7.43849912578935e-06\\
260.326642062517	7.26423001753047e-06\\
258.393027942593	7.09448739736079e-06\\
256.473776023691	6.92915670748602e-06\\
254.568779628468	6.76812570731841e-06\\
252.677932871941	6.61128445478072e-06\\
250.801130655604	6.45852528656354e-06\\
248.938268661586	6.30974279738368e-06\\
247.089243346851	6.16483381829118e-06\\
245.253951937446	6.02369739407479e-06\\
243.432292422782	5.88623475981467e-06\\
241.624163549975	5.75234931663318e-06\\
239.829464818207	5.62194660669272e-06\\
238.048096473146	5.49493428749146e-06\\
236.279959501398	5.37122210550537e-06\\
234.524955625009	5.25072186922661e-06\\
232.782987295997	5.13334742164558e-06\\
231.05395769093	5.01901461222492e-06\\
229.33777070555	4.9076412684117e-06\\
227.634330949423	4.79914716673347e-06\\
225.943543740646	4.69345400352291e-06\\
224.265315100576	4.5904853653142e-06\\
222.599551748611	4.49016669895365e-06\\
220.946161097006	4.39242528146515e-06\\
219.305051245723	4.29719018971036e-06\\
217.676130977326	4.20439226988181e-06\\
216.059309751909	4.11396410686585e-06\\
214.454497702065	4.02583999351121e-06\\
212.86160562789	3.93995589983697e-06\\
211.280544992026	3.85624944221332e-06\\
209.711227914737	3.77465985254602e-06\\
208.15356716903	3.69512794749508e-06\\
206.6074761758	3.61759609775596e-06\\
205.072868999023	3.54200819743108e-06\\
203.549660340977	3.46830963351726e-06\\
202.037765537499	3.39644725553412e-06\\
200.537100553285	3.32636934531675e-06\\
199.047581977213	3.25802558699478e-06\\
197.569127017711	3.19136703717891e-06\\
196.101653498152	3.12634609537445e-06\\
194.645079852288	3.06291647464063e-06\\
193.199325119717	3.0010331725128e-06\\
191.764308941382	2.94065244220404e-06\\
190.339951555105	2.8817317641012e-06\\
188.926173791152	2.82422981756955e-06\\
187.522897067834	2.76810645307909e-06\\
186.13004338714	2.71332266466472e-06\\
184.7475353304	2.65984056273124e-06\\
183.375296053983	2.60762334721352e-06\\
182.013249285024	2.55663528110122e-06\\
180.661319317186	2.5068416643362e-06\\
179.319431006454	2.45820880809068e-06\\
177.987509766954	2.4107040094325e-06\\
176.665481566809	2.36429552638398e-06\\
175.353272924028	2.31895255337926e-06\\
174.050810902413	2.27464519712509e-06\\
172.758023107516	2.23134445286873e-06\\
171.474837682604	2.18902218107634e-06\\
170.201183304674	2.14765108452454e-06\\
168.936989180483	2.10720468580711e-06\\
167.682185042617	2.06765730525842e-06\\
166.436701145581	2.02898403929446e-06\\
165.200468261928	1.99116073917197e-06\\
163.973417678405	1.95416399016565e-06\\
162.755481192139	1.91797109116286e-06\\
161.546591106842	1.88256003467495e-06\\
160.34668022905	1.84790948726384e-06\\
159.15568186439	1.81399877038217e-06\\
157.97352981387	1.78080784162473e-06\\
156.800158370201	1.74831727638898e-06\\
155.635502314145	1.71650824994163e-06\\
154.479496910888	1.68536251988836e-06\\
153.332077906443	1.65486240904324e-06\\
152.193181524082	1.62499078869424e-06\\
151.062744460785	1.59573106226087e-06\\
149.940703883725	1.56706714933997e-06\\
148.826997426776	1.5389834701352e-06\\
147.721563187044	1.51146493026572e-06\\
146.624339721429	1.48449690594949e-06\\
145.535266043209	1.45806522955603e-06\\
144.454281618647	1.43215617552398e-06\\
143.381326363632	1.40675644663795e-06\\
142.316340640336	1.38185316065964e-06\\
141.259265253899	1.35743383730764e-06\\
140.21004144914	1.33348638558058e-06\\
139.168610907289	1.30999909141791e-06\\
138.134915742751	1.28696060569273e-06\\
137.108898499881	1.26435993253089e-06\\
136.090502149795	1.2421864179507e-06\\
135.0796700872	1.22042973881725e-06\\
134.076346127247	1.19907989210565e-06\\
133.080474502409	1.1781271844672e-06\\
132.091999859379	1.15756222209263e-06\\
131.110867255994	1.13737590086657e-06\\
130.137022158185	1.1175593968071e-06\\
129.17041043694	1.09810415678481e-06\\
128.2109783653	1.07900188951523e-06\\
127.258672615369	1.06024455681876e-06\\
126.313440255352	1.04182436514245e-06\\
125.375228746615	1.02373375733752e-06\\
124.44398594076	1.0059654046871e-06\\
123.519660076729	9.88512199178222e-07\\
122.602199777928	9.71367246012541e-07\\
121.691554049369	9.54523856349989e-07\\
120.787672274837	9.37975540279857e-07\\
119.890504214077	9.21716000013708e-07\\
119	9.05739123294672e-07\\
}--cycle;
\addplot[ybar interval, fill=black, fill opacity=0.35, area legend, draw=none] table[row sep=crcr, x=Lower, y=Count] {%
Lower	Upper	Count\\
119	166.945095217458	0.000111238351058511\\
166.945095217458	234.207267371144	0.00027752102064167\\
234.207267371144	328.569365982322	0.000254339404837665\\
328.569365982322	460.949950331585	0.000161151527153326\\
460.949950331585	646.666666794865	0.000358969660546674\\
646.666666794865	907.208640857353	0.000327522402639305\\
907.208640857353	1272.72296579858	0.000423148030358411\\
1272.72296579858	1785.50299756883	0.000239218883471713\\
1785.50299756883	2504.88208353096	0.000159396719899502\\
2504.88208353096	3514.09897431581	0.000118904074134826\\
3514.09897431581	4929.92930983802	6.78047345020274e-05\\
4929.92930983802	6916.19763064073	2.81935725468193e-05\\
6916.19763064073	9702.73337806785	1.53117241384977e-05\\
9702.73337806785	13611.9642661441	8.18575339144199e-06\\
13611.9642661441	19096.2241219165	1.45872008445765e-06\\
19096.2241219165	26790.0920531703	1.03978909847186e-06\\
26790.0920531703	37583.8190647142	9.88228315878168e-07\\
37583.8190647142	52726.3382554153	3.52209118322166e-07\\
52726.3382554153	73969.7778194808	1.25528950178929e-07\\
73969.7778194808	103772.198330147	0\\
103772.198330147	145582.012866817	6.37808776771247e-08\\
145582.012866817	204237	4.5463596481774e-08\\
204237	204237	4.5463596481774e-08\\
};
\addlegendentry{Istogramma reale}

\addplot [color=black]
table[row sep=crcr]{%
119	6.15433567615765e-05\\
119.890504214077	6.24706820220498e-05\\
120.787672274837	6.34079858824207e-05\\
121.691554049369	6.43552990703441e-05\\
122.602199777928	6.53126508675256e-05\\
123.519660076729	6.62800690806998e-05\\
124.44398594076	6.72575800126463e-05\\
125.375228746615	6.8245208433252e-05\\
126.313440255352	6.92429775506336e-05\\
127.258672615369	7.02509089823314e-05\\
128.2109783653	7.12690227265871e-05\\
129.17041043694	7.22973371337154e-05\\
130.137022158185	7.33358688775852e-05\\
131.110867255994	7.43846329272213e-05\\
132.091999859379	7.54436425185364e-05\\
133.080474502409	7.65129091262124e-05\\
134.076346127247	7.75924424357363e-05\\
135.0796700872	7.86822503156108e-05\\
136.090502149795	7.97823387897457e-05\\
137.108898499881	8.08927120100506e-05\\
138.134915742751	8.20133722292341e-05\\
139.168610907289	8.31443197738299e-05\\
140.21004144914	8.42855530174598e-05\\
141.259265253899	8.54370683543451e-05\\
142.316340640336	8.65988601730861e-05\\
143.381326363632	8.77709208307162e-05\\
144.454281618647	8.89532406270501e-05\\
145.535266043209	9.01458077793349e-05\\
146.624339721429	9.13486083972235e-05\\
147.721563187044	9.25616264580771e-05\\
148.826997426776	9.37848437826191e-05\\
149.940703883725	9.50182400109448e-05\\
151.062744460785	9.62617925789103e-05\\
152.193181524082	9.7515476694907e-05\\
153.332077906443	9.87792653170396e-05\\
154.479496910888	0.000100053129130722\\
155.635502314145	0.0001013370365267\\
156.800158370201	0.000102630953579524\\
157.97352981387	0.000103934844026472\\
159.15568186439	0.000105248669246958\\
160.34668022905	0.000106572388242409\\
161.546591106842	0.000107905957616656\\
162.755481192139	0.000109249331556834\\
163.973417678405	0.00011060246181481\\
165.200468261928	0.000111965297689149\\
166.436701145581	0.000113337786007639\\
167.682185042617	0.000114719871110382\\
168.936989180483	0.000116111494833468\\
170.201183304674	0.000117512596493243\\
171.474837682604	0.000118923112871184\\
172.758023107516	0.000120342978199402\\
174.050810902413	0.000121772124146765\\
175.353272924028	0.000123210479805689\\
176.665481566809	0.000124657971679569\\
177.987509766954	0.000126114523670893\\
179.319431006454	0.000127580057070043\\
180.661319317186	0.000129054490544787\\
182.013249285024	0.000130537740130483\\
183.375296053983	0.000132029719221008\\
184.7475353304	0.000133530338560412\\
186.13004338714	0.000135039506235326\\
187.522897067834	0.000136557127668123\\
188.926173791152	0.000138083105610844\\
190.339951555105	0.000139617340139902\\
191.764308941382	0.000141159728651583\\
193.199325119717	0.000142710165858337\\
194.645079852288	0.000144268543785884\\
196.101653498152	0.000145834751771141\\
197.569127017711	0.000147408676460974\\
199.047581977213	0.000148990201811795\\
200.537100553285	0.000150579209090004\\
202.037765537499	0.00015217557687329\\
203.549660340977	0.000153779181052795\\
205.072868999023	0.00015538989483616\\
206.6074761758	0.000157007588751446\\
208.15356716903	0.000158632130651951\\
209.711227914737	0.000160263385721917\\
211.280544992026	0.000161901216483158\\
212.86160562789	0.00016354548280258\\
214.454497702065	0.000165196041900635\\
216.059309751909	0.000166852748360693\\
217.676130977326	0.000168515454139346\\
219.305051245723	0.000170184008577645\\
220.946161097006	0.000171858258413285\\
222.599551748611	0.000173538047793727\\
224.265315100576	0.000175223218290276\\
225.943543740646	0.000176913608913115\\
227.634330949423	0.000178609056127285\\
229.33777070555	0.000180309393869643\\
231.05395769093	0.00018201445356677\\
232.782987295997	0.000183724064153848\\
234.524955625009	0.000185438052094511\\
236.279959501398	0.000187156241401662\\
238.048096473146	0.000188878453659255\\
239.829464818207	0.000190604508045058\\
241.624163549975	0.000192334221354384\\
243.432292422782	0.000194067408024791\\
245.253951937446	0.00019580388016176\\
247.089243346851	0.000197543447565335\\
248.938268661586	0.000199285917757737\\
250.801130655604	0.000201031096011948\\
252.677932871941	0.000202778785381251\\
254.568779628468	0.000204528786729739\\
256.473776023691	0.000206280898763782\\
258.393027942593	0.000208034918064446\\
260.326642062517	0.000209790639120867\\
262.274725859099	0.000211547854364566\\
264.237387612236	0.000213306354204713\\
266.214736412113	0.000215065927064321\\
268.206882165259	0.000216826359417371\\
270.213935600659	0.000218587435826866\\
272.236008275909	0.000220348938983791\\
274.273212583414	0.000222110649746995\\
276.325661756641	0.000223872347183965\\
278.393469876404	0.0002256338086125\\
280.476751877215	0.000227394809643262\\
282.575623553662	0.000229155124223215\\
284.690201566855	0.000230914524679913\\
286.820603450903	0.00023267278176666\\
288.96694761945	0.000234429664708507\\
291.129353372257	0.000236184941249081\\
293.307940901833	0.000237938377698245\\
295.502831300113	0.000239689738980561\\
297.714146565192	0.000241438788684555\\
299.942009608104	0.00024318528911277\\
302.186544259657	0.000244929001332583\\
304.447875277308	0.000246669685227792\\
306.726128352108	0.000248407099550941\\
309.02143011568	0.000250141001976374\\
311.333908147261	0.000251871149154006\\
313.663690980792	0.000253597296763796\\
316.010908112063	0.000255319199570902\\
318.375690005912	0.000257036611481498\\
320.758168103475	0.000258749285599256\\
323.158474829489	0.000260456974282443\\
325.57674359966	0.000262159429201651\\
328.013108828071	0.000263856401398117\\
330.467705934659	0.000265547641342626\\
332.940671352735	0.000267232898994974\\
335.432142536577	0.000268911923863977\\
337.942257969062	0.000270584465067997\\
340.471157169366	0.000272250271395982\\
343.018980700718	0.000273909091368981\\
345.585870178217	0.000275560673302125\\
348.171968276697	0.000277204765367053\\
350.777418738662	0.000278841115654756\\
353.402366382274	0.000280469472238825\\
356.046957109401	0.000282089583239075\\
358.711337913731	0.000283701196885528\\
361.395656888935	0.000285304061582725\\
364.100063236907	0.000286897925974362\\
366.824707276051	0.0002884825390082\\
369.569740449639	0.000290057650001255\\
372.335315334224	0.000291623008705221\\
375.12158564813	0.000293178365372119\\
377.928706259986	0.000294723470820138\\
380.75683319734	0.000296258076499648\\
383.60612365533	0.000297781934559367\\
386.476736005421	0.000299294797912638\\
389.368829804207	0.000300796420303823\\
392.282565802281	0.000302286556374751\\
395.21810595317	0.000303764961731233\\
398.175613422337	0.000305231393009589\\
401.155252596246	0.000306685607943183\\
404.157189091507	0.000308127365428926\\
407.181589764074	0.000309556425593725\\
410.228622718523	0.000310972549860868\\
413.298457317395	0.000312375501016291\\
416.391264190611	0.000313765043274725\\
419.507215244952	0.000315140942345691\\
422.646483673618	0.00031650296549931\\
425.809243965854	0.000317850881631909\\
428.995671916648	0.000319184461331401\\
432.205944636502	0.0003205034769424\\
435.440240561274	0.00032180770263106\\
438.698739462102	0.000323096914449605\\
441.981622455389	0.000324370890400516\\
445.289072012877	0.000325629410500373\\
448.621271971783	0.0003268722568433\\
451.978407545023	0.000328099213664001\\
455.360665331498	0.000329310067400362\\
458.768233326478	0.000330504606755593\\
462.201300932039	0.000331682622759872\\
465.660058967601	0.000332843908831491\\
469.144699680525	0.000333988260837455\\
472.655416756805	0.000335115477153521\\
476.192405331832	0.00033622535872365\\
479.755862001239	0.000337317709118845\\
483.345984831829	0.00033839233459536\\
486.962973372584	0.000339449044152243\\
490.607028665758	0.000340487649588191\\
494.278353258049	0.000341507965557711\\
497.977151211859	0.000342509809626532\\
501.703628116633	0.000343493002326277\\
505.457991100293	0.000344457367208351\\
509.240448840744	0.000345402730897028\\
513.051211577476	0.00034632892314172\\
516.890491123249	0.000347235776868395\\
520.758500875867	0.000348123128230135\\
524.655455830038	0.000348990816656802\\
528.581572589324	0.000349838684903794\\
532.537069378183	0.000350666579099879\\
536.522166054094	0.000351474348794067\\
540.537084119782	0.000352261847001522\\
544.582046735526	0.000353028930248476\\
548.657278731565	0.000353775458616143\\
552.763006620593	0.000354501295783596\\
556.899458610353	0.000355206309069601\\
561.066864616317	0.000355890369473389\\
565.265456274466	0.000356553351714344\\
569.495466954168	0.000357195134270587\\
573.757131771146	0.000357815599416454\\
578.050687600549	0.000358414633258833\\
582.376373090114	0.000358992125772356\\
586.734428673439	0.000359547970833432\\
591.125096583335	0.000360082066253094\\
595.548620865302	0.000360594313808658\\
600.005247391086	0.00036108461927418\\
604.495223872346	0.000361552892449684\\
609.018799874428	0.000361999047189168\\
613.576226830228	0.000362423001427358\\
618.167758054176	0.000362824677205209\\
622.793648756308	0.00036320400069414\\
627.454156056458	0.000363560902218991\\
632.149538998544	0.000363895316279688\\
636.880058564972	0.000364207181571616\\
641.645977691138	0.000364496441004685\\
646.447561280041	0.000364763041721074\\
651.285076217013	0.000365006935111666\\
656.158791384547	0.000365228076831142\\
661.06897767725	0.000365426426811744\\
666.015908016889	0.000365601949275691\\
670.999857367572	0.000365754612746258\\
676.021102751025	0.000365884390057487\\
681.079923261988	0.000365991258362556\\
686.176600083736	0.000366075199140777\\
691.311416503699	0.000366136198203236\\
696.484657929211	0.000366174245697065\\
701.696611903378	0.000366189336108349\\
706.947568121055	0.000366181468263655\\
712.237818444947	0.000366150645330204\\
717.56765692184	0.000366096874814667\\
722.937379798933	0.000366020168560586\\
728.347285540317	0.000365920542744439\\
733.797674843554	0.000365798017870328\\
739.288850656394	0.000365652618763311\\
744.821118193618	0.000365484374561366\\
750.394784953994	0.000365293318706004\\
756.01016073738	0.000365079488931525\\
761.667557661931	0.000364842927252928\\
767.367290181458	0.000364583679952474\\
773.109675102898	0.000364301797564917\\
778.89503160393	0.000363997334861401\\
784.72368125071	0.000363670350832035\\
790.59594801575	0.000363320908667153\\
796.512158295919	0.000362949075737264\\
802.472640930591	0.000362554923571706\\
808.47772721992	0.000362138527836009\\
814.527750943253	0.000361699968307982\\
820.623048377686	0.000361239328852526\\
826.763958316753	0.000360756697395194\\
832.950822089257	0.000360252165894505\\
839.183983578243	0.000359725830313021\\
845.463789240111	0.000359177790587206\\
851.790588123873	0.000358608150596073\\
858.164731890556	0.000358017018128644\\
864.586574832748	0.000357404504850221\\
871.056473894285	0.0003567707262675\\
877.574788690099	0.000356115801692531\\
884.141881526201	0.00035543985420554\\
890.758117419823	0.000354743010616642\\
897.423864119701	0.000354025401426445\\
904.139492126523	0.00035328716078557\\
910.905374713515	0.000352528426453111\\
917.721887947192	0.000351749339754033\\
924.589410708265	0.000350950045535554\\
931.508324712691	0.000350130692122507\\
938.479014532895	0.000349291431271708\\
945.501867619149	0.000348432418125357\\
952.577274321103	0.00034755381116348\\
959.705627909479	0.000346655772155446\\
966.88732459794	0.000345738466110558\\
974.122763565099	0.000344802061227776\\
981.412346976721	0.000343846728844542\\
988.756480008064	0.000342872643384783\\
996.155570866409	0.000341879982306074\\
1003.61003081374	0.000340868926045999\\
1011.12027418962	0.000339839657967735\\
1018.6867184342	0.000338792364304878\\
1026.30978411142	0.00033772723410553\\
1033.98989493243	0.000336644459175681\\
1041.72747777907	0.000335544234021896\\
1049.52296272766	0.000334426755793345\\
1057.37678307286	0.000333292224223194\\
1065.2893753518	0.000332140841569372\\
1073.26117936829	0.000330972812554755\\
1081.2926382173	0.000329788344306786\\
1089.38419830959	0.000328587646296545\\
1097.53630939651	0.000327370930277311\\
1105.749424595	0.000326138410222628\\
1114.02400041277	0.00032489030226391\\
1122.3604967737	0.000323626824627597\\
1130.75937704337	0.00032234819757191\\
1139.22110805483	0.000321054643323199\\
1147.74616013456	0.000319746386011945\\
1156.3350071286	0.00031842365160841\\
1164.98812642888	0.000317086667857984\\
1173.70599899975	0.000315735664216239\\
1182.48910940477	0.000314370871783724\\
1191.33794583355	0.000312992523240529\\
1200.25300012896	0.000311600852780635\\
1209.23476781446	0.000310196096046081\\
1218.28374812157	0.000308778490060974\\
1227.40044401774	0.000307348273165368\\
1236.58536223419	0.000305905684949036\\
1245.83901329415	0.000304450966185159\\
1255.16191154121	0.00030298435876396\\
1264.5545751679	0.000301506105626311\\
1274.01752624451	0.000300016450697333\\
1283.55129074812	0.000298515638820012\\
1293.15639859177	0.000297003915688868\\
1302.83338365401	0.000295481527783689\\
1312.5827838085	0.00029394872230336\\
1322.40514095393	0.000292405747099813\\
1332.30100104415	0.000290852850612119\\
1342.27091411851	0.000289290281800749\\
1352.31543433242	0.000287718290082025\\
1362.43511998817	0.00028613712526279\\
1372.63053356595	0.00028454703747531\\
1382.90224175512	0.000282948277112448\\
1393.2508154857	0.000281341094763113\\
1403.67682996012	0.000279725741148025\\
1414.18086468518	0.000278102467055806\\
1424.76350350425	0.000276471523279424\\
1435.42533462974	0.000274833160553015\\
1446.16695067579	0.000273187629489099\\
1456.98894869122	0.000271535180516212\\
1467.89193019267	0.000269876063816979\\
1478.8765011981	0.000268210529266647\\
1489.94327226042	0.000266538826372096\\
1501.09285850146	0.00026486120421135\\
1512.32587964614	0.00026317791137361\\
1523.64296005691	0.000261489195899823\\
1535.0447287685	0.000259795305223809\\
1546.53181952283	0.000258096486113968\\
1558.10487080425	0.000256392984615578\\
1569.76452587505	0.000254685045993705\\
1581.51143281119	0.000252972914676749\\
1593.34624453832	0.00025125683420063\\
1605.26961886811	0.000249537047153643\\
1617.28221853476	0.00024781379512199\\
1629.38471123188	0.000246087318636008\\
1641.57776964957	0.000244357857117114\\
1653.86207151182	0.000242625648825466\\
1666.2382996142	0.000240890930808374\\
1678.70714186178	0.000239153938849465\\
1691.26929130741	0.000237414907418612\\
1703.92544619016	0.000235674069622648\\
1716.67630997424	0.000233931657156878\\
1729.522591388	0.000232187900257394\\
1742.4650044634	0.00023044302765421\\
1755.50426857564	0.000228697266525232\\
1768.64110848317	0.000226950842451062\\
1781.876254368	0.000225203979370667\\
1795.21044187622	0.000223456899537894\\
1808.64441215895	0.000221709823478873\\
1822.17891191353	0.000219962969950293\\
1835.81469342496	0.000218216555898569\\
1849.55251460781	0.00021647079641991\\
1863.39313904828	0.000214725904721292\\
1877.33733604663	0.000212982092082346\\
1891.38588066002	0.000211239567818164\\
1905.53955374551	0.000209498539243033\\
1919.7991420035	0.000207759211635107\\
1934.16543802145	0.000206021788202011\\
1948.63924031791	0.00020428647004739\\
1963.22135338698	0.000202553456138409\\
1977.91258774292	0.000200822943274204\\
1992.71375996528	0.000199095126055291\\
2007.62569274426	0.00019737019685393\\
2022.64921492642	0.000195648345785452\\
2037.7851615608	0.000193929760680558\\
2053.03437394529	0.000192214627058573\\
2068.39769967338	0.000190503128101675\\
2083.87599268133	0.000188795444630095\\
2099.47011329558	0.000187091755078275\\
2115.18092828061	0.000185392235472011\\
2131.00931088707	0.000183697059406553\\
2146.95614090037	0.000182006398025679\\
2163.02230468953	0.000180320420001736\\
2179.20869525649	0.000178639291516645\\
2195.51621228573	0.000176963176243875\\
2211.94576219425	0.000175292235331371\\
2228.49825818201	0.000173626627385446\\
2245.17462028264	0.000171966508455627\\
2261.97577541457	0.000170312032020456\\
2278.90265743261	0.000168663348974233\\
2295.95620717979	0.00016702060761471\\
2313.1373725397	0.000165383953631718\\
2330.44710848916	0.000163753530096733\\
2347.88637715129	0.000162129477453368\\
2365.45614784899	0.000160511933508786\\
2383.15739715885	0.000158901033426038\\
2400.9911089654	0.0001572969097173\\
2418.95827451578	0.000155699692238024\\
2437.05989247487	0.000154109508181976\\
2455.29696898081	0.000152526482077175\\
2473.67051770087	0.000150950735782697\\
2492.18155988785	0.00014938238848637\\
2510.8311244368	0.000147821556703319\\
2529.62024794223	0.00014626835427538\\
2548.54997475574	0.000144722892371345\\
2567.62135704402	0.000143185279488066\\
2586.83545484739	0.000141655621452367\\
2606.1933361387	0.000140134021423791\\
2625.69607688265	0.000138620579898144\\
2645.34476109567	0.000137115394711847\\
2665.14048090611	0.000135618561047068\\
2685.08433661496	0.000134130171437636\\
2705.17743675704	0.000132650315775724\\
2725.42089816257	0.000131179081319278\\
2745.81584601927	0.000129716552700199\\
2766.3634139349	0.000128262811933256\\
2787.06474400025	0.000126817938425714\\
2807.92098685266	0.000125382008987678\\
2828.93330173995	0.000123955097843126\\
2850.10285658484	0.000122537276641634\\
2871.4308280499	0.000121128614470767\\
2892.91840160291	0.000119729177869136\\
2914.56677158281	0.00011833903084009\\
2936.37714126603	0.000116958234866055\\
2958.3507229334	0.000115586848923486\\
2980.48873793751	0.000114224929498426\\
3002.79241677064	0.000112872530602668\\
3025.2629991331	0.000111529703790484\\
3047.90173400216	0.000110196498175937\\
3070.7098797015	0.000108872960450736\\
3093.68870397109	0.000107559134902637\\
3116.83948403772	0.000106255063434369\\
3140.16350668593	0.000104960785583074\\
3163.66206832958	0.000103676338540252\\
3187.33647508389	0.000102401757172176\\
3211.18804283803	0.000101137074040801\\
3235.21809732829	9.98823194251117e-05\\
3259.42797421173	9.86375213429245e-05\\
3283.81901914043	9.74027055731192e-05\\
3308.39258783631	9.61778956782835e-05\\
3333.15004616647	9.49631130277623e-05\\
3358.09277021909	9.37583768210949e-05\\
3383.22214637995	9.25637041118269e-05\\
3408.53957140944	9.13791098316827e-05\\
3434.04645252026	9.02046068150839e-05\\
3459.7442074556	8.90402058240023e-05\\
3485.63426456793	8.78859155731309e-05\\
3511.71806289842	8.67417427553602e-05\\
3537.99705225692	8.56076920675463e-05\\
3564.47269330252	8.44837662365574e-05\\
3591.14645762477	8.33699660455842e-05\\
3618.01982782546	8.22662903607014e-05\\
3645.09429760103	8.11727361576664e-05\\
3672.37137182558	8.00892985489419e-05\\
3699.85256663454	7.90159708109293e-05\\
3727.53940950892	7.79527444114009e-05\\
3755.43343936023	7.6899609037114e-05\\
3783.53620661599	7.58565526215985e-05\\
3811.84927330594	7.48235613731023e-05\\
3840.37421314884	7.3800619802683e-05\\
3869.11261163994	7.27877107524313e-05\\
3898.06606613913	7.17848154238169e-05\\
3927.23618595969	7.07919134061408e-05\\
3956.62459245778	6.98089827050847e-05\\
3986.23291912252	6.88359997713438e-05\\
4016.06281166681	6.78729395293304e-05\\
4046.1159281188	6.69197754059387e-05\\
4076.39393891404	6.59764793593567e-05\\
4106.89852698833	6.50430219079152e-05\\
4137.63138787127	6.41193721589619e-05\\
4168.59422978048	6.32054978377485e-05\\
4199.78877371658	6.23013653163216e-05\\
4231.21675355883	6.1406939642405e-05\\
4262.8799161615	6.05221845682632e-05\\
4294.78002145095	5.96470625795352e-05\\
4326.91884252352	5.87815349240279e-05\\
4359.29816574399	5.79255616404599e-05\\
4391.91979084494	5.70791015871453e-05\\
4424.78553102675	5.62421124706063e-05\\
4457.89721305839	5.54145508741065e-05\\
4491.25667737898	5.45963722860944e-05\\
4524.86577820005	5.37875311285491e-05\\
4558.72638360862	5.29879807852169e-05\\
4592.84037567103	5.21976736297316e-05\\
4627.20965053756	5.14165610536089e-05\\
4661.83611854782	5.06445934941069e-05\\
4696.72170433693	4.98817204619439e-05\\
4731.86834694246	4.91278905688654e-05\\
4767.27799991228	4.83830515550516e-05\\
4802.95263141311	4.76471503163594e-05\\
4838.89422433987	4.69201329313902e-05\\
4875.10477642598	4.62019446883746e-05\\
4911.58630035433	4.54925301118708e-05\\
4948.34082386919	4.47918329892644e-05\\
4985.37038988889	4.40997963970679e-05\\
5022.6770566194	4.34163627270103e-05\\
5060.2628976687	4.27414737119108e-05\\
5098.13000216207	4.20750704513305e-05\\
5136.28047485818	4.14170934369975e-05\\
5174.7164362661	4.07674825779975e-05\\
5213.44002276314	4.01261772257253e-05\\
5252.45338671363	3.94931161985917e-05\\
5291.7586965885	3.88682378064814e-05\\
5331.3581370859	3.82514798749548e-05\\
5371.25390925253	3.76427797691916e-05\\
5411.44823060603	3.70420744176691e-05\\
5451.94333525824	3.64493003355719e-05\\
5492.74147403939	3.586439364793e-05\\
5533.84491462315	3.52872901124785e-05\\
5575.25594165272	3.47179251422373e-05\\
5616.97685686784	3.41562338278064e-05\\
5659.00997923266	3.3602150959373e-05\\
5701.35764506468	3.30556110484279e-05\\
5744.02220816459	3.25165483491873e-05\\
5787.00603994713	3.19848968797178e-05\\
5830.31152957285	3.14605904427611e-05\\
5873.94108408097	3.09435626462567e-05\\
5917.8971285231	3.04337469235594e-05\\
5962.18210609809	2.99310765533504e-05\\
6006.79847828778	2.94354846792385e-05\\
6051.74872499389	2.89469043290512e-05\\
6097.03534467575	2.84652684338132e-05\\
6142.66085448928	2.79905098464102e-05\\
6188.62779042682	2.75225613599383e-05\\
6234.93870745815	2.70613557257357e-05\\
6281.59617967246	2.66068256710975e-05\\
6328.60280042144	2.6158903916672e-05\\
6375.9611824634	2.57175231935377e-05\\
6423.67395810857	2.52826162599604e-05\\
6471.74377936531	2.48541159178301e-05\\
6520.17331808759	2.44319550287779e-05\\
6568.96526612346	2.40160665299724e-05\\
6618.12233546469	2.36063834495946e-05\\
6667.64725839753	2.3202838921994e-05\\
6717.54278765452	2.28053662025234e-05\\
6767.81169656754	2.24138986820548e-05\\
6818.45677922192	2.20283699011764e-05\\
6869.48085061182	2.16487135640707e-05\\
6920.88674679662	2.12748635520755e-05\\
6972.67732505857	2.09067539369289e-05\\
7024.85546406162	2.05443189936977e-05\\
7077.42406401143	2.01874932133927e-05\\
7130.38604681656	1.98362113152703e-05\\
7183.7443562509	1.9490408258823e-05\\
7237.50195811723	1.91500192554598e-05\\
7291.66184041212	1.88149797798786e-05\\
7346.22701349204	1.84852255811309e-05\\
7401.20051024062	1.81606926933831e-05\\
7456.58538623724	1.78413174463742e-05\\
7512.38471992689	1.75270364755729e-05\\
7568.60161279127	1.7217786732036e-05\\
7625.23918952119	1.69135054919702e-05\\
7682.30059819021	1.66141303660001e-05\\
7739.78901042966	1.63195993081441e-05\\
7797.70762160489	1.60298506245014e-05\\
7856.05965099295	1.57448229816515e-05\\
7914.84834196143	1.54644554147702e-05\\
7974.0769621488	1.51886873354643e-05\\
8033.748803646	1.49174585393269e-05\\
8093.86718317946	1.46507092132169e-05\\
8154.43544229543	1.43883799422667e-05\\
8215.4569475457	1.41304117166181e-05\\
8276.93509067475	1.38767459378923e-05\\
8338.87328880825	1.3627324425396e-05\\
8401.27498464301	1.33820894220657e-05\\
8464.14364663834	1.31409836001556e-05\\
8527.48276920879	1.29039500666699e-05\\
8591.29587291844	1.26709323685444e-05\\
8655.58650467655	1.24418744975804e-05\\
8720.35823793472	1.22167208951337e-05\\
8785.61467288549	1.19954164565621e-05\\
8851.35943666247	1.17779065354363e-05\\
8917.59618354194	1.15641369475148e-05\\
8984.32859514597	1.13540539744896e-05\\
9051.56038064706	1.11476043675033e-05\\
9119.29527697427	1.0944735350443e-05\\
9187.53704902097	1.07453946230138e-05\\
9256.28948985408	1.05495303635959e-05\\
9325.55642092493	1.03570912318877e-05\\
9395.34169228163	1.01680263713404e-05\\
9465.64918278304	9.98228541138656e-06\\
9536.48280031448	9.79981846946555e-06\\
9607.84648200483	9.62057615285162e-06\\
9679.74419444542	9.44450956028615e-06\\
9752.17993391047	9.27157028341911e-06\\
9825.15772657923	9.10171040806262e-06\\
9898.68162875981	8.93488251526058e-06\\
9972.75572711457	8.77103968217818e-06\\
10047.3841388873	8.61013548281467e-06\\
10122.5710121321	8.45212398854311e-06\\
10198.3205259438	8.29695976848078e-06\\
10274.6368906905	8.1445978896941e-06\\
10351.5243482474	7.99499391724133e-06\\
10428.9871722326	7.84810391405707e-06\\
10507.0296682446	7.7038844406816e-06\\
10585.6561741017	7.56229255483945e-06\\
10664.8710600833	7.42328581087016e-06\\
10744.6787291723	7.28682225901497e-06\\
10825.0836173003	7.15286044456331e-06\\
10906.0901935939	7.0213594068619e-06\\
10987.7029606233	6.89227867819083e-06\\
11069.9264546524	6.76557828250964e-06\\
11152.765245891	6.64121873407669e-06\\
11236.2239387488	6.51916103594578e-06\\
11320.3071720913	6.39936667834311e-06\\
11405.019619498	6.28179763692793e-06\\
11490.3659895215	6.1664163709406e-06\\
11576.3510259497	6.0531858212408e-06\\
11662.9795080693	5.94206940823985e-06\\
11750.2562509318	5.83303102972999e-06\\
11838.1861056203	5.72603505861398e-06\\
11926.7739595202	5.6210463405384e-06\\
12016.0247365899	5.51803019143335e-06\\
12105.9433976352	5.41695239496236e-06\\
12196.5349405845	5.3177791998852e-06\\
12287.8044007671	5.22047731733644e-06\\
12379.7568511926	5.1250139180237e-06\\
12472.3974028332	5.0313566293474e-06\\
12565.7312049077	4.9394735324459e-06\\
12659.7634451676	4.8493331591687e-06\\
12754.4993501855	4.76090448898006e-06\\
12849.9441856459	4.67415694579706e-06\\
12946.1032566372	4.58906039476348e-06\\
13042.9819079474	4.50558513896344e-06\\
13140.5855243605	4.42370191607697e-06\\
13238.9195309561	4.34338189498013e-06\\
13337.9893934111	4.26459667229273e-06\\
13437.8006183031	4.18731826887593e-06\\
13538.3587534167	4.11151912628219e-06\\
13639.669388052	4.03717210316068e-06\\
13741.738153335	3.96425047161977e-06\\
13844.5707225307	3.89272791354978e-06\\
13948.1728113584	3.82257851690803e-06\\
14052.5501783096	3.75377677196837e-06\\
14157.7086249677	3.68629756753823e-06\\
14263.6539963308	3.62011618714431e-06\\
14370.3921811364	3.55520830519029e-06\\
14477.9291121888	3.49154998308796e-06\\
14586.2707666889	3.42911766536414e-06\\
14695.4231665662	3.36788817574567e-06\\
14805.3923788139	3.30783871322416e-06\\
14916.1845158256	3.24894684810281e-06\\
15027.8057357356	3.19119051802714e-06\\
15140.2622427609	3.13454802400146e-06\\
15253.5602875458	3.0789980263931e-06\\
15367.70616751	3.02451954092631e-06\\
15482.7062271979	2.97109193466717e-06\\
15598.5668586318	2.91869492200199e-06\\
15715.2945016669	2.86730856061016e-06\\
15832.8956443492	2.81691324743359e-06\\
15951.3768232763	2.76748971464435e-06\\
16070.7446239608	2.71901902561164e-06\\
16191.0056811961	2.67148257087034e-06\\
16312.166679425	2.62486206409201e-06\\
16434.234353112	2.5791395380601e-06\\
16557.2154871169	2.53429734065083e-06\\
16681.116917072	2.49031813082092e-06\\
16805.9455297624	2.44718487460368e-06\\
16931.7082635086	2.40488084111474e-06\\
17058.4121085521	2.36338959856848e-06\\
17186.0641074439	2.32269501030686e-06\\
17314.6713554361	2.28278123084119e-06\\
17444.2410008763	2.24363270190853e-06\\
17574.7802456044	2.20523414854366e-06\\
17706.2963453539	2.16757057516742e-06\\
17838.7966101542	2.13062726169309e-06\\
17972.2884047374	2.09438975965107e-06\\
18106.7791489477	2.0588438883334e-06\\
18242.2763181536	2.02397573095881e-06\\
18378.7874436634	1.98977163085914e-06\\
18516.3201131441	1.95621818768824e-06\\
18654.8819710428	1.92330225365389e-06\\
18794.480719012	1.89101092977377e-06\\
18935.1241163369	1.85933156215618e-06\\
19076.8199803677	1.82825173830607e-06\\
19219.5761869535	1.79775928345744e-06\\
19363.4006708799	1.76784225693246e-06\\
19508.3014263109	1.73848894852804e-06\\
19654.286507232	1.70968787493068e-06\\
19801.3640278989	1.68142777615975e-06\\
19949.5421632881	1.65369761204017e-06\\
20098.8291495512	1.62648655870488e-06\\
20249.233284473	1.59978400512735e-06\\
20400.7629279322	1.57357954968512e-06\\
20553.4265023667	1.54786299675424e-06\\
20707.2324932414	1.5226243533355e-06\\
20862.1894495195	1.49785382571253e-06\\
21018.3059841386	1.4735418161423e-06\\
21175.5907744885	1.44967891957829e-06\\
21334.0525628939	1.42625592042673e-06\\
21493.7001571006	1.40326378933603e-06\\
21654.5424307645	1.38069368001994e-06\\
21816.5883239452	1.35853692611442e-06\\
21979.8468436028	1.3367850380686e-06\\
22144.3270640985	1.31542970007009e-06\\
22310.0381276992	1.29446276700454e-06\\
22476.9892450851	1.27387626145003e-06\\
22645.1896958625	1.25366237070607e-06\\
22814.6488290788	1.2338134438575e-06\\
22985.3760637425	1.21432198887348e-06\\
23157.3808893468	1.19518066974139e-06\\
23330.6728663967	1.17638230363597e-06\\
23505.261626941	1.15791985812368e-06\\
23681.1568751072	1.13978644840218e-06\\
23858.3683876408	1.12197533457516e-06\\
24036.9060144492	1.10447991896234e-06\\
24216.7796791487	1.08729374344477e-06\\
24397.9993796163	1.07041048684528e-06\\
24580.5751885457	1.05382396234409e-06\\
24764.5172540065	1.03752811492958e-06\\
24949.8358000088	1.02151701888402e-06\\
25136.5411270713	1.00578487530427e-06\\
25324.6436127937	9.90326009657387e-07\\
25514.153712434	9.75134869370807e-07\\
25705.0819594888	9.60206021457348e-07\\
25897.4389662797	9.45534150174525e-07\\
26091.2354245425	9.31114054718273e-07\\
26286.4821060218	9.16940646950849e-07\\
26483.1898630694	9.03008949162704e-07\\
26681.3696292481	8.89314091868216e-07\\
26881.0324199387	8.75851311635083e-07\\
27082.189332953	8.62615948947101e-07\\
27284.8515491498	8.49603446100281e-07\\
27489.0303330572	8.36809345131912e-07\\
27694.7370334982	8.24229285782479e-07\\
27901.9830842215	8.1185900349017e-07\\
28110.7800045375	7.9969432741767e-07\\
28321.1393999579	7.8773117851114e-07\\
28533.0729628413	7.7596556759097e-07\\
28746.5924730428	7.64393593474166e-07\\
28961.7097985688	7.53011441128065e-07\\
29178.4368962368	7.41815379855086e-07\\
29396.78581234	7.30801761508296e-07\\
29616.7686833165	7.19967018737469e-07\\
29838.3977364244	7.09307663265371e-07\\
30061.6852904209	6.98820284193985e-07\\
30286.6437562476	6.88501546340359e-07\\
30513.2856377198	6.78348188601812e-07\\
30741.6235322216	6.68357022350198e-07\\
30971.6701314066	6.58524929854857e-07\\
31203.4382219025	6.48848862734066e-07\\
31436.9406860226	6.39325840434549e-07\\
31672.1905024814	6.2995294873883e-07\\
31909.2007471158	6.20727338300071e-07\\
32147.9845936127	6.11646223204058e-07\\
32388.5553142404	6.02706879558095e-07\\
32630.9262805866	5.93906644106364e-07\\
32875.1109643017	5.85242912871543e-07\\
33121.1229378475	5.76713139822288e-07\\
33368.9758752518	5.68314835566272e-07\\
33618.6835528682	5.60045566068455e-07\\
33870.2598501416	5.51902951394279e-07\\
34123.7187503806	5.43884664477396e-07\\
34379.0743415334	5.35988429911695e-07\\
34636.340816972	5.28212022767207e-07\\
34895.5324762807	5.20553267429617e-07\\
35156.6637260502	5.13010036463052e-07\\
35419.7490806799	5.05580249495766e-07\\
35684.8031631832	4.98261872128477e-07\\
35951.840706001	4.91052914864945e-07\\
36220.8765518205	4.83951432064522e-07\\
36491.9256543999	4.76955520916337e-07\\
36765.0030794002	4.7006332043475e-07\\
37040.1240052216	4.63273010475838e-07\\
37317.3037238483	4.56582810774489e-07\\
37596.5576416978	4.49990980001861e-07\\
37877.9012804769	4.43495814842862e-07\\
38161.3502780454	4.37095649093309e-07\\
38446.9203892846	4.30788852776492e-07\\
38734.6274869731	4.2457383127881e-07\\
39024.4875626694	4.18449024504142e-07\\
39316.5167276	4.12412906046707e-07\\
39610.7312135559	4.06463982382025e-07\\
39907.1473737941	4.00600792075744e-07\\
40205.7816839466	3.94821905009993e-07\\
40506.6507429366	3.89125921626941e-07\\
40809.7712739007	3.83511472189345e-07\\
41115.1601251185	3.77977216057673e-07\\
41422.8342709494	3.7252184098361e-07\\
41732.8108127753	3.67144062419614e-07\\
42045.1069799522	3.61842622844217e-07\\
42359.7401307671	3.56616291102819e-07\\
42676.7277534027	3.51463861763692e-07\\
42996.0874669105	3.46384154488861e-07\\
43317.8370221886	3.41376013419675e-07\\
43641.9943029698	3.36438306576703e-07\\
43968.5773268145	3.31569925273744e-07\\
44297.6042461126	3.26769783545668e-07\\
44629.0933490932	3.22036817589795e-07\\
44963.0630608394	3.1736998522061e-07\\
45299.531944314	3.12768265337483e-07\\
45638.5187013907	3.08230657405193e-07\\
45980.0421738933	3.03756180946993e-07\\
46324.1213446437	2.99343875049926e-07\\
46670.775338516	2.94992797882216e-07\\
47020.0234235008	2.90702026222426e-07\\
47371.885011775	2.86470655000178e-07\\
47726.3796607813	2.82297796848194e-07\\
48083.5270743154	2.78182581665412e-07\\
48443.347103621	2.74124156190964e-07\\
48805.8597484928	2.70121683588774e-07\\
49171.0851583894	2.66174343042556e-07\\
49539.0436335514	2.62281329361006e-07\\
49909.7556261313	2.58441852592945e-07\\
50283.2417413299	2.54655137652214e-07\\
50659.5227385407	2.50920423952124e-07\\
51038.6195325055	2.4723696504921e-07\\
51420.5531944749	2.43604028296141e-07\\
51805.3449533811	2.40020894503532e-07\\
52193.0161970173	2.36486857610503e-07\\
52583.5884732259	2.33001224363757e-07\\
52977.0834910972	2.29563314005006e-07\\
53373.5231221757	2.26172457966547e-07\\
53772.929401675	2.2282799957481e-07\\
54175.3245297041	2.19529293761675e-07\\
54580.7308724997	2.16275706783423e-07\\
54989.1709636708	2.13066615947085e-07\\
55400.6675054502	2.09901409344066e-07\\
55815.2433699566	2.06779485590851e-07\\
56232.9216004666	2.03700253576622e-07\\
56653.725412694	2.00663132217648e-07\\
57077.6781960819	1.97667550218244e-07\\
57504.8035151016	1.94712945838181e-07\\
57935.1251105625	1.91798766666379e-07\\
58368.6669009326	1.88924469400709e-07\\
58805.4529836665	1.86089519633796e-07\\
59245.5076365459	1.83293391644631e-07\\
59688.8553190289	1.80535568195887e-07\\
60135.5206736089	1.77815540336771e-07\\
60585.528527185	1.75132807211288e-07\\
61038.9038924417	1.72486875871771e-07\\
61495.6719692387	1.69877261097563e-07\\
61955.8581460127	1.67303485218681e-07\\
62419.4880011873	1.64765077944395e-07\\
62886.5873045955	1.62261576196532e-07\\
63357.182018912	1.59792523947427e-07\\
63831.2983010957	1.57357472062388e-07\\
64308.9625038448	1.54955978146543e-07\\
64790.2011770598	1.52587606395984e-07\\
65275.0410693208	1.50251927453059e-07\\
65763.5091293736	1.4794851826574e-07\\
66255.6325076273	1.45676961950922e-07\\
66751.438557664	1.43436847661574e-07\\
67250.9548377589	1.41227770457628e-07\\
67754.209112412	1.39049331180506e-07\\
68261.2293538915	1.36901136331184e-07\\
68772.0437437882	1.34782797951706e-07\\
69286.6806745825	1.32693933510033e-07\\
69805.1687512222	1.30634165788162e-07\\
70327.5367927121	1.28603122773405e-07\\
70853.8138337169	1.26600437552735e-07\\
71384.0291261735	1.24625748210148e-07\\
71918.2121409184	1.22678697726901e-07\\
72456.392569325	1.20758933884602e-07\\
72998.6003249533	1.18866109171028e-07\\
73544.8655452147	1.16999880688609e-07\\
74095.2185930443	1.15159910065517e-07\\
74649.690058591	1.13345863369247e-07\\
75208.3107609164	1.1155741102266e-07\\
75771.111749708	1.09794227722384e-07\\
76338.1243070055	1.08055992359521e-07\\
76909.3799489393	1.06342387942589e-07\\
77484.9104274818	1.04653101522631e-07\\
78064.7477322134	1.02987824120419e-07\\
78648.9240920989	1.01346250655717e-07\\
79237.4719772807	9.97280798785028e-08\\
79830.4241008822	9.81330143021242e-08\\
80427.8134208264	9.6560760138312e-08\\
81029.6731416689	9.50110272339945e-08\\
81636.0367164413	9.34835290098716e-08\\
82246.937848513	9.19779824006749e-08\\
82862.4104934629	9.04941077970785e-08\\
83482.488860967	8.90316289892036e-08\\
84107.2074167008	8.75902731116613e-08\\
84736.6008842538	8.61697705900971e-08\\
85370.7042470603	8.47698550891814e-08\\
86009.5527503438	8.33902634620029e-08\\
86653.1819030753	8.20307357008229e-08\\
87301.627479948	8.06910148891421e-08\\
87954.9255233655	7.93708471550435e-08\\
88613.1123454441	7.80699816257694e-08\\
89276.2245300332	7.67881703834853e-08\\
89944.2989347464	7.55251684222087e-08\\
90617.3726930118	7.42807336058451e-08\\
91295.4832161355	7.3054626627311e-08\\
91978.6681953803	7.18466109687023e-08\\
92666.9656040625	7.06564528624683e-08\\
93360.4136996602	6.94839212535731e-08\\
94059.0510259418	6.83287877625935e-08\\
94762.9164151073	6.71908266497362e-08\\
95472.0489899465	6.60698147797389e-08\\
96186.4881660146	6.49655315876222e-08\\
96906.2736538223	6.38777590452704e-08\\
97631.4454610423	6.280628162881e-08\\
98362.0438947356	6.17508862867545e-08\\
99098.1095635883	6.0711362408902e-08\\
99839.6833801717	5.96875017959442e-08\\
100586.806563215	5.86790986297744e-08\\
101339.520639896	5.76859494444671e-08\\
102097.867448152	5.67078530979017e-08\\
102861.889138999	5.57446107440178e-08\\
103631.628178883	5.47960258056691e-08\\
104407.127352034	5.38619039480633e-08\\
105188.429762846	5.29420530527659e-08\\
105975.578838274	5.20362831922447e-08\\
106768.618330246	5.11444066049411e-08\\
107567.592318097	5.02662376708468e-08\\
108372.545211017	4.94015928875686e-08\\
109183.521750518	4.85502908468663e-08\\
110000.567012926	4.7712152211643e-08\\
110823.726411883	4.68869996933756e-08\\
111653.04570087	4.60746580299701e-08\\
112488.570975754	4.52749539640217e-08\\
113330.348677345	4.44877162214738e-08\\
114178.425593984	4.37127754906531e-08\\
115032.848864136	4.29499644016737e-08\\
115893.665979016	4.21991175061956e-08\\
116760.924785228	4.14600712575215e-08\\
117634.673487419	4.07326639910276e-08\\
118514.960650966	4.00167359049083e-08\\
119401.835204671	3.93121290412288e-08\\
120295.34644348	3.86186872672754e-08\\
121195.544031226	3.7936256257187e-08\\
122102.478003387	3.72646834738668e-08\\
123016.198769868	3.66038181511573e-08\\
123936.757117803	3.59535112762709e-08\\
124864.204214378	3.53136155724718e-08\\
125798.591609674	3.46839854819934e-08\\
126739.971239535	3.40644771491884e-08\\
127688.395428448	3.34549484039028e-08\\
128643.916892463	3.28552587450617e-08\\
129606.588742111	3.22652693244687e-08\\
130576.464485363	3.16848429308007e-08\\
131553.598030603	3.11138439738014e-08\\
132538.043689621	3.05521384686608e-08\\
133529.856180639	2.99995940205761e-08\\
134529.090631344	2.94560798094895e-08\\
135535.802581959	2.89214665749958e-08\\
136550.047988325	2.8395626601414e-08\\
137571.883225014	2.78784337030196e-08\\
138601.365088463	2.73697632094298e-08\\
139638.550800127	2.68694919511396e-08\\
140683.498009666	2.63774982452031e-08\\
141736.264798142	2.58936618810537e-08\\
142796.909681254	2.54178641064642e-08\\
143865.491612584	2.49499876136357e-08\\
144942.069986881	2.44899165254183e-08\\
146026.704643354	2.40375363816555e-08\\
147119.455869007	2.359273412565e-08\\
148220.384401982	2.31553980907495e-08\\
149329.55143494	2.2725417987046e-08\\
150447.018618462	2.23026848881881e-08\\
151572.84806447	2.18870912183035e-08\\
152707.102349689	2.14785307390262e-08\\
153849.844519117	2.10768985366299e-08\\
155001.138089531	2.06820910092626e-08\\
156161.047053023	2.02940058542788e-08\\
157329.635880547	1.9912542055672e-08\\
158506.969525512	1.95375998715996e-08\\
159693.113427386	1.91690808220019e-08\\
160888.133515336	1.88068876763133e-08\\
162092.096211894	1.84509244412606e-08\\
163305.068436644	1.8101096348752e-08\\
164527.117609946	1.775730984385e-08\\
165758.311656683	1.74194725728303e-08\\
166998.719010032	1.70874933713237e-08\\
168248.408615274	1.67612822525399e-08\\
169507.449933624	1.64407503955721e-08\\
170775.912946089	1.61258101337818e-08\\
172053.868157361	1.58163749432612e-08\\
173341.386599734	1.55123594313739e-08\\
174638.539837053	1.52136793253714e-08\\
175945.399968693	1.49202514610857e-08\\
177262.039633563	1.46319937716972e-08\\
178588.532014148	1.4348825276575e-08\\
179924.950840571	1.40706660701926e-08\\
181271.370394698	1.37974373111143e-08\\
182627.865514261	1.35290612110546e-08\\
183994.51159702	1.32654610240092e-08\\
185371.384604954	1.30065610354556e-08\\
186758.561068483	1.27522865516257e-08\\
188156.118090722	1.25025638888468e-08\\
189564.133351766	1.2257320362953e-08\\
190982.685113006	1.20164842787656e-08\\
192411.852221483	1.17799849196411e-08\\
193851.71411427	1.15477525370897e-08\\
195302.350822881	1.13197183404597e-08\\
196763.842977729	1.1095814486691e-08\\
198236.2718126	1.0875974070136e-08\\
199719.719169172	1.06601311124474e-08\\
201214.267501562	1.04482205525338e-08\\
202719.999880911	1.02401782365818e-08\\
204237	1.00359409081449e-08\\
};
\addlegendentry{Adatt. BLN reale (\textit{ML})}


        \nextgroupplot[%
            plotColumn1,plotLegend1,xmode=log,
            xmin=100,xmax=1000000,
            ymin=0,ymax=0.00045,
        ] 
% !TEX root = ../../../../Esperimenti/.tex/MEF.tex

% This file was created by matlab2tikz.
%
\definecolor{mycolor1}{rgb}{0.83529,0.36863,0.00000}%

\addplot[ybar interval, fill=mycolor1, fill opacity=0.35, area legend, draw=none] table[row sep=crcr, x=Lower, y=Count] {%
Lower	Upper	Count\\
111.473131801004	156.850807223597	1.17532096645874e-06\\
156.850807223597	220.700498220617	7.51765580231869e-06\\
220.700498220617	310.541659153798	2.5526532710758e-05\\
310.541659153798	436.954709425234	6.18079645776798e-05\\
436.954709425234	614.827068964461	0.000154267926006505\\
614.827068964461	865.10642082028	0.000248788668361812\\
865.10642082028	1217.26767919475	0.00033022750392854\\
1217.26767919475	1712.78419296345	0.000332689887728512\\
1712.78419296345	2410.01198159317	0.000241068991714072\\
2410.01198159317	3391.06221045479	0.00014436229919747\\
3391.06221045479	4771.47126362945	7.22877515452229e-05\\
4771.47126362945	6713.807239941	3.44602242607061e-05\\
6713.807239941	9446.8152828814	1.75728230257952e-05\\
9446.8152828814	13292.3564528293	8.30052223844292e-06\\
13292.3564528293	18703.3126803322	4.38616250722947e-06\\
18703.3126803322	26316.9217933377	2.2696200636939e-06\\
26316.9217933377	37029.8237811609	1.09276335020829e-06\\
37029.8237811609	52103.6563482497	3.82119144176778e-07\\
52103.6563482497	73313.6355414593	1.30756061006473e-07\\
73313.6355414593	103157.619503348	2.23383938122341e-08\\
103157.619503348	145150.276384529	3.81018996851571e-09\\
145150.276384529	204237	9.02628036721417e-10\\
204237	204237	9.02628036721417e-10\\
};
\addlegendentry{Istogramma appr.}

\addplot [color=mycolor1, only marks, every error bar/.append style={opacity=0.45}, mark=*, mark size=0pt, draw=none, forget plot]
plot [error bars/.cd, y dir=both, y explicit, error bar style={line width=1pt, color=mycolor1}, error mark options={mark=none,mark size=0pt}]
table[row sep=crcr, y error plus index=2, y error minus index=3]{%
132.229537950981	1.17532096645874e-06	1.64068830288434e-06	1.17532096645874e-06\\
186.056580911705	7.51765580231869e-06	3.60742995943025e-06	3.60742995943025e-06\\
261.795146810441	2.5526532710758e-05	6.9545240059212e-06	6.9545240059212e-06\\
368.36481976429	6.18079645776798e-05	1.12448207145163e-05	1.12448207145163e-05\\
518.316103614517	0.000154267926006505	1.86291307218131e-05	1.86291307218131e-05\\
729.308470439819	0.000248788668361812	2.24411541826347e-05	2.24411541826348e-05\\
1026.19008235725	0.00033022750392854	1.91677267606224e-05	1.91677267606224e-05\\
1443.92411141655	0.000332689887728512	1.09038551397073e-05	1.09038551397073e-05\\
2031.70628461038	0.000241068991714072	5.36562578657546e-06	5.36562578657546e-06\\
2858.75856929609	0.00014436229919747	5.46106788915819e-06	5.4610678891582e-06\\
4022.48131013244	7.22877515452229e-05	3.65681982894919e-06	3.65681982894919e-06\\
5659.92387889853	3.44602242607061e-05	2.06579169976343e-06	2.06579169976343e-06\\
7963.92471339317	1.75728230257952e-05	1.03640209749361e-06	1.03640209749361e-06\\
11205.8215265145	8.30052223844292e-06	4.64223842147829e-07	4.6422384214783e-07\\
15767.4062228287	4.38616250722947e-06	3.26402913075513e-07	3.26402913075513e-07\\
22185.8877912209	2.2696200636939e-06	1.8146326991141e-07	1.8146326991141e-07\\
31217.1583663518	1.09276335020829e-06	1.1273781617234e-07	1.1273781617234e-07\\
43924.8131803637	3.82119144176778e-07	5.57185303016802e-08	5.57185303016802e-08\\
61805.4081120173	1.30756061006473e-07	2.48003762760863e-08	2.48003762760863e-08\\
86964.7061720613	2.23383938122341e-08	8.11694387701619e-09	8.11694387701619e-09\\
122365.669131832	3.81018996851571e-09	3.00750590802471e-09	3.00750590802471e-09\\
172177.399788553	9.02628036721417e-10	1.26002283968979e-09	9.02628036721417e-10\\
};
\addplot [color=mycolor1]
table[row sep=crcr]{%
111.473131801004	2.1987140199919e-06\\
112.314656473384	2.24170395328633e-06\\
113.162533921208	2.28561719255329e-06\\
114.016812102373	2.3304765428334e-06\\
114.877539336821	2.3763054160293e-06\\
115.744764309265	2.42312784591946e-06\\
116.618536071944	2.4709685034442e-06\\
117.4989040474	2.51985271226425e-06\\
118.385918031273	2.56980646459179e-06\\
119.279628195113	2.62085643729378e-06\\
120.180085089226	2.67303000826696e-06\\
121.087339645525	2.72635527308391e-06\\
122.001443180416	2.78086106190884e-06\\
122.922447397698	2.83657695668194e-06\\
123.850404391489	2.89353330857046e-06\\
124.785366649173	2.95176125568444e-06\\
125.727387054366	3.01129274105484e-06\\
126.676518889911	3.07216053087115e-06\\
127.632815840887	3.13439823297558e-06\\
128.596331997653	3.19804031561003e-06\\
129.567121858898	3.26312212641226e-06\\
130.545240334732	3.32967991165675e-06\\
131.530742749786	3.39775083573553e-06\\
132.523684846345	3.46737300087383e-06\\
133.524122787498	3.53858546707495e-06\\
134.53211316032	3.6114282722882e-06\\
135.547712979064	3.68594245279325e-06\\
136.570979688393	3.76217006379399e-06\\
137.601971166627	3.84015420021409e-06\\
138.640745729018	3.91993901768628e-06\\
139.687362131045	4.00156975372656e-06\\
140.74187957174	4.08509274908423e-06\\
141.804357697036	4.17055546925786e-06\\
142.874856603141	4.25800652616678e-06\\
143.953436839938	4.34749569996722e-06\\
145.040159414406	4.4390739610013e-06\\
146.135085794075	4.53279349186693e-06\\
147.2382779105	4.62870770959538e-06\\
148.349798162768	4.72687128792327e-06\\
149.469709421021	4.82734017964467e-06\\
150.598075030017	4.93017163902832e-06\\
151.734958812712	5.03542424428457e-06\\
152.880425073868	5.14315792006555e-06\\
154.034538603695	5.25343395998172e-06\\
155.197364681509	5.36631504911685e-06\\
156.368969079428	5.48186528652298e-06\\
157.549418066095	5.60015020767629e-06\\
158.73877841042	5.72123680687348e-06\\
159.937117385361	5.84519355954811e-06\\
161.144502771729	5.97209044448511e-06\\
162.36100286202	6.10199896591129e-06\\
163.586686464278	6.23499217543827e-06\\
164.821622905988	6.3711446938341e-06\\
166.065882037999	6.5105327325985e-06\\
167.319534238469	6.65323411531623e-06\\
168.582650416852	6.79932829876171e-06\\
169.855302017906	6.9488963937279e-06\\
171.137561025735	7.10202118555099e-06\\
172.429499967858	7.25878715430172e-06\\
173.731191919316	7.41928049461369e-06\\
175.042710506801	7.58358913511758e-06\\
176.364129912823	7.75180275744982e-06\\
177.695524879903	7.92401281480312e-06\\
179.036970714807	8.10031254998547e-06\\
180.388543292795	8.28079701295339e-06\\
181.750319061926	8.46556307778462e-06\\
183.122375047369	8.65470945905389e-06\\
184.504788855768	8.84833672757537e-06\\
185.897638679631	9.04654732547413e-06\\
187.301003301749	9.24944558054821e-06\\
188.714962099654	9.45713771988196e-06\\
190.13959505011	9.66973188267099e-06\\
191.574982733635	9.88733813221748e-06\\
193.02120633906	1.01100684670547e-05\\
194.478347668119	1.03380368311579e-05\\
195.946489140081	1.05713591231995e-05\\
197.425713796405	1.08101532048035e-05\\
198.916105305443	1.10545389077551e-05\\
200.417747967167	1.1304638040122e-05\\
201.930726717942	1.15605743912378e-05\\
203.455127135329	1.18224737355057e-05\\
204.991035442923	1.20904638349718e-05\\
206.538538515233	1.23646744406223e-05\\
208.097723882594	1.26452372923558e-05\\
209.668679736118	1.29322861175827e-05\\
211.251494932683	1.32259566284014e-05\\
212.846258999961	1.35263865173038e-05\\
214.453062141476	1.38337154513579e-05\\
216.071995241712	1.41480850648189e-05\\
217.70314987125	1.44696389501178e-05\\
219.346618291952	1.47985226471766e-05\\
221.002493462171	1.51348836309988e-05\\
222.67086904202	1.54788712974855e-05\\
224.35183939866	1.58306369474231e-05\\
226.045499611642	1.61903337685952e-05\\
227.751945478287	1.65581168159628e-05\\
229.471273519099	1.6934142989866e-05\\
231.20358098323	1.73185710121928e-05\\
232.948965853976	1.77115614004661e-05\\
234.707526854322	1.81132764397969e-05\\
236.479363452525	1.85238801526548e-05\\
238.264575867741	1.89435382664042e-05\\
240.063265075692	1.9372418178558e-05\\
241.875532814379	1.98106889196999e-05\\
243.701481589837	2.02585211140248e-05\\
245.54121468193	2.07160869374539e-05\\
247.394836150195	2.11835600732726e-05\\
249.262450839729	2.16611156652489e-05\\
251.144164387117	2.2148930268186e-05\\
253.040083226407	2.26471817958641e-05\\
254.950314595131	2.31560494663297e-05\\
256.874966540373	2.367571374449e-05\\
258.814147924875	2.42063562819723e-05\\
260.767968433199	2.47481598542077e-05\\
262.73653857793	2.5301308294704e-05\\
264.719969705925	2.58659864264693e-05\\
266.718374004613	2.64423799905528e-05\\
268.731864508341	2.70306755716701e-05\\
270.760555104764	2.76310605208813e-05\\
272.804560541293	2.82437228752944e-05\\
274.863996431577	2.88688512747648e-05\\
276.938979262052	2.95066348755689e-05\\
279.02962639852	3.01572632610257e-05\\
281.136056092795	3.08209263490501e-05\\
283.258387489388	3.14978142966166e-05\\
285.396740632247	3.21881174011203e-05\\
287.551236471548	3.28920259986212e-05\\
289.721996870532	3.36097303589631e-05\\
291.909144612403	3.43414205777593e-05\\
294.112803407272	3.50872864652405e-05\\
296.33309789915	3.58475174319651e-05\\
298.570153673005	3.66223023713926e-05\\
300.824097261858	3.74118295393269e-05\\
303.095056153947	3.82162864302358e-05\\
305.383158799932	3.90358596504612e-05\\
307.688534620164	3.98707347883343e-05\\
310.011314012007	4.07210962812148e-05\\
312.351628357206	4.15871272794774e-05\\
314.709610029328	4.24690095074727e-05\\
317.085392401243	4.33669231214913e-05\\
319.479109852668	4.42810465647684e-05\\
321.890897777773	4.52115564195646e-05\\
324.32089259283	4.61586272563672e-05\\
326.769231743941	4.7122431480259e-05\\
329.236053714802	4.81031391745045e-05\\
331.72149803454	4.91009179414099e-05\\
334.225705285606	5.01159327405171e-05\\
336.748817111725	5.11483457241946e-05\\
339.290976225908	5.21983160706957e-05\\
341.852326418525	5.32659998147567e-05\\
344.433012565439	5.43515496758127e-05\\
347.033180636195	5.54551148839144e-05\\
349.652977702284	5.65768410034337e-05\\
352.292551945457	5.77168697546484e-05\\
354.952052666108	5.88753388333051e-05\\
357.631630291718	6.00523817282597e-05\\
360.331436385364	6.12481275373043e-05\\
363.051623654292	6.24627007812897e-05\\
365.792345958554	6.3696221216661e-05\\
368.55375831971	6.49488036465253e-05\\
371.336016929598	6.62205577303795e-05\\
374.13927915917	6.75115877926278e-05\\
376.963703567387	6.8821992630021e-05\\
379.809449910193	7.01518653181637e-05\\
382.676679149551	7.15012930172279e-05\\
385.565553462546	7.28703567770249e-05\\
388.476236250556	7.42591313415901e-05\\
391.408892148499	7.56676849534369e-05\\
394.363687034141	7.70960791576441e-05\\
397.340788037481	7.85443686059416e-05\\
400.340363550203	8.00126008609713e-05\\
403.3625832352	8.15008162008896e-05\\
406.407618036173	8.30090474245019e-05\\
409.475640187299	8.4537319657104e-05\\
412.56682322297	8.60856501572236e-05\\
415.681341987614	8.76540481244497e-05\\
418.819372645578	8.92425145085484e-05\\
421.981092691098	9.08510418200604e-05\\
425.166680958337	9.24796139425831e-05\\
428.376317631498	9.41282059469444e-05\\
431.610184255017	9.57967839074717e-05\\
434.868463743833	9.7485304720573e-05\\
438.151340393733	9.91937159258376e-05\\
441.458999891773	0.000100921955529877\\
444.791629326788	0.000102669951833121\\
448.149417199969	0.000104437623259796\\
451.532553435523	0.000106224878191294\\
454.941229391423	0.000108031614803173\\
458.375637870226	0.000109857720906006\\
461.835973129977	0.000111703073790297\\
465.322430895204	0.000113567540075715\\
468.835208367981	0.000115450975564844\\
472.374504239085	0.000117353225101706\\
475.940518699236	0.000119274122435264\\
479.533453450418	0.000121213490088146\\
483.15351171729	0.000123171139230822\\
486.800898258679	0.000125146869561447\\
490.47581937916	0.000127140469191609\\
494.178482940729	0.000129151714538207\\
497.90909837456	0.000131180370221679\\
501.667876692845	0.000133226188970809\\
505.455030500736	0.000135288911534325\\
509.27077400837	0.000137368266599514\\
513.115323042978	0.000139463970718069\\
516.988895061104	0.000141575728239379\\
520.891709160893	0.000143703231251464\\
524.823986094494	0.000145846159529784\\
528.78594828054	0.000148004180494096\\
532.777819816728	0.000150176949173572\\
536.799826492501	0.000152364108180379\\
540.85219580181	0.000154565287691886\\
544.935156955991	0.000156780105441709\\
549.048940896723	0.000159008166719749\\
553.193780309092	0.000161249064381412\\
557.369909634754	0.000163502378866166\\
561.577565085194	0.0001657676782256\\
565.816984655088	0.000168044518161143\\
570.088408135764	0.000170332442071576\\
574.392077128762	0.000172630981110489\\
578.728235059506	0.000174939654253807\\
583.097127191066	0.000177257968377512\\
587.499000638037	0.000179585418345674\\
591.93410438051	0.000181921487108896\\
596.402689278159	0.000184265645813279\\
600.905008084431	0.000186617353919989\\
605.441315460838	0.0001889760593355\\
610.011867991365	0.000191341198552603\\
614.61692419698	0.000193712196802218\\
619.256744550263	0.000196088468216075\\
623.931591490128	0.0001984694160003\\
628.641729436677	0.000200854432619926\\
633.387424806152	0.00020324289999437\\
638.168946026005	0.000205634189703854\\
642.986563550079	0.000208027663206793\\
647.840549873909	0.000210422672068117\\
652.731179550132	0.000212818558198514\\
657.658729204016	0.000215214654104538\\
662.623477549112	0.000217610283149555\\
667.625705403009	0.000220004759825444\\
672.665695703228	0.000222397390034989\\
677.743733523219	0.000224787471384872\\
682.860106088486	0.000227174293489174\\
688.015102792837	0.000229557138283253\\
693.209015214748	0.000231935280347905\\
698.442137133859	0.00023430798724364\\
703.714764547589	0.000236674519854942\\
709.02719568788	0.000239034132744348\\
714.379731038059	0.000241386074516155\\
719.772673349847	0.000243729588189594\\
725.20632766047	0.000246063911581241\\
730.681001309921	0.000248388277696464\\
736.197003958341	0.000250701915129688\\
741.754647603534	0.000253004048473214\\
747.354246598613	0.000255293898734365\\
752.996117669786	0.000257570683760669\\
758.680579934264	0.000259833618672827\\
764.407954918312	0.000262081916305152\\
770.178566575443	0.000264314787653191\\
775.99274130473	0.000266531442328211\\
781.850807969277	0.000268731089018221\\
787.753097914816	0.000270912935955194\\
793.699944988449	0.000273076191388135\\
799.691685557532	0.000275220064061645\\
805.728658528701	0.000277343763699588\\
811.81120536704	0.000279446501493503\\
817.939670115397	0.000281527490595345\\
824.114399413841	0.000283585946614171\\
830.33574251927	0.000285621088116336\\
836.604051325169	0.000287632137128803\\
842.919680381509	0.000289618319645105\\
849.282986914804	0.000291578866133545\\
855.694330848315	0.000293513012047165\\
862.154074822409	0.000295419998335038\\
868.662584215073	0.000297299071954418\\
875.220227162575	0.000299149486383272\\
881.827374580292	0.000300970502132717\\
888.484400183686	0.000302761387258882\\
895.191680509447	0.000304521417873703\\
901.949594936784	0.00030624987865416\\
908.758525708891	0.000307946063349447\\
915.61885795456	0.000309609275285587\\
922.530979709972	0.00031123882786698\\
929.495281940641	0.000312834045074373\\
936.512158563526	0.000314394261958747\\
943.582006469321	0.000315918825130603\\
950.705225544888	0.000317407093244149\\
957.882218695896	0.000318858437475856\\
965.113391869591	0.000320272241996889\\
972.39915407777	0.000321647904438891\\
979.73991741991	0.000322984836352626\\
987.13609710648	0.000324282463658953\\
994.588111482424	0.000325540227091656\\
1002.09638205082	0.000326757582631612\\
1009.66133349674	0.000327934001931815\\
1017.28339371124	0.000329068972732762\\
1024.96299381559	0.000330161999267726\\
1032.70056818564	0.000331212602657444\\
1040.49655447641	0.000332220321293738\\
1048.35139364682	0.000333184711211639\\
1056.26552998465	0.000334105346449538\\
1064.23941113167	0.000334981819396948\\
1072.27348810894	0.000335813741129429\\
1080.36821534236	0.000336600741730287\\
1088.5240506883	0.000337342470598612\\
1096.74145545961	0.000338038596743289\\
1105.02089445159	0.00033868880906259\\
1113.36283596838	0.000339292816608997\\
1121.76775184938	0.000339850348838886\\
1130.23611749598	0.000340361155846767\\
1138.76841189842	0.000340825008583742\\
1147.36511766291	0.000341241699059887\\
1156.02672103892	0.000341611040530279\\
1164.75371194667	0.000341932867664403\\
1173.54658400484	0.000342207036698682\\
1182.40583455852	0.000342433425571922\\
1191.3319647073	0.000342611934043443\\
1200.32547933364	0.000342742483793726\\
1209.38688713144	0.000342825018507392\\
1218.51670063476	0.000342859503938381\\
1227.71543624688	0.000342845927957194\\
1236.98361426944	0.000342784300580099\\
1246.32175893192	0.000342674653980214\\
1255.73039842129	0.000342517042480415\\
1265.21006491183	0.000342311542528017\\
1274.7612945953	0.000342058252651223\\
1284.38462771124	0.000341757293397344\\
1294.0806085775	0.000341408807252816\\
1303.84978562109	0.000341012958545069\\
1313.69271140914	0.000340569933326328\\
1323.60994268017	0.000340079939239425\\
1333.60204037562	0.000339543205365769\\
1343.66956967153	0.000338959982055593\\
1353.81310001052	0.000338330540740658\\
1364.03320513401	0.000337655173729593\\
1374.33046311466	0.000336934193986091\\
1384.70545638909	0.000336167934890182\\
1395.1587717908	0.000335356749982842\\
1405.69100058334	0.000334501012694218\\
1416.30273849384	0.000333601116055751\\
1426.99458574659	0.000332657472396525\\
1437.7671470971	0.00033167051302418\\
1448.62103186621	0.000330640687890729\\
1459.55685397464	0.000329568465243674\\
1470.57523197764	0.000328454331262802\\
1481.67678910003	0.000327298789683073\\
1492.86215327143	0.000326102361404035\\
1504.13195716179	0.000324865584086202\\
1515.48683821714	0.000323589011734864\\
1526.92743869569	0.000322273214271804\\
1538.45440570412	0.000320918777095407\\
1550.06839123423	0.000319526300629668\\
1561.77005219976	0.000318096399862618\\
1573.56005047358	0.000316629703874699\\
1585.43905292513	0.000315126855357617\\
1597.40773145812	0.000313588510124238\\
1609.46676304856	0.000312015336610071\\
1621.61682978302	0.000310408015366923\\
1633.85861889724	0.000308767238549288\\
1646.19282281499	0.000307093709394056\\
1658.62013918722	0.000305388141694135\\
1671.14127093155	0.000303651259266567\\
1683.756926272	0.000301883795415751\\
1696.46781877908	0.000300086492392356\\
1709.2746674101	0.000298260100848532\\
1722.17819654992	0.000296405379290023\\
1735.17913605182	0.000294523093525779\\
1748.27822127887	0.000292614016115671\\
1761.47619314548	0.000290678925816905\\
1774.77379815929	0.000288718607029733\\
1788.17178846346	0.000286733849243056\\
1801.67092187916	0.000284725446480495\\
1815.27196194842	0.000282694196747529\\
1828.97567797739	0.00028064090148026\\
1842.78284507977	0.000278566364996381\\
1856.6942442207	0.000276471393948904\\
1870.71066226095	0.000274356796783203\\
1884.83289200137	0.000272223383197911\\
1899.06173222778	0.000270071963610193\\
1913.39798775612	0.000267903348625933\\
1927.84246947799	0.000265718348515329\\
1942.39599440654	0.000263517772694388\\
1957.05938572261	0.000261302429212824\\
1971.83347282137	0.000259073124248794\\
1986.71909135918	0.000256830661610962\\
2001.71708330089	0.000254575842248298\\
2016.82829696743	0.00025230946376806\\
2032.05358708382	0.000250032319962347\\
2047.39381482752	0.000247745200343629\\
2062.8498478771	0.000245448889689618\\
2078.42256046136	0.000243144167597843\\
2094.11283340877	0.000240831808050272\\
2109.92155419728	0.000238512578988285\\
2125.8496170045	0.000236187241898333\\
2141.89792275834	0.00023385655140854\\
2158.06737918789	0.000231521254896529\\
2174.35890087482	0.000229182092108724\\
2190.77340930509	0.000226839794791354\\
2207.31183292107	0.000224495086333365\\
2223.97510717407	0.000222148681421453\\
2240.76417457722	0.000219801285707367\\
2257.67998475881	0.000217453595487662\\
2274.72349451599	0.000215106297396027\\
2291.89566786891	0.000212760068108302\\
2309.19747611519	0.00021041557406031\\
2326.62989788494	0.000208073471178555\\
2344.19391919604	0.000205734404623877\\
2361.89053350997	0.000203399008548092\\
2379.72074178795	0.000201067905863673\\
2397.6855525476	0.000198741708026452\\
2415.78598191995	0.000196421014831366\\
2434.02305370695	0.000194106414221221\\
2452.39779943935	0.000191798482108426\\
2470.91125843504	0.000189497782209664\\
2489.56447785789	0.000187204865893414\\
2508.3585127769	0.000184920272040266\\
2527.29442622594	0.000182644526915899\\
2546.37328926386	0.000180378144056654\\
2565.59618103505	0.000178121624167538\\
2584.96418883051	0.000175875455032551\\
2604.47840814933	0.000173640111437171\\
2624.13994276068	0.000171416055102848\\
2643.94990476618	0.00016920373463332\\
2663.90941466289	0.000167003585472581\\
2684.01960140661	0.000164816029874298\\
2704.28160247578	0.000162641476882484\\
2724.69656393583	0.000160480322323193\\
2745.26564050395	0.000158332948807042\\
2765.98999561444	0.00015619972574231\\
2786.87080148452	0.00015408100935838\\
2807.90923918064	0.000151977142739287\\
2829.10649868523	0.000149888455867111\\
2850.46377896409	0.000147815265674958\\
2871.98228803412	0.000145757876109275\\
2893.66324303174	0.00014371657820121\\
2915.50787028164	0.000141691650146778\\
2937.51740536621	0.00013968335739551\\
2959.69309319542	0.000137691952747353\\
2982.03618807719	0.000135717676457495\\
3004.5479537884	0.000133760756348863\\
3027.22966364631	0.00013182140793198\\
3050.08260058064	0.000129899834531902\\
3073.10805720609	0.000127996227421959\\
3096.30733589548	0.000126110765963969\\
3119.68174885339	0.000124243617754682\\
3143.23261819043	0.000122394938778124\\
3166.96127599795	0.000120564873563566\\
3190.86906442344	0.000118753555348833\\
3214.95733574645	0.000116961106248649\\
3239.22745245503	0.000115187637427749\\
3263.68078732284	0.000113433249278458\\
3288.31872348677	0.000111698031602462\\
3313.14265452519	0.000109982063796492\\
3338.15398453677	0.00010828541504164\\
3363.3541282199	0.000106608144496039\\
3388.74451095268	0.000104950301490634\\
3414.32656887361	0.000103311925727774\\
3440.10174896275	0.000101693047482379\\
3466.07150912361	0.000100093687805402\\
3492.23731826558	9.85138587293501e-05\\
3518.60065638705	9.69535634756155e-05\\
3545.16301465908	9.54127966633619e-05\\
3571.9258955098	9.38915445197423e-05\\
3598.89081270933	9.23897850912047e-05\\
3626.05929145545	9.09074884556669e-05\\
3653.43286845983	8.94446169353316e-05\\
3681.01309203499	8.80011253099314e-05\\
3708.80152218184	8.65769610301907e-05\\
3736.79973067795	8.51720644312998e-05\\
3765.00930116641	8.37863689462068e-05\\
3793.43182924546	8.24198013185278e-05\\
3822.06892255869	8.10722818148994e-05\\
3850.92220088601	7.97437244365815e-05\\
3879.99329623524	7.8434037130149e-05\\
3909.28385293442	7.71431219970936e-05\\
3938.79552772486	7.58708755021829e-05\\
3968.5299898548	7.46171886804156e-05\\
3998.48892117385	7.33819473424285e-05\\
4028.67401622809	7.21650322782094e-05\\
4059.08698235599	7.09663194589788e-05\\
4089.72953978488	6.97856802371159e-05\\
4120.60342172835	6.86229815439949e-05\\
4151.71037448419	6.74780860856241e-05\\
4183.05215753326	6.63508525359663e-05\\
4214.63054363891	6.52411357278407e-05\\
4246.44731894735	6.41487868413035e-05\\
4278.50428308861	6.30736535894116e-05\\
4310.80324927834	6.20155804012843e-05\\
4343.34604442043	6.09744086023751e-05\\
4376.13450921022	5.99499765918819e-05\\
4409.17049823878	5.89421200172163e-05\\
4442.45588009764	5.79506719454743e-05\\
4475.99253748463	5.69754630318381e-05\\
4509.78236731024	5.6016321684861e-05\\
4543.82728080503	5.50730742285771e-05\\
4578.12920362765	5.41455450613958e-05\\
4612.69007597378	5.32335568117357e-05\\
4647.51185268592	5.23369304903642e-05\\
4682.59650336387	5.14554856394093e-05\\
4717.94601247624	5.0589040478016e-05\\
4753.5623794726	4.9737412044627e-05\\
4789.44761889663	4.89004163358646e-05\\
4825.60376050007	4.80778684420045e-05\\
4862.03284935749	4.72695826790258e-05\\
4898.73694598198	4.64753727172339e-05\\
4935.71812644172	4.56950517064525e-05\\
4972.9784824774	4.49284323977846e-05\\
5010.52012162047	4.41753272619482e-05\\
5048.34516731247	4.34355486041913e-05\\
5086.45575902501	4.27089086758025e-05\\
5124.85405238087	4.19952197822231e-05\\
5163.54221927589	4.12942943877867e-05\\
5202.52244800183	4.0605945217098e-05\\
5241.79694337011	3.99299853530782e-05\\
5281.3679268366	3.92662283317006e-05\\
5321.23763662717	3.86144882334435e-05\\
5361.40832786437	3.79745797714916e-05\\
5401.88227269494	3.73463183767167e-05\\
5442.66176041833	3.67295202794733e-05\\
5483.74909761624	3.61240025882437e-05\\
5525.14660828299	3.55295833651738e-05\\
5566.85663395707	3.49460816985362e-05\\
5608.88153385351	3.43733177721649e-05\\
5651.22368499737	3.3811112931903e-05\\
5693.88548235815	3.32592897491098e-05\\
5736.8693389853	3.27176720812723e-05\\
5780.17768614467	3.21860851297678e-05\\
5823.81297345603	3.16643554948286e-05\\
5867.77766903166	3.11523112277541e-05\\
5912.07425961591	3.06497818804261e-05\\
5956.7052507259	3.01565985521726e-05\\
6001.67316679318	2.96725939340379e-05\\
6046.98055130657	2.91976023505072e-05\\
6092.62996695602	2.87314597987414e-05\\
6138.62399577752	2.82740039853757e-05\\
6184.9652392992	2.78250743609354e-05\\
6231.65631868842	2.7384512151924e-05\\
6278.6998749001	2.69521603906381e-05\\
6326.09856882603	2.65278639427653e-05\\
6373.85508144543	2.61114695328176e-05\\
6421.97211397655	2.57028257674594e-05\\
6470.45238802948	2.53017831567806e-05\\
6519.29864576009	2.49081941335743e-05\\
6568.51365002513	2.45219130706712e-05\\
6618.10018453851	2.41427962963861e-05\\
6668.06105402872	2.37707021081321e-05\\
6718.39908439753	2.34054907842548e-05\\
6769.11712287976	2.30470245941423e-05\\
6820.21803820442	2.26951678066624e-05\\
6871.70472075685	2.23497866969826e-05\\
6923.58008274232	2.20107495518219e-05\\
6975.84705835067	2.167792667319e-05\\
7028.50860392234	2.13511903806625e-05\\
7081.56769811553	2.10304150122445e-05\\
7135.02734207472	2.07154769238717e-05\\
7188.8905596004	2.04062544875995e-05\\
7243.16039732009	2.01026280885277e-05\\
7297.83992486074	1.98044801205094e-05\\
7352.93223502222	1.95116949806921e-05\\
7408.44044395243	1.92241590629356e-05\\
7464.36769132336	1.89417607501546e-05\\
7520.71714050886	1.86643904056293e-05\\
7577.49197876344	1.83919403633286e-05\\
7634.69541740261	1.81243049172891e-05\\
7692.33069198451	1.78613803100925e-05\\
7750.40106249289	1.76030647204829e-05\\
7808.90981352158	1.73492582501626e-05\\
7867.86025446014	1.70998629098108e-05\\
7927.25571968122	1.68547826043583e-05\\
7987.09956872899	1.66139231175605e-05\\
8047.39518650928	1.63771920959037e-05\\
8108.14598348099	1.6144499031881e-05\\
8169.35539584901	1.59157552466743e-05\\
8231.02688575857	1.56908738722741e-05\\
8293.16394149105	1.54697698330738e-05\\
8355.77007766133	1.52523598269685e-05\\
8418.84883541654	1.50385623059912e-05\\
8482.40378263641	1.48282974565164e-05\\
8546.43851413497	1.46214871790623e-05\\
8610.95665186405	1.44180550677188e-05\\
8675.96184511796	1.42179263892311e-05\\
8741.45777074004	1.40210280617651e-05\\
8807.44813333057	1.38272886333822e-05\\
8873.93666545632	1.36366382602474e-05\\
8940.9271278617	1.34490086845974e-05\\
9008.42330968139	1.32643332124922e-05\\
9076.42902865481	1.3082546691372e-05\\
9144.94813134188	1.29035854874441e-05\\
9213.98449334078	1.27273874629199e-05\\
9283.54201950699	1.25538919531239e-05\\
9353.62464417428	1.23830397434934e-05\\
9424.23633137717	1.22147730464908e-05\\
9495.38107507519	1.20490354784452e-05\\
9567.0628993788	1.18857720363416e-05\\
9639.28585877689	1.17249290745767e-05\\
9712.05403836631	1.15664542816961e-05\\
9785.37155408272	1.14102966571301e-05\\
9859.24255293356	1.12564064879435e-05\\
9933.67121323252	1.11047353256135e-05\\
10008.6617448359	1.09552359628513e-05\\
10084.2183893808	1.08078624104798e-05\\
10160.3454205248	1.0662569874381e-05\\
10237.047144188	1.05193147325258e-05\\
10314.3278987966	1.03780545120983e-05\\
10392.1920555277	1.02387478667258e-05\\
10470.6440185573	1.01013545538252e-05\\
10549.6882253089	9.96583541207715e-06\\
10629.3291467048	9.83215233903785e-06\\
10709.5712874188	9.70026826889681e-06\\
10790.419186131	9.5701471503916e-06\\
10871.8774157846	9.44175392488687e-06\\
10953.9505838446	9.31505450462634e-06\\
11036.6433325583	9.19001575116587e-06\\
11119.9603392177	9.06660545399484e-06\\
11203.9063164246	8.94479230935276e-06\\
11288.4860123565	8.82454589924806e-06\\
11373.7042110358	8.7058366706856e-06\\
11459.5657325999	8.58863591510834e-06\\
11546.075433574	8.47291574805943e-06\\
11633.2382071459	8.3586490890698e-06\\
11721.0589834426	8.24580964177607e-06\\
11809.5427298094	8.13437187427409e-06\\
11898.6944510906	8.02431099971212e-06\\
11988.5191899127	7.91560295712777e-06\\
12079.0220269697	7.80822439253277e-06\\
12170.2080813103	7.70215264024921e-06\\
12262.0825106276	7.59736570450057e-06\\
12354.6505115508	7.49384224126039e-06\\
12447.9173199389	7.39156154036184e-06\\
12541.8882111773	7.29050350787049e-06\\
12636.568500476	7.19064864872294e-06\\
12731.9635431699	7.09197804963341e-06\\
12828.0787350221	6.99447336227005e-06\\
12924.9195125292	6.89811678670297e-06\\
13022.4913532284	6.80289105512547e-06\\
13120.7997760075	6.70877941584996e-06\\
13219.8503414172	6.61576561757937e-06\\
13319.6486519853	6.52383389395542e-06\\
13420.200352534	6.43296894838432e-06\\
13521.511130499	6.34315593914067e-06\\
13623.5867162509	6.25438046474997e-06\\
13726.4328834197	6.16662854964993e-06\\
13830.0554492214	6.07988663013074e-06\\
13934.4602747868	5.99414154055447e-06\\
14039.6532654931	5.90938049985317e-06\\
14145.6403712978	5.82559109830538e-06\\
14252.4275870757	5.74276128459064e-06\\
14360.0209529573	5.66087935312132e-06\\
14468.4265546712	5.57993393165099e-06\\
14577.6505238876	5.49991396915859e-06\\
14687.6990385654	5.42080872400704e-06\\
14798.5783233021	5.34260775237535e-06\\
14910.2946496851	5.26530089696293e-06\\
15022.8543366469	5.18887827596473e-06\\
15136.2637508225	5.11333027231539e-06\\
15250.5293069092	5.03864752320105e-06\\
15365.65746803	4.96482090983685e-06\\
15481.6547460986	4.89184154750846e-06\\
15598.527702188	4.81970077587535e-06\\
15716.2829469014	4.74839014953397e-06\\
15834.9271407466	4.67790142883832e-06\\
15954.4669945122	4.60822657097603e-06\\
16074.9092696475	4.53935772129721e-06\\
16196.2607786446	4.47128720489363e-06\\
16318.5283854242	4.40400751842576e-06\\
16441.7190057235	4.33751132219482e-06\\
16565.8396074876	4.27179143245726e-06\\
16690.8972112632	4.20684081397876e-06\\
16816.8988905963	4.14265257282462e-06\\
16943.8517724318	4.07921994938385e-06\\
17071.7630375169	4.01653631162361e-06\\
17200.639920807	3.95459514857115e-06\\
17330.4897118753	3.8933900640196e-06\\
17461.3197553249	3.83291477045472e-06\\
17593.1374512039	3.7731630831991e-06\\
17725.9502554248	3.71412891477028e-06\\
17859.7656801852	3.65580626944966e-06\\
17994.5912943937	3.59818923805833e-06\\
18130.4347240971	3.54127199293636e-06\\
18267.3036529127	3.48504878312214e-06\\
18405.2058224618	3.42951392972788e-06\\
18544.1490328086	3.37466182150762e-06\\
18684.1411429008	3.32048691061409e-06\\
18825.1900710143	3.26698370854065e-06\\
18967.3037952012	3.21414678224447e-06\\
19110.4903537406	3.16197075044717e-06\\
19254.7578455939	3.11045028010905e-06\\
19400.1144308624	3.05958008307307e-06\\
19546.5683312491	3.00935491287474e-06\\
19694.1278305236	2.959769561714e-06\\
19842.8012749908	2.91081885758515e-06\\
19992.5970739628	2.86249766156099e-06\\
20143.5237002349	2.81480086522725e-06\\
20295.5896905642	2.76772338826343e-06\\
20448.8036461532	2.72126017616605e-06\\
20603.1742331357	2.67540619811057e-06\\
20758.7101830675	2.63015644494786e-06\\
20915.4202934198	2.58550592733168e-06\\
21073.313428077	2.54144967397301e-06\\
21232.3985178383	2.49798273001754e-06\\
21392.6845609224	2.45510015554244e-06\\
21554.1806234768	2.41279702416868e-06\\
21716.8958400905	2.37106842178515e-06\\
21880.8394143105	2.32990944538069e-06\\
22046.0206191627	2.28931520198051e-06\\
22212.4487976761	2.24928080768322e-06\\
22380.1333634117	2.20980138679491e-06\\
22549.0838009943	2.17087207105661e-06\\
22719.3096666494	2.13248799896168e-06\\
22890.8205887437	2.09464431515946e-06\\
23063.6262683299	2.05733616994193e-06\\
23237.7364796946	2.02055871880986e-06\\
23413.1610709123	1.98430712211498e-06\\
23589.9099644014	1.94857654477513e-06\\
23767.9931574861	1.91336215605884e-06\\
23947.4207229617	1.87865912943632e-06\\
24128.2028096642	1.84446264249375e-06\\
24310.3496430445	1.81076787690757e-06\\
24493.8715257466	1.77757001847595e-06\\
24678.7788381905	1.74486425720439e-06\\
24865.0820391595	1.71264578744254e-06\\
25052.7916663912	1.68090980806943e-06\\
25241.9183371742	1.64965152272431e-06\\
25432.4727489482	1.61886614008032e-06\\
25624.4656799093	1.58854887415849e-06\\
25817.9079896195	1.55869494467941e-06\\
26012.8106196209	1.52929957744995e-06\\
26209.1845940549	1.50035800478274e-06\\
26407.0410202855	1.47186546594595e-06\\
26606.3910895274	1.44381720764107e-06\\
26807.2460774792	1.41620848450635e-06\\
27009.6173449613	1.38903455964394e-06\\
27213.5163385584	1.36229070516831e-06\\
27418.9545912668	1.33597220277418e-06\\
27625.9437231468	1.31007434432181e-06\\
27834.49544198	1.28459243243779e-06\\
28044.6215439317	1.25952178112944e-06\\
28256.3339142177	1.23485771641113e-06\\
28469.6445277769	1.21059557694076e-06\\
28684.5654499485	1.18673071466462e-06\\
28901.1088371544	1.16325849546924e-06\\
29119.286937587	1.1401742998386e-06\\
29339.1120919019	1.1174735235153e-06\\
29560.5967339154	1.09515157816412e-06\\
29783.7533913087	1.07320389203687e-06\\
30008.594686336	1.05162591063709e-06\\
30235.1333365383	1.03041309738343e-06\\
30463.382155463	1.00956093427053e-06\\
30693.3540533885	9.89064922526333e-07\\
30925.0620380546	9.68920583264743e-07\\
31158.5192153983	9.49123458132635e-07\\
31393.7387902947	9.29669109950327e-07\\
31630.7340673043	9.10553123344559e-07\\
31869.5184514253	8.91771105373191e-07\\
32110.105448852	8.73318686140799e-07\\
32352.5086677388	8.55191519404484e-07\\
32596.7418189694	8.37385283169165e-07\\
32842.818716933	8.19895680271709e-07\\
33090.7532803053	8.0271843895332e-07\\
33340.5595328357	7.85849313419636e-07\\
33592.2516041407	7.6928408438804e-07\\
33845.8437305032	7.53018559621653e-07\\
34101.3502556773	7.37048574449657e-07\\
34358.7856317002	7.21369992273493e-07\\
34618.1644197092	7.05978705058644e-07\\
34879.5012907654	6.9087063381165e-07\\
35142.8110266837	6.76041729042069e-07\\
35408.1085208687	6.61487971209134e-07\\
35675.4087791572	6.47205371152942e-07\\
35944.726920667	6.33189970509939e-07\\
36216.078178652	6.19437842112541e-07\\
36489.4779013636	6.05945090372819e-07\\
36764.9415529195	5.92707851650085e-07\\
37042.4847141778	5.79722294602371e-07\\
37322.1230836184	5.66984620521713e-07\\
37603.8724782308	5.54491063653223e-07\\
37887.7488344093	5.42237891497948e-07\\
38173.768208854	5.30221405099567e-07\\
38461.9467794789	5.18437939314934e-07\\
38752.3008463271	5.06883863068541e-07\\
39044.8468324927	4.95555579590983e-07\\
39339.60128505	4.84449526641518e-07\\
39636.5808759893	4.73562176714855e-07\\
39935.8024031596	4.62890037232279e-07\\
40237.2827912191	4.52429650717268e-07\\
40541.0390925923	4.42177594955747e-07\\
40847.0884884347	4.32130483141175e-07\\
41155.4482896045	4.22284964004642e-07\\
41466.1359376416	4.12637721930148e-07\\
41779.1690057544	4.0318547705528e-07\\
42094.5651998136	3.93924985357543e-07\\
42412.3423593538	3.84853038726481e-07\\
42732.5184585825	3.75966465021917e-07\\
43055.1116073965	3.67262128118475e-07\\
43380.1400524069	3.58736927936665e-07\\
43707.6221779705	3.50387800460785e-07\\
44037.57650723	3.42211717743883e-07\\
44370.0217031614	3.34205687900046e-07\\
44704.9765696303	3.26366755084276e-07\\
45042.4600524546	3.18691999460253e-07\\
45382.4912404769	3.11178537156201e-07\\
45725.0893666436	3.03823520209201e-07\\
46070.2738090931	2.96624136498176e-07\\
46418.0640922519	2.89577609665845e-07\\
46768.479887939	2.82681199029944e-07\\
47121.541016478	2.75932199483975e-07\\
47477.2674478187	2.69327941387776e-07\\
47835.679302667	2.62865790448182e-07\\
48196.7968536222	2.56543147590087e-07\\
48560.6405263239	2.50357448818171e-07\\
48927.2309006076	2.44306165069562e-07\\
49296.5887116685	2.38386802057749e-07\\
49668.7348512343	2.32596900107988e-07\\
50043.6903687473	2.26934033984507e-07\\
50421.4764725545	2.21395812709775e-07\\
50802.114531107	2.15979879376105e-07\\
51185.6260741696	2.10683910949866e-07\\
51572.0327940378	2.05505618068582e-07\\
51961.3565467648	2.00442744831176e-07\\
52353.6193533982	1.95493068581621e-07\\
52748.8434012247	1.90654399686266e-07\\
53147.0510450263	1.85924581305091e-07\\
53548.2648083438	1.81301489157144e-07\\
53952.5073847507	1.76783031280427e-07\\
54359.8016391376	1.72367147786447e-07\\
54770.1706090047	1.68051810609706e-07\\
55183.6375057655	1.63835023252363e-07\\
55600.2257160592	1.59714820524295e-07\\
56019.9588030736	1.55689268278805e-07\\
56442.8605078779	1.51756463144191e-07\\
56868.9547507659	1.4791453225141e-07\\
57298.2656326087	1.44161632958068e-07\\
57730.8174362176	1.40495952568932e-07\\
58166.634627718	1.36915708053192e-07\\
58605.7418579333	1.33419145758677e-07\\
59048.163963779	1.30004541123237e-07\\
59493.9259696676	1.26670198383484e-07\\
59943.0530889239	1.23414450281091e-07\\
60395.5707252113	1.20235657766843e-07\\
60851.5044739687	1.17132209702629e-07\\
61310.8801238584	1.14102522561559e-07\\
61773.7236582241	1.1114504012639e-07\\
62240.0612565613	1.08258233186426e-07\\
62709.9192959978	1.05440599233075e-07\\
63183.3243527857	1.02690662154229e-07\\
63660.3032038044	1.00006971927632e-07\\
64140.8828280752	9.73881043133863e-08\\
64625.090408288	9.48326605457629e-08\\
65112.9533323374	9.233926702447e-08\\
65604.4991948736	8.99065750055129e-08\\
66099.7557988615	8.75332602918088e-08\\
66598.7511571543	8.52180229236782e-08\\
67101.5134940779	8.29595868693595e-08\\
67608.0712470273	8.07566997156777e-08\\
68118.4530680746	7.86081323589948e-08\\
68632.68782559	7.65126786965631e-08\\
69150.8046058751	7.44691553184048e-08\\
69672.8327148071	7.2476401199842e-08\\
70198.8016794974	7.05332773947741e-08\\
70728.7412499609	6.86386667298321e-08\\
71262.6814007993	6.67914734994994e-08\\
71800.6523328963	6.49906231623116e-08\\
72342.684475126	6.32350620382327e-08\\
72888.8084860737	6.15237570073029e-08\\
73439.0552557701	5.98556952096491e-08\\
73993.4559074389	5.82298837469488e-08\\
74552.041799257	5.66453493854347e-08\\
75114.8445261277	5.51011382605215e-08\\
75681.8959214685	5.3596315583132e-08\\
76253.2280590112	5.21299653478026e-08\\
76828.8732546163	5.07011900426386e-08\\
77408.8640681011	4.93091103611919e-08\\
77993.2333050806	4.79528649163279e-08\\
78582.0140188235	4.66316099561428e-08\\
79175.239512122	4.5344519081996e-08\\
79772.9433391755	4.40907829687157e-08\\
80375.1593074877	4.28696090870335e-08\\
80981.9214797798	4.1680221428299e-08\\
81593.2641759163	4.05218602315267e-08\\
82209.2219748472	3.93937817128225e-08\\
82829.8297165634	3.82952577972351e-08\\
83455.1225040667	3.72255758530751e-08\\
84085.1357053561	3.61840384287402e-08\\
84719.9049554282	3.51699629920851e-08\\
85359.4661582928	3.41826816723733e-08\\
86003.8554890032	3.32215410048413e-08\\
86653.109395703	3.22859016779059e-08\\
87307.2646016869	3.13751382830449e-08\\
87966.3581074791	3.04886390673742e-08\\
88630.4271929253	2.96258056889503e-08\\
89299.509419301	2.87860529748156e-08\\
89973.6426314364	2.7968808681808e-08\\
90652.8649598577	2.71735132601536e-08\\
91337.214822943	2.63996196198576e-08\\
92026.7309290954	2.56465928999089e-08\\
92721.4522789325	2.49139102403078e-08\\
93421.4181674926	2.4201060556931e-08\\
94126.6681864573	2.3507544319239e-08\\
94837.2422263909	2.28328733308367e-08\\
95553.1804789962	2.21765705128898e-08\\
96274.5234393881	2.15381696904016e-08\\
97001.3119083847	2.09172153813523e-08\\
97733.5869948146	2.03132625887024e-08\\
98471.3901178417	1.97258765952592e-08\\
99214.7630093084	1.91546327614024e-08\\
99963.7477160966	1.85991163256673e-08\\
100718.386602504	1.80589222081816e-08\\
101478.722352644	1.75336548169451e-08\\
102244.797972856	1.70229278569511e-08\\
103016.65679414	1.65263641421356e-08\\
103794.342474607	1.60435954101464e-08\\
104577.899001948	1.55742621399231e-08\\
105367.370695925	1.51180133720726e-08\\
106162.802210873	1.46745065320295e-08\\
106964.23853823	1.42434072559854e-08\\
107771.725009077	1.38243892195753e-08\\
108585.307296709	1.34171339693004e-08\\
109405.031419212	1.30213307566758e-08\\
110230.943742068	1.26366763750805e-08\\
111063.09098078	1.22628749992953e-08\\
111901.52020351	1.18996380277062e-08\\
112746.278833744	1.15466839271558e-08\\
113597.414652975	1.12037380804204e-08\\
114454.975803403	1.08705326362918e-08\\
115319.010790661	1.05468063622421e-08\\
116189.568486556	1.02323044996495e-08\\
117066.698131834	9.92677862156034e-09\\
117950.449338968	9.62998649296519e-09\\
118840.872094958	9.34169193356387e-09\\
119738.016764165	9.06166468299568e-09\\
120641.934091157	8.78968026850912e-09\\
121552.675203577	8.52551987504551e-09\\
122470.291615039	8.26897021771077e-09\\
123394.835228039	8.01982341660966e-09\\
124326.358336891	7.77787687401534e-09\\
125264.913630686	7.54293315384728e-09\\
126210.554196271	7.31479986343103e-09\\
127163.333521252	7.09328953751173e-09\\
128123.305497023	6.87821952449457e-09\\
129090.524421805	6.6694118748839e-09\\
130065.045003728	6.46669323189296e-09\\
131046.922363918	6.26989472419622e-09\\
132036.212039619	6.07885186079658e-09\\
133032.969987331	5.89340442797837e-09\\
134037.252585978	5.71339638831799e-09\\
135049.116640094	5.53867578172383e-09\\
136068.619383039	5.36909462847664e-09\\
137095.818480235	5.20450883424216e-09\\
138130.772032427	5.04477809702721e-09\\
139173.538578968	4.88976581605049e-09\\
140224.177101134	4.73933900249975e-09\\
141282.747025459	4.593368192147e-09\\
142349.308227094	4.45172735979311e-09\\
143423.921033194	4.31429383551321e-09\\
144506.646226332	4.18094822267516e-09\\
145597.545047938	4.05157431770228e-09\\
146696.679201759	3.92605903155295e-09\\
147804.110857352	3.80429231288865e-09\\
148919.902653601	3.68616707290278e-09\\
150044.117702257	3.57157911178262e-09\\
151176.819591511	3.46042704677715e-09\\
152318.072389588	3.35261224184323e-09\\
153467.940648372	3.24803873884304e-09\\
154626.48940706	3.14661319026581e-09\\
155793.784195833	3.04824479344763e-09\\
156969.891039574	2.95284522626193e-09\\
158154.87646159	2.86032858425539e-09\\
159348.807487385	2.77061131920239e-09\\
160551.751648444	2.68361217905267e-09\\
161763.776986059	2.59925214924661e-09\\
162984.952055171	2.51745439537262e-09\\
164215.345928253	2.43814420714176e-09\\
165455.028199213	2.36124894365469e-09\\
166704.068987334	2.28669797993665e-09\\
167962.538941238	2.21442265471582e-09\\
169230.50924288	2.14435621942165e-09\\
170508.051611579	2.07643378837885e-09\\
171795.238308072	2.01059229017416e-09\\
173092.142138601	1.94677042017245e-09\\
174398.83645903	1.88490859415962e-09\\
175715.395178998	1.82494890308936e-09\\
177041.892766095	1.76683506891195e-09\\
178378.404250078	1.71051240146294e-09\\
179725.005227113	1.6559277563902e-09\\
181081.771864049	1.60302949409777e-09\\
182448.780902729	1.55176743968585e-09\\
183826.109664331	1.50209284386581e-09\\
185213.83605374	1.45395834483017e-09\\
186612.038563953	1.40731793105711e-09\\
188020.796280522	1.36212690502996e-09\\
189440.188886025	1.31834184785192e-09\\
190870.296664576	1.27592058473703e-09\\
192311.200506361	1.23482215135842e-09\\
193762.981912217	1.19500676103513e-09\\
195225.722998241	1.15643577273936e-09\\
196699.506500436	1.11907165990609e-09\\
198184.415779389	1.08287798002745e-09\\
199680.534824985	1.04781934501434e-09\\
201187.94826116	1.01386139230829e-09\\
202706.741350687	9.80970756726708e-10\\
204237	9.49115043025144e-10\\
};
\addlegendentry{Adatt. BLN appr. (\textit{ML})}


\addplot[area legend, draw=none, fill=mycolor1, fill opacity=0.15, forget plot]
table[row sep=crcr] {%
x	y\\
111.473131801004	2.90708557314055e-06\\
112.314656473384	2.9595550548364e-06\\
113.162533921208	3.0130641993651e-06\\
114.016812102373	3.06763796699213e-06\\
114.877539336821	3.12330201202353e-06\\
115.744764309265	3.18008270153571e-06\\
116.618536071944	3.23800713447547e-06\\
117.4989040474	3.29710316112935e-06\\
118.385918031273	3.35739940296009e-06\\
119.279628195113	3.41892527280786e-06\\
120.180085089226	3.48171099545286e-06\\
121.087339645525	3.54578762853551e-06\\
122.001443180416	3.61118708382926e-06\\
122.922447397698	3.67794214886074e-06\\
123.850404391489	3.74608650887069e-06\\
124.785366649173	3.81565476910863e-06\\
125.727387054366	3.886682477453e-06\\
126.676518889911	3.9592061473479e-06\\
127.632815840887	4.03326328104632e-06\\
128.596331997653	4.10889239314892e-06\\
129.567121858898	4.18613303442653e-06\\
130.545240334732	4.26502581591314e-06\\
131.530742749786	4.34561243325551e-06\\
132.523684846345	4.42793569130429e-06\\
133.524122787498	4.51203952893035e-06\\
134.53211316032	4.5979690440492e-06\\
135.547712979064	4.68577051883491e-06\\
136.570979688393	4.77549144510445e-06\\
137.601971166627	4.86718054985139e-06\\
138.640745729018	4.9608878209078e-06\\
139.687362131045	5.05666453271134e-06\\
140.74187957174	5.15456327215389e-06\\
141.804357697036	5.25463796448695e-06\\
142.874856603141	5.35694389925793e-06\\
143.953436839938	5.46153775625075e-06\\
145.040159414406	5.56847763140285e-06\\
146.135085794075	5.67782306267034e-06\\
147.2382779105	5.78963505581166e-06\\
148.349798162768	5.9039761100598e-06\\
149.469709421021	6.0209102436522e-06\\
150.598075030017	6.14050301918678e-06\\
151.734958812712	6.26282156877224e-06\\
152.880425073868	6.38793461893998e-06\\
154.034538603695	6.51591251528476e-06\\
155.197364681509	6.6468272468009e-06\\
156.368969079428	6.78075246988032e-06\\
157.549418066095	6.91776353193928e-06\\
158.73877841042	7.05793749463935e-06\\
159.937117385361	7.20135315666956e-06\\
161.144502771729	7.34809107605568e-06\\
162.36100286202	7.49823359196339e-06\\
163.586686464278	7.65186484596194e-06\\
164.821622905988	7.80907080271547e-06\\
166.065882037999	7.96993927006938e-06\\
167.319534238469	8.13455991849973e-06\\
168.582650416852	8.30302429989363e-06\\
169.855302017906	8.47542586563005e-06\\
171.137561025735	8.65185998392981e-06\\
172.429499967858	8.83242395644501e-06\\
173.731191919316	9.01721703405825e-06\\
175.042710506801	9.20634043186269e-06\\
176.364129912823	9.39989734329451e-06\\
177.695524879903	9.5979929533898e-06\\
179.036970714807	9.80073445113839e-06\\
180.388543292795	1.00082310409075e-05\\
181.750319061926	1.0220593952909e-05\\
183.122375047369	1.04379364526827e-05\\
184.504788855768	1.06603738495712e-05\\
185.897638679631	1.08880235041581e-05\\
187.301003301749	1.11210048346458e-05\\
188.714962099654	1.13594393221447e-05\\
190.13959505011	1.16034505148485e-05\\
191.574982733635	1.18531640310689e-05\\
193.02120633906	1.21087075611028e-05\\
194.478347668119	1.23702108679036e-05\\
195.946489140081	1.26378057865296e-05\\
197.425713796405	1.29116262223399e-05\\
198.916105305443	1.31918081479077e-05\\
200.417747967167	1.34784895986222e-05\\
201.930726717942	1.37718106669446e-05\\
203.455127135329	1.40719134952883e-05\\
204.991035442923	1.43789422674874e-05\\
206.538538515233	1.46930431988198e-05\\
208.097723882594	1.50143645245476e-05\\
209.668679736118	1.5343056486938e-05\\
211.251494932683	1.56792713207248e-05\\
212.846258999961	1.60231632369718e-05\\
214.453062141476	1.63748884052948e-05\\
216.071995241712	1.67346049344011e-05\\
217.70314987125	1.71024728509013e-05\\
219.346618291952	1.74786540763489e-05\\
221.002493462171	1.78633124024606e-05\\
222.67086904202	1.82566134644711e-05\\
224.35183939866	1.86587247125726e-05\\
226.045499611642	1.9069815381391e-05\\
227.751945478287	1.94900564574474e-05\\
229.471273519099	1.9919620644556e-05\\
231.20358098323	2.03586823271049e-05\\
232.948965853976	2.08074175311704e-05\\
234.707526854322	2.12660038834107e-05\\
236.479363452525	2.17346205676886e-05\\
238.264575867741	2.22134482793703e-05\\
240.063265075692	2.27026691772475e-05\\
241.875532814379	2.32024668330341e-05\\
243.701481589837	2.37130261783811e-05\\
245.54121468193	2.42345334493653e-05\\
247.394836150195	2.47671761283968e-05\\
249.262450839729	2.53111428834986e-05\\
251.144164387117	2.58666235049111e-05\\
253.040083226407	2.64338088389726e-05\\
254.950314595131	2.70128907192323e-05\\
256.874966540373	2.76040618947516e-05\\
258.814147924875	2.8207515955551e-05\\
260.767968433199	2.88234472551623e-05\\
262.73653857793	2.94520508302491e-05\\
264.719969705925	3.00935223172569e-05\\
266.718374004613	3.0748057866061e-05\\
268.731864508341	3.14158540505789e-05\\
270.760555104764	3.20971077763184e-05\\
272.804560541293	3.27920161848352e-05\\
274.863996431577	3.35007765550753e-05\\
276.938979262052	3.42235862015822e-05\\
279.02962639852	3.49606423695484e-05\\
281.136056092795	3.57121421266999e-05\\
283.258387489388	3.64782822519982e-05\\
285.396740632247	3.72592591211536e-05\\
287.551236471548	3.80552685889431e-05\\
289.721996870532	3.88665058683326e-05\\
291.909144612403	3.96931654064039e-05\\
294.112803407272	4.05354407570938e-05\\
296.33309789915	4.13935244507529e-05\\
298.570153673005	4.22676078605384e-05\\
300.824097261858	4.31578810656588e-05\\
303.095056153947	4.40645327114892e-05\\
305.383158799932	4.49877498665861e-05\\
307.688534620164	4.59277178766288e-05\\
310.011314012007	4.68846202153215e-05\\
312.351628357206	4.78586383322956e-05\\
314.709610029328	4.88499514980541e-05\\
317.085392401243	4.98587366460042e-05\\
319.479109852668	5.08851682116321e-05\\
321.890897777773	5.19294179688734e-05\\
324.32089259283	5.29916548637427e-05\\
326.769231743941	5.40720448452865e-05\\
329.236053714802	5.51707506939296e-05\\
331.72149803454	5.62879318472907e-05\\
334.225705285606	5.74237442235478e-05\\
336.748817111725	5.8578340042436e-05\\
339.290976225908	5.97518676439706e-05\\
341.852326418525	6.09444713049875e-05\\
344.433012565439	6.21562910536012e-05\\
347.033180636195	6.33874624816855e-05\\
349.652977702284	6.46381165554866e-05\\
352.292551945457	6.59083794244815e-05\\
354.952052666108	6.71983722286026e-05\\
357.631630291718	6.85082109039524e-05\\
360.331436385364	6.98380059871373e-05\\
363.051623654292	7.11878624183555e-05\\
365.792345958554	7.25578793433773e-05\\
368.55375831971	7.39481499145621e-05\\
371.336016929598	7.5358761091061e-05\\
374.13927915917	7.67897934383607e-05\\
376.963703567387	7.82413209273206e-05\\
379.809449910193	7.97134107328756e-05\\
382.676679149551	8.12061230325647e-05\\
385.565553462546	8.27195108050601e-05\\
388.476236250556	8.42536196288752e-05\\
391.408892148499	8.58084874814301e-05\\
394.363687034141	8.73841445386617e-05\\
397.340788037481	8.89806129753659e-05\\
400.340363550203	9.05979067664701e-05\\
403.3625832352	9.22360314894264e-05\\
406.407618036173	9.3894984127935e-05\\
409.475640187299	9.55747528771971e-05\\
412.56682322297	9.7275316950909e-05\\
415.681341987614	9.89966463902082e-05\\
418.819372645578	0.000100738701874789\\
421.981092691098	0.000102501434536402\\
425.166680958337	0.000104284785774969\\
428.376317631498	0.000106088687077518\\
431.610184255017	0.000107913059840192\\
434.868463743833	0.000109757815193536\\
438.151340393733	0.000111622853831312\\
441.458999891773	0.00011350806584306\\
444.791629326788	0.000115413330550655\\
448.149417199969	0.000117338516349085\\
451.532553435523	0.000119283480551694\\
454.941229391423	0.000121248069240127\\
458.375637870226	0.000123232117119218\\
461.835973129977	0.00012523544737706\\
465.322430895204	0.000127257871550497\\
468.835208367981	0.000129299189396279\\
472.374504239085	0.00013135918876812\\
475.940518699236	0.000133437645499906\\
479.533453450418	0.00013553432329527\\
483.15351171729	0.000137648973623802\\
486.800898258679	0.000139781335624101\\
490.47581937916	0.000141931136013921\\
494.178482940729	0.000144098089007641\\
497.90909837456	0.000146281896241278\\
501.667876692845	0.000148482246705279\\
505.455030500736	0.000150698816685316\\
509.27077400837	0.000152931269711294\\
513.115323042978	0.000155179256514799\\
516.988895061104	0.000157442414995193\\
520.891709160893	0.000159720370194555\\
524.823986094494	0.000162012734281692\\
528.78594828054	0.000164319106545386\\
532.777819816728	0.000166639073397099\\
536.799826492501	0.000168972208383296\\
540.85219580181	0.000171318072207577\\
544.935156955991	0.000173676212762794\\
549.048940896723	0.0001760461651733\\
553.193780309092	0.000178427451847504\\
557.369909634754	0.000180819582540874\\
561.577565085194	0.000183222054429526\\
565.816984655088	0.000185634352194535\\
570.088408135764	0.000188055948117093\\
574.392077128762	0.000190486302184615\\
578.728235059506	0.000192924862207916\\
583.097127191066	0.000195371063949531\\
587.499000638037	0.000197824331263287\\
591.93410438051	0.000200284076245165\\
596.402689278159	0.00020274969939555\\
600.905008084431	0.000205220589792893\\
605.441315460838	0.000207696125278831\\
610.011867991365	0.000210175672654795\\
614.61692419698	0.000212658587890113\\
619.256744550263	0.000215144216341617\\
623.931591490128	0.000217631892984731\\
628.641729436677	0.00022012094265603\\
633.387424806152	0.000222610680307219\\
638.168946026005	0.000225100411270488\\
642.986563550079	0.000227589431535175\\
647.840549873909	0.00023007702803566\\
652.731179550132	0.000232562478950397\\
657.658729204016	0.000235045054011973\\
662.623477549112	0.000237524014828084\\
667.625705403009	0.00023999861521327\\
672.665695703228	0.000242468101531295\\
677.743733523219	0.000244931713047968\\
682.860106088486	0.000247388682294253\\
688.015102792837	0.000249838235439475\\
693.209015214748	0.00025227959267439\\
698.442137133859	0.000254711968603935\\
703.714764547589	0.000257134572649383\\
709.02719568788	0.00025954660945969\\
714.379731038059	0.000261947279331736\\
719.772673349847	0.000264335778639215\\
725.20632766047	0.000266711300269851\\
730.681001309921	0.000269073034070668\\
736.197003958341	0.000271420167300968\\
741.754647603534	0.000273751885092703\\
747.354246598613	0.000276067370917897\\
752.996117669786	0.000278365807062753\\
758.680579934264	0.000280646375108093\\
764.407954918312	0.00028290825641574\\
770.178566575443	0.000285150632620462\\
775.99274130473	0.000287372686127064\\
781.850807969277	0.000289573600612244\\
787.753097914816	0.000291752561530759\\
793.699944988449	0.000293908756625508\\
799.691685557532	0.000296041376441059\\
805.728658528701	0.000298149614840209\\
811.81120536704	0.000300232669523091\\
817.939670115397	0.000302289742548402\\
824.114399413841	0.00030432004085625\\
830.33574251927	0.000306322776792177\\
836.604051325169	0.000308297168631868\\
842.919680381509	0.000310242441106062\\
849.282986914804	0.000312157825925186\\
855.694330848315	0.000314042562303224\\
862.154074822409	0.000315895897480323\\
868.662584215073	0.000317717087243647\\
875.220227162575	0.000319505396445993\\
881.827374580292	0.000321260099521658\\
888.484400183686	0.000322980480999083\\
895.191680509447	0.000324665836009761\\
901.949594936784	0.000326315470792939\\
908.758525708891	0.00032792870319561\\
915.61885795456	0.000329504863167309\\
922.530979709972	0.000331043293249253\\
929.495281940641	0.000332543349057313\\
936.512158563526	0.000334004399758386\\
943.582006469321	0.00033542582853966\\
950.705225544888	0.000336807033070352\\
957.882218695896	0.00033814742595544\\
965.113391869591	0.000339446435180953\\
972.39915407777	0.000340703504550391\\
979.73991741991	0.000341918094111837\\
987.13609710648	0.000343089680575356\\
994.588111482424	0.00034421775772026\\
1002.09638205082	0.000345301836791864\\
1009.66133349674	0.000346341446887337\\
1017.28339371124	0.000347336135330274\\
1024.96299381559	0.000348285468033643\\
1032.70056818564	0.000349189029850757\\
1040.49655447641	0.000350046424913937\\
1048.35139364682	0.000350857276960555\\
1056.26552998465	0.000351621229646153\\
1064.23941113167	0.000352337946844354\\
1072.27348810894	0.000353007112933303\\
1080.36821534236	0.000353628433068373\\
1088.5240506883	0.000354201633440907\\
1096.74145545961	0.000354726461522793\\
1105.02089445159	0.000355202686296647\\
1113.36283596838	0.000355630098471453\\
1121.76775184938	0.000356008510683479\\
1130.23611749598	0.000356337757682353\\
1138.76841189842	0.000356617696502155\\
1147.36511766291	0.000356848206617461\\
1156.02672103892	0.000357029190084238\\
1164.75371194667	0.000357160571665561\\
1173.54658400484	0.000357242298942122\\
1182.40583455852	0.000357274342407528\\
1191.3319647073	0.000357256695548419\\
1200.32547933364	0.000357189374909466\\
1209.38688713144	0.000357072420143323\\
1218.51670063476	0.000356905894045654\\
1227.71543624688	0.000356689882575373\\
1236.98361426944	0.000356424494860282\\
1246.32175893192	0.000356109863188314\\
1255.73039842129	0.000355746142984619\\
1265.21006491183	0.000355333512774799\\
1274.7612945953	0.000354872174134584\\
1284.38462771124	0.000354362351626346\\
1294.0806085775	0.000353804292722827\\
1303.84978562109	0.000353198267718551\\
1313.69271140914	0.000352544569629414\\
1323.60994268017	0.000351843514080986\\
1333.60204037562	0.000351095439186139\\
1343.66956967153	0.000350300705412632\\
1353.81310001052	0.000349459695441366\\
1364.03320513401	0.000348572814016043\\
1374.33046311466	0.000347640487785064\\
1384.70545638909	0.000346663165136496\\
1395.1587717908	0.00034564131602704\\
1405.69100058334	0.000344575431805942\\
1416.30273849384	0.000343466025034849\\
1426.99458574659	0.000342313629304642\\
1437.7671470971	0.000341118799050288\\
1448.62103186621	0.000339882109364762\\
1459.55685397464	0.000338604155813081\\
1470.57523197764	0.000337285554247411\\
1481.67678910003	0.000335926940624148\\
1492.86215327143	0.000334528970823739\\
1504.13195716179	0.00033309232047375\\
1515.48683821714	0.000331617684775486\\
1526.92743869569	0.000330105778334\\
1538.45440570412	0.000328557334990871\\
1550.06839123423	0.000326973107658415\\
1561.77005219976	0.00032535386815318\\
1573.56005047358	0.000323700407025434\\
1585.43905292513	0.000322013533380019\\
1597.40773145812	0.000320294074682278\\
1609.46676304856	0.000318542876540653\\
1621.61682978302	0.000316760802455137\\
1633.85861889724	0.000314948733517806\\
1646.19282281499	0.000313107568048257\\
1658.62013918722	0.000311238221142872\\
1671.14127093155	0.000309341624112532\\
1683.756926272	0.0003074187237787\\
1696.46781877908	0.000305470481593006\\
1709.2746674101	0.000303497872540823\\
1722.17819654992	0.000301501883785233\\
1735.17913605182	0.000299483513005003\\
1748.27822127887	0.000297443766379303\\
1761.47619314548	0.000295383656174123\\
1774.77379815929	0.000293304197891516\\
1788.17178846346	0.000291206406954339\\
1801.67092187916	0.000289091294916846\\
1815.27196194842	0.000286959865216187\\
1828.97567797739	0.000284813108511482\\
1842.78284507977	0.00028265199769486\\
1856.6942442207	0.000280477482700427\\
1870.71066226095	0.00027829048527904\\
1884.83289200137	0.000276091893944271\\
1899.06173222778	0.000273882559322143\\
1913.39798775612	0.00027166329014857\\
1927.84246947799	0.000269434850149059\\
1942.39599440654	0.000267197956002546\\
1957.05938572261	0.000264953276536063\\
1971.83347282137	0.000262701433223212\\
1986.71909135918	0.000260443001974988\\
2001.71708330089	0.000258178516125742\\
2016.82829696743	0.000255908470440459\\
2032.05358708382	0.000253633325910533\\
2047.39381482752	0.000251353515069966\\
2062.8498478771	0.000249069447554223\\
2078.42256046136	0.000246781515638313\\
2094.11283340877	0.000244490099524275\\
2109.92155419728	0.000242195572194647\\
2125.8496170045	0.00023989830370092\\
2141.89792275834	0.000237598664808136\\
2158.06737918789	0.000235297029964064\\
2174.35890087482	0.000232993779600592\\
2190.77340930509	0.000230689301804968\\
2207.31183292107	0.000228383993418999\\
2223.97510717407	0.000226078260636463\\
2240.76417457722	0.000223772519174006\\
2257.67998475881	0.000221467194090531\\
2274.72349451599	0.000219162719325934\\
2291.89566786891	0.000216859537023542\\
2309.19747611519	0.000214558096692852\\
2326.62989788494	0.000212258854260903\\
2344.19391919604	0.000209962271052656\\
2361.89053350997	0.000207668812733218\\
2379.72074178795	0.000205378948238114\\
2397.6855525476	0.000203093148711952\\
2415.78598191995	0.000200811886470899\\
2434.02305370695	0.000198535634000287\\
2452.39779943935	0.000196264862995312\\
2470.91125843504	0.000194000043450101\\
2489.56447785789	0.000191741642798323\\
2508.3585127769	0.000189490125106881\\
2527.29442622594	0.000187245950322974\\
2546.37328926386	0.000185009573573905\\
2565.59618103505	0.000182781444518364\\
2584.96418883051	0.000180562006747419\\
2604.47840814933	0.00017835169723319\\
2624.13994276068	0.000176150945822958\\
2643.94990476618	0.000173960174776409\\
2663.90941466289	0.000171779798343664\\
2684.01960140661	0.000169610222381788\\
2704.28160247578	0.000167451844007529\\
2724.69656393583	0.000165305051284125\\
2745.26564050395	0.000163170222940121\\
2765.98999561444	0.000161047728118242\\
2786.87080148452	0.000158937926152484\\
2807.90923918064	0.000156841166371694\\
2829.10649868523	0.000154757787928038\\
2850.46377896409	0.000152688119648825\\
2871.98228803412	0.000150632479910305\\
2893.66324303174	0.000148591176532112\\
2915.50787028164	0.000146564506691137\\
2937.51740536621	0.000144552756853686\\
2959.69309319542	0.000142556202724871\\
2982.03618807719	0.000140575109214224\\
3004.5479537884	0.000138609730416624\\
3027.22966364631	0.000136660309607653\\
3050.08260058064	0.000134727079252571\\
3073.10805720609	0.00013281026102816\\
3096.30733589548	0.000130910065856687\\
3119.68174885339	0.000129026693951336\\
3143.23261819043	0.000127160334872451\\
3166.96127599795	0.000125311167593991\\
3190.86906442344	0.000123479360579616\\
3214.95733574645	0.000121665071867866\\
3239.22745245503	0.000119868449165899\\
3263.68078732284	0.000118089629951308\\
3288.31872348677	0.000116328741581528\\
3313.14265452519	0.00011458590141039\\
3338.15398453677	0.000112861216911377\\
3363.3541282199	0.000111154785807177\\
3388.74451095268	0.000109466696205116\\
3414.32656887361	0.000107797026738099\\
3440.10174896275	0.00010614584671068\\
3466.07150912361	0.000104513216249899\\
3492.23731826558	0.000102899186460546\\
3518.60065638705	0.000101303799584516\\
3545.16301465908	9.97270891639207e-05\\
3571.9258955098	9.81690802076679e-05\\
3598.89081270933	9.66297893611755e-05\\
3626.05929145545	9.51092250789538e-05\\
3653.43286845983	9.36073877997542e-05\\
3681.01309203499	9.21242701240204e-05\\
3708.80152218184	9.0659856993372e-05\\
3736.79973067795	8.92141258718626e-05\\
3765.00930116641	8.77870469287679e-05\\
3793.43182924546	8.63785832226558e-05\\
3822.06892255869	8.49886908865152e-05\\
3850.92220088601	8.36173193137075e-05\\
3879.99329623524	8.22644113445342e-05\\
3909.28385293442	8.09299034532013e-05\\
3938.79552772486	7.96137259349844e-05\\
3968.5299898548	7.83158030933958e-05\\
3998.48892117385	7.70360534271666e-05\\
4028.67401622809	7.57743898168628e-05\\
4059.08698235599	7.45307197109591e-05\\
4089.72953978488	7.33049453112082e-05\\
4120.60342172835	7.20969637571372e-05\\
4151.71037448419	7.09066673095269e-05\\
4183.05215753326	6.97339435327169e-05\\
4214.63054363891	6.85786754756043e-05\\
4246.44731894735	6.74407418511956e-05\\
4278.50428308861	6.63200172145878e-05\\
4310.80324927834	6.52163721392554e-05\\
4343.34604442043	6.41296733915274e-05\\
4376.13450921022	6.30597841031477e-05\\
4409.17049823878	6.20065639418097e-05\\
4442.45588009764	6.09698692795767e-05\\
4475.99253748463	5.99495533590862e-05\\
4509.78236731024	5.89454664574613e-05\\
4543.82728080503	5.79574560478418e-05\\
4578.12920362765	5.69853669584646e-05\\
4612.69007597378	5.60290415292225e-05\\
4647.51185268592	5.50883197656365e-05\\
4682.59650336387	5.41630394901843e-05\\
4717.94601247624	5.32530364909282e-05\\
4753.5623794726	5.23581446673986e-05\\
4789.44761889663	5.14781961736812e-05\\
4825.60376050007	5.06130215586777e-05\\
4862.03284935749	4.97624499034961e-05\\
4898.73694598198	4.89263089559469e-05\\
4935.71812644172	4.81044252621159e-05\\
4972.9784824774	4.72966242949923e-05\\
5010.52012162047	4.65027305801354e-05\\
5048.34516731247	4.57225678183631e-05\\
5086.45575902501	4.49559590054579e-05\\
5124.85405238087	4.42027265488766e-05\\
5163.54221927589	4.34626923814684e-05\\
5202.52244800183	4.27356780721968e-05\\
5241.79694337011	4.20215049338718e-05\\
5281.3679268366	4.13199941278995e-05\\
5321.23763662717	4.06309667660584e-05\\
5361.40832786437	3.99542440093167e-05\\
5401.88227269494	3.92896471637058e-05\\
5442.66176041833	3.8636997773269e-05\\
5483.74909761624	3.79961177101048e-05\\
5525.14660828299	3.73668292615325e-05\\
5566.85663395707	3.67489552143993e-05\\
5608.88153385351	3.61423189365633e-05\\
5651.22368499737	3.55467444555765e-05\\
5693.88548235815	3.49620565346032e-05\\
5736.8693389853	3.43880807456046e-05\\
5780.17768614467	3.3824643539825e-05\\
5823.81297345603	3.32715723156163e-05\\
5867.77766903166	3.27286954836363e-05\\
5912.07425961591	3.21958425294624e-05\\
5956.7052507259	3.16728440736551e-05\\
6001.67316679318	3.11595319293183e-05\\
6046.98055130657	3.06557391571905e-05\\
6092.62996695602	3.01613001183139e-05\\
6138.62399577752	2.96760505243202e-05\\
6184.9652392992	2.91998274853776e-05\\
6231.65631868842	2.87324695558407e-05\\
6278.6998749001	2.82738167776453e-05\\
6326.09856882603	2.78237107214942e-05\\
6373.85508144543	2.73819945258721e-05\\
6421.97211397655	2.69485129339371e-05\\
6470.45238802948	2.65231123283271e-05\\
6519.29864576009	2.61056407639255e-05\\
6568.51365002513	2.56959479986279e-05\\
6618.10018453851	2.52938855221487e-05\\
6668.06105402872	2.48993065829111e-05\\
6718.39908439753	2.45120662130584e-05\\
6769.11712287976	2.41320212516267e-05\\
6820.21803820442	2.37590303659179e-05\\
6871.70472075685	2.33929540711107e-05\\
6923.58008274232	2.3033654748147e-05\\
6975.84705835067	2.26809966599308e-05\\
7028.50860392234	2.23348459658744e-05\\
7081.56769811553	2.19950707348289e-05\\
7135.02734207472	2.16615409564326e-05\\
7188.8905596004	2.13341285509111e-05\\
7243.16039732009	2.1012707377364e-05\\
7297.83992486074	2.06971532405697e-05\\
7352.93223502222	2.03873438963425e-05\\
7408.44044395243	2.00831590554724e-05\\
7464.36769132336	1.97844803862822e-05\\
7520.71714050886	1.949119151583e-05\\
7577.49197876344	1.92031780297931e-05\\
7634.69541740261	1.89203274710603e-05\\
7692.33069198451	1.86425293370682e-05\\
7750.40106249289	1.83696750759099e-05\\
7808.90981352158	1.81016580812497e-05\\
7867.86025446014	1.78383736860753e-05\\
7927.25571968122	1.75797191553198e-05\\
7987.09956872899	1.73255936773853e-05\\
8047.39518650928	1.70758983546026e-05\\
8108.14598348099	1.68305361926588e-05\\
8169.35539584901	1.65894120890278e-05\\
8231.02688575857	1.63524328204373e-05\\
8293.16394149105	1.61195070294073e-05\\
8355.77007766133	1.58905452098955e-05\\
8418.84883541654	1.56654596920854e-05\\
8482.40378263641	1.54441646263527e-05\\
8546.43851413497	1.52265759664483e-05\\
8610.95665186405	1.50126114519317e-05\\
8675.96184511796	1.48021905898968e-05\\
8741.45777074004	1.45952346360232e-05\\
8807.44813333057	1.43916665749939e-05\\
8873.93666545632	1.41914111003163e-05\\
8940.9271278617	1.39943945935843e-05\\
9008.42330968139	1.38005451032211e-05\\
9076.42902865481	1.36097923227378e-05\\
9144.94813134188	1.34220675685497e-05\\
9213.98449334078	1.3237303757383e-05\\
9283.54201950699	1.30554353833137e-05\\
9353.62464417428	1.28763984944711e-05\\
9424.23633137717	1.27001306694456e-05\\
9495.38107507519	1.25265709934329e-05\\
9567.0628993788	1.23556600341526e-05\\
9639.28585877689	1.21873398175711e-05\\
9712.05403836631	1.20215538034657e-05\\
9785.37155408272	1.18582468608581e-05\\
9859.24255293356	1.16973652433515e-05\\
9933.67121323252	1.15388565643973e-05\\
10008.6617448359	1.13826697725239e-05\\
10084.2183893808	1.1228755126551e-05\\
10160.3454205248	1.1077064170819e-05\\
10237.047144188	1.09275497104546e-05\\
10314.3278987966	1.07801657866995e-05\\
10392.1920555277	1.06348676523215e-05\\
10470.6440185573	1.04916117471308e-05\\
10549.6882253089	1.03503556736186e-05\\
10629.3291467048	1.02110581727385e-05\\
10709.5712874188	1.0073679099846e-05\\
10790.419186131	9.93817940081147e-06\\
10871.8774157846	9.80452108832144e-06\\
10953.9505838446	9.6726672183807e-06\\
11036.6433325583	9.54258186702715e-06\\
11119.9603392177	9.41423010727002e-06\\
11203.9063164246	9.28757798626078e-06\\
11288.4860123565	9.16259250270534e-06\\
11373.7042110358	9.03924158452538e-06\\
11459.5657325999	8.91749406677457e-06\\
11546.075433574	8.79731966981643e-06\\
11633.2382071459	8.67868897776795e-06\\
11721.0589834426	8.56157341721332e-06\\
11809.5427298094	8.44594523619175e-06\\
11898.6944510906	8.33177748346191e-06\\
11988.5191899127	8.21904398804522e-06\\
12079.0220269697	8.10771933905029e-06\\
12170.2080813103	7.99777886577981e-06\\
12262.0825106276	7.88919861812113e-06\\
12354.6505115508	7.78195534722103e-06\\
12447.9173199389	7.67602648644563e-06\\
12541.8882111773	7.57139013262555e-06\\
12636.568500476	7.46802502758663e-06\\
12731.9635431699	7.36591053996633e-06\\
12828.0787350221	7.26502664731563e-06\\
12924.9195125292	7.16535391848634e-06\\
13022.4913532284	7.06687349630412e-06\\
13120.7997760075	6.96956708052682e-06\\
13219.8503414172	6.87341691108788e-06\\
13319.6486519853	6.77840575162535e-06\\
13420.200352534	6.684516873296e-06\\
13521.511130499	6.59173403887492e-06\\
13623.5867162509	6.50004148714079e-06\\
13726.4328834197	6.40942391754683e-06\\
13830.0554492214	6.31986647517777e-06\\
13934.4602747868	6.23135473599339e-06\\
14039.6532654931	6.14387469235887e-06\\
14145.6403712978	6.05741273886229e-06\\
14252.4275870757	5.97195565841992e-06\\
14360.0209529573	5.88749060866969e-06\\
14468.4265546712	5.80400510865339e-06\\
14577.6505238876	5.72148702578806e-06\\
14687.6990385654	5.63992456312684e-06\\
14798.5783233021	5.55930624690978e-06\\
14910.2946496851	5.47962091440501e-06\\
15022.8543366469	5.40085770204034e-06\\
15136.2637508225	5.32300603382532e-06\\
15250.5293069092	5.24605561006387e-06\\
15365.65746803	5.16999639635721e-06\\
15481.6547460986	5.09481861289678e-06\\
15598.527702188	5.02051272404662e-06\\
15716.2829469014	4.9470694282144e-06\\
15834.9271407466	4.87447964801015e-06\\
15954.4669945122	4.80273452069166e-06\\
16074.9092696475	4.73182538889505e-06\\
16196.2607786446	4.66174379164878e-06\\
16318.5283854242	4.5924814556693e-06\\
16441.7190057235	4.52403028693616e-06\\
16565.8396074876	4.45638236254414e-06\\
16690.8972112632	4.38952992282985e-06\\
16816.8988905963	4.32346536376957e-06\\
16943.8517724318	4.25818122964537e-06\\
17071.7630375169	4.193670205976e-06\\
17200.639920807	4.12992511270895e-06\\
17330.4897118753	4.0669388976694e-06\\
17461.3197553249	4.00470463026226e-06\\
17593.1374512039	3.94321549542279e-06\\
17725.9502554248	3.88246478781101e-06\\
17859.7656801852	3.82244590624546e-06\\
17994.5912943937	3.76315234837097e-06\\
18130.4347240971	3.70457770555545e-06\\
18267.3036529127	3.64671565801056e-06\\
18405.2058224618	3.58955997013052e-06\\
18544.1490328086	3.53310448604382e-06\\
18684.1411429008	3.47734312537184e-06\\
18825.1900710143	3.42226987918909e-06\\
18967.3037952012	3.36787880617896e-06\\
19110.4903537406	3.31416402897926e-06\\
19254.7578455939	3.2611197307117e-06\\
19400.1144308624	3.20874015168937e-06\\
19546.5683312491	3.15701958629654e-06\\
19694.1278305236	3.10595238003473e-06\\
19842.8012749908	3.05553292672936e-06\\
19992.5970739628	3.00575566589117e-06\\
20143.5237002349	2.95661508022679e-06\\
20295.5896905642	2.9081056932928e-06\\
20448.8036461532	2.86022206728775e-06\\
20603.1742331357	2.81295880097671e-06\\
20758.7101830675	2.76631052774305e-06\\
20915.4202934198	2.7202719137622e-06\\
21073.313428077	2.67483765629246e-06\\
21232.3985178383	2.63000248207764e-06\\
21392.6845609224	2.58576114585693e-06\\
21554.1806234768	2.54210842897724e-06\\
21716.8958400905	2.49903913810353e-06\\
21880.8394143105	2.45654810402261e-06\\
22046.0206191627	2.4146301805363e-06\\
22212.4487976761	2.37328024343974e-06\\
22380.1333634117	2.33249318958099e-06\\
22549.0838009943	2.29226393599807e-06\\
22719.3096666494	2.25258741912978e-06\\
22890.8205887437	2.21345859409663e-06\\
23063.6262683299	2.17487243404891e-06\\
23237.7364796946	2.13682392957815e-06\\
23413.1610709123	2.09930808818917e-06\\
23589.9099644014	2.06231993382972e-06\\
23767.9931574861	2.02585450647471e-06\\
23947.4207229617	1.98990686176241e-06\\
24128.2028096642	1.95447207068004e-06\\
24310.3496430445	1.91954521929612e-06\\
24493.8715257466	1.88512140853725e-06\\
24678.7788381905	1.85119575400709e-06\\
24865.0820391595	1.81776338584524e-06\\
25052.7916663912	1.78481944862404e-06\\
25241.9183371742	1.75235910128119e-06\\
25432.4727489482	1.72037751708625e-06\\
25624.4656799093	1.68886988363934e-06\\
25817.9079896195	1.65783140290018e-06\\
26012.8106196209	1.62725729124568e-06\\
26209.1845940549	1.59714277955467e-06\\
26407.0410202855	1.56748311331805e-06\\
26606.3910895274	1.53827355277303e-06\\
26807.2460774792	1.50950937305985e-06\\
27009.6173449613	1.4811858643997e-06\\
27213.5163385584	1.4532983322926e-06\\
27418.9545912668	1.42584209773378e-06\\
27625.9437231468	1.39881249744753e-06\\
27834.49544198	1.37220488413716e-06\\
28044.6215439317	1.3460146267501e-06\\
28256.3339142177	1.32023711075691e-06\\
28469.6445277769	1.29486773844321e-06\\
28684.5654499485	1.26990192921352e-06\\
28901.1088371544	1.24533511990594e-06\\
29119.286937587	1.22116276511676e-06\\
29339.1120919019	1.19738033753421e-06\\
29560.5967339154	1.17398332828011e-06\\
29783.7533913087	1.15096724725901e-06\\
30008.594686336	1.12832762351354e-06\\
30235.1333365383	1.10606000558549e-06\\
30463.382155463	1.08415996188173e-06\\
30693.3540533885	1.06262308104411e-06\\
30925.0620380546	1.04144497232271e-06\\
31158.5192153983	1.02062126595182e-06\\
31393.7387902947	1.0001476135277e-06\\
31630.7340673043	9.80019688387756e-07\\
31869.5184514253	9.60233185990272e-07\\
32110.105448852	9.40783824294163e-07\\
32352.5086677388	9.21667344138184e-07\\
32596.7418189694	9.02879509618969e-07\\
32842.818716933	8.84416108467354e-07\\
33090.7532803053	8.66272952422439e-07\\
33340.5595328357	8.4844587760292e-07\\
33592.2516041407	8.30930744875169e-07\\
33845.8437305032	8.13723440217552e-07\\
34101.3502556773	7.96819875080596e-07\\
34358.7856317002	7.80215986742505e-07\\
34618.1644197092	7.63907738659663e-07\\
34879.5012907654	7.478911208117e-07\\
35142.8110266837	7.32162150040728e-07\\
35408.1085208687	7.16716870384377e-07\\
35675.4087791572	7.0155135340235e-07\\
35944.726920667	6.86661698496105e-07\\
36216.078178652	6.72044033221359e-07\\
36489.4779013636	6.57694513593165e-07\\
36764.9415529195	6.43609324383224e-07\\
37042.4847141778	6.29784679409245e-07\\
37322.1230836184	6.16216821816052e-07\\
37603.8724782308	6.02902024348233e-07\\
37887.7488344093	5.89836589614103e-07\\
38173.768208854	5.77016850340827e-07\\
38461.9467794789	5.64439169620485e-07\\
38752.3008463271	5.52099941146913e-07\\
39044.8468324927	5.39995589443187e-07\\
39339.60128505	5.28122570079598e-07\\
39636.5808759893	5.16477369882008e-07\\
39935.8024031596	5.05056507130479e-07\\
40237.2827912191	4.93856531748048e-07\\
40541.0390925923	4.82874025479595e-07\\
40847.0884884347	4.72105602060729e-07\\
41155.4482896045	4.61547907376635e-07\\
41466.1359376416	4.51197619610813e-07\\
41779.1690057544	4.41051449383687e-07\\
42094.5651998136	4.3110613988109e-07\\
42412.3423593538	4.21358466972549e-07\\
42732.5184585825	4.11805239319434e-07\\
43055.1116073965	4.02443298472925e-07\\
43380.1400524069	3.93269518961849e-07\\
43707.6221779705	3.84280808370394e-07\\
44037.57650723	3.75474107405763e-07\\
44370.0217031614	3.66846389955771e-07\\
44704.9765696303	3.58394663136475e-07\\
45042.4600524546	3.50115967329897e-07\\
45382.4912404769	3.42007376211876e-07\\
45725.0893666436	3.34065996770178e-07\\
46070.2738090931	3.26288969312905e-07\\
46418.0640922519	3.18673467467314e-07\\
46768.479887939	3.11216698169145e-07\\
47121.541016478	3.0391590164257e-07\\
47477.2674478187	2.96768351370851e-07\\
47835.679302667	2.89771354057842e-07\\
48196.7968536222	2.82922249580456e-07\\
48560.6405263239	2.76218410932231e-07\\
48927.2309006076	2.69657244158098e-07\\
49296.5887116685	2.63236188280535e-07\\
49668.7348512343	2.56952715217196e-07\\
50043.6903687473	2.50804329690214e-07\\
50421.4764725545	2.44788569127293e-07\\
50802.114531107	2.38903003554748e-07\\
51185.6260741696	2.33145235482654e-07\\
51572.0327940378	2.27512899782274e-07\\
51961.3565467648	2.22003663555915e-07\\
52353.6193533982	2.16615225999382e-07\\
52748.8434012247	2.11345318257198e-07\\
53147.0510450263	2.06191703270766e-07\\
53548.2648083438	2.0115217561964e-07\\
53952.5073847507	1.96224561356089e-07\\
54359.8016391376	1.91406717833116e-07\\
54770.1706090047	1.86696533526114e-07\\
55183.6375057655	1.82091927848356e-07\\
55600.2257160592	1.77590850960482e-07\\
56019.9588030736	1.73191283574155e-07\\
56442.8605078779	1.68891236750101e-07\\
56868.9547507659	1.64688751690683e-07\\
57298.2656326087	1.60581899527206e-07\\
57730.8174362176	1.56568781102153e-07\\
58166.634627718	1.52647526746495e-07\\
58605.7418579333	1.48816296052298e-07\\
59048.163963779	1.45073277640792e-07\\
59493.9259696676	1.41416688926087e-07\\
59943.0530889239	1.37844775874724e-07\\
60395.5707252113	1.34355812761229e-07\\
60851.5044739687	1.30948101919876e-07\\
61310.8801238584	1.27619973492806e-07\\
61773.7236582241	1.24369785174726e-07\\
62240.0612565613	1.21195921954313e-07\\
62709.9192959978	1.18096795852544e-07\\
63183.3243527857	1.15070845658109e-07\\
63660.3032038044	1.12116536660081e-07\\
64140.8828280752	1.09232360378016e-07\\
64625.090408288	1.06416834289662e-07\\
65112.9533323374	1.03668501556436e-07\\
65604.4991948736	1.00985930746836e-07\\
66099.7557988615	9.83677155579725e-08\\
66598.7511571543	9.58124745353495e-08\\
67101.5134940779	9.33188507910842e-08\\
67608.0712470273	9.08855117207131e-08\\
68118.4530680746	8.85111487187422e-08\\
68632.68782559	8.61944768930927e-08\\
69150.8046058751	8.39342347785946e-08\\
69672.8327148071	8.17291840496855e-08\\
70198.8016794974	7.95781092324446e-08\\
70728.7412499609	7.74798174161223e-08\\
71262.6814007993	7.54331379642942e-08\\
71800.6523328963	7.34369222257817e-08\\
72342.684475126	7.14900432454767e-08\\
72888.8084860737	6.95913954752007e-08\\
73439.0552557701	6.77398944847244e-08\\
73993.4559074389	6.59344766730776e-08\\
74552.041799257	6.41740989802735e-08\\
75114.8445261277	6.24577385995654e-08\\
75681.8959214685	6.07843926903516e-08\\
76253.2280590112	5.91530780918455e-08\\
76828.8732546163	5.75628310376184e-08\\
77408.8640681011	5.60127068711255e-08\\
77993.2333050806	5.45017797623194e-08\\
78582.0140188235	5.30291424254471e-08\\
79175.239512122	5.15939058381342e-08\\
79772.9433391755	5.01951989618485e-08\\
80375.1593074877	4.88321684638361e-08\\
80981.9214797798	4.75039784406138e-08\\
81593.2641759163	4.62098101431081e-08\\
82209.2219748472	4.49488617035179e-08\\
82829.8297165634	4.37203478639839e-08\\
83455.1225040667	4.25234997071372e-08\\
84085.1357053561	4.13575643885986e-08\\
84719.9049554282	4.02218048714992e-08\\
85359.4661582928	3.91154996630881e-08\\
86003.8554890032	3.8037942553491e-08\\
86653.109395703	3.6988442356675e-08\\
87307.2646016869	3.59663226536823e-08\\
87966.3581074791	3.49709215381811e-08\\
88630.4271929253	3.40015913643883e-08\\
89299.509419301	3.30576984974111e-08\\
89973.6426314364	3.2138623066049e-08\\
90652.8649598577	3.12437587181026e-08\\
91337.214822943	3.03725123782264e-08\\
92026.7309290954	2.95243040083633e-08\\
92721.4522789325	2.86985663707919e-08\\
93421.4181674926	2.78947447938213e-08\\
94126.6681864573	2.71122969401594e-08\\
94837.2422263909	2.63506925779823e-08\\
95553.1804789962	2.56094133547283e-08\\
96274.5234393881	2.48879525736364e-08\\
97001.3119083847	2.41858149730486e-08\\
97733.5869948146	2.35025165084942e-08\\
98471.3901178417	2.28375841375689e-08\\
99214.7630093084	2.21905556076196e-08\\
99963.7477160966	2.15609792462467e-08\\
100718.386602504	2.09484137546331e-08\\
101478.722352644	2.03524280037008e-08\\
102244.797972856	1.97726008331057e-08\\
103016.65679414	1.92085208530667e-08\\
103794.342474607	1.86597862490328e-08\\
104577.899001948	1.81260045891863e-08\\
105367.370695925	1.76067926347785e-08\\
106162.802210873	1.71017761532926e-08\\
106964.23853823	1.66105897344279e-08\\
107771.725009077	1.61328766088988e-08\\
108585.307296709	1.56682884700359e-08\\
109405.031419212	1.52164852981828e-08\\
110230.943742068	1.47771351878734e-08\\
111063.09098078	1.43499141777764e-08\\
111901.52020351	1.3934506083395e-08\\
112746.278833744	1.35306023325042e-08\\
113597.414652975	1.3137901803309e-08\\
114454.975803403	1.27561106653053e-08\\
115319.010790661	1.23849422228268e-08\\
116189.568486556	1.20241167612555e-08\\
117066.698131834	1.16733613958769e-08\\
117950.449338968	1.13324099233575e-08\\
118840.872094958	1.10010026758221e-08\\
119738.016764165	1.06788863775095e-08\\
120641.934091157	1.03658140039806e-08\\
121552.675203577	1.00615446438562e-08\\
122470.291615039	9.76584336305812e-09\\
123394.835228039	9.47848107152974e-09\\
124326.358336891	9.19923439240835e-09\\
125264.913630686	8.92788553362317e-09\\
126210.554196271	8.66422216189249e-09\\
127163.333521252	8.40803727909166e-09\\
128123.305497023	8.15912910096499e-09\\
129090.524421805	7.91730093815291e-09\\
130065.045003728	7.68236107950585e-09\\
131046.922363918	7.45412267765622e-09\\
132036.212039619	7.23240363681999e-09\\
133032.969987331	7.01702650279827e-09\\
134037.252585978	6.80781835514941e-09\\
135049.116640094	6.60461070150267e-09\\
136068.619383039	6.40723937398342e-09\\
137095.818480235	6.2155444277206e-09\\
138130.772032427	6.02937004140651e-09\\
139173.538578968	5.848564419879e-09\\
140224.177101134	5.67297969869632e-09\\
141282.747025459	5.50247185067528e-09\\
142349.308227094	5.3369005943625e-09\\
143423.921033194	5.17612930440893e-09\\
144506.646226332	5.02002492381858e-09\\
145597.545047938	4.86845787804114e-09\\
146696.679201759	4.72130199087985e-09\\
147804.110857352	4.57843440218465e-09\\
148919.902653601	4.43973548730164e-09\\
150044.117702257	4.30508877824969e-09\\
151176.819591511	4.17438088659558e-09\\
152318.072389588	4.04750142799876e-09\\
153467.940648372	3.92434294839707e-09\\
154626.48940706	3.8048008518053e-09\\
155793.784195833	3.68877332969868e-09\\
156969.891039574	3.57616129195274e-09\\
158154.87646159	3.46686829931284e-09\\
159348.807487385	3.36080049736505e-09\\
160551.751648444	3.2578665519818e-09\\
161763.776986059	3.15797758621524e-09\\
162984.952055171	3.06104711861175e-09\\
164215.345928253	2.96699100292104e-09\\
165455.028199213	2.87572736917393e-09\\
166704.068987334	2.7871765661032e-09\\
167962.538941238	2.70126110488145e-09\\
169230.50924288	2.61790560415159e-09\\
170508.051611579	2.53703673632422e-09\\
171795.238308072	2.45858317511784e-09\\
173092.142138601	2.38247554431745e-09\\
174398.83645903	2.30864636772769e-09\\
175715.395178998	2.23703002029645e-09\\
177041.892766095	2.16756268038608e-09\\
178378.404250078	2.10018228316894e-09\\
179725.005227113	2.0348284751246e-09\\
181081.771864049	1.97144256961631e-09\\
182448.780902729	1.90996750352463e-09\\
183826.109664331	1.85034779491651e-09\\
185213.83605374	1.79252950172846e-09\\
186612.038563953	1.73646018144264e-09\\
188020.796280522	1.68208885173499e-09\\
189440.188886025	1.62936595207512e-09\\
190870.296664576	1.57824330625774e-09\\
192311.200506361	1.52867408584583e-09\\
193762.981912217	1.48061277450604e-09\\
195225.722998241	1.43401513321713e-09\\
196699.506500436	1.38883816633259e-09\\
198184.415779389	1.34504008847901e-09\\
199680.534824985	1.30258029227181e-09\\
201187.94826116	1.26141931683046e-09\\
202706.741350687	1.22151881707561e-09\\
204237	1.18284153379085e-09\\
204237	7.15388552259442e-10\\
202706.741350687	7.40422696377809e-10\\
201187.94826116	7.66303467786111e-10\\
199680.534824985	7.93058397756867e-10\\
198184.415779389	8.20715871575889e-10\\
196699.506500436	8.49305153479591e-10\\
195225.722998241	8.78856412261583e-10\\
193762.981912217	9.09400747564215e-10\\
192311.200506361	9.40970216871015e-10\\
190870.296664576	9.73597863216324e-10\\
189440.188886025	1.00731774362871e-09\\
188020.796280522	1.04216495832493e-09\\
186612.038563953	1.07817568067158e-09\\
185213.83605374	1.11538718793188e-09\\
183826.109664331	1.15383789281512e-09\\
182448.780902729	1.19356737584706e-09\\
181081.771864049	1.23461641857922e-09\\
179725.005227113	1.27702703765579e-09\\
178378.404250078	1.32084251975695e-09\\
177041.892766095	1.36610745743781e-09\\
175715.395178998	1.41286778588226e-09\\
174398.83645903	1.46117082059155e-09\\
173092.142138601	1.51106529602745e-09\\
171795.238308072	1.56260140523048e-09\\
170508.051611579	1.61583084043348e-09\\
169230.50924288	1.6708068346917e-09\\
167962.538941238	1.7275842045502e-09\\
166704.068987334	1.7862193937701e-09\\
165455.028199213	1.84677051813544e-09\\
164215.345928253	1.90929741136248e-09\\
162984.952055171	1.97386167213348e-09\\
161763.776986059	2.04052671227797e-09\\
160551.751648444	2.10935780612354e-09\\
159348.807487385	2.18042214103972e-09\\
158154.87646159	2.25378886919794e-09\\
156969.891039574	2.32952916057113e-09\\
155793.784195833	2.40771625719659e-09\\
154626.48940706	2.48842552872633e-09\\
153467.940648372	2.57173452928901e-09\\
152318.072389588	2.65772305568771e-09\\
151176.819591511	2.74647320695872e-09\\
150044.117702257	2.83806944531555e-09\\
148919.902653601	2.93259865850392e-09\\
147804.110857352	3.03015022359265e-09\\
146696.679201759	3.13081607222605e-09\\
145597.545047938	3.23469075736342e-09\\
144506.646226332	3.34187152153174e-09\\
143423.921033194	3.45245836661749e-09\\
142349.308227094	3.56655412522372e-09\\
141282.747025459	3.68426453361872e-09\\
140224.177101134	3.80569830630318e-09\\
139173.538578968	3.93096721222198e-09\\
138130.772032427	4.06018615264791e-09\\
137095.818480235	4.19347324076372e-09\\
136068.619383039	4.33094988296985e-09\\
135049.116640094	4.47274086194498e-09\\
134037.252585978	4.61897442148657e-09\\
133032.969987331	4.76978235315847e-09\\
132036.212039619	4.92530008477318e-09\\
131046.922363918	5.08566677073623e-09\\
130065.045003728	5.25102538428006e-09\\
129090.524421805	5.4215228116149e-09\\
128123.305497023	5.59730994802416e-09\\
127163.333521252	5.77854179593179e-09\\
126210.554196271	5.96537756496956e-09\\
125264.913630686	6.15798077407139e-09\\
124326.358336891	6.35651935562232e-09\\
123394.835228039	6.56116576168957e-09\\
122470.291615039	6.77209707236342e-09\\
121552.675203577	6.98949510623483e-09\\
120641.934091157	7.21354653303763e-09\\
119738.016764165	7.44444298848184e-09\\
118840.872094958	7.68238119130559e-09\\
117950.449338968	7.92756306257288e-09\\
117066.698131834	8.18019584724376e-09\\
116189.568486556	8.44049223804348e-09\\
115319.010790661	8.70867050165734e-09\\
114454.975803403	8.9849546072782e-09\\
113597.414652975	9.26957435753182e-09\\
112746.278833744	9.56276552180735e-09\\
111901.52020351	9.86476997201741e-09\\
111063.09098078	1.01758358208143e-08\\
110230.943742068	1.04962175622877e-08\\
109405.031419212	1.08261762151687e-08\\
108585.307296709	1.1165979468565e-08\\
107771.725009077	1.15159018302518e-08\\
106964.23853823	1.18762247775429e-08\\
106162.802210873	1.22472369107663e-08\\
105367.370695925	1.26292341093667e-08\\
104577.899001948	1.30225196906599e-08\\
103794.342474607	1.34274045712601e-08\\
103016.65679414	1.38442074312044e-08\\
102244.797972856	1.42732548807966e-08\\
101478.722352644	1.47148816301894e-08\\
100718.386602504	1.51694306617301e-08\\
99963.7477160966	1.56372534050879e-08\\
99214.7630093084	1.61187099151851e-08\\
98471.3901178417	1.66141690529494e-08\\
97733.5869948146	1.71240086689106e-08\\
97001.3119083847	1.7648615789656e-08\\
96274.5234393881	1.81883868071668e-08\\
95553.1804789962	1.87437276710513e-08\\
94837.2422263909	1.93150540836911e-08\\
94126.6681864573	1.99027916983186e-08\\
93421.4181674926	2.05073763200407e-08\\
92721.4522789325	2.11292541098238e-08\\
92026.7309290954	2.17688817914545e-08\\
91337.214822943	2.24267268614888e-08\\
90652.8649598577	2.31032678022046e-08\\
89973.6426314364	2.3798994297567e-08\\
89299.509419301	2.451440745222e-08\\
88630.4271929253	2.52500200135123e-08\\
87966.3581074791	2.60063565965674e-08\\
87307.2646016869	2.67839539124074e-08\\
86653.109395703	2.75833609991369e-08\\
86003.8554890032	2.84051394561916e-08\\
85359.4661582928	2.92498636816584e-08\\
84719.9049554282	3.01181211126711e-08\\
84085.1357053561	3.10105124688818e-08\\
83455.1225040667	3.19276519990131e-08\\
82829.8297165634	3.28701677304864e-08\\
82209.2219748472	3.38387017221271e-08\\
81593.2641759163	3.48339103199453e-08\\
80981.9214797798	3.58564644159841e-08\\
80375.1593074877	3.69070497102309e-08\\
79772.9433391755	3.79863669755828e-08\\
79175.239512122	3.90951323258578e-08\\
78582.0140188235	4.02340774868385e-08\\
77993.2333050806	4.14039500703364e-08\\
77408.8640681011	4.26055138512582e-08\\
76828.8732546163	4.38395490476588e-08\\
76253.2280590112	4.51068526037597e-08\\
75681.8959214685	4.64082384759124e-08\\
75114.8445261277	4.77445379214777e-08\\
74552.041799257	4.91165997905959e-08\\
73993.4559074389	5.052529082082e-08\\
73439.0552557701	5.19714959345738e-08\\
72888.8084860737	5.34561185394051e-08\\
72342.684475126	5.49800808309887e-08\\
71800.6523328963	5.65443240988415e-08\\
71262.6814007993	5.81498090347045e-08\\
70728.7412499609	5.97975160435418e-08\\
70198.8016794974	6.14884455571037e-08\\
69672.8327148071	6.32236183499984e-08\\
69150.8046058751	6.50040758582151e-08\\
68632.68782559	6.68308805000334e-08\\
68118.4530680746	6.87051159992474e-08\\
67608.0712470273	7.06278877106424e-08\\
67101.5134940779	7.26003229476348e-08\\
66598.7511571543	7.46235713120068e-08\\
66099.7557988615	7.6698805025645e-08\\
65604.4991948736	7.88272192641897e-08\\
65112.9533323374	8.10100324925043e-08\\
64625.090408288	8.32484868018636e-08\\
64140.8828280752	8.55438482487561e-08\\
63660.3032038044	8.78974071951829e-08\\
63183.3243527857	9.03104786503482e-08\\
62709.9192959978	9.27844026136048e-08\\
62240.0612565613	9.53205444185389e-08\\
61773.7236582241	9.79202950780543e-08\\
61310.8801238584	1.00585071630312e-07\\
60851.5044739687	1.03316317485382e-07\\
60395.5707252113	1.06115502772456e-07\\
59943.0530889239	1.08984124687458e-07\\
59493.9259696676	1.11923707840881e-07\\
59048.163963779	1.14935804605682e-07\\
58605.7418579333	1.18021995465056e-07\\
58166.634627718	1.21183889359889e-07\\
57730.8174362176	1.24423124035711e-07\\
57298.2656326087	1.27741366388931e-07\\
56868.9547507659	1.31140312812138e-07\\
56442.8605078779	1.3462168953828e-07\\
56019.9588030736	1.38187252983456e-07\\
55600.2257160592	1.41838790088109e-07\\
55183.6375057655	1.45578118656369e-07\\
54770.1706090047	1.49407087693299e-07\\
54359.8016391376	1.53327577739778e-07\\
53952.5073847507	1.57341501204765e-07\\
53548.2648083438	1.61450802694649e-07\\
53147.0510450263	1.65657459339416e-07\\
52748.8434012247	1.69963481115334e-07\\
52353.6193533982	1.74370911163859e-07\\
51961.3565467648	1.78881826106437e-07\\
51572.0327940378	1.8349833635489e-07\\
51185.6260741696	1.88222586417079e-07\\
50802.114531107	1.93056755197463e-07\\
50421.4764725545	1.98003056292257e-07\\
50043.6903687473	2.030637382788e-07\\
49668.7348512343	2.0824108499878e-07\\
49296.5887116685	2.13537415834963e-07\\
48927.2309006076	2.18955085981025e-07\\
48560.6405263239	2.24496486704111e-07\\
48196.7968536222	2.30164045599719e-07\\
47835.679302667	2.35960226838522e-07\\
47477.2674478187	2.41887531404701e-07\\
47121.541016478	2.4794849732538e-07\\
46768.479887939	2.54145699890742e-07\\
46418.0640922519	2.60481751864376e-07\\
46070.2738090931	2.66959303683446e-07\\
45725.0893666436	2.73581043648224e-07\\
45382.4912404769	2.80349698100525e-07\\
45042.4600524546	2.87268031590609e-07\\
44704.9765696303	2.94338847032078e-07\\
44370.0217031614	3.01564985844321e-07\\
44037.57650723	3.08949328082003e-07\\
43707.6221779705	3.16494792551176e-07\\
43380.1400524069	3.24204336911482e-07\\
43055.1116073965	3.32080957764024e-07\\
42732.5184585825	3.401276907244e-07\\
42412.3423593538	3.48347610480413e-07\\
42094.5651998136	3.56743830833996e-07\\
41779.1690057544	3.65319504726873e-07\\
41466.1359376416	3.74077824249482e-07\\
41155.4482896045	3.83022020632649e-07\\
40847.0884884347	3.92155364221621e-07\\
40541.0390925923	4.01481164431899e-07\\
40237.2827912191	4.11002769686489e-07\\
39935.8024031596	4.20723567334079e-07\\
39636.5808759893	4.30646983547701e-07\\
39339.60128505	4.40776483203439e-07\\
39044.8468324927	4.51115569738779e-07\\
38752.3008463271	4.6166778499017e-07\\
38461.9467794789	4.72436709009384e-07\\
38173.768208854	4.83425959858307e-07\\
37887.7488344093	4.94639193381794e-07\\
37603.8724782308	5.06080102958213e-07\\
37322.1230836184	5.17752419227373e-07\\
37042.4847141778	5.29659909795496e-07\\
36764.9415529195	5.41806378916945e-07\\
36489.4779013636	5.54195667152472e-07\\
36216.078178652	5.66831651003722e-07\\
35944.726920667	5.79718242523773e-07\\
35675.4087791572	5.92859388903535e-07\\
35408.1085208687	6.06259072033891e-07\\
35142.8110266837	6.1992130804341e-07\\
34879.5012907654	6.33850146811601e-07\\
34618.1644197092	6.48049671457626e-07\\
34358.7856317002	6.62523997804481e-07\\
34101.3502556773	6.77277273818718e-07\\
33845.8437305032	6.92313679025755e-07\\
33592.2516041407	7.07637423900911e-07\\
33340.5595328357	7.23252749236352e-07\\
33090.7532803053	7.39163925484201e-07\\
32842.818716933	7.55375252076065e-07\\
32596.7418189694	7.71891056719361e-07\\
32352.5086677388	7.88715694670785e-07\\
32110.105448852	8.05853547987434e-07\\
31869.5184514253	8.23309024756109e-07\\
31630.7340673043	8.41086558301363e-07\\
31393.7387902947	8.59190606372957e-07\\
31158.5192153983	8.77625650313453e-07\\
30925.0620380546	8.96396194206773e-07\\
30693.3540533885	9.1550676400856e-07\\
30463.382155463	9.34961906659322e-07\\
30235.1333365383	9.54766189181369e-07\\
30008.594686336	9.74924197760652e-07\\
29783.7533913087	9.95440536814728e-07\\
29560.5967339154	1.01631982804812e-06\\
29339.1120919019	1.03756670949639e-06\\
29119.286937587	1.05918583456044e-06\\
28901.1088371544	1.08118187103253e-06\\
28684.5654499485	1.10355950011571e-06\\
28469.6445277769	1.12632341543832e-06\\
28256.3339142177	1.14947832206536e-06\\
28044.6215439317	1.17302893550877e-06\\
27834.49544198	1.19697998073841e-06\\
27625.9437231468	1.22133619119609e-06\\
27418.9545912668	1.24610230781458e-06\\
27213.5163385584	1.27128307804401e-06\\
27009.6173449613	1.29688325488817e-06\\
26807.2460774792	1.32290759595285e-06\\
26606.3910895274	1.3493608625091e-06\\
26407.0410202855	1.37624781857386e-06\\
26209.1845940549	1.40357323001082e-06\\
26012.8106196209	1.43134186365422e-06\\
25817.9079896195	1.45955848645863e-06\\
25624.4656799093	1.48822786467764e-06\\
25432.4727489482	1.51735476307438e-06\\
25241.9183371742	1.54694394416743e-06\\
25052.7916663912	1.57700016751482e-06\\
24865.0820391595	1.60752818903984e-06\\
24678.7788381905	1.63853276040169e-06\\
24493.8715257466	1.67001862841465e-06\\
24310.3496430445	1.70199053451902e-06\\
24128.2028096642	1.73445321430746e-06\\
23947.4207229617	1.76741139711023e-06\\
23767.9931574861	1.80086980564297e-06\\
23589.9099644014	1.83483315572055e-06\\
23413.1610709123	1.8693061560408e-06\\
23237.7364796946	1.90429350804157e-06\\
23063.6262683299	1.93979990583495e-06\\
22890.8205887437	1.97583003622229e-06\\
22719.3096666494	2.01238857879359e-06\\
22549.0838009943	2.04948020611515e-06\\
22380.1333634117	2.08710958400883e-06\\
22212.4487976761	2.1252813719267e-06\\
22046.0206191627	2.16400022342472e-06\\
21880.8394143105	2.20327078673877e-06\\
21716.8958400905	2.24309770546677e-06\\
21554.1806234768	2.28348561936013e-06\\
21392.6845609224	2.32443916522795e-06\\
21232.3985178383	2.36596297795744e-06\\
21073.313428077	2.40806169165356e-06\\
20915.4202934198	2.45073994090116e-06\\
20758.7101830675	2.49400236215268e-06\\
20603.1742331357	2.53785359524443e-06\\
20448.8036461532	2.58229828504436e-06\\
20295.5896905642	2.62734108323406e-06\\
20143.5237002349	2.67298665022771e-06\\
19992.5970739628	2.71923965723081e-06\\
19842.8012749908	2.76610478844094e-06\\
19694.1278305236	2.81358674339326e-06\\
19546.5683312491	2.86169023945294e-06\\
19400.1144308624	2.91042001445677e-06\\
19254.7578455939	2.9597808295064e-06\\
19110.4903537406	3.00977747191508e-06\\
18967.3037952012	3.06041475830998e-06\\
18825.1900710143	3.11169753789221e-06\\
18684.1411429008	3.16363069585635e-06\\
18544.1490328086	3.21621915697142e-06\\
18405.2058224618	3.26946788932523e-06\\
18267.3036529127	3.32338190823373e-06\\
18130.4347240971	3.37796628031728e-06\\
17994.5912943937	3.4332261277457e-06\\
17859.7656801852	3.48916663265386e-06\\
17725.9502554248	3.54579304172954e-06\\
17593.1374512039	3.6031106709754e-06\\
17461.3197553249	3.66112491064718e-06\\
17330.4897118753	3.71984123036981e-06\\
17200.639920807	3.77926518443335e-06\\
17071.7630375169	3.83940241727123e-06\\
16943.8517724318	3.90025866912234e-06\\
16816.8988905963	3.96183978187968e-06\\
16690.8972112632	4.02415170512766e-06\\
16565.8396074876	4.08720050237037e-06\\
16441.7190057235	4.15099235745347e-06\\
16318.5283854242	4.21553358118221e-06\\
16196.2607786446	4.28083061813847e-06\\
16074.9092696475	4.34689005369936e-06\\
15954.4669945122	4.4137186212604e-06\\
15834.9271407466	4.48132320966649e-06\\
15716.2829469014	4.54971087085354e-06\\
15598.527702188	4.61888882770409e-06\\
15481.6547460986	4.68886448212013e-06\\
15365.65746803	4.75964542331649e-06\\
15250.5293069092	4.83123943633822e-06\\
15136.2637508225	4.90365451080546e-06\\
15022.8543366469	4.97689884988912e-06\\
14910.2946496851	5.05098087952085e-06\\
14798.5783233021	5.12590925784091e-06\\
14687.6990385654	5.20169288488725e-06\\
14577.6505238876	5.27834091252912e-06\\
14468.4265546712	5.35586275464859e-06\\
14360.0209529573	5.43426809757295e-06\\
14252.4275870757	5.51356691076137e-06\\
14145.6403712978	5.59376945774847e-06\\
14039.6532654931	5.67488630734746e-06\\
13934.4602747868	5.75692834511555e-06\\
13830.0554492214	5.83990678508371e-06\\
13726.4328834197	5.92383318175302e-06\\
13623.5867162509	6.00871944235915e-06\\
13521.511130499	6.09457783940642e-06\\
13420.200352534	6.18142102347264e-06\\
13319.6486519853	6.26926203628548e-06\\
13219.8503414172	6.35811432407086e-06\\
13120.7997760075	6.44799175117311e-06\\
13022.4913532284	6.53890861394681e-06\\
12924.9195125292	6.6308796549196e-06\\
12828.0787350221	6.72392007722448e-06\\
12731.9635431699	6.81804555930048e-06\\
12636.568500476	6.91327226985925e-06\\
12541.8882111773	7.00961688311542e-06\\
12447.9173199389	7.10709659427805e-06\\
12354.6505115508	7.20572913529975e-06\\
12262.0825106276	7.30553279088001e-06\\
12170.2080813103	7.40652641471861e-06\\
12079.0220269697	7.50872944601525e-06\\
11988.5191899127	7.61216192621033e-06\\
11898.6944510906	7.71684451596234e-06\\
11809.5427298094	7.82279851235644e-06\\
11721.0589834426	7.93004586633883e-06\\
11633.2382071459	8.03860920037164e-06\\
11546.075433574	8.14851182630242e-06\\
11459.5657325999	8.25977776344211e-06\\
11373.7042110358	8.37243175684583e-06\\
11288.4860123565	8.48649929579077e-06\\
11203.9063164246	8.60200663244474e-06\\
11119.9603392177	8.71898080071966e-06\\
11036.6433325583	8.83744963530459e-06\\
10953.9505838446	8.95744179087197e-06\\
10871.8774157846	9.07898676145229e-06\\
10790.419186131	9.20211489997174e-06\\
10709.5712874188	9.32685743794761e-06\\
10629.3291467048	9.45324650533719e-06\\
10549.6882253089	9.58131515053572e-06\\
10470.6440185573	9.71109736051948e-06\\
10392.1920555277	9.84262808113004e-06\\
10314.3278987966	9.9759432374972e-06\\
10237.047144188	1.0111079754597e-05\\
10160.3454205248	1.0248075577943e-05\\
10084.2183893808	1.03869696944086e-05\\
10008.6617448359	1.05278021531788e-05\\
9933.67121323252	1.06706140868297e-05\\
9859.24255293356	1.08154477325355e-05\\
9785.37155408272	1.09623464534021e-05\\
9712.05403836631	1.11113547599265e-05\\
9639.28585877689	1.12625183315823e-05\\
9567.0628993788	1.14158840385306e-05\\
9495.38107507519	1.15714999634575e-05\\
9424.23633137717	1.17294154235361e-05\\
9353.62464417428	1.18896809925156e-05\\
9283.54201950699	1.2052348522934e-05\\
9213.98449334078	1.22174711684568e-05\\
9144.94813134188	1.23851034063385e-05\\
9076.42902865481	1.25553010600061e-05\\
9008.42330968139	1.27281213217633e-05\\
8940.9271278617	1.29036227756105e-05\\
8873.93666545632	1.30818654201784e-05\\
8807.44813333057	1.32629106917705e-05\\
8741.45777074004	1.3446821487507e-05\\
8675.96184511796	1.36336621885653e-05\\
8610.95665186405	1.38234986835059e-05\\
8546.43851413497	1.40163983916764e-05\\
8482.40378263641	1.42124302866801e-05\\
8418.84883541654	1.44116649198971e-05\\
8355.77007766133	1.46141744440416e-05\\
8293.16394149105	1.48200326367403e-05\\
8231.02688575857	1.50293149241108e-05\\
8169.35539584901	1.52420984043207e-05\\
8108.14598348099	1.54584618711033e-05\\
8047.39518650928	1.56784858372047e-05\\
7987.09956872899	1.59022525577357e-05\\
7927.25571968122	1.61298460533968e-05\\
7867.86025446014	1.63613521335463e-05\\
7808.90981352158	1.65968584190755e-05\\
7750.40106249289	1.68364543650558e-05\\
7692.33069198451	1.70802312831169e-05\\
7634.69541740261	1.73282823635178e-05\\
7577.49197876344	1.75807026968641e-05\\
7520.71714050886	1.78375892954287e-05\\
7464.36769132336	1.80990411140271e-05\\
7408.44044395243	1.83651590703988e-05\\
7352.93223502222	1.86360460650418e-05\\
7297.83992486074	1.89118070004492e-05\\
7243.16039732009	1.91925487996914e-05\\
7188.8905596004	1.94783804242879e-05\\
7135.02734207472	1.97694128913108e-05\\
7081.56769811553	2.006575928966e-05\\
7028.50860392234	2.03675347954506e-05\\
6975.84705835067	2.06748566864492e-05\\
6923.58008274232	2.09878443554968e-05\\
6871.70472075685	2.13066193228546e-05\\
6820.21803820442	2.1631305247407e-05\\
6769.11712287976	2.19620279366578e-05\\
6718.39908439753	2.22989153554512e-05\\
6668.06105402872	2.26420976333531e-05\\
6618.10018453851	2.29917070706236e-05\\
6568.51365002513	2.33478781427145e-05\\
6519.29864576009	2.37107475032232e-05\\
6470.45238802948	2.40804539852342e-05\\
6421.97211397655	2.44571386009817e-05\\
6373.85508144543	2.48409445397632e-05\\
6326.09856882603	2.52320171640364e-05\\
6278.6998749001	2.56305040036309e-05\\
6231.65631868842	2.60365547480073e-05\\
6184.9652392992	2.64503212364931e-05\\
6138.62399577752	2.68719574464312e-05\\
6092.62996695602	2.73016194791689e-05\\
6046.98055130657	2.77394655438238e-05\\
6001.67316679318	2.81856559387575e-05\\
5956.7052507259	2.86403530306901e-05\\
5912.07425961591	2.91037212313898e-05\\
5867.77766903166	2.95759269718719e-05\\
5823.81297345603	3.00571386740409e-05\\
5780.17768614467	3.05475267197106e-05\\
5736.8693389853	3.10472634169399e-05\\
5693.88548235815	3.15565229636164e-05\\
5651.22368499737	3.20754814082295e-05\\
5608.88153385351	3.26043166077665e-05\\
5566.85663395707	3.31432081826731e-05\\
5525.14660828299	3.36923374688152e-05\\
5483.74909761624	3.42518874663825e-05\\
5442.66176041833	3.48220427856776e-05\\
5401.88227269494	3.54029895897276e-05\\
5361.40832786437	3.59949155336665e-05\\
5321.23763662717	3.65980097008286e-05\\
5281.3679268366	3.72124625355016e-05\\
5241.79694337011	3.78384657722846e-05\\
5202.52244800183	3.84762123619993e-05\\
5163.54221927589	3.91258963941051e-05\\
5124.85405238087	3.97877130155696e-05\\
5086.45575902501	4.0461858346147e-05\\
5048.34516731247	4.11485293900195e-05\\
5010.52012162047	4.18479239437609e-05\\
4972.9784824774	4.25602405005769e-05\\
4935.71812644172	4.32856781507891e-05\\
4898.73694598198	4.40244364785209e-05\\
4862.03284935749	4.47767154545555e-05\\
4825.60376050007	4.55427153253313e-05\\
4789.44761889663	4.6322636498048e-05\\
4753.5623794726	4.71166794218555e-05\\
4717.94601247624	4.79250444651038e-05\\
4682.59650336387	4.87479317886344e-05\\
4647.51185268592	4.95855412150919e-05\\
4612.69007597378	5.04380720942489e-05\\
4578.12920362765	5.1305723164327e-05\\
4543.82728080503	5.21886924093125e-05\\
4509.78236731024	5.30871769122607e-05\\
4475.99253748463	5.400137270459e-05\\
4442.45588009764	5.49314746113718e-05\\
4409.17049823878	5.58776760926228e-05\\
4376.13450921022	5.68401690806161e-05\\
4343.34604442043	5.78191438132228e-05\\
4310.80324927834	5.88147886633132e-05\\
4278.50428308861	5.98272899642355e-05\\
4246.44731894735	6.08568318314114e-05\\
4214.63054363891	6.19035959800772e-05\\
4183.05215753326	6.29677615392156e-05\\
4151.71037448419	6.40495048617214e-05\\
4120.60342172835	6.51489993308527e-05\\
4089.72953978488	6.62664151630237e-05\\
4059.08698235599	6.74019192069986e-05\\
4028.67401622809	6.8555674739556e-05\\
3998.48892117385	6.97278412576905e-05\\
3968.5299898548	7.09185742674355e-05\\
3938.79552772486	7.21280250693813e-05\\
3909.28385293442	7.33563405409859e-05\\
3879.99329623524	7.46036629157637e-05\\
3850.92220088601	7.58701295594554e-05\\
3822.06892255869	7.71558727432835e-05\\
3793.43182924546	7.84610194143998e-05\\
3765.00930116641	7.97856909636456e-05\\
3736.79973067795	8.1130002990737e-05\\
3708.80152218184	8.24940650670094e-05\\
3681.01309203499	8.38779804958424e-05\\
3653.43286845983	8.5281846070909e-05\\
3626.05929145545	8.67057518323799e-05\\
3598.89081270933	8.81497808212338e-05\\
3571.9258955098	8.96140088318168e-05\\
3545.16301465908	9.1098504162803e-05\\
3518.60065638705	9.26033273667152e-05\\
3492.23731826558	9.41285309981538e-05\\
3466.07150912361	9.5674159360905e-05\\
3440.10174896275	9.7240248254079e-05\\
3414.32656887361	9.88268247174495e-05\\
3388.74451095268	0.000100433906776151\\
3363.3541282199	0.000102061503184901\\
3338.15398453677	0.000103709613171902\\
3313.14265452519	0.000105378226182594\\
3288.31872348677	0.000107067321623396\\
3263.68078732284	0.000108776868605608\\
3239.22745245503	0.000110506825689599\\
3214.95733574645	0.000112257140629431\\
3190.86906442344	0.000114027750118049\\
3166.96127599795	0.000115818579533142\\
3143.23261819043	0.000117629542683797\\
3119.68174885339	0.000119460541558029\\
3096.30733589548	0.00012131146607125\\
3073.10805720609	0.000123182193815757\\
3050.08260058064	0.000125072589811233\\
3027.22966364631	0.000126982506256306\\
3004.5479537884	0.000128911782281102\\
2982.03618807719	0.000130860243700766\\
2959.69309319542	0.000132827702769835\\
2937.51740536621	0.000134813957937334\\
2915.50787028164	0.000136818793602418\\
2893.66324303174	0.000138841979870308\\
2871.98228803412	0.000140883272308244\\
2850.46377896409	0.000142942411701092\\
2829.10649868523	0.000145019123806184\\
2807.90923918064	0.00014711311910688\\
2786.87080148452	0.000149224092564276\\
2765.98999561444	0.000151351723366378\\
2745.26564050395	0.000153495674673964\\
2724.69656393583	0.000155655593362261\\
2704.28160247578	0.000157831109757439\\
2684.01960140661	0.000160021837366808\\
2663.90941466289	0.000162227372601497\\
2643.94990476618	0.000164447294490232\\
2624.13994276068	0.000166681164382739\\
2604.47840814933	0.000168928525641152\\
2584.96418883051	0.000171188903317682\\
2565.59618103505	0.000173461803816713\\
2546.37328926386	0.000175746714539404\\
2527.29442622594	0.000178043103508825\\
2508.3585127769	0.00018035041897365\\
2489.56447785789	0.000182668088988506\\
2470.91125843504	0.000184995520969227\\
2452.39779943935	0.00018733210122154\\
2434.02305370695	0.000189677194442155\\
2415.78598191995	0.000192030143191833\\
2397.6855525476	0.000194390267340951\\
2379.72074178795	0.000196756863489232\\
2361.89053350997	0.000199129204362966\\
2344.19391919604	0.000201506538195097\\
2326.62989788494	0.000203888088096207\\
2309.19747611519	0.000206273051427769\\
2291.89566786891	0.000208660599193062\\
2274.72349451599	0.00021104987546612\\
2257.67998475881	0.000213439996884793\\
2240.76417457722	0.000215830052240728\\
2223.97510717407	0.000218219102206442\\
2207.31183292107	0.000220606179247731\\
2190.77340930509	0.00022299028777774\\
2174.35890087482	0.000225370404616856\\
2158.06737918789	0.000227745479828993\\
2141.89792275834	0.000230114438008943\\
2125.8496170045	0.000232476180095746\\
2109.92155419728	0.000234829585781922\\
2094.11283340877	0.000237173516576268\\
2078.42256046136	0.000239506819557374\\
2062.8498478771	0.000241828331825013\\
2047.39381482752	0.000244136885617292\\
2032.05358708382	0.000246431314014162\\
2016.82829696743	0.000248710457095662\\
2001.71708330089	0.000250973168370854\\
1986.71909135918	0.000253218321246936\\
1971.83347282137	0.000255444815274376\\
1957.05938572261	0.000257651581889586\\
1942.39599440654	0.000259837589386231\\
1927.84246947799	0.000262001846881599\\
1913.39798775612	0.000264143407103296\\
1899.06173222778	0.000266261367898243\\
1884.83289200137	0.00026835487245155\\
1870.71066226095	0.000270423108287366\\
1856.6942442207	0.00027246530519738\\
1842.78284507977	0.000274480732297901\\
1828.97567797739	0.000276468694449039\\
1815.27196194842	0.000278428528278872\\
1801.67092187916	0.000280359598044144\\
1788.17178846346	0.000282261291531773\\
1774.77379815929	0.00028413301616795\\
1761.47619314548	0.000285974195459686\\
1748.27822127887	0.000287784265852038\\
1735.17913605182	0.000289562674046555\\
1722.17819654992	0.000291308874794813\\
1709.2746674101	0.000293022329156241\\
1696.46781877908	0.000294702503191706\\
1683.756926272	0.000296348867052803\\
1671.14127093155	0.000297960894420601\\
1658.62013918722	0.000299538062245398\\
1646.19282281499	0.000301079850739856\\
1633.85861889724	0.00030258574358077\\
1621.61682978302	0.000304055228278709\\
1609.46676304856	0.000305487796679488\\
1597.40773145812	0.000306882945566198\\
1585.43905292513	0.000308240177335216\\
1573.56005047358	0.000309559000723965\\
1561.77005219976	0.000310838931572055\\
1550.06839123423	0.000312079493600921\\
1538.45440570412	0.000313280219199944\\
1526.92743869569	0.000314440650209608\\
1515.48683821714	0.000315560338694242\\
1504.13195716179	0.000316638847698654\\
1492.86215327143	0.000317675751984332\\
1481.67678910003	0.000318670638741998\\
1470.57523197764	0.000319623108278193\\
1459.55685397464	0.000320532774674266\\
1448.62103186621	0.000321399266416695\\
1437.7671470971	0.000322222226998073\\
1426.99458574659	0.000323001315488409\\
1416.30273849384	0.000323736207076653\\
1405.69100058334	0.000324426593582495\\
1395.1587717908	0.000325072183938644\\
1384.70545638909	0.000325672704643867\\
1374.33046311466	0.000326227900187118\\
1364.03320513401	0.000326737533443143\\
1353.81310001052	0.00032720138603995\\
1343.66956967153	0.000327619258698554\\
1333.60204037562	0.0003279909715454\\
1323.60994268017	0.000328316364397864\\
1313.69271140914	0.000328595297023241\\
1303.84978562109	0.000328827649371586\\
1294.0806085775	0.000329013321782804\\
1284.38462771124	0.000329152235168342\\
1274.7612945953	0.000329244331167862\\
1265.21006491183	0.000329289572281235\\
1255.73039842129	0.000329287941976211\\
1246.32175893192	0.000329239444772114\\
1236.98361426944	0.000329144106299915\\
1227.71543624688	0.000329001973339015\\
1218.51670063476	0.000328813113831108\\
1209.38688713144	0.00032857761687146\\
1200.32547933364	0.000328295592677986\\
1191.3319647073	0.000327967172538467\\
1182.40583455852	0.000327592508736315\\
1173.54658400484	0.000327171774455242\\
1164.75371194667	0.000326705163663245\\
1156.02672103892	0.000326192890976321\\
1147.36511766291	0.000325635191502313\\
1138.76841189842	0.000325032320665329\\
1130.23611749598	0.000324384554011181\\
1121.76775184938	0.000323692186994292\\
1113.36283596838	0.000322955534746541\\
1105.02089445159	0.000322174931828534\\
1096.74145545961	0.000321350731963785\\
1088.5240506883	0.000320483307756318\\
1080.36821534236	0.000319573050392202\\
1072.27348810894	0.000318620369325555\\
1064.23941113167	0.000317625691949541\\
1056.26552998465	0.000316589463252923\\
1048.35139364682	0.000315512145462722\\
1040.49655447641	0.000314394217673538\\
1032.70056818564	0.00031323617546413\\
1024.96299381559	0.00031203853050181\\
1017.28339371124	0.00031080181013525\\
1009.66133349674	0.000309526556976293\\
1002.09638205082	0.00030821332847136\\
994.588111482424	0.000306862696463052\\
987.13609710648	0.00030547524674255\\
979.73991741991	0.000304051578593414\\
972.39915407777	0.000302592304327392\\
965.113391869591	0.000301098048812824\\
957.882218695896	0.000299569448996272\\
950.705225544888	0.000298007153417946\\
943.582006469321	0.000296411821721546\\
936.512158563526	0.000294784124159107\\
929.495281940641	0.000293124741091433\\
922.530979709972	0.000291434362484707\\
915.61885795456	0.000289713687403865\\
908.758525708891	0.000287963423503284\\
901.949594936784	0.000286184286515381\\
895.191680509447	0.000284376999737646\\
888.484400183686	0.00028254229351868\\
881.827374580292	0.000280680904743776\\
875.220227162575	0.000278793576320552\\
868.662584215073	0.00027688105666519\\
862.154074822409	0.000274944099189754\\
855.694330848315	0.000272983461791106\\
849.282986914804	0.000270999906341905\\
842.919680381509	0.000268994198184149\\
836.604051325169	0.000266967105625738\\
830.33574251927	0.000264919399440495\\
824.114399413841	0.000262851852372091\\
817.939670115397	0.000260765238642288\\
811.81120536704	0.000258660333463915\\
805.728658528701	0.000256537912558968\\
799.691685557532	0.000254398751682231\\
793.699944988449	0.000252243626150762\\
787.753097914816	0.000250073310379628\\
781.850807969277	0.000247888577424199\\
775.99274130473	0.000245690198529358\\
770.178566575443	0.00024347894268592\\
764.407954918312	0.000241255576194563\\
758.680579934264	0.000239020862237562\\
752.996117669786	0.000236775560458585\\
747.354246598613	0.000234520426550833\\
741.754647603534	0.000232256211853725\\
736.197003958341	0.000229983662958408\\
730.681001309921	0.00022770352132226\\
725.20632766047	0.00022541652289263\\
719.772673349847	0.000223123397739974\\
714.379731038059	0.000220824869700573\\
709.02719568788	0.000218521656029005\\
703.714764547589	0.000216214467060501\\
698.442137133859	0.000213904005883345\\
693.209015214748	0.00021159096802142\\
688.015102792837	0.000209276041127032\\
682.860106088486	0.000206959904684094\\
677.743733523219	0.000204643229721777\\
672.665695703228	0.000202326678538682\\
667.625705403009	0.000200010904437617\\
662.623477549112	0.000197696551471026\\
657.658729204016	0.000195384254197102\\
652.731179550132	0.000193074637446631\\
647.840549873909	0.000190768316100575\\
642.986563550079	0.00018846589487841\\
638.168946026005	0.000186167968137219\\
633.387424806152	0.00018387511968152\\
628.641729436677	0.000181587922583822\\
623.931591490128	0.000179306939015868\\
619.256744550263	0.000177032720090534\\
614.61692419698	0.000174765805714323\\
610.011867991365	0.000172506724450411\\
605.441315460838	0.000170255993392169\\
600.905008084431	0.000168014118047084\\
596.402689278159	0.000165781592231009\\
591.93410438051	0.000163558897972627\\
587.499000638037	0.000161346505428061\\
583.097127191066	0.000159144872805493\\
578.728235059506	0.000156954446299699\\
574.392077128762	0.000154775660036362\\
570.088408135764	0.000152608936026059\\
565.816984655088	0.00015045468412775\\
561.577565085194	0.000148313302021674\\
557.369909634754	0.000146185175191457\\
553.193780309092	0.00014407067691532\\
549.048940896723	0.000141970168266198\\
544.935156955991	0.000139883998120623\\
540.85219580181	0.000137812503176194\\
536.799826492501	0.000135756007977461\\
532.777819816728	0.000133714824950045\\
528.78594828054	0.000131689254442805\\
524.823986094494	0.000129679584777877\\
520.891709160893	0.000127686092308373\\
516.988895061104	0.000125709041483564\\
513.115323042978	0.000123748684921339\\
509.27077400837	0.000121805263487734\\
505.455030500736	0.000119879006383334\\
501.667876692845	0.000117970131236339\\
497.90909837456	0.00011607884420208\\
494.178482940729	0.000114205340068772\\
490.47581937916	0.000112349802369297\\
486.800898258679	0.000110512403498794\\
483.15351171729	0.000108693304837843\\
479.533453450418	0.000106892656881021\\
475.940518699236	0.000105110599370621\\
472.374504239085	0.000103347261435292\\
468.835208367981	0.00010160276173341\\
465.322430895204	9.98772086009322e-05\\
461.835973129977	9.8170700203533e-05\\
458.375637870226	9.64833246927928e-05\\
454.941229391423	9.48151603662198e-05\\
451.532553435523	9.31662758308943e-05\\
448.149417199969	9.15367301705075e-05\\
444.791629326788	8.99265731155876e-05\\
441.458999891773	8.83358452166933e-05\\
438.151340393733	8.6764578020363e-05\\
434.868463743833	8.52127942476099e-05\\
431.610184255017	8.36805079747518e-05\\
428.376317631498	8.21677248163709e-05\\
425.166680958337	8.06744421101974e-05\\
421.981092691098	7.92006491037182e-05\\
418.819372645578	7.77463271423081e-05\\
415.681341987614	7.63114498586912e-05\\
412.56682322297	7.48959833635381e-05\\
409.475640187299	7.3499886437011e-05\\
406.407618036173	7.21231107210688e-05\\
403.3625832352	7.07656009123528e-05\\
400.340363550203	6.94272949554724e-05\\
397.340788037481	6.81081242365173e-05\\
394.363687034141	6.68080137766264e-05\\
391.408892148499	6.55268824254438e-05\\
388.476236250556	6.4264643054305e-05\\
385.565553462546	6.30212027489897e-05\\
382.676679149551	6.1796463001891e-05\\
379.809449910193	6.05903199034518e-05\\
376.963703567387	5.94026643327214e-05\\
374.13927915917	5.82333821468948e-05\\
371.336016929598	5.70823543696979e-05\\
368.55375831971	5.59494573784886e-05\\
365.792345958554	5.48345630899447e-05\\
363.051623654292	5.37375391442239e-05\\
360.331436385364	5.26582490874712e-05\\
357.631630291718	5.15965525525669e-05\\
354.952052666108	5.05523054380076e-05\\
352.292551945457	4.95253600848154e-05\\
349.652977702284	4.85155654513808e-05\\
347.033180636195	4.75227672861433e-05\\
344.433012565439	4.65468082980241e-05\\
341.852326418525	4.55875283245258e-05\\
339.290976225908	4.46447644974208e-05\\
336.748817111725	4.37183514059532e-05\\
334.225705285606	4.28081212574865e-05\\
331.72149803454	4.19139040355291e-05\\
329.236053714802	4.10355276550794e-05\\
326.769231743941	4.01728181152316e-05\\
324.32089259283	3.93255996489918e-05\\
321.890897777773	3.84936948702558e-05\\
319.479109852668	3.76769249179047e-05\\
317.085392401243	3.68751095969783e-05\\
314.709610029328	3.60880675168912e-05\\
312.351628357206	3.53156162266592e-05\\
310.011314012007	3.45575723471081e-05\\
307.688534620164	3.38137517000398e-05\\
305.383158799932	3.30839694343362e-05\\
303.095056153947	3.23680401489824e-05\\
300.824097261858	3.16657780129951e-05\\
298.570153673005	3.09769968822468e-05\\
296.33309789915	3.03015104131773e-05\\
294.112803407272	2.96391321733872e-05\\
291.909144612403	2.89896757491147e-05\\
289.721996870532	2.83529548495937e-05\\
287.551236471548	2.77287834082992e-05\\
285.396740632247	2.71169756810869e-05\\
283.258387489388	2.6517346341235e-05\\
281.136056092795	2.59297105714003e-05\\
279.02962639852	2.5353884152503e-05\\
276.938979262052	2.47896835495555e-05\\
274.863996431577	2.42369259944543e-05\\
272.804560541293	2.36954295657536e-05\\
270.760555104764	2.31650132654443e-05\\
268.731864508341	2.26454970927613e-05\\
266.718374004613	2.21367021150446e-05\\
264.719969705925	2.16384505356817e-05\\
262.73653857793	2.11505657591589e-05\\
260.767968433199	2.0672872453253e-05\\
258.814147924875	2.02051966083936e-05\\
256.874966540373	1.97473655942284e-05\\
254.950314595131	1.92992082134271e-05\\
253.040083226407	1.88605547527556e-05\\
251.144164387117	1.8431237031461e-05\\
249.262450839729	1.80110884469993e-05\\
247.394836150195	1.75999440181484e-05\\
245.54121468193	1.71976404255425e-05\\
243.701481589837	1.68040160496686e-05\\
241.875532814379	1.64189110063656e-05\\
240.063265075692	1.60421671798685e-05\\
238.264575867741	1.56736282534381e-05\\
236.479363452525	1.53131397376209e-05\\
234.707526854322	1.49605489961831e-05\\
232.948965853976	1.46157052697618e-05\\
231.20358098323	1.42784596972808e-05\\
229.471273519099	1.39486653351761e-05\\
227.751945478287	1.36261771744783e-05\\
226.045499611642	1.33108521557994e-05\\
224.35183939866	1.30025491822736e-05\\
222.67086904202	1.27011291304998e-05\\
221.002493462171	1.2406454859537e-05\\
219.346618291952	1.21183912180042e-05\\
217.70314987125	1.18368050493343e-05\\
216.071995241712	1.15615651952367e-05\\
214.453062141476	1.1292542497421e-05\\
212.846258999961	1.10296097976358e-05\\
211.251494932683	1.07726419360779e-05\\
209.668679736118	1.05215157482273e-05\\
208.097723882594	1.0276110060164e-05\\
206.538538515233	1.00363056824248e-05\\
204.991035442923	9.80198540245624e-06\\
203.455127135329	9.57303397572308e-06\\
201.930726717942	9.34933811553089e-06\\
200.417747967167	9.13078648162169e-06\\
198.916105305443	8.91726966760256e-06\\
197.425713796405	8.70868018726705e-06\\
195.946489140081	8.50491245986946e-06\\
194.478347668119	8.30586279441224e-06\\
193.02120633906	8.11142937300658e-06\\
191.574982733635	7.92151223336609e-06\\
190.13959505011	7.73601325049353e-06\\
188.714962099654	7.5548361176192e-06\\
187.301003301749	7.37788632645057e-06\\
185.897638679631	7.20507114679016e-06\\
184.504788855768	7.03629960557957e-06\\
183.122375047369	6.87148246542505e-06\\
181.750319061926	6.71053220266024e-06\\
180.388543292795	6.55336298499924e-06\\
179.036970714807	6.39989064883255e-06\\
177.695524879903	6.25003267621645e-06\\
176.364129912823	6.10370817160514e-06\\
175.042710506801	5.96083783837247e-06\\
173.731191919316	5.82134395516913e-06\\
172.429499967858	5.68515035215843e-06\\
171.137561025735	5.55218238717216e-06\\
169.855302017906	5.42236692182575e-06\\
168.582650416852	5.29563229762978e-06\\
167.319534238469	5.17190831213273e-06\\
166.065882037999	5.05112619512761e-06\\
164.821622905988	4.93321858495272e-06\\
163.586686464278	4.81811950491461e-06\\
162.36100286202	4.7057643398592e-06\\
161.144502771729	4.59608981291454e-06\\
159.937117385361	4.48903396242666e-06\\
158.73877841042	4.38453611910762e-06\\
157.549418066095	4.2825368834133e-06\\
156.368969079428	4.18297810316563e-06\\
155.197364681509	4.08580285143281e-06\\
154.034538603695	3.99095540467868e-06\\
152.880425073868	3.89838122119112e-06\\
151.734958812712	3.80802691979689e-06\\
150.598075030017	3.71984025886986e-06\\
149.469709421021	3.63377011563714e-06\\
148.349798162768	3.54976646578674e-06\\
147.2382779105	3.4677803633791e-06\\
146.135085794075	3.38776392106352e-06\\
145.040159414406	3.30967029059976e-06\\
143.953436839938	3.23345364368369e-06\\
142.874856603141	3.15906915307563e-06\\
141.804357697036	3.08647297402876e-06\\
140.74187957174	3.01562222601456e-06\\
139.687362131045	2.94647497474177e-06\\
138.640745729018	2.87899021446476e-06\\
137.601971166627	2.8131278505768e-06\\
136.570979688393	2.74884868248353e-06\\
135.547712979064	2.68611438675158e-06\\
134.53211316032	2.6248875005272e-06\\
133.524122787498	2.56513140521955e-06\\
132.523684846345	2.50681031044337e-06\\
131.530742749786	2.44988923821555e-06\\
130.545240334732	2.39433400740037e-06\\
129.567121858898	2.34011121839799e-06\\
128.596331997653	2.28718823807113e-06\\
127.632815840887	2.23553318490484e-06\\
126.676518889911	2.18511491439441e-06\\
125.727387054366	2.13590300465668e-06\\
124.785366649173	2.08786774226026e-06\\
123.850404391489	2.04098010827023e-06\\
122.922447397698	1.99521176450314e-06\\
122.001443180416	1.95053503998841e-06\\
121.087339645525	1.90692291763231e-06\\
120.180085089226	1.86434902108107e-06\\
119.279628195113	1.8227876017797e-06\\
118.385918031273	1.78221352622349e-06\\
117.4989040474	1.74260226339914e-06\\
116.618536071944	1.70392987241293e-06\\
115.744764309265	1.66617299030321e-06\\
114.877539336821	1.62930882003507e-06\\
114.016812102373	1.59331511867467e-06\\
113.162533921208	1.55817018574147e-06\\
112.314656473384	1.52385285173626e-06\\
111.473131801004	1.49034246684325e-06\\
}--cycle;
\addplot[ybar interval, fill=black, fill opacity=0.35, area legend, draw=none] table[row sep=crcr, x=Lower, y=Count] {%
Lower	Upper	Count\\
111.473131801004	156.850807223597	0.000117532096645874\\
156.850807223597	220.700498220617	0.000250588526743956\\
220.700498220617	310.541659153798	0.000207774103459658\\
310.541659153798	436.954709425234	0.000189853816108914\\
436.954709425234	614.827068964461	0.000344816549868768\\
614.827068964461	865.10642082028	0.000351607111603417\\
865.10642082028	1217.26767919475	0.000401331293469678\\
1217.26767919475	1712.78419296345	0.000247553135482231\\
1712.78419296345	2410.01198159317	0.000175934850370416\\
2410.01198159317	3391.06221045479	0.000133190598016871\\
3391.06221045479	4771.47126362945	6.76128087675789e-05\\
4771.47126362945	6713.807239941	3.15770979281371e-05\\
6713.807239941	9446.8152828814	1.46358881391965e-05\\
9446.8152828814	13292.3564528293	7.62788175629675e-06\\
13292.3564528293	18703.3126803322	2.46413624001655e-06\\
18703.3126803322	26316.9217933377	1.05075002948792e-06\\
26316.9217933377	37029.8237811609	7.46763109481746e-07\\
37029.8237811609	52103.6563482497	5.30721033578858e-07\\
52103.6563482497	73313.6355414593	1.25726981736993e-07\\
73313.6355414593	103157.619503348	0\\
103157.619503348	145150.276384529	6.35031661419285e-08\\
145150.276384529	204237	4.51314018360709e-08\\
204237	204237	4.51314018360709e-08\\
};
\addlegendentry{Istogramma reale}

\addplot [color=black]
table[row sep=crcr]{%
111.473131801004	5.38377661953373e-05\\
112.314656473384	5.46865853473044e-05\\
113.162533921208	5.55452268613916e-05\\
114.016812102373	5.64137339988677e-05\\
114.877539336821	5.72921487863371e-05\\
115.744764309265	5.81805119855834e-05\\
116.618536071944	5.9078863063351e-05\\
117.4989040474	5.99872401610765e-05\\
118.385918031273	6.09056800645806e-05\\
119.279628195113	6.18342181737318e-05\\
120.180085089226	6.27728884720918e-05\\
121.087339645525	6.37217234965575e-05\\
122.001443180416	6.46807543070064e-05\\
122.922447397698	6.56500104559629e-05\\
123.850404391489	6.66295199582925e-05\\
124.785366649173	6.76193092609386e-05\\
125.727387054366	6.8619403212715e-05\\
126.676518889911	6.96298250341647e-05\\
127.632815840887	7.06505962874984e-05\\
128.596331997653	7.16817368466261e-05\\
129.567121858898	7.27232648672947e-05\\
130.545240334732	7.37751967573435e-05\\
131.530742749786	7.48375471470926e-05\\
132.523684846345	7.5910328859876e-05\\
133.524122787498	7.6993552882735e-05\\
134.53211316032	7.80872283372832e-05\\
135.547712979064	7.91913624507575e-05\\
136.570979688393	8.03059605272717e-05\\
137.601971166627	8.14310259192815e-05\\
138.640745729018	8.25665599992806e-05\\
139.687362131045	8.3712562131737e-05\\
140.74187957174	8.48690296452883e-05\\
141.804357697036	8.60359578052073e-05\\
142.874856603141	8.72133397861529e-05\\
143.953436839938	8.84011666452224e-05\\
145.040159414406	8.95994272953166e-05\\
146.135085794075	9.08081084788372e-05\\
147.2382779105	9.20271947417258e-05\\
148.349798162768	9.3256668407863e-05\\
149.469709421021	9.44965095538405e-05\\
150.598075030017	9.57466959841217e-05\\
151.734958812712	9.7007203206606e-05\\
152.880425073868	9.82780044086105e-05\\
154.034538603695	9.95590704332852e-05\\
155.197364681509	0.000100850369756475\\
156.368969079428	0.000102151868464044\\
157.549418066095	0.00010346353022968\\
158.73877841042	0.000104785316293189\\
159.937117385361	0.000106117185439292\\
161.144502771729	0.000107459093976955\\
162.36100286202	0.000108810995719243\\
163.586686464278	0.000110172841963731\\
164.821622905988	0.000111544581473484\\
166.065882037999	0.000112926160458607\\
167.319534238469	0.000114317522558395\\
168.582650416852	0.000115718608824099\\
169.855302017906	0.000117129357702296\\
171.137561025735	0.000118549705018917\\
172.429499967858	0.000119979583963913\\
173.731191919316	0.000121418925076585\\
175.042710506801	0.000122867656231605\\
176.364129912823	0.000124325702625712\\
177.695524879903	0.000125792986765129\\
179.036970714807	0.000127269428453687\\
180.388543292795	0.000128754944781691\\
181.750319061926	0.000130249450115523\\
183.122375047369	0.000131752856088006\\
184.504788855768	0.000133265071589539\\
185.897638679631	0.000134786002760014\\
187.301003301749	0.000136315552981524\\
188.714962099654	0.000137853622871888\\
190.13959505011	0.00013940011027898\\
191.574982733635	0.0001409549102759\\
193.02120633906	0.000142517915156981\\
194.478347668119	0.000144089014434644\\
195.946489140081	0.00014566809483713\\
197.425713796405	0.000147255040307084\\
198.916105305443	0.000148849732001043\\
200.417747967167	0.000150452048289804\\
201.930726717942	0.000152061864759697\\
203.455127135329	0.000153679054214777\\
204.991035442923	0.00015530348667993\\
206.538538515233	0.000156935029404906\\
208.097723882594	0.000158573546869303\\
209.668679736118	0.000160218900788482\\
211.251494932683	0.000161870950120443\\
212.846258999961	0.000163529551073665\\
214.453062141476	0.000165194557115904\\
216.071995241712	0.000166865818983975\\
217.70314987125	0.000168543184694509\\
219.346618291952	0.000170226499555698\\
221.002493462171	0.00017191560618003\\
222.67086904202	0.000173610344498026\\
224.35183939866	0.000175310551772969\\
226.045499611642	0.000177016062616648\\
227.751945478287	0.000178726709006108\\
229.471273519099	0.000180442320301407\\
231.20358098323	0.000182162723264401\\
232.948965853976	0.000183887742078532\\
234.707526854322	0.000185617198369653\\
236.479363452525	0.000187350911227857\\
238.264575867741	0.000189088697230348\\
240.063265075692	0.000190830370465317\\
241.875532814379	0.000192575742556859\\
243.701481589837	0.000194324622690902\\
245.54121468193	0.000196076817642167\\
247.394836150195	0.000197832131802146\\
249.262450839729	0.0001995903672081\\
251.144164387117	0.000201351323573083\\
253.040083226407	0.000203114798316971\\
254.950314595131	0.000204880586598517\\
256.874966540373	0.000206648481348403\\
258.814147924875	0.000208418273303307\\
260.767968433199	0.000210189751040966\\
262.73653857793	0.000211962701016241\\
264.719969705925	0.000213736907598166\\
266.718374004613	0.000215512153107987\\
268.731864508341	0.000217288217858177\\
270.760555104764	0.000219064880192425\\
272.804560541293	0.000220841916526585\\
274.863996431577	0.000222619101390588\\
276.938979262052	0.000224396207471295\\
279.02962639852	0.000226173005656293\\
281.136056092795	0.000227949265078619\\
283.258387489388	0.000229724753162403\\
285.396740632247	0.000231499235669427\\
287.551236471548	0.000233272476746568\\
289.721996870532	0.000235044238974146\\
291.909144612403	0.000236814283415129\\
294.112803407272	0.000238582369665218\\
296.33309789915	0.000240348255903769\\
298.570153673005	0.000242111698945557\\
300.824097261858	0.000243872454293365\\
303.095056153947	0.000245630276191373\\
305.383158799932	0.000247384917679354\\
307.688534620164	0.000249136130647636\\
310.011314012007	0.000250883665892836\\
312.351628357206	0.000252627273174336\\
314.709610029328	0.000254366701271488\\
317.085392401243	0.000256101698041539\\
319.479109852668	0.000257832010478248\\
321.890897777773	0.000259557384771177\\
324.32089259283	0.000261277566365649\\
326.769231743941	0.000262992300023341\\
329.236053714802	0.000264701329883502\\
331.72149803454	0.000266404399524776\\
334.225705285606	0.000268101252027596\\
336.748817111725	0.000269791630037156\\
339.290976225908	0.000271475275826908\\
341.852326418525	0.000273151931362595\\
344.433012565439	0.000274821338366765\\
347.033180636195	0.00027648323838378\\
349.652977702284	0.000278137372845266\\
352.292551945457	0.000279783483135998\\
354.952052666108	0.000281421310660202\\
357.631630291718	0.00028305059690823\\
360.331436385364	0.000284671083523611\\
363.051623654292	0.000286282512370426\\
365.792345958554	0.000287884625601014\\
368.55375831971	0.000289477165723955\\
371.336016929598	0.000291059875672322\\
374.13927915917	0.000292632498872183\\
376.963703567387	0.0002941947793113\\
379.809449910193	0.000295746461608038\\
382.676679149551	0.000297287291080424\\
385.565553462546	0.000298817013815353\\
388.476236250556	0.000300335376737904\\
391.408892148499	0.000301842127680742\\
394.363687034141	0.000303337015453582\\
397.340788037481	0.000304819789912685\\
400.340363550203	0.00030629020203036\\
403.3625832352	0.000307748003964453\\
406.407618036173	0.000309192949127779\\
409.475640187299	0.000310624792257489\\
412.56682322297	0.000312043289484331\\
415.681341987614	0.000313448198401786\\
418.819372645578	0.000314839278135045\\
421.981092691098	0.000316216289409807\\
425.166680958337	0.000317578994620865\\
428.376317631498	0.000318927157900456\\
431.610184255017	0.000320260545186343\\
434.868463743833	0.000321578924289606\\
438.151340393733	0.000322882064962117\\
441.458999891773	0.000324169738963661\\
444.791629326788	0.000325441720128688\\
448.149417199969	0.000326697784432661\\
451.532553435523	0.000327937710057975\\
454.941229391423	0.000329161277459421\\
458.375637870226	0.000330368269429166\\
461.835973129977	0.000331558471161221\\
465.322430895204	0.000332731670315377\\
468.835208367981	0.000333887657080575\\
472.374504239085	0.000335026224237685\\
475.940518699236	0.000336147167221669\\
479.533453450418	0.000337250284183106\\
483.15351171729	0.000338335376049042\\
486.800898258679	0.000339402246583151\\
490.47581937916	0.000340450702445178\\
494.178482940729	0.000341480553249626\\
497.90909837456	0.000342491611623695\\
501.667876692845	0.000343483693264403\\
505.455030500736	0.000344456616994903\\
509.27077400837	0.000345410204819956\\
513.115323042978	0.00034634428198053\\
516.988895061104	0.000347258677007517\\
520.891709160893	0.000348153221774533\\
524.823986094494	0.000349027751549782\\
528.78594828054	0.00034988210504696\\
532.777819816728	0.000350716124475185\\
536.799826492501	0.000351529655587915\\
540.85219580181	0.00035232254773085\\
544.935156955991	0.000353094653888788\\
549.048940896723	0.000353845830731414\\
553.193780309092	0.000354575938658006\\
557.369909634754	0.000355284841841033\\
561.577565085194	0.000355972408268638\\
565.816984655088	0.000356638509785969\\
570.088408135764	0.000357283022135356\\
574.392077128762	0.000357905824995302\\
578.728235059506	0.000358506802018295\\
583.097127191066	0.000359085840867385\\
587.499000638037	0.000359642833251557\\
591.93410438051	0.000360177674959838\\
596.402689278159	0.000360690265894164\\
600.905008084431	0.000361180510100958\\
605.441315460838	0.000361648315801434\\
610.011867991365	0.00036209359542059\\
614.61692419698	0.000362516265614889\\
619.256744550263	0.000362916247298623\\
623.931591490128	0.000363293465668923\\
628.641729436677	0.000363647850229436\\
633.387424806152	0.000363979334812632\\
638.168946026005	0.000364287857600745\\
642.986563550079	0.000364573361145336\\
647.840549873909	0.000364835792385467\\
652.731179550132	0.000365075102664481\\
657.658729204016	0.000365291247745375\\
662.623477549112	0.00036548418782477\\
667.625705403009	0.000365653887545454\\
672.665695703228	0.000365800316007519\\
677.743733523219	0.000365923446778051\\
682.860106088486	0.000366023257899409\\
688.015102792837	0.000366099731896054\\
693.209015214748	0.000366152855779948\\
698.442137133859	0.000366182621054511\\
703.714764547589	0.000366189023717136\\
709.02719568788	0.000366172064260259\\
714.379731038059	0.000366131747670988\\
719.772673349847	0.00036606808342929\\
725.20632766047	0.00036598108550473\\
730.681001309921	0.000365870772351774\\
736.197003958341	0.000365737166903652\\
741.754647603534	0.000365580296564783\\
747.354246598613	0.00036540019320177\\
752.996117669786	0.000365196893132966\\
758.680579934264	0.000364970437116614\\
764.407954918312	0.00036472087033757\\
770.178566575443	0.000364448242392617\\
775.99274130473	0.000364152607274365\\
781.850807969277	0.000363834023353761\\
787.753097914816	0.000363492553361205\\
793.699944988449	0.000363128264366279\\
799.691685557532	0.000362741227756112\\
805.728658528701	0.000362331519212376\\
811.81120536704	0.000361899218686927\\
817.939670115397	0.000361444410376111\\
824.114399413841	0.000360967182693732\\
830.33574251927	0.000360467628242707\\
836.604051325169	0.000359945843785411\\
842.919680381509	0.000359401930212732\\
849.282986914804	0.000358835992511845\\
855.694330848315	0.000358248139732725\\
862.154074822409	0.000357638484953408\\
868.662584215073	0.000357007145244014\\
875.220227162575	0.000356354241629563\\
881.827374580292	0.000355679899051575\\
888.484400183686	0.000354984246328498\\
895.191680509447	0.000354267416114961\\
901.949594936784	0.00035352954485988\\
908.758525708891	0.000352770772763433\\
915.61885795456	0.000351991243732926\\
922.530979709972	0.000351191105337561\\
929.495281940641	0.000350370508762138\\
936.512158563526	0.000349529608759698\\
943.582006469321	0.00034866856360314\\
950.705225544888	0.000347787535035824\\
957.882218695896	0.000346886688221188\\
965.113391869591	0.000345966191691396\\
972.39915407777	0.00034502621729504\\
979.73991741991	0.000344066940143924\\
987.13609710648	0.000343088538558945\\
994.588111482424	0.000342091194015102\\
1002.09638205082	0.000341075091085649\\
1009.66133349674	0.000340040417385422\\
1017.28339371124	0.000338987363513363\\
1024.96299381559	0.000337916122994265\\
1032.70056818564	0.000336826892219761\\
1040.49655447641	0.000335719870388584\\
1048.35139364682	0.000334595259446125\\
1056.26552998465	0.000333453264023313\\
1064.23941113167	0.000332294091374836\\
1072.27348810894	0.00033111795131675\\
1080.36821534236	0.000329925056163471\\
1088.5240506883	0.000328715620664204\\
1096.74145545961	0.000327489861938823\\
1105.02089445159	0.00032624799941323\\
1113.36283596838	0.000324990254754216\\
1121.76775184938	0.000323716851803864\\
1130.23611749598	0.000322428016513504\\
1138.76841189842	0.000321123976877262\\
1147.36511766291	0.000319804962865217\\
1156.02672103892	0.00031847120635621\\
1164.75371194667	0.000317122941070308\\
1173.54658400484	0.000315760402500981\\
1182.40583455852	0.000314383827846991\\
1191.3319647073	0.000312993455944036\\
1200.32547933364	0.000311589527196179\\
1209.38688713144	0.000310172283507071\\
1218.51670063476	0.00030874196821102\\
1227.71543624688	0.000307298826003913\\
1236.98361426944	0.00030584310287403\\
1246.32175893192	0.000304375046032774\\
1255.73039842129	0.000302894903845337\\
1265.21006491183	0.000301402925761344\\
1274.7612945953	0.000299899362245489\\
1284.38462771124	0.000298384464708191\\
1294.0806085775	0.000296858485436305\\
1303.84978562109	0.000295321677523905\\
1313.69271140914	0.000293774294803173\\
1323.60994268017	0.000292216591775408\\
1333.60204037562	0.000290648823542201\\
1343.66956967153	0.000289071245736778\\
1353.81310001052	0.000287484114455557\\
1364.03320513401	0.000285887686189927\\
1374.33046311466	0.00028428221775829\\
1384.70545638909	0.00028266796623838\\
1395.1587717908	0.000281045188899882\\
1405.69100058334	0.000279414143137388\\
1416.30273849384	0.000277775086403698\\
1426.99458574659	0.0002761282761435\\
1437.7671470971	0.000274473969727451\\
1448.62103186621	0.000272812424386677\\
1459.55685397464	0.000271143897147716\\
1470.57523197764	0.000269468644767939\\
1481.67678910003	0.000267786923671443\\
1492.86215327143	0.000266098989885474\\
1504.13195716179	0.000264405098977361\\
1515.48683821714	0.000262705505992023\\
1526.92743869569	0.000261000465390035\\
1538.45440570412	0.000259290230986293\\
1550.06839123423	0.00025757505588929\\
1561.77005219976	0.000255855192441029\\
1573.56005047358	0.000254130892157579\\
1585.43905292513	0.000252402405670306\\
1597.40773145812	0.000250669982667795\\
1609.46676304856	0.000248933871838465\\
1621.61682978302	0.000247194320813919\\
1633.85861889724	0.00024545157611303\\
1646.19282281499	0.000243705883086773\\
1658.62013918722	0.000241957485863842\\
1671.14127093155	0.000240206627297044\\
1683.756926272	0.000238453548910499\\
1696.46781877908	0.00023669849084765\\
1709.2746674101	0.00023494169182011\\
1722.17819654992	0.000233183389057348\\
1735.17913605182	0.00023142381825723\\
1748.27822127887	0.000229663213537432\\
1761.47619314548	0.000227901807387732\\
1774.77379815929	0.000226139830623193\\
1788.17178846346	0.000224377512338252\\
1801.67092187916	0.000222615079861718\\
1815.27196194842	0.000220852758712702\\
1828.97567797739	0.000219090772557472\\
1842.78284507977	0.000217329343167251\\
1856.6942442207	0.000215568690376976\\
1870.71066226095	0.000213809032045005\\
1884.83289200137	0.000212050584013792\\
1899.06173222778	0.000210293560071546\\
1913.39798775612	0.000208538171914855\\
1927.84246947799	0.000206784629112308\\
1942.39599440654	0.000205033139069108\\
1957.05938572261	0.000203283906992674\\
1971.83347282137	0.000201537135859255\\
1986.71909135918	0.000199793026381546\\
2001.71708330089	0.000198051776977306\\
2016.82829696743	0.000196313583739007\\
2032.05358708382	0.000194578640404478\\
2047.39381482752	0.000192847138328579\\
2062.8498478771	0.000191119266455894\\
2078.42256046136	0.000189395211294438\\
2094.11283340877	0.00018767515689039\\
2109.92155419728	0.000185959284803851\\
2125.8496170045	0.000184247774085614\\
2141.89792275834	0.000182540801254965\\
2158.06737918789	0.000180838540278497\\
2174.35890087482	0.000179141162549947\\
2190.77340930509	0.000177448836871047\\
2207.31183292107	0.000175761729433388\\
2223.97510717407	0.000174080003801298\\
2240.76417457722	0.00017240382089573\\
2257.67998475881	0.000170733338979146\\
2274.72349451599	0.000169068713641413\\
2291.89566786891	0.000167410097786681\\
2309.19747611519	0.000165757641621269\\
2326.62989788494	0.000164111492642521\\
2344.19391919604	0.000162471795628651\\
2361.89053350997	0.000160838692629558\\
2379.72074178795	0.000159212322958613\\
2397.6855525476	0.000157592823185403\\
2415.78598191995	0.000155980327129427\\
2434.02305370695	0.000154374965854745\\
2452.39779943935	0.00015277686766556\\
2470.91125843504	0.000151186158102731\\
2489.56447785789	0.000149602959941207\\
2508.3585127769	0.000148027393188377\\
2527.29442622594	0.000146459575083321\\
2546.37328926386	0.000144899620096957\\
2565.59618103505	0.000143347639933066\\
2584.96418883051	0.000141803743530203\\
2604.47840814933	0.000140268037064464\\
2624.13994276068	0.000138740623953105\\
2643.94990476618	0.000137221604859011\\
2663.90941466289	0.000135711077695988\\
2684.01960140661	0.000134209137634886\\
2704.28160247578	0.000132715877110516\\
2724.69656393583	0.000131231385829377\\
2745.26564050395	0.000129755750778159\\
2765.98999561444	0.000128289056233024\\
2786.87080148452	0.000126831383769645\\
2807.90923918064	0.000125382812273987\\
2829.10649868523	0.000123943417953831\\
2850.46377896409	0.000122513274351016\\
2871.98228803412	0.000121092452354382\\
2893.66324303174	0.000119681020213421\\
2915.50787028164	0.000118279043552599\\
2937.51740536621	0.000116886585386346\\
2959.69309319542	0.000115503706134709\\
2982.03618807719	0.00011413046363963\\
3004.5479537884	0.000112766913181862\\
3027.22966364631	0.000111413107498495\\
3050.08260058064	0.00011006909680107\\
3073.10805720609	0.000108734928794288\\
3096.30733589548	0.000107410648695278\\
3119.68174885339	0.000106096299253427\\
3143.23261819043	0.000104791920770744\\
3166.96127599795	0.00010349755112275\\
3190.86906442344	0.000102213225779877\\
3214.95733574645	0.000100938977829368\\
3239.22745245503	9.9674837997651e-05\\
3263.68078732284	9.8420834673181e-05\\
3288.31872348677	9.71769939297365e-05\\
3313.14265452519	9.59433395501459e-05\\
3338.15398453677	9.47198930504423e-05\\
3363.3541282199	9.35066737044192e-05\\
3388.74451095268	9.23036985685816e-05\\
3414.32656887361	9.11109825074727e-05\\
3440.10174896275	8.99285382193638e-05\\
3466.07150912361	8.8756376262291e-05\\
3492.23731826558	8.75945050804249e-05\\
3518.60065638705	8.64429310307607e-05\\
3545.16301465908	8.53016584101077e-05\\
3571.9258955098	8.41706894823739e-05\\
3598.89081270933	8.30500245061193e-05\\
3626.05929145545	8.19396617623758e-05\\
3653.43286845983	8.08395975827076e-05\\
3681.01309203499	7.97498263775079e-05\\
3708.80152218184	7.86703406645114e-05\\
3736.79973067795	7.76011310975111e-05\\
3765.00930116641	7.6542186495266e-05\\
3793.43182924546	7.54934938705826e-05\\
3822.06892255869	7.44550384595632e-05\\
3850.92220088601	7.34268037509982e-05\\
3879.99329623524	7.24087715158985e-05\\
3909.28385293442	7.1400921837147e-05\\
3938.79552772486	7.04032331392613e-05\\
3968.5299898548	6.94156822182515e-05\\
3998.48892117385	6.84382442715617e-05\\
4028.67401622809	6.74708929280825e-05\\
4059.08698235599	6.651360027822e-05\\
4089.72953978488	6.5566336904013e-05\\
4120.60342172835	6.46290719092809e-05\\
4151.71037448419	6.37017729497952e-05\\
4183.05215753326	6.27844062634589e-05\\
4214.63054363891	6.18769367004843e-05\\
4246.44731894735	6.09793277535569e-05\\
4278.50428308861	6.00915415879745e-05\\
4310.80324927834	5.92135390717488e-05\\
4343.34604442043	5.8345279805661e-05\\
4376.13450921022	5.74867221532587e-05\\
4409.17049823878	5.66378232707829e-05\\
4442.45588009764	5.57985391370186e-05\\
4475.99253748463	5.49688245830522e-05\\
4509.78236731024	5.41486333219335e-05\\
4543.82728080503	5.33379179782252e-05\\
4578.12920362765	5.25366301174347e-05\\
4612.69007597378	5.17447202753182e-05\\
4647.51185268592	5.09621379870467e-05\\
4682.59650336387	5.01888318162254e-05\\
4717.94601247624	4.94247493837574e-05\\
4753.5623794726	4.86698373965454e-05\\
4789.44761889663	4.79240416760181e-05\\
4825.60376050007	4.71873071864796e-05\\
4862.03284935749	4.64595780632674e-05\\
4898.73694598198	4.57407976407164e-05\\
4935.71812644172	4.50309084799184e-05\\
4972.9784824774	4.43298523962704e-05\\
5010.52012162047	4.36375704868061e-05\\
5048.34516731247	4.29540031573009e-05\\
5086.45575902501	4.2279090149147e-05\\
5124.85405238087	4.16127705659887e-05\\
5163.54221927589	4.09549829001158e-05\\
5202.52244800183	4.03056650586041e-05\\
5241.79694337011	3.96647543892021e-05\\
5281.3679268366	3.90321877059546e-05\\
5321.23763662717	3.84079013145601e-05\\
5361.40832786437	3.77918310374555e-05\\
5401.88227269494	3.71839122386239e-05\\
5442.66176041833	3.65840798481206e-05\\
5483.74909761624	3.5992268386312e-05\\
5525.14660828299	3.54084119878249e-05\\
5566.85663395707	3.48324444251989e-05\\
5608.88153385351	3.42642991322427e-05\\
5651.22368499737	3.37039092270849e-05\\
5693.88548235815	3.31512075349207e-05\\
5736.8693389853	3.26061266104487e-05\\
5780.17768614467	3.2068598759994e-05\\
5823.81297345603	3.1538556063318e-05\\
5867.77766903166	3.10159303951072e-05\\
5912.07425961591	3.0500653446144e-05\\
5956.7052507259	2.99926567441516e-05\\
6001.67316679318	2.94918716743162e-05\\
6046.98055130657	2.89982294994796e-05\\
6092.62996695602	2.8511661380004e-05\\
6138.62399577752	2.80320983933056e-05\\
6184.9652392992	2.7559471553056e-05\\
6231.65631868842	2.709371182805e-05\\
6278.6998749001	2.66347501607389e-05\\
6326.09856882603	2.61825174854282e-05\\
6373.85508144543	2.57369447461391e-05\\
6421.97211397655	2.5297962914133e-05\\
6470.45238802948	2.48655030050984e-05\\
6519.29864576009	2.44394960960007e-05\\
6568.51365002513	2.40198733415939e-05\\
6618.10018453851	2.36065659905941e-05\\
6668.06105402872	2.31995054015166e-05\\
6718.39908439753	2.27986230581748e-05\\
6769.11712287976	2.24038505848428e-05\\
6820.21803820442	2.20151197610809e-05\\
6871.70472075685	2.16323625362279e-05\\
6923.58008274232	2.1255511043556e-05\\
6975.84705835067	2.08844976140948e-05\\
7028.50860392234	2.0519254790121e-05\\
7081.56769811553	2.01597153383184e-05\\
7135.02734207472	1.98058122626072e-05\\
7188.8905596004	1.94574788166461e-05\\
7243.16039732009	1.9114648516007e-05\\
7297.83992486074	1.87772551500252e-05\\
7352.93223502222	1.84452327933274e-05\\
7408.44044395243	1.8118515817037e-05\\
7464.36769132336	1.7797038899663e-05\\
7520.71714050886	1.74807370376698e-05\\
7577.49197876344	1.71695455557341e-05\\
7634.69541740261	1.68634001166888e-05\\
7692.33069198451	1.65622367311583e-05\\
7750.40106249289	1.62659917668857e-05\\
7808.90981352158	1.59746019577561e-05\\
7867.86025446014	1.56880044125182e-05\\
7927.25571968122	1.54061366232068e-05\\
7987.09956872899	1.51289364732694e-05\\
8047.39518650928	1.48563422453995e-05\\
8108.14598348099	1.45882926290801e-05\\
8169.35539584901	1.432472672784e-05\\
8231.02688575857	1.40655840662264e-05\\
8293.16394149105	1.38108045964964e-05\\
8355.77007766133	1.3560328705032e-05\\
8418.84883541654	1.33140972184803e-05\\
8482.40378263641	1.3072051409623e-05\\
8546.43851413497	1.28341330029797e-05\\
8610.95665186405	1.2600284180146e-05\\
8675.96184511796	1.23704475848724e-05\\
8741.45777074004	1.21445663278861e-05\\
8807.44813333057	1.192258399146e-05\\
8873.93666545632	1.17044446337319e-05\\
8940.9271278617	1.14900927927781e-05\\
9008.42330968139	1.12794734904455e-05\\
9076.42902865481	1.10725322359442e-05\\
9144.94813134188	1.08692150292067e-05\\
9213.98449334078	1.06694683640158e-05\\
9283.54201950699	1.04732392309054e-05\\
9353.62464417428	1.02804751198387e-05\\
9424.23633137717	1.00911240226665e-05\\
9495.38107507519	9.90513443537053e-06\\
9567.0628993788	9.72245536009477e-06\\
9639.28585877689	9.5430363069696e-06\\
9712.05403836631	9.36682729573158e-06\\
9785.37155408272	9.19377885714383e-06\\
9859.24255293356	9.02384203421988e-06\\
9933.67121323252	8.85696838325593e-06\\
10008.6617448359	8.69310997467432e-06\\
10084.2183893808	8.53221939368312e-06\\
10160.3454205248	8.37424974075502e-06\\
10237.047144188	8.2191546319296e-06\\
10314.3278987966	8.06688819894307e-06\\
10392.1920555277	7.91740508918917e-06\\
10470.6440185573	7.77066046551491e-06\\
10549.6882253089	7.62661000585517e-06\\
10629.3291467048	7.48520990271028e-06\\
10709.5712874188	7.34641686246957e-06\\
10790.419186131	7.21018810458571e-06\\
10871.8774157846	7.07648136060272e-06\\
10953.9505838446	6.94525487304181e-06\\
11036.6433325583	6.81646739414879e-06\\
11119.9603392177	6.6900781845068e-06\\
11203.9063164246	6.56604701151764e-06\\
11288.4860123565	6.44433414775569e-06\\
11373.7042110358	6.32490036919824e-06\\
11459.5657325999	6.20770695333507e-06\\
11546.075433574	6.0927156771618e-06\\
11633.2382071459	5.97988881505975e-06\\
11721.0589834426	5.86918913656611e-06\\
11809.5427298094	5.76057990403807e-06\\
11898.6944510906	5.65402487021404e-06\\
11988.5191899127	5.54948827567547e-06\\
12079.0220269697	5.4469348462125e-06\\
12170.2080813103	5.34632979009711e-06\\
12262.0825106276	5.24763879526671e-06\\
12354.6505115508	5.15082802642145e-06\\
12447.9173199389	5.05586412203862e-06\\
12541.8882111773	4.96271419130715e-06\\
12636.568500476	4.8713458109854e-06\\
12731.9635431699	4.7817270221854e-06\\
12828.0787350221	4.69382632708636e-06\\
12924.9195125292	4.60761268558058e-06\\
13022.4913532284	4.5230555118548e-06\\
13120.7997760075	4.44012467090982e-06\\
13219.8503414172	4.35879047502112e-06\\
13319.6486519853	4.2790236801435e-06\\
13420.200352534	4.20079548226246e-06\\
13521.511130499	4.12407751369498e-06\\
13623.5867162509	4.04884183934259e-06\\
13726.4328834197	3.97506095289903e-06\\
13830.0554492214	3.90270777301539e-06\\
13934.4602747868	3.8317556394252e-06\\
14039.6532654931	3.76217830903199e-06\\
14145.6403712978	3.69394995196163e-06\\
14252.4275870757	3.62704514758209e-06\\
14360.0209529573	3.56143888049282e-06\\
14468.4265546712	3.49710653648611e-06\\
14577.6505238876	3.43402389848286e-06\\
14687.6990385654	3.37216714244468e-06\\
14798.5783233021	3.31151283326468e-06\\
14910.2946496851	3.25203792063921e-06\\
15022.8543366469	3.19371973492239e-06\\
15136.2637508225	3.13653598296564e-06\\
15250.5293069092	3.08046474394408e-06\\
15365.65746803	3.02548446517189e-06\\
15481.6547460986	2.97157395790848e-06\\
15598.527702188	2.91871239315729e-06\\
15716.2829469014	2.86687929745903e-06\\
15834.9271407466	2.81605454868125e-06\\
15954.4669945122	2.7662183718058e-06\\
16074.9092696475	2.71735133471604e-06\\
16196.2607786446	2.66943434398524e-06\\
16318.5283854242	2.62244864066794e-06\\
16441.7190057235	2.5763757960957e-06\\
16565.8396074876	2.5311977076788e-06\\
16690.8972112632	2.48689659471544e-06\\
16816.8988905963	2.44345499420967e-06\\
16943.8517724318	2.40085575669952e-06\\
17071.7630375169	2.35908204209676e-06\\
17200.639920807	2.31811731553954e-06\\
17330.4897118753	2.27794534325904e-06\\
17461.3197553249	2.23855018846146e-06\\
17593.1374512039	2.19991620722666e-06\\
17725.9502554248	2.16202804442423e-06\\
17859.7656801852	2.12487062964854e-06\\
17994.5912943937	2.0884291731734e-06\\
18130.4347240971	2.05268916192767e-06\\
18267.3036529127	2.01763635549262e-06\\
18405.2058224618	1.98325678212222e-06\\
18544.1490328086	1.94953673478688e-06\\
18684.1411429008	1.91646276724193e-06\\
18825.1900710143	1.88402169012144e-06\\
18967.3037952012	1.85220056705832e-06\\
19110.4903537406	1.82098671083129e-06\\
19254.7578455939	1.79036767953977e-06\\
19400.1144308624	1.76033127280697e-06\\
19546.5683312491	1.73086552801232e-06\\
19694.1278305236	1.70195871655358e-06\\
19842.8012749908	1.67359934013927e-06\\
19992.5970739628	1.6457761271121e-06\\
20143.5237002349	1.61847802880393e-06\\
20295.5896905642	1.59169421592275e-06\\
20448.8036461532	1.56541407497212e-06\\
20603.1742331357	1.5396272047037e-06\\
20758.7101830675	1.51432341260309e-06\\
20915.4202934198	1.48949271140956e-06\\
21073.313428077	1.46512531567006e-06\\
21232.3985178383	1.44121163832762e-06\\
21392.6845609224	1.41774228734483e-06\\
21554.1806234768	1.39470806236239e-06\\
21716.8958400905	1.37209995139325e-06\\
21880.8394143105	1.34990912755243e-06\\
22046.0206191627	1.3281269458228e-06\\
22212.4487976761	1.30674493985707e-06\\
22380.1333634117	1.28575481881617e-06\\
22549.0838009943	1.26514846424412e-06\\
22719.3096666494	1.24491792697964e-06\\
22890.8205887437	1.22505542410447e-06\\
23063.6262683299	1.20555333592872e-06\\
23237.7364796946	1.18640420301311e-06\\
23413.1610709123	1.16760072322836e-06\\
23589.9099644014	1.14913574885157e-06\\
23767.9931574861	1.13100228369986e-06\\
23947.4207229617	1.11319348030103e-06\\
24128.2028096642	1.0957026371015e-06\\
24310.3496430445	1.07852319571123e-06\\
24493.8715257466	1.06164873818576e-06\\
24678.7788381905	1.04507298434533e-06\\
24865.0820391595	1.02878978913089e-06\\
25052.7916663912	1.01279313999698e-06\\
25241.9183371742	9.97077154341423e-07\\
25432.4727489482	9.81636076971622e-07\\
25624.4656799093	9.66464277607437e-07\\
25817.9079896195	9.51556248420448e-07\\
26012.8106196209	9.36906601609455e-07\\
26209.1845940549	9.22510067012053e-07\\
26407.0410202855	9.08361489752155e-07\\
26606.3910895274	8.94455827923219e-07\\
26807.2460774792	8.8078815030702e-07\\
27009.6173449613	8.67353634127727e-07\\
27213.5163385584	8.54147562841144e-07\\
27418.9545912668	8.41165323958792e-07\\
27625.9437231468	8.28402406906694e-07\\
27834.49544198	8.15854400918556e-07\\
28044.6215439317	8.03516992963114e-07\\
28256.3339142177	7.91385965705413e-07\\
28469.6445277769	7.79457195501733e-07\\
28684.5654499485	7.67726650427887e-07\\
28901.1088371544	7.5619038834061e-07\\
29119.286937587	7.44844554971791e-07\\
29339.1120919019	7.33685382055219e-07\\
29560.5967339154	7.22709185485562e-07\\
29783.7533913087	7.11912363509272e-07\\
30008.594686336	7.0129139494712e-07\\
30235.1333365383	6.90842837448045e-07\\
30463.382155463	6.80563325774018e-07\\
30693.3540533885	6.70449570115557e-07\\
30925.0620380546	6.60498354437617e-07\\
31158.5192153983	6.50706534855516e-07\\
31393.7387902947	6.41071038040559e-07\\
31630.7340673043	6.31588859655029e-07\\
31869.5184514253	6.22257062816217e-07\\
32110.105448852	6.13072776589137e-07\\
32352.5086677388	6.04033194507623e-07\\
32596.7418189694	5.95135573123422e-07\\
32842.818716933	5.86377230582967e-07\\
33090.7532803053	5.77755545231474e-07\\
33340.5595328357	5.69267954244016e-07\\
33592.2516041407	5.60911952283258e-07\\
33845.8437305032	5.52685090183439e-07\\
34101.3502556773	5.44584973660326e-07\\
34358.7856317002	5.36609262046726e-07\\
34618.1644197092	5.28755667053269e-07\\
34879.5012907654	5.21021951554064e-07\\
35142.8110266837	5.13405928396894e-07\\
35408.1085208687	5.05905459237609e-07\\
35675.4087791572	4.98518453398365e-07\\
35944.726920667	4.91242866749355e-07\\
36216.078178652	4.84076700613693e-07\\
36489.4779013636	4.77018000695099e-07\\
36764.9415529195	4.70064856028036e-07\\
37042.4847141778	4.63215397949978e-07\\
37322.1230836184	4.56467799095439e-07\\
37603.8724782308	4.49820272411434e-07\\
37887.7488344093	4.43271070194035e-07\\
38173.768208854	4.36818483145684e-07\\
38461.9467794789	4.30460839452924e-07\\
38752.3008463271	4.24196503884214e-07\\
39044.8468324927	4.18023876907506e-07\\
39339.60128505	4.11941393827229e-07\\
39636.5808759893	4.05947523940398e-07\\
39935.8024031596	4.00040769711478e-07\\
40237.2827912191	3.942196659657e-07\\
40541.0390925923	3.88482779100515e-07\\
40847.0884884347	3.82828706314858e-07\\
41155.4482896045	3.77256074855916e-07\\
41466.1359376416	3.71763541283083e-07\\
41779.1690057544	3.6634979074878e-07\\
42094.5651998136	3.61013536295882e-07\\
42412.3423593538	3.55753518171373e-07\\
42732.5184585825	3.50568503156009e-07\\
43055.1116073965	3.45457283909632e-07\\
43380.1400524069	3.40418678331865e-07\\
43707.6221779705	3.35451528937903e-07\\
44037.57650723	3.30554702249107e-07\\
44370.0217031614	3.25727088198102e-07\\
44704.9765696303	3.2096759954812e-07\\
45042.4600524546	3.16275171326302e-07\\
45382.4912404769	3.11648760270672e-07\\
45725.0893666436	3.07087344290541e-07\\
46070.2738090931	3.02589921940041e-07\\
46418.0640922519	2.98155511904538e-07\\
46768.479887939	2.93783152499674e-07\\
47121.541016478	2.89471901182767e-07\\
47477.2674478187	2.85220834076315e-07\\
47835.679302667	2.81029045503354e-07\\
48196.7968536222	2.76895647534432e-07\\
48560.6405263239	2.72819769545949e-07\\
48927.2309006076	2.68800557789613e-07\\
49296.5887116685	2.64837174972797e-07\\
49668.7348512343	2.60928799849534e-07\\
50043.6903687473	2.57074626821956e-07\\
50421.4764725545	2.53273865551921e-07\\
50802.114531107	2.49525740582613e-07\\
51185.6260741696	2.45829490969906e-07\\
51572.0327940378	2.42184369923259e-07\\
51961.3565467648	2.38589644455951e-07\\
52353.6193533982	2.35044595044416e-07\\
52748.8434012247	2.3154851529651e-07\\
53147.0510450263	2.28100711628463e-07\\
53548.2648083438	2.24700502950368e-07\\
53952.5073847507	2.21347220359967e-07\\
54359.8016391376	2.1804020684456e-07\\
54770.1706090047	2.14778816990856e-07\\
55183.6375057655	2.11562416702557e-07\\
55600.2257160592	2.08390382925524e-07\\
56019.9588030736	2.05262103380311e-07\\
56442.8605078779	2.02176976301918e-07\\
56868.9547507659	1.99134410186574e-07\\
57298.2656326087	1.96133823545389e-07\\
57730.8174362176	1.93174644664705e-07\\
58166.634627718	1.90256311372985e-07\\
58605.7418579333	1.87378270814067e-07\\
59048.163963779	1.84539979226647e-07\\
59493.9259696676	1.81740901729825e-07\\
59943.0530889239	1.78980512114551e-07\\
60395.5707252113	1.76258292640846e-07\\
60851.5044739687	1.73573733840635e-07\\
61310.8801238584	1.70926334326058e-07\\
61773.7236582241	1.68315600603115e-07\\
62240.0612565613	1.65741046890508e-07\\
62709.9192959978	1.63202194943544e-07\\
63183.3243527857	1.60698573882981e-07\\
63660.3032038044	1.58229720028664e-07\\
64140.8828280752	1.55795176737851e-07\\
64625.090408288	1.53394494248083e-07\\
65112.9533323374	1.51027229524496e-07\\
65604.4991948736	1.48692946111439e-07\\
66099.7557988615	1.463912139883e-07\\
66598.7511571543	1.44121609429418e-07\\
67101.5134940779	1.4188371486796e-07\\
67608.0712470273	1.39677118763678e-07\\
68118.4530680746	1.37501415474425e-07\\
68632.68782559	1.35356205131322e-07\\
69150.8046058751	1.33241093517488e-07\\
69672.8327148071	1.3115569195023e-07\\
70198.8016794974	1.29099617166586e-07\\
70728.7412499609	1.27072491212146e-07\\
71262.6814007993	1.25073941333042e-07\\
71800.6523328963	1.23103599871017e-07\\
72342.684475126	1.21161104161505e-07\\
72888.8084860737	1.19246096434611e-07\\
73439.0552557701	1.17358223718926e-07\\
73993.4559074389	1.15497137748076e-07\\
74552.041799257	1.1366249486995e-07\\
75114.8445261277	1.1185395595851e-07\\
75681.8959214685	1.10071186328114e-07\\
76253.2280590112	1.08313855650271e-07\\
76828.8732546163	1.06581637872772e-07\\
77408.8640681011	1.0487421114111e-07\\
77993.2333050806	1.03191257722125e-07\\
78582.0140188235	1.01532463929817e-07\\
79175.239512122	9.98975200532372e-08\\
79772.9433391755	9.8286120286431e-08\\
80375.1593074877	9.66979626603331e-08\\
80981.9214797798	9.51327489765789e-08\\
81593.2641759163	9.35901847431644e-08\\
82209.2219748472	9.20699791118969e-08\\
82829.8297165634	9.0571844817586e-08\\
83455.1225040667	8.90954981189155e-08\\
84085.1357053561	8.7640658740946e-08\\
84719.9049554282	8.62070498191941e-08\\
85359.4661582928	8.47943978452439e-08\\
86003.8554890032	8.34024326138381e-08\\
86653.109395703	8.20308871714e-08\\
87307.2646016869	8.0679497765945e-08\\
87966.3581074791	7.93480037983329e-08\\
88630.4271929253	7.80361477748217e-08\\
89299.509419301	7.67436752608775e-08\\
89973.6426314364	7.54703348361996e-08\\
90652.8649598577	7.42158780509223e-08\\
91337.214822943	7.2980059382956e-08\\
92026.7309290954	7.17626361964274e-08\\
92721.4522789325	7.05633687011831e-08\\
93421.4181674926	6.93820199133226e-08\\
94126.6681864573	6.82183556167241e-08\\
94837.2422263909	6.70721443255328e-08\\
95553.1804789962	6.59431572475776e-08\\
96274.5234393881	6.48311682486843e-08\\
97001.3119083847	6.37359538178546e-08\\
97733.5869948146	6.2657293033285e-08\\
98471.3901178417	6.1594967529194e-08\\
99214.7630093084	6.05487614634293e-08\\
99963.7477160966	5.95184614858289e-08\\
100718.386602504	5.85038567073146e-08\\
101478.722352644	5.75047386696831e-08\\
102244.797972856	5.65209013160814e-08\\
103016.65679414	5.55521409621341e-08\\
103794.342474607	5.4598256267704e-08\\
104577.899001948	5.36590482092639e-08\\
105367.370695925	5.27343200528577e-08\\
106162.802210873	5.18238773276292e-08\\
106964.23853823	5.09275277998983e-08\\
107771.725009077	5.00450814477698e-08\\
108585.307296709	4.91763504362484e-08\\
109405.031419212	4.83211490928498e-08\\
110230.943742068	4.74792938836849e-08\\
111063.09098078	4.66506033900015e-08\\
111901.52020351	4.583489828517e-08\\
112746.278833744	4.50320013120945e-08\\
113597.414652975	4.42417372610337e-08\\
114454.975803403	4.34639329478197e-08\\
115319.010790661	4.26984171924584e-08\\
116189.568486556	4.19450207980998e-08\\
117066.698131834	4.12035765303621e-08\\
117950.449338968	4.04739190970002e-08\\
118840.872094958	3.97558851279043e-08\\
119738.016764165	3.90493131554177e-08\\
120641.934091157	3.83540435949632e-08\\
121552.675203577	3.76699187259656e-08\\
122470.291615039	3.69967826730598e-08\\
123394.835228039	3.6334481387577e-08\\
124326.358336891	3.56828626292973e-08\\
125264.913630686	3.5041775948458e-08\\
126210.554196271	3.44110726680127e-08\\
127163.333521252	3.37906058661279e-08\\
128123.305497023	3.31802303589126e-08\\
129090.524421805	3.25798026833717e-08\\
130065.045003728	3.19891810805741e-08\\
131046.922363918	3.14082254790301e-08\\
132036.212039619	3.08367974782708e-08\\
133032.969987331	3.02747603326222e-08\\
134037.252585978	2.97219789351673e-08\\
135049.116640094	2.91783198018912e-08\\
136068.619383039	2.86436510560019e-08\\
137095.818480235	2.8117842412422e-08\\
138130.772032427	2.76007651624468e-08\\
139173.538578968	2.70922921585605e-08\\
140224.177101134	2.65922977994086e-08\\
141282.747025459	2.6100658014921e-08\\
142349.308227094	2.56172502515808e-08\\
143423.921033194	2.5141953457834e-08\\
144506.646226332	2.46746480696374e-08\\
145597.545047938	2.42152159961391e-08\\
146696.679201759	2.37635406054898e-08\\
147804.110857352	2.33195067107799e-08\\
148919.902653601	2.28830005560986e-08\\
150044.117702257	2.24539098027136e-08\\
151176.819591511	2.20321235153658e-08\\
152318.072389588	2.161753214868e-08\\
153467.940648372	2.12100275336831e-08\\
154626.48940706	2.08095028644327e-08\\
155793.784195833	2.04158526847505e-08\\
156969.891039574	2.00289728750585e-08\\
158154.87646159	1.9648760639317e-08\\
159348.807487385	1.92751144920604e-08\\
160551.751648444	1.89079342455301e-08\\
161763.776986059	1.85471209969029e-08\\
162984.952055171	1.81925771156112e-08\\
164215.345928253	1.7844206230756e-08\\
165455.028199213	1.75019132186079e-08\\
166704.068987334	1.71656041901979e-08\\
167962.538941238	1.68351864789941e-08\\
169230.50924288	1.65105686286641e-08\\
170508.051611579	1.61916603809217e-08\\
171795.238308072	1.58783726634561e-08\\
173092.142138601	1.55706175779437e-08\\
174398.83645903	1.52683083881404e-08\\
175715.395178998	1.49713595080536e-08\\
177041.892766095	1.46796864901937e-08\\
178378.404250078	1.43932060139029e-08\\
179725.005227113	1.41118358737629e-08\\
181081.771864049	1.38354949680771e-08\\
182448.780902729	1.35641032874306e-08\\
183826.109664331	1.32975819033246e-08\\
185213.83605374	1.30358529568854e-08\\
186612.038563953	1.27788396476486e-08\\
188020.796280522	1.25264662224156e-08\\
189440.188886025	1.22786579641847e-08\\
190870.296664576	1.20353411811535e-08\\
192311.200506361	1.17964431957958e-08\\
193762.981912217	1.15618923340086e-08\\
195225.722998241	1.1331617914332e-08\\
196699.506500436	1.110555023724e-08\\
198184.415779389	1.08836205745033e-08\\
199680.534824985	1.06657611586224e-08\\
201187.94826116	1.04519051723315e-08\\
202706.741350687	1.02419867381735e-08\\
204237	1.00359409081449e-08\\
};
\addlegendentry{Adatt. BLN reale (\textit{ML})}

        \nextgroupplot[%
            plotColumn2,plotLegend1,xmode=log,
            xmin=100,xmax=1000000,
            ymin=0,ymax=0.00045,
        ] 
% !TEX root = ../../../../Esperimenti/.tex/MEF.tex

% This file was created by matlab2tikz.
%
\definecolor{mycolor1}{rgb}{0.83529,0.36863,0.00000}%

\addplot[ybar interval, fill=mycolor1, fill opacity=0.35, area legend, draw=none] table[row sep=crcr, x=Lower, y=Count] {%
Lower	Upper	Count\\
119	166.945095217458	1.11238351058511e-06\\
166.945095217458	234.207267371144	2.7752102064167e-06\\
234.207267371144	328.569365982322	7.91278148383845e-06\\
328.569365982322	460.949950331585	3.7064851245265e-05\\
460.949950331585	646.666666794865	9.59166932980712e-05\\
646.666666794865	907.208640857353	0.000210740195948228\\
907.208640857353	1272.72296579858	0.000311159715427349\\
1272.72296579858	1785.50299756883	0.000337142613379807\\
1785.50299756883	2504.88208353096	0.000250289919246847\\
2504.88208353096	3514.09897431581	0.000147863821968552\\
3514.09897431581	4929.92930983802	7.15716641965845e-05\\
4929.92930983802	6916.19763064073	3.09189512263452e-05\\
6916.19763064073	9702.73337806785	1.63069862075001e-05\\
9702.73337806785	13611.9642661441	8.69736297840712e-06\\
13611.9642661441	19096.2241219165	4.55120666350786e-06\\
19096.2241219165	26790.0920531703	2.43310649042416e-06\\
26790.0920531703	37583.8190647142	1.00799288219573e-06\\
37583.8190647142	52726.3382554153	3.8214689337955e-07\\
52726.3382554153	73969.7778194808	1.35571266193244e-07\\
73969.7778194808	103772.198330147	2.68434572189764e-09\\
103772.198330147	145582.012866817	0\\
145582.012866817	204237	0\\
204237	204237	0\\
};
\addlegendentry{Istogramma appr.}

\addplot [color=mycolor1, only marks, every error bar/.append style={opacity=0.45}, mark=*, mark size=0pt, draw=none, forget plot]
plot [error bars/.cd, y dir=both, y explicit, error bar style={line width=1pt, color=mycolor1}, error mark options={mark=none,mark size=0pt}]
table[row sep=crcr, y error plus index=2, y error minus index=3]{%
140.948452743822	1.11238351058511e-06	1.55283081491976e-06	1.11238351058511e-06\\
197.736578689671	2.7752102064167e-06	2.01725080098544e-06	2.01725080098544e-06\\
277.404638296819	7.91278148383845e-06	2.9906047912031e-06	2.9906047912031e-06\\
389.170955917874	3.7064851245265e-05	5.10412816699938e-06	5.10412816699938e-06\\
545.967918416627	9.59166932980712e-05	6.21941950524437e-06	6.21941950524437e-06\\
765.938370804547	0.000210740195948228	8.33153173681027e-06	8.33153173681028e-06\\
1074.53490961907	0.000311159715427349	9.01897024273471e-06	9.01897024273469e-06\\
1507.46498151966	0.000337142613379807	7.93079916252218e-06	7.9307991625222e-06\\
2114.82256198977	0.000250289919246847	5.23395039719159e-06	5.23395039719161e-06\\
2966.88448722194	0.000147863821968552	2.98722479853788e-06	2.98722479853789e-06\\
4162.24212788627	7.15716641965845e-05	1.87274928280596e-06	1.87274928280597e-06\\
5839.20931393352	3.09189512263452e-05	1.24974676291599e-06	1.24974676291598e-06\\
8191.82651184286	1.63069862075001e-05	7.5082496507373e-07	7.50824965073729e-07\\
11492.3130842395	8.69736297840712e-06	4.3504453055823e-07	4.3504453055823e-07\\
16122.5655640101	4.55120666350786e-06	2.56373587033532e-07	2.56373587033532e-07\\
22618.3465817931	2.43310649042416e-06	1.47266632832467e-07	1.47266632832467e-07\\
31731.2775105792	1.00799288219573e-06	7.52630941431503e-08	7.52630941431504e-08\\
44515.8079443296	3.8214689337955e-07	3.58334299941275e-08	3.58334299941275e-08\\
62451.2251760353	1.35571266193244e-07	1.60992528030993e-08	1.60992528030993e-08\\
87612.8212895809	2.68434572189764e-09	3.04393410592121e-09	2.68434572189764e-09\\
122912.023466044	0	0	0\\
172433.272780749	0	0	0\\
};
\addplot [color=mycolor1]
table[row sep=crcr]{%
119	9.03372815198184e-07\\
119.890504214077	9.20625786741225e-07\\
120.787672274837	9.38222444531328e-07\\
121.691554049369	9.561711099858e-07\\
122.602199777928	9.74480363752054e-07\\
123.519660076729	9.93159054631823e-07\\
124.44398594076	1.01221630878737e-06\\
125.375228746615	1.03166153923643e-06\\
126.313440255352	1.05150445564284e-06\\
127.258672615369	1.07175507440961e-06\\
128.2109783653	1.09242372908151e-06\\
129.17041043694	1.11352108106408e-06\\
130.137022158185	1.13505813066615e-06\\
131.110867255994	1.15704622847286e-06\\
132.091999859379	1.17949708705623e-06\\
133.080474502409	1.20242279303046e-06\\
134.076346127247	1.22583581945893e-06\\
135.0796700872	1.24974903861998e-06\\
136.090502149795	1.27417573513864e-06\\
137.108898499881	1.29912961949114e-06\\
138.134915742751	1.32462484188947e-06\\
139.168610907289	1.3506760065526e-06\\
140.21004144914	1.37729818637163e-06\\
141.259265253899	1.40450693797531e-06\\
142.316340640336	1.43231831720305e-06\\
143.381326363632	1.46074889499169e-06\\
144.454281618647	1.48981577368291e-06\\
145.535266043209	1.51953660375746e-06\\
146.624339721429	1.54992960100262e-06\\
147.721563187044	1.58101356411891e-06\\
148.826997426776	1.61280789277214e-06\\
149.940703883725	1.64533260609645e-06\\
151.062744460785	1.67860836165405e-06\\
152.193181524082	1.71265647485681e-06\\
153.332077906443	1.7474989388551e-06\\
154.479496910888	1.78315844489844e-06\\
155.635502314145	1.8196584031728e-06\\
156.800158370201	1.85702296411867e-06\\
157.97352981387	1.89527704023393e-06\\
159.15568186439	1.93444632836522e-06\\
160.34668022905	1.9745573324909e-06\\
161.546591106842	2.01563738699874e-06\\
162.755481192139	2.05771468046053e-06\\
163.973417678405	2.10081827990595e-06\\
165.200468261928	2.14497815559709e-06\\
166.436701145581	2.19022520630474e-06\\
167.682185042617	2.23659128508723e-06\\
168.936989180483	2.28410922557161e-06\\
170.201183304674	2.33281286873698e-06\\
171.474837682604	2.38273709019861e-06\\
172.758023107516	2.43391782799133e-06\\
174.050810902413	2.48639211084977e-06\\
175.353272924028	2.54019808698244e-06\\
176.665481566809	2.59537505333586e-06\\
177.987509766954	2.6519634853444e-06\\
179.319431006454	2.71000506716045e-06\\
180.661319317186	2.76954272235895e-06\\
182.013249285024	2.83062064510943e-06\\
183.375296053983	2.89328433180772e-06\\
184.7475353304	2.95758061315888e-06\\
186.13004338714	3.02355768670152e-06\\
187.522897067834	3.09126514976333e-06\\
188.926173791152	3.16075403283594e-06\\
190.339951555105	3.23207683335675e-06\\
191.764308941382	3.30528754988418e-06\\
193.199325119717	3.38044171665141e-06\\
194.645079852288	3.45759643848289e-06\\
196.101653498152	3.53681042605658e-06\\
197.569127017711	3.61814403149375e-06\\
199.047581977213	3.70165928425668e-06\\
200.537100553285	3.78741992733389e-06\\
202.037765537499	3.87549145369039e-06\\
203.549660340977	3.9659411429601e-06\\
205.072868999023	4.05883809835527e-06\\
206.6074761758	4.15425328376703e-06\\
208.15356716903	4.25225956102931e-06\\
209.711227914737	4.35293172731726e-06\\
211.280544992026	4.45634655264946e-06\\
212.86160562789	4.56258281746201e-06\\
214.454497702065	4.67172135022068e-06\\
216.059309751909	4.78384506503609e-06\\
217.676130977326	4.89903899924503e-06\\
219.305051245723	5.01739035091933e-06\\
220.946161097006	5.13898851626235e-06\\
222.599551748611	5.26392512685116e-06\\
224.265315100576	5.39229408668074e-06\\
225.943543740646	5.52419160896528e-06\\
227.634330949423	5.6597162526491e-06\\
229.33777070555	5.79896895857888e-06\\
231.05395769093	5.94205308528609e-06\\
232.782987295997	6.08907444432784e-06\\
234.524955625009	6.24014133513123e-06\\
236.279959501398	6.39536457928598e-06\\
238.048096473146	6.55485755422677e-06\\
239.829464818207	6.718736226246e-06\\
241.624163549975	6.88711918277546e-06\\
243.432292422782	7.06012766387312e-06\\
245.253951937446	7.23788559285059e-06\\
247.089243346851	7.42051960597323e-06\\
248.938268661586	7.60815908116501e-06\\
250.801130655604	7.80093616564664e-06\\
252.677932871941	7.99898580243523e-06\\
254.568779628468	8.20244575563035e-06\\
256.473776023691	8.41145663441156e-06\\
258.393027942593	8.62616191566844e-06\\
260.326642062517	8.84670796518489e-06\\
262.274725859099	9.0732440572956e-06\\
264.237387612236	9.3059223929323e-06\\
266.214736412113	9.54489811597582e-06\\
268.206882165259	9.79032932782716e-06\\
270.213935600659	1.00423771001113e-05\\
272.236008275909	1.03012054854234e-05\\
274.273212583414	1.05669815260292e-05\\
276.325661756641	1.08398752604259e-05\\
278.393469876404	1.1120059727672e-05\\
280.476751877215	1.14077109693918e-05\\
282.575623553662	1.17030080293596e-05\\
284.690201566855	1.20061329505667e-05\\
286.820603450903	1.23172707696751e-05\\
288.96694761945	1.26366095087592e-05\\
291.129353372257	1.29643401642373e-05\\
293.307940901833	1.33006566928934e-05\\
295.502831300113	1.36457559948891e-05\\
297.714146565192	1.39998378936656e-05\\
299.942009608104	1.43631051126354e-05\\
302.186544259657	1.47357632485619e-05\\
304.447875277308	1.51180207415282e-05\\
306.726128352108	1.55100888413907e-05\\
309.02143011568	1.59121815706212e-05\\
311.333908147261	1.63245156834335e-05\\
313.663690980792	1.67473106210976e-05\\
316.010908112063	1.71807884633408e-05\\
318.375690005912	1.7625173875738e-05\\
320.758168103475	1.80806940529945e-05\\
323.158474829489	1.85475786580241e-05\\
325.57674359966	1.90260597567293e-05\\
328.013108828071	1.95163717483895e-05\\
330.467705934659	2.00187512915657e-05\\
332.940671352735	2.05334372254325e-05\\
335.432142536577	2.10606704864508e-05\\
337.942257969062	2.16006940202938e-05\\
340.471157169366	2.21537526889463e-05\\
343.018980700718	2.27200931728962e-05\\
345.585870178217	2.32999638683407e-05\\
348.171968276697	2.38936147793343e-05\\
350.777418738662	2.45012974048087e-05\\
353.402366382274	2.51232646203958e-05\\
356.046957109401	2.57597705549928e-05\\
358.711337913731	2.64110704620093e-05\\
361.395656888935	2.70774205852423e-05\\
364.100063236907	2.77590780193288e-05\\
366.824707276051	2.84563005647298e-05\\
369.569740449639	2.91693465772072e-05\\
372.335315334224	2.98984748117558e-05\\
375.12158564813	3.06439442609636e-05\\
377.928706259986	3.14060139877734e-05\\
380.75683319734	3.21849429526302e-05\\
383.60612365533	3.29809898350005e-05\\
386.476736005421	3.37944128492598e-05\\
389.368829804207	3.46254695549481e-05\\
392.282565802281	3.54744166614014e-05\\
395.21810595317	3.63415098267765e-05\\
398.175613422337	3.7227003451488e-05\\
401.155252596246	3.81311504660921e-05\\
404.157189091507	3.90542021136506e-05\\
407.181589764074	3.99964077266241e-05\\
410.228622718523	4.09580144983478e-05\\
413.298457317395	4.1939267249152e-05\\
416.391264190611	4.29404081871991e-05\\
419.507215244952	4.39616766641158e-05\\
422.646483673618	4.50033089255129e-05\\
425.809243965854	4.60655378564857e-05\\
428.995671916648	4.7148592722209e-05\\
432.205944636502	4.82526989037378e-05\\
435.440240561274	4.93780776291444e-05\\
438.698739462102	5.0524945700127e-05\\
441.981622455389	5.16935152142379e-05\\
445.289072012877	5.28839932828852e-05\\
448.621271971783	5.40965817452771e-05\\
451.978407545023	5.53314768784864e-05\\
455.360665331498	5.65888691038187e-05\\
458.768233326478	5.78689426896891e-05\\
462.201300932039	5.91718754512067e-05\\
465.660058967601	6.04978384466957e-05\\
469.144699680525	6.18469956713723e-05\\
472.655416756805	6.32195037484269e-05\\
476.192405331832	6.46155116177538e-05\\
479.755862001239	6.60351602225943e-05\\
483.345984831829	6.74785821943641e-05\\
486.962973372584	6.89459015359424e-05\\
490.607028665758	7.04372333037214e-05\\
494.278353258049	7.19526832887112e-05\\
497.977151211859	7.3492347697019e-05\\
501.703628116633	7.50563128300176e-05\\
505.457991100293	7.66446547645422e-05\\
509.240448840744	7.82574390334509e-05\\
513.051211577476	7.9894720306902e-05\\
516.890491123249	8.15565420747089e-05\\
520.758500875867	8.32429363301397e-05\\
524.655455830038	8.49539232555378e-05\\
528.581572589324	8.66895109101507e-05\\
532.537069378183	8.8449694920561e-05\\
536.522166054094	9.02344581741167e-05\\
540.537084119782	9.20437705157751e-05\\
544.582046735526	9.38775884487685e-05\\
548.657278731565	9.57358548395192e-05\\
552.763006620593	9.76184986272272e-05\\
556.899458610353	9.95254345385671e-05\\
561.066864616317	0.000101456562807929\\
565.265456274466	0.000103411768903647\\
569.495466954168	0.000105390923260668\\
573.757131771146	0.000107393881020094\\
578.050687600549	0.000109420481776079\\
582.376373090114	0.000111470549330505\\
586.734428673439	0.000113543891455919\\
591.125096583335	0.000115640299667174\\
595.548620865302	0.000117759549002241\\
600.005247391086	0.000119901397812647\\
604.495223872346	0.000122065587563996\\
609.018799874428	0.000124251842647035\\
613.576226830228	0.000126459870199707\\
618.167758054176	0.000128689359940658\\
622.793648756308	0.000130939984014628\\
627.454156056458	0.000133211396850183\\
632.149538998544	0.00013550323503021\\
636.880058564972	0.000137815117175623\\
641.645977691138	0.000140146643842693\\
646.447561280041	0.000142497397434414\\
651.285076217013	0.000144866942126329\\
656.158791384547	0.000147254823807199\\
661.06897767725	0.000149660570034911\\
666.015908016889	0.000152083690008001\\
670.999857367572	0.000154523674553156\\
676.021102751025	0.000156979996129053\\
681.079923261988	0.000159452108846864\\
686.176600083736	0.000161939448507768\\
691.311416503699	0.000164441432657762\\
696.484657929211	0.000166957460660083\\
701.696611903378	0.00016948691378551\\
706.947568121055	0.000172029155320798\\
712.237818444947	0.000174583530695506\\
717.56765692184	0.000177149367627429\\
722.937379798933	0.000179725976286829\\
728.347285540317	0.000182312649479676\\
733.797674843554	0.000184908662850027\\
739.288850656394	0.00018751327510171\\
744.821118193618	0.000190125728239412\\
750.394784953994	0.000192745247829276\\
756.01016073738	0.000195371043279077\\
761.667557661931	0.000198002308138011\\
767.367290181458	0.000200638220416135\\
773.109675102898	0.000203277942923445\\
778.89503160393	0.000205920623628555\\
784.72368125071	0.000208565396036938\\
790.59594801575	0.000211211379588627\\
796.512158295919	0.000213857680075276\\
802.472640930591	0.000216503390076443\\
808.47772721992	0.000219147589414922\\
814.527750943253	0.000221789345630933\\
820.623048377686	0.000224427714474955\\
826.763958316753	0.000227061740418938\\
832.950822089257	0.000229690457185621\\
839.183983578243	0.00023231288829565\\
845.463789240111	0.000234928047632157\\
851.790588123873	0.000237534940022433\\
858.164731890556	0.0002401325618363\\
864.586574832748	0.000242719901600763\\
871.056473894285	0.000245295940630489\\
877.574788690099	0.000247859653673636\\
884.141881526201	0.000250410009572522\\
890.758117419823	0.000252945971938599\\
897.423864119701	0.000255466499841179\\
904.139492126523	0.000257970548509317\\
910.905374713515	0.000260457070046237\\
917.721887947192	0.000262925014155666\\
924.589410708265	0.00026537332887941\\
931.508324712691	0.000267800961345485\\
938.479014532895	0.000270206858526087\\
945.501867619149	0.000272589968004673\\
952.577274321103	0.000274949238751395\\
959.705627909479	0.000277283621906099\\
966.88732459794	0.000279592071568116\\
974.122763565099	0.000281873545592\\
981.412346976721	0.000284127006388397\\
988.756480008064	0.000286351421729193\\
996.155570866409	0.000288545765556061\\
1003.61003081374	0.00029070901879154\\
1011.12027418962	0.000292840170151739\\
1018.6867184342	0.000294938216959766\\
1026.30978411142	0.000297002165958956\\
1033.98989493243	0.000299031034124975\\
1041.72747777907	0.000301023849475865\\
1049.52296272766	0.000302979651879078\\
1057.37678307286	0.000304897493854549\\
1065.2893753518	0.000306776441372874\\
1073.26117936829	0.000308615574647607\\
1081.2926382173	0.000310413988920742\\
1089.38419830959	0.000312170795240415\\
1097.53630939651	0.000313885121229867\\
1105.749424595	0.000315556111846722\\
1114.02400041277	0.000317182930131626\\
1122.3604967737	0.000318764757945322\\
1130.75937704337	0.000320300796693204\\
1139.22110805483	0.000321790268036451\\
1147.74616013456	0.000323232414588808\\
1156.3350071286	0.000324626500598136\\
1164.98812642888	0.000325971812611828\\
1173.70599899975	0.000327267660125226\\
1182.48910940477	0.00032851337621219\\
1191.33794583355	0.000329708318136987\\
1200.25300012896	0.000330851867946676\\
1209.23476781446	0.000331943433043203\\
1218.28374812157	0.000332982446734439\\
1227.40044401774	0.000333968368763414\\
1236.58536223419	0.000334900685815014\\
1245.83901329415	0.000335778911999478\\
1255.16191154121	0.000336602589311994\\
1264.5545751679	0.00033737128806779\\
1274.01752624451	0.000338084607312092\\
1283.55129074812	0.000338742175204393\\
1293.15639859177	0.000339343649376479\\
1302.83338365401	0.000339888717263714\\
1312.5827838085	0.000340377096409111\\
1322.40514095393	0.000340808534739743\\
1332.30100104415	0.000341182810815107\\
1342.27091411851	0.000341499734047074\\
1352.31543433242	0.000341759144891092\\
1362.43511998817	0.000341960915008363\\
1372.63053356595	0.000342104947398751\\
1382.90224175512	0.000342191176504198\\
1393.2508154857	0.000342219568282497\\
1403.67682996012	0.000342190120251285\\
1414.18086468518	0.000342102861502166\\
1424.76350350425	0.000341957852684934\\
1435.42533462974	0.000341755185961869\\
1446.16695067579	0.000341494984932169\\
1456.98894869122	0.000341177404526562\\
1467.89193019267	0.000340802630872253\\
1478.8765011981	0.000340370881128339\\
1489.94327226042	0.0003398824032919\\
1501.09285850146	0.000339337475975013\\
1512.32587964614	0.000338736408152954\\
1523.64296005691	0.00033807953888393\\
1535.0447287685	0.000337367237000675\\
1546.53181952283	0.000336599900774327\\
1558.10487080425	0.000335777957551007\\
1569.76452587505	0.000334901863361578\\
1581.51143281119	0.000333972102505098\\
1593.34624453832	0.000332989187106496\\
1605.26961886811	0.000331953656649051\\
1617.28221853476	0.000330866077482306\\
1629.38471123188	0.000329727042306023\\
1641.57776964957	0.000328537169630888\\
1653.86207151182	0.000327297103216645\\
1666.2382996142	0.000326007511488411\\
1678.70714186178	0.000324669086931913\\
1691.26929130741	0.000323282545468456\\
1703.92544619016	0.000321848625810409\\
1716.67630997424	0.000320368088798061\\
1729.522591388	0.000318841716718701\\
1742.4650044634	0.000317270312608795\\
1755.50426857564	0.000315654699540152\\
1768.64110848317	0.000313995719891008\\
1781.876254368	0.000312294234602939\\
1795.21044187622	0.000310551122424567\\
1808.64441215895	0.000308767279142991\\
1822.17891191353	0.000306943616803937\\
1835.81469342496	0.000305081062921585\\
1849.55251460781	0.000303180559679067\\
1863.39313904828	0.000301243063120623\\
1877.33733604663	0.000299269542336421\\
1891.38588066002	0.000297260978641025\\
1905.53955374551	0.000295218364746523\\
1919.7991420035	0.000293142703931309\\
1934.16543802145	0.000291035009205528\\
1948.63924031791	0.000288896302474154\\
1963.22135338698	0.000286727613698729\\
1977.91258774292	0.000284529980058707\\
1992.71375996528	0.00028230444511341\\
2007.62569274426	0.000280052057965539\\
2022.64921492642	0.000277773872427204\\
2037.7851615608	0.000275470946189426\\
2053.03437394529	0.000273144339996018\\
2068.39769967338	0.000270795116822771\\
2083.87599268133	0.000268424341062843\\
2099.47011329558	0.000266033077719225\\
2115.18092828061	0.000263622391605149\\
2131.00931088707	0.000261193346553267\\
2146.95614090037	0.000258747004634437\\
2163.02230468953	0.000256284425386897\\
2179.20869525649	0.000253806665056604\\
2195.51621228573	0.000251314775849497\\
2211.94576219425	0.000248809805196387\\
2228.49825818201	0.000246292795031192\\
2245.17462028264	0.000243764781083171\\
2261.97577541457	0.000241226792183812\\
2278.90265743261	0.000238679849588983\\
2295.95620717979	0.000236124966316924\\
2313.1373725397	0.00023356314650266\\
2330.44710848916	0.000230995384769324\\
2347.88637715129	0.000228422665616926\\
2365.45614784899	0.000225845962829005\\
2383.15739715885	0.000223266238897605\\
2400.9911089654	0.000220684444466985\\
2418.95827451578	0.000218101517796428\\
2437.05989247487	0.000215518384242488\\
2455.29696898081	0.000212935955760988\\
2473.67051770087	0.000210355130429036\\
2492.18155988785	0.000207776791987321\\
2510.8311244368	0.000205201809402882\\
2529.62024794223	0.000202631036452545\\
2548.54997475574	0.000200065311327179\\
2567.62135704402	0.000197505456256877\\
2586.83545484739	0.000194952277157181\\
2606.1933361387	0.000192406563296371\\
2625.69607688265	0.000189869086983884\\
2645.34476109567	0.00018734060327984\\
2665.14048090611	0.000184821849725655\\
2685.08433661496	0.000182313546095693\\
2705.17743675704	0.000179816394169854\\
2725.42089816257	0.000177331077527004\\
2745.81584601927	0.000174858261359098\\
2766.3634139349	0.000172398592305852\\
2787.06474400025	0.000169952698309745\\
2807.92098685266	0.000167521188491173\\
2828.93330173995	0.000165104653043502\\
2850.10285658484	0.000162703663147758\\
2871.4308280499	0.000160318770906697\\
2892.91840160291	0.000157950509297931\\
2914.56677158281	0.000155599392145806\\
2936.37714126603	0.000153265914111694\\
2958.3507229334	0.000150950550702331\\
2980.48873793751	0.000148653758295842\\
3002.79241677064	0.000146375974185055\\
3025.2629991331	0.000144117616637703\\
3047.90173400216	0.000141879084973096\\
3070.7098797015	0.000139660759654828\\
3093.68870397109	0.00013746300239908\\
3116.83948403772	0.000135286156298058\\
3140.16350668593	0.000133130545958109\\
3163.66206832958	0.00013099647765203\\
3187.33647508389	0.000128884239485094\\
3211.18804283803	0.000126794101574299\\
3235.21809732829	0.000124726316240348\\
3259.42797421173	0.000122681118211848\\
3283.81901914043	0.000120658724841228\\
3308.39258783631	0.000118659336331865\\
3333.15004616647	0.000116683135975899\\
3358.09277021909	0.000114730290402222\\
3383.22214637995	0.000112800949834134\\
3408.53957140944	0.000110895248356132\\
3434.04645252026	0.000109013304189321\\
3459.7442074556	0.000107155219974946\\
3485.63426456793	0.000105321083065505\\
3511.71806289842	0.000103510965822955\\
3537.99705225692	0.000101724925923485\\
3564.47269330252	9.99630066683765e-05\\
3591.14645762477	9.82252373004221e-05\\
3618.01982782546	9.6511633325439e-05\\
3645.09429760103	9.48221968383706e-05\\
3672.37137182558	9.31569168535051e-05\\
3699.85256663454	9.15157696383362e-05\\
3727.53940950892	8.98987190506012e-05\\
3755.43343936023	8.83057168780394e-05\\
3783.53620661599	8.67367031804208e-05\\
3811.84927330594	8.51916066334111e-05\\
3840.37421314884	8.36703448738397e-05\\
3869.11261163994	8.21728248459507e-05\\
3898.06606613913	8.06989431482334e-05\\
3927.23618595969	7.92485863804267e-05\\
3956.62459245778	7.7821631490316e-05\\
3986.23291912252	7.64179461199445e-05\\
4016.06281166681	7.50373889508729e-05\\
4046.1159281188	7.36798100481358e-05\\
4076.39393891404	7.2345051202555e-05\\
4106.89852698833	7.10329462710786e-05\\
4137.63138787127	6.97433215148304e-05\\
4168.59422978048	6.84759959345636e-05\\
4199.78877371658	6.72307816032292e-05\\
4231.21675355883	6.60074839953765e-05\\
4262.8799161615	6.48059023131215e-05\\
4294.78002145095	6.36258298084249e-05\\
4326.91884252352	6.24670541014402e-05\\
4359.29816574399	6.13293574947011e-05\\
4391.91979084494	6.02125172829315e-05\\
4424.78553102675	5.91163060582713e-05\\
4457.89721305839	5.80404920107266e-05\\
4491.25667737898	5.69848392236638e-05\\
4524.86577820005	5.59491079641777e-05\\
4558.72638360862	5.49330549681784e-05\\
4592.84037567103	5.39364337200503e-05\\
4627.20965053756	5.29589947267502e-05\\
4661.83611854782	5.20004857862241e-05\\
4696.72170433693	5.10606522500287e-05\\
4731.86834694246	5.01392372800605e-05\\
4767.27799991228	4.92359820993006e-05\\
4802.95263141311	4.83506262364991e-05\\
4838.89422433987	4.74829077647295e-05\\
4875.10477642598	4.66325635337525e-05\\
4911.58630035433	4.57993293961459e-05\\
4948.34082386919	4.4982940427155e-05\\
4985.37038988889	4.41831311382387e-05\\
5022.6770566194	4.33996356842882e-05\\
5060.2628976687	4.2632188064505e-05\\
5098.13000216207	4.18805223169365e-05\\
5136.28047485818	4.11443727066719e-05\\
5174.7164362661	4.04234739077108e-05\\
5213.44002276314	3.97175611785241e-05\\
5252.45338671363	3.90263705313327e-05\\
5291.7586965885	3.83496388951401e-05\\
5331.3581370859	3.76871042725563e-05\\
5371.25390925253	3.70385058904611e-05\\
5411.44823060603	3.64035843445593e-05\\
5451.94333525824	3.57820817378841e-05\\
5492.74147403939	3.51737418133161e-05\\
5533.84491462315	3.45783100801825e-05\\
5575.25594165272	3.3995533935013e-05\\
5616.97685686784	3.34251627765286e-05\\
5659.00997923266	3.28669481149477e-05\\
5701.35764506468	3.2320643675694e-05\\
5744.02220816459	3.17860054975975e-05\\
5787.00603994713	3.12627920256816e-05\\
5830.31152957285	3.07507641986331e-05\\
5873.94108408097	3.02496855310551e-05\\
5917.8971285231	2.97593221906049e-05\\
5962.18210609809	2.92794430701211e-05\\
6006.79847828778	2.88098198548478e-05\\
6051.74872499389	2.83502270848641e-05\\
6097.03534467575	2.79004422128295e-05\\
6142.66085448928	2.74602456571574e-05\\
6188.62779042682	2.70294208507302e-05\\
6234.93870745815	2.6607754285269e-05\\
6281.59617967246	2.61950355514761e-05\\
6328.60280042144	2.57910573750625e-05\\
6375.9611824634	2.53956156487797e-05\\
6423.67395810857	2.50085094605706e-05\\
6471.74377936531	2.46295411179575e-05\\
6520.17331808759	2.4258516168783e-05\\
6568.96526612346	2.38952434184211e-05\\
6618.12233546469	2.35395349435733e-05\\
6667.64725839753	2.31912061027673e-05\\
6717.54278765452	2.28500755436715e-05\\
6767.81169656754	2.25159652073403e-05\\
6818.45677922192	2.21887003295036e-05\\
6869.48085061182	2.18681094390119e-05\\
6920.88674679662	2.15540243535491e-05\\
6972.67732505857	2.12462801727228e-05\\
7024.85546406162	2.09447152686394e-05\\
7077.42406401143	2.06491712740724e-05\\
7130.38604681656	2.03594930683298e-05\\
7183.7443562509	2.0075528760922e-05\\
7237.50195811723	1.97971296731364e-05\\
7291.66184041212	1.95241503176153e-05\\
7346.22701349204	1.92564483760391e-05\\
7401.20051024062	1.89938846750093e-05\\
7456.58538623724	1.87363231602279e-05\\
7512.38471992689	1.8483630869065e-05\\
7568.60161279127	1.82356779016069e-05\\
7625.23918952119	1.79923373902722e-05\\
7682.30059819021	1.77534854680842e-05\\
7739.78901042966	1.75190012356835e-05\\
7797.70762160489	1.7288766727164e-05\\
7856.05965099295	1.70626668748124e-05\\
7914.84834196143	1.68405894728304e-05\\
7974.0769621488	1.66224251401152e-05\\
8033.748803646	1.64080672821727e-05\\
8093.86718317946	1.61974120522351e-05\\
8154.43544229543	1.59903583116533e-05\\
8215.4569475457	1.57868075896311e-05\\
8276.93509067475	1.55866640423671e-05\\
8338.87328880825	1.5389834411667e-05\\
8401.27498464301	1.51962279830889e-05\\
8464.14364663834	1.50057565436788e-05\\
8527.48276920879	1.48183343393552e-05\\
8591.29587291844	1.46338780319965e-05\\
8655.58650467655	1.44523066562839e-05\\
8720.35823793472	1.42735415763527e-05\\
8785.61467288549	1.40975064422978e-05\\
8851.35943666247	1.39241271465829e-05\\
8917.59618354194	1.37533317803973e-05\\
8984.32859514597	1.35850505900027e-05\\
9051.56038064706	1.34192159331128e-05\\
9119.29527697427	1.32557622353439e-05\\
9187.53704902097	1.30946259467736e-05\\
9256.28948985408	1.29357454986449e-05\\
9325.55642092493	1.27790612602482e-05\\
9395.34169228163	1.26245154960146e-05\\
9465.64918278304	1.24720523228505e-05\\
9536.48280031448	1.23216176677425e-05\\
9607.84648200483	1.21731592256611e-05\\
9679.74419444542	1.20266264177876e-05\\
9752.17993391047	1.18819703500898e-05\\
9825.15772657923	1.17391437722691e-05\\
9898.68162875981	1.15981010371001e-05\\
9972.75572711457	1.14587980601842e-05\\
10047.3841388873	1.13211922801341e-05\\
10122.5710121321	1.11852426192084e-05\\
10198.3205259438	1.10509094444121e-05\\
10274.6368906905	1.09181545290775e-05\\
10351.5243482474	1.07869410149399e-05\\
10428.9871722326	1.06572333747214e-05\\
10507.0296682446	1.05289973752332e-05\\
10585.6561741017	1.04022000410079e-05\\
10664.8710600833	1.02768096184711e-05\\
10744.6787291723	1.01527955406612e-05\\
10825.0836173003	1.00301283925045e-05\\
10906.0901935939	9.90877987665281e-06\\
10987.7029606233	9.78872277988929e-06\\
11069.9264546524	9.66993094010767e-06\\
11152.765245891	9.55237921386866e-06\\
11236.2239387488	9.43604344453769e-06\\
11320.3071720913	9.32090043100628e-06\\
11405.019619498	9.20692789699908e-06\\
11490.3659895215	9.09410446096824e-06\\
11576.3510259497	8.98240960657539e-06\\
11662.9795080693	8.87182365376188e-06\\
11750.2562509318	8.76232773040639e-06\\
11838.1861056203	8.65390374456919e-06\\
11926.7739595202	8.54653435732148e-06\\
12016.0247365899	8.44020295615754e-06\\
12105.9433976352	8.33489362898764e-06\\
12196.5349405845	8.23059113870868e-06\\
12287.8044007671	8.12728089834869e-06\\
12379.7568511926	8.02494894678265e-06\\
12472.3974028332	7.92358192501411e-06\\
12565.7312049077	7.82316705301942e-06\\
12659.7634451676	7.72369210714917e-06\\
12754.4993501855	7.62514539808171e-06\\
12849.9441856459	7.52751574932418e-06\\
12946.1032566372	7.43079247625441e-06\\
13042.9819079474	7.33496536569878e-06\\
13140.5855243605	7.24002465603954e-06\\
13238.9195309561	7.14596101784528e-06\\
13337.9893934111	7.05276553501851e-06\\
13437.8006183031	6.96042968645345e-06\\
13538.3587534167	6.8689453281971e-06\\
13639.669388052	6.77830467610736e-06\\
13741.738153335	6.68850028900032e-06\\
13844.5707225307	6.5995250522802e-06\\
13948.1728113584	6.51137216204463e-06\\
14052.5501783096	6.42403510965738e-06\\
14157.7086249677	6.33750766678225e-06\\
14263.6539963308	6.25178387086933e-06\\
14370.3921811364	6.16685801108723e-06\\
14477.9291121888	6.08272461469313e-06\\
14586.2707666889	5.99937843383311e-06\\
14695.4231665662	5.91681443276538e-06\\
14805.3923788139	5.83502777549857e-06\\
14916.1845158256	5.75401381383756e-06\\
15027.8057357356	5.6737680758291e-06\\
15140.2622427609	5.59428625459967e-06\\
15253.5602875458	5.51556419757799e-06\\
15367.70616751	5.43759789609459e-06\\
15482.7062271979	5.36038347535058e-06\\
15598.5668586318	5.28391718474879e-06\\
15715.2945016669	5.20819538857888e-06\\
15832.8956443492	5.13321455704963e-06\\
15951.3768232763	5.05897125766093e-06\\
16070.7446239608	4.98546214690756e-06\\
16191.0056811961	4.91268396230842e-06\\
16312.166679425	4.84063351475297e-06\\
16434.234353112	4.7693076811584e-06\\
16557.2154871169	4.69870339743022e-06\\
16681.116917072	4.62881765171899e-06\\
16805.9455297624	4.55964747796658e-06\\
16931.7082635086	4.49118994973494e-06\\
17058.4121085521	4.42344217431025e-06\\
17186.0641074439	4.35640128707641e-06\\
17314.6713554361	4.29006444615031e-06\\
17444.2410008763	4.22442882727307e-06\\
17574.7802456044	4.15949161895035e-06\\
17706.2963453539	4.09525001783523e-06\\
17838.7966101542	4.03170122434779e-06\\
17972.2884047374	3.96884243852457e-06\\
18106.7791489477	3.90667085609209e-06\\
18242.2763181536	3.8451836647584e-06\\
18378.7874436634	3.78437804071631e-06\\
18516.3201131441	3.72425114535283e-06\\
18654.8819710428	3.66480012215873e-06\\
18794.480719012	3.60602209383254e-06\\
18935.1241163369	3.54791415957348e-06\\
19076.8199803677	3.49047339255755e-06\\
19219.5761869535	3.43369683759149e-06\\
19363.4006708799	3.37758150893931e-06\\
19508.3014263109	3.32212438831564e-06\\
19654.286507232	3.26732242304152e-06\\
19801.3640278989	3.21317252435657e-06\\
19949.5421632881	3.15967156588329e-06\\
20098.8291495512	3.10681638223818e-06\\
20249.233284473	3.0546037677847e-06\\
20400.7629279322	3.00303047552387e-06\\
20553.4265023667	2.95209321611708e-06\\
20707.2324932414	2.90178865703708e-06\\
20862.1894495195	2.85211342184245e-06\\
21018.3059841386	2.80306408957104e-06\\
21175.5907744885	2.75463719424808e-06\\
21334.0525628939	2.70682922450491e-06\\
21493.7001571006	2.65963662330364e-06\\
21654.5424307645	2.61305578776431e-06\\
21816.5883239452	2.56708306908984e-06\\
21979.8468436028	2.5217147725854e-06\\
22144.3270640985	2.47694715776797e-06\\
22310.0381276992	2.43277643856237e-06\\
22476.9892450851	2.38919878358032e-06\\
22645.1896958625	2.34621031647841e-06\\
22814.6488290788	2.30380711639185e-06\\
22985.3760637425	2.26198521844038e-06\\
23157.3808893468	2.22074061430277e-06\\
23330.6728663967	2.18006925285685e-06\\
23505.261626941	2.13996704088162e-06\\
23681.1568751072	2.1004298438183e-06\\
23858.3683876408	2.06145348658725e-06\\
24036.9060144492	2.0230337544576e-06\\
24216.7796791487	1.98516639396679e-06\\
24397.9993796163	1.94784711388693e-06\\
24580.5751885457	1.91107158623518e-06\\
24764.5172540065	1.87483544732566e-06\\
24949.8358000088	1.83913429885961e-06\\
25136.5411270713	1.80396370905178e-06\\
25324.6436127937	1.76931921379014e-06\\
25514.153712434	1.73519631782632e-06\\
25705.0819594888	1.70159049599484e-06\\
25897.4389662797	1.66849719445799e-06\\
26091.2354245425	1.63591183197481e-06\\
26286.4821060218	1.60382980119149e-06\\
26483.1898630694	1.57224646995096e-06\\
26681.3696292481	1.5411571826199e-06\\
26881.0324199387	1.51055726143078e-06\\
27082.189332953	1.48044200783696e-06\\
27284.8515491498	1.45080670387924e-06\\
27489.0303330572	1.42164661356139e-06\\
27694.7370334982	1.39295698423346e-06\\
27901.9830842215	1.3647330479806e-06\\
28110.7800045375	1.33697002301597e-06\\
28321.1393999579	1.30966311507602e-06\\
28533.0729628413	1.28280751881643e-06\\
28746.5924730428	1.25639841920727e-06\\
28961.7097985688	1.23043099292601e-06\\
29178.4368962368	1.2049004097465e-06\\
29396.78581234	1.17980183392316e-06\\
29616.7686833165	1.15513042556845e-06\\
29838.3977364244	1.13088134202278e-06\\
30061.6852904209	1.10704973921538e-06\\
30286.6437562476	1.08363077301499e-06\\
30513.2856377198	1.06061960056931e-06\\
30741.6235322216	1.03801138163214e-06\\
30971.6701314066	1.01580127987693e-06\\
31203.4382219025	9.93984464196155e-07\\
31436.9406860226	9.72556109985204e-07\\
31672.1905024814	9.51511400410069e-07\\
31909.2007471158	9.30845527657997e-07\\
32147.9845936127	9.10553694170132e-07\\
32388.5553142404	8.90631113855628e-07\\
32630.9262805866	8.71073013286264e-07\\
32875.1109643017	8.51874632871031e-07\\
33121.1229378475	8.33031228010046e-07\\
33368.9758752518	8.14538070227073e-07\\
33618.6835528682	7.96390448280247e-07\\
33870.2598501416	7.78583669250394e-07\\
34123.7187503806	7.61113059606405e-07\\
34379.0743415334	7.43973966247414e-07\\
34636.340816972	7.27161757521115e-07\\
34895.5324762807	7.10671824218039e-07\\
35156.6637260502	6.94499580541397e-07\\
35419.7490806799	6.78640465052075e-07\\
35684.8031631832	6.63089941588736e-07\\
35951.840706001	6.47843500162527e-07\\
36220.8765518205	6.32896657826377e-07\\
36491.9256543999	6.18244959518638e-07\\
36765.0030794002	6.03883978880857e-07\\
37040.1240052216	5.89809319049721e-07\\
37317.3037238483	5.76016613422894e-07\\
37596.5576416978	5.6250152639882e-07\\
37877.9012804769	5.49259754090412e-07\\
38161.3502780454	5.36287025012584e-07\\
38446.9203892846	5.23579100743685e-07\\
38734.6274869731	5.11131776560874e-07\\
39024.4875626694	4.98940882049426e-07\\
39316.5167276	4.87002281686188e-07\\
39610.7312135559	4.75311875397133e-07\\
39907.1473737941	4.63865599089262e-07\\
40205.7816839466	4.52659425156963e-07\\
40506.6507429366	4.41689362962943e-07\\
40809.7712739007	4.30951459294036e-07\\
41115.1601251185	4.2044179879195e-07\\
41422.8342709494	4.10156504359293e-07\\
41732.8108127753	4.00091737541066e-07\\
42045.1069799522	3.90243698881845e-07\\
42359.7401307671	3.80608628258995e-07\\
42676.7277534027	3.71182805192159e-07\\
42996.0874669105	3.61962549129275e-07\\
43317.8370221886	3.52944219709552e-07\\
43641.9943029698	3.44124217003584e-07\\
43968.5773268145	3.35498981731038e-07\\
44297.6042461126	3.27064995456237e-07\\
44629.0933490932	3.18818780761936e-07\\
44963.0630608394	3.10756901401768e-07\\
45299.531944314	3.02875962431608e-07\\
45638.5187013907	2.95172610320327e-07\\
45980.0421738933	2.87643533040298e-07\\
46324.1213446437	2.80285460138012e-07\\
46670.775338516	2.73095162785293e-07\\
47020.0234235008	2.66069453811413e-07\\
47371.885011775	2.59205187716612e-07\\
47726.3796607813	2.524992606674e-07\\
48083.5270743154	2.45948610474044e-07\\
48443.347103621	2.39550216550717e-07\\
48805.8597484928	2.33301099858706e-07\\
49171.0851583894	2.27198322833088e-07\\
49539.0436335514	2.21238989293379e-07\\
49909.7556261313	2.15420244338507e-07\\
50283.2417413299	2.09739274226601e-07\\
50659.5227385407	2.0419330624003e-07\\
51038.6195325055	1.98779608536077e-07\\
51420.5531944749	1.9349548998377e-07\\
51805.3449533811	1.88338299987222e-07\\
52193.0161970173	1.83305428295971e-07\\
52583.5884732259	1.7839430480273e-07\\
52977.0834910972	1.73602399328983e-07\\
53373.5231221757	1.68927221398854e-07\\
53772.929401675	1.64366320001695e-07\\
54175.3245297041	1.59917283343768e-07\\
54580.7308724997	1.55577738589516e-07\\
54989.1709636708	1.51345351592771e-07\\
55400.6675054502	1.4721782661835e-07\\
55815.2433699566	1.43192906054446e-07\\
56232.9216004666	1.39268370116202e-07\\
56653.725412694	1.35442036540899e-07\\
57077.6781960819	1.31711760275113e-07\\
57504.8035151016	1.28075433154273e-07\\
57935.1251105625	1.24530983574982e-07\\
58368.6669009326	1.21076376160481e-07\\
58805.4529836665	1.17709611419652e-07\\
59245.5076365459	1.14428725399894e-07\\
59688.8553190289	1.11231789334271e-07\\
60135.5206736089	1.08116909283269e-07\\
60585.528527185	1.0508222577151e-07\\
61038.9038924417	1.02125913419781e-07\\
61495.6719692387	9.92461805727078e-08\\
61955.8581460127	9.64412689223972e-08\\
62419.4880011873	9.3709453128398e-08\\
62886.5873045955	9.10490404342605e-08\\
63357.182018912	8.84583702810393e-08\\
63831.2983010957	8.59358139180285e-08\\
64308.9625038448	8.34797740110172e-08\\
64790.2011770598	8.10886842483828e-08\\
65275.0410693208	7.87610089452704e-08\\
65763.5091293736	7.64952426461617e-08\\
66255.6325076273	7.42899097260901e-08\\
66751.438557664	7.21435639907596e-08\\
67250.9548377589	7.00547882758337e-08\\
67754.209112412	6.80221940456304e-08\\
68261.2293538915	6.60444209914694e-08\\
68772.0437437882	6.41201366299043e-08\\
69286.6806745825	6.22480359010597e-08\\
69805.1687512222	6.04268407673008e-08\\
70327.5367927121	5.86552998124433e-08\\
70853.8138337169	5.69321878417009e-08\\
71384.0291261735	5.52563054825873e-08\\
71918.2121409184	5.36264787869406e-08\\
72456.392569325	5.20415588342716e-08\\
72998.6003249533	5.05004213366082e-08\\
73544.8655452147	4.90019662449983e-08\\
74095.2185930443	4.75451173578554e-08\\
74649.690058591	4.61288219312795e-08\\
75208.3107609164	4.47520502915249e-08\\
75771.111749708	4.34137954497508e-08\\
76338.1243070055	4.21130727191907e-08\\
76909.3799489393	4.08489193348811e-08\\
77484.9104274818	3.96203940760705e-08\\
78064.7477322134	3.8426576891426e-08\\
78648.9240920989	3.72665685271632e-08\\
79237.4719772807	3.61394901581938e-08\\
79830.4241008822	3.50444830224048e-08\\
80427.8134208264	3.39807080581631e-08\\
81029.6731416689	3.29473455451334e-08\\
81636.0367164413	3.19435947485062e-08\\
82246.937848513	3.09686735667046e-08\\
82862.4104934629	3.00218181826569e-08\\
83482.488860967	2.91022827187012e-08\\
84107.2074167008	2.82093388951875e-08\\
84736.6008842538	2.73422756928427e-08\\
85370.7042470603	2.65003990189527e-08\\
86009.5527503438	2.56830313774145e-08\\
86653.1819030753	2.48895115427083e-08\\
87301.627479948	2.41191942378282e-08\\
87954.9255233655	2.33714498162184e-08\\
88613.1123454441	2.26456639477449e-08\\
89276.2245300332	2.19412373087335e-08\\
89944.2989347464	2.12575852761093e-08\\
90617.3726930118	2.05941376256508e-08\\
91295.4832161355	1.99503382343884e-08\\
91978.6681953803	1.9325644787159e-08\\
92666.9656040625	1.87195284873304e-08\\
93360.4136996602	1.81314737717099e-08\\
94059.0510259418	1.75609780296377e-08\\
94762.9164151073	1.70075513262743e-08\\
95472.0489899465	1.64707161300816e-08\\
96186.4881660146	1.59500070444942e-08\\
96906.2736538223	1.54449705437782e-08\\
97631.4454610423	1.49551647130717e-08\\
98362.0438947356	1.44801589925922e-08\\
99098.1095635883	1.40195339260065e-08\\
99839.6833801717	1.35728809129415e-08\\
100586.806563215	1.31398019656238e-08\\
101339.520639896	1.27199094696294e-08\\
102097.867448152	1.23128259487183e-08\\
102861.889138999	1.19181838337391e-08\\
103631.628178883	1.15356252355713e-08\\
104407.127352034	1.11648017220844e-08\\
105188.429762846	1.08053740990837e-08\\
105975.578838274	1.04570121952141e-08\\
106768.618330246	1.01193946507905e-08\\
107567.592318097	9.79220871052336e-09\\
108372.545211017	9.4751500201044e-09\\
109183.521750518	9.16792242661937e-09\\
110000.567012926	8.87023778274988e-09\\
110823.726411883	8.58181575472865e-09\\
111653.04570087	8.30238363400972e-09\\
112488.570975754	8.031676152613e-09\\
113330.348677345	7.76943530210621e-09\\
114178.425593984	7.51541015617962e-09\\
115032.848864136	7.26935669677507e-09\\
115893.665979016	7.03103764372556e-09\\
116760.924785228	6.80022228786188e-09\\
117634.673487419	6.57668632754507e-09\\
118514.960650966	6.36021170857758e-09\\
119401.835204671	6.15058646745131e-09\\
120295.34644348	5.94760457788636e-09\\
121195.544031226	5.75106580061496e-09\\
122102.478003387	5.56077553636602e-09\\
123016.198769868	5.37654468200394e-09\\
123936.757117803	5.19818948977493e-09\\
124864.204214378	5.02553142961649e-09\\
125798.591609674	4.85839705448132e-09\\
126739.971239535	4.69661786863128e-09\\
127688.395428448	4.54003019885394e-09\\
128643.916892463	4.38847506855479e-09\\
129606.588742111	4.2417980746803e-09\\
130576.464485363	4.09984926742325e-09\\
131553.598030603	3.96248303266581e-09\\
132538.043689621	3.82955797711336e-09\\
133529.856180639	3.70093681607289e-09\\
134529.090631344	3.57648626383068e-09\\
135535.802581959	3.45607692658321e-09\\
136550.047988325	3.33958319787591e-09\\
137571.883225014	3.226883156505e-09\\
138601.365088463	3.11785846683685e-09\\
139638.550800127	3.01239428150102e-09\\
140683.498009666	2.91037914641251e-09\\
141736.264798142	2.81170490807885e-09\\
142796.909681254	2.71626662314963e-09\\
143865.491612584	2.623962470164e-09\\
144942.069986881	2.53469366345449e-09\\
146026.704643354	2.44836436916436e-09\\
147119.455869007	2.36488162333628e-09\\
148220.384401982	2.28415525203188e-09\\
149329.55143494	2.20609779343995e-09\\
150447.018618462	2.13062442193364e-09\\
151572.84806447	2.05765287403634e-09\\
152707.102349689	1.98710337625643e-09\\
153849.844519117	1.91889857475219e-09\\
155001.138089531	1.85296346678819e-09\\
156161.047053023	1.78922533394464e-09\\
157329.635880547	1.72761367704314e-09\\
158506.969525512	1.66806015275052e-09\\
159693.113427386	1.61049851182536e-09\\
160888.133515336	1.55486453897061e-09\\
162092.096211894	1.50109599425677e-09\\
163305.068436644	1.44913255608144e-09\\
164527.117609946	1.39891576562975e-09\\
165758.311656683	1.35038897280285e-09\\
166998.719010032	1.30349728358056e-09\\
168248.408615274	1.2581875087855e-09\\
169507.449933624	1.21440811421674e-09\\
170775.912946089	1.17210917212093e-09\\
172053.868157361	1.13124231397e-09\\
173341.386599734	1.09176068451478e-09\\
174638.539837053	1.05361889708434e-09\\
175945.399968693	1.01677299010173e-09\\
177262.039633563	9.81180384787005e-10\\
178588.532014148	9.46799844018855e-10\\
179924.950840571	9.13591432327459e-10\\
181271.370394698	8.81516476990438e-10\\
182627.865514261	8.50537530205543e-10\\
183994.51159702	8.20618332313462e-10\\
185371.384604954	7.91723776044804e-10\\
186758.561068483	7.63819871766169e-10\\
188156.118090722	7.36873713700077e-10\\
189564.133351766	7.10853447094684e-10\\
190982.685113006	6.85728236319285e-10\\
192411.852221483	6.61468233862177e-10\\
193851.71411427	6.38044550208076e-10\\
195302.350822881	6.15429224572644e-10\\
196763.842977729	5.93595196472021e-10\\
198236.2718126	5.72516278106167e-10\\
199719.719169172	5.52167127534644e-10\\
201214.267501562	5.32523222624517e-10\\
202719.999880911	5.13560835750145e-10\\
204237	4.95257009225103e-10\\
};
\addlegendentry{Adatt. BLN appr. (\textit{ML})}


\addplot[area legend, draw=none, fill=mycolor1, fill opacity=0.15, forget plot]
table[row sep=crcr] {%
x	y\\
119	1.22416820860006e-06\\
119.890504214077	1.24633723002206e-06\\
120.787672274837	1.26890760056988e-06\\
121.691554049369	1.29188791286206e-06\\
122.602199777928	1.31528700982365e-06\\
123.519660076729	1.33911399337072e-06\\
124.44398594076	1.36337823338154e-06\\
125.375228746615	1.38808937696214e-06\\
126.313440255352	1.41325735801365e-06\\
127.258672615369	1.43889240710933e-06\\
128.2109783653	1.46500506168903e-06\\
129.17041043694	1.49160617657892e-06\\
130.137022158185	1.51870693484447e-06\\
131.110867255994	1.54631885898481e-06\\
132.091999859379	1.57445382247646e-06\\
133.080474502409	1.60312406167464e-06\\
134.076346127247	1.6323421880804e-06\\
135.0796700872	1.66212120098183e-06\\
136.090502149795	1.69247450047762e-06\\
137.108898499881	1.7234159008913e-06\\
138.134915742751	1.75495964458461e-06\\
139.168610907289	1.78712041617818e-06\\
140.21004144914	1.81991335718807e-06\\
141.259265253899	1.85335408108634e-06\\
142.316340640336	1.88745868879416e-06\\
143.381326363632	1.92224378461554e-06\\
144.454281618647	1.95772649262013e-06\\
145.535266043209	1.99392447348303e-06\\
146.624339721429	2.03085594178998e-06\\
147.721563187044	2.06853968381565e-06\\
148.826997426776	2.10699507578317e-06\\
149.940703883725	2.14624210261243e-06\\
151.062744460785	2.18630137716507e-06\\
152.193181524082	2.22719415999325e-06\\
153.332077906443	2.2689423795998e-06\\
154.479496910888	2.31156865321667e-06\\
155.635502314145	2.35509630810853e-06\\
156.800158370201	2.39954940340814e-06\\
157.97352981387	2.44495275248979e-06\\
159.15568186439	2.49133194588696e-06\\
160.34668022905	2.53871337475966e-06\\
161.546591106842	2.58712425491713e-06\\
162.755481192139	2.63659265140055e-06\\
163.973417678405	2.68714750363065e-06\\
165.200468261928	2.73881865112406e-06\\
166.436701145581	2.79163685978228e-06\\
167.682185042617	2.8456338487564e-06\\
168.936989180483	2.90084231789005e-06\\
170.201183304674	2.95729597574286e-06\\
171.474837682604	3.01502956819564e-06\\
172.758023107516	3.07407890763828e-06\\
174.050810902413	3.13448090274027e-06\\
175.353272924028	3.1962735888035e-06\\
176.665481566809	3.25949615869547e-06\\
177.987509766954	3.32418899436124e-06\\
179.319431006454	3.39039369891061e-06\\
180.661319317186	3.45815312927666e-06\\
182.013249285024	3.52751142944073e-06\\
183.375296053983	3.59851406421779e-06\\
184.7475353304	3.67120785359515e-06\\
186.13004338714	3.74564100761628e-06\\
187.522897067834	3.82186316180054e-06\\
188.926173791152	3.89992541308799e-06\\
190.339951555105	3.97988035629747e-06\\
191.764308941382	4.06178212108485e-06\\
193.199325119717	4.14568640938643e-06\\
194.645079852288	4.23165053333157e-06\\
196.101653498152	4.31973345360678e-06\\
197.569127017711	4.40999581825176e-06\\
199.047581977213	4.50250000186654e-06\\
200.537100553285	4.59731014520677e-06\\
202.037765537499	4.69449219514257e-06\\
203.549660340977	4.79411394495444e-06\\
205.072868999023	4.89624507493753e-06\\
206.6074761758	5.00095719328397e-06\\
208.15356716903	5.10832387721028e-06\\
209.711227914737	5.21842071429509e-06\\
211.280544992026	5.33132534399013e-06\\
212.86160562789	5.44711749926465e-06\\
214.454497702065	5.56587904834174e-06\\
216.059309751909	5.68769403648175e-06\\
217.676130977326	5.81264872776629e-06\\
219.305051245723	5.94083164683277e-06\\
220.946161097006	6.07233362050767e-06\\
222.599551748611	6.20724781928328e-06\\
224.265315100576	6.34566979858022e-06\\
225.943543740646	6.48769753973532e-06\\
227.634330949423	6.63343149065106e-06\\
229.33777070555	6.78297460604064e-06\\
231.05395769093	6.93643238719885e-06\\
232.782987295997	7.09391292122713e-06\\
234.524955625009	7.25552691963693e-06\\
236.279959501398	7.42138775625377e-06\\
238.048096473146	7.59161150434065e-06\\
239.829464818207	7.76631697285654e-06\\
241.624163549975	7.94562574176386e-06\\
243.432292422782	8.12966219629405e-06\\
245.253951937446	8.3185535600801e-06\\
247.089243346851	8.51242992705969e-06\\
248.938268661586	8.71142429205223e-06\\
250.801130655604	8.91567257990901e-06\\
252.677932871941	9.1253136731347e-06\\
254.568779628468	9.34048943787471e-06\\
256.473776023691	9.56134474816324e-06\\
258.393027942593	9.78802750832218e-06\\
260.326642062517	1.00206886734026e-05\\
262.274725859099	1.02594822675566e-05\\
264.237387612236	1.05045654002271e-05\\
266.214736412113	1.07560982800446e-05\\
268.206882165259	1.10142442263142e-05\\
270.213935600659	1.12791696779819e-05\\
272.236008275909	1.1551044199965e-05\\
274.273212583414	1.18300404867341e-05\\
276.325661756641	1.21163343630334e-05\\
278.393469876404	1.24101047816284e-05\\
280.476751877215	1.27115338179713e-05\\
282.575623553662	1.30208066616739e-05\\
284.690201566855	1.33381116046845e-05\\
286.820603450903	1.3663640026062e-05\\
288.96694761945	1.39975863732482e-05\\
291.129353372257	1.4340148139738e-05\\
293.307940901833	1.46915258390541e-05\\
295.502831300113	1.50519229749332e-05\\
297.714146565192	1.54215460076365e-05\\
299.942009608104	1.58006043163005e-05\\
302.186544259657	1.61893101572472e-05\\
304.447875277308	1.6587878618179e-05\\
306.726128352108	1.69965275681857e-05\\
309.02143011568	1.74154776034974e-05\\
311.333908147261	1.78449519889189e-05\\
313.663690980792	1.8285176594889e-05\\
316.010908112063	1.87363798301085e-05\\
318.375690005912	1.91987925696876e-05\\
320.758168103475	1.96726480787677e-05\\
323.158474829489	2.01581819315729e-05\\
325.57674359966	2.06556319258559e-05\\
328.013108828071	2.11652379927013e-05\\
330.467705934659	2.16872421016563e-05\\
332.940671352735	2.22218881611594e-05\\
335.432142536577	2.27694219142434e-05\\
337.942257969062	2.33300908294882e-05\\
340.471157169366	2.39041439872054e-05\\
343.018980700718	2.44918319608353e-05\\
345.585870178217	2.50934066935421e-05\\
348.171968276697	2.57091213699922e-05\\
350.777418738662	2.63392302833052e-05\\
353.402366382274	2.69839886971666e-05\\
356.046957109401	2.76436527030944e-05\\
358.711337913731	2.83184790728526e-05\\
361.395656888935	2.90087251060056e-05\\
364.100063236907	2.97146484726127e-05\\
366.824707276051	3.04365070510571e-05\\
369.569740449639	3.11745587610141e-05\\
372.335315334224	3.19290613915565e-05\\
375.12158564813	3.27002724244066e-05\\
377.928706259986	3.34884488523373e-05\\
380.75683319734	3.42938469927364e-05\\
383.60612365533	3.5116722296345e-05\\
386.476736005421	3.59573291511887e-05\\
389.368829804207	3.68159206817212e-05\\
392.282565802281	3.76927485432049e-05\\
395.21810595317	3.85880627113612e-05\\
398.175613422337	3.95021112673217e-05\\
401.155252596246	4.04351401779244e-05\\
404.157189091507	4.13873930713992e-05\\
407.181589764074	4.23591110084955e-05\\
410.228622718523	4.33505322491146e-05\\
413.298457317395	4.43618920145113e-05\\
416.391264190611	4.53934222451397e-05\\
419.507215244952	4.64453513542273e-05\\
422.646483673618	4.7517903977169e-05\\
425.809243965854	4.86113007168382e-05\\
428.995671916648	4.97257578849279e-05\\
432.205944636502	5.08614872394383e-05\\
435.440240561274	5.20186957184371e-05\\
438.698739462102	5.3197585170236e-05\\
441.981622455389	5.43983520801265e-05\\
445.289072012877	5.56211872938353e-05\\
448.621271971783	5.68662757378681e-05\\
451.978407545023	5.81337961369234e-05\\
455.360665331498	5.94239207285613e-05\\
458.768233326478	6.07368149753356e-05\\
462.201300932039	6.20726372745929e-05\\
465.660058967601	6.34315386661692e-05\\
469.144699680525	6.48136625382108e-05\\
472.655416756805	6.62191443313689e-05\\
476.192405331832	6.7648111241619e-05\\
479.755862001239	6.91006819219724e-05\\
483.345984831829	7.05769661833573e-05\\
486.962973372584	7.20770646949513e-05\\
490.607028665758	7.3601068684269e-05\\
494.278353258049	7.51490596373026e-05\\
497.977151211859	7.67211089990421e-05\\
501.703628116633	7.83172778746902e-05\\
505.457991100293	7.99376167319178e-05\\
509.240448840744	8.15821651044966e-05\\
513.051211577476	8.32509512976658e-05\\
516.890491123249	8.49439920955943e-05\\
520.758500875867	8.66612924713081e-05\\
524.655455830038	8.84028452994591e-05\\
528.581572589324	9.01686310723205e-05\\
532.537069378183	9.19586176194051e-05\\
536.522166054094	9.37727598310967e-05\\
540.537084119782	9.5610999386708e-05\\
544.582046735526	9.74732644873696e-05\\
548.657278731565	9.93594695941678e-05\\
552.763006620593	0.000101269515171952\\
556.899458610353	0.000103203287439237\\
561.066864616317	0.000105160658124625\\
565.265456274466	0.000107141484230189\\
569.495466954168	0.000109145607802232\\
573.757131771146	0.00011117285570989\\
578.050687600549	0.000113223039431983\\
582.376373090114	0.000115295954852576\\
586.734428673439	0.000117391382065688\\
591.125096583335	0.000119509085189573\\
595.548620865302	0.000121648812191034\\
600.005247391086	0.000123810294720187\\
604.495223872346	0.000125993247956126\\
609.018799874428	0.000128197370463912\\
613.576226830228	0.000130422344063317\\
618.167758054176	0.000132667833709758\\
622.793648756308	0.000134933487387822\\
627.454156056458	0.000137218936017819\\
632.149538998544	0.000139523793375758\\
636.880058564972	0.000141847656027154\\
641.645977691138	0.000144190103275064\\
646.447561280041	0.000146550697122727\\
651.285076217013	0.000148928982251208\\
656.158791384547	0.000151324486012395\\
661.06897767725	0.000153736718437716\\
666.015908016889	0.00015616517226293\\
670.999857367572	0.000158609322969313\\
676.021102751025	0.000161068628841586\\
681.079923261988	0.000163542531042871\\
686.176600083736	0.000166030453706995\\
691.311416503699	0.000168531804048405\\
696.484657929211	0.000171045972489983\\
701.696611903378	0.000173572332808996\\
706.947568121055	0.000176110242301419\\
712.237818444947	0.00017865904196485\\
717.56765692184	0.000181218056700221\\
722.937379798933	0.000183786595532468\\
728.347285540317	0.00018636395185033\\
733.797674843554	0.000188949403665405\\
739.288850656394	0.000191542213890575\\
744.821118193618	0.000194141630637887\\
750.394784953994	0.000196746887535953\\
756.01016073738	0.000199357204066883\\
761.667557661931	0.00020197178592278\\
767.367290181458	0.00020458982538173\\
773.109675102898	0.000207210501703242\\
778.89503160393	0.000209832981543023\\
784.72368125071	0.00021245641938693\\
790.59594801575	0.000215079958003905\\
796.512158295919	0.000217702728917646\\
802.472640930591	0.000220323852896711\\
808.47772721992	0.000222942440462665\\
814.527750943253	0.000225557592415858\\
820.623048377686	0.000228168400378292\\
826.763958316753	0.000230773947352992\\
832.950822089257	0.000233373308299158\\
839.183983578243	0.000235965550722307\\
845.463789240111	0.000238549735278464\\
851.790588123873	0.00024112491639134\\
858.164731890556	0.000243690142881286\\
864.586574832748	0.00024624445860464\\
871.056473894285	0.000248786903101924\\
877.574788690099	0.000251316512253118\\
884.141881526201	0.000253832318938066\\
890.758117419823	0.000256333353699795\\
897.423864119701	0.000258818645408327\\
904.139492126523	0.000261287221922282\\
910.905374713515	0.000263738110745316\\
917.721887947192	0.000266170339674182\\
924.589410708265	0.000268582937434939\\
931.508324712691	0.000270974934303607\\
938.479014532895	0.000273345362707362\\
945.501867619149	0.000275693257802194\\
952.577274321103	0.000278017658022897\\
959.705627909479	0.000280317605601231\\
966.88732459794	0.00028259214704825\\
974.122763565099	0.000284840333597035\\
981.412346976721	0.000287061221602531\\
988.756480008064	0.000289253872895803\\
996.155570866409	0.000291417355090921\\
1003.61003081374	0.00029355074184376\\
1011.12027418962	0.000295653113063365\\
1018.6867184342	0.000297723555078108\\
1026.30978411142	0.000299761160760677\\
1033.98989493243	0.000301765029617894\\
1041.72747777907	0.00030373426785344\\
1049.52296272766	0.000305667988413634\\
1057.37678307286	0.000307565311028381\\
1065.2893753518	0.000309425362261161\\
1073.26117936829	0.000311247275583263\\
1081.2926382173	0.000313030191488356\\
1089.38419830959	0.000314773257663685\\
1097.53630939651	0.000316475629233699\\
1105.749424595	0.000318136469090637\\
1114.02400041277	0.000319754948324538\\
1122.3604967737	0.000321330246762391\\
1130.75937704337	0.00032286155362276\\
1139.22110805483	0.000324348068288442\\
1147.74616013456	0.000325789001195666\\
1156.3350071286	0.000327183574834354\\
1164.98812642888	0.000328531024850174\\
1173.70599899975	0.000329830601235781\\
1182.48910940477	0.000331081569595932\\
1191.33794583355	0.000332283212469193\\
1200.25300012896	0.000333434830687764\\
1209.23476781446	0.000334535744756608\\
1218.28374812157	0.000335585296233419\\
1227.40044401774	0.000336582849092029\\
1236.58536223419	0.000337527791053392\\
1245.83901329415	0.000338419534870242\\
1255.16191154121	0.000339257519553706\\
1264.5545751679	0.000340041211532415\\
1274.01752624451	0.000340770105736945\\
1283.55129074812	0.00034144372660453\\
1293.15639859177	0.00034206162900097\\
1302.83338365401	0.000342623399058329\\
1312.5827838085	0.000343128654928521\\
1322.40514095393	0.000343577047453989\\
1332.30100104415	0.000343968260757664\\
1342.27091411851	0.000344302012755023\\
1352.31543433242	0.00034457805559154\\
1362.43511998817	0.000344796176009109\\
1372.63053356595	0.000344956195645126\\
1382.90224175512	0.000345057971267933\\
1393.2508154857	0.000345101394952238\\
1403.67682996012	0.000345086394197964\\
1414.18086468518	0.000345012931995783\\
1424.76350350425	0.000344881006842352\\
1435.42533462974	0.000344690652708047\\
1446.16695067579	0.000344441938959712\\
1456.98894869122	0.000344134970240733\\
1467.89193019267	0.000343769886310521\\
1478.8765011981	0.000343346861845236\\
1489.94327226042	0.000342866106201463\\
1501.09285850146	0.000342327863144322\\
1512.32587964614	0.000341732410541385\\
1523.64296005691	0.000341080060023616\\
1535.0447287685	0.000340371156614469\\
1546.53181952283	0.000339606078328162\\
1558.10487080425	0.000338785235738099\\
1569.76452587505	0.000337909071516308\\
1581.51143281119	0.000336978059944779\\
1593.34624453832	0.000335992706399483\\
1605.26961886811	0.000334953546807884\\
1617.28221853476	0.000333861147080699\\
1629.38471123188	0.000332716102518674\\
1641.57776964957	0.000331519037195135\\
1653.86207151182	0.000330270603315057\\
1666.2382996142	0.000328971480551433\\
1678.70714186178	0.000327622375359695\\
1691.26929130741	0.000326224020270965\\
1703.92544619016	0.000324777173164928\\
1716.67630997424	0.000323282616523118\\
1729.522591388	0.000321741156663426\\
1742.4650044634	0.00032015362295665\\
1755.50426857564	0.000318520867025899\\
1768.64110848317	0.000316843761929699\\
1781.876254368	0.000315123201329632\\
1795.21044187622	0.000313360098643344\\
1808.64441215895	0.000311555386183777\\
1822.17891191353	0.000309710014285466\\
1835.81469342496	0.00030782495041873\\
1849.55251460781	0.000305901178292614\\
1863.39313904828	0.000303939696947388\\
1877.33733604663	0.000301941519837449\\
1891.38588066002	0.000299907673905413\\
1905.53955374551	0.000297839198648206\\
1919.7991420035	0.000295737145175947\\
1934.16543802145	0.000293602575264382\\
1948.63924031791	0.000291436560401636\\
1963.22135338698	0.000289240180830031\\
1977.91258774292	0.000287014524583719\\
1992.71375996528	0.00028476068652285\\
2007.62569274426	0.000282479767365017\\
2022.64921492642	0.000280172872714722\\
2037.7851615608	0.000277841112091603\\
2053.03437394529	0.000275485597958215\\
2068.39769967338	0.000273107444748153\\
2083.87599268133	0.00027070776789538\\
2099.47011329558	0.000268287682865676\\
2115.18092828061	0.000265848304191197\\
2131.00931088707	0.000263390744509209\\
2146.95614090037	0.000260916113606209\\
2163.02230468953	0.000258425517468729\\
2179.20869525649	0.00025592005734229\\
2195.51621228573	0.000253400828800127\\
2211.94576219425	0.000250868920823452\\
2228.49825818201	0.000248325414895233\\
2245.17462028264	0.000245771384109636\\
2261.97577541457	0.000243207892299442\\
2278.90265743261	0.000240635993183944\\
2295.95620717979	0.000238056729539963\\
2313.1373725397	0.000235471132398748\\
2330.44710848916	0.000232880220271618\\
2347.88637715129	0.000230284998407229\\
2365.45614784899	0.000227686458083332\\
2383.15739715885	0.000225085575935817\\
2400.9911089654	0.000222483313327667\\
2418.95827451578	0.000219880615760232\\
2437.05989247487	0.000217278412328916\\
2455.29696898081	0.000214677615225015\\
2473.67051770087	0.000212079119284993\\
2492.18155988785	0.000209483801587989\\
2510.8311244368	0.000206892521101837\\
2529.62024794223	0.000204306118377289\\
2548.54997475574	0.000201725415289606\\
2567.62135704402	0.000199151214826106\\
2586.83545484739	0.000196584300917768\\
2606.1933361387	0.000194025438312492\\
2625.69607688265	0.00019147537248726\\
2645.34476109567	0.000188934829596087\\
2665.14048090611	0.000186404516450447\\
2685.08433661496	0.000183885120528706\\
2705.17743675704	0.000181377310011048\\
2725.42089816257	0.000178881733836425\\
2745.81584601927	0.000176399021778198\\
2766.3634139349	0.000173929784535298\\
2787.06474400025	0.000171474613836007\\
2807.92098685266	0.000169034082551767\\
2828.93330173995	0.000166608744818707\\
2850.10285658484	0.000164199136164972\\
2871.4308280499	0.000161805773642238\\
2892.91840160291	0.000159429155960148\\
2914.56677158281	0.000157069763622749\\
2936.37714126603	0.000154728059066252\\
2958.3507229334	0.000152404486797754\\
2980.48873793751	0.000150099473534757\\
3002.79241677064	0.000147813428345521\\
3025.2629991331	0.000145546742790476\\
3047.90173400216	0.000143299791064993\\
3070.7098797015	0.000141072930143949\\
3093.68870397109	0.000138866499928576\\
3116.83948403772	0.000136680823396096\\
3140.16350668593	0.000134516206752703\\
3163.66206832958	0.0001323729395904\\
3187.33647508389	0.00013025129504821\\
3211.18804283803	0.000128151529978236\\
3235.21809732829	0.000126073885117006\\
3259.42797421173	0.000124018585262464\\
3283.81901914043	0.00012198583945696\\
3308.39258783631	0.000119975841176479\\
3333.15004616647	0.000117988768526331\\
3358.09277021909	0.000116024784443449\\
3383.22214637995	0.000114084036905379\\
3408.53957140944	0.000112166659146006\\
3434.04645252026	0.000110272769877986\\
3459.7442074556	0.000108402473521835\\
3485.63426456793	0.000106555860441555\\
3511.71806289842	0.000104733007186639\\
3537.99705225692	0.000102933976740276\\
3564.47269330252	0.000101158818773528\\
3591.14645762477	9.94075699052238e-05\\
3618.01982782546	9.76802539672823e-05\\
3645.09429760103	9.59768822751776e-05\\
3672.37137182558	9.42974539032026e-05\\
3699.85256663454	9.2641955964201e-05\\
3727.53940950892	9.10103638934047e-05\\
3755.43343936023	8.94026417360064e-05\\
3783.53620661599	8.78187424380867e-05\\
3811.84927330594	8.62586081405069e-05\\
3840.37421314884	8.47221704753679e-05\\
3869.11261163994	8.32093508646301e-05\\
3898.06606613913	8.172006082049e-05\\
3927.23618595969	8.02542022470952e-05\\
3956.62459245778	7.8811667743185e-05\\
3986.23291912252	7.73923409052365e-05\\
4016.06281166681	7.59960966306936e-05\\
4046.1159281188	7.46228014208552e-05\\
4076.39393891404	7.32723136830002e-05\\
4106.89852698833	7.19444840313172e-05\\
4137.63138787127	7.06391555862142e-05\\
4168.59422978048	6.93561642715757e-05\\
4199.78877371658	6.8095339109538e-05\\
4231.21675355883	6.68565025123511e-05\\
4262.8799161615	6.56394705708959e-05\\
4294.78002145095	6.44440533394264e-05\\
4326.91884252352	6.32700551161108e-05\\
4359.29816574399	6.21172747189508e-05\\
4391.91979084494	6.09855057566632e-05\\
4424.78553102675	5.98745368941208e-05\\
4457.89721305839	5.87841521119645e-05\\
4491.25667737898	5.77141309600218e-05\\
4524.86577820005	5.66642488041875e-05\\
4558.72638360862	5.56342770664648e-05\\
4592.84037567103	5.46239834579014e-05\\
4627.20965053756	5.36331322042138e-05\\
4661.83611854782	5.26614842639532e-05\\
4696.72170433693	5.17087975391446e-05\\
4731.86834694246	5.07748270784149e-05\\
4767.27799991228	4.9859325272729e-05\\
4802.95263141311	4.89620420439624e-05\\
4838.89422433987	4.80827250266594e-05\\
4875.10477642598	4.7221119743452e-05\\
4911.58630035433	4.63769697747549e-05\\
4948.34082386919	4.55500169234741e-05\\
4985.37038988889	4.47400013756036e-05\\
5022.6770566194	4.39466618576939e-05\\
5060.2628976687	4.3169735792275e-05\\
5098.13000216207	4.24089594523886e-05\\
5136.28047485818	4.16640681164196e-05\\
5174.7164362661	4.09347962244172e-05\\
5213.44002276314	4.02208775370474e-05\\
5252.45338671363	3.95220452982231e-05\\
5291.7586965885	3.88380324023163e-05\\
5331.3581370859	3.81685715666607e-05\\
5371.25390925253	3.75133955098207e-05\\
5411.44823060603	3.68722371358352e-05\\
5451.94333525824	3.62448297243488e-05\\
5492.74147403939	3.56309071262432e-05\\
5533.84491462315	3.50302039640826e-05\\
5575.25594165272	3.44424558364026e-05\\
5616.97685686784	3.38673995246285e-05\\
5659.00997923266	3.33047732012015e-05\\
5701.35764506468	3.2754316637343e-05\\
5744.02220816459	3.22157714087947e-05\\
5787.00603994713	3.16888810978444e-05\\
5830.31152957285	3.11733914899793e-05\\
5873.94108408097	3.06690507635973e-05\\
5917.8971285231	3.01756096713403e-05\\
5962.18210609809	2.96928217117908e-05\\
6006.79847828778	2.92204432904779e-05\\
6051.74872499389	2.87582338693546e-05\\
6097.03534467575	2.83059561041406e-05\\
6142.66085448928	2.78633759691443e-05\\
6188.62779042682	2.74302628693936e-05\\
6234.93870745815	2.70063897400964e-05\\
6281.59617967246	2.65915331336272e-05\\
6328.60280042144	2.61854732943816e-05\\
6375.9611824634	2.578799422196e-05\\
6423.67395810857	2.53988837232385e-05\\
6471.74377936531	2.50179334539527e-05\\
6520.17331808759	2.46449389504668e-05\\
6568.96526612346	2.42796996524244e-05\\
6618.12233546469	2.39220189169887e-05\\
6667.64725839753	2.35717040253702e-05\\
6717.54278765452	2.32285661823232e-05\\
6767.81169656754	2.2892420509266e-05\\
6818.45677922192	2.25630860316428e-05\\
6869.48085061182	2.22403856611109e-05\\
6920.88674679662	2.1924146173094e-05\\
6972.67732505857	2.16141981801999e-05\\
7024.85546406162	2.13103761019612e-05\\
7077.42406401143	2.10125181313143e-05\\
7130.38604681656	2.07204661981956e-05\\
7183.7443562509	2.04340659305958e-05\\
7237.50195811723	2.0153166613378e-05\\
7291.66184041212	1.9877621145135e-05\\
7346.22701349204	1.96072859933329e-05\\
7401.20051024062	1.93420211479596e-05\\
7456.58538623724	1.9081690073876e-05\\
7512.38471992689	1.88261596620444e-05\\
7568.60161279127	1.85753001797911e-05\\
7625.23918952119	1.83289852202435e-05\\
7682.30059819021	1.80870916510671e-05\\
7739.78901042966	1.78494995626141e-05\\
7797.70762160489	1.76160922155866e-05\\
7856.05965099295	1.73867559883031e-05\\
7914.84834196143	1.71613803236531e-05\\
7974.0769621488	1.6939857675814e-05\\
8033.748803646	1.67220834567977e-05\\
8093.86718317946	1.65079559828908e-05\\
8154.43544229543	1.62973764210443e-05\\
8215.4569475457	1.6090248735267e-05\\
8276.93509067475	1.58864796330701e-05\\
8338.87328880825	1.56859785120083e-05\\
8401.27498464301	1.54886574063605e-05\\
8464.14364663834	1.5294430933988e-05\\
8527.48276920879	1.51032162434078e-05\\
8591.29587291844	1.49149329611135e-05\\
8655.58650467655	1.47295031391774e-05\\
8720.35823793472	1.45468512031633e-05\\
8785.61467288549	1.4366903900377e-05\\
8851.35943666247	1.41895902484811e-05\\
8917.59618354194	1.40148414844994e-05\\
8984.32859514597	1.38425910142316e-05\\
9051.56038064706	1.36727743621008e-05\\
9119.29527697427	1.35053291214527e-05\\
9187.53704902097	1.33401949053235e-05\\
9256.28948985408	1.31773132976938e-05\\
9325.55642092493	1.30166278052426e-05\\
9395.34169228163	1.28580838096135e-05\\
9465.64918278304	1.27016285202061e-05\\
9536.48280031448	1.25472109275009e-05\\
9607.84648200483	1.23947817569271e-05\\
9679.74419444542	1.22442934232791e-05\\
9752.17993391047	1.20956999856865e-05\\
9825.15772657923	1.19489571031425e-05\\
9898.68162875981	1.18040219905902e-05\\
9972.75572711457	1.1660853375569e-05\\
10047.3841388873	1.15194114554187e-05\\
10122.5710121321	1.13796578550383e-05\\
10198.3205259438	1.12415555851957e-05\\
10274.6368906905	1.11050690013814e-05\\
10351.5243482474	1.09701637632007e-05\\
10428.9871722326	1.08368067942938e-05\\
10507.0296682446	1.07049662427763e-05\\
10585.6561741017	1.05746114421888e-05\\
10664.8710600833	1.04457128729445e-05\\
10744.6787291723	1.03182421242647e-05\\
10825.0836173003	1.0192171856589e-05\\
10906.0901935939	1.00674757644519e-05\\
10987.7029606233	9.94412853981386e-06\\
11069.9264546524	9.8221058358388e-06\\
11152.765245891	9.70138423111261e-06\\
11236.2239387488	9.5819411942977e-06\\
11320.3071720913	9.46375504922395e-06\\
11405.019619498	9.34680494041968e-06\\
11490.3659895215	9.23107079909176e-06\\
11576.3510259497	9.11653330956892e-06\\
11662.9795080693	9.00317387623006e-06\\
11750.2562509318	8.89097459094554e-06\\
11838.1861056203	8.77991820106776e-06\\
11926.7739595202	8.66998807801586e-06\\
12016.0247365899	8.56116818650705e-06\\
12105.9433976352	8.45344305449707e-06\\
12196.5349405845	8.34679774389924e-06\\
12287.8044007671	8.24121782215928e-06\\
12379.7568511926	8.13668933477098e-06\\
12472.3974028332	8.03319877882029e-06\\
12565.7312049077	7.930733077651e-06\\
12659.7634451676	7.82927955674405e-06\\
12754.4993501855	7.72882592090102e-06\\
12849.9441856459	7.62936023281791e-06\\
12946.1032566372	7.53087089312545e-06\\
13042.9819079474	7.43334662196204e-06\\
13140.5855243605	7.33677644212967e-06\\
13238.9195309561	7.24114966386514e-06\\
13337.9893934111	7.14645587123935e-06\\
13437.8006183031	7.05268491017403e-06\\
13538.3587534167	6.95982687804181e-06\\
13639.669388052	6.86787211479254e-06\\
13741.738153335	6.77681119552325e-06\\
13844.5707225307	6.68663492438887e-06\\
13948.1728113584	6.59733432973007e-06\\
14052.5501783096	6.50890066027754e-06\\
14157.7086249677	6.42132538228043e-06\\
14263.6539963308	6.33460017739566e-06\\
14370.3921811364	6.24871694117268e-06\\
14477.9291121888	6.16366778196697e-06\\
14586.2707666889	6.07944502012021e-06\\
14695.4231665662	5.99604118725401e-06\\
14805.3923788139	5.91344902553538e-06\\
14916.1845158256	5.83166148678708e-06\\
15027.8057357356	5.75067173133266e-06\\
15140.2622427609	5.67047312648409e-06\\
15253.5602875458	5.59105924459903e-06\\
15367.70616751	5.51242386065329e-06\\
15482.7062271979	5.43456094929179e-06\\
15598.5668586318	5.35746468133985e-06\\
15715.2945016669	5.28112941977e-06\\
15832.8956443492	5.20554971513538e-06\\
15951.3768232763	5.13072030049132e-06\\
16070.7446239608	5.0566360858365e-06\\
16191.0056811961	4.98329215211415e-06\\
16312.166679425	4.91068374481725e-06\\
16434.234353112	4.83880626724823e-06\\
16557.2154871169	4.76765527348425e-06\\
16681.116917072	4.69722646110078e-06\\
16805.9455297624	4.6275156637059e-06\\
16931.7082635086	4.55851884333666e-06\\
17058.4121085521	4.49023208276577e-06\\
17186.0641074439	4.4226515777659e-06\\
17314.6713554361	4.35577362937354e-06\\
17444.2410008763	4.28959463619286e-06\\
17574.7802456044	4.22411108677503e-06\\
17706.2963453539	4.15931955210514e-06\\
17838.7966101542	4.09521667822569e-06\\
17972.2884047374	4.0317991790207e-06\\
18106.7791489477	3.96906382918266e-06\\
18242.2763181536	3.90700745737995e-06\\
18378.7874436634	3.84562693963995e-06\\
18516.3201131441	3.78491919296015e-06\\
18654.8819710428	3.72488116915698e-06\\
18794.480719012	3.66550984895951e-06\\
18935.1241163369	3.60680223635358e-06\\
19076.8199803677	3.54875535317934e-06\\
19219.5761869535	3.4913662339843e-06\\
19363.4006708799	3.43463192113203e-06\\
19508.3014263109	3.37854946016518e-06\\
19654.286507232	3.32311589542146e-06\\
19801.3640278989	3.26832826589871e-06\\
19949.5421632881	3.21418360136588e-06\\
20098.8291495512	3.16067891871529e-06\\
20249.233284473	3.10781121855071e-06\\
20400.7629279322	3.05557748200672e-06\\
20553.4265023667	3.00397466779243e-06\\
20707.2324932414	2.95299970945436e-06\\
20862.1894495195	2.90264951285169e-06\\
21018.3059841386	2.8529209538375e-06\\
21175.5907744885	2.80381087613953e-06\\
21334.0525628939	2.75531608943392e-06\\
21493.7001571006	2.70743336760488e-06\\
21654.5424307645	2.6601594471844e-06\\
21816.5883239452	2.61349102596467e-06\\
21979.8468436028	2.5674247617772e-06\\
22144.3270640985	2.52195727143212e-06\\
22310.0381276992	2.47708512981108e-06\\
22476.9892450851	2.432804869108e-06\\
22645.1896958625	2.38911297821105e-06\\
22814.6488290788	2.34600590222026e-06\\
22985.3760637425	2.3034800420948e-06\\
23157.3808893468	2.26153175442415e-06\\
23330.6728663967	2.2201573513177e-06\\
23505.261626941	2.17935310040727e-06\\
23681.1568751072	2.13911522495731e-06\\
23858.3683876408	2.09943990407764e-06\\
24036.9060144492	2.06032327303356e-06\\
24216.7796791487	2.02176142364872e-06\\
24397.9993796163	1.98375040479582e-06\\
24580.5751885457	1.94628622297041e-06\\
24764.5172540065	1.90936484294377e-06\\
24949.8358000088	1.87298218848991e-06\\
25136.5411270713	1.83713414318292e-06\\
25324.6436127937	1.80181655126041e-06\\
25514.153712434	1.76702521854883e-06\\
25705.0819594888	1.73275591344729e-06\\
25897.4389662797	1.69900436796545e-06\\
26091.2354245425	1.66576627881241e-06\\
26286.4821060218	1.63303730853266e-06\\
26483.1898630694	1.60081308668565e-06\\
26681.3696292481	1.56908921106579e-06\\
26881.0324199387	1.53786124895936e-06\\
27082.189332953	1.50712473843514e-06\\
27284.8515491498	1.47687518966588e-06\\
27489.0303330572	1.44710808627702e-06\\
27694.7370334982	1.41781888672022e-06\\
27901.9830842215	1.38900302566834e-06\\
28110.7800045375	1.36065591542911e-06\\
28321.1393999579	1.33277294737471e-06\\
28533.0729628413	1.30534949338428e-06\\
28746.5924730428	1.27838090729669e-06\\
28961.7097985688	1.25186252637084e-06\\
29178.4368962368	1.2257896727505e-06\\
29396.78581234	1.20015765493129e-06\\
29616.7686833165	1.17496176922676e-06\\
29838.3977364244	1.15019730123098e-06\\
30061.6852904209	1.12585952727474e-06\\
30286.6437562476	1.1019437158726e-06\\
30513.2856377198	1.07844512915797e-06\\
30741.6235322216	1.05535902430327e-06\\
30971.6701314066	1.03268065492216e-06\\
31203.4382219025	1.01040527245095e-06\\
31436.9406860226	9.88528127506063e-07\\
31672.1905024814	9.6704447121433e-07\\
31909.2007471158	9.45949556513147e-07\\
32147.9845936127	9.25238639416988e-07\\
32388.5553142404	9.04906980247321e-07\\
32630.9262805866	8.84949844822402e-07\\
32875.1109643017	8.65362505603918e-07\\
33121.1229378475	8.4614024279729e-07\\
33368.9758752518	8.27278345402549e-07\\
33618.6835528682	8.08772112213141e-07\\
33870.2598501416	7.9061685276008e-07\\
34123.7187503806	7.72807888199291e-07\\
34379.0743415334	7.55340552140772e-07\\
34636.340816972	7.38210191418396e-07\\
34895.5324762807	7.21412166800424e-07\\
35156.6637260502	7.04941853641526e-07\\
35419.7490806799	6.88794642478166e-07\\
35684.8031631832	6.72965939570846e-07\\
35951.840706001	6.57451167397647e-07\\
36220.8765518205	6.42245765105559e-07\\
36491.9256543999	6.2734518892747e-07\\
36765.0030794002	6.12744912574435e-07\\
37040.1240052216	5.98440427614764e-07\\
37317.3037238483	5.84427243852699e-07\\
37596.5576416978	5.70700889721223e-07\\
37877.9012804769	5.57256912704428e-07\\
38161.3502780454	5.44090879805592e-07\\
38446.9203892846	5.31198378077414e-07\\
38734.6274869731	5.18575015230322e-07\\
39024.4875626694	5.06216420333637e-07\\
39316.5167276	4.94118244622763e-07\\
39610.7312135559	4.82276162422673e-07\\
39907.1473737941	4.70685872195066e-07\\
40205.7816839466	4.59343097712538e-07\\
40506.6507429366	4.48243589358869e-07\\
40809.7712739007	4.3738312555015e-07\\
41115.1601251185	4.26757514266665e-07\\
41422.8342709494	4.16362594681331e-07\\
41732.8108127753	4.06194238866383e-07\\
42045.1069799522	3.96248353556719e-07\\
42359.7401307671	3.86520881945905e-07\\
42676.7277534027	3.77007805489198e-07\\
42996.0874669105	3.67705145687367e-07\\
43317.8370221886	3.58608965825691e-07\\
43641.9943029698	3.49715372643652e-07\\
43968.5773268145	3.41020517913249e-07\\
44297.6042461126	3.3252059990662e-07\\
44629.0933490932	3.24211864736965e-07\\
44963.0630608394	3.16090607560557e-07\\
45299.531944314	3.08153173631154e-07\\
45638.5187013907	3.00395959201977e-07\\
45980.0421738933	2.92815412273766e-07\\
46324.1213446437	2.85408033190558e-07\\
46670.775338516	2.78170375087619e-07\\
47020.0234235008	2.71099044198054e-07\\
47371.885011775	2.64190700026589e-07\\
47726.3796607813	2.57442055400148e-07\\
48083.5270743154	2.50849876405824e-07\\
48443.347103621	2.4441098222727e-07\\
48805.8597484928	2.38122244890678e-07\\
49171.0851583894	2.31980588931309e-07\\
49539.0436335514	2.25982990991257e-07\\
49909.7556261313	2.20126479358394e-07\\
50283.2417413299	2.14408133455943e-07\\
50659.5227385407	2.08825083291214e-07\\
51038.6195325055	2.03374508871273e-07\\
51420.5531944749	1.9805363959256e-07\\
51805.3449533811	1.92859753610508e-07\\
52193.0161970173	1.87790177194615e-07\\
52583.5884732259	1.82842284073551e-07\\
52977.0834910972	1.78013494774264e-07\\
53373.5231221757	1.7330127595844e-07\\
53772.929401675	1.68703139759076e-07\\
54175.3245297041	1.64216643119428e-07\\
54580.7308724997	1.59839387136228e-07\\
54989.1709636708	1.55569016408561e-07\\
55400.6675054502	1.5140321839359e-07\\
55815.2433699566	1.47339722769937e-07\\
56232.9216004666	1.4337630080934e-07\\
56653.725412694	1.39510764757022e-07\\
57077.6781960819	1.35740967220968e-07\\
57504.8035151016	1.32064800570269e-07\\
57935.1251105625	1.28480196342514e-07\\
58368.6669009326	1.24985124660141e-07\\
58805.4529836665	1.21577593655629e-07\\
59245.5076365459	1.18255648905293e-07\\
59688.8553190289	1.15017372871483e-07\\
60135.5206736089	1.11860884352912e-07\\
60585.528527185	1.08784337942835e-07\\
61038.9038924417	1.05785923494818e-07\\
61495.6719692387	1.02863865595791e-07\\
61955.8581460127	1.00016423046118e-07\\
62419.4880011873	9.72418883464241e-08\\
62886.5873045955	9.45385871908883e-08\\
63357.182018912	9.19048779667865e-08\\
63831.2983010957	8.93391512600313e-08\\
64308.9625038448	8.68398293664848e-08\\
64790.2011770598	8.4405365808866e-08\\
65275.0410693208	8.20342448590277e-08\\
65763.5091293736	7.97249810654624e-08\\
66255.6325076273	7.74761187858625e-08\\
66751.438557664	7.52862317245918e-08\\
67250.9548377589	7.31539224749505e-08\\
67754.209112412	7.10778220661077e-08\\
68261.2293538915	6.90565895146052e-08\\
68772.0437437882	6.70889113803433e-08\\
69286.6806745825	6.51735013269633e-08\\
69805.1687512222	6.33090996865675e-08\\
70327.5367927121	6.14944730287142e-08\\
70853.8138337169	5.97284137336306e-08\\
71384.0291261735	5.80097395696212e-08\\
71918.2121409184	5.63372932746132e-08\\
72456.392569325	5.47099421418306e-08\\
72998.6003249533	5.31265776095684e-08\\
73544.8655452147	5.1586114855044e-08\\
74095.2185930443	5.0087492392331e-08\\
74649.690058591	4.86296716743476e-08\\
75208.3107609164	4.7211636698912e-08\\
75771.111749708	4.58323936188573e-08\\
76338.1243070055	4.44909703562048e-08\\
76909.3799489393	4.31864162204075e-08\\
77484.9104274818	4.19178015306656e-08\\
78064.7477322134	4.06842172423156e-08\\
78648.9240920989	3.94847745773172e-08\\
79237.4719772807	3.83186046588282e-08\\
79830.4241008822	3.71848581498914e-08\\
80427.8134208264	3.60827048962387e-08\\
81029.6731416689	3.50113335732177e-08\\
81636.0367164413	3.39699513368633e-08\\
82246.937848513	3.29577834791075e-08\\
82862.4104934629	3.19740730871508e-08\\
83482.488860967	3.10180807069949e-08\\
84107.2074167008	3.00890840111467e-08\\
84736.6008842538	2.91863774705006e-08\\
85370.7042470603	2.83092720304046e-08\\
86009.5527503438	2.74570947909125e-08\\
86653.1819030753	2.6629188691229e-08\\
87301.627479948	2.58249121983444e-08\\
87954.9255233655	2.50436389998642e-08\\
88613.1123454441	2.42847577010312e-08\\
89276.2245300332	2.35476715259337e-08\\
89944.2989347464	2.28317980229049e-08\\
90617.3726930118	2.21365687740974e-08\\
91295.4832161355	2.14614291092347e-08\\
91978.6681953803	2.08058378235279e-08\\
92666.9656040625	2.01692668997453e-08\\
93360.4136996602	1.95512012344314e-08\\
94059.0510259418	1.89511383682522e-08\\
94762.9164151073	1.8368588220462e-08\\
95472.0489899465	1.78030728274706e-08\\
96186.4881660146	1.72541260854944e-08\\
96906.2736538223	1.6721293497276e-08\\
97631.4454610423	1.62041319228494e-08\\
98362.0438947356	1.57022093343289e-08\\
99098.1095635883	1.5215104574705e-08\\
99839.6833801717	1.47424071206156e-08\\
100586.806563215	1.42837168490736e-08\\
101339.520639896	1.38386438081227e-08\\
102097.867448152	1.34068079913924e-08\\
102861.889138999	1.29878391165282e-08\\
103631.628178883	1.25813764074622e-08\\
104407.127352034	1.21870683804973e-08\\
105188.429762846	1.18045726341732e-08\\
105975.578838274	1.14335556428788e-08\\
106768.618330246	1.10736925541821e-08\\
107567.592318097	1.07246669898395e-08\\
108372.545211017	1.03861708504518e-08\\
109183.521750518	1.00579041237309e-08\\
110000.567012926	9.7395746963388e-09\\
110823.726411883	9.43089816926454e-09\\
111653.04570087	9.13159767669973e-09\\
112488.570975754	8.84140370837319e-09\\
113330.348677345	8.56005393530909e-09\\
114178.425593984	8.28729303896485e-09\\
115032.848864136	8.02287254371233e-09\\
115893.665979016	7.76655065262044e-09\\
116760.924785228	7.5180920864976e-09\\
117634.673487419	7.27726792615551e-09\\
118514.960650966	7.04385545784937e-09\\
119401.835204671	6.81763802185579e-09\\
120295.34644348	6.59840486414491e-09\\
121195.544031226	6.38595099110439e-09\\
122102.478003387	6.18007702727386e-09\\
123016.198769868	5.98058907604695e-09\\
123936.757117803	5.78729858329718e-09\\
124864.204214378	5.60002220388752e-09\\
125798.591609674	5.41858167101784e-09\\
126739.971239535	5.24280366836996e-09\\
127688.395428448	5.07251970500647e-09\\
128643.916892463	4.90756599298e-09\\
129606.588742111	4.74778332761225e-09\\
130576.464485363	4.59301697039779e-09\\
131553.598030603	4.44311653449201e-09\\
132538.043689621	4.29793587274025e-09\\
133529.856180639	4.15733296820562e-09\\
134529.090631344	4.02116982715422e-09\\
135535.802581959	3.88931237445552e-09\\
136550.047988325	3.76163035135647e-09\\
137571.883225014	3.63799721558819e-09\\
138601.365088463	3.51829004376378e-09\\
139638.550800127	3.40238943602702e-09\\
140683.498009666	3.29017942291122e-09\\
141736.264798142	3.18154737436763e-09\\
142796.909681254	3.07638391092467e-09\\
143865.491612584	2.97458281693716e-09\\
144942.069986881	2.87604095588731e-09\\
146026.704643354	2.78065818769845e-09\\
147119.455869007	2.68833728802264e-09\\
148220.384401982	2.59898386946505e-09\\
149329.55143494	2.51250630470639e-09\\
150447.018618462	2.42881565148693e-09\\
151572.84806447	2.34782557941514e-09\\
152707.102349689	2.2694522985641e-09\\
153849.844519117	2.19361448982041e-09\\
155001.138089531	2.12023323694958e-09\\
156161.047053023	2.04923196034273e-09\\
157329.635880547	1.98053635241064e-09\\
158506.969525512	1.91407431458995e-09\\
159693.113427386	1.84977589592861e-09\\
160888.133515336	1.78757323321706e-09\\
162092.096211894	1.72740049263185e-09\\
163305.068436644	1.66919381286036e-09\\
164527.117609946	1.61289124967366e-09\\
165758.311656683	1.55843272191684e-09\\
166998.719010032	1.50575995888565e-09\\
168248.408615274	1.45481644905886e-09\\
169507.449933624	1.40554739015666e-09\\
170775.912946089	1.35789964049527e-09\\
172053.868157361	1.31182167160897e-09\\
173341.386599734	1.26726352211096e-09\\
174638.539837053	1.22417675276474e-09\\
175945.399968693	1.18251440273871e-09\\
177262.039633563	1.1422309470168e-09\\
178588.532014148	1.10328225493809e-09\\
179924.950840571	1.06562554983983e-09\\
181271.370394698	1.02921936977761e-09\\
182627.865514261	9.94023529297573e-10\\
183994.51159702	9.59999082236063e-10\\
185371.384604954	9.27108285521876e-10\\
186758.561068483	8.9531456395779e-10\\
188156.118090722	8.6458247595737e-10\\
189564.133351766	8.34877680214419e-10\\
190982.685113006	8.06166903282353e-10\\
192411.852221483	7.7841790804132e-10\\
193851.71411427	7.51599463031485e-10\\
195302.350822881	7.25681312631171e-10\\
196763.842977729	7.00634148058881e-10\\
198236.2718126	6.76429579179067e-10\\
199719.719169172	6.53040107091285e-10\\
201214.267501562	6.30439097483415e-10\\
202719.999880911	6.08600754729645e-10\\
204237	5.87500096714342e-10\\
204237	4.03013921735864e-10\\
202719.999880911	4.18520916770645e-10\\
201214.267501562	4.34607347765619e-10\\
199719.719169172	4.51294147978003e-10\\
198236.2718126	4.68602977033267e-10\\
196763.842977729	4.86556244885161e-10\\
195302.350822881	5.05177136514117e-10\\
193851.71411427	5.24489637384668e-10\\
192411.852221483	5.44518559683033e-10\\
190982.685113006	5.65289569356217e-10\\
189564.133351766	5.86829213974948e-10\\
188156.118090722	6.09164951442784e-10\\
186758.561068483	6.32325179574548e-10\\
185371.384604954	6.56339266567733e-10\\
183994.51159702	6.81237582390862e-10\\
182627.865514261	7.07051531113513e-10\\
181271.370394698	7.33813584203271e-10\\
179924.950840571	7.61557314815084e-10\\
178588.532014148	7.90317433099625e-10\\
177262.039633563	8.20129822557206e-10\\
175945.399968693	8.51031577464758e-10\\
174638.539837053	8.8306104140394e-10\\
173341.386599734	9.16257846918588e-10\\
172053.868157361	9.50662956331024e-10\\
170775.912946089	9.86318703746593e-10\\
169507.449933624	1.02326883827682e-09\\
168248.408615274	1.06155856851213e-09\\
166998.719010032	1.10123460827546e-09\\
165758.311656683	1.14234522368887e-09\\
164527.117609946	1.18494028158585e-09\\
163305.068436644	1.22907129930252e-09\\
162092.096211894	1.2747914958817e-09\\
160888.133515336	1.32215584472416e-09\\
159693.113427386	1.37122112772212e-09\\
158506.969525512	1.4220459909111e-09\\
157329.635880547	1.47469100167563e-09\\
156161.047053023	1.52921870754655e-09\\
155001.138089531	1.58569369662679e-09\\
153849.844519117	1.64418265968398e-09\\
152707.102349689	1.70475445394876e-09\\
151572.84806447	1.76748016865755e-09\\
150447.018618462	1.83243319238035e-09\\
149329.55143494	1.89968928217352e-09\\
148220.384401982	1.96932663459871e-09\\
147119.455869007	2.04142595864992e-09\\
146026.704643354	2.11607055063027e-09\\
144942.069986881	2.19334637102168e-09\\
143865.491612584	2.27334212339083e-09\\
142796.909681254	2.35614933537458e-09\\
141736.264798142	2.44186244179007e-09\\
140683.498009666	2.5305788699138e-09\\
139638.550800127	2.62239912697502e-09\\
138601.365088463	2.71742688990992e-09\\
137571.883225014	2.81576909742182e-09\\
136550.047988325	2.91753604439535e-09\\
135535.802581959	3.02284147871089e-09\\
134529.090631344	3.13180270050715e-09\\
133529.856180639	3.24454066394016e-09\\
132538.043689621	3.36118008148647e-09\\
131553.598030603	3.48184953083961e-09\\
130576.464485363	3.60668156444871e-09\\
129606.588742111	3.73581282174835e-09\\
128643.916892463	3.86938414412958e-09\\
127688.395428448	4.00754069270141e-09\\
126739.971239535	4.15043206889259e-09\\
125798.591609674	4.2982124379448e-09\\
124864.204214378	4.45104065534546e-09\\
123936.757117803	4.60908039625268e-09\\
123016.198769868	4.77250028796093e-09\\
122102.478003387	4.94147404545817e-09\\
121195.544031226	5.11618061012553e-09\\
120295.34644348	5.29680429162781e-09\\
119401.835204671	5.48353491304684e-09\\
118514.960650966	5.67656795930579e-09\\
117634.673487419	5.87610472893463e-09\\
116760.924785228	6.08235248922616e-09\\
115893.665979016	6.29552463483068e-09\\
115032.848864136	6.51584084983781e-09\\
114178.425593984	6.74352727339439e-09\\
113330.348677345	6.97881666890332e-09\\
112488.570975754	7.2219485968528e-09\\
111653.04570087	7.47316959131972e-09\\
110823.726411883	7.73273334019275e-09\\
110000.567012926	8.00090086916096e-09\\
109183.521750518	8.27794072950783e-09\\
108372.545211017	8.56412918975697e-09\\
107567.592318097	8.85975043120726e-09\\
106768.618330246	9.16509674739894e-09\\
105975.578838274	9.48046874754932e-09\\
105188.429762846	9.8061755639942e-09\\
104407.127352034	1.01425350636714e-08\\
103631.628178883	1.04898740636805e-08\\
102861.889138999	1.084852855095e-08\\
102097.867448152	1.12188439060441e-08\\
101339.520639896	1.1601175131136e-08\\
100586.806563215	1.19958870821741e-08\\
99839.6833801717	1.24033547052674e-08\\
99098.1095635883	1.28239632773081e-08\\
98362.0438947356	1.32581086508556e-08\\
97631.4454610423	1.37061975032941e-08\\
96906.2736538223	1.41686475902804e-08\\
96186.4881660146	1.46458880034939e-08\\
95472.0489899465	1.51383594326927e-08\\
94762.9164151073	1.56465144320865e-08\\
94059.0510259418	1.61708176910231e-08\\
93360.4136996602	1.67117463089885e-08\\
92666.9656040625	1.72697900749154e-08\\
91978.6681953803	1.78454517507901e-08\\
91295.4832161355	1.8439247359542e-08\\
90617.3726930118	1.90517064772042e-08\\
89944.2989347464	1.96833725293136e-08\\
89276.2245300332	2.03348030915332e-08\\
88613.1123454441	2.10065701944585e-08\\
87954.9255233655	2.16992606325726e-08\\
87301.627479948	2.2413476277312e-08\\
86653.1819030753	2.31498343941875e-08\\
86009.5527503438	2.39089679639166e-08\\
85370.7042470603	2.46915260075008e-08\\
84736.6008842538	2.54981739151848e-08\\
84107.2074167008	2.63295937792284e-08\\
83482.488860967	2.71864847304075e-08\\
82862.4104934629	2.8069563278163e-08\\
82246.937848513	2.89795636543016e-08\\
81636.0367164413	2.99172381601491e-08\\
81029.6731416689	3.08833575170491e-08\\
80427.8134208264	3.18787112200876e-08\\
79830.4241008822	3.29041078949181e-08\\
79237.4719772807	3.39603756575594e-08\\
78648.9240920989	3.50483624770092e-08\\
78064.7477322134	3.61689365405364e-08\\
77484.9104274818	3.73229866214755e-08\\
76909.3799489393	3.85114224493547e-08\\
76338.1243070055	3.97351750821767e-08\\
75771.111749708	4.09951972806444e-08\\
75208.3107609164	4.22924638841379e-08\\
74649.690058591	4.36279721882114e-08\\
74095.2185930443	4.50027423233798e-08\\
73544.8655452147	4.64178176349527e-08\\
72998.6003249533	4.78742650636479e-08\\
72456.392569325	4.93731755267127e-08\\
71918.2121409184	5.0915664299268e-08\\
71384.0291261735	5.25028713955533e-08\\
70853.8138337169	5.41359619497711e-08\\
70327.5367927121	5.58161265961724e-08\\
69805.1687512222	5.75445818480341e-08\\
69286.6806745825	5.93225704751561e-08\\
68772.0437437882	6.11513618794653e-08\\
68261.2293538915	6.30322524683335e-08\\
67754.209112412	6.49665660251531e-08\\
67250.9548377589	6.69556540767169e-08\\
66751.438557664	6.90008962569274e-08\\
66255.6325076273	7.11037006663177e-08\\
65763.5091293736	7.3265504226861e-08\\
65275.0410693208	7.54877730315131e-08\\
64790.2011770598	7.77720026878997e-08\\
64308.9625038448	8.01197186555495e-08\\
63831.2983010957	8.25324765760257e-08\\
63357.182018912	8.50118625952921e-08\\
62886.5873045955	8.75594936776326e-08\\
62419.4880011873	9.01770179103719e-08\\
61955.8581460127	9.28661147986762e-08\\
61495.6719692387	9.56284955496243e-08\\
61038.9038924417	9.84659033447441e-08\\
60585.528527185	1.01380113600185e-07\\
60135.5206736089	1.04372934213627e-07\\
59688.8553190289	1.07446205797059e-07\\
59245.5076365459	1.10601801894495e-07\\
58805.4529836665	1.13841629183675e-07\\
58368.6669009326	1.17167627660822e-07\\
57935.1251105625	1.20581770807451e-07\\
57504.8035151016	1.24086065738277e-07\\
57077.6781960819	1.27682553329259e-07\\
56653.725412694	1.31373308324777e-07\\
56232.9216004666	1.35160439423065e-07\\
55815.2433699566	1.39046089338955e-07\\
55400.6675054502	1.43032434843109e-07\\
54989.1709636708	1.47121686776981e-07\\
54580.7308724997	1.51316090042805e-07\\
54175.3245297041	1.55617923568107e-07\\
53772.929401675	1.60029500244314e-07\\
53373.5231221757	1.64553166839269e-07\\
52977.0834910972	1.69191303883701e-07\\
52583.5884732259	1.73946325531909e-07\\
52193.0161970173	1.78820679397326e-07\\
51805.3449533811	1.83816846363936e-07\\
51420.5531944749	1.88937340374981e-07\\
51038.6195325055	1.94184708200881e-07\\
50659.5227385407	1.99561529188847e-07\\
50283.2417413299	2.05070414997259e-07\\
49909.7556261313	2.10714009318619e-07\\
49539.0436335514	2.16494987595502e-07\\
49171.0851583894	2.22416056734866e-07\\
48805.8597484928	2.28479954826734e-07\\
48443.347103621	2.34689450874164e-07\\
48083.5270743154	2.41047344542263e-07\\
47726.3796607813	2.47556465934651e-07\\
47371.885011775	2.54219675406636e-07\\
47020.0234235008	2.61039863424771e-07\\
46670.775338516	2.68019950482968e-07\\
46324.1213446437	2.75162887085467e-07\\
45980.0421738933	2.8247165380683e-07\\
45638.5187013907	2.89949261438677e-07\\
45299.531944314	2.97598751232061e-07\\
44963.0630608394	3.0542319524298e-07\\
44629.0933490932	3.13425696786908e-07\\
44297.6042461126	3.21609391005854e-07\\
43968.5773268145	3.29977445548827e-07\\
43641.9943029698	3.38533061363516e-07\\
43317.8370221886	3.47279473593413e-07\\
42996.0874669105	3.56219952571183e-07\\
42676.7277534027	3.65357804895119e-07\\
42359.7401307671	3.74696374572085e-07\\
42045.1069799522	3.84239044206971e-07\\
41732.8108127753	3.93989236215749e-07\\
41422.8342709494	4.03950414037256e-07\\
41115.1601251185	4.14126083317234e-07\\
40809.7712739007	4.24519793037921e-07\\
40506.6507429366	4.35135136567017e-07\\
40205.7816839466	4.45975752601388e-07\\
39907.1473737941	4.57045325983457e-07\\
39610.7312135559	4.68347588371593e-07\\
39316.5167276	4.79886318749614e-07\\
39024.4875626694	4.91665343765215e-07\\
38734.6274869731	5.03688537891426e-07\\
38446.9203892846	5.15959823409955e-07\\
38161.3502780454	5.28483170219576e-07\\
37877.9012804769	5.41262595476397e-07\\
37596.5576416978	5.54302163076416e-07\\
37317.3037238483	5.67605982993089e-07\\
37040.1240052216	5.81178210484678e-07\\
36765.0030794002	5.95023045187278e-07\\
36491.9256543999	6.09144730109807e-07\\
36220.8765518205	6.23547550547196e-07\\
35951.840706001	6.38235832927407e-07\\
35684.8031631832	6.53213943606626e-07\\
35419.7490806799	6.68486287625983e-07\\
35156.6637260502	6.84057307441268e-07\\
34895.5324762807	6.99931481635655e-07\\
34636.340816972	7.16113323623834e-07\\
34379.0743415334	7.32607380354056e-07\\
34123.7187503806	7.49418231013519e-07\\
33870.2598501416	7.66550485740708e-07\\
33618.6835528682	7.84008784347353e-07\\
33368.9758752518	8.01797795051598e-07\\
33121.1229378475	8.19922213222802e-07\\
32875.1109643017	8.38386760138143e-07\\
32630.9262805866	8.57196181750125e-07\\
32388.5553142404	8.76355247463936e-07\\
32147.9845936127	8.95868748923275e-07\\
31909.2007471158	9.15741498802847e-07\\
31672.1905024814	9.35978329605808e-07\\
31436.9406860226	9.56584092464345e-07\\
31203.4382219025	9.77563655941358e-07\\
30971.6701314066	9.989219048317e-07\\
30741.6235322216	1.02066373896101e-06\\
30513.2856377198	1.04279407198066e-06\\
30286.6437562476	1.06531783015738e-06\\
30061.6852904209	1.08823995115602e-06\\
29838.3977364244	1.11156538281458e-06\\
29616.7686833165	1.13529908191014e-06\\
29396.78581234	1.15944601291503e-06\\
29178.4368962368	1.18401114674251e-06\\
28961.7097985688	1.20899945948118e-06\\
28746.5924730428	1.23441593111786e-06\\
28533.0729628413	1.26026554424857e-06\\
28321.1393999579	1.28655328277734e-06\\
28110.7800045375	1.31328413060283e-06\\
27901.9830842215	1.34046307029286e-06\\
27694.7370334982	1.36809508174669e-06\\
27489.0303330572	1.39618514084576e-06\\
27284.8515491498	1.42473821809259e-06\\
27082.189332953	1.45375927723878e-06\\
26881.0324199387	1.4832532739022e-06\\
26681.3696292481	1.51322515417401e-06\\
26483.1898630694	1.54367985321627e-06\\
26286.4821060218	1.57462229385032e-06\\
26091.2354245425	1.60605738513721e-06\\
25897.4389662797	1.63799002095052e-06\\
25705.0819594888	1.67042507854239e-06\\
25514.153712434	1.70336741710382e-06\\
25324.6436127937	1.73682187631986e-06\\
25136.5411270713	1.77079327492063e-06\\
24949.8358000088	1.80528640922931e-06\\
24764.5172540065	1.84030605170755e-06\\
24580.5751885457	1.87585694949995e-06\\
24397.9993796163	1.91194382297803e-06\\
24216.7796791487	1.94857136428485e-06\\
24036.9060144492	1.98574423588165e-06\\
23858.3683876408	2.02346706909685e-06\\
23681.1568751072	2.06174446267928e-06\\
23505.261626941	2.10058098135597e-06\\
23330.6728663967	2.139981154396e-06\\
23157.3808893468	2.17994947418139e-06\\
22985.3760637425	2.22049039478596e-06\\
22814.6488290788	2.26160833056344e-06\\
22645.1896958625	2.30330765474577e-06\\
22476.9892450851	2.34559269805264e-06\\
22310.0381276992	2.38846774731366e-06\\
22144.3270640985	2.43193704410382e-06\\
21979.8468436028	2.4760047833936e-06\\
21816.5883239452	2.52067511221502e-06\\
21654.5424307645	2.56595212834422e-06\\
21493.7001571006	2.61183987900241e-06\\
21334.0525628939	2.6583423595759e-06\\
21175.5907744885	2.70546351235662e-06\\
21018.3059841386	2.75320722530457e-06\\
20862.1894495195	2.80157733083321e-06\\
20707.2324932414	2.85057760461979e-06\\
20553.4265023667	2.90021176444173e-06\\
20400.7629279322	2.95048346904103e-06\\
20249.233284473	3.00139631701868e-06\\
20098.8291495512	3.05295384576107e-06\\
19949.5421632881	3.1051595304007e-06\\
19801.3640278989	3.15801678281444e-06\\
19654.286507232	3.21152895066159e-06\\
19508.3014263109	3.26569931646611e-06\\
19363.4006708799	3.32053109674659e-06\\
19219.5761869535	3.37602744119869e-06\\
19076.8199803677	3.43219143193576e-06\\
18935.1241163369	3.48902608279339e-06\\
18794.480719012	3.54653433870557e-06\\
18654.8819710428	3.60471907516049e-06\\
18516.3201131441	3.66358309774552e-06\\
18378.7874436634	3.72312914179268e-06\\
18242.2763181536	3.78335987213685e-06\\
18106.7791489477	3.84427788300152e-06\\
17972.2884047374	3.90588569802843e-06\\
17838.7966101542	3.96818577046989e-06\\
17706.2963453539	4.03118048356531e-06\\
17574.7802456044	4.09487215112568e-06\\
17444.2410008763	4.15926301835328e-06\\
17314.6713554361	4.22435526292708e-06\\
17186.0641074439	4.29015099638692e-06\\
17058.4121085521	4.35665226585473e-06\\
16931.7082635086	4.42386105613322e-06\\
16805.9455297624	4.49177929222726e-06\\
16681.116917072	4.56040884233721e-06\\
16557.2154871169	4.62975152137619e-06\\
16434.234353112	4.69980909506857e-06\\
16312.166679425	4.77058328468869e-06\\
16191.0056811961	4.84207577250269e-06\\
16070.7446239608	4.91428820797861e-06\\
15951.3768232763	4.98722221483055e-06\\
15832.8956443492	5.06087939896388e-06\\
15715.2945016669	5.13526135738775e-06\\
15598.5668586318	5.21036968815774e-06\\
15482.7062271979	5.28620600140937e-06\\
15367.70616751	5.3627719315359e-06\\
15253.5602875458	5.44006915055695e-06\\
15140.2622427609	5.51809938271526e-06\\
15027.8057357356	5.59686442032554e-06\\
14916.1845158256	5.67636614088803e-06\\
14805.3923788139	5.75660652546177e-06\\
14695.4231665662	5.83758767827675e-06\\
14586.2707666889	5.919311847546e-06\\
14477.9291121888	6.00178144741928e-06\\
14370.3921811364	6.08499908100178e-06\\
14263.6539963308	6.168967564343e-06\\
14157.7086249677	6.25368995128406e-06\\
14052.5501783096	6.33916955903722e-06\\
13948.1728113584	6.4254099943592e-06\\
13844.5707225307	6.51241518017154e-06\\
13741.738153335	6.60018938247739e-06\\
13639.669388052	6.68873723742218e-06\\
13538.3587534167	6.77806377835239e-06\\
13437.8006183031	6.86817446273287e-06\\
13337.9893934111	6.95907519879766e-06\\
13238.9195309561	7.05077237182542e-06\\
13140.5855243605	7.14327286994942e-06\\
13042.9819079474	7.23658410943552e-06\\
12946.1032566372	7.33071405938337e-06\\
12849.9441856459	7.42567126583044e-06\\
12754.4993501855	7.5214648752624e-06\\
12659.7634451676	7.61810465755429e-06\\
12565.7312049077	7.71560102838785e-06\\
12472.3974028332	7.81396507120794e-06\\
12379.7568511926	7.91320855879432e-06\\
12287.8044007671	8.01334397453811e-06\\
12196.5349405845	8.11438453351812e-06\\
12105.9433976352	8.21634420347821e-06\\
12016.0247365899	8.31923772580803e-06\\
11926.7739595202	8.4230806366271e-06\\
11838.1861056203	8.52788928807061e-06\\
11750.2562509318	8.63368086986723e-06\\
11662.9795080693	8.74047343129369e-06\\
11576.3510259497	8.84828590358186e-06\\
11490.3659895215	8.95713812284471e-06\\
11405.019619498	9.06705085357848e-06\\
11320.3071720913	9.17804581278862e-06\\
11236.2239387488	9.29014569477768e-06\\
11152.765245891	9.4033741966247e-06\\
11069.9264546524	9.51775604437655e-06\\
10987.7029606233	9.63331701996473e-06\\
10906.0901935939	9.75008398885367e-06\\
10825.0836173003	9.86808492842002e-06\\
10744.6787291723	9.98734895705776e-06\\
10664.8710600833	1.01079063639977e-05\\
10585.6561741017	1.02297886398271e-05\\
10507.0296682446	1.03530285076902e-05\\
10428.9871722326	1.04776599551491e-05\\
10351.5243482474	1.06037182666791e-05\\
10274.6368906905	1.07312400567735e-05\\
10198.3205259438	1.08602633036286e-05\\
10122.5710121321	1.09908273833785e-05\\
10047.3841388873	1.11229731048495e-05\\
9972.75572711457	1.12567427447994e-05\\
9898.68162875981	1.13921800836101e-05\\
9825.15772657923	1.15293304413956e-05\\
9752.17993391047	1.16682407144931e-05\\
9679.74419444542	1.18089594122961e-05\\
9607.84648200483	1.19515366943951e-05\\
9536.48280031448	1.20960244079841e-05\\
9465.64918278304	1.22424761254949e-05\\
9395.34169228163	1.23909471824157e-05\\
9325.55642092493	1.25414947152538e-05\\
9256.28948985408	1.2694177699596e-05\\
9187.53704902097	1.28490569882238e-05\\
9119.29527697427	1.30061953492352e-05\\
9051.56038064706	1.31656575041249e-05\\
8984.32859514597	1.33275101657738e-05\\
8917.59618354194	1.34918220762952e-05\\
8851.35943666247	1.36586640446848e-05\\
8785.61467288549	1.38281089842186e-05\\
8720.35823793472	1.40002319495421e-05\\
8655.58650467655	1.41751101733904e-05\\
8591.29587291844	1.43528231028794e-05\\
8527.48276920879	1.45334524353026e-05\\
8464.14364663834	1.47170821533696e-05\\
8401.27498464301	1.49037985598173e-05\\
8338.87328880825	1.50936903113258e-05\\
8276.93509067475	1.5286848451664e-05\\
8215.4569475457	1.54833664439952e-05\\
8154.43544229543	1.56833402022623e-05\\
8093.86718317946	1.58868681215794e-05\\
8033.748803646	1.60940511075476e-05\\
7974.0769621488	1.63049926044164e-05\\
7914.84834196143	1.65197986220076e-05\\
7856.05965099295	1.67385777613217e-05\\
7797.70762160489	1.69614412387414e-05\\
7739.78901042966	1.7188502908753e-05\\
7682.30059819021	1.74198792851014e-05\\
7625.23918952119	1.76556895603009e-05\\
7568.60161279127	1.78960556234227e-05\\
7512.38471992689	1.81411020760855e-05\\
7456.58538623724	1.83909562465797e-05\\
7401.20051024062	1.8645748202059e-05\\
7346.22701349204	1.89056107587453e-05\\
7291.66184041212	1.91706794900956e-05\\
7237.50195811723	1.94410927328947e-05\\
7183.7443562509	1.97169915912482e-05\\
7130.38604681656	1.9998519938464e-05\\
7077.42406401143	2.02858244168306e-05\\
7024.85546406162	2.05790544353175e-05\\
6972.67732505857	2.08783621652457e-05\\
6920.88674679662	2.11839025340043e-05\\
6869.48085061182	2.14958332169129e-05\\
6818.45677922192	2.18143146273644e-05\\
6767.81169656754	2.21395099054145e-05\\
6717.54278765452	2.24715849050197e-05\\
6667.64725839753	2.28107081801644e-05\\
6618.12233546469	2.31570509701578e-05\\
6568.96526612346	2.35107871844178e-05\\
6520.17331808759	2.38720933870993e-05\\
6471.74377936531	2.42411487819623e-05\\
6423.67395810857	2.46181351979028e-05\\
6375.9611824634	2.50032370755994e-05\\
6328.60280042144	2.53966414557434e-05\\
6281.59617967246	2.5798537969325e-05\\
6234.93870745815	2.62091188304417e-05\\
6188.62779042682	2.66285788320667e-05\\
6142.66085448928	2.70571153451705e-05\\
6097.03534467575	2.74949283215184e-05\\
6051.74872499389	2.79422203003735e-05\\
6006.79847828778	2.83991964192177e-05\\
5962.18210609809	2.88660644284514e-05\\
5917.8971285231	2.93430347098696e-05\\
5873.94108408097	2.98303202985129e-05\\
5830.31152957285	3.03281369072869e-05\\
5787.00603994713	3.08367029535188e-05\\
5744.02220816459	3.13562395864002e-05\\
5701.35764506468	3.18869707140449e-05\\
5659.00997923266	3.24291230286939e-05\\
5616.97685686784	3.29829260284286e-05\\
5575.25594165272	3.35486120336233e-05\\
5533.84491462315	3.41264161962824e-05\\
5492.74147403939	3.4716576500389e-05\\
5451.94333525824	3.53193337514194e-05\\
5411.44823060603	3.59349315532833e-05\\
5371.25390925253	3.65636162711016e-05\\
5331.3581370859	3.72056369784519e-05\\
5291.7586965885	3.78612453879639e-05\\
5252.45338671363	3.85306957644423e-05\\
5213.44002276314	3.92142448200007e-05\\
5174.7164362661	3.99121515910044e-05\\
5136.28047485818	4.06246772969242e-05\\
5098.13000216207	4.13520851814845e-05\\
5060.2628976687	4.2094640336735e-05\\
5022.6770566194	4.28526095108825e-05\\
4985.37038988889	4.36262609008738e-05\\
4948.34082386919	4.44158639308358e-05\\
4911.58630035433	4.52216890175369e-05\\
4875.10477642598	4.60440073240531e-05\\
4838.89422433987	4.68830905027996e-05\\
4802.95263141311	4.77392104290358e-05\\
4767.27799991228	4.86126389258722e-05\\
4731.86834694246	4.9503647481706e-05\\
4696.72170433693	5.04125069609127e-05\\
4661.83611854782	5.13394873084949e-05\\
4627.20965053756	5.22848572492867e-05\\
4592.84037567103	5.32488839821991e-05\\
4558.72638360862	5.4231832869892e-05\\
4524.86577820005	5.52339671241678e-05\\
4491.25667737898	5.62555474873058e-05\\
4457.89721305839	5.72968319094887e-05\\
4424.78553102675	5.83580752224219e-05\\
4391.91979084494	5.94395288091997e-05\\
4359.29816574399	6.05414402704513e-05\\
4326.91884252352	6.16640530867696e-05\\
4294.78002145095	6.28076062774234e-05\\
4262.8799161615	6.3972334055347e-05\\
4231.21675355883	6.51584654784018e-05\\
4199.78877371658	6.63662240969203e-05\\
4168.59422978048	6.75958275975515e-05\\
4137.63138787127	6.88474874434466e-05\\
4106.89852698833	7.012140851084e-05\\
4076.39393891404	7.14177887221098e-05\\
4046.1159281188	7.27368186754164e-05\\
4016.06281166681	7.40786812710523e-05\\
3986.23291912252	7.54435513346525e-05\\
3956.62459245778	7.68315952374469e-05\\
3927.23618595969	7.82429705137581e-05\\
3898.06606613913	7.96778254759768e-05\\
3869.11261163994	8.11362988272714e-05\\
3840.37421314884	8.26185192723114e-05\\
3811.84927330594	8.41246051263153e-05\\
3783.53620661599	8.56546639227549e-05\\
3755.43343936023	8.72087920200724e-05\\
3727.53940950892	8.87870742077977e-05\\
3699.85256663454	9.03895833124713e-05\\
3672.37137182558	9.20163798038077e-05\\
3645.09429760103	9.36675114015636e-05\\
3618.01982782546	9.53430126835956e-05\\
3591.14645762477	9.70429046956204e-05\\
3564.47269330252	9.87671945632247e-05\\
3537.99705225692	0.000100515875106695\\
3511.71806289842	0.000102288924459271\\
3485.63426456793	0.000104086305689456\\
3459.7442074556	0.000105907966428057\\
3434.04645252026	0.000107753838500656\\
3408.53957140944	0.000109623837566257\\
3383.22214637995	0.00011151786276289\\
3358.09277021909	0.000113435796360995\\
3333.15004616647	0.000115377503425467\\
3308.39258783631	0.000117342831487252\\
3283.81901914043	0.000119331610225496\\
3259.42797421173	0.000121343651161232\\
3235.21809732829	0.00012337874736369\\
3211.18804283803	0.000125436673170362\\
3187.33647508389	0.000127517183921978\\
3163.66206832958	0.00012962001571366\\
3140.16350668593	0.000131744885163515\\
3116.83948403772	0.000133891489200021\\
3093.68870397109	0.000136059504869585\\
3070.7098797015	0.000138248589165708\\
3047.90173400216	0.000140458378881199\\
3025.2629991331	0.00014268849048493\\
3002.79241677064	0.000144938520024589\\
2980.48873793751	0.000147208043056927\\
2958.3507229334	0.000149496614606907\\
2936.37714126603	0.000151803769157137\\
2914.56677158281	0.000154129020668864\\
2892.91840160291	0.000156471862635714\\
2871.4308280499	0.000158831768171157\\
2850.10285658484	0.000161208190130544\\
2828.93330173995	0.000163600561268298\\
2807.92098685266	0.00016600829443058\\
2787.06474400025	0.000168430782783482\\
2766.3634139349	0.000170867400076406\\
2745.81584601927	0.000173317500939998\\
2725.42089816257	0.000175780421217583\\
2705.17743675704	0.000178255478328661\\
2685.08433661496	0.000180741971662679\\
2665.14048090611	0.000183239183000862\\
2645.34476109567	0.000185746376963592\\
2625.69607688265	0.000188262801480509\\
2606.1933361387	0.00019078768828025\\
2586.83545484739	0.000193320253396594\\
2567.62135704402	0.000195859697687648\\
2548.54997475574	0.000198405207364751\\
2529.62024794223	0.000200955954527802\\
2510.8311244368	0.000203511097703927\\
2492.18155988785	0.000206069782386653\\
2473.67051770087	0.000208631141573079\\
2455.29696898081	0.00021119429629696\\
2437.05989247487	0.00021375835615606\\
2418.95827451578	0.000216322419832624\\
2400.9911089654	0.000218885575606303\\
2383.15739715885	0.000221446901859393\\
2365.45614784899	0.000224005467574679\\
2347.88637715129	0.000226560332826623\\
2330.44710848916	0.000229110549267029\\
2313.1373725397	0.000231655160606572\\
2295.95620717979	0.000234193203093886\\
2278.90265743261	0.000236723705994021\\
2261.97577541457	0.000239245692068182\\
2245.17462028264	0.000241758178056706\\
2228.49825818201	0.000244260175167151\\
2211.94576219425	0.000246750689569323\\
2195.51621228573	0.000249228722898866\\
2179.20869525649	0.000251693272770918\\
2163.02230468953	0.000254143333305065\\
2146.95614090037	0.000256577895662665\\
2131.00931088707	0.000258995948597324\\
2115.18092828061	0.000261396479019101\\
2099.47011329558	0.000263778472572774\\
2083.87599268133	0.000266140914230305\\
2068.39769967338	0.000268482788897388\\
2053.03437394529	0.000270803082033821\\
2037.7851615608	0.00027310078028725\\
2022.64921492642	0.000275374872139686\\
2007.62569274426	0.00027762434856606\\
1992.71375996528	0.000279848203703971\\
1977.91258774292	0.000282045435533695\\
1963.22135338698	0.000284215046567426\\
1948.63924031791	0.000286356044546672\\
1934.16543802145	0.000288467443146673\\
1919.7991420035	0.000290548262686672\\
1905.53955374551	0.00029259753084484\\
1891.38588066002	0.000294614283376637\\
1877.33733604663	0.000296597564835392\\
1863.39313904828	0.000298546429293857\\
1849.55251460781	0.00030045994106552\\
1835.81469342496	0.000302337175424441\\
1822.17891191353	0.000304177219322409\\
1808.64441215895	0.000305979172102205\\
1795.21044187622	0.00030774214620579\\
1781.876254368	0.000309465267876246\\
1768.64110848317	0.000311147677852316\\
1755.50426857564	0.000312788532054406\\
1742.4650044634	0.000314387002260941\\
1729.522591388	0.000315942276773977\\
1716.67630997424	0.000317453561073004\\
1703.92544619016	0.000318920078455889\\
1691.26929130741	0.000320341070665948\\
1678.70714186178	0.000321715798504132\\
1666.2382996142	0.000323043542425389\\
1653.86207151182	0.000324323603118234\\
1641.57776964957	0.000325555302066641\\
1629.38471123188	0.000326737982093372\\
1617.28221853476	0.000327871007883914\\
1605.26961886811	0.000328953766490219\\
1593.34624453832	0.000329985667813509\\
1581.51143281119	0.000330966145065418\\
1569.76452587505	0.000331894655206848\\
1558.10487080425	0.000332770679363914\\
1546.53181952283	0.000333593723220492\\
1535.0447287685	0.000334363317386882\\
1523.64296005691	0.000335079017744243\\
1512.32587964614	0.000335740405764522\\
1501.09285850146	0.000336347088805703\\
1489.94327226042	0.000336898700382338\\
1478.8765011981	0.000337394900411442\\
1467.89193019267	0.000337835375433985\\
1456.98894869122	0.00033821983881239\\
1446.16695067579	0.000338548030904627\\
1435.42533462974	0.000338819719215692\\
1424.76350350425	0.000339034698527515\\
1414.18086468518	0.000339192791008549\\
1403.67682996012	0.000339293846304606\\
1393.2508154857	0.000339337741612757\\
1382.90224175512	0.000339324381740462\\
1372.63053356595	0.000339253699152375\\
1362.43511998817	0.000339125654007617\\
1352.31543433242	0.000338940234190644\\
1342.27091411851	0.000338697455339126\\
1332.30100104415	0.00033839736087255\\
1322.40514095393	0.000338040022025496\\
1312.5827838085	0.000337625537889701\\
1302.83338365401	0.000337154035469099\\
1293.15639859177	0.000336625669751987\\
1283.55129074812	0.000336040623804255\\
1274.01752624451	0.000335399108887239\\
1264.5545751679	0.000334701364603165\\
1255.16191154121	0.000333947659070283\\
1245.83901329415	0.000333138289128714\\
1236.58536223419	0.000332273580576637\\
1227.40044401774	0.000331353888434799\\
1218.28374812157	0.00033037959723546\\
1209.23476781446	0.000329351121329798\\
1200.25300012896	0.000328268905205588\\
1191.33794583355	0.000327133423804781\\
1182.48910940477	0.000325945182828448\\
1173.70599899975	0.00032470471901467\\
1164.98812642888	0.000323412600373482\\
1156.3350071286	0.000322069426361918\\
1147.74616013456	0.000320675827981949\\
1139.22110805483	0.00031923246778446\\
1130.75937704337	0.000317740039763649\\
1122.3604967737	0.000316199269128254\\
1114.02400041277	0.000314610911938714\\
1105.749424595	0.000312975754602806\\
1097.53630939651	0.000311294613226036\\
1089.38419830959	0.000309568332817146\\
1081.2926382173	0.000307797786353127\\
1073.26117936829	0.00030598387371195\\
1065.2893753518	0.000304127520484588\\
1057.37678307286	0.000302229676680718\\
1049.52296272766	0.000300291315344522\\
1041.72747777907	0.000298313431098291\\
1033.98989493243	0.000296297038632056\\
1026.30978411142	0.000294243171157234\\
1018.6867184342	0.000292152878841424\\
1011.12027418962	0.000290027227240114\\
1003.61003081374	0.000287867295739321\\
996.155570866409	0.000285674176021202\\
988.756480008064	0.000283448970562583\\
981.412346976721	0.000281192791174263\\
974.122763565099	0.000278906757586965\\
966.88732459794	0.000276591996087983\\
959.705627909479	0.000274249638210968\\
952.577274321103	0.000271880819479892\\
945.501867619149	0.000269486678207152\\
938.479014532895	0.000267068354344812\\
931.508324712691	0.000264626988387363\\
924.589410708265	0.000262163720323882\\
917.721887947192	0.00025967968863715\\
910.905374713515	0.000257176029347158\\
904.139492126523	0.000254653875096353\\
897.423864119701	0.000252114354274031\\
890.758117419823	0.000249558590177403\\
884.141881526201	0.000246987700206978\\
877.574788690099	0.000244402795094154\\
871.056473894285	0.000241804978159054\\
864.586574832748	0.000239195344596886\\
858.164731890556	0.000236574980791315\\
851.790588123873	0.000233944963653527\\
845.463789240111	0.000231306359985851\\
839.183983578243	0.000228660225868993\\
832.950822089257	0.000226007606072084\\
826.763958316753	0.000223349533484885\\
820.623048377686	0.000220687028571619\\
814.527750943253	0.000218021098846009\\
808.47772721992	0.000215352738367178\\
802.472640930591	0.000212682927256175\\
796.512158295919	0.000210012631232906\\
790.59594801575	0.00020734280117335\\
784.72368125071	0.000204674372686947\\
778.89503160393	0.000202008265714088\\
773.109675102898	0.000199345384143648\\
767.367290181458	0.000196686615450541\\
761.667557661931	0.000194032830353241\\
756.01016073738	0.00019138488249127\\
750.394784953994	0.000188743608122599\\
744.821118193618	0.000186109825840936\\
739.288850656394	0.000183484336312846\\
733.797674843554	0.00018086792203465\\
728.347285540317	0.000178261347109023\\
722.937379798933	0.000175665357041191\\
717.56765692184	0.000173080678554637\\
712.237818444947	0.000170508019426163\\
706.947568121055	0.000167948068340177\\
701.696611903378	0.000165401494762023\\
696.484657929211	0.000162868948830184\\
691.311416503699	0.000160351061267119\\
686.176600083736	0.000157848443308541\\
681.079923261988	0.000155361686650858\\
676.021102751025	0.00015289136341652\\
670.999857367572	0.000150438026137\\
666.015908016889	0.000148002207753072\\
661.06897767725	0.000145584421632105\\
656.158791384547	0.000143185161602002\\
651.285076217013	0.000140804902001449\\
646.447561280041	0.000138444097746101\\
641.645977691138	0.000136103184410322\\
636.880058564972	0.000133782578324092\\
632.149538998544	0.000131482676684661\\
627.454156056458	0.000129203857682547\\
622.793648756308	0.000126946480641435\\
618.167758054176	0.000124710886171558\\
613.576226830228	0.000122497396336096\\
609.018799874428	0.000120306314830158\\
604.495223872346	0.000118137927171866\\
600.005247391086	0.000115992500905107\\
595.548620865302	0.000113870285813448\\
591.125096583335	0.000111771514144775\\
586.734428673439	0.00010969640084615\\
582.376373090114	0.000107645143808433\\
578.050687600549	0.000105617924120174\\
573.757131771146	0.000103614906330297\\
569.495466954168	0.000101636238719103\\
565.265456274466	9.96820535771064e-05\\
561.066864616317	9.77524674912327e-05\\
556.899458610353	9.58475816378976e-05\\
552.763006620593	9.39674820825025e-05\\
548.657278731565	9.21122400848706e-05\\
544.582046735526	9.02819124101674e-05\\
540.537084119782	8.84765416448421e-05\\
536.522166054094	8.66961565171368e-05\\
532.537069378183	8.49407722217168e-05\\
528.581572589324	8.32103907479808e-05\\
524.655455830038	8.15050012116166e-05\\
520.758500875867	7.98245801889712e-05\\
516.890491123249	7.81690920538235e-05\\
513.051211577476	7.65384893161382e-05\\
509.240448840744	7.49327129624051e-05\\
505.457991100293	7.33516927971666e-05\\
501.703628116633	7.17953477853449e-05\\
497.977151211859	7.02635863949959e-05\\
494.278353258049	6.87563069401198e-05\\
490.607028665758	6.72733979231738e-05\\
486.962973372584	6.58147383769335e-05\\
483.345984831829	6.4380198205371e-05\\
479.755862001239	6.29696385232162e-05\\
476.192405331832	6.15829119938885e-05\\
472.655416756805	6.02198631654849e-05\\
469.144699680525	5.88803288045339e-05\\
465.660058967601	5.75641382272222e-05\\
462.201300932039	5.62711136278206e-05\\
458.768233326478	5.50010704040425e-05\\
455.360665331498	5.37538174790761e-05\\
451.978407545023	5.25291576200494e-05\\
448.621271971783	5.13268877526862e-05\\
445.289072012877	5.01467992719351e-05\\
441.981622455389	4.89886783483492e-05\\
438.698739462102	4.78523062300179e-05\\
435.440240561274	4.67374595398516e-05\\
432.205944636502	4.56439105680374e-05\\
428.995671916648	4.457142755949e-05\\
425.809243965854	4.35197749961333e-05\\
422.646483673618	4.24887138738568e-05\\
419.507215244952	4.14780019740044e-05\\
416.391264190611	4.04873941292585e-05\\
413.298457317395	3.95166424837927e-05\\
410.228622718523	3.85654967475809e-05\\
407.181589764074	3.76337044447527e-05\\
404.157189091507	3.6721011155902e-05\\
401.155252596246	3.58271607542598e-05\\
398.175613422337	3.49518956356544e-05\\
395.21810595317	3.40949569421918e-05\\
392.282565802281	3.3256084779598e-05\\
389.368829804207	3.2435018428175e-05\\
386.476736005421	3.16314965473309e-05\\
383.60612365533	3.0845257373656e-05\\
380.75683319734	3.0076038912524e-05\\
377.928706259986	2.93235791232095e-05\\
375.12158564813	2.85876160975205e-05\\
372.335315334224	2.78678882319551e-05\\
369.569740449639	2.71641343934003e-05\\
366.824707276051	2.64760940784025e-05\\
364.100063236907	2.58035075660449e-05\\
361.395656888935	2.51461160644789e-05\\
358.711337913731	2.4503661851166e-05\\
356.046957109401	2.38758884068911e-05\\
353.402366382274	2.32625405436251e-05\\
350.777418738662	2.26633645263122e-05\\
348.171968276697	2.20781081886764e-05\\
345.585870178217	2.15065210431392e-05\\
343.018980700718	2.09483543849571e-05\\
340.471157169366	2.04033613906873e-05\\
337.942257969062	1.98712972110993e-05\\
335.432142536577	1.93519190586582e-05\\
332.940671352735	1.88449862897056e-05\\
330.467705934659	1.8350260481475e-05\\
328.013108828071	1.78675055040777e-05\\
325.57674359966	1.73964875876027e-05\\
323.158474829489	1.69369753844753e-05\\
320.758168103475	1.64887400272213e-05\\
318.375690005912	1.60515551817884e-05\\
316.010908112063	1.56251970965731e-05\\
313.663690980792	1.52094446473062e-05\\
311.333908147261	1.4804079377948e-05\\
309.02143011568	1.4408885537745e-05\\
306.726128352108	1.40236501145958e-05\\
304.447875277308	1.36481628648774e-05\\
302.186544259657	1.32822163398767e-05\\
299.942009608104	1.29256059089702e-05\\
297.714146565192	1.25781297796948e-05\\
295.502831300113	1.22395890148451e-05\\
293.307940901833	1.19097875467328e-05\\
291.129353372257	1.15885321887365e-05\\
288.96694761945	1.12756326442701e-05\\
286.820603450903	1.09709015132882e-05\\
284.690201566855	1.0674154296449e-05\\
282.575623553662	1.03852093970454e-05\\
280.476751877215	1.01038881208123e-05\\
278.393469876404	9.83001467371549e-06\\
276.325661756641	9.56341615781843e-06\\
274.273212583414	9.30392256532434e-06\\
272.236008275909	9.05136677088177e-06\\
270.213935600659	8.80558452224064e-06\\
268.206882165259	8.56641442934015e-06\\
266.214736412113	8.33369795190703e-06\\
264.237387612236	8.1072793856375e-06\\
262.274725859099	7.88700584703465e-06\\
260.326642062517	7.67272725696716e-06\\
258.393027942593	7.4642963230147e-06\\
256.473776023691	7.26156852065989e-06\\
254.568779628468	7.06440207338599e-06\\
252.677932871941	6.87265793173576e-06\\
250.801130655604	6.68619975138426e-06\\
248.938268661586	6.50489387027779e-06\\
247.089243346851	6.32860928488677e-06\\
245.253951937446	6.15721762562107e-06\\
243.432292422782	5.99059313145219e-06\\
241.624163549975	5.82861262378705e-06\\
239.829464818207	5.67115547963546e-06\\
238.048096473146	5.5181036041129e-06\\
236.279959501398	5.36934140231818e-06\\
234.524955625009	5.22475575062554e-06\\
232.782987295997	5.08423596742855e-06\\
231.05395769093	4.94767378337334e-06\\
229.33777070555	4.81496331111712e-06\\
227.634330949423	4.68600101464714e-06\\
225.943543740646	4.56068567819524e-06\\
224.265315100576	4.43891837478127e-06\\
222.599551748611	4.32060243441905e-06\\
220.946161097006	4.20564341201703e-06\\
219.305051245723	4.09394905500588e-06\\
217.676130977326	3.98542927072377e-06\\
216.059309751909	3.87999609359043e-06\\
214.454497702065	3.77756365209962e-06\\
212.86160562789	3.67804813565936e-06\\
211.280544992026	3.58136776130879e-06\\
209.711227914737	3.48744274033942e-06\\
208.15356716903	3.39619524484835e-06\\
206.6074761758	3.30754937425008e-06\\
205.072868999023	3.22143112177301e-06\\
203.549660340977	3.13776834096576e-06\\
202.037765537499	3.0564907122382e-06\\
200.537100553285	2.977529709461e-06\\
199.047581977213	2.90081856664683e-06\\
197.569127017711	2.82629224473573e-06\\
196.101653498152	2.75388739850638e-06\\
194.645079852288	2.68354234363421e-06\\
193.199325119717	2.6151970239164e-06\\
191.764308941382	2.54879297868352e-06\\
190.339951555105	2.48427331041603e-06\\
188.926173791152	2.42158265258389e-06\\
187.522897067834	2.36066713772613e-06\\
186.13004338714	2.30147436578676e-06\\
184.7475353304	2.24395337272261e-06\\
183.375296053983	2.18805459939765e-06\\
182.013249285024	2.13372986077813e-06\\
180.661319317186	2.08093231544125e-06\\
179.319431006454	2.02961643541029e-06\\
177.987509766954	1.97973797632756e-06\\
176.665481566809	1.93125394797626e-06\\
175.353272924028	1.88412258516139e-06\\
174.050810902413	1.83830331895927e-06\\
172.758023107516	1.79375674834439e-06\\
171.474837682604	1.75044461220157e-06\\
170.201183304674	1.7083297617311e-06\\
168.936989180483	1.66737613325316e-06\\
167.682185042617	1.62754872141806e-06\\
166.436701145581	1.58881355282721e-06\\
165.200468261928	1.55113766007013e-06\\
163.973417678405	1.51448905618125e-06\\
162.755481192139	1.4788367095205e-06\\
161.546591106842	1.44415051908035e-06\\
160.34668022905	1.41040129022215e-06\\
159.15568186439	1.37756071084348e-06\\
157.97352981387	1.34560132797807e-06\\
156.800158370201	1.31449652482919e-06\\
155.635502314145	1.28422049823706e-06\\
154.479496910888	1.25474823658021e-06\\
153.332077906443	1.2260554981104e-06\\
152.193181524082	1.19811878972037e-06\\
151.062744460785	1.17091534614302e-06\\
149.940703883725	1.14442310958047e-06\\
148.826997426776	1.11862070976111e-06\\
147.721563187044	1.09348744442217e-06\\
146.624339721429	1.06900326021527e-06\\
145.535266043209	1.0451487340319e-06\\
144.454281618647	1.02190505474569e-06\\
143.381326363632	9.99254005367829e-07\\
142.316340640336	9.77177945611941e-07\\
141.259265253899	9.55659794864288e-07\\
140.21004144914	9.34683015555185e-07\\
139.168610907289	9.14231596927019e-07\\
138.134915742751	8.94290039194322e-07\\
137.108898499881	8.74843338090985e-07\\
136.090502149795	8.55876969799656e-07\\
135.0796700872	8.37376876258135e-07\\
134.076346127247	8.19329450837457e-07\\
133.080474502409	8.01721524386284e-07\\
132.091999859379	7.84540351635993e-07\\
131.110867255994	7.67773597960907e-07\\
130.137022158185	7.51409326487842e-07\\
129.17041043694	7.3543598554925e-07\\
128.2109783653	7.19842396473995e-07\\
127.258672615369	7.04617741709903e-07\\
126.313440255352	6.89751553272038e-07\\
125.375228746615	6.75233701510714e-07\\
124.44398594076	6.61054384193186e-07\\
123.519660076729	6.47204115892931e-07\\
122.602199777928	6.3367371768046e-07\\
121.691554049369	6.20454307109543e-07\\
120.787672274837	6.07537288492774e-07\\
119.890504214077	5.9491434346039e-07\\
119	5.82577421796305e-07\\
}--cycle;
\addplot[ybar interval, fill=black, fill opacity=0.35, area legend, draw=none] table[row sep=crcr, x=Lower, y=Count] {%
Lower	Upper	Count\\
119	166.945095217458	0.000111238351058511\\
166.945095217458	234.207267371144	0.00027752102064167\\
234.207267371144	328.569365982322	0.000254339404837665\\
328.569365982322	460.949950331585	0.000161151527153326\\
460.949950331585	646.666666794865	0.000358969660546674\\
646.666666794865	907.208640857353	0.000327522402639305\\
907.208640857353	1272.72296579858	0.000423148030358411\\
1272.72296579858	1785.50299756883	0.000239218883471713\\
1785.50299756883	2504.88208353096	0.000159396719899502\\
2504.88208353096	3514.09897431581	0.000118904074134826\\
3514.09897431581	4929.92930983802	6.78047345020274e-05\\
4929.92930983802	6916.19763064073	2.81935725468193e-05\\
6916.19763064073	9702.73337806785	1.53117241384977e-05\\
9702.73337806785	13611.9642661441	8.18575339144199e-06\\
13611.9642661441	19096.2241219165	1.45872008445765e-06\\
19096.2241219165	26790.0920531703	1.03978909847186e-06\\
26790.0920531703	37583.8190647142	9.88228315878168e-07\\
37583.8190647142	52726.3382554153	3.52209118322166e-07\\
52726.3382554153	73969.7778194808	1.25528950178929e-07\\
73969.7778194808	103772.198330147	0\\
103772.198330147	145582.012866817	6.37808776771247e-08\\
145582.012866817	204237	4.5463596481774e-08\\
204237	204237	4.5463596481774e-08\\
};
\addlegendentry{Istogramma reale}

\addplot [color=black]
table[row sep=crcr]{%
119	6.15433567615765e-05\\
119.890504214077	6.24706820220498e-05\\
120.787672274837	6.34079858824207e-05\\
121.691554049369	6.43552990703441e-05\\
122.602199777928	6.53126508675256e-05\\
123.519660076729	6.62800690806998e-05\\
124.44398594076	6.72575800126463e-05\\
125.375228746615	6.8245208433252e-05\\
126.313440255352	6.92429775506336e-05\\
127.258672615369	7.02509089823314e-05\\
128.2109783653	7.12690227265871e-05\\
129.17041043694	7.22973371337154e-05\\
130.137022158185	7.33358688775852e-05\\
131.110867255994	7.43846329272213e-05\\
132.091999859379	7.54436425185364e-05\\
133.080474502409	7.65129091262124e-05\\
134.076346127247	7.75924424357363e-05\\
135.0796700872	7.86822503156108e-05\\
136.090502149795	7.97823387897457e-05\\
137.108898499881	8.08927120100506e-05\\
138.134915742751	8.20133722292341e-05\\
139.168610907289	8.31443197738299e-05\\
140.21004144914	8.42855530174598e-05\\
141.259265253899	8.54370683543451e-05\\
142.316340640336	8.65988601730861e-05\\
143.381326363632	8.77709208307162e-05\\
144.454281618647	8.89532406270501e-05\\
145.535266043209	9.01458077793349e-05\\
146.624339721429	9.13486083972235e-05\\
147.721563187044	9.25616264580771e-05\\
148.826997426776	9.37848437826191e-05\\
149.940703883725	9.50182400109448e-05\\
151.062744460785	9.62617925789103e-05\\
152.193181524082	9.7515476694907e-05\\
153.332077906443	9.87792653170396e-05\\
154.479496910888	0.000100053129130722\\
155.635502314145	0.0001013370365267\\
156.800158370201	0.000102630953579524\\
157.97352981387	0.000103934844026472\\
159.15568186439	0.000105248669246958\\
160.34668022905	0.000106572388242409\\
161.546591106842	0.000107905957616656\\
162.755481192139	0.000109249331556834\\
163.973417678405	0.00011060246181481\\
165.200468261928	0.000111965297689149\\
166.436701145581	0.000113337786007639\\
167.682185042617	0.000114719871110382\\
168.936989180483	0.000116111494833468\\
170.201183304674	0.000117512596493243\\
171.474837682604	0.000118923112871184\\
172.758023107516	0.000120342978199402\\
174.050810902413	0.000121772124146765\\
175.353272924028	0.000123210479805689\\
176.665481566809	0.000124657971679569\\
177.987509766954	0.000126114523670893\\
179.319431006454	0.000127580057070043\\
180.661319317186	0.000129054490544787\\
182.013249285024	0.000130537740130483\\
183.375296053983	0.000132029719221008\\
184.7475353304	0.000133530338560412\\
186.13004338714	0.000135039506235326\\
187.522897067834	0.000136557127668123\\
188.926173791152	0.000138083105610844\\
190.339951555105	0.000139617340139902\\
191.764308941382	0.000141159728651583\\
193.199325119717	0.000142710165858337\\
194.645079852288	0.000144268543785884\\
196.101653498152	0.000145834751771141\\
197.569127017711	0.000147408676460974\\
199.047581977213	0.000148990201811795\\
200.537100553285	0.000150579209090004\\
202.037765537499	0.00015217557687329\\
203.549660340977	0.000153779181052795\\
205.072868999023	0.00015538989483616\\
206.6074761758	0.000157007588751446\\
208.15356716903	0.000158632130651951\\
209.711227914737	0.000160263385721917\\
211.280544992026	0.000161901216483158\\
212.86160562789	0.00016354548280258\\
214.454497702065	0.000165196041900635\\
216.059309751909	0.000166852748360693\\
217.676130977326	0.000168515454139346\\
219.305051245723	0.000170184008577645\\
220.946161097006	0.000171858258413285\\
222.599551748611	0.000173538047793727\\
224.265315100576	0.000175223218290276\\
225.943543740646	0.000176913608913115\\
227.634330949423	0.000178609056127285\\
229.33777070555	0.000180309393869643\\
231.05395769093	0.00018201445356677\\
232.782987295997	0.000183724064153848\\
234.524955625009	0.000185438052094511\\
236.279959501398	0.000187156241401662\\
238.048096473146	0.000188878453659255\\
239.829464818207	0.000190604508045058\\
241.624163549975	0.000192334221354384\\
243.432292422782	0.000194067408024791\\
245.253951937446	0.00019580388016176\\
247.089243346851	0.000197543447565335\\
248.938268661586	0.000199285917757737\\
250.801130655604	0.000201031096011948\\
252.677932871941	0.000202778785381251\\
254.568779628468	0.000204528786729739\\
256.473776023691	0.000206280898763782\\
258.393027942593	0.000208034918064446\\
260.326642062517	0.000209790639120867\\
262.274725859099	0.000211547854364566\\
264.237387612236	0.000213306354204713\\
266.214736412113	0.000215065927064321\\
268.206882165259	0.000216826359417371\\
270.213935600659	0.000218587435826866\\
272.236008275909	0.000220348938983791\\
274.273212583414	0.000222110649746995\\
276.325661756641	0.000223872347183965\\
278.393469876404	0.0002256338086125\\
280.476751877215	0.000227394809643262\\
282.575623553662	0.000229155124223215\\
284.690201566855	0.000230914524679913\\
286.820603450903	0.00023267278176666\\
288.96694761945	0.000234429664708507\\
291.129353372257	0.000236184941249081\\
293.307940901833	0.000237938377698245\\
295.502831300113	0.000239689738980561\\
297.714146565192	0.000241438788684555\\
299.942009608104	0.00024318528911277\\
302.186544259657	0.000244929001332583\\
304.447875277308	0.000246669685227792\\
306.726128352108	0.000248407099550941\\
309.02143011568	0.000250141001976374\\
311.333908147261	0.000251871149154006\\
313.663690980792	0.000253597296763796\\
316.010908112063	0.000255319199570902\\
318.375690005912	0.000257036611481498\\
320.758168103475	0.000258749285599256\\
323.158474829489	0.000260456974282443\\
325.57674359966	0.000262159429201651\\
328.013108828071	0.000263856401398117\\
330.467705934659	0.000265547641342626\\
332.940671352735	0.000267232898994974\\
335.432142536577	0.000268911923863977\\
337.942257969062	0.000270584465067997\\
340.471157169366	0.000272250271395982\\
343.018980700718	0.000273909091368981\\
345.585870178217	0.000275560673302125\\
348.171968276697	0.000277204765367053\\
350.777418738662	0.000278841115654756\\
353.402366382274	0.000280469472238825\\
356.046957109401	0.000282089583239075\\
358.711337913731	0.000283701196885528\\
361.395656888935	0.000285304061582725\\
364.100063236907	0.000286897925974362\\
366.824707276051	0.0002884825390082\\
369.569740449639	0.000290057650001255\\
372.335315334224	0.000291623008705221\\
375.12158564813	0.000293178365372119\\
377.928706259986	0.000294723470820138\\
380.75683319734	0.000296258076499648\\
383.60612365533	0.000297781934559367\\
386.476736005421	0.000299294797912638\\
389.368829804207	0.000300796420303823\\
392.282565802281	0.000302286556374751\\
395.21810595317	0.000303764961731233\\
398.175613422337	0.000305231393009589\\
401.155252596246	0.000306685607943183\\
404.157189091507	0.000308127365428926\\
407.181589764074	0.000309556425593725\\
410.228622718523	0.000310972549860868\\
413.298457317395	0.000312375501016291\\
416.391264190611	0.000313765043274725\\
419.507215244952	0.000315140942345691\\
422.646483673618	0.00031650296549931\\
425.809243965854	0.000317850881631909\\
428.995671916648	0.000319184461331401\\
432.205944636502	0.0003205034769424\\
435.440240561274	0.00032180770263106\\
438.698739462102	0.000323096914449605\\
441.981622455389	0.000324370890400516\\
445.289072012877	0.000325629410500373\\
448.621271971783	0.0003268722568433\\
451.978407545023	0.000328099213664001\\
455.360665331498	0.000329310067400362\\
458.768233326478	0.000330504606755593\\
462.201300932039	0.000331682622759872\\
465.660058967601	0.000332843908831491\\
469.144699680525	0.000333988260837455\\
472.655416756805	0.000335115477153521\\
476.192405331832	0.00033622535872365\\
479.755862001239	0.000337317709118845\\
483.345984831829	0.00033839233459536\\
486.962973372584	0.000339449044152243\\
490.607028665758	0.000340487649588191\\
494.278353258049	0.000341507965557711\\
497.977151211859	0.000342509809626532\\
501.703628116633	0.000343493002326277\\
505.457991100293	0.000344457367208351\\
509.240448840744	0.000345402730897028\\
513.051211577476	0.00034632892314172\\
516.890491123249	0.000347235776868395\\
520.758500875867	0.000348123128230135\\
524.655455830038	0.000348990816656802\\
528.581572589324	0.000349838684903794\\
532.537069378183	0.000350666579099879\\
536.522166054094	0.000351474348794067\\
540.537084119782	0.000352261847001522\\
544.582046735526	0.000353028930248476\\
548.657278731565	0.000353775458616143\\
552.763006620593	0.000354501295783596\\
556.899458610353	0.000355206309069601\\
561.066864616317	0.000355890369473389\\
565.265456274466	0.000356553351714344\\
569.495466954168	0.000357195134270587\\
573.757131771146	0.000357815599416454\\
578.050687600549	0.000358414633258833\\
582.376373090114	0.000358992125772356\\
586.734428673439	0.000359547970833432\\
591.125096583335	0.000360082066253094\\
595.548620865302	0.000360594313808658\\
600.005247391086	0.00036108461927418\\
604.495223872346	0.000361552892449684\\
609.018799874428	0.000361999047189168\\
613.576226830228	0.000362423001427358\\
618.167758054176	0.000362824677205209\\
622.793648756308	0.00036320400069414\\
627.454156056458	0.000363560902218991\\
632.149538998544	0.000363895316279688\\
636.880058564972	0.000364207181571616\\
641.645977691138	0.000364496441004685\\
646.447561280041	0.000364763041721074\\
651.285076217013	0.000365006935111666\\
656.158791384547	0.000365228076831142\\
661.06897767725	0.000365426426811744\\
666.015908016889	0.000365601949275691\\
670.999857367572	0.000365754612746258\\
676.021102751025	0.000365884390057487\\
681.079923261988	0.000365991258362556\\
686.176600083736	0.000366075199140777\\
691.311416503699	0.000366136198203236\\
696.484657929211	0.000366174245697065\\
701.696611903378	0.000366189336108349\\
706.947568121055	0.000366181468263655\\
712.237818444947	0.000366150645330204\\
717.56765692184	0.000366096874814667\\
722.937379798933	0.000366020168560586\\
728.347285540317	0.000365920542744439\\
733.797674843554	0.000365798017870328\\
739.288850656394	0.000365652618763311\\
744.821118193618	0.000365484374561366\\
750.394784953994	0.000365293318706004\\
756.01016073738	0.000365079488931525\\
761.667557661931	0.000364842927252928\\
767.367290181458	0.000364583679952474\\
773.109675102898	0.000364301797564917\\
778.89503160393	0.000363997334861401\\
784.72368125071	0.000363670350832035\\
790.59594801575	0.000363320908667153\\
796.512158295919	0.000362949075737264\\
802.472640930591	0.000362554923571706\\
808.47772721992	0.000362138527836009\\
814.527750943253	0.000361699968307982\\
820.623048377686	0.000361239328852526\\
826.763958316753	0.000360756697395194\\
832.950822089257	0.000360252165894505\\
839.183983578243	0.000359725830313021\\
845.463789240111	0.000359177790587206\\
851.790588123873	0.000358608150596073\\
858.164731890556	0.000358017018128644\\
864.586574832748	0.000357404504850221\\
871.056473894285	0.0003567707262675\\
877.574788690099	0.000356115801692531\\
884.141881526201	0.00035543985420554\\
890.758117419823	0.000354743010616642\\
897.423864119701	0.000354025401426445\\
904.139492126523	0.00035328716078557\\
910.905374713515	0.000352528426453111\\
917.721887947192	0.000351749339754033\\
924.589410708265	0.000350950045535554\\
931.508324712691	0.000350130692122507\\
938.479014532895	0.000349291431271708\\
945.501867619149	0.000348432418125357\\
952.577274321103	0.00034755381116348\\
959.705627909479	0.000346655772155446\\
966.88732459794	0.000345738466110558\\
974.122763565099	0.000344802061227776\\
981.412346976721	0.000343846728844542\\
988.756480008064	0.000342872643384783\\
996.155570866409	0.000341879982306074\\
1003.61003081374	0.000340868926045999\\
1011.12027418962	0.000339839657967735\\
1018.6867184342	0.000338792364304878\\
1026.30978411142	0.00033772723410553\\
1033.98989493243	0.000336644459175681\\
1041.72747777907	0.000335544234021896\\
1049.52296272766	0.000334426755793345\\
1057.37678307286	0.000333292224223194\\
1065.2893753518	0.000332140841569372\\
1073.26117936829	0.000330972812554755\\
1081.2926382173	0.000329788344306786\\
1089.38419830959	0.000328587646296545\\
1097.53630939651	0.000327370930277311\\
1105.749424595	0.000326138410222628\\
1114.02400041277	0.00032489030226391\\
1122.3604967737	0.000323626824627597\\
1130.75937704337	0.00032234819757191\\
1139.22110805483	0.000321054643323199\\
1147.74616013456	0.000319746386011945\\
1156.3350071286	0.00031842365160841\\
1164.98812642888	0.000317086667857984\\
1173.70599899975	0.000315735664216239\\
1182.48910940477	0.000314370871783724\\
1191.33794583355	0.000312992523240529\\
1200.25300012896	0.000311600852780635\\
1209.23476781446	0.000310196096046081\\
1218.28374812157	0.000308778490060974\\
1227.40044401774	0.000307348273165368\\
1236.58536223419	0.000305905684949036\\
1245.83901329415	0.000304450966185159\\
1255.16191154121	0.00030298435876396\\
1264.5545751679	0.000301506105626311\\
1274.01752624451	0.000300016450697333\\
1283.55129074812	0.000298515638820012\\
1293.15639859177	0.000297003915688868\\
1302.83338365401	0.000295481527783689\\
1312.5827838085	0.00029394872230336\\
1322.40514095393	0.000292405747099813\\
1332.30100104415	0.000290852850612119\\
1342.27091411851	0.000289290281800749\\
1352.31543433242	0.000287718290082025\\
1362.43511998817	0.00028613712526279\\
1372.63053356595	0.00028454703747531\\
1382.90224175512	0.000282948277112448\\
1393.2508154857	0.000281341094763113\\
1403.67682996012	0.000279725741148025\\
1414.18086468518	0.000278102467055806\\
1424.76350350425	0.000276471523279424\\
1435.42533462974	0.000274833160553015\\
1446.16695067579	0.000273187629489099\\
1456.98894869122	0.000271535180516212\\
1467.89193019267	0.000269876063816979\\
1478.8765011981	0.000268210529266647\\
1489.94327226042	0.000266538826372096\\
1501.09285850146	0.00026486120421135\\
1512.32587964614	0.00026317791137361\\
1523.64296005691	0.000261489195899823\\
1535.0447287685	0.000259795305223809\\
1546.53181952283	0.000258096486113968\\
1558.10487080425	0.000256392984615578\\
1569.76452587505	0.000254685045993705\\
1581.51143281119	0.000252972914676749\\
1593.34624453832	0.00025125683420063\\
1605.26961886811	0.000249537047153643\\
1617.28221853476	0.00024781379512199\\
1629.38471123188	0.000246087318636008\\
1641.57776964957	0.000244357857117114\\
1653.86207151182	0.000242625648825466\\
1666.2382996142	0.000240890930808374\\
1678.70714186178	0.000239153938849465\\
1691.26929130741	0.000237414907418612\\
1703.92544619016	0.000235674069622648\\
1716.67630997424	0.000233931657156878\\
1729.522591388	0.000232187900257394\\
1742.4650044634	0.00023044302765421\\
1755.50426857564	0.000228697266525232\\
1768.64110848317	0.000226950842451062\\
1781.876254368	0.000225203979370667\\
1795.21044187622	0.000223456899537894\\
1808.64441215895	0.000221709823478873\\
1822.17891191353	0.000219962969950293\\
1835.81469342496	0.000218216555898569\\
1849.55251460781	0.00021647079641991\\
1863.39313904828	0.000214725904721292\\
1877.33733604663	0.000212982092082346\\
1891.38588066002	0.000211239567818164\\
1905.53955374551	0.000209498539243033\\
1919.7991420035	0.000207759211635107\\
1934.16543802145	0.000206021788202011\\
1948.63924031791	0.00020428647004739\\
1963.22135338698	0.000202553456138409\\
1977.91258774292	0.000200822943274204\\
1992.71375996528	0.000199095126055291\\
2007.62569274426	0.00019737019685393\\
2022.64921492642	0.000195648345785452\\
2037.7851615608	0.000193929760680558\\
2053.03437394529	0.000192214627058573\\
2068.39769967338	0.000190503128101675\\
2083.87599268133	0.000188795444630095\\
2099.47011329558	0.000187091755078275\\
2115.18092828061	0.000185392235472011\\
2131.00931088707	0.000183697059406553\\
2146.95614090037	0.000182006398025679\\
2163.02230468953	0.000180320420001736\\
2179.20869525649	0.000178639291516645\\
2195.51621228573	0.000176963176243875\\
2211.94576219425	0.000175292235331371\\
2228.49825818201	0.000173626627385446\\
2245.17462028264	0.000171966508455627\\
2261.97577541457	0.000170312032020456\\
2278.90265743261	0.000168663348974233\\
2295.95620717979	0.00016702060761471\\
2313.1373725397	0.000165383953631718\\
2330.44710848916	0.000163753530096733\\
2347.88637715129	0.000162129477453368\\
2365.45614784899	0.000160511933508786\\
2383.15739715885	0.000158901033426038\\
2400.9911089654	0.0001572969097173\\
2418.95827451578	0.000155699692238024\\
2437.05989247487	0.000154109508181976\\
2455.29696898081	0.000152526482077175\\
2473.67051770087	0.000150950735782697\\
2492.18155988785	0.00014938238848637\\
2510.8311244368	0.000147821556703319\\
2529.62024794223	0.00014626835427538\\
2548.54997475574	0.000144722892371345\\
2567.62135704402	0.000143185279488066\\
2586.83545484739	0.000141655621452367\\
2606.1933361387	0.000140134021423791\\
2625.69607688265	0.000138620579898144\\
2645.34476109567	0.000137115394711847\\
2665.14048090611	0.000135618561047068\\
2685.08433661496	0.000134130171437636\\
2705.17743675704	0.000132650315775724\\
2725.42089816257	0.000131179081319278\\
2745.81584601927	0.000129716552700199\\
2766.3634139349	0.000128262811933256\\
2787.06474400025	0.000126817938425714\\
2807.92098685266	0.000125382008987678\\
2828.93330173995	0.000123955097843126\\
2850.10285658484	0.000122537276641634\\
2871.4308280499	0.000121128614470767\\
2892.91840160291	0.000119729177869136\\
2914.56677158281	0.00011833903084009\\
2936.37714126603	0.000116958234866055\\
2958.3507229334	0.000115586848923486\\
2980.48873793751	0.000114224929498426\\
3002.79241677064	0.000112872530602668\\
3025.2629991331	0.000111529703790484\\
3047.90173400216	0.000110196498175937\\
3070.7098797015	0.000108872960450736\\
3093.68870397109	0.000107559134902637\\
3116.83948403772	0.000106255063434369\\
3140.16350668593	0.000104960785583074\\
3163.66206832958	0.000103676338540252\\
3187.33647508389	0.000102401757172176\\
3211.18804283803	0.000101137074040801\\
3235.21809732829	9.98823194251117e-05\\
3259.42797421173	9.86375213429245e-05\\
3283.81901914043	9.74027055731192e-05\\
3308.39258783631	9.61778956782835e-05\\
3333.15004616647	9.49631130277623e-05\\
3358.09277021909	9.37583768210949e-05\\
3383.22214637995	9.25637041118269e-05\\
3408.53957140944	9.13791098316827e-05\\
3434.04645252026	9.02046068150839e-05\\
3459.7442074556	8.90402058240023e-05\\
3485.63426456793	8.78859155731309e-05\\
3511.71806289842	8.67417427553602e-05\\
3537.99705225692	8.56076920675463e-05\\
3564.47269330252	8.44837662365574e-05\\
3591.14645762477	8.33699660455842e-05\\
3618.01982782546	8.22662903607014e-05\\
3645.09429760103	8.11727361576664e-05\\
3672.37137182558	8.00892985489419e-05\\
3699.85256663454	7.90159708109293e-05\\
3727.53940950892	7.79527444114009e-05\\
3755.43343936023	7.6899609037114e-05\\
3783.53620661599	7.58565526215985e-05\\
3811.84927330594	7.48235613731023e-05\\
3840.37421314884	7.3800619802683e-05\\
3869.11261163994	7.27877107524313e-05\\
3898.06606613913	7.17848154238169e-05\\
3927.23618595969	7.07919134061408e-05\\
3956.62459245778	6.98089827050847e-05\\
3986.23291912252	6.88359997713438e-05\\
4016.06281166681	6.78729395293304e-05\\
4046.1159281188	6.69197754059387e-05\\
4076.39393891404	6.59764793593567e-05\\
4106.89852698833	6.50430219079152e-05\\
4137.63138787127	6.41193721589619e-05\\
4168.59422978048	6.32054978377485e-05\\
4199.78877371658	6.23013653163216e-05\\
4231.21675355883	6.1406939642405e-05\\
4262.8799161615	6.05221845682632e-05\\
4294.78002145095	5.96470625795352e-05\\
4326.91884252352	5.87815349240279e-05\\
4359.29816574399	5.79255616404599e-05\\
4391.91979084494	5.70791015871453e-05\\
4424.78553102675	5.62421124706063e-05\\
4457.89721305839	5.54145508741065e-05\\
4491.25667737898	5.45963722860944e-05\\
4524.86577820005	5.37875311285491e-05\\
4558.72638360862	5.29879807852169e-05\\
4592.84037567103	5.21976736297316e-05\\
4627.20965053756	5.14165610536089e-05\\
4661.83611854782	5.06445934941069e-05\\
4696.72170433693	4.98817204619439e-05\\
4731.86834694246	4.91278905688654e-05\\
4767.27799991228	4.83830515550516e-05\\
4802.95263141311	4.76471503163594e-05\\
4838.89422433987	4.69201329313902e-05\\
4875.10477642598	4.62019446883746e-05\\
4911.58630035433	4.54925301118708e-05\\
4948.34082386919	4.47918329892644e-05\\
4985.37038988889	4.40997963970679e-05\\
5022.6770566194	4.34163627270103e-05\\
5060.2628976687	4.27414737119108e-05\\
5098.13000216207	4.20750704513305e-05\\
5136.28047485818	4.14170934369975e-05\\
5174.7164362661	4.07674825779975e-05\\
5213.44002276314	4.01261772257253e-05\\
5252.45338671363	3.94931161985917e-05\\
5291.7586965885	3.88682378064814e-05\\
5331.3581370859	3.82514798749548e-05\\
5371.25390925253	3.76427797691916e-05\\
5411.44823060603	3.70420744176691e-05\\
5451.94333525824	3.64493003355719e-05\\
5492.74147403939	3.586439364793e-05\\
5533.84491462315	3.52872901124785e-05\\
5575.25594165272	3.47179251422373e-05\\
5616.97685686784	3.41562338278064e-05\\
5659.00997923266	3.3602150959373e-05\\
5701.35764506468	3.30556110484279e-05\\
5744.02220816459	3.25165483491873e-05\\
5787.00603994713	3.19848968797178e-05\\
5830.31152957285	3.14605904427611e-05\\
5873.94108408097	3.09435626462567e-05\\
5917.8971285231	3.04337469235594e-05\\
5962.18210609809	2.99310765533504e-05\\
6006.79847828778	2.94354846792385e-05\\
6051.74872499389	2.89469043290512e-05\\
6097.03534467575	2.84652684338132e-05\\
6142.66085448928	2.79905098464102e-05\\
6188.62779042682	2.75225613599383e-05\\
6234.93870745815	2.70613557257357e-05\\
6281.59617967246	2.66068256710975e-05\\
6328.60280042144	2.6158903916672e-05\\
6375.9611824634	2.57175231935377e-05\\
6423.67395810857	2.52826162599604e-05\\
6471.74377936531	2.48541159178301e-05\\
6520.17331808759	2.44319550287779e-05\\
6568.96526612346	2.40160665299724e-05\\
6618.12233546469	2.36063834495946e-05\\
6667.64725839753	2.3202838921994e-05\\
6717.54278765452	2.28053662025234e-05\\
6767.81169656754	2.24138986820548e-05\\
6818.45677922192	2.20283699011764e-05\\
6869.48085061182	2.16487135640707e-05\\
6920.88674679662	2.12748635520755e-05\\
6972.67732505857	2.09067539369289e-05\\
7024.85546406162	2.05443189936977e-05\\
7077.42406401143	2.01874932133927e-05\\
7130.38604681656	1.98362113152703e-05\\
7183.7443562509	1.9490408258823e-05\\
7237.50195811723	1.91500192554598e-05\\
7291.66184041212	1.88149797798786e-05\\
7346.22701349204	1.84852255811309e-05\\
7401.20051024062	1.81606926933831e-05\\
7456.58538623724	1.78413174463742e-05\\
7512.38471992689	1.75270364755729e-05\\
7568.60161279127	1.7217786732036e-05\\
7625.23918952119	1.69135054919702e-05\\
7682.30059819021	1.66141303660001e-05\\
7739.78901042966	1.63195993081441e-05\\
7797.70762160489	1.60298506245014e-05\\
7856.05965099295	1.57448229816515e-05\\
7914.84834196143	1.54644554147702e-05\\
7974.0769621488	1.51886873354643e-05\\
8033.748803646	1.49174585393269e-05\\
8093.86718317946	1.46507092132169e-05\\
8154.43544229543	1.43883799422667e-05\\
8215.4569475457	1.41304117166181e-05\\
8276.93509067475	1.38767459378923e-05\\
8338.87328880825	1.3627324425396e-05\\
8401.27498464301	1.33820894220657e-05\\
8464.14364663834	1.31409836001556e-05\\
8527.48276920879	1.29039500666699e-05\\
8591.29587291844	1.26709323685444e-05\\
8655.58650467655	1.24418744975804e-05\\
8720.35823793472	1.22167208951337e-05\\
8785.61467288549	1.19954164565621e-05\\
8851.35943666247	1.17779065354363e-05\\
8917.59618354194	1.15641369475148e-05\\
8984.32859514597	1.13540539744896e-05\\
9051.56038064706	1.11476043675033e-05\\
9119.29527697427	1.0944735350443e-05\\
9187.53704902097	1.07453946230138e-05\\
9256.28948985408	1.05495303635959e-05\\
9325.55642092493	1.03570912318877e-05\\
9395.34169228163	1.01680263713404e-05\\
9465.64918278304	9.98228541138656e-06\\
9536.48280031448	9.79981846946555e-06\\
9607.84648200483	9.62057615285162e-06\\
9679.74419444542	9.44450956028615e-06\\
9752.17993391047	9.27157028341911e-06\\
9825.15772657923	9.10171040806262e-06\\
9898.68162875981	8.93488251526058e-06\\
9972.75572711457	8.77103968217818e-06\\
10047.3841388873	8.61013548281467e-06\\
10122.5710121321	8.45212398854311e-06\\
10198.3205259438	8.29695976848078e-06\\
10274.6368906905	8.1445978896941e-06\\
10351.5243482474	7.99499391724133e-06\\
10428.9871722326	7.84810391405707e-06\\
10507.0296682446	7.7038844406816e-06\\
10585.6561741017	7.56229255483945e-06\\
10664.8710600833	7.42328581087016e-06\\
10744.6787291723	7.28682225901497e-06\\
10825.0836173003	7.15286044456331e-06\\
10906.0901935939	7.0213594068619e-06\\
10987.7029606233	6.89227867819083e-06\\
11069.9264546524	6.76557828250964e-06\\
11152.765245891	6.64121873407669e-06\\
11236.2239387488	6.51916103594578e-06\\
11320.3071720913	6.39936667834311e-06\\
11405.019619498	6.28179763692793e-06\\
11490.3659895215	6.1664163709406e-06\\
11576.3510259497	6.0531858212408e-06\\
11662.9795080693	5.94206940823985e-06\\
11750.2562509318	5.83303102972999e-06\\
11838.1861056203	5.72603505861398e-06\\
11926.7739595202	5.6210463405384e-06\\
12016.0247365899	5.51803019143335e-06\\
12105.9433976352	5.41695239496236e-06\\
12196.5349405845	5.3177791998852e-06\\
12287.8044007671	5.22047731733644e-06\\
12379.7568511926	5.1250139180237e-06\\
12472.3974028332	5.0313566293474e-06\\
12565.7312049077	4.9394735324459e-06\\
12659.7634451676	4.8493331591687e-06\\
12754.4993501855	4.76090448898006e-06\\
12849.9441856459	4.67415694579706e-06\\
12946.1032566372	4.58906039476348e-06\\
13042.9819079474	4.50558513896344e-06\\
13140.5855243605	4.42370191607697e-06\\
13238.9195309561	4.34338189498013e-06\\
13337.9893934111	4.26459667229273e-06\\
13437.8006183031	4.18731826887593e-06\\
13538.3587534167	4.11151912628219e-06\\
13639.669388052	4.03717210316068e-06\\
13741.738153335	3.96425047161977e-06\\
13844.5707225307	3.89272791354978e-06\\
13948.1728113584	3.82257851690803e-06\\
14052.5501783096	3.75377677196837e-06\\
14157.7086249677	3.68629756753823e-06\\
14263.6539963308	3.62011618714431e-06\\
14370.3921811364	3.55520830519029e-06\\
14477.9291121888	3.49154998308796e-06\\
14586.2707666889	3.42911766536414e-06\\
14695.4231665662	3.36788817574567e-06\\
14805.3923788139	3.30783871322416e-06\\
14916.1845158256	3.24894684810281e-06\\
15027.8057357356	3.19119051802714e-06\\
15140.2622427609	3.13454802400146e-06\\
15253.5602875458	3.0789980263931e-06\\
15367.70616751	3.02451954092631e-06\\
15482.7062271979	2.97109193466717e-06\\
15598.5668586318	2.91869492200199e-06\\
15715.2945016669	2.86730856061016e-06\\
15832.8956443492	2.81691324743359e-06\\
15951.3768232763	2.76748971464435e-06\\
16070.7446239608	2.71901902561164e-06\\
16191.0056811961	2.67148257087034e-06\\
16312.166679425	2.62486206409201e-06\\
16434.234353112	2.5791395380601e-06\\
16557.2154871169	2.53429734065083e-06\\
16681.116917072	2.49031813082092e-06\\
16805.9455297624	2.44718487460368e-06\\
16931.7082635086	2.40488084111474e-06\\
17058.4121085521	2.36338959856848e-06\\
17186.0641074439	2.32269501030686e-06\\
17314.6713554361	2.28278123084119e-06\\
17444.2410008763	2.24363270190853e-06\\
17574.7802456044	2.20523414854366e-06\\
17706.2963453539	2.16757057516742e-06\\
17838.7966101542	2.13062726169309e-06\\
17972.2884047374	2.09438975965107e-06\\
18106.7791489477	2.0588438883334e-06\\
18242.2763181536	2.02397573095881e-06\\
18378.7874436634	1.98977163085914e-06\\
18516.3201131441	1.95621818768824e-06\\
18654.8819710428	1.92330225365389e-06\\
18794.480719012	1.89101092977377e-06\\
18935.1241163369	1.85933156215618e-06\\
19076.8199803677	1.82825173830607e-06\\
19219.5761869535	1.79775928345744e-06\\
19363.4006708799	1.76784225693246e-06\\
19508.3014263109	1.73848894852804e-06\\
19654.286507232	1.70968787493068e-06\\
19801.3640278989	1.68142777615975e-06\\
19949.5421632881	1.65369761204017e-06\\
20098.8291495512	1.62648655870488e-06\\
20249.233284473	1.59978400512735e-06\\
20400.7629279322	1.57357954968512e-06\\
20553.4265023667	1.54786299675424e-06\\
20707.2324932414	1.5226243533355e-06\\
20862.1894495195	1.49785382571253e-06\\
21018.3059841386	1.4735418161423e-06\\
21175.5907744885	1.44967891957829e-06\\
21334.0525628939	1.42625592042673e-06\\
21493.7001571006	1.40326378933603e-06\\
21654.5424307645	1.38069368001994e-06\\
21816.5883239452	1.35853692611442e-06\\
21979.8468436028	1.3367850380686e-06\\
22144.3270640985	1.31542970007009e-06\\
22310.0381276992	1.29446276700454e-06\\
22476.9892450851	1.27387626145003e-06\\
22645.1896958625	1.25366237070607e-06\\
22814.6488290788	1.2338134438575e-06\\
22985.3760637425	1.21432198887348e-06\\
23157.3808893468	1.19518066974139e-06\\
23330.6728663967	1.17638230363597e-06\\
23505.261626941	1.15791985812368e-06\\
23681.1568751072	1.13978644840218e-06\\
23858.3683876408	1.12197533457516e-06\\
24036.9060144492	1.10447991896234e-06\\
24216.7796791487	1.08729374344477e-06\\
24397.9993796163	1.07041048684528e-06\\
24580.5751885457	1.05382396234409e-06\\
24764.5172540065	1.03752811492958e-06\\
24949.8358000088	1.02151701888402e-06\\
25136.5411270713	1.00578487530427e-06\\
25324.6436127937	9.90326009657387e-07\\
25514.153712434	9.75134869370807e-07\\
25705.0819594888	9.60206021457348e-07\\
25897.4389662797	9.45534150174525e-07\\
26091.2354245425	9.31114054718273e-07\\
26286.4821060218	9.16940646950849e-07\\
26483.1898630694	9.03008949162704e-07\\
26681.3696292481	8.89314091868216e-07\\
26881.0324199387	8.75851311635083e-07\\
27082.189332953	8.62615948947101e-07\\
27284.8515491498	8.49603446100281e-07\\
27489.0303330572	8.36809345131912e-07\\
27694.7370334982	8.24229285782479e-07\\
27901.9830842215	8.1185900349017e-07\\
28110.7800045375	7.9969432741767e-07\\
28321.1393999579	7.8773117851114e-07\\
28533.0729628413	7.7596556759097e-07\\
28746.5924730428	7.64393593474166e-07\\
28961.7097985688	7.53011441128065e-07\\
29178.4368962368	7.41815379855086e-07\\
29396.78581234	7.30801761508296e-07\\
29616.7686833165	7.19967018737469e-07\\
29838.3977364244	7.09307663265371e-07\\
30061.6852904209	6.98820284193985e-07\\
30286.6437562476	6.88501546340359e-07\\
30513.2856377198	6.78348188601812e-07\\
30741.6235322216	6.68357022350198e-07\\
30971.6701314066	6.58524929854857e-07\\
31203.4382219025	6.48848862734066e-07\\
31436.9406860226	6.39325840434549e-07\\
31672.1905024814	6.2995294873883e-07\\
31909.2007471158	6.20727338300071e-07\\
32147.9845936127	6.11646223204058e-07\\
32388.5553142404	6.02706879558095e-07\\
32630.9262805866	5.93906644106364e-07\\
32875.1109643017	5.85242912871543e-07\\
33121.1229378475	5.76713139822288e-07\\
33368.9758752518	5.68314835566272e-07\\
33618.6835528682	5.60045566068455e-07\\
33870.2598501416	5.51902951394279e-07\\
34123.7187503806	5.43884664477396e-07\\
34379.0743415334	5.35988429911695e-07\\
34636.340816972	5.28212022767207e-07\\
34895.5324762807	5.20553267429617e-07\\
35156.6637260502	5.13010036463052e-07\\
35419.7490806799	5.05580249495766e-07\\
35684.8031631832	4.98261872128477e-07\\
35951.840706001	4.91052914864945e-07\\
36220.8765518205	4.83951432064522e-07\\
36491.9256543999	4.76955520916337e-07\\
36765.0030794002	4.7006332043475e-07\\
37040.1240052216	4.63273010475838e-07\\
37317.3037238483	4.56582810774489e-07\\
37596.5576416978	4.49990980001861e-07\\
37877.9012804769	4.43495814842862e-07\\
38161.3502780454	4.37095649093309e-07\\
38446.9203892846	4.30788852776492e-07\\
38734.6274869731	4.2457383127881e-07\\
39024.4875626694	4.18449024504142e-07\\
39316.5167276	4.12412906046707e-07\\
39610.7312135559	4.06463982382025e-07\\
39907.1473737941	4.00600792075744e-07\\
40205.7816839466	3.94821905009993e-07\\
40506.6507429366	3.89125921626941e-07\\
40809.7712739007	3.83511472189345e-07\\
41115.1601251185	3.77977216057673e-07\\
41422.8342709494	3.7252184098361e-07\\
41732.8108127753	3.67144062419614e-07\\
42045.1069799522	3.61842622844217e-07\\
42359.7401307671	3.56616291102819e-07\\
42676.7277534027	3.51463861763692e-07\\
42996.0874669105	3.46384154488861e-07\\
43317.8370221886	3.41376013419675e-07\\
43641.9943029698	3.36438306576703e-07\\
43968.5773268145	3.31569925273744e-07\\
44297.6042461126	3.26769783545668e-07\\
44629.0933490932	3.22036817589795e-07\\
44963.0630608394	3.1736998522061e-07\\
45299.531944314	3.12768265337483e-07\\
45638.5187013907	3.08230657405193e-07\\
45980.0421738933	3.03756180946993e-07\\
46324.1213446437	2.99343875049926e-07\\
46670.775338516	2.94992797882216e-07\\
47020.0234235008	2.90702026222426e-07\\
47371.885011775	2.86470655000178e-07\\
47726.3796607813	2.82297796848194e-07\\
48083.5270743154	2.78182581665412e-07\\
48443.347103621	2.74124156190964e-07\\
48805.8597484928	2.70121683588774e-07\\
49171.0851583894	2.66174343042556e-07\\
49539.0436335514	2.62281329361006e-07\\
49909.7556261313	2.58441852592945e-07\\
50283.2417413299	2.54655137652214e-07\\
50659.5227385407	2.50920423952124e-07\\
51038.6195325055	2.4723696504921e-07\\
51420.5531944749	2.43604028296141e-07\\
51805.3449533811	2.40020894503532e-07\\
52193.0161970173	2.36486857610503e-07\\
52583.5884732259	2.33001224363757e-07\\
52977.0834910972	2.29563314005006e-07\\
53373.5231221757	2.26172457966547e-07\\
53772.929401675	2.2282799957481e-07\\
54175.3245297041	2.19529293761675e-07\\
54580.7308724997	2.16275706783423e-07\\
54989.1709636708	2.13066615947085e-07\\
55400.6675054502	2.09901409344066e-07\\
55815.2433699566	2.06779485590851e-07\\
56232.9216004666	2.03700253576622e-07\\
56653.725412694	2.00663132217648e-07\\
57077.6781960819	1.97667550218244e-07\\
57504.8035151016	1.94712945838181e-07\\
57935.1251105625	1.91798766666379e-07\\
58368.6669009326	1.88924469400709e-07\\
58805.4529836665	1.86089519633796e-07\\
59245.5076365459	1.83293391644631e-07\\
59688.8553190289	1.80535568195887e-07\\
60135.5206736089	1.77815540336771e-07\\
60585.528527185	1.75132807211288e-07\\
61038.9038924417	1.72486875871771e-07\\
61495.6719692387	1.69877261097563e-07\\
61955.8581460127	1.67303485218681e-07\\
62419.4880011873	1.64765077944395e-07\\
62886.5873045955	1.62261576196532e-07\\
63357.182018912	1.59792523947427e-07\\
63831.2983010957	1.57357472062388e-07\\
64308.9625038448	1.54955978146543e-07\\
64790.2011770598	1.52587606395984e-07\\
65275.0410693208	1.50251927453059e-07\\
65763.5091293736	1.4794851826574e-07\\
66255.6325076273	1.45676961950922e-07\\
66751.438557664	1.43436847661574e-07\\
67250.9548377589	1.41227770457628e-07\\
67754.209112412	1.39049331180506e-07\\
68261.2293538915	1.36901136331184e-07\\
68772.0437437882	1.34782797951706e-07\\
69286.6806745825	1.32693933510033e-07\\
69805.1687512222	1.30634165788162e-07\\
70327.5367927121	1.28603122773405e-07\\
70853.8138337169	1.26600437552735e-07\\
71384.0291261735	1.24625748210148e-07\\
71918.2121409184	1.22678697726901e-07\\
72456.392569325	1.20758933884602e-07\\
72998.6003249533	1.18866109171028e-07\\
73544.8655452147	1.16999880688609e-07\\
74095.2185930443	1.15159910065517e-07\\
74649.690058591	1.13345863369247e-07\\
75208.3107609164	1.1155741102266e-07\\
75771.111749708	1.09794227722384e-07\\
76338.1243070055	1.08055992359521e-07\\
76909.3799489393	1.06342387942589e-07\\
77484.9104274818	1.04653101522631e-07\\
78064.7477322134	1.02987824120419e-07\\
78648.9240920989	1.01346250655717e-07\\
79237.4719772807	9.97280798785028e-08\\
79830.4241008822	9.81330143021242e-08\\
80427.8134208264	9.6560760138312e-08\\
81029.6731416689	9.50110272339945e-08\\
81636.0367164413	9.34835290098716e-08\\
82246.937848513	9.19779824006749e-08\\
82862.4104934629	9.04941077970785e-08\\
83482.488860967	8.90316289892036e-08\\
84107.2074167008	8.75902731116613e-08\\
84736.6008842538	8.61697705900971e-08\\
85370.7042470603	8.47698550891814e-08\\
86009.5527503438	8.33902634620029e-08\\
86653.1819030753	8.20307357008229e-08\\
87301.627479948	8.06910148891421e-08\\
87954.9255233655	7.93708471550435e-08\\
88613.1123454441	7.80699816257694e-08\\
89276.2245300332	7.67881703834853e-08\\
89944.2989347464	7.55251684222087e-08\\
90617.3726930118	7.42807336058451e-08\\
91295.4832161355	7.3054626627311e-08\\
91978.6681953803	7.18466109687023e-08\\
92666.9656040625	7.06564528624683e-08\\
93360.4136996602	6.94839212535731e-08\\
94059.0510259418	6.83287877625935e-08\\
94762.9164151073	6.71908266497362e-08\\
95472.0489899465	6.60698147797389e-08\\
96186.4881660146	6.49655315876222e-08\\
96906.2736538223	6.38777590452704e-08\\
97631.4454610423	6.280628162881e-08\\
98362.0438947356	6.17508862867545e-08\\
99098.1095635883	6.0711362408902e-08\\
99839.6833801717	5.96875017959442e-08\\
100586.806563215	5.86790986297744e-08\\
101339.520639896	5.76859494444671e-08\\
102097.867448152	5.67078530979017e-08\\
102861.889138999	5.57446107440178e-08\\
103631.628178883	5.47960258056691e-08\\
104407.127352034	5.38619039480633e-08\\
105188.429762846	5.29420530527659e-08\\
105975.578838274	5.20362831922447e-08\\
106768.618330246	5.11444066049411e-08\\
107567.592318097	5.02662376708468e-08\\
108372.545211017	4.94015928875686e-08\\
109183.521750518	4.85502908468663e-08\\
110000.567012926	4.7712152211643e-08\\
110823.726411883	4.68869996933756e-08\\
111653.04570087	4.60746580299701e-08\\
112488.570975754	4.52749539640217e-08\\
113330.348677345	4.44877162214738e-08\\
114178.425593984	4.37127754906531e-08\\
115032.848864136	4.29499644016737e-08\\
115893.665979016	4.21991175061956e-08\\
116760.924785228	4.14600712575215e-08\\
117634.673487419	4.07326639910276e-08\\
118514.960650966	4.00167359049083e-08\\
119401.835204671	3.93121290412288e-08\\
120295.34644348	3.86186872672754e-08\\
121195.544031226	3.7936256257187e-08\\
122102.478003387	3.72646834738668e-08\\
123016.198769868	3.66038181511573e-08\\
123936.757117803	3.59535112762709e-08\\
124864.204214378	3.53136155724718e-08\\
125798.591609674	3.46839854819934e-08\\
126739.971239535	3.40644771491884e-08\\
127688.395428448	3.34549484039028e-08\\
128643.916892463	3.28552587450617e-08\\
129606.588742111	3.22652693244687e-08\\
130576.464485363	3.16848429308007e-08\\
131553.598030603	3.11138439738014e-08\\
132538.043689621	3.05521384686608e-08\\
133529.856180639	2.99995940205761e-08\\
134529.090631344	2.94560798094895e-08\\
135535.802581959	2.89214665749958e-08\\
136550.047988325	2.8395626601414e-08\\
137571.883225014	2.78784337030196e-08\\
138601.365088463	2.73697632094298e-08\\
139638.550800127	2.68694919511396e-08\\
140683.498009666	2.63774982452031e-08\\
141736.264798142	2.58936618810537e-08\\
142796.909681254	2.54178641064642e-08\\
143865.491612584	2.49499876136357e-08\\
144942.069986881	2.44899165254183e-08\\
146026.704643354	2.40375363816555e-08\\
147119.455869007	2.359273412565e-08\\
148220.384401982	2.31553980907495e-08\\
149329.55143494	2.2725417987046e-08\\
150447.018618462	2.23026848881881e-08\\
151572.84806447	2.18870912183035e-08\\
152707.102349689	2.14785307390262e-08\\
153849.844519117	2.10768985366299e-08\\
155001.138089531	2.06820910092626e-08\\
156161.047053023	2.02940058542788e-08\\
157329.635880547	1.9912542055672e-08\\
158506.969525512	1.95375998715996e-08\\
159693.113427386	1.91690808220019e-08\\
160888.133515336	1.88068876763133e-08\\
162092.096211894	1.84509244412606e-08\\
163305.068436644	1.8101096348752e-08\\
164527.117609946	1.775730984385e-08\\
165758.311656683	1.74194725728303e-08\\
166998.719010032	1.70874933713237e-08\\
168248.408615274	1.67612822525399e-08\\
169507.449933624	1.64407503955721e-08\\
170775.912946089	1.61258101337818e-08\\
172053.868157361	1.58163749432612e-08\\
173341.386599734	1.55123594313739e-08\\
174638.539837053	1.52136793253714e-08\\
175945.399968693	1.49202514610857e-08\\
177262.039633563	1.46319937716972e-08\\
178588.532014148	1.4348825276575e-08\\
179924.950840571	1.40706660701926e-08\\
181271.370394698	1.37974373111143e-08\\
182627.865514261	1.35290612110546e-08\\
183994.51159702	1.32654610240092e-08\\
185371.384604954	1.30065610354556e-08\\
186758.561068483	1.27522865516257e-08\\
188156.118090722	1.25025638888468e-08\\
189564.133351766	1.2257320362953e-08\\
190982.685113006	1.20164842787656e-08\\
192411.852221483	1.17799849196411e-08\\
193851.71411427	1.15477525370897e-08\\
195302.350822881	1.13197183404597e-08\\
196763.842977729	1.1095814486691e-08\\
198236.2718126	1.0875974070136e-08\\
199719.719169172	1.06601311124474e-08\\
201214.267501562	1.04482205525338e-08\\
202719.999880911	1.02401782365818e-08\\
204237	1.00359409081449e-08\\
};
\addlegendentry{Adatt. BLN reale (\textit{ML})}


        \nextgroupplot[%
            plotColumn1,plotLegend2,
            xmode=log,ymode=log,
            xmin=3637.5,xmax=204237,
            ymin=0.00531914893617021,ymax=1,
        ] 
% !TEX root = ../../../../Esperimenti/.tex/MEF.tex

% This file was created by matlab2tikz.
%
\definecolor{mycolor1}{rgb}{0.00000,0.44706,0.69804}%
\definecolor{mycolor2}{rgb}{0.00000,0.44700,0.74100}%
\definecolor{mycolor3}{rgb}{0.49400,0.18400,0.55600}%

\addplot[only marks, mark=*, mark size=1.3693pt, color=mycolor1, fill=mycolor1, opacity=0.60, draw=none, mark options={draw=none,line width=0pt}] table[row sep=crcr]{%
x	y\\
4150.33692939717	0.99468085106383\\
4187.02801314734	0.984042553191489\\
4227.99416515043	0.973404255319149\\
4260.55797877454	0.962765957446808\\
4300.72242733043	0.952127659574468\\
4335.42461784776	0.941489361702128\\
4376.00387751075	0.930851063829787\\
4422.91035118834	0.920212765957447\\
4464.7116944447	0.909574468085106\\
4515.35164007382	0.898936170212766\\
4562.49789965917	0.888297872340426\\
4605.23709177358	0.877659574468085\\
4659.71291074476	0.867021276595745\\
4711.70093000889	0.856382978723404\\
4755.0080995876	0.845744680851064\\
4798.19726500855	0.835106382978723\\
4857.23702099794	0.824468085106383\\
4910.39137337255	0.813829787234043\\
4965.82878091041	0.803191489361702\\
5029.0568906941	0.792553191489362\\
5088.42255877955	0.781914893617021\\
5171.03587863455	0.771276595744681\\
5221.89140695905	0.76063829787234\\
5290.11800762439	0.75\\
5369.92682059452	0.73936170212766\\
5438.76894957928	0.728723404255319\\
5516.660516035	0.718085106382979\\
5593.03432546002	0.707446808510638\\
5668.83503101414	0.696808510638298\\
5750.33490761774	0.686170212765957\\
5833.0693059852	0.675531914893617\\
5900.22787063534	0.664893617021277\\
5991.5370166347	0.654255319148936\\
6071.94084668974	0.643617021276596\\
6152.42965115096	0.632978723404255\\
6251.32147576442	0.622340425531915\\
6365.20174111441	0.611702127659574\\
6482.67813301117	0.601063829787234\\
6594.87718985963	0.590425531914894\\
6695.81342093645	0.579787234042553\\
6823.05234956022	0.569148936170213\\
6944.31061950605	0.558510638297872\\
7063.28271349974	0.547872340425532\\
7180.97539058113	0.537234042553192\\
7327.55937965199	0.526595744680851\\
7492.2266170938	0.515957446808511\\
7640.78205299079	0.50531914893617\\
7800.24929157451	0.49468085106383\\
7930.06288587192	0.484042553191489\\
8103.82504159792	0.473404255319149\\
8310.96252663182	0.462765957446808\\
8505.47450486619	0.452127659574468\\
8684.98997422434	0.441489361702128\\
8894.0175522291	0.430851063829787\\
9139.10626698926	0.420212765957447\\
9388.67502484604	0.409574468085106\\
9617.91714837575	0.398936170212766\\
9861.38901527143	0.388297872340426\\
10131.8757616707	0.377659574468085\\
10399.1494067797	0.367021276595745\\
10637.8048752088	0.356382978723404\\
10947.7930459521	0.345744680851064\\
11243.6249013708	0.335106382978723\\
11643.9756479125	0.324468085106383\\
11970.1650568534	0.313829787234043\\
12458.9411105912	0.303191489361702\\
12846.7303484796	0.292553191489362\\
13278.5607159556	0.281914893617021\\
13743.0095674602	0.271276595744681\\
14212.5048575402	0.26063829787234\\
14642.8026364531	0.25\\
15176.9046108407	0.23936170212766\\
15642.3138108233	0.228723404255319\\
16233.7245571048	0.218085106382979\\
16790.8618042594	0.207446808510638\\
17466.542192993	0.196808510638298\\
18092.3497227154	0.186170212765957\\
18752.9103088033	0.175531914893617\\
19433.7432986731	0.164893617021277\\
20272.6425785745	0.154255319148936\\
21341.7762265443	0.143617021276596\\
22337.6168011567	0.132978723404255\\
23414.9137117763	0.122340425531915\\
24737.4256596548	0.111702127659574\\
25792.0193418334	0.101063829787234\\
27119.2945375304	0.0904255319148936\\
28494.9658179533	0.0797872340425532\\
30682.1417334986	0.0691489361702128\\
33295.028254727	0.0585106382978723\\
35836.8405111506	0.0478723404255319\\
38503.3120853778	0.0372340425531915\\
42826.0977100476	0.0265957446808511\\
47805.9263257852	0.0159574468085106\\
60247.2514362665	0.00531914893617021\\
};
\addlegendentry{FRC empirica esatta}

\addplot [color=mycolor1, only marks, every error bar/.append style={opacity=0.30}, mark=*, mark size=0pt, draw=none, forget plot]
plot [error bars/.cd, x dir=both, x explicit, error bar style={line width=1pt, color=mycolor1}, error mark options={mark=none,mark size=0pt}]
table[row sep=crcr, x error plus index=2, x error minus index=3]{%
4150.33692939717	0.99468085106383	169.976563227838	169.976563227838\\
4187.02801314734	0.984042553191489	170.803714607782	170.803714607782\\
4227.99416515043	0.973404255319149	172.090538162676	172.090538162676\\
4260.55797877454	0.962765957446808	173.862364112151	173.862364112151\\
4300.72242733043	0.952127659574468	176.233690858732	176.233690858732\\
4335.42461784776	0.941489361702128	177.903950164512	177.903950164512\\
4376.00387751075	0.930851063829787	178.401001170612	178.401001170612\\
4422.91035118834	0.920212765957447	179.587836328601	179.587836328601\\
4464.7116944447	0.909574468085106	180.057657675379	180.057657675379\\
4515.35164007382	0.898936170212766	181.943575694853	181.943575694853\\
4562.49789965917	0.888297872340426	186.56785173259	186.56785173259\\
4605.23709177358	0.877659574468085	188.248498137471	188.248498137471\\
4659.71291074476	0.867021276595745	189.528464884198	189.528464884198\\
4711.70093000889	0.856382978723404	192.305037581581	192.305037581581\\
4755.0080995876	0.845744680851064	194.853994281051	194.853994281051\\
4798.19726500855	0.835106382978723	195.378296551023	195.378296551023\\
4857.23702099794	0.824468085106383	199.575704996736	199.575704996736\\
4910.39137337255	0.813829787234043	201.44898135535	201.44898135535\\
4965.82878091041	0.803191489361702	204.454853417	204.454853417\\
5029.0568906941	0.792553191489362	207.034785580243	207.034785580243\\
5088.42255877955	0.781914893617021	210.393003991369	210.393003991369\\
5171.03587863455	0.771276595744681	214.48250010172	214.48250010172\\
5221.89140695905	0.76063829787234	216.028636445941	216.028636445941\\
5290.11800762439	0.75	219.51100058042	219.51100058042\\
5369.92682059452	0.73936170212766	227.972949116076	227.972949116076\\
5438.76894957928	0.728723404255319	232.156300290993	232.156300290993\\
5516.660516035	0.718085106382979	235.510191798361	235.510191798361\\
5593.03432546002	0.707446808510638	240.645345316544	240.645345316544\\
5668.83503101414	0.696808510638298	241.719659869157	241.719659869157\\
5750.33490761774	0.686170212765957	246.168376859479	246.168376859479\\
5833.0693059852	0.675531914893617	251.794391826736	251.794391826736\\
5900.22787063534	0.664893617021277	251.918637338232	251.918637338232\\
5991.5370166347	0.654255319148936	256.53710341495	256.53710341495\\
6071.94084668974	0.643617021276596	257.696945720065	257.696945720065\\
6152.42965115096	0.632978723404255	260.782510089085	260.782510089085\\
6251.32147576442	0.622340425531915	269.573294025571	269.573294025571\\
6365.20174111441	0.611702127659574	276.361432806123	276.361432806123\\
6482.67813301117	0.601063829787234	283.990187574019	283.990187574019\\
6594.87718985963	0.590425531914894	287.516358166977	287.516358166977\\
6695.81342093645	0.579787234042553	291.863430559429	291.863430559429\\
6823.05234956022	0.569148936170213	295.358082145184	295.358082145184\\
6944.31061950605	0.558510638297872	302.153138635742	302.153138635742\\
7063.28271349974	0.547872340425532	305.428107406512	305.428107406512\\
7180.97539058113	0.537234042553192	308.036150091488	308.036150091488\\
7327.55937965199	0.526595744680851	310.520431962096	310.520431962096\\
7492.2266170938	0.515957446808511	314.699512899538	314.699512899538\\
7640.78205299079	0.50531914893617	316.91019399724	316.91019399724\\
7800.24929157451	0.49468085106383	322.015446547718	322.015446547718\\
7930.06288587192	0.484042553191489	327.479464536693	327.479464536693\\
8103.82504159792	0.473404255319149	333.305728030455	333.305728030455\\
8310.96252663182	0.462765957446808	339.202053947573	339.202053947573\\
8505.47450486619	0.452127659574468	345.343145549605	345.343145549605\\
8684.98997422434	0.441489361702128	347.662426204301	347.662426204301\\
8894.0175522291	0.430851063829787	353.808810759062	353.808810759062\\
9139.10626698926	0.420212765957447	369.46410500229	369.46410500229\\
9388.67502484604	0.409574468085106	379.658777166131	379.658777166131\\
9617.91714837575	0.398936170212766	385.132581074676	385.132581074676\\
9861.38901527143	0.388297872340426	397.846787609668	397.846787609668\\
10131.8757616707	0.377659574468085	402.971972219338	402.971972219338\\
10399.1494067797	0.367021276595745	418.410578103319	418.410578103319\\
10637.8048752088	0.356382978723404	427.439557118643	427.439557118643\\
10947.7930459521	0.345744680851064	441.885185493965	441.885185493965\\
11243.6249013708	0.335106382978723	458.759404727146	458.759404727146\\
11643.9756479125	0.324468085106383	490.674908895074	490.674908895074\\
11970.1650568534	0.313829787234043	494.465820248422	494.465820248422\\
12458.9411105912	0.303191489361702	516.678759212836	516.678759212836\\
12846.7303484796	0.292553191489362	547.20246860532	547.20246860532\\
13278.5607159556	0.281914893617021	571.952111046299	571.952111046299\\
13743.0095674602	0.271276595744681	601.394168796818	601.394168796818\\
14212.5048575402	0.26063829787234	614.559976254256	614.559976254256\\
14642.8026364531	0.25	633.96250429779	633.96250429779\\
15176.9046108407	0.23936170212766	661.922780740718	661.922780740718\\
15642.3138108233	0.228723404255319	680.383536462998	680.383536462998\\
16233.7245571048	0.218085106382979	700.491293268708	700.491293268708\\
16790.8618042594	0.207446808510638	709.393919590814	709.393919590814\\
17466.542192993	0.196808510638298	751.793211398035	751.793211398035\\
18092.3497227154	0.186170212765957	795.771383628601	795.771383628601\\
18752.9103088033	0.175531914893617	836.96701403669	836.96701403669\\
19433.7432986731	0.164893617021277	860.234431563034	860.234431563034\\
20272.6425785745	0.154255319148936	912.986871352462	912.986871352462\\
21341.7762265443	0.143617021276596	962.469529926972	962.469529926972\\
22337.6168011567	0.132978723404255	999.31345257153	999.31345257153\\
23414.9137117763	0.122340425531915	1110.09972286582	1110.09972286582\\
24737.4256596548	0.111702127659574	1137.47591914502	1137.47591914502\\
25792.0193418334	0.101063829787234	1130.89880255318	1130.89880255318\\
27119.2945375304	0.0904255319148936	1221.13682638719	1221.13682638719\\
28494.9658179533	0.0797872340425532	1312.433492292	1312.433492292\\
30682.1417334986	0.0691489361702128	1506.30650715445	1506.30650715445\\
33295.028254727	0.0585106382978723	1676.83369541404	1676.83369541404\\
35836.8405111506	0.0478723404255319	1915.75212640388	1915.75212640388\\
38503.3120853778	0.0372340425531915	2105.43586935611	2105.43586935611\\
42826.0977100476	0.0265957446808511	2278.41949441374	2278.41949441374\\
47805.9263257852	0.0159574468085106	2531.06086974032	2531.06086974032\\
60247.2514362665	0.00531914893617021	3932.42233731382	3932.42233731382\\
};
\addplot [color=mycolor1]
table[row sep=crcr]{%
4172.07132263442	0.896158559033291\\
4339.30559463463	0.872198409665676\\
4513.24332387914	0.846474617794401\\
4694.15321329881	0.819021696793244\\
4882.31473657492	0.790385027191211\\
5078.01856987647	0.760206455493003\\
5281.56704090323	0.729384985285657\\
5493.27459592846	0.699196127001848\\
5713.46828556252	0.669239802412873\\
5942.48826998815	0.639969098130032\\
6180.68834444751	0.610816443673002\\
6428.4364857931	0.582329531363922\\
6686.11542094667	0.554843616317876\\
6954.12321814448	0.528415386121498\\
7232.87390188203	0.503249558078688\\
7522.79809250848	0.479285658649111\\
7824.34367145866	0.456466119490573\\
8137.97647315046	0.434736137489878\\
8464.18100461634	0.414043541556656\\
8803.46119398071	0.394338665852437\\
9156.34116893949	0.375574229143215\\
9523.36606644442	0.35770521997882\\
9905.10287484291	0.340688787416921\\
10302.1413097745	0.324484137023104\\
10715.0947251767	0.30905243189158\\
11144.6010608082	0.294356698443432\\
11591.3238277527	0.280361736771142\\
12055.9531334255	0.26703403530937\\
12539.2067476672	0.254341689622597\\
13041.8312115707	0.242254325110465\\
13564.6029907548	0.230743023441266\\
14108.3296748659	0.219780252533232\\
14673.8512251603	0.209339799912063\\
15262.0412720959	0.199396709281409\\
15873.8084649359	0.18992722015096\\
16510.0978754506	0.180908710374341\\
17171.8924578859	0.17231964145614\\
17860.2145674531	0.164139506494266\\
18576.1275396862	0.156348780630266\\
19320.7373331067	0.148928873886453\\
20095.1942377335	0.141862086274523\\
20900.6946520764	0.135131565065953\\
21738.4829313601	0.128721264119784\\
22609.8533098318	0.122615905168437\\
23516.1519001238	0.116800940967028\\
24458.7787727585	0.111262520216235\\
25439.1901190089	0.105987454173102\\
26458.9005004558	0.100963184868329\\
27519.485188717	0.0961777548525354\\
28622.5825989621	0.091619778397722\\
29769.8968209732	0.087278414083748\\
30963.2002516618	0.0831433387030116\\
32204.3363331067	0.0792047224197722\\
33495.222400346	0.0754532051236182\\
34837.8526433185	0.0718798739195088\\
36234.301187534	0.0684762416996074\\
37686.725298229	0.0652342267447641\\
39197.368712959	0.0621461333060351\\
40768.5651077743	0.0592046331190147\\
42402.7417023361	0.0564027478060447\\
44102.42300954	0.0537338321235323\\
45870.2347354405	0.0511915580136764\\
47708.9078355006	0.0487698994218671\\
49621.2827334344	0.0464631178428916\\
51610.3137091567	0.0442657485608658\\
53679.0734626221	0.0421725875494963\\
55830.757860601	0.0401786790008962\\
58068.6908737265	0.0382793034527088\\
60396.3297114396	0.0364699664847535\\
62817.2701627643	0.0347463879577964\\
65335.2521511636	0.0331044917683707\\
67954.1655120575	0.0315403960948304\\
70678.0560019295	0.0300504041110144\\
73511.1315483009	0.0286309951450424\\
76457.7687502338	0.0272788162618426\\
79522.5196393993	0.0259906742490465\\
82710.1187121592	0.0247635279868652\\
86025.4902435224	0.0235944811834994\\
89473.7558942766	0.0224807754585183\\
93060.2426230453	0.0214197837574957\\
96790.4909154932	0.020409004081992\\
100670.263343395	0.0194460535197378\\
104705.553466784	0.0185286625606066\\
108902.595092942	0.0176546696846552\\
113267.871906523	0.0168220162091727\\
117808.127485696	0.0160287413823068\\
122530.37571978	0.0152729777114349\\
127441.911644451	0.0145529465150174\\
132550.322711287	0.0138669536872106\\
137863.500509028	0.0132133856650353\\
143389.652954685	0.0125907055883854\\
149137.316973309	0.0119974496436297\\
155115.371686033	0.0114322235820043\\
161333.052126735	0.0108936994044166\\
167799.963508525	0.0103806122046828\\
174526.096062101	0.0098917571636085\\
181521.840468875	0.00942598668668178\\
188798.003912741	0.0089822076784989\\
196365.826775247	0.00855937894737315\\
204237	0.00815650873388985\\
};
\addlegendentry{Adatt. Pareto esatto (\textit{ML})}


\addplot[area legend, draw=none, fill=mycolor1, fill opacity=0.15, forget plot]
table[row sep=crcr] {%
x	y\\
4172.07132263442	0.920934456622767\\
4339.30559463463	0.899077692913512\\
4513.24332387914	0.875203247009936\\
4694.15321329881	0.849233895163214\\
4882.31473657492	0.821724065586837\\
5078.01856987647	0.792095117638882\\
5281.56704090323	0.7614185650501\\
5493.27459592846	0.731271102617904\\
5713.46828556252	0.701058209426262\\
5942.48826998815	0.671368693496505\\
6180.68834444751	0.641330348352662\\
6428.4364857931	0.611691533252499\\
6686.11542094667	0.582935750270436\\
6954.12321814448	0.555154266913134\\
7232.87390188203	0.528704070836868\\
7522.79809250848	0.503521016070272\\
7824.34367145866	0.479544063096854\\
8137.97647315046	0.456715127777053\\
8464.18100461634	0.434978937642445\\
8803.46119398071	0.414282895201216\\
9156.34116893949	0.394576947911823\\
9523.36606644442	0.375813464498568\\
9905.10287484291	0.35794711729886\\
10302.1413097745	0.340934770347159\\
10715.0947251767	0.324735372915075\\
11144.6010608082	0.309309858240871\\
11591.3238277527	0.294621047194693\\
12055.9531334255	0.280633556638311\\
12539.2067476672	0.267313712249951\\
13041.8312115707	0.254629465596093\\
13564.6029907548	0.242550315242739\\
14108.3296748659	0.231047231708885\\
14673.8512251603	0.220092586074537\\
15262.0412720959	0.209660082064849\\
15873.8084649359	0.199724691440654\\
16510.0978754506	0.190262592534009\\
17171.8924578859	0.181251111775225\\
17860.2145674531	0.172668668065411\\
18576.1275396862	0.164494719855663\\
19320.7373331067	0.156709714800848\\
20095.1942377335	0.149295041862365\\
20900.6946520764	0.142232985740431\\
21738.4829313601	0.135506683522248\\
22609.8533098318	0.129100083437995\\
23516.1519001238	0.122997905621829\\
24458.7787727585	0.117185604780146\\
25439.1901190089	0.111649334674086\\
26458.9005004558	0.106375914327831\\
27519.485188717	0.101352795878562\\
28622.5825989621	0.0965680339880544\\
29769.8968209732	0.0920102567397713\\
30963.2002516618	0.0876686379490737\\
32204.3363331067	0.0835328708176614\\
33495.222400346	0.0795931428667444\\
34837.8526433185	0.0758401120866344\\
36234.301187534	0.0722648842434872\\
37686.725298229	0.0688589912868198\\
39197.368712959	0.0656143708041817\\
40768.5651077743	0.062523346471972\\
42402.7417023361	0.0595786094538879\\
44102.42300954	0.0567732007008544\\
45870.2347354405	0.0541004941085423\\
47708.9078355006	0.0515541804907205\\
49621.2827334344	0.049128252328727\\
51610.3137091567	0.0468169892592861\\
53679.0734626221	0.0446149442647348\\
55830.757860601	0.0425169305314858\\
58068.6908737265	0.0405180089442127\\
60396.3297114396	0.0386134761848395\\
62817.2701627643	0.0367988534069164\\
65335.2521511636	0.0350698754574087\\
67954.1655120575	0.0334224806192847\\
70678.0560019295	0.0318528008495898\\
73511.1315483009	0.0303571524889316\\
76457.7687502338	0.0289320274194696\\
79522.5196393993	0.0275740846496306\\
82710.1187121592	0.0262801423048224\\
86025.4902435224	0.0250471700044399\\
89473.7558942766	0.0238722816064132\\
93060.2426230453	0.0227527283014627\\
96790.4909154932	0.0216858920400995\\
100670.263343395	0.0206692792762319\\
104705.553466784	0.0197005150120299\\
108902.595092942	0.0187773371294455\\
113267.871906523	0.0178975909945005\\
117808.127485696	0.0170592243211268\\
122530.37571978	0.0162602822819936\\
127441.911644451	0.0154989028543631\\
132550.322711287	0.0147733123896014\\
137863.500509028	0.0140818213955254\\
143389.652954685	0.0134228205212914\\
149137.316973309	0.0127947767350345\\
155115.371686033	0.0121962296849425\\
161333.052126735	0.0116257882349022\\
167799.963508525	0.0110821271662874\\
174526.096062101	0.0105639840378662\\
181521.840468875	0.0100701561961977\\
188798.003912741	0.00959949792925632\\
196365.826775247	0.00915091775637693\\
204237	0.00872337584794784\\
204237	0.00758964161983187\\
196365.826775247	0.00796784013836936\\
188798.003912741	0.00836491742774148\\
181521.840468875	0.00878181717716584\\
174526.096062101	0.00921953028935079\\
167799.963508525	0.00967909724307814\\
161333.052126735	0.0101616105739309\\
155115.371686033	0.0106682174790662\\
149137.316973309	0.0112001225522248\\
143389.652954685	0.0117585906554794\\
137863.500509028	0.0123449499345453\\
132550.322711287	0.0129605949848198\\
127441.911644451	0.0136069901756717\\
122530.37571978	0.0142856731408762\\
117808.127485696	0.0149982584434868\\
113267.871906523	0.015746441423845\\
108902.595092942	0.0165320022398649\\
104705.553466784	0.0173568101091834\\
100670.263343395	0.0182228277632437\\
96790.4909154932	0.0191321161238845\\
93060.2426230453	0.0200868392135287\\
89473.7558942766	0.0210892693106234\\
86025.4902435224	0.0221417923625589\\
82710.1187121592	0.0232469136689081\\
79522.5196393993	0.0244072638484623\\
76457.7687502338	0.0256256051042156\\
73511.1315483009	0.0269048378011533\\
70678.0560019295	0.0282480073724389\\
67954.1655120575	0.0296583115703761\\
65335.2521511636	0.0311391080793327\\
62817.2701627643	0.0326939225086764\\
60396.3297114396	0.0343264567846675\\
58068.6908737265	0.0360405979612049\\
55830.757860601	0.0378404274703066\\
53679.0734626221	0.0397302308342578\\
51610.3137091567	0.0417145078624455\\
49621.2827334344	0.0437979833570561\\
47708.9078355006	0.0459856183530137\\
45870.2347354405	0.0482826219188106\\
44102.42300954	0.0506944635462103\\
42402.7417023361	0.0532268861582015\\
40768.5651077743	0.0558859197660573\\
39197.368712959	0.0586778958078885\\
37686.725298229	0.0616094622027084\\
36234.301187534	0.0646875991557275\\
34837.8526433185	0.0679196357523832\\
33495.222400346	0.071313267380492\\
32204.3363331067	0.0748765740218831\\
30963.2002516618	0.0786180394569494\\
29769.8968209732	0.0825465714277247\\
28622.5825989621	0.0866715228073895\\
27519.485188717	0.0910027138265083\\
26458.9005004558	0.0955504554088278\\
25439.1901190089	0.100325573672118\\
24458.7787727585	0.105339435652324\\
23516.1519001238	0.110603976312228\\
22609.8533098318	0.116131726898879\\
21738.4829313601	0.12193584471732\\
20900.6946520764	0.128030144391476\\
20095.1942377335	0.134429130686682\\
19320.7373331067	0.141148032972059\\
18576.1275396862	0.148202841404868\\
17860.2145674531	0.15561034492312\\
17171.8924578859	0.163388171137055\\
16510.0978754506	0.171554828214673\\
15873.8084649359	0.180129748861267\\
15262.0412720959	0.189133336497969\\
14673.8512251603	0.198587013749589\\
14108.3296748659	0.20851327335758\\
13564.6029907548	0.218935731639794\\
13041.8312115707	0.229879184624838\\
12539.2067476672	0.241369666995242\\
12055.9531334255	0.253434513980429\\
11591.3238277527	0.266102426347592\\
11144.6010608082	0.279403538645993\\
10715.0947251767	0.293369490868085\\
10302.1413097745	0.308033503699049\\
9905.10287484291	0.323430457534981\\
9523.36606644442	0.339596975459072\\
9156.34116893949	0.356571510374607\\
8803.46119398071	0.374394436503658\\
8464.18100461634	0.393108145470866\\
8137.97647315046	0.412757147202704\\
7824.34367145866	0.433388175884293\\
7522.79809250848	0.45505030122795\\
7232.87390188203	0.477795045320509\\
6954.12321814448	0.501676505329863\\
6686.11542094667	0.526751482365316\\
6428.4364857931	0.552967529475345\\
6180.68834444751	0.580302538993342\\
5942.48826998815	0.608569502763559\\
5713.46828556252	0.637421395399485\\
5493.27459592846	0.667121151385792\\
5281.56704090323	0.697351405521215\\
5078.01856987647	0.728317793347124\\
4882.31473657492	0.759045988795585\\
4694.15321329881	0.788809498423274\\
4513.24332387914	0.817745988578866\\
4339.30559463463	0.845319126417839\\
4172.07132263442	0.871382661443816\\
}--cycle;
\addplot[only marks, mark=*, mark size=1.3693pt, color=black, fill=black, opacity=0.60, draw=none, mark options={draw=none,line width=0pt}] table[row sep=crcr]{%
x	y\\
3650	0.99468085106383\\
3735	0.984042553191489\\
3741	0.973404255319149\\
3761	0.962765957446808\\
3772	0.952127659574468\\
3785	0.941489361702128\\
3825	0.930851063829787\\
3847	0.920212765957447\\
3900	0.909574468085106\\
3928	0.898936170212766\\
3945	0.888297872340426\\
3996	0.877659574468085\\
4009	0.867021276595745\\
4024	0.856382978723404\\
4061	0.845744680851064\\
4061	0.835106382978723\\
4071	0.824468085106383\\
4113	0.813829787234043\\
4228	0.803191489361702\\
4348	0.792553191489362\\
4430	0.781914893617021\\
4503	0.771276595744681\\
4539	0.76063829787234\\
4632	0.75\\
4654	0.73936170212766\\
4663	0.728723404255319\\
4679	0.718085106382979\\
4702	0.707446808510638\\
4779	0.696808510638298\\
4861	0.686170212765957\\
4953	0.675531914893617\\
5029	0.664893617021277\\
5048	0.654255319148936\\
5236	0.643617021276596\\
5264	0.632978723404255\\
5327	0.622340425531915\\
5379	0.611702127659574\\
5458	0.601063829787234\\
5463	0.590425531914894\\
5524	0.579787234042553\\
5607	0.569148936170213\\
5857	0.558510638297872\\
5868	0.547872340425532\\
5982	0.537234042553192\\
6317	0.526595744680851\\
6332	0.515957446808511\\
6342	0.50531914893617\\
6356	0.49468085106383\\
6367	0.484042553191489\\
6468	0.473404255319149\\
6629	0.462765957446808\\
6926	0.452127659574468\\
7252	0.441489361702128\\
7294	0.430851063829787\\
7320	0.420212765957447\\
7348	0.409574468085106\\
7596	0.398936170212766\\
7724	0.388297872340426\\
7877	0.377659574468085\\
8035	0.367021276595745\\
8499	0.356382978723404\\
8518	0.345744680851064\\
8994	0.335106382978723\\
9128	0.324468085106383\\
9267	0.313829787234043\\
9435	0.303191489361702\\
9599	0.292553191489362\\
9837	0.281914893617021\\
10119	0.271276595744681\\
10336	0.26063829787234\\
10377	0.25\\
11048	0.23936170212766\\
11424	0.228723404255319\\
11830	0.218085106382979\\
12182	0.207446808510638\\
12313	0.196808510638298\\
13086	0.186170212765957\\
13380	0.175531914893617\\
13398	0.164893617021277\\
13899	0.154255319148936\\
14984	0.143617021276596\\
16428	0.132978723404255\\
20491	0.122340425531915\\
21264	0.111702127659574\\
23237	0.101063829787234\\
30134	0.0904255319148936\\
30990	0.0797872340425532\\
32887	0.0691489361702128\\
37527	0.0585106382978723\\
39026	0.0478723404255319\\
41095	0.0372340425531915\\
61636	0.0265957446808511\\
122339	0.0159574468085106\\
204237	0.00531914893617021\\
};
\addlegendentry{FRC empirica reale}

\addplot [color=black]
table[row sep=crcr]{%
3637.5	1\\
3788.54993935839	0.949537816810858\\
3945.87234172165	0.901622065553929\\
4109.72767586131	0.856124247714574\\
4280.38722671172	0.812922349093734\\
4458.13354451935	0.771900512595218\\
4643.26091264341	0.732948727524845\\
4836.07583478224	0.695962534568237\\
5036.89754243213	0.660842745656075\\
5246.05852341874	0.627495177965562\\
5463.90507237627	0.59583040134476\\
5690.79786408554	0.565763498482441\\
5927.11255062052	0.53721383718029\\
6173.24038329176	0.510104854116755\\
6429.58886041643	0.484363849522645\\
6696.58240198765	0.459921792217835\\
6974.66305235981	0.436713134486259\\
7264.29121211354	0.414675636292709\\
7565.94640031188	0.393750198370032\\
7880.12804840974	0.373880703729123\\
8207.35632713103	0.355013867166658\\
8548.17300768248	0.337099092367008\\
8903.14235873003	0.32008833621509\\
9272.85208062291	0.303935979956297\\
9657.9142784119	0.28859870685797\\
10058.9664752731	0.274035386044354\\
10476.6726680149	0.260206962193476\\
10911.7244264153	0.247076350800179\\
11364.8420382106	0.234608338724395\\
11836.7757016304	0.222769489757984\\
12328.3067674531	0.211528054956865\\
12840.2490326395	0.200853887497989\\
13373.4500876847	0.190718361832814\\
13928.7927199204	0.181094296920473\\
14507.19637509	0.171955883334763\\
15109.6186796172	0.163278614049474\\
15737.0570260872	0.15503921871644\\
16390.550224567	0.147215601260069\\
17071.1802224973	0.139786780620984\\
17780.0738960051	0.132732834489867\\
18518.4049156008	0.126034845880625\\
19287.3956893508	0.119674852399582\\
20088.3193867412	0.113635798074661\\
20922.5020465844	0.107901487615373\\
21791.3247724572	0.102456542980945\\
22696.2260193077	0.0972863621401141\\
23638.7039750138	0.0923770799119943\\
24620.3190408384	0.0877155307829971\\
25642.6964148873	0.0832892136000928\\
26707.5287828473	0.0790862580457252\\
27816.5791204588	0.0750953928044779\\
28971.6836123634	0.0713059153361178\\
30174.7546921595	0.0677076631739571\\
31427.7842086969	0.0642909866715642\\
32732.8467238564	0.061046723124733\\
34092.1029472698	0.0579661721993158\\
35507.8033136712	0.0550410725990206\\
36982.2917087997	0.0522635799106019\\
38518.0093500227	0.0496262455670326\\
40117.4988281057	0.0471219968722397\\
41783.4083168193	0.0447441180338345\\
43518.4959573534	0.0424862321529745\\
45325.6344247971	0.0403422841230547\\
47207.8156842464	0.0383065243913687\\
49168.1559444106	0.0363734935401921\\
51209.900816924	0.0345380076459378\\
53336.4306898988	0.0327951443771205\\
55551.2663246212	0.0311402297938478\\
57858.074684653	0.0295688258134387\\
60260.6750069927	0.0280767183085531\\
62763.0451253437	0.026659905805917\\
65369.3280559641	0.0253145887553335\\
68083.8388569957	0.024037159340204\\
70911.0717726344	0.022824191802232\\
73855.7076739664	0.0216724332543637\\
76922.6218087908	0.0205787949573275\\
80116.89187326	0.0195403440363791\\
83443.8064187003	0.0185542956160364\\
86908.8736075329	0.0176180053517145\\
90517.8303327903	0.016728962338229\\
94276.651716329	0.015884782376153\\
98191.5610014598	0.0150832015779679\\
102269.039856381	0.0143220702968617\\
106515.839105467	0.0135993473618937\\
110938.989906178	0.012913094604065\\
115545.815390112	0.012261471658616\\
120343.942787444	0.0116427310296104\\
125341.316054851	0.0110552134035723\\
130546.209027823	0.0104973431996062\\
135967.239119128	0.00996762434406833\\
141613.381586117	0.00946463625845741\\
147493.984390494	0.00898703004976452\\
153618.783675143	0.00853352489306699\\
159997.919883649	0.00810290459666392\\
166641.954549187	0.00769401434054292\\
173561.887780587	0.00730575757943056\\
180769.176474522	0.00693709310212186\\
188275.753283964	0.00658703223920245\\
196094.046374327	0.00625463621167503\\
204237	0.00593901361338004\\
};
\addlegendentry{Adatt. Pareto reale (\textit{ML})}

        \nextgroupplot[%
            plotColumn2,plotLegend2,
            xmode=log,ymode=log,
            xmin=3637.5,xmax=204237,
            ymin=0.00531914893617021,ymax=1,
        ] 
% !TEX root = ../../../../Esperimenti/.tex/MEF.tex

% This file was created by matlab2tikz.
%
\definecolor{mycolor1}{rgb}{0.00000,0.44706,0.69804}%
\definecolor{mycolor2}{rgb}{0.00000,0.44700,0.74100}%
\definecolor{mycolor3}{rgb}{0.49400,0.18400,0.55600}%

\addplot[only marks, mark=*, mark size=1.3693pt, color=mycolor1, fill=mycolor1, opacity=0.60, draw=none, mark options={draw=none,line width=0pt}] table[row sep=crcr]{%
x	y\\
3992.21901287406	0.99468085106383\\
4035.70712062431	0.984042553191489\\
4071.79187007323	0.973404255319149\\
4109.99987537909	0.962765957446808\\
4148.37378839266	0.952127659574468\\
4191.11022078168	0.941489361702128\\
4233.67010131831	0.930851063829787\\
4276.18786631414	0.920212765957447\\
4322.69703062875	0.909574468085106\\
4373.86784906093	0.898936170212766\\
4429.42241797401	0.888297872340426\\
4488.27036791876	0.877659574468085\\
4531.08400121108	0.867021276595745\\
4583.6696277312	0.856382978723404\\
4636.661594009	0.845744680851064\\
4682.64785922942	0.835106382978723\\
4723.40219917704	0.824468085106383\\
4781.45828106834	0.813829787234043\\
4837.6139874997	0.803191489361702\\
4890.9460296608	0.792553191489362\\
4964.6915877558	0.781914893617021\\
5027.23664881734	0.771276595744681\\
5090.46224303436	0.76063829787234\\
5145.17361009948	0.75\\
5230.73315983136	0.73936170212766\\
5309.23585926379	0.728723404255319\\
5389.2464473355	0.718085106382979\\
5451.15340178209	0.707446808510638\\
5513.75038061141	0.696808510638298\\
5580.49396960848	0.686170212765957\\
5665.68461156702	0.675531914893617\\
5747.44769911652	0.664893617021277\\
5845.42076390995	0.654255319148936\\
5940.37542079474	0.643617021276596\\
6033.71769568608	0.632978723404255\\
6117.06475145149	0.622340425531915\\
6230.16326627369	0.611702127659574\\
6327.14225349641	0.601063829787234\\
6448.56672058875	0.590425531914894\\
6547.77801503347	0.579787234042553\\
6668.3988545149	0.569148936170213\\
6782.43324968638	0.558510638297872\\
6907.45969695854	0.547872340425532\\
7051.00328256838	0.537234042553192\\
7200.9720083471	0.526595744680851\\
7344.76484244162	0.515957446808511\\
7484.13719823487	0.50531914893617\\
7629.65460827699	0.49468085106383\\
7773.36176427919	0.484042553191489\\
7929.64873614779	0.473404255319149\\
8105.11761585431	0.462765957446808\\
8298.15579839243	0.452127659574468\\
8521.10134053963	0.441489361702128\\
8723.13555707264	0.430851063829787\\
8939.90353328617	0.420212765957447\\
9164.01645919119	0.409574468085106\\
9402.11959628572	0.398936170212766\\
9707.47429499043	0.388297872340426\\
9989.07495897613	0.377659574468085\\
10274.0247025048	0.367021276595745\\
10560.6118642168	0.356382978723404\\
10892.4037361983	0.345744680851064\\
11247.0199356664	0.335106382978723\\
11588.2272573391	0.324468085106383\\
12017.3114524898	0.313829787234043\\
12370.6880062903	0.303191489361702\\
12854.0502391796	0.292553191489362\\
13283.105353645	0.281914893617021\\
13738.8570598731	0.271276595744681\\
14301.2271115371	0.26063829787234\\
14801.9384204441	0.25\\
15414.9477386648	0.23936170212766\\
15957.81166874	0.228723404255319\\
16540.500092895	0.218085106382979\\
17102.2823224419	0.207446808510638\\
17776.5549347508	0.196808510638298\\
18415.9281927802	0.186170212765957\\
19140.8385790227	0.175531914893617\\
19888.1839358292	0.164893617021277\\
20731.1123371328	0.154255319148936\\
21634.2649092012	0.143617021276596\\
22644.8761215826	0.132978723404255\\
23612.1010443776	0.122340425531915\\
24770.4490417905	0.111702127659574\\
25668.8046063559	0.101063829787234\\
26704.5437787447	0.0904255319148936\\
27974.1365140605	0.0797872340425532\\
29306.6120797963	0.0691489361702128\\
31035.873228833	0.0585106382978723\\
32973.6149110452	0.0478723404255319\\
35261.7054855083	0.0372340425531915\\
38059.9769107571	0.0265957446808511\\
41156.020851293	0.0159574468085106\\
46060.6159360783	0.00531914893617021\\
};
\addlegendentry{FRC empirica esatta}

\addplot [color=mycolor1, only marks, every error bar/.append style={opacity=0.30}, mark=*, mark size=0pt, draw=none, forget plot]
plot [error bars/.cd, x dir=both, x explicit, error bar style={line width=1pt, color=mycolor1}, error mark options={mark=none,mark size=0pt}]
table[row sep=crcr, x error plus index=2, x error minus index=3]{%
3992.21901287406	0.99468085106383	24.0929963046435	24.0929963046435\\
4035.70712062431	0.984042553191489	24.7205781163324	24.7205781163324\\
4071.79187007323	0.973404255319149	26.9774693367253	26.9774693367253\\
4109.99987537909	0.962765957446808	26.4486511911181	26.4486511911181\\
4148.37378839266	0.952127659574468	27.8647942308759	27.8647942308759\\
4191.11022078168	0.941489361702128	27.2912716631838	27.2912716631838\\
4233.67010131831	0.930851063829787	28.1370913416694	28.1370913416694\\
4276.18786631414	0.920212765957447	28.5529755718965	28.5529755718965\\
4322.69703062875	0.909574468085106	28.8991874779666	28.8991874779666\\
4373.86784906093	0.898936170212766	30.6756589305794	30.6756589305794\\
4429.42241797401	0.888297872340426	32.1028322424308	32.1028322424308\\
4488.27036791876	0.877659574468085	33.3343773486832	33.3343773486832\\
4531.08400121108	0.867021276595745	32.7177708313179	32.7177708313179\\
4583.6696277312	0.856382978723404	33.1145413012851	33.1145413012851\\
4636.661594009	0.845744680851064	34.0549702487333	34.0549702487333\\
4682.64785922942	0.835106382978723	35.5219194435424	35.5219194435424\\
4723.40219917704	0.824468085106383	35.9837461124071	35.9837461124071\\
4781.45828106834	0.813829787234043	34.724504365569	34.724504365569\\
4837.6139874997	0.803191489361702	37.3491122474413	37.3491122474413\\
4890.9460296608	0.792553191489362	36.2186799502016	36.2186799502016\\
4964.6915877558	0.781914893617021	39.4705490753678	39.4705490753678\\
5027.23664881734	0.771276595744681	40.2663812228522	40.2663812228522\\
5090.46224303436	0.76063829787234	41.8311452529706	41.8311452529706\\
5145.17361009948	0.75	42.7817528955582	42.7817528955582\\
5230.73315983136	0.73936170212766	45.9488626165249	45.9488626165249\\
5309.23585926379	0.728723404255319	46.9105132622091	46.9105132622091\\
5389.2464473355	0.718085106382979	47.835238887165	47.835238887165\\
5451.15340178209	0.707446808510638	48.479481424285	48.479481424285\\
5513.75038061141	0.696808510638298	50.9510175144446	50.9510175144446\\
5580.49396960848	0.686170212765957	51.6133782369635	51.6133782369635\\
5665.68461156702	0.675531914893617	51.1150426783279	51.1150426783279\\
5747.44769911652	0.664893617021277	52.0477633700786	52.0477633700786\\
5845.42076390995	0.654255319148936	53.0872208994149	53.0872208994149\\
5940.37542079474	0.643617021276596	52.5638825929713	52.5638825929713\\
6033.71769568608	0.632978723404255	54.1947369216084	54.1947369216084\\
6117.06475145149	0.622340425531915	53.9015791110814	53.9015791110814\\
6230.16326627369	0.611702127659574	55.2799783153436	55.2799783153436\\
6327.14225349641	0.601063829787234	57.7825764842437	57.7825764842437\\
6448.56672058875	0.590425531914894	64.3031647130056	64.3031647130056\\
6547.77801503347	0.579787234042553	65.0813656585476	65.0813656585476\\
6668.3988545149	0.569148936170213	67.133473648557	67.133473648557\\
6782.43324968638	0.558510638297872	68.1878977435861	68.1878977435861\\
6907.45969695854	0.547872340425532	71.1662719678658	71.1662719678658\\
7051.00328256838	0.537234042553192	72.737219438316	72.737219438316\\
7200.9720083471	0.526595744680851	73.6586593752944	73.6586593752944\\
7344.76484244162	0.515957446808511	79.5414190059575	79.5414190059575\\
7484.13719823487	0.50531914893617	83.4252137927848	83.4252137927848\\
7629.65460827699	0.49468085106383	83.2105013690511	83.2105013690511\\
7773.36176427919	0.484042553191489	83.5251133267155	83.5251133267155\\
7929.64873614779	0.473404255319149	89.49761607267	89.49761607267\\
8105.11761585431	0.462765957446808	93.1631776006354	93.1631776006354\\
8298.15579839243	0.452127659574468	102.88839010754	102.88839010754\\
8521.10134053963	0.441489361702128	102.461744649762	102.461744649762\\
8723.13555707264	0.430851063829787	110.046474494388	110.046474494388\\
8939.90353328617	0.420212765957447	111.070407369721	111.070407369721\\
9164.01645919119	0.409574468085106	116.10347242335	116.10347242335\\
9402.11959628572	0.398936170212766	119.649835504628	119.649835504628\\
9707.47429499043	0.388297872340426	121.605415296429	121.605415296429\\
9989.07495897613	0.377659574468085	125.767136680712	125.767136680712\\
10274.0247025048	0.367021276595745	132.398640697653	132.398640697653\\
10560.6118642168	0.356382978723404	133.30515703914	133.30515703914\\
10892.4037361983	0.345744680851064	138.57168694667	138.57168694667\\
11247.0199356664	0.335106382978723	145.150897489162	145.150897489162\\
11588.2272573391	0.324468085106383	146.895755031666	146.895755031666\\
12017.3114524898	0.313829787234043	141.235613894037	141.235613894037\\
12370.6880062903	0.303191489361702	136.681220515636	136.681220515636\\
12854.0502391796	0.292553191489362	142.95108886342	142.95108886342\\
13283.105353645	0.281914893617021	156.529646294123	156.529646294123\\
13738.8570598731	0.271276595744681	168.497850336368	168.497850336368\\
14301.2271115371	0.26063829787234	180.337992503896	180.337992503896\\
14801.9384204441	0.25	187.58238704516	187.58238704516\\
15414.9477386648	0.23936170212766	190.520933630729	190.520933630729\\
15957.81166874	0.228723404255319	207.896403333284	207.896403333284\\
16540.500092895	0.218085106382979	219.720800190695	219.720800190695\\
17102.2823224419	0.207446808510638	223.293317163294	223.293317163294\\
17776.5549347508	0.196808510638298	236.649062658165	236.649062658165\\
18415.9281927802	0.186170212765957	242.26189107995	242.26189107995\\
19140.8385790227	0.175531914893617	234.818445220524	234.818445220524\\
19888.1839358292	0.164893617021277	255.421754471547	255.421754471547\\
20731.1123371328	0.154255319148936	249.893248268887	249.893248268887\\
21634.2649092012	0.143617021276596	257.844869876992	257.844869876992\\
22644.8761215826	0.132978723404255	269.912249617495	269.912249617495\\
23612.1010443776	0.122340425531915	268.356513256938	268.356513256938\\
24770.4490417905	0.111702127659574	278.870851616435	278.870851616435\\
25668.8046063559	0.101063829787234	291.400560450634	291.400560450634\\
26704.5437787447	0.0904255319148936	356.056304764243	356.056304764243\\
27974.1365140605	0.0797872340425532	380.526676401661	380.526676401661\\
29306.6120797963	0.0691489361702128	391.248314963009	391.248314963009\\
31035.873228833	0.0585106382978723	407.047404181406	407.047404181406\\
32973.6149110452	0.0478723404255319	427.970440862631	427.970440862631\\
35261.7054855083	0.0372340425531915	421.930290430503	421.930290430503\\
38059.9769107571	0.0265957446808511	532.738630095942	532.738630095942\\
41156.020851293	0.0159574468085106	695.565885713892	695.565885713892\\
46060.6159360783	0.00531914893617021	921.332606778658	921.332606778658\\
};
\addplot [color=mycolor1]
table[row sep=crcr]{%
3985.7162970862	0.985737496181037\\
4147.39454893027	0.950502350306669\\
4315.63118455555	0.907170644741589\\
4490.69224096661	0.864331689836275\\
4672.85454680357	0.823518416699039\\
4862.40616009855	0.78463491050967\\
5059.64682378936	0.747589804043749\\
5264.88843971087	0.712296061769516\\
5478.45556181355	0.678670774208701\\
5700.68590938921	0.646634962072767\\
5931.93090111527	0.616113389709222\\
6172.55621076243	0.587034387414894\\
6422.94234544414	0.559329682194135\\
6683.48524732248	0.532934236560071\\
6954.59691972191	0.507786094996148\\
7236.70607864091	0.483826237713451\\
7530.25883069187	0.460998441356685\\
7835.7193785411	0.439249146328174\\
8153.5707549648	0.418527330415065\\
8484.31558668133	0.398784388419872\\
8828.47688916811	0.379974017508772\\
9186.5988937197	0.362052108005714\\
9559.24790805498	0.344976639373316\\
9947.01321183443	0.328707581133866\\
10350.5079885035	0.313206798495507\\
10770.3702949358	0.298437962459849\\
11207.2640704089	0.284366464197945\\
11661.880186509	0.270959333491646\\
12134.9375396247	0.258185161047096\\
12627.1841877562	0.246014024496263\\
13139.398533439	0.234417417911179\\
13672.3905546528	0.223368184663953\\
14227.0030856605	0.212840453473493\\
14804.1131498041	0.202809577487505\\
15404.633346365	0.193252076255507\\
16029.5132936809	0.184145580455488\\
16679.7411308022	0.175468779243353\\
17356.3450800625	0.167201370100529\\
18060.3950730332	0.159324011061069\\
18793.0044424346	0.151818275205167\\
19555.3316826779	0.144666607311448\\
20348.5822818231	0.137852282565485\\
21174.0106278484	0.131359367226851\\
22032.9219922476	0.125172681161715\\
22926.6745940892	0.119277762152364\\
23856.6817478045	0.113660831899263\\
24824.4140980988	0.108308763635298\\
25831.4019455198	0.103209051275636\\
26879.2376663628	0.0983497800302959\\
27969.5782307366	0.0937195984099828\\
29104.1478227738	0.089307691559051\\
30284.7405671288	0.0851037558525847\\
31513.2233660736	0.0810979746976006\\
32791.5388516798	0.0772809954812166\\
34121.7084577534	0.073643907611351\\
35505.8356163813	0.0701782215981026\\
36946.1090841431	0.066875849126428\\
38444.8064032501	0.0637290840730751\\
40004.2975030821	0.0607305844229726\\
41627.0484478186	0.0578733550423974\\
43315.6253360912	0.0551507312682761\\
45072.6983588229	0.0525563632748993\\
46901.04602167	0.0500842011811772\\
48803.5595387472	0.0477284808633064\\
50783.2474045797	0.0454837104393919\\
52843.2401515145	0.0433446573941577\\
54986.7953001132	0.0413063363133879\\
57217.3025103552	0.0393639971991871\\
59538.2889417954	0.0375131143385193\\
61953.4248311545	0.0357493756987909\\
64466.5292961603	0.0340686728254921\\
67081.576374819	0.0324670912180944\\
69802.7013096654	0.030940901161533\\
72634.2070869308	0.0294865489916786\\
75580.5712409674	0.0281006487742297\\
78646.4529346918	0.0267799743774291\\
81836.7003272403	0.0255214519199429\\
85156.3582404912	0.0243221525761222\\
88610.6761365729	0.0231792857217136\\
92205.1164189772	0.022090192403886\\
95945.3630704002	0.02105233912021\\
99837.3306409738	0.0200633118919503\\
103887.173601099	0.0191208106177315\\
108101.296073671	0.0182226436942939\\
112486.361961086	0.0173667228916889\\
117049.305483048	0.0165510584708632\\
121797.342141828	0.0157737545321523\\
126737.980132329	0.0150330045837467\\
131879.032214993	0.0143270873197164\\
137228.628070322	0.013654362597669\\
142795.227154554	0.0130132676065907\\
148587.632076826	0.0124023132158645\\
154615.002518971	0.0118200804968904\\
160886.869719957	0.0112652174091345\\
167413.151547894	0.0107364356428265\\
174204.168183413	0.0102325076108883\\
181270.658439238	0.00975226358303312\\
188623.796741757	0.0092945889553052\\
196275.210801431	0.00885842164865161\\
204237	0.00844274963042014\\
};
\addlegendentry{Adatt. Pareto esatto (\textit{ML})}


\addplot[area legend, draw=none, fill=mycolor1, fill opacity=0.15, forget plot]
table[row sep=crcr] {%
x	y\\
3985.7162970862	0.989941789532646\\
4147.39454893027	0.956479779588114\\
4315.63118455555	0.913234470257356\\
4490.69224096661	0.869836525229641\\
4672.85454680357	0.828508035149297\\
4862.40616009855	0.789150658075715\\
5059.64682378936	0.751670806391365\\
5264.88843971087	0.715979428749156\\
5478.45556181355	0.681991802640957\\
5700.68590938921	0.649627335867146\\
5931.93090111527	0.618809374279075\\
6172.55621076243	0.589465011833717\\
6422.94234544414	0.561524897297776\\
6683.48524732248	0.534923030288175\\
6954.59691972191	0.509596538871755\\
7236.70607864091	0.485485433763872\\
7530.25883069187	0.462532342800684\\
7835.7193785411	0.440682244555705\\
8153.5707549648	0.419882236920187\\
8484.31558668133	0.400081383322203\\
8828.47688916811	0.381230663989205\\
9186.5988937197	0.363283023822634\\
9559.24790805498	0.346193472307193\\
9947.01321183443	0.329919177733809\\
10350.5079885035	0.314419512931701\\
10770.3702949358	0.299656037794063\\
11207.2640704089	0.28559242708472\\
11661.880186509	0.272194362493818\\
12134.9375396247	0.259429407869229\\
12627.1841877562	0.247266881575148\\
13139.398533439	0.235677734294645\\
13672.3905546528	0.224634436223061\\
14227.0030856605	0.214110874830654\\
14804.1131498041	0.204082262878668\\
15404.633346365	0.194525055705621\\
16029.5132936809	0.185416876601227\\
16679.7411308022	0.176736449119146\\
17356.3450800625	0.168463535311121\\
18060.3950730332	0.160578879022825\\
18793.0044424346	0.153064153542801\\
19555.3316826779	0.145901913027173\\
20348.5822818231	0.139075547231376\\
21174.0106278484	0.132569239167508\\
22032.9219922476	0.126367925375113\\
22926.6745940892	0.120457258547678\\
23856.6817478045	0.114823572299828\\
24824.4140980988	0.109453847893893\\
25831.4019455198	0.104335682770999\\
26879.2376663628	0.0994572607529866\\
27969.5782307366	0.0948073237983691\\
29104.1478227738	0.0903751452092198\\
30284.7405671288	0.0861505041970574\\
31513.2233660736	0.0821236617250508\\
32791.5388516798	0.0782853375515684\\
34121.7084577534	0.0746266884066281\\
35505.8356163813	0.0711392872383687\\
36946.1090841431	0.0678151034714879\\
38444.8064032501	0.0646464842237807\\
40004.2975030821	0.0616261364306299\\
41627.0484478186	0.0587471098305816\\
43315.6253360912	0.0560027807681033\\
45072.6983588229	0.0533868367722841\\
46901.04602167	0.0508932618726715\\
48803.5595387472	0.0485163226156566\\
50783.2474045797	0.0462505547468661\\
52843.2401515145	0.0440907505269093\\
54986.7953001132	0.0420319466495807\\
57217.3025103552	0.0400694127332492\\
59538.2889417954	0.0381986403576956\\
61953.4248311545	0.0364153326200783\\
64466.5292961603	0.0347153941850578\\
67081.576374819	0.0330949218053609\\
69802.7013096654	0.0315501952902591\\
72634.2070869308	0.0300776689005508\\
75580.5712409674	0.0286739631496984\\
78646.4529346918	0.0273358569917673\\
81836.7003272403	0.0260602803777625\\
85156.3582404912	0.0248443071628545\\
88610.6761365729	0.0236851483478354\\
92205.1164189772	0.0225801456389508\\
95945.3630704002	0.0215267653110214\\
99837.3306409738	0.0205225923594887\\
103887.173601099	0.0195653249277138\\
108101.296073671	0.0186527689965105\\
112486.361961086	0.0177828333235177\\
117049.305483048	0.016953524620607\\
121797.342141828	0.0161629429580855\\
126737.980132329	0.0154092773849886\\
131879.032214993	0.0146908017552668\\
137228.628070322	0.0140058707501551\\
142795.227154554	0.0133529160874753\\
148587.632076826	0.0127304429090588\\
154615.002518971	0.0121370263378978\\
160886.869719957	0.0115713081970275\\
167413.151547894	0.0110319938825235\\
174204.168183413	0.0105178493833553\\
181270.658439238	0.0100276984411849\\
188623.796741757	0.00956041984352129\\
196275.210801431	0.00911494484395621\\
204237	0.00869025470350256\\
204237	0.00819524455733771\\
196275.210801431	0.00860189845334701\\
188623.796741757	0.0090287580670891\\
181270.658439238	0.0094768287248813\\
174204.168183413	0.00994716583842133\\
167413.151547894	0.0104408774031296\\
160886.869719957	0.0109591266212415\\
154615.002518971	0.011503134655883\\
148587.632076826	0.0120741835226702\\
142795.227154554	0.012673619125706\\
137228.628070322	0.0133028544451828\\
131879.032214993	0.013963372884166\\
126737.980132329	0.0146567317825048\\
121797.342141828	0.0153845661062191\\
117049.305483048	0.0161485923211195\\
112486.361961086	0.0169506124598601\\
108101.296073671	0.0177925183920773\\
103887.173601099	0.0186762963077493\\
99837.3306409738	0.0196040314244118\\
95945.3630704002	0.0205779129293985\\
92205.1164189772	0.0216002391688213\\
88610.6761365729	0.0226734230955919\\
85156.3582404912	0.0237999979893899\\
81836.7003272403	0.0249826234621233\\
78646.4529346918	0.0262240917630909\\
75580.5712409674	0.027527334398761\\
72634.2070869308	0.0288954290828064\\
69802.7013096654	0.0303316070328069\\
67081.576374819	0.0318392606308279\\
64466.5292961603	0.0334219514659264\\
61953.4248311545	0.0350834187775034\\
59538.2889417954	0.0368275883193429\\
57217.3025103552	0.0386585816651249\\
54986.7953001132	0.0405807259771951\\
52843.2401515145	0.0425985642614061\\
50783.2474045797	0.0447168661319178\\
48803.5595387472	0.0469406391109561\\
46901.04602167	0.049275140489683\\
45072.6983588229	0.0517258897775145\\
43315.6253360912	0.0542986817684489\\
41627.0484478186	0.0569996002542133\\
40004.2975030821	0.0598350324153153\\
38444.8064032501	0.0628116839223696\\
36946.1090841431	0.0659365947813682\\
35505.8356163813	0.0692171559578365\\
34121.7084577534	0.072661126816074\\
32791.5388516798	0.0762766534108647\\
31513.2233660736	0.0800722876701505\\
30284.7405671288	0.084057007508112\\
29104.1478227738	0.0882402379088822\\
27969.5782307366	0.0926318730215964\\
26879.2376663628	0.0972422993076052\\
25831.4019455198	0.102082419780273\\
24824.4140980988	0.107163679376702\\
23856.6817478045	0.112498091498698\\
22926.6745940892	0.11809826575705\\
22032.9219922476	0.123977436948317\\
21174.0106278484	0.130149495286194\\
20348.5822818231	0.136629017899593\\
19555.3316826779	0.143431301595723\\
18793.0044424346	0.150572396867532\\
18060.3950730332	0.158069143099314\\
17356.3450800625	0.165939204889938\\
16679.7411308022	0.174201109367559\\
16029.5132936809	0.18287428430975\\
15404.633346365	0.191979096805392\\
14804.1131498041	0.201536892096342\\
14227.0030856605	0.211570032116333\\
13672.3905546528	0.222101933104845\\
13139.398533439	0.233157101527713\\
12627.1841877562	0.244761167417378\\
12134.9375396247	0.256940914224964\\
11661.880186509	0.269724304489473\\
11207.2640704089	0.283140501311169\\
10770.3702949358	0.297219887125635\\
10350.5079885035	0.311994084059313\\
9947.01321183443	0.327495984533923\\
9559.24790805498	0.343759806439438\\
9186.5988937197	0.360821192188793\\
8828.47688916811	0.378717371028338\\
8484.31558668133	0.397487393517542\\
8153.5707549648	0.417172423909944\\
7835.7193785411	0.437816048100642\\
7530.25883069187	0.459464539912686\\
7236.70607864091	0.482167041663031\\
6954.59691972191	0.50597565112054\\
6683.48524732248	0.530945442831966\\
6422.94234544414	0.557134467090494\\
6172.55621076243	0.584603762996071\\
5931.93090111527	0.613417405139369\\
5700.68590938921	0.643642588278388\\
5478.45556181355	0.675349745776446\\
5264.88843971087	0.708612694789876\\
5059.64682378936	0.743508801696133\\
4862.40616009855	0.780119162943624\\
4672.85454680357	0.818528798248781\\
4490.69224096661	0.858826854442908\\
4315.63118455555	0.901106819225821\\
4147.39454893027	0.944524921025225\\
3985.7162970862	0.981533202829427\\
}--cycle;
\addplot[only marks, mark=*, mark size=1.3693pt, color=black, fill=black, opacity=0.60, draw=none, mark options={draw=none,line width=0pt}] table[row sep=crcr]{%
x	y\\
3650	0.99468085106383\\
3735	0.984042553191489\\
3741	0.973404255319149\\
3761	0.962765957446808\\
3772	0.952127659574468\\
3785	0.941489361702128\\
3825	0.930851063829787\\
3847	0.920212765957447\\
3900	0.909574468085106\\
3928	0.898936170212766\\
3945	0.888297872340426\\
3996	0.877659574468085\\
4009	0.867021276595745\\
4024	0.856382978723404\\
4061	0.845744680851064\\
4061	0.835106382978723\\
4071	0.824468085106383\\
4113	0.813829787234043\\
4228	0.803191489361702\\
4348	0.792553191489362\\
4430	0.781914893617021\\
4503	0.771276595744681\\
4539	0.76063829787234\\
4632	0.75\\
4654	0.73936170212766\\
4663	0.728723404255319\\
4679	0.718085106382979\\
4702	0.707446808510638\\
4779	0.696808510638298\\
4861	0.686170212765957\\
4953	0.675531914893617\\
5029	0.664893617021277\\
5048	0.654255319148936\\
5236	0.643617021276596\\
5264	0.632978723404255\\
5327	0.622340425531915\\
5379	0.611702127659574\\
5458	0.601063829787234\\
5463	0.590425531914894\\
5524	0.579787234042553\\
5607	0.569148936170213\\
5857	0.558510638297872\\
5868	0.547872340425532\\
5982	0.537234042553192\\
6317	0.526595744680851\\
6332	0.515957446808511\\
6342	0.50531914893617\\
6356	0.49468085106383\\
6367	0.484042553191489\\
6468	0.473404255319149\\
6629	0.462765957446808\\
6926	0.452127659574468\\
7252	0.441489361702128\\
7294	0.430851063829787\\
7320	0.420212765957447\\
7348	0.409574468085106\\
7596	0.398936170212766\\
7724	0.388297872340426\\
7877	0.377659574468085\\
8035	0.367021276595745\\
8499	0.356382978723404\\
8518	0.345744680851064\\
8994	0.335106382978723\\
9128	0.324468085106383\\
9267	0.313829787234043\\
9435	0.303191489361702\\
9599	0.292553191489362\\
9837	0.281914893617021\\
10119	0.271276595744681\\
10336	0.26063829787234\\
10377	0.25\\
11048	0.23936170212766\\
11424	0.228723404255319\\
11830	0.218085106382979\\
12182	0.207446808510638\\
12313	0.196808510638298\\
13086	0.186170212765957\\
13380	0.175531914893617\\
13398	0.164893617021277\\
13899	0.154255319148936\\
14984	0.143617021276596\\
16428	0.132978723404255\\
20491	0.122340425531915\\
21264	0.111702127659574\\
23237	0.101063829787234\\
30134	0.0904255319148936\\
30990	0.0797872340425532\\
32887	0.0691489361702128\\
37527	0.0585106382978723\\
39026	0.0478723404255319\\
41095	0.0372340425531915\\
61636	0.0265957446808511\\
122339	0.0159574468085106\\
204237	0.00531914893617021\\
};
\addlegendentry{FRC empirica reale}

\addplot [color=black]
table[row sep=crcr]{%
3637.5	1\\
3788.54993935839	0.949537816810858\\
3945.87234172165	0.901622065553929\\
4109.72767586131	0.856124247714574\\
4280.38722671172	0.812922349093734\\
4458.13354451935	0.771900512595218\\
4643.26091264341	0.732948727524845\\
4836.07583478224	0.695962534568237\\
5036.89754243213	0.660842745656075\\
5246.05852341874	0.627495177965562\\
5463.90507237627	0.59583040134476\\
5690.79786408554	0.565763498482441\\
5927.11255062052	0.53721383718029\\
6173.24038329176	0.510104854116755\\
6429.58886041643	0.484363849522645\\
6696.58240198765	0.459921792217835\\
6974.66305235981	0.436713134486259\\
7264.29121211354	0.414675636292709\\
7565.94640031188	0.393750198370032\\
7880.12804840974	0.373880703729123\\
8207.35632713103	0.355013867166658\\
8548.17300768248	0.337099092367008\\
8903.14235873003	0.32008833621509\\
9272.85208062291	0.303935979956297\\
9657.9142784119	0.28859870685797\\
10058.9664752731	0.274035386044354\\
10476.6726680149	0.260206962193476\\
10911.7244264153	0.247076350800179\\
11364.8420382106	0.234608338724395\\
11836.7757016304	0.222769489757984\\
12328.3067674531	0.211528054956865\\
12840.2490326395	0.200853887497989\\
13373.4500876847	0.190718361832814\\
13928.7927199204	0.181094296920473\\
14507.19637509	0.171955883334763\\
15109.6186796172	0.163278614049474\\
15737.0570260872	0.15503921871644\\
16390.550224567	0.147215601260069\\
17071.1802224973	0.139786780620984\\
17780.0738960051	0.132732834489867\\
18518.4049156008	0.126034845880625\\
19287.3956893508	0.119674852399582\\
20088.3193867412	0.113635798074661\\
20922.5020465844	0.107901487615373\\
21791.3247724572	0.102456542980945\\
22696.2260193077	0.0972863621401141\\
23638.7039750138	0.0923770799119943\\
24620.3190408384	0.0877155307829971\\
25642.6964148873	0.0832892136000928\\
26707.5287828473	0.0790862580457252\\
27816.5791204588	0.0750953928044779\\
28971.6836123634	0.0713059153361178\\
30174.7546921595	0.0677076631739571\\
31427.7842086969	0.0642909866715642\\
32732.8467238564	0.061046723124733\\
34092.1029472698	0.0579661721993158\\
35507.8033136712	0.0550410725990206\\
36982.2917087997	0.0522635799106019\\
38518.0093500227	0.0496262455670326\\
40117.4988281057	0.0471219968722397\\
41783.4083168193	0.0447441180338345\\
43518.4959573534	0.0424862321529745\\
45325.6344247971	0.0403422841230547\\
47207.8156842464	0.0383065243913687\\
49168.1559444106	0.0363734935401921\\
51209.900816924	0.0345380076459378\\
53336.4306898988	0.0327951443771205\\
55551.2663246212	0.0311402297938478\\
57858.074684653	0.0295688258134387\\
60260.6750069927	0.0280767183085531\\
62763.0451253437	0.026659905805917\\
65369.3280559641	0.0253145887553335\\
68083.8388569957	0.024037159340204\\
70911.0717726344	0.022824191802232\\
73855.7076739664	0.0216724332543637\\
76922.6218087908	0.0205787949573275\\
80116.89187326	0.0195403440363791\\
83443.8064187003	0.0185542956160364\\
86908.8736075329	0.0176180053517145\\
90517.8303327903	0.016728962338229\\
94276.651716329	0.015884782376153\\
98191.5610014598	0.0150832015779679\\
102269.039856381	0.0143220702968617\\
106515.839105467	0.0135993473618937\\
110938.989906178	0.012913094604065\\
115545.815390112	0.012261471658616\\
120343.942787444	0.0116427310296104\\
125341.316054851	0.0110552134035723\\
130546.209027823	0.0104973431996062\\
135967.239119128	0.00996762434406833\\
141613.381586117	0.00946463625845741\\
147493.984390494	0.00898703004976452\\
153618.783675143	0.00853352489306699\\
159997.919883649	0.00810290459666392\\
166641.954549187	0.00769401434054292\\
173561.887780587	0.00730575757943056\\
180769.176474522	0.00693709310212186\\
188275.753283964	0.00658703223920245\\
196094.046374327	0.00625463621167503\\
204237	0.00593901361338004\\
};
\addlegendentry{Adatt. Pareto reale (\textit{ML})}

        \nextgroupplot[%
            plotColumn1,plotLegend2,
            xmode=log,ymode=log,
            xmin=3637.5,xmax=204237,
            ymin=0.00531914893617021,ymax=1,
            xlabel={\(s\)},
        ] 
% !TEX root = ../../../../Esperimenti/.tex/MEF.tex

% This file was created by matlab2tikz.
%
\definecolor{mycolor1}{rgb}{0.83529,0.36863,0.00000}%
\definecolor{mycolor2}{rgb}{0.00000,0.44700,0.74100}%
\definecolor{mycolor3}{rgb}{0.49400,0.18400,0.55600}%

\addplot[only marks, mark=*, mark size=1.3693pt, color=mycolor1, fill=mycolor1, opacity=0.60, draw=none, mark options={draw=none,line width=0pt}] table[row sep=crcr]{%
x	y\\
4090.18789382316	0.99468085106383\\
4122.09547982752	0.984042553191489\\
4155.04168189341	0.973404255319149\\
4194.69949416662	0.962765957446808\\
4239.74118419993	0.952127659574468\\
4284.14775374348	0.941489361702128\\
4330.24604033271	0.930851063829787\\
4374.42901462382	0.920212765957447\\
4425.13150595817	0.909574468085106\\
4469.71677077288	0.898936170212766\\
4520.54761538257	0.888297872340426\\
4560.18451428432	0.877659574468085\\
4605.50393519694	0.867021276595745\\
4657.30327134483	0.856382978723404\\
4700.97627303685	0.845744680851064\\
4764.37249076017	0.835106382978723\\
4808.31084550097	0.824468085106383\\
4874.99236961615	0.813829787234043\\
4926.42597563426	0.803191489361702\\
4973.82225422745	0.792553191489362\\
5018.37572312591	0.781914893617021\\
5084.18360572234	0.771276595744681\\
5148.06580759128	0.76063829787234\\
5204.23223982815	0.75\\
5274.07005104382	0.73936170212766\\
5337.89874570431	0.728723404255319\\
5415.64072264054	0.718085106382979\\
5492.76880574403	0.707446808510638\\
5565.72397109615	0.696808510638298\\
5634.05504390289	0.686170212765957\\
5703.6717794501	0.675531914893617\\
5784.95733651798	0.664893617021277\\
5867.14096454924	0.654255319148936\\
5952.74586240556	0.643617021276596\\
6041.49472536148	0.632978723404255\\
6130.22261687393	0.622340425531915\\
6230.18175789701	0.611702127659574\\
6335.72240830055	0.601063829787234\\
6429.80027789145	0.590425531914894\\
6538.27224602741	0.579787234042553\\
6634.36178031828	0.569148936170213\\
6749.76659591886	0.558510638297872\\
6864.05526878626	0.547872340425532\\
7010.08105950752	0.537234042553192\\
7143.49211427853	0.526595744680851\\
7261.02306424498	0.515957446808511\\
7387.77746796246	0.50531914893617\\
7513.94326822623	0.49468085106383\\
7642.19981546648	0.484042553191489\\
7783.36360893193	0.473404255319149\\
7940.17715306541	0.462765957446808\\
8092.78906026177	0.452127659574468\\
8277.82714000132	0.441489361702128\\
8451.81740192651	0.430851063829787\\
8623.7856090143	0.420212765957447\\
8806.05785187102	0.409574468085106\\
9033.84717483674	0.398936170212766\\
9247.1834759257	0.388297872340426\\
9450.43484427712	0.377659574468085\\
9664.94657901608	0.367021276595745\\
9897.67501014668	0.356382978723404\\
10141.3064587777	0.345744680851064\\
10422.2118059154	0.335106382978723\\
10727.76232014	0.324468085106383\\
11045.1971940702	0.313829787234043\\
11350.7470178221	0.303191489361702\\
11741.8345825804	0.292553191489362\\
12086.8201259851	0.281914893617021\\
12495.1086998833	0.271276595744681\\
12874.6765953411	0.26063829787234\\
13249.5964673784	0.25\\
13706.5394621796	0.23936170212766\\
14123.5348041346	0.228723404255319\\
14564.9355335215	0.218085106382979\\
15180.7478800256	0.207446808510638\\
15793.3084745867	0.196808510638298\\
16500.4549809074	0.186170212765957\\
17155.3858329086	0.175531914893617\\
17904.9389199844	0.164893617021277\\
18673.2121393376	0.154255319148936\\
19513.0759429737	0.143617021276596\\
20512.5326585322	0.132978723404255\\
21534.1123741018	0.122340425531915\\
22627.4606917686	0.111702127659574\\
23813.8581385178	0.101063829787234\\
25082.3723158769	0.0904255319148936\\
26637.3422859925	0.0797872340425532\\
28639.6059362482	0.0691489361702128\\
31065.3781275239	0.0585106382978723\\
33881.7568032755	0.0478723404255319\\
37558.3652092283	0.0372340425531915\\
42879.3474844928	0.0265957446808511\\
50610.4225690929	0.0159574468085106\\
66648.0604708262	0.00531914893617021\\
};
\addlegendentry{FRC empirica appr.}

\addplot [color=mycolor1, only marks, every error bar/.append style={opacity=0.30}, mark=*, mark size=0pt, draw=none, forget plot]
plot [error bars/.cd, x dir=both, x explicit, error bar style={line width=1pt, color=mycolor1}, error mark options={mark=none,mark size=0pt}]
table[row sep=crcr, x error plus index=2, x error minus index=3]{%
4090.18789382316	0.99468085106383	166.458766544861	166.458766544861\\
4122.09547982752	0.984042553191489	167.7433334849	167.7433334849\\
4155.04168189341	0.973404255319149	169.010385830021	169.010385830021\\
4194.69949416662	0.962765957446808	169.325108951434	169.325108951434\\
4239.74118419993	0.952127659574468	171.823764486805	171.823764486805\\
4284.14775374348	0.941489361702128	176.857996456427	176.857996456427\\
4330.24604033271	0.930851063829787	179.561779416429	179.561779416429\\
4374.42901462382	0.920212765957447	181.698430631808	181.698430631808\\
4425.13150595817	0.909574468085106	183.909255659238	183.909255659238\\
4469.71677077288	0.898936170212766	186.363091466225	186.363091466225\\
4520.54761538257	0.888297872340426	189.513037538107	189.513037538107\\
4560.18451428432	0.877659574468085	191.866018519974	191.866018519974\\
4605.50393519694	0.867021276595745	192.916580605487	192.916580605487\\
4657.30327134483	0.856382978723404	194.937802461083	194.937802461083\\
4700.97627303685	0.845744680851064	195.053159326233	195.053159326233\\
4764.37249076017	0.835106382978723	197.692778547019	197.692778547019\\
4808.31084550097	0.824468085106383	198.442260038738	198.442260038738\\
4874.99236961615	0.813829787234043	205.183610217414	205.183610217414\\
4926.42597563426	0.803191489361702	207.700978451702	207.700978451702\\
4973.82225422745	0.792553191489362	208.305633210768	208.305633210768\\
5018.37572312591	0.781914893617021	209.180972486008	209.180972486008\\
5084.18360572234	0.771276595744681	211.589719867085	211.589719867085\\
5148.06580759128	0.76063829787234	217.995888398394	217.995888398394\\
5204.23223982815	0.75	219.607469820092	219.607469820092\\
5274.07005104382	0.73936170212766	223.211377796883	223.211377796883\\
5337.89874570431	0.728723404255319	225.802515824091	225.802515824091\\
5415.64072264054	0.718085106382979	226.186647359474	226.186647359474\\
5492.76880574403	0.707446808510638	229.680932209235	229.680932209235\\
5565.72397109615	0.696808510638298	234.315607480887	234.315607480887\\
5634.05504390289	0.686170212765957	235.867618217599	235.867618217599\\
5703.6717794501	0.675531914893617	237.810031631943	237.810031631943\\
5784.95733651798	0.664893617021277	241.325290460416	241.325290460416\\
5867.14096454924	0.654255319148936	248.800916471037	248.800916471037\\
5952.74586240556	0.643617021276596	254.170554596976	254.170554596976\\
6041.49472536148	0.632978723404255	254.261673265285	254.261673265285\\
6130.22261687393	0.622340425531915	256.192940115814	256.192940115814\\
6230.18175789701	0.611702127659574	257.57736591349	257.57736591349\\
6335.72240830055	0.601063829787234	260.074250314107	260.074250314107\\
6429.80027789145	0.590425531914894	264.397525074104	264.397525074104\\
6538.27224602741	0.579787234042553	264.856409803303	264.856409803303\\
6634.36178031828	0.569148936170213	267.584879142862	267.584879142862\\
6749.76659591886	0.558510638297872	268.508763266451	268.508763266451\\
6864.05526878626	0.547872340425532	272.602933109253	272.602933109253\\
7010.08105950752	0.537234042553192	281.632062863936	281.632062863936\\
7143.49211427853	0.526595744680851	289.906335821306	289.906335821306\\
7261.02306424498	0.515957446808511	290.716699344693	290.716699344693\\
7387.77746796246	0.50531914893617	293.604697431073	293.604697431073\\
7513.94326822623	0.49468085106383	298.329793092426	298.329793092426\\
7642.19981546648	0.484042553191489	301.594290328255	301.594290328255\\
7783.36360893193	0.473404255319149	309.134495430939	309.134495430939\\
7940.17715306541	0.462765957446808	314.736067193438	314.736067193438\\
8092.78906026177	0.452127659574468	322.922007691259	322.922007691259\\
8277.82714000132	0.441489361702128	333.711215938974	333.711215938974\\
8451.81740192651	0.430851063829787	339.141136735576	339.141136735576\\
8623.7856090143	0.420212765957447	346.824249339896	346.824249339896\\
8806.05785187102	0.409574468085106	354.137732011337	354.137732011337\\
9033.84717483674	0.398936170212766	373.783381206163	373.783381206163\\
9247.1834759257	0.388297872340426	389.610240377353	389.610240377353\\
9450.43484427712	0.377659574468085	393.112608453473	393.112608453473\\
9664.94657901608	0.367021276595745	398.397791825441	398.397791825441\\
9897.67501014668	0.356382978723404	417.514715448383	417.514715448383\\
10141.3064587777	0.345744680851064	425.337140513658	425.337140513658\\
10422.2118059154	0.335106382978723	437.758779113336	437.758779113336\\
10727.76232014	0.324468085106383	454.95355504627	454.95355504627\\
11045.1971940702	0.313829787234043	474.006239439475	474.006239439475\\
11350.7470178221	0.303191489361702	489.37808609872	489.37808609872\\
11741.8345825804	0.292553191489362	513.64331648725	513.64331648725\\
12086.8201259851	0.281914893617021	542.562595298593	542.562595298593\\
12495.1086998833	0.271276595744681	569.421478529583	569.421478529583\\
12874.6765953411	0.26063829787234	580.320113281177	580.320113281177\\
13249.5964673784	0.25	601.218553201691	601.218553201691\\
13706.5394621796	0.23936170212766	617.024831062476	617.024831062476\\
14123.5348041346	0.228723404255319	636.911612206262	636.911612206262\\
14564.9355335215	0.218085106382979	645.567118911371	645.567118911371\\
15180.7478800256	0.207446808510638	656.782201446524	656.782201446524\\
15793.3084745867	0.196808510638298	671.980914163578	671.980914163578\\
16500.4549809074	0.186170212765957	703.564939247906	703.564939247906\\
17155.3858329086	0.175531914893617	757.8807600079	757.8807600079\\
17904.9389199844	0.164893617021277	805.174208435116	805.174208435116\\
18673.2121393376	0.154255319148936	840.906376829417	840.906376829417\\
19513.0759429737	0.143617021276596	850.83013936028	850.83013936028\\
20512.5326585322	0.132978723404255	905.256668019495	905.256668019495\\
21534.1123741018	0.122340425531915	955.37865371116	955.37865371116\\
22627.4606917686	0.111702127659574	984.542344845506	984.542344845506\\
23813.8581385178	0.101063829787234	1037.83859244011	1037.83859244011\\
25082.3723158769	0.0904255319148936	1072.56169294271	1072.56169294271\\
26637.3422859925	0.0797872340425532	1154.37796393949	1154.37796393949\\
28639.6059362482	0.0691489361702128	1260.96972527419	1260.96972527419\\
31065.3781275239	0.0585106382978723	1401.20512261195	1401.20512261195\\
33881.7568032755	0.0478723404255319	1478.42565752305	1478.42565752305\\
37558.3652092283	0.0372340425531915	1838.09768744092	1838.09768744092\\
42879.3474844928	0.0265957446808511	2461.58459353075	2461.58459353075\\
50610.4225690929	0.0159574468085106	2926.19509112914	2926.19509112914\\
66648.0604708262	0.00531914893617021	4980.53227793302	4980.53227793302\\
};
\addplot [color=mycolor1]
table[row sep=crcr]{%
4068.45705100297	0.900525103157597\\
4232.61309178989	0.874743130628581\\
4403.39257861275	0.847776104916631\\
4581.06275742351	0.819601632011465\\
4765.90165714095	0.789820234141071\\
4958.19852472696	0.7588963626295\\
5158.25427781745	0.727741086474661\\
5366.3819756163	0.696512612109698\\
5582.90730878905	0.665206283098123\\
5808.16910912323	0.634073675301564\\
6042.51987975248	0.603823908604417\\
6286.32634677464	0.574956916658895\\
6539.9700331266	0.54752049913212\\
6803.84785561425	0.52100825858129\\
7078.37274603159	0.495746142987926\\
7363.97429734111	0.471369875775326\\
7661.09943592655	0.448027740387064\\
7970.21312097002	0.425844234848318\\
8291.79907204793	0.404761712988198\\
8626.36052608429	0.384725403471335\\
8974.42102484597	0.36568326609467\\
9336.52523421214	0.347585855284122\\
9713.23979649998	0.330386190429535\\
10105.1542171805	0.314039632714575\\
10512.8817873719	0.29850376811549\\
10937.0605435543	0.283738296259053\\
11378.3542660072	0.2697049248456\\
11837.4535175326	0.256367269357882\\
12315.0767240891	0.243690757790452\\
12811.9712990276	0.231642540147716\\
13328.9148126885	0.220191402471384\\
13866.7162091903	0.209307685170126\\
14426.2170723137	0.198963205435644\\
15008.292942462	0.189131183540218\\
15613.8546867595	0.179786172821105\\
16243.8499244305	0.170903993166919\\
16899.2645096906	0.162461667830439\\
17581.1240744703	0.154437363401102\\
18290.4956333853	0.146810332778813\\
19028.4892534646	0.139560860998667\\
19796.25979125	0.132670213763713\\
20595.0086999843	0.126120588550114\\
21425.9859097174	0.119895068155789\\
22290.4917832717	0.113977576570191\\
23189.8791511274	0.108352837048927\\
24125.5554284124	0.103006332282833\\
25098.9848173099	0.0979242665566221\\
26111.6905983294	0.0930935297974879\\
27165.2575140274	0.0885016634190651\\
28261.3342489067	0.0841368278708746\\
29401.636009377	0.0799877718078977\\
30587.9472078122	0.0760438027992091\\
31822.1242549057	0.0722947594986565\\
33106.098464693	0.0687309852044444\\
34441.8790767867	0.0653433027381424\\
35831.5564005558	0.0621229905771233\\
37277.3050861659	0.0590617601777473\\
38781.3875276022	0.056151734429746\\
40346.1574029988	0.0533854271852485\\
41974.0633578151	0.0507557238087245\\
43667.6528366225	0.0482558626968107\\
45429.5760694978	0.0458794177195471\\
47262.5902192622	0.0436202815369717\\
49169.5636960539	0.0414726497473366\\
51153.4806459882	0.0394310058253932\\
53217.4456209277	0.0374901068112775\\
55364.6884366707	0.0356449697125025\\
57598.5692271598	0.0338908585834425\\
59922.5837026207	0.0322232722484736\\
62340.3686198586	0.0306379326366356\\
64855.7074732724	0.0291307736972785\\
67472.5364154933	0.0276979308676958\\
70194.9504169115	0.0263357310651887\\
73027.2096737307	0.0250406831773898\\
75973.7462745781	0.0238094690259809\\
79039.1711361016	0.0226389347801858\\
82228.2812184075	0.0215260827975961\\
85546.0670316312	0.0204680638710163\\
88997.7204453849	0.0194621698610736\\
92588.6428133067	0.0185058266953557\\
96324.4534254207	0.0175965877157982\\
100210.998301539	0.0167321273569587\\
104254.359339461	0.0159102351386807\\
108460.863832294	0.0151288099574769\\
112837.094369777	0.014385854661742\\
117389.899139113	0.0136794708966514\\
122126.402641418	0.013007854205306\\
127054.016840571	0.012369289373358\\
132180.452761894	0.0117621460049859\\
137513.732558829	0.0111848743186972\\
143062.202066483	0.0106360011520074\\
148834.543861686	0.0101141261645971\\
154839.79085001	0.00961791823006052\\
161087.340401	0.00914611200685869\\
167586.969053735	0.00869750467955547\\
174348.847815749	0.00827095286186019\\
181383.558079223	0.00786536965342464\\
188702.108179387	0.00747972184274269\\
196315.950621017	0.00711302724888323\\
204237	0.00676435219514924\\
};
\addlegendentry{Adatt. Pareto appr. (\textit{ML})}


\addplot[area legend, draw=none, fill=mycolor1, fill opacity=0.15, forget plot]
table[row sep=crcr] {%
x	y\\
4068.45705100297	0.923711143870348\\
4232.61309178989	0.900219650189486\\
4403.39257861275	0.875371433153945\\
4581.06275742351	0.849001834001422\\
4765.90165714095	0.820521454014548\\
4958.19852472696	0.790415632652346\\
5158.25427781745	0.7597317040739\\
5366.3819756163	0.728606165272565\\
5582.90730878905	0.696950396178654\\
5808.16910912323	0.665027459863733\\
6042.51987975248	0.633798873509941\\
6286.32634677464	0.604004255753895\\
6539.9700331266	0.575743679624617\\
6803.84785561425	0.548215122237351\\
7078.37274603159	0.521979590612608\\
7363.97429734111	0.49641689919322\\
7661.09943592655	0.471814225528751\\
7970.21312097002	0.44843604009419\\
8291.79907204793	0.426221147170205\\
8626.36052608429	0.405111421139046\\
8974.42102484597	0.385051651940115\\
9336.52523421214	0.365989398327821\\
9713.23979649998	0.347874848536757\\
10105.1542171805	0.330660687979319\\
10512.8817873719	0.314301973619874\\
10937.0605435543	0.298756014687668\\
11378.3542660072	0.283982259407791\\
11837.4535175326	0.269942187445782\\
12315.0767240891	0.256599207776895\\
12811.9712990276	0.24391856170571\\
13328.9148126885	0.231867230775668\\
13866.7162091903	0.220413849321323\\
14426.2170723137	0.209528621428623\\
15008.292942462	0.199183242080435\\
15613.8546867595	0.189350822275804\\
16243.8499244305	0.180005817922178\\
16899.2645096906	0.171123962309964\\
17581.1240744703	0.162682201988471\\
18290.4956333853	0.154658635871433\\
19028.4892534646	0.147032457409018\\
19796.25979125	0.139783899671487\\
20595.0086999843	0.132894183197486\\
21425.9859097174	0.12634546646742\\
22290.4917832717	0.120120798869408\\
23189.8791511274	0.114204076032009\\
24125.5554284124	0.108579997404299\\
25098.9848173099	0.103234025969903\\
26111.6905983294	0.0981523499873209\\
27165.2575140274	0.0933218466543433\\
28261.3342489067	0.0887300475994992\\
29401.636009377	0.0843651061084091\\
30587.9472078122	0.0802157659975544\\
31822.1242549057	0.0762713320524062\\
33106.098464693	0.072521641951051\\
34441.8790767867	0.0689570395984352\\
35831.5564005558	0.0655683498001347\\
37277.3050861659	0.0623468542081465\\
38781.3875276022	0.0592842684746103\\
40346.1574029988	0.0563727205526034\\
41974.0633578151	0.0536047300862271\\
43667.6528366225	0.0509731888351197\\
45429.5760694978	0.0484713420813029\\
47262.5902192622	0.0460927709688954\\
49169.5636960539	0.0438313757297281\\
51153.4806459882	0.0416813597502639\\
53217.4456209277	0.0396372144374755\\
55364.6884366707	0.0376937048434747\\
57598.5692271598	0.0358458560107113\\
59922.5837026207	0.0340889400014894\\
62340.3686198586	0.0324184635773777\\
64855.7074732724	0.0308301564958244\\
67472.5364154933	0.0293199603929401\\
70194.9504169115	0.0278840182229729\\
73027.2096737307	0.0265186642264919\\
75973.7462745781	0.0252204144007009\\
79039.1711361016	0.0239859574466483\\
82228.2812184075	0.0228121461693694\\
85546.0670316312	0.0216959893082055\\
88997.7204453849	0.0206346437756932\\
92588.6428133067	0.0196254072845019\\
96324.4534254207	0.0186657113429374\\
100210.998301539	0.0177531146005065\\
104254.359339461	0.0168852965259711\\
108460.863832294	0.0160600514012084\\
112837.094369777	0.0152752826150289\\
117389.899139113	0.0145289972419064\\
122126.402641418	0.0138193008913304\\
127054.016840571	0.0131443928142099\\
132180.452761894	0.0125025612534423\\
137513.732558829	0.0118921790264083\\
143062.202066483	0.0113116993277716\\
148834.543861686	0.0107596517415442\\
154839.79085001	0.0102346384519363\\
161087.340401	0.00973533064303498\\
167586.969053735	0.00926046507785814\\
174348.847815749	0.00880884084780359\\
181383.558079223	0.00837931628396707\\
188702.108179387	0.00797080602222956\\
196315.950621017	0.00758227821442242\\
204237	0.00721275187826487\\
204237	0.0063159525120336\\
196315.950621017	0.00664377628334405\\
188702.108179387	0.00698863766325582\\
181383.558079223	0.00735142302288221\\
174348.847815749	0.00773306487591679\\
167586.969053735	0.0081345442812528\\
161087.340401	0.00855689337068241\\
154839.79085001	0.00900119800818478\\
148834.543861686	0.00946860058765006\\
143062.202066483	0.00996030297624323\\
137513.732558829	0.010477569610986\\
132180.452761894	0.0110217307565296\\
127054.016840571	0.011594185932506\\
122126.402641418	0.0121964075192815\\
117389.899139113	0.0128299445513963\\
112837.094369777	0.0134964267084551\\
108460.863832294	0.0141975685137453\\
104254.359339461	0.0149351737513903\\
100210.998301539	0.0157111401134109\\
96324.4534254207	0.0165274640886589\\
92588.6428133067	0.0173862461062095\\
88997.7204453849	0.018289695946454\\
85546.0670316312	0.019240138433827\\
82228.2812184075	0.0202400194258228\\
79039.1711361016	0.0212919121137232\\
75973.7462745781	0.022398523651261\\
73027.2096737307	0.0235627021282877\\
70194.9504169115	0.0247874439074046\\
67472.5364154933	0.0260759013424516\\
64855.7074732724	0.0274313908987326\\
62340.3686198586	0.0288574016958935\\
59922.5837026207	0.0303576044954579\\
57598.5692271598	0.0319358611561736\\
55364.6884366707	0.0335962345815304\\
53217.4456209277	0.0353429991850794\\
51153.4806459882	0.0371806519005225\\
49169.5636960539	0.0391139237649451\\
47262.5902192622	0.041147792105048\\
45429.5760694978	0.0432874933577913\\
43667.6528366225	0.0455385365585017\\
41974.0633578151	0.0479067175312218\\
40346.1574029988	0.0503981338178936\\
38781.3875276022	0.0530192003848817\\
37277.3050861659	0.0557766661473482\\
35831.5564005558	0.0586776313541119\\
34441.8790767867	0.0617295658778495\\
33106.098464693	0.0649403284578379\\
31822.1242549057	0.0683181869449068\\
30587.9472078122	0.0718718396008637\\
29401.636009377	0.0756104375073862\\
28261.3342489067	0.0795436081422499\\
27165.2575140274	0.083681480183787\\
26111.6905983294	0.0880347096076548\\
25098.9848173099	0.0926145071433415\\
24125.5554284124	0.0974326671613673\\
23189.8791511274	0.102501598065845\\
22290.4917832717	0.107834354270974\\
21425.9859097174	0.113444669844159\\
20595.0086999843	0.119346993902742\\
19796.25979125	0.12555652785594\\
19028.4892534646	0.132089264588315\\
18290.4956333853	0.138962029686194\\
17581.1240744703	0.146192524813732\\
16899.2645096906	0.153799373350913\\
16243.8499244305	0.16180216841166\\
15613.8546867595	0.170221523366407\\
15008.292942462	0.179079125000001\\
14426.2170723137	0.188397789442664\\
13866.7162091903	0.19820152101893\\
13328.9148126885	0.208515574167101\\
12811.9712990276	0.219366518589722\\
12315.0767240891	0.230782307804009\\
11837.4535175326	0.242792351269982\\
11378.3542660072	0.25542759028341\\
10937.0605435543	0.268720577830438\\
10512.8817873719	0.282705562611107\\
10105.1542171805	0.297418577449831\\
9713.23979649998	0.312897532322312\\
9336.52523421214	0.329182312240424\\
8974.42102484597	0.346314880249225\\
8626.36052608429	0.364339385803623\\
8291.79907204793	0.383302278806191\\
7970.21312097002	0.403252429602445\\
7661.09943592655	0.424241255245377\\
7363.97429734111	0.446322852357432\\
7078.37274603159	0.469512695363244\\
6803.84785561425	0.493801394925229\\
6539.9700331266	0.519297318639622\\
6286.32634677464	0.545909577563894\\
6042.51987975248	0.573848943698893\\
5808.16910912323	0.603119890739395\\
5582.90730878905	0.633462170017592\\
5366.3819756163	0.664419058946831\\
5158.25427781745	0.695750468875422\\
4958.19852472696	0.727377092606655\\
4765.90165714095	0.759119014267593\\
4581.06275742351	0.790201430021507\\
4403.39257861275	0.820180776679316\\
4232.61309178989	0.849266611067676\\
4068.45705100297	0.877339062444847\\
}--cycle;
\addplot[only marks, mark=*, mark size=1.3693pt, color=black, fill=black, opacity=0.60, draw=none, mark options={draw=none,line width=0pt}] table[row sep=crcr]{%
x	y\\
3650	0.99468085106383\\
3735	0.984042553191489\\
3741	0.973404255319149\\
3761	0.962765957446808\\
3772	0.952127659574468\\
3785	0.941489361702128\\
3825	0.930851063829787\\
3847	0.920212765957447\\
3900	0.909574468085106\\
3928	0.898936170212766\\
3945	0.888297872340426\\
3996	0.877659574468085\\
4009	0.867021276595745\\
4024	0.856382978723404\\
4061	0.845744680851064\\
4061	0.835106382978723\\
4071	0.824468085106383\\
4113	0.813829787234043\\
4228	0.803191489361702\\
4348	0.792553191489362\\
4430	0.781914893617021\\
4503	0.771276595744681\\
4539	0.76063829787234\\
4632	0.75\\
4654	0.73936170212766\\
4663	0.728723404255319\\
4679	0.718085106382979\\
4702	0.707446808510638\\
4779	0.696808510638298\\
4861	0.686170212765957\\
4953	0.675531914893617\\
5029	0.664893617021277\\
5048	0.654255319148936\\
5236	0.643617021276596\\
5264	0.632978723404255\\
5327	0.622340425531915\\
5379	0.611702127659574\\
5458	0.601063829787234\\
5463	0.590425531914894\\
5524	0.579787234042553\\
5607	0.569148936170213\\
5857	0.558510638297872\\
5868	0.547872340425532\\
5982	0.537234042553192\\
6317	0.526595744680851\\
6332	0.515957446808511\\
6342	0.50531914893617\\
6356	0.49468085106383\\
6367	0.484042553191489\\
6468	0.473404255319149\\
6629	0.462765957446808\\
6926	0.452127659574468\\
7252	0.441489361702128\\
7294	0.430851063829787\\
7320	0.420212765957447\\
7348	0.409574468085106\\
7596	0.398936170212766\\
7724	0.388297872340426\\
7877	0.377659574468085\\
8035	0.367021276595745\\
8499	0.356382978723404\\
8518	0.345744680851064\\
8994	0.335106382978723\\
9128	0.324468085106383\\
9267	0.313829787234043\\
9435	0.303191489361702\\
9599	0.292553191489362\\
9837	0.281914893617021\\
10119	0.271276595744681\\
10336	0.26063829787234\\
10377	0.25\\
11048	0.23936170212766\\
11424	0.228723404255319\\
11830	0.218085106382979\\
12182	0.207446808510638\\
12313	0.196808510638298\\
13086	0.186170212765957\\
13380	0.175531914893617\\
13398	0.164893617021277\\
13899	0.154255319148936\\
14984	0.143617021276596\\
16428	0.132978723404255\\
20491	0.122340425531915\\
21264	0.111702127659574\\
23237	0.101063829787234\\
30134	0.0904255319148936\\
30990	0.0797872340425532\\
32887	0.0691489361702128\\
37527	0.0585106382978723\\
39026	0.0478723404255319\\
41095	0.0372340425531915\\
61636	0.0265957446808511\\
122339	0.0159574468085106\\
204237	0.00531914893617021\\
};
\addlegendentry{FRC empirica reale}

\addplot [color=black]
table[row sep=crcr]{%
3637.5	1\\
3788.54993935839	0.949537816810858\\
3945.87234172165	0.901622065553929\\
4109.72767586131	0.856124247714574\\
4280.38722671172	0.812922349093734\\
4458.13354451935	0.771900512595218\\
4643.26091264341	0.732948727524845\\
4836.07583478224	0.695962534568237\\
5036.89754243213	0.660842745656075\\
5246.05852341874	0.627495177965562\\
5463.90507237627	0.59583040134476\\
5690.79786408554	0.565763498482441\\
5927.11255062052	0.53721383718029\\
6173.24038329176	0.510104854116755\\
6429.58886041643	0.484363849522645\\
6696.58240198765	0.459921792217835\\
6974.66305235981	0.436713134486259\\
7264.29121211354	0.414675636292709\\
7565.94640031188	0.393750198370032\\
7880.12804840974	0.373880703729123\\
8207.35632713103	0.355013867166658\\
8548.17300768248	0.337099092367008\\
8903.14235873003	0.32008833621509\\
9272.85208062291	0.303935979956297\\
9657.9142784119	0.28859870685797\\
10058.9664752731	0.274035386044354\\
10476.6726680149	0.260206962193476\\
10911.7244264153	0.247076350800179\\
11364.8420382106	0.234608338724395\\
11836.7757016304	0.222769489757984\\
12328.3067674531	0.211528054956865\\
12840.2490326395	0.200853887497989\\
13373.4500876847	0.190718361832814\\
13928.7927199204	0.181094296920473\\
14507.19637509	0.171955883334763\\
15109.6186796172	0.163278614049474\\
15737.0570260872	0.15503921871644\\
16390.550224567	0.147215601260069\\
17071.1802224973	0.139786780620984\\
17780.0738960051	0.132732834489867\\
18518.4049156008	0.126034845880625\\
19287.3956893508	0.119674852399582\\
20088.3193867412	0.113635798074661\\
20922.5020465844	0.107901487615373\\
21791.3247724572	0.102456542980945\\
22696.2260193077	0.0972863621401141\\
23638.7039750138	0.0923770799119943\\
24620.3190408384	0.0877155307829971\\
25642.6964148873	0.0832892136000928\\
26707.5287828473	0.0790862580457252\\
27816.5791204588	0.0750953928044779\\
28971.6836123634	0.0713059153361178\\
30174.7546921595	0.0677076631739571\\
31427.7842086969	0.0642909866715642\\
32732.8467238564	0.061046723124733\\
34092.1029472698	0.0579661721993158\\
35507.8033136712	0.0550410725990206\\
36982.2917087997	0.0522635799106019\\
38518.0093500227	0.0496262455670326\\
40117.4988281057	0.0471219968722397\\
41783.4083168193	0.0447441180338345\\
43518.4959573534	0.0424862321529745\\
45325.6344247971	0.0403422841230547\\
47207.8156842464	0.0383065243913687\\
49168.1559444106	0.0363734935401921\\
51209.900816924	0.0345380076459378\\
53336.4306898988	0.0327951443771205\\
55551.2663246212	0.0311402297938478\\
57858.074684653	0.0295688258134387\\
60260.6750069927	0.0280767183085531\\
62763.0451253437	0.026659905805917\\
65369.3280559641	0.0253145887553335\\
68083.8388569957	0.024037159340204\\
70911.0717726344	0.022824191802232\\
73855.7076739664	0.0216724332543637\\
76922.6218087908	0.0205787949573275\\
80116.89187326	0.0195403440363791\\
83443.8064187003	0.0185542956160364\\
86908.8736075329	0.0176180053517145\\
90517.8303327903	0.016728962338229\\
94276.651716329	0.015884782376153\\
98191.5610014598	0.0150832015779679\\
102269.039856381	0.0143220702968617\\
106515.839105467	0.0135993473618937\\
110938.989906178	0.012913094604065\\
115545.815390112	0.012261471658616\\
120343.942787444	0.0116427310296104\\
125341.316054851	0.0110552134035723\\
130546.209027823	0.0104973431996062\\
135967.239119128	0.00996762434406833\\
141613.381586117	0.00946463625845741\\
147493.984390494	0.00898703004976452\\
153618.783675143	0.00853352489306699\\
159997.919883649	0.00810290459666392\\
166641.954549187	0.00769401434054292\\
173561.887780587	0.00730575757943056\\
180769.176474522	0.00693709310212186\\
188275.753283964	0.00658703223920245\\
196094.046374327	0.00625463621167503\\
204237	0.00593901361338004\\
};
\addlegendentry{Adatt. Pareto reale (\textit{ML})}

        \nextgroupplot[%
            plotColumn2,plotLegend2,
            xmode=log,ymode=log,
            xmin=3637.5,xmax=204237,
            ymin=0.00531914893617021,ymax=1,
            xlabel={\(s\)},
        ] 
% !TEX root = ../../../../Esperimenti/.tex/MEF.tex

% This file was created by matlab2tikz.
%
\definecolor{mycolor1}{rgb}{0.83529,0.36863,0.00000}%
\definecolor{mycolor2}{rgb}{0.00000,0.44700,0.74100}%
\definecolor{mycolor3}{rgb}{0.49400,0.18400,0.55600}%

\addplot[only marks, mark=*, mark size=1.3693pt, color=mycolor1, fill=mycolor1, opacity=0.60, draw=none, mark options={draw=none,line width=0pt}] table[row sep=crcr]{%
x	y\\
4144.75288262406	0.99468085106383\\
4179.62398058572	0.984042553191489\\
4215.76024339066	0.973404255319149\\
4256.14087550221	0.962765957446808\\
4303.86970447649	0.952127659574468\\
4343.18737876807	0.941489361702128\\
4390.37266294463	0.930851063829787\\
4424.74479568227	0.920212765957447\\
4471.75403931124	0.909574468085106\\
4520.01015855406	0.898936170212766\\
4570.52426240352	0.888297872340426\\
4624.65483241745	0.877659574468085\\
4663.77914308373	0.867021276595745\\
4717.50439294368	0.856382978723404\\
4771.488031781	0.845744680851064\\
4831.00107994651	0.835106382978723\\
4887.72135056638	0.824468085106383\\
4940.41835653766	0.813829787234043\\
4996.35032779911	0.803191489361702\\
5050.17202057511	0.792553191489362\\
5121.20281532318	0.781914893617021\\
5180.84404747116	0.771276595744681\\
5250.93167895285	0.76063829787234\\
5321.93926594653	0.75\\
5391.82927229045	0.73936170212766\\
5465.67895842508	0.728723404255319\\
5535.0906657203	0.718085106382979\\
5607.35195203084	0.707446808510638\\
5685.49652017257	0.696808510638298\\
5781.98891406254	0.686170212765957\\
5872.38215431129	0.675531914893617\\
5952.95952902381	0.664893617021277\\
6034.8274489136	0.654255319148936\\
6133.10853535397	0.643617021276596\\
6233.47090227089	0.632978723404255\\
6352.62425722548	0.622340425531915\\
6466.02604386459	0.611702127659574\\
6580.12268256736	0.601063829787234\\
6691.57471525446	0.590425531914894\\
6791.29299466151	0.579787234042553\\
6900.90962910194	0.569148936170213\\
7039.73958208549	0.558510638297872\\
7186.99464757832	0.547872340425532\\
7307.30519723254	0.537234042553192\\
7438.19254289039	0.526595744680851\\
7579.65013434	0.515957446808511\\
7724.10340491736	0.50531914893617\\
7861.75403655186	0.49468085106383\\
8023.32242809818	0.484042553191489\\
8193.73163164291	0.473404255319149\\
8371.78995311987	0.462765957446808\\
8541.75850645131	0.452127659574468\\
8735.99262940824	0.441489361702128\\
8948.36212983917	0.430851063829787\\
9105.90763001572	0.420212765957447\\
9294.05030927583	0.409574468085106\\
9519.62368952413	0.398936170212766\\
9726.82183739778	0.388297872340426\\
9907.03624847639	0.377659574468085\\
10127.9353127805	0.367021276595745\\
10383.059062363	0.356382978723404\\
10652.6734202282	0.345744680851064\\
10907.8009417955	0.335106382978723\\
11162.9043994948	0.324468085106383\\
11457.2729231007	0.313829787234043\\
11769.2942768269	0.303191489361702\\
12187.7978838888	0.292553191489362\\
12561.9176585095	0.281914893617021\\
12929.5137952714	0.271276595744681\\
13362.7089927517	0.26063829787234\\
13783.3662032309	0.25\\
14178.2168808309	0.23936170212766\\
14607.2562288249	0.228723404255319\\
15180.502836343	0.218085106382979\\
15737.7901099512	0.207446808510638\\
16397.0435955197	0.196808510638298\\
16994.933946114	0.186170212765957\\
17589.6442569585	0.175531914893617\\
18218.9254411867	0.164893617021277\\
19020.8357830833	0.154255319148936\\
19882.2098921862	0.143617021276596\\
20751.0796200933	0.132978723404255\\
21726.8444487339	0.122340425531915\\
22639.3437118845	0.111702127659574\\
23704.2117880131	0.101063829787234\\
25098.2757067462	0.0904255319148936\\
26770.285221367	0.0797872340425532\\
28461.9339637705	0.0691489361702128\\
30765.0569667666	0.0585106382978723\\
33379.7839544832	0.0478723404255319\\
36663.2525500314	0.0372340425531915\\
41450.1635496427	0.0265957446808511\\
48171.8277553066	0.0159574468085106\\
60318.6063434032	0.00531914893617021\\
};
\addlegendentry{FRC empirica appr.}

\addplot [color=mycolor1, only marks, every error bar/.append style={opacity=0.30}, mark=*, mark size=0pt, draw=none, forget plot]
plot [error bars/.cd, x dir=both, x explicit, error bar style={line width=1pt, color=mycolor1}, error mark options={mark=none,mark size=0pt}]
table[row sep=crcr, x error plus index=2, x error minus index=3]{%
4144.75288262406	0.99468085106383	31.6609730361255	31.6609730361255\\
4179.62398058572	0.984042553191489	32.342536214027	32.342536214027\\
4215.76024339066	0.973404255319149	33.5616182317548	33.5616182317548\\
4256.14087550221	0.962765957446808	34.4839063949425	34.4839063949425\\
4303.86970447649	0.952127659574468	35.8731136633835	35.8731136633835\\
4343.18737876807	0.941489361702128	36.7631206246115	36.7631206246115\\
4390.37266294463	0.930851063829787	37.2102891741294	37.2102891741294\\
4424.74479568227	0.920212765957447	37.5915183149188	37.5915183149188\\
4471.75403931124	0.909574468085106	38.6686046463841	38.6686046463841\\
4520.01015855406	0.898936170212766	39.8824454105956	39.8824454105956\\
4570.52426240352	0.888297872340426	40.4230926017322	40.4230926017322\\
4624.65483241745	0.877659574468085	40.2194844001611	40.2194844001611\\
4663.77914308373	0.867021276595745	39.0026901582762	39.0026901582762\\
4717.50439294368	0.856382978723404	40.4401252546307	40.4401252546307\\
4771.488031781	0.845744680851064	40.2260010668223	40.2260010668223\\
4831.00107994651	0.835106382978723	41.422386966584	41.422386966584\\
4887.72135056638	0.824468085106383	41.226407559053	41.226407559053\\
4940.41835653766	0.813829787234043	42.3655443425873	42.3655443425873\\
4996.35032779911	0.803191489361702	40.309090158689	40.309090158689\\
5050.17202057511	0.792553191489362	42.4424800667169	42.4424800667169\\
5121.20281532318	0.781914893617021	44.3333364936642	44.3333364936642\\
5180.84404747116	0.771276595744681	46.5537585489176	46.5537585489176\\
5250.93167895285	0.76063829787234	47.7288335859149	47.7288335859149\\
5321.93926594653	0.75	48.8025814181248	48.8025814181248\\
5391.82927229045	0.73936170212766	48.9931503633503	48.9931503633503\\
5465.67895842508	0.728723404255319	50.34897526967	50.34897526967\\
5535.0906657203	0.718085106382979	53.6190972199121	53.6190972199121\\
5607.35195203084	0.707446808510638	52.5512860205778	52.5512860205778\\
5685.49652017257	0.696808510638298	52.6415275844415	52.6415275844415\\
5781.98891406254	0.686170212765957	53.0507922110169	53.0507922110169\\
5872.38215431129	0.675531914893617	55.3541226781098	55.3541226781098\\
5952.95952902381	0.664893617021277	55.9135835001205	55.9135835001205\\
6034.8274489136	0.654255319148936	57.7180972131915	57.7180972131915\\
6133.10853535397	0.643617021276596	58.4036202546019	58.4036202546019\\
6233.47090227089	0.632978723404255	58.3888884078917	58.3888884078917\\
6352.62425722548	0.622340425531915	61.2945458195759	61.2945458195759\\
6466.02604386459	0.611702127659574	63.2783048765789	63.2783048765789\\
6580.12268256736	0.601063829787234	67.2923693037132	67.2923693037132\\
6691.57471525446	0.590425531914894	70.7869375154473	70.7869375154473\\
6791.29299466151	0.579787234042553	75.3399779782439	75.3399779782439\\
6900.90962910194	0.569148936170213	77.0653254727872	77.0653254727872\\
7039.73958208549	0.558510638297872	77.7570297994815	77.7570297994815\\
7186.99464757832	0.547872340425532	78.6752103943273	78.6752103943273\\
7307.30519723254	0.537234042553192	81.8321363707108	81.8321363707108\\
7438.19254289039	0.526595744680851	79.9838899199234	79.9838899199234\\
7579.65013434	0.515957446808511	81.7500773913838	81.7500773913838\\
7724.10340491736	0.50531914893617	86.6441189049122	86.6441189049122\\
7861.75403655186	0.49468085106383	93.2202420880527	93.2202420880527\\
8023.32242809818	0.484042553191489	92.1137245152766	92.1137245152766\\
8193.73163164291	0.473404255319149	89.7121580344394	89.7121580344394\\
8371.78995311987	0.462765957446808	88.8495455637563	88.8495455637563\\
8541.75850645131	0.452127659574468	92.4286911553245	92.4286911553245\\
8735.99262940824	0.441489361702128	96.3801534185793	96.3801534185793\\
8948.36212983917	0.430851063829787	95.8296681222396	95.8296681222396\\
9105.90763001572	0.420212765957447	99.3668197166248	99.3668197166248\\
9294.05030927583	0.409574468085106	102.223147570175	102.223147570175\\
9519.62368952413	0.398936170212766	102.794370881455	102.794370881455\\
9726.82183739778	0.388297872340426	105.096730835378	105.096730835378\\
9907.03624847639	0.377659574468085	101.472967518765	101.472967518765\\
10127.9353127805	0.367021276595745	103.276236087283	103.276236087283\\
10383.059062363	0.356382978723404	112.355746146727	112.355746146727\\
10652.6734202282	0.345744680851064	122.938192011411	122.938192011411\\
10907.8009417955	0.335106382978723	127.438950813625	127.438950813625\\
11162.9043994948	0.324468085106383	127.572645839786	127.572645839786\\
11457.2729231007	0.313829787234043	139.939871814126	139.939871814126\\
11769.2942768269	0.303191489361702	148.56740103998	148.56740103998\\
12187.7978838888	0.292553191489362	161.92042273555	161.92042273555\\
12561.9176585095	0.281914893617021	155.895537261709	155.895537261709\\
12929.5137952714	0.271276595744681	159.715527082339	159.715527082339\\
13362.7089927517	0.26063829787234	172.172336466328	172.172336466328\\
13783.3662032309	0.25	181.699739106327	181.699739106327\\
14178.2168808309	0.23936170212766	181.796635197945	181.796635197945\\
14607.2562288249	0.228723404255319	191.099879387254	191.099879387254\\
15180.502836343	0.218085106382979	191.105856854057	191.105856854057\\
15737.7901099512	0.207446808510638	209.612324685417	209.612324685417\\
16397.0435955197	0.196808510638298	224.769820544273	224.769820544273\\
16994.933946114	0.186170212765957	222.790562802458	222.790562802458\\
17589.6442569585	0.175531914893617	211.996581061599	211.996581061599\\
18218.9254411867	0.164893617021277	232.850942333069	232.850942333069\\
19020.8357830833	0.154255319148936	247.278841092114	247.278841092114\\
19882.2098921862	0.143617021276596	245.272027795904	245.272027795904\\
20751.0796200933	0.132978723404255	272.409829038342	272.409829038342\\
21726.8444487339	0.122340425531915	292.121936646406	292.121936646406\\
22639.3437118845	0.111702127659574	290.898602668675	290.898602668675\\
23704.2117880131	0.101063829787234	303.872581891293	303.872581891293\\
25098.2757067462	0.0904255319148936	368.896741638298	368.896741638298\\
26770.285221367	0.0797872340425532	363.967534374096	363.967534374096\\
28461.9339637705	0.0691489361702128	436.582202432453	436.582202432453\\
30765.0569667666	0.0585106382978723	473.329613520878	473.329613520878\\
33379.7839544832	0.0478723404255319	612.199499630426	612.199499630426\\
36663.2525500314	0.0372340425531915	730.610234020293	730.610234020293\\
41450.1635496427	0.0265957446808511	879.965291884655	879.965291884655\\
48171.8277553066	0.0159574468085106	1053.00817522267	1053.00817522267\\
60318.6063434032	0.00531914893617021	1346.27555551412	1346.27555551412\\
};
\addplot [color=mycolor1]
table[row sep=crcr]{%
4121.27247091489	0.983240745572417\\
4287.00096819517	0.949582922549605\\
4459.39389618335	0.907108023712944\\
4638.71925125536	0.863560406380808\\
4825.25580671027	0.821944574316917\\
5019.29354614212	0.781955330538238\\
5221.13411423881	0.74391610290176\\
5431.09128570909	0.707731575441243\\
5649.49145306632	0.67331110274834\\
5876.67413402763	0.64056848051825\\
6112.9924993169	0.609421727398617\\
6358.81392169242	0.579792877583263\\
6614.52054705249	0.551607783620014\\
6880.50988850682	0.524795928928183\\
7157.1954443374	0.499290249546256\\
7445.00734080926	0.475026964654067\\
7744.39300083068	0.451945415436334\\
8055.81783950207	0.429987911875837\\
8379.76598763506	0.409099587084943\\
8716.74104436623	0.389228258803511\\
9067.26686003583	0.370324297709648\\
9431.8883505483	0.35234050220725\\
9811.17234448058	0.335231979370921\\
10205.7084642552	0.318956031744626\\
10616.1100427478	0.303472049705477\\
11043.0150767544	0.288741409118311\\
11487.0872188003	0.274727374020279\\
11949.0168088322	0.261395004087563\\
12429.521947398	0.248711066648589\\
12929.3496119827	0.236643953019727\\
13449.2768182345	0.225163598950586\\
13990.1118278881	0.214241408976459\\
14552.6954052618	0.203850184485545\\
15137.9021242818	0.193964055318013\\
15746.6417280664	0.184558414723049\\
16379.8605431825	0.17560985750859\\
17038.5429507739	0.167096121226607\\
17723.7129168476	0.158996030244575\\
18436.435584098	0.1512894425611\\
19177.8189277426	0.143957199230729\\
19949.0154779436	0.136981076269581\\
20751.2241114938	0.130343738919808\\
21585.6919155506	0.124028698156875\\
22453.716126317	0.118020269329403\\
23356.6461456822	0.112303532826726\\
24295.8856389574	0.106864296674498\\
25272.8947169677	0.101689060963595\\
26289.1922058925	0.0967649840222249\\
27346.3580083826	0.0920798502455905\\
28446.0355596248	0.0876220395016887\\
29589.9343821726	0.0833804980358193\\
30779.8327435131	0.0793447108002017\\
32017.5804205038	0.0755046751387198\\
33305.1015749745	0.0718508757602594\\
34644.3977449659	0.0683742609373813\\
36037.5509562552	0.0650662198701796\\
37486.7269590047	0.0619185611581471\\
38994.1785945654	0.0589234923256693\\
40562.2492976707	0.0560736003494556\\
42193.3767394626	0.0533618331387518\\
43890.0966170153	0.0507814819216003\\
45655.046595246	0.0483261644927097\\
47490.9704073427	0.0459898092806817\\
49400.722120081	0.043766640194422\\
51387.2705706629	0.0416511622105343\\
53453.703981973	0.0396381476653759\\
55603.2347634286	0.0377226232172377\\
57839.2045048855	0.0358998574458082\\
60165.0891713642	0.0341653490576963\\
62584.5045066697	0.0325148156683193\\
65101.2116543091	0.0309441831319217\\
67719.1230044409	0.0294495753928787\\
70442.3082759484	0.0280273048327548\\
73275.0008430902	0.02667386308884\\
76221.6043165656	0.0253859123210831\\
79286.6993892222	0.0241602769054673\\
82475.0509570505	0.0229939355329579\\
85791.6155265345	0.0218840136941695\\
89241.548919874	0.02082777653088\\
92830.2142900556	0.0198226220364407\\
96563.1904582344	0.0188660745880124\\
100446.280586386	0.0179557787943976\\
104485.521198711	0.017089493644032\\
108687.191565816	0.0162650869384551\\
113057.823466263	0.0154805299973024\\
117604.211340657	0.0147338926215428\\
122333.42285406	0.0140233383023363\\
127252.809883148	0.0133471196635076\\
132370.019945201	0.0127035741262164\\
137693.008086678	0.0120911197849671\\
143230.049249875	0.0115082514846314\\
148989.751136874	0.0109535370886627\\
154981.067590793	0.0104256139291642\\
161213.312515141	0.00992318542992575\\
167696.174352902	0.00944501789398531\\
174439.731147871	0.00898993744767856\\
181454.46621165	0.00855682713353707\\
188751.284420657	0.00814462414476873\\
196341.529168492	0.0077523171944084\\
204237	0.00737894401256679\\
};
\addlegendentry{Adatt. Pareto appr. (\textit{ML})}


\addplot[area legend, draw=none, fill=mycolor1, fill opacity=0.15, forget plot]
table[row sep=crcr] {%
x	y\\
4121.27247091489	0.987910052798955\\
4287.00096819517	0.956397468515223\\
4459.39389618335	0.914528500809317\\
4638.71925125536	0.870614191574209\\
4825.25580671027	0.828632799600922\\
5019.29354614212	0.787970926758448\\
5221.13411423881	0.749314910092052\\
5431.09128570909	0.712566196876232\\
5649.49145306632	0.677631222967765\\
5876.67413402763	0.644421187600926\\
6112.9924993169	0.612851842159063\\
6358.81392169242	0.582843289918462\\
6614.52054705249	0.55431979085781\\
6880.50988850682	0.527209560719153\\
7157.1954443374	0.50144454639452\\
7445.00734080926	0.476960152095487\\
7744.39300083068	0.453694888873424\\
8055.81783950207	0.43158993652733\\
8379.76598763506	0.410588654850327\\
8716.74104436623	0.390636152272475\\
9067.26686003583	0.371679061697391\\
9431.8883505483	0.353665617131502\\
9811.17234448058	0.336545977731673\\
10205.7084642552	0.320272622489463\\
10616.1100427478	0.304800639296059\\
11043.0150767544	0.290087829964551\\
11487.0872188003	0.276094650865409\\
11949.0168088322	0.262784052216178\\
12429.521947398	0.250121275924999\\
12929.3496119827	0.238073650949174\\
13449.2768182345	0.226610405341332\\
13990.1118278881	0.215702501365161\\
14552.6954052618	0.205322493508565\\
15137.9021242818	0.195444406581369\\
15746.6417280664	0.186043630460208\\
16379.8605431825	0.177096828288867\\
17038.5429507739	0.168581855472328\\
17723.7129168476	0.160477687355794\\
18436.435584098	0.152764353959398\\
19177.8189277426	0.14542288052263\\
19949.0154779436	0.138435232906829\\
20751.2241114938	0.131784267125329\\
21585.6919155506	0.125453682435593\\
22453.716126317	0.119427977549972\\
23356.6461456822	0.113692409612693\\
24295.8856389574	0.108232955658651\\
25272.8947169677	0.103036276320823\\
26289.1922058925	0.0980896815919657\\
27346.3580083826	0.093381098476221\\
28446.0355596248	0.0888990403894075\\
29589.9343821726	0.0846325781851063\\
30779.8327435131	0.080571312698206\\
32017.5804205038	0.0767053487093724\\
33305.1015749745	0.0730252702435748\\
34644.3977449659	0.0695221171238438\\
36037.5509562552	0.0661873627082066\\
37486.7269590047	0.0630128927435397\\
38994.1785945654	0.0599909852750713\\
40562.2492976707	0.0571142915546479\\
42193.3767394626	0.0543758178947379\\
43890.0966170153	0.051768908418595\\
45655.046595246	0.0492872286601067\\
47490.9704073427	0.0469247499696603\\
49400.722120081	0.0446757346849304\\
51387.2705706629	0.04253472202784\\
53453.703981973	0.0404965146911223\\
55603.2347634286	0.0385561660799226\\
57839.2045048855	0.0367089681757452\\
60165.0891713642	0.0349504399918007\\
62584.5045066697	0.0332763165904363\\
65101.2116543091	0.031682538634861\\
67719.1230044409	0.0301652424488187\\
70442.3082759484	0.0287207505592084\\
73275.0008430902	0.0273455626979274\\
76221.6043165656	0.026036347240416\\
79286.6993892222	0.0247899330595144\\
82475.0509570505	0.0236033017743192\\
85791.6155265345	0.0224735803747383\\
89241.548919874	0.021398034203408\\
92830.2142900556	0.0203740602775444\\
96563.1904582344	0.0193991809341655\\
100446.280586386	0.0184710377829374\\
104485.521198711	0.0175873859516778\\
108687.191565816	0.0167460886102817\\
113057.823466263	0.0159451117595395\\
117604.211340657	0.0151825192719772\\
122333.42285406	0.0144564681724804\\
127252.809883148	0.0137652041470646\\
132370.019945201	0.0131070572687218\\
137693.008086678	0.0124804379298132\\
143230.049249875	0.011883832970995\\
148989.751136874	0.0113158019971495\\
154981.067590793	0.0107749738712603\\
161213.312515141	0.0102600433776108\\
167696.174352902	0.00976976804610451\\
174439.731147871	0.00930296512990629\\
181454.46621165	0.00885850872898117\\
188751.284420657	0.00843532705246999\\
196341.529168492	0.00803239981318248\\
204237	0.00764875574781651\\
204237	0.00710913227731707\\
196341.529168492	0.00747223457563433\\
188751.284420657	0.00785392123706747\\
181454.46621165	0.00825514553809297\\
174439.731147871	0.00867690976545083\\
167696.174352902	0.00912026774186611\\
161213.312515141	0.00958632748224067\\
154981.067590793	0.010076253987068\\
148989.751136874	0.010591272180176\\
143230.049249875	0.0111326699982677\\
137693.008086678	0.011701801640121\\
132370.019945201	0.012300090983711\\
127252.809883148	0.0129290351799506\\
122333.42285406	0.0135902084321923\\
117604.211340657	0.0142852659711084\\
113057.823466263	0.0150159482350653\\
108687.191565816	0.0157840852666285\\
104485.521198711	0.0165916013363862\\
100446.280586386	0.0174405198058577\\
96563.1904582344	0.0183329682418593\\
92830.2142900556	0.019271183795337\\
89241.548919874	0.0202575188583521\\
85791.6155265345	0.0212944470136006\\
82475.0509570505	0.0223845692915965\\
79286.6993892222	0.0235306207514203\\
76221.6043165656	0.0247354774017502\\
73275.0008430902	0.0260021634797526\\
70442.3082759484	0.0273338591063011\\
67719.1230044409	0.0287339083369387\\
65101.2116543091	0.0302058276289823\\
62584.5045066697	0.0317533147462023\\
60165.0891713642	0.0333802581235919\\
57839.2045048855	0.0350907467158712\\
55603.2347634286	0.0368890803545528\\
53453.703981973	0.0387797806396295\\
51387.2705706629	0.0407676023932286\\
49400.722120081	0.0428575457039135\\
47490.9704073427	0.0450548685917031\\
45655.046595246	0.0473651003253128\\
43890.0966170153	0.0497940554246056\\
42193.3767394626	0.0523478483827656\\
40562.2492976707	0.0550329091442632\\
38994.1785945654	0.0578559993762674\\
37486.7269590047	0.0608242295727545\\
36037.5509562552	0.0639450770321525\\
34644.3977449659	0.0672264047509187\\
33305.1015749745	0.070676481276944\\
32017.5804205038	0.0743040015680671\\
30779.8327435131	0.0781181089021974\\
29589.9343821726	0.0821284178865324\\
28446.0355596248	0.0863450386139699\\
27346.3580083826	0.0907786020149599\\
26289.1922058925	0.0954402864524842\\
25272.8947169677	0.100341845606368\\
24295.8856389574	0.105495637690344\\
23356.6461456822	0.11091465604076\\
22453.716126317	0.116612561108835\\
21585.6919155506	0.122603713878158\\
20751.2241114938	0.128903210714287\\
19949.0154779436	0.135526919632334\\
19177.8189277426	0.142491517938828\\
18436.435584098	0.149814531162803\\
17723.7129168476	0.157514373133356\\
17038.5429507739	0.165610386980887\\
16379.8605431825	0.174122886728313\\
15746.6417280664	0.183073198985891\\
15137.9021242818	0.192483704054657\\
14552.6954052618	0.202377875462525\\
13990.1118278881	0.212780316587756\\
13449.2768182345	0.223716792559839\\
12929.3496119827	0.235214255090281\\
12429.521947398	0.24730085737218\\
11949.0168088322	0.260005955958949\\
11487.0872188003	0.273360097175149\\
11043.0150767544	0.287394988272071\\
10616.1100427478	0.302143460114894\\
10205.7084642552	0.317639440999789\\
9811.17234448058	0.333917981010168\\
9431.8883505483	0.351015387282997\\
9067.26686003583	0.368969533721904\\
8716.74104436623	0.387820365334547\\
8379.76598763506	0.407610519319559\\
8055.81783950207	0.428385887224345\\
7744.39300083068	0.450195941999243\\
7445.00734080926	0.473093777212647\\
7157.1954443374	0.497135952697992\\
6880.50988850682	0.522382297137214\\
6614.52054705249	0.548895776382219\\
6358.81392169242	0.576742465248064\\
6112.9924993169	0.605991612638171\\
5876.67413402763	0.636715773435574\\
5649.49145306632	0.668990982528916\\
5431.09128570909	0.702896954006255\\
5221.13411423881	0.738517295711468\\
5019.29354614212	0.775939734318028\\
4825.25580671027	0.815256349032912\\
4638.71925125536	0.856506621187406\\
4459.39389618335	0.89968754661657\\
4287.00096819517	0.942768376583987\\
4121.27247091489	0.978571438345879\\
}--cycle;
\addplot[only marks, mark=*, mark size=1.3693pt, color=black, fill=black, opacity=0.60, draw=none, mark options={draw=none,line width=0pt}] table[row sep=crcr]{%
x	y\\
3650	0.99468085106383\\
3735	0.984042553191489\\
3741	0.973404255319149\\
3761	0.962765957446808\\
3772	0.952127659574468\\
3785	0.941489361702128\\
3825	0.930851063829787\\
3847	0.920212765957447\\
3900	0.909574468085106\\
3928	0.898936170212766\\
3945	0.888297872340426\\
3996	0.877659574468085\\
4009	0.867021276595745\\
4024	0.856382978723404\\
4061	0.845744680851064\\
4061	0.835106382978723\\
4071	0.824468085106383\\
4113	0.813829787234043\\
4228	0.803191489361702\\
4348	0.792553191489362\\
4430	0.781914893617021\\
4503	0.771276595744681\\
4539	0.76063829787234\\
4632	0.75\\
4654	0.73936170212766\\
4663	0.728723404255319\\
4679	0.718085106382979\\
4702	0.707446808510638\\
4779	0.696808510638298\\
4861	0.686170212765957\\
4953	0.675531914893617\\
5029	0.664893617021277\\
5048	0.654255319148936\\
5236	0.643617021276596\\
5264	0.632978723404255\\
5327	0.622340425531915\\
5379	0.611702127659574\\
5458	0.601063829787234\\
5463	0.590425531914894\\
5524	0.579787234042553\\
5607	0.569148936170213\\
5857	0.558510638297872\\
5868	0.547872340425532\\
5982	0.537234042553192\\
6317	0.526595744680851\\
6332	0.515957446808511\\
6342	0.50531914893617\\
6356	0.49468085106383\\
6367	0.484042553191489\\
6468	0.473404255319149\\
6629	0.462765957446808\\
6926	0.452127659574468\\
7252	0.441489361702128\\
7294	0.430851063829787\\
7320	0.420212765957447\\
7348	0.409574468085106\\
7596	0.398936170212766\\
7724	0.388297872340426\\
7877	0.377659574468085\\
8035	0.367021276595745\\
8499	0.356382978723404\\
8518	0.345744680851064\\
8994	0.335106382978723\\
9128	0.324468085106383\\
9267	0.313829787234043\\
9435	0.303191489361702\\
9599	0.292553191489362\\
9837	0.281914893617021\\
10119	0.271276595744681\\
10336	0.26063829787234\\
10377	0.25\\
11048	0.23936170212766\\
11424	0.228723404255319\\
11830	0.218085106382979\\
12182	0.207446808510638\\
12313	0.196808510638298\\
13086	0.186170212765957\\
13380	0.175531914893617\\
13398	0.164893617021277\\
13899	0.154255319148936\\
14984	0.143617021276596\\
16428	0.132978723404255\\
20491	0.122340425531915\\
21264	0.111702127659574\\
23237	0.101063829787234\\
30134	0.0904255319148936\\
30990	0.0797872340425532\\
32887	0.0691489361702128\\
37527	0.0585106382978723\\
39026	0.0478723404255319\\
41095	0.0372340425531915\\
61636	0.0265957446808511\\
122339	0.0159574468085106\\
204237	0.00531914893617021\\
};
\addlegendentry{FRC empirica reale}

\addplot [color=black]
table[row sep=crcr]{%
3637.5	1\\
3788.54993935839	0.949537816810858\\
3945.87234172165	0.901622065553929\\
4109.72767586131	0.856124247714574\\
4280.38722671172	0.812922349093734\\
4458.13354451935	0.771900512595218\\
4643.26091264341	0.732948727524845\\
4836.07583478224	0.695962534568237\\
5036.89754243213	0.660842745656075\\
5246.05852341874	0.627495177965562\\
5463.90507237627	0.59583040134476\\
5690.79786408554	0.565763498482441\\
5927.11255062052	0.53721383718029\\
6173.24038329176	0.510104854116755\\
6429.58886041643	0.484363849522645\\
6696.58240198765	0.459921792217835\\
6974.66305235981	0.436713134486259\\
7264.29121211354	0.414675636292709\\
7565.94640031188	0.393750198370032\\
7880.12804840974	0.373880703729123\\
8207.35632713103	0.355013867166658\\
8548.17300768248	0.337099092367008\\
8903.14235873003	0.32008833621509\\
9272.85208062291	0.303935979956297\\
9657.9142784119	0.28859870685797\\
10058.9664752731	0.274035386044354\\
10476.6726680149	0.260206962193476\\
10911.7244264153	0.247076350800179\\
11364.8420382106	0.234608338724395\\
11836.7757016304	0.222769489757984\\
12328.3067674531	0.211528054956865\\
12840.2490326395	0.200853887497989\\
13373.4500876847	0.190718361832814\\
13928.7927199204	0.181094296920473\\
14507.19637509	0.171955883334763\\
15109.6186796172	0.163278614049474\\
15737.0570260872	0.15503921871644\\
16390.550224567	0.147215601260069\\
17071.1802224973	0.139786780620984\\
17780.0738960051	0.132732834489867\\
18518.4049156008	0.126034845880625\\
19287.3956893508	0.119674852399582\\
20088.3193867412	0.113635798074661\\
20922.5020465844	0.107901487615373\\
21791.3247724572	0.102456542980945\\
22696.2260193077	0.0972863621401141\\
23638.7039750138	0.0923770799119943\\
24620.3190408384	0.0877155307829971\\
25642.6964148873	0.0832892136000928\\
26707.5287828473	0.0790862580457252\\
27816.5791204588	0.0750953928044779\\
28971.6836123634	0.0713059153361178\\
30174.7546921595	0.0677076631739571\\
31427.7842086969	0.0642909866715642\\
32732.8467238564	0.061046723124733\\
34092.1029472698	0.0579661721993158\\
35507.8033136712	0.0550410725990206\\
36982.2917087997	0.0522635799106019\\
38518.0093500227	0.0496262455670326\\
40117.4988281057	0.0471219968722397\\
41783.4083168193	0.0447441180338345\\
43518.4959573534	0.0424862321529745\\
45325.6344247971	0.0403422841230547\\
47207.8156842464	0.0383065243913687\\
49168.1559444106	0.0363734935401921\\
51209.900816924	0.0345380076459378\\
53336.4306898988	0.0327951443771205\\
55551.2663246212	0.0311402297938478\\
57858.074684653	0.0295688258134387\\
60260.6750069927	0.0280767183085531\\
62763.0451253437	0.026659905805917\\
65369.3280559641	0.0253145887553335\\
68083.8388569957	0.024037159340204\\
70911.0717726344	0.022824191802232\\
73855.7076739664	0.0216724332543637\\
76922.6218087908	0.0205787949573275\\
80116.89187326	0.0195403440363791\\
83443.8064187003	0.0185542956160364\\
86908.8736075329	0.0176180053517145\\
90517.8303327903	0.016728962338229\\
94276.651716329	0.015884782376153\\
98191.5610014598	0.0150832015779679\\
102269.039856381	0.0143220702968617\\
106515.839105467	0.0135993473618937\\
110938.989906178	0.012913094604065\\
115545.815390112	0.012261471658616\\
120343.942787444	0.0116427310296104\\
125341.316054851	0.0110552134035723\\
130546.209027823	0.0104973431996062\\
135967.239119128	0.00996762434406833\\
141613.381586117	0.00946463625845741\\
147493.984390494	0.00898703004976452\\
153618.783675143	0.00853352489306699\\
159997.919883649	0.00810290459666392\\
166641.954549187	0.00769401434054292\\
173561.887780587	0.00730575757943056\\
180769.176474522	0.00693709310212186\\
188275.753283964	0.00658703223920245\\
196094.046374327	0.00625463621167503\\
204237	0.00593901361338004\\
};
\addlegendentry{Adatt. Pareto reale (\textit{ML})}
    \end{groupplot}

    \begin{groupplot}[group style=plotGroup2]
        \nextgroupplot[%
            plotColumn1,plotLegend3,
            at={($(Group1 c1r4.south west)-(0,\yPlotSepII em)-(0,\plotHeight cm)$)},
            xmin=0,xmax=10000,
            ymin=4100,ymax=4800,
            xlabel={\(t\)},ylabel={\(\langle s\rangle\)},
        ] 
% !TEX root = ../../../../Esperimenti/.tex/MEF.tex

% This file was created by matlab2tikz.
%
\definecolor{mycolor1}{rgb}{0.00000,0.44706,0.69804}%
\definecolor{mycolor2}{rgb}{0.83529,0.36863,0.00000}%

\addplot [color=mycolor1]
table[row sep=crcr]{%
0	4395.32799999998\\
200	4397.81480380004\\
400	4398.59154339039\\
600	4386.5534186326\\
800	4395.77440176183\\
1000	4413.07349685022\\
1200	4414.51206573803\\
1400	4416.21374450608\\
1600	4407.3728462828\\
1800	4408.60823632873\\
2000	4415.3856716495\\
2200	4415.5550503352\\
2400	4427.68066504751\\
2600	4441.59776196881\\
2800	4442.26315388597\\
3000	4443.65444446686\\
3200	4450.54554252936\\
3400	4439.70459624788\\
3600	4431.89675498212\\
3800	4435.52340582072\\
4000	4439.81531473571\\
4200	4463.79544605168\\
4400	4473.96888834266\\
4600	4474.10946153821\\
4800	4470.82552437785\\
5000	4470.38321411354\\
5200	4467.25530540575\\
5400	4463.44730277309\\
5600	4467.21633015159\\
5800	4465.06869141576\\
6000	4464.03054637483\\
6200	4457.85247196245\\
6400	4433.19737202102\\
6600	4444.38584467653\\
6800	4449.70810713132\\
7000	4454.564242123\\
7200	4466.78714575986\\
7400	4463.54400356082\\
7600	4457.20069746067\\
7800	4456.26763364014\\
8000	4466.67282781017\\
8200	4485.98247581196\\
8400	4484.01231307893\\
8600	4491.8795654758\\
8800	4490.37103478397\\
9000	4499.09500562178\\
9200	4509.71743686936\\
9400	4503.81980864499\\
9600	4511.49459777515\\
9800	4516.89806047319\\
10000	4520.59938481386\\
};
\addlegendentry{Taglia media esatta}


\addplot[area legend, draw=none, fill=mycolor1, fill opacity=0.15, forget plot]
table[row sep=crcr] {%
x	y\\
0	4395.32799999998\\
200	4403.1609677162\\
400	4411.1545944306\\
600	4403.97563615797\\
800	4419.52859634565\\
1000	4445.77596484198\\
1200	4454.22808535139\\
1400	4464.01011989168\\
1600	4460.6407359315\\
1800	4468.87003204591\\
2000	4480.81398641477\\
2200	4481.88799321949\\
2400	4496.57761094044\\
2600	4514.76649250219\\
2800	4517.83984469743\\
3000	4523.04203791725\\
3200	4533.76584740167\\
3400	4526.169771118\\
3600	4519.34751182437\\
3800	4525.13733700006\\
4000	4529.43341905219\\
4200	4558.79119378858\\
4400	4574.03885726873\\
4600	4580.119138888\\
4800	4583.27601235343\\
5000	4585.08365052708\\
5200	4582.69804800546\\
5400	4581.0244255695\\
5600	4588.7435748833\\
5800	4590.05442880285\\
6000	4591.39130262885\\
6200	4589.509016984\\
6400	4567.97216342918\\
6600	4585.5171266754\\
6800	4592.8172861822\\
7000	4601.41080567684\\
7200	4617.61204459318\\
7400	4619.79301619542\\
7600	4614.75345957821\\
7800	4616.8416452301\\
8000	4628.05782817275\\
8200	4653.63335617668\\
8400	4650.72192730257\\
8600	4657.71556186245\\
8800	4658.82275813608\\
9000	4671.99923783209\\
9200	4689.60198546443\\
9400	4685.0372998761\\
9600	4693.26659860731\\
9800	4702.27723078616\\
10000	4707.82282516017\\
10000	4333.37594446754\\
9800	4331.51889016022\\
9600	4329.722596943\\
9400	4322.60231741389\\
9200	4329.8328882743\\
9000	4326.19077341147\\
8800	4321.91931143185\\
8600	4326.04356908915\\
8400	4317.30269885529\\
8200	4318.33159544725\\
8000	4305.28782744758\\
7800	4295.69362205018\\
7600	4299.64793534313\\
7400	4307.29499092623\\
7200	4315.96224692655\\
7000	4307.71767856917\\
6800	4306.59892808043\\
6600	4303.25456267765\\
6400	4298.42258061285\\
6200	4326.19592694089\\
6000	4336.66979012081\\
5800	4340.08295402866\\
5600	4345.68908541987\\
5400	4345.87017997668\\
5200	4351.81256280605\\
5000	4355.68277769999\\
4800	4358.37503640228\\
4600	4368.09978418843\\
4400	4373.8989194166\\
4200	4368.79969831479\\
4000	4350.19721041923\\
3800	4345.90947464139\\
3600	4344.44599813987\\
3400	4353.23942137777\\
3200	4367.32523765705\\
3000	4364.26685101647\\
2800	4366.68646307452\\
2600	4368.42903143544\\
2400	4358.78371915458\\
2200	4349.22210745091\\
2000	4349.95735688423\\
1800	4348.34644061156\\
1600	4354.10495663411\\
1400	4368.41736912047\\
1200	4374.79604612467\\
1000	4380.37102885845\\
800	4372.02020717801\\
600	4369.13120110722\\
400	4386.02849235018\\
200	4392.46863988389\\
0	4395.32799999998\\
}--cycle;
\addplot [color=mycolor2]
table[row sep=crcr]{%
0	4395.32799999998\\
200	4393.35357108239\\
400	4390.39178272287\\
600	4388.08712871615\\
800	4388.24035565726\\
1000	4380.39937006586\\
1200	4363.88013015198\\
1400	4356.91008961955\\
1600	4358.37266537895\\
1800	4357.67502854488\\
2000	4356.25705531124\\
2200	4357.09546240025\\
2400	4367.69928025848\\
2600	4368.2580485144\\
2800	4350.2826010156\\
3000	4336.82534866117\\
3200	4352.41576588186\\
3400	4367.6953290692\\
3600	4355.19349353067\\
3800	4347.93191382238\\
4000	4346.97920917123\\
4200	4348.67300610494\\
4400	4365.38482034145\\
4600	4369.83281969643\\
4800	4368.46886442797\\
5000	4380.98568526529\\
5200	4385.12190701182\\
5400	4385.65042981328\\
5600	4397.4211931682\\
5800	4396.82940005403\\
6000	4382.68736030867\\
6200	4373.44845644619\\
6400	4375.66905730304\\
6600	4369.10212657111\\
6800	4357.93898845775\\
7000	4358.42975600171\\
7200	4346.91302887737\\
7400	4335.92035068224\\
7600	4337.98321122892\\
7800	4363.68445644914\\
8000	4372.60859536783\\
8200	4368.26623391021\\
8400	4358.28434975177\\
8600	4337.49552520189\\
8800	4332.53605487679\\
9000	4326.86430049632\\
9200	4303.1316526372\\
9400	4298.49465876291\\
9600	4294.51637995403\\
9800	4297.22790588663\\
10000	4308.16842389739\\
};
\addlegendentry{Taglia media appr.}


\addplot[area legend, draw=none, fill=mycolor2, fill opacity=0.15, forget plot]
table[row sep=crcr] {%
x	y\\
0	4395.32799999998\\
200	4398.28433995475\\
400	4401.73016976747\\
600	4408.63581522444\\
800	4416.84297025615\\
1000	4420.45415172743\\
1200	4408.19572010923\\
1400	4409.28844673208\\
1600	4414.46307430171\\
1800	4418.24509045573\\
2000	4423.20735373223\\
2200	4430.54666385923\\
2400	4448.18433429311\\
2600	4449.72389015849\\
2800	4433.9812085517\\
3000	4424.29748044106\\
3200	4444.4026106397\\
3400	4463.22392095664\\
3600	4453.39763442513\\
3800	4451.7416228604\\
4000	4454.74895524689\\
4200	4457.44181961677\\
4400	4480.43611372995\\
4600	4488.19060018582\\
4800	4490.12721954834\\
5000	4507.14426506977\\
5200	4516.31726382548\\
5400	4523.62512038082\\
5600	4539.82662605082\\
5800	4541.85811814064\\
6000	4528.59789987939\\
6200	4517.84727911194\\
6400	4522.76989972311\\
6600	4516.10933016534\\
6800	4505.61475961175\\
7000	4506.64040933018\\
7200	4496.9284016303\\
7400	4489.7174930784\\
7600	4491.64676884636\\
7800	4520.04428137881\\
8000	4530.97609129997\\
8200	4529.40591185119\\
8400	4519.49503691801\\
8600	4495.27534950596\\
8800	4489.76556346238\\
9000	4488.37804521118\\
9200	4464.16961690506\\
9400	4457.16659883004\\
9600	4457.46338940466\\
9800	4463.86473599028\\
10000	4479.37004977918\\
10000	4136.96679801561\\
9800	4130.59107578299\\
9600	4131.56937050339\\
9400	4139.82271869579\\
9200	4142.09368836933\\
9000	4165.35055578146\\
8800	4175.3065462912\\
8600	4179.71570089781\\
8400	4197.07366258554\\
8200	4207.12655596923\\
8000	4214.2410994357\\
7800	4207.32463151947\\
7600	4184.31965361148\\
7400	4182.12320828608\\
7200	4196.89765612444\\
7000	4210.21910267325\\
6800	4210.26321730376\\
6600	4222.09492297688\\
6400	4228.56821488298\\
6200	4229.04963378045\\
6000	4236.77682073794\\
5800	4251.80068196742\\
5600	4255.01576028558\\
5400	4247.67573924574\\
5200	4253.92655019816\\
5000	4254.82710546081\\
4800	4246.81050930759\\
4600	4251.47503920704\\
4400	4250.33352695295\\
4200	4239.90419259312\\
4000	4239.20946309558\\
3800	4244.12220478436\\
3600	4256.9893526362\\
3400	4272.16673718176\\
3200	4260.42892112403\\
3000	4249.35321688128\\
2800	4266.5839934795\\
2600	4286.79220687032\\
2400	4287.21422622385\\
2200	4283.64426094127\\
2000	4289.30675689025\\
1800	4297.10496663402\\
1600	4302.2822564562\\
1400	4304.53173250701\\
1200	4319.56454019474\\
1000	4340.34458840429\\
800	4359.63774105838\\
600	4367.53844220785\\
400	4379.05339567826\\
200	4388.42280221004\\
0	4395.32799999998\\
}--cycle;
\addplot [color=black]
table[row sep=crcr]{%
0	4395.328\\
200	4395.328\\
400	4395.328\\
600	4395.328\\
800	4395.328\\
1000	4395.328\\
1200	4395.328\\
1400	4395.328\\
1600	4395.328\\
1800	4395.328\\
2000	4395.328\\
2200	4395.328\\
2400	4395.328\\
2600	4395.328\\
2800	4395.328\\
3000	4395.328\\
3200	4395.328\\
3400	4395.328\\
3600	4395.328\\
3800	4395.328\\
4000	4395.328\\
4200	4395.328\\
4400	4395.328\\
4600	4395.328\\
4800	4395.328\\
5000	4395.328\\
5200	4395.328\\
5400	4395.328\\
5600	4395.328\\
5800	4395.328\\
6000	4395.328\\
6200	4395.328\\
6400	4395.328\\
6600	4395.328\\
6800	4395.328\\
7000	4395.328\\
7200	4395.328\\
7400	4395.328\\
7600	4395.328\\
7800	4395.328\\
8000	4395.328\\
8200	4395.328\\
8400	4395.328\\
8600	4395.328\\
8800	4395.328\\
9000	4395.328\\
9200	4395.328\\
9400	4395.328\\
9600	4395.328\\
9800	4395.328\\
10000	4395.328\\
};
\addlegendentry{Taglia media reale}

        \nextgroupplot[%
            plotColumn2,plotLegend3,
            xmin=0,xmax=10000,
            ymin=4395.3279999995,ymax=4395.3280000005,
            xlabel={\(t\)},ylabel={\(\langle s\rangle\)},
        ] 
% !TEX root = ../../../../Esperimenti/.tex/MEF.tex

% This file was created by matlab2tikz.
%
\definecolor{mycolor1}{rgb}{0.00000,0.44706,0.69804}%
\definecolor{mycolor2}{rgb}{0.83529,0.36863,0.00000}%

\addplot [color=mycolor1]
table[row sep=crcr]{%
0	4395.32799999998\\
200	4395.32799999999\\
400	4395.32799999999\\
600	4395.32799999999\\
800	4395.32799999999\\
1000	4395.32799999999\\
1200	4395.32799999999\\
1400	4395.32799999999\\
1600	4395.32799999999\\
1800	4395.32799999999\\
2000	4395.32799999999\\
2200	4395.32799999999\\
2400	4395.32799999999\\
2600	4395.32799999999\\
2800	4395.32799999999\\
3000	4395.32799999999\\
3200	4395.32799999999\\
3400	4395.32799999999\\
3600	4395.32799999999\\
3800	4395.32799999999\\
4000	4395.32799999999\\
4200	4395.32799999999\\
4400	4395.32799999999\\
4600	4395.32799999999\\
4800	4395.32799999999\\
5000	4395.32799999999\\
5200	4395.32799999999\\
5400	4395.32799999999\\
5600	4395.32799999999\\
5800	4395.32799999999\\
6000	4395.32799999999\\
6200	4395.32799999999\\
6400	4395.32799999999\\
6600	4395.32799999999\\
6800	4395.32799999999\\
7000	4395.32799999999\\
7200	4395.32799999999\\
7400	4395.32799999999\\
7600	4395.32799999999\\
7800	4395.32799999999\\
8000	4395.32799999999\\
8200	4395.32799999999\\
8400	4395.32799999999\\
8600	4395.32799999999\\
8800	4395.32799999999\\
9000	4395.32799999999\\
9200	4395.32799999999\\
9400	4395.32799999999\\
9600	4395.32799999999\\
9800	4395.32799999999\\
10000	4395.32799999999\\
};
\addlegendentry{Taglia media esatta}


\addplot[area legend, draw=none, fill=mycolor1, fill opacity=0.15, forget plot]
table[row sep=crcr] {%
x	y\\
0	4395.32799999998\\
200	4395.32799999999\\
400	4395.32799999999\\
600	4395.32799999999\\
800	4395.32799999999\\
1000	4395.32799999999\\
1200	4395.32799999999\\
1400	4395.32799999999\\
1600	4395.32799999999\\
1800	4395.32799999999\\
2000	4395.32799999999\\
2200	4395.32799999999\\
2400	4395.32799999999\\
2600	4395.32799999999\\
2800	4395.32799999999\\
3000	4395.32799999999\\
3200	4395.32799999999\\
3400	4395.32799999999\\
3600	4395.32799999999\\
3800	4395.32799999999\\
4000	4395.32799999999\\
4200	4395.32799999999\\
4400	4395.32799999999\\
4600	4395.32799999999\\
4800	4395.32799999999\\
5000	4395.32799999999\\
5200	4395.32799999999\\
5400	4395.32799999999\\
5600	4395.32799999999\\
5800	4395.32799999999\\
6000	4395.32799999999\\
6200	4395.32799999999\\
6400	4395.32799999999\\
6600	4395.32799999999\\
6800	4395.32799999999\\
7000	4395.32799999999\\
7200	4395.32799999999\\
7400	4395.32799999999\\
7600	4395.32799999999\\
7800	4395.32799999999\\
8000	4395.32799999999\\
8200	4395.32799999999\\
8400	4395.32799999999\\
8600	4395.32799999999\\
8800	4395.32799999999\\
9000	4395.32799999999\\
9200	4395.32799999999\\
9400	4395.32799999999\\
9600	4395.32799999999\\
9800	4395.32799999999\\
10000	4395.32799999999\\
10000	4395.32799999999\\
9800	4395.32799999998\\
9600	4395.32799999999\\
9400	4395.32799999999\\
9200	4395.32799999999\\
9000	4395.32799999999\\
8800	4395.32799999999\\
8600	4395.32799999999\\
8400	4395.32799999999\\
8200	4395.32799999999\\
8000	4395.32799999999\\
7800	4395.32799999999\\
7600	4395.32799999999\\
7400	4395.32799999999\\
7200	4395.32799999999\\
7000	4395.32799999999\\
6800	4395.32799999999\\
6600	4395.32799999999\\
6400	4395.32799999999\\
6200	4395.32799999999\\
6000	4395.32799999999\\
5800	4395.32799999999\\
5600	4395.32799999999\\
5400	4395.32799999999\\
5200	4395.32799999999\\
5000	4395.32799999999\\
4800	4395.32799999999\\
4600	4395.32799999999\\
4400	4395.32799999999\\
4200	4395.32799999999\\
4000	4395.32799999999\\
3800	4395.32799999999\\
3600	4395.32799999999\\
3400	4395.32799999999\\
3200	4395.32799999999\\
3000	4395.32799999999\\
2800	4395.32799999999\\
2600	4395.32799999999\\
2400	4395.32799999999\\
2200	4395.32799999999\\
2000	4395.32799999999\\
1800	4395.32799999999\\
1600	4395.32799999999\\
1400	4395.32799999999\\
1200	4395.32799999999\\
1000	4395.32799999999\\
800	4395.32799999999\\
600	4395.32799999999\\
400	4395.32799999999\\
200	4395.32799999999\\
0	4395.32799999998\\
}--cycle;
\addplot [color=mycolor2]
table[row sep=crcr]{%
0	4395.32799999998\\
200	4395.32799999999\\
400	4395.32799999999\\
600	4395.32799999999\\
800	4395.32799999999\\
1000	4395.32799999999\\
1200	4395.32799999999\\
1400	4395.32799999999\\
1600	4395.32799999999\\
1800	4395.32799999999\\
2000	4395.32799999999\\
2200	4395.32799999999\\
2400	4395.32799999999\\
2600	4395.32799999999\\
2800	4395.32799999999\\
3000	4395.32799999999\\
3200	4395.32799999999\\
3400	4395.32799999999\\
3600	4395.32799999999\\
3800	4395.32799999999\\
4000	4395.32799999999\\
4200	4395.32799999999\\
4400	4395.32799999999\\
4600	4395.32799999999\\
4800	4395.32799999999\\
5000	4395.32799999999\\
5200	4395.32799999999\\
5400	4395.32799999999\\
5600	4395.32799999999\\
5800	4395.32799999999\\
6000	4395.32799999999\\
6200	4395.32799999999\\
6400	4395.32799999999\\
6600	4395.32799999999\\
6800	4395.32799999999\\
7000	4395.32799999999\\
7200	4395.32799999999\\
7400	4395.32799999999\\
7600	4395.32799999999\\
7800	4395.32799999999\\
8000	4395.32799999999\\
8200	4395.32799999999\\
8400	4395.32799999999\\
8600	4395.32799999999\\
8800	4395.32799999999\\
9000	4395.32799999999\\
9200	4395.32799999999\\
9400	4395.32799999999\\
9600	4395.32799999999\\
9800	4395.32799999999\\
10000	4395.32799999999\\
};
\addlegendentry{Taglia media appr.}


\addplot[area legend, draw=none, fill=mycolor2, fill opacity=0.15, forget plot]
table[row sep=crcr] {%
x	y\\
0	4395.32799999998\\
200	4395.32799999999\\
400	4395.32799999999\\
600	4395.32799999999\\
800	4395.32799999999\\
1000	4395.32799999999\\
1200	4395.32799999999\\
1400	4395.32799999999\\
1600	4395.32799999999\\
1800	4395.32799999999\\
2000	4395.32799999999\\
2200	4395.32799999999\\
2400	4395.32799999999\\
2600	4395.32799999999\\
2800	4395.32799999999\\
3000	4395.32799999999\\
3200	4395.32799999999\\
3400	4395.32799999999\\
3600	4395.32799999999\\
3800	4395.32799999999\\
4000	4395.32799999999\\
4200	4395.32799999999\\
4400	4395.32799999999\\
4600	4395.32799999999\\
4800	4395.32799999999\\
5000	4395.32799999999\\
5200	4395.32799999999\\
5400	4395.32799999999\\
5600	4395.32799999999\\
5800	4395.32799999999\\
6000	4395.32799999999\\
6200	4395.32799999999\\
6400	4395.32799999999\\
6600	4395.32799999999\\
6800	4395.32799999999\\
7000	4395.32799999999\\
7200	4395.32799999999\\
7400	4395.32799999999\\
7600	4395.32799999999\\
7800	4395.32799999999\\
8000	4395.32799999999\\
8200	4395.32799999999\\
8400	4395.32799999999\\
8600	4395.32799999999\\
8800	4395.32799999999\\
9000	4395.32799999999\\
9200	4395.32799999999\\
9400	4395.32799999999\\
9600	4395.32799999999\\
9800	4395.32799999999\\
10000	4395.32799999999\\
10000	4395.32799999999\\
9800	4395.32799999999\\
9600	4395.32799999999\\
9400	4395.32799999999\\
9200	4395.32799999999\\
9000	4395.32799999999\\
8800	4395.32799999999\\
8600	4395.32799999999\\
8400	4395.32799999999\\
8200	4395.32799999999\\
8000	4395.32799999999\\
7800	4395.32799999999\\
7600	4395.32799999999\\
7400	4395.32799999999\\
7200	4395.32799999999\\
7000	4395.32799999999\\
6800	4395.32799999999\\
6600	4395.32799999999\\
6400	4395.32799999999\\
6200	4395.32799999999\\
6000	4395.32799999999\\
5800	4395.32799999999\\
5600	4395.32799999999\\
5400	4395.32799999999\\
5200	4395.32799999999\\
5000	4395.32799999999\\
4800	4395.32799999999\\
4600	4395.32799999999\\
4400	4395.32799999999\\
4200	4395.32799999999\\
4000	4395.32799999999\\
3800	4395.32799999999\\
3600	4395.32799999999\\
3400	4395.32799999999\\
3200	4395.32799999999\\
3000	4395.32799999999\\
2800	4395.32799999999\\
2600	4395.32799999999\\
2400	4395.32799999999\\
2200	4395.32799999999\\
2000	4395.32799999999\\
1800	4395.32799999999\\
1600	4395.32799999999\\
1400	4395.32799999999\\
1200	4395.32799999999\\
1000	4395.32799999999\\
800	4395.32799999999\\
600	4395.32799999999\\
400	4395.32799999999\\
200	4395.32799999999\\
0	4395.32799999998\\
}--cycle;
\addplot [color=black]
table[row sep=crcr]{%
0	4395.328\\
200	4395.328\\
400	4395.328\\
600	4395.328\\
800	4395.328\\
1000	4395.328\\
1200	4395.328\\
1400	4395.328\\
1600	4395.328\\
1800	4395.328\\
2000	4395.328\\
2200	4395.328\\
2400	4395.328\\
2600	4395.328\\
2800	4395.328\\
3000	4395.328\\
3200	4395.328\\
3400	4395.328\\
3600	4395.328\\
3800	4395.328\\
4000	4395.328\\
4200	4395.328\\
4400	4395.328\\
4600	4395.328\\
4800	4395.328\\
5000	4395.328\\
5200	4395.328\\
5400	4395.328\\
5600	4395.328\\
5800	4395.328\\
6000	4395.328\\
6200	4395.328\\
6400	4395.328\\
6600	4395.328\\
6800	4395.328\\
7000	4395.328\\
7200	4395.328\\
7400	4395.328\\
7600	4395.328\\
7800	4395.328\\
8000	4395.328\\
8200	4395.328\\
8400	4395.328\\
8600	4395.328\\
8800	4395.328\\
9000	4395.328\\
9200	4395.328\\
9400	4395.328\\
9600	4395.328\\
9800	4395.328\\
10000	4395.328\\
};
\addlegendentry{Taglia media reale}
    \end{groupplot}

    %\begingroup Didascalie
        \tikzset{
            Titolo/.style={
                text width=\plotWidth cm,
                align=center,
                % anchor=center,
                anchor=south,
                font=\dimTesto{\titleSize}\bfseries,
            }
        }
        \node[Titolo] at (Group1 c1r1.north) {Con fluttuazioni \(\gamma\)};
        \node[Titolo] at (Group1 c2r1.north) {Senza fluttuazioni \(\gamma\)};

        \tikzset{
            SubCaption/.style={
                % text width=\plotWidth cm,
                align=center,
                font=\dimTesto{\subCaptionSize},
            },
            SubCaption1/.style={SubCaption,anchor=south east},
            SubCaption2/.style={SubCaption,anchor=north east},
            SubCaption3/.style={SubCaption,anchor=south west},
        }

        \node[SubCaption1] at (Group1 c1r1.south east) {\sottoDidascalia[]{}};
        \node[SubCaption1] at (Group1 c2r1.south east) {\sottoDidascalia[]{}};
        \node[SubCaption1] at (Group1 c1r2.south east) {\sottoDidascalia[]{}};
        \node[SubCaption1] at (Group1 c2r2.south east) {\sottoDidascalia[]{}};
    
        \node[SubCaption2] at (Group1 c1r3.north east) {\sottoDidascalia[]{}};
        \node[SubCaption2] at (Group1 c2r3.north east) {\sottoDidascalia[]{}};
        \node[SubCaption2] at (Group1 c1r4.north east) {\sottoDidascalia[]{}};
        \node[SubCaption2] at (Group1 c2r4.north east) {\sottoDidascalia[]{}};

        \node[SubCaption3] at (Group2 c1r1.south west) {\sottoDidascalia[]{}};
        \node[SubCaption3] at (Group2 c2r1.south west) {\sottoDidascalia[]{}};
    
        % https://tex.stackexchange.com/questions/632045/get-subcaption-for-every-plot-using-groupplot-pgf-tikz-after-2020-update
    %\endgroup

    \redefineTikZbounds{Group1 c1r1}{Group2 c2r1}
\end{tikzpicture}%
}
                {Confronto di una simulazione con e senza fluttuazioni; la regola d'emigrazione è la \cref{eqVarianteRegolaTagliaForzaFrazionata} mentre i parametri sono illustrati nella \cref{tabParametriConEsenzaGamma}.}
                {.6\linewidth}{figConfrontoConESenzaFluttuazioni}

    \section{Regole d'emigrazione}\label{secRegoleEmigrazione}

        In questo paragrafo si specializza la regola d'emigrazione \(E\) definita nelle \ref{eqRegoleInterazioneUrbaneEsplicite} per completare le \ref{eqRegoleInterazioneUrbane}; tutte le forme qui presenti condividono comunque tre aspetti: sono simmetriche, non lineari e associano i microstati \((i,s)\) e \((i_*,s_*)\) rispettivamente alla città interagente e ricevente.

        Per brevità si limita lo studio a venire su tre aspetti:

        \begin{enumerate}[
            label=\arabic*.,
            %topsep=0.5em,
            %parsep=0em,
            %itemsep=0.25em,
            %leftmargin=2em,
            %rightmargin=1.5em,
            % \leftmargin + \itemindent = \labelindent + \labelwidth + \labelsep
            %itemindent=!,
            %labelindent=3em,
            %labelwidth=!,
            %labelsep=!,
        ]
            \item si considera solo la regione della Sardegna sia perché molti risultati di \cite{DeMontis2007} sono già stati riprodotti nel \cref{secCenniSuiDati}, sia poiché per le altre 19 regioni sono stati trovati risultati analoghi;
            \item si analizza esclusivamente la configurazione che meglio fitta i dati reali, almeno tra quelle esplorate dall'autore, e
            \item si approfondiscono, vista la loro lunghezza, due soli studi parametrici per l'ultima regola d'emigrazione proposta.
        \end{enumerate}
       
        Pertanto, data l'importanza della Sardegna, si elencano nella \cref{tabDatiParametriSardegna} i dati principali della regione assieme ai parametri adattati dell'indice di Pareto e della lognormale bimodale.

        \begin{table}[htb]
            \centering
            \begin{tabular}{cccc}
                \hline
                \(N\)&\(P\)&\(\beta\)&\(\xi\)\\
                \hline
                \(\num{375}\)&\(\approx\num{1648248}\)&\(\num{1.27}\)&\(\num{0.93}\)\\
                \hline
            \end{tabular}
            \caption{Dati e parametri della Sardegna.}
            \label{tabDatiParametriSardegna}
        \end{table}

        Cionnonostante, nei \cref{secCasoItalia,secValleDAosta} si studiano comunque la Valle d'Aosta e l'Italia essendo casi estremi rispetto al numero di nodi.

        \subsection{Regola taglia}

            Una prima possibilità consiste nella
            %
            \eqtag{\([\text{RE}]_T\)}{
                \label{eqLeggeEmigrazioneTaglia}
                E(s,s_*)\equiv\lambda\frac{(s_*/s)^\alpha}{1+(s_*/s)^\alpha},
            }%
            in cui \(\lambda\in(0,1)\) e \(\alpha\in\mathbb R^+\); in essenza è la \cite[(2.2), § 2, p. 223]{Gualandi2019SDC} modificata mediante la \cite[(4.5), § 4, p. 228]{Gualandi2019SDC}, ossia è una funzione di Hill di ordine \(\alpha\), in cui v'è un tasso di emigrazione maggiore verso città con popolazione relativa, data dal rapporto \(s_*/s\), maggiore.

            % \tikzfigure{p}
            % {
% !TEX root = ../../../../Esperimenti/.tex/MEF.tex
% LTeX: language=it

\begin{tikzpicture}%[trim axis left,trim axis right]
    %\begingroup Comandi
        \definecolor{colore}{rgb}{0.50196,0.50196,0.50196}%
    
        % \newcommand\titleSize{12}
        \newcommand\captionSize{12}
        \newcommand\labelSize{12}
        \newcommand\tickSize{12}
        \newcommand\legendSize{11}

        \pgfmathsetmacro\widthPlot{10}
        \pgfmathsetmacro\heightPlot{.6*\widthPlot}
        \pgfmathsetmacro\halfHeightPlot{\heightPlot/2}
        \pgfmathsetmacro\lineWidth{.5pt}
        \pgfmathsetmacro\lineMidWidth{.75pt}
        \pgfmathsetmacro\lineThickWidth{1.2pt}
        
        \pgfmathsetmacro\xPlotSep{2}
        \pgfmathsetmacro\yPlotSep{6}
        \pgfmathsetmacro\yPlotSepPartial{.8*\yPlotSep}
        \pgfmathsetmacro\xLegendSep{.25cm}
    %\endgroup

    %\begingroup Impostazioni pgfplots - Prima colonna
        \pgfplotsset{
            /pgfplots/group/plotGroup/.style={
                group name=Col1,
                group size=1 by 5,
                vertical sep=\yPlotSep em,
                % x descriptions at=edge bottom,
                % y descriptions at=edge left,
            },
            plotColumn/.style={
                width=\widthPlot cm,% 10.45in
                height=\heightPlot cm,% height=6.328in,
                scale only axis, % Fondamentale per avere coerenza tra i due grafici,altrimenti non scalano correttamente
                % xmin=0,xmax=60,
                % xlabel={},
                xlabel style={
                    font=\color{white!15!black}\dimTesto{\labelSize},
                    % at={(ticklabel cs:1.0)},
                    % anchor=east,
                    % yshift=-.3cm,
                },
                % xtick distance=5,
                % ymin=0,
                % ylabel={},
                ylabel style={
                    font=\color{white!15!black}\dimTesto{\labelSize},
                    % at={(ticklabel cs:1.0)},
                    % anchor=east,
                    % yshift=.3cm,
                },
                xticklabel style={font=\dimTesto{\tickSize}},
                yticklabel style={font=\dimTesto{\tickSize}},
                % axis background/.style={fill=white},
                axis x line*=bottom,
                axis y line*=left,
                xmajorgrids,
                ymajorgrids,
                grid style={dashed}
            },
            plotLegend/.style={
                legend cell align=left,
                align=left,
                legend pos=north east,
                % at={(0,.5)},
                % anchor=west,
                xshift=\xLegendSep,
                draw=white!15!black,
                font=\dimTesto{\legendSize},
                % fill=none,
                % draw=none,
            },
        }
    %\endgroup

    \begin{groupplot}[group style=plotGroup,plotColumn]
        % This file was created by matlab2tikz.
%
\nextgroupplot[
    % title style={font=\dimTesto{\titleSize}},
    % title={Prova},
    % width=4.521in,
    % height=3.566in,
    % at={(0.758in,0.481in)},
    scale only axis,
    % xmin=-3.65000000000001,
    xmin=0,
    xmax=300,
    xlabel style={font=\color{white!15!black}},
    xlabel={$k$},
    ymin=0,
    ymax=0.03,
    ylabel style={font=\color{white!15!black}},
    ylabel={$P(k)$},
    axis background/.style={fill=white},
    % axis x line*=bottom,
    % axis y line*=left,
    xmajorgrids,
    ymajorgrids,
    grid style={dashed},
    legend style=plotLegend,
    % legend style={
    %     legend cell align=left,
    %     align=left,
    %     draw=white!15!black
    % },
]
    \addplot[
        ybar interval,
        fill=colore,
        area legend,
        draw=none
    ] table[
        row sep=crcr,
        x=Lower,
        y=Count
    ] {%
    Lower	Upper	Count\\
    10	20.92	0.00927960927960928\\
    20.92	31.84	0.0212454212454212\\
    31.84	42.76	0.0251526251526252\\
    42.76	53.68	0.0151404151404151\\
    53.68	64.6	0.00537240537240538\\
    64.6	75.52	0.00390720390720391\\
    75.52	86.44	0.00268620268620269\\
    86.44	97.36	0.00317460317460317\\
    97.36	108.28	0.00146520146520146\\
    108.28	119.2	0.000976800976800977\\
    119.2	130.12	0.000976800976800977\\
    130.12	141.04	0.000732600732600733\\
    141.04	151.96	0.000244200244200244\\
    151.96	162.88	0.000244200244200244\\
    162.88	173.8	0\\
    173.8	184.72	0.000244200244200244\\
    184.72	195.64	0\\
    195.64	206.56	0.000488400488400488\\
    206.56	217.48	0\\
    217.48	228.4	0\\
    228.4	239.32	0\\
    239.32	250.24	0\\
    250.24	261.16	0\\
    261.16	272.08	0\\
    272.08	283	0.000244200244200244\\
    283	283	0.000244200244200244\\
    };
    \addlegendentry{Istogramma}

    \addplot [color=black]
    table[row sep=crcr]{%
    10	0.00293376181378972\\
    10.5470941883768	0.00355750249447913\\
    11.0941883767535	0.00423360895998364\\
    11.6412825651303	0.00495554107665914\\
    12.188376753507	0.0057161768479886\\
    12.7354709418838	0.00650807002096127\\
    13.2825651302605	0.00732367401393397\\
    13.8296593186373	0.00815553005334939\\
    14.376753507014	0.0089964199162941\\
    14.9238476953908	0.00983948533947425\\
    15.4709418837675	0.0106783171626301\\
    16.0180360721443	0.0115070177933485\\
    16.565130260521	0.0123202407500482\\
    17.1122244488978	0.0131132109715274\\
    17.6593186372745	0.0138817293600731\\
    18.2064128256513	0.0146221647141141\\
    18.7535070140281	0.0153314358514356\\
    19.3006012024048	0.0160069863568422\\
    19.8476953907816	0.0166467540300424\\
    20.3947895791583	0.017249136773729\\
    20.9418837675351	0.017812956355913\\
    21.4889779559118	0.0183374212080772\\
    22.0360721442886	0.018822089182422\\
    22.5831663326653	0.019266830986348\\
    23.1302605210421	0.019671794838215\\
    23.6773547094188	0.0200373727425699\\
    24.2244488977956	0.0203641686624064\\
    24.7715430861723	0.0206529687675336\\
    25.3186372745491	0.0209047138588067\\
    25.8657314629259	0.0211204740050476\\
    26.4128256513026	0.0213014253804247\\
    26.9599198396794	0.0214488292526203\\
    27.5070140280561	0.0215640130442982\\
    28.0541082164329	0.0216483533704807\\
    28.6012024048096	0.0217032609409554\\
    29.1482965931864	0.0217301672085133\\
    29.6953907815631	0.0217305126395896\\
    30.2424849699399	0.021705736482854\\
    30.7895791583166	0.0216572679127372\\
    31.3366733466934	0.0215865184281694\\
    31.8837675350701	0.0214948753914493\\
    32.4308617234469	0.0213836965977531\\
    32.9779559118236	0.0212543057720069\\
    33.5250501002004	0.021107988896423\\
    34.0721442885771	0.0209459912787428\\
    34.6192384769539	0.0207695152779714\\
    35.1663326653307	0.0205797186110216\\
    35.7134268537074	0.020377713170112\\
    36.2605210420842	0.0201645642869268\\
    36.8076152304609	0.019941290385392\\
    37.3547094188377	0.0197088629704415\\
    37.9018036072144	0.0194682069053028\\
    38.4488977955912	0.019220200934636\\
    38.9959919839679	0.018965678415304\\
    39.5430861723447	0.0187054282206517\\
    40.0901803607214	0.0184401957879286\\
    40.6372745490982	0.0181706842819329\\
    41.1843687374749	0.0178975558510833\\
    41.7314629258517	0.0176214329549776\\
    42.2785571142285	0.0173428997450696\\
    42.8256513026052	0.0170625034824291\\
    43.372745490982	0.0167807559786421\\
    43.9198396793587	0.0164981350477976\\
    44.4669338677355	0.0162150859591888\\
    45.0140280561122	0.0159320228818702\\
    45.561122244489	0.015649330313552\\
    46.1082164328657	0.0153673644875122\\
    46.6553106212425	0.0150864547522614\\
    47.2024048096192	0.0148069049196376\\
    47.749498997996	0.0145289945778295\\
    48.2965931863727	0.0142529803665555\\
    48.8436873747495	0.0139790972122597\\
    49.3907815631263	0.0137075595217438\\
    49.937875751503	0.0134385623331319\\
    50.4849699398798	0.0131722824234889\\
    51.0320641282565	0.0129088793727688\\
    51.5791583166333	0.0126484965840788\\
    52.12625250501	0.0123912622605108\\
    52.6733466933868	0.0121372903390099\\
    53.2204408817635	0.0118866813819393\\
    53.7675350701403	0.0116395234271544\\
    54.314629258517	0.0113958927975256\\
    54.8617234468938	0.0111558548709518\\
    55.4088176352705	0.0109194648119883\\
    55.9559118236473	0.0106867682662725\\
    56.503006012024	0.0104578020189804\\
    57.0501002004008	0.0102325946185732\\
    57.5971943887776	0.0100111669671184\\
    58.1442885771543	0.00979353287847373\\
    58.6913827655311	0.00957969960562572\\
    59.2384769539078	0.00936966833846407\\
    59.7855711422846	0.00916343467326103\\
    60.3326653306613	0.00896098905510394\\
    60.8797595190381	0.00876231719450554\\
    61.4268537074148	0.00856740045938896\\
    61.9739478957916	0.00837621624361331\\
    62.5210420841683	0.00818873831317316\\
    63.0681362725451	0.00800493713117022\\
    63.6152304609218	0.00782478016261966\\
    64.1623246492986	0.00764823216011622\\
    64.7094188376753	0.00747525543134844\\
    65.2565130260521	0.00730581008941083\\
    65.8036072144289	0.00713985428682687\\
    66.3507014028056	0.00697734443415777\\
    66.8977955911824	0.0068182354040352\\
    67.4448897795591	0.00666248072141974\\
    67.9919839679359	0.00651003274085098\\
    68.5390781563126	0.0063608428114204\\
    69.0861723446894	0.00621486143016416\\
    69.6332665330661	0.00607203838453997\\
    70.1803607214429	0.00593232288461991\\
    70.7274549098196	0.00579566368560042\\
    71.2745490981964	0.00566200920120061\\
    71.8216432865731	0.00553130760849127\\
    72.3687374749499	0.00540350694466921\\
    72.9158316633267	0.00527855519626493\\
    73.4629258517034	0.00515640038124609\\
    74.0100200400802	0.00503699062445489\\
    74.5571142284569	0.00492027422679375\\
    75.1042084168337	0.00480619972855155\\
    75.6513026052104	0.0046947159672414\\
    76.1983967935872	0.00458577213030005\\
    76.7454909819639	0.0044793178029802\\
    77.2925851703407	0.00437530301174793\\
    77.8396793587174	0.00427367826348035\\
    78.3867735470942	0.00417439458074129\\
    78.9338677354709	0.00407740353339739\\
    79.4809619238477	0.00398265726682154\\
    80.0280561122244	0.00389010852691643\\
    80.5751503006012	0.00379971068217719\\
    81.122244488978	0.00371141774299945\\
    81.6693386773547	0.00362518437842671\\
    82.2164328657315	0.0035409659305193\\
    82.7635270541082	0.00345871842651662\\
    83.310621242485	0.0033783985889534\\
    83.8577154308617	0.0032999638438815\\
    84.4048096192385	0.00322337232733906\\
    84.9519038076152	0.0031485828902004\\
    85.498997995992	0.00307555510153141\\
    86.0460921843687	0.00300424925056784\\
    86.5931863727455	0.00293462634742606\\
    87.1402805611222	0.00286664812264927\\
    87.687374749499	0.00280027702568531\\
    88.2344689378757	0.00273547622238632\\
    88.7815631262525	0.00267220959161437\\
    89.3286573146293	0.00261044172103195\\
    89.875751503006	0.00255013790215083\\
    90.4228456913828	0.00249126412470817\\
    90.9699398797595	0.00243378707043392\\
    91.5170340681363	0.0023776741062694\\
    92.064128256513	0.00232289327709295\\
    92.6112224448898	0.00226941329800448\\
    93.1583166332665	0.00221720354621745\\
    93.7054108216433	0.00216623405260333\\
    94.25250501002	0.00211647549293023\\
    94.7995991983968	0.00206789917883495\\
    95.3466933867736	0.00202047704856411\\
    95.8937875751503	0.00197418165751833\\
    96.4408817635271	0.00192898616863018\\
    96.9879759519038	0.0018848643426047\\
    97.5350701402806	0.00184179052804921\\
    98.0821643286573	0.00179973965151682\\
    98.6292585170341	0.00175868720748637\\
    99.1763527054108	0.00171860924829966\\
    99.7234468937876	0.00167948237407524\\
    100.270541082164	0.00164128372261633\\
    100.817635270541	0.00160399095932926\\
    101.364729458918	0.00156758226716705\\
    101.911823647295	0.00153203633661196\\
    102.458917835671	0.00149733235570921\\
    103.006012024048	0.00146345000016324\\
    103.553106212425	0.00143036942350678\\
    104.100200400802	0.00139807124735194\\
    104.647294589178	0.00136653655173182\\
    105.194388777555	0.00133574686554003\\
    105.741482965932	0.00130568415707505\\
    106.288577154309	0.00127633082469529\\
    106.835671342685	0.00124766968759043\\
    107.382765531062	0.0012196839766735\\
    107.929859719439	0.00119235732559813\\
    108.476953907816	0.00116567376190434\\
    109.024048096192	0.0011396176982961\\
    109.571142284569	0.0011141739240533\\
    110.118236472946	0.0010893275965802\\
    110.665330661323	0.00106506423309242\\
    111.212424849699	0.00104136970244356\\
    111.759519038076	0.00101823021709293\\
    112.306613226453	0.000995632325214829\\
    112.85370741483	0.000973562902950099\\
    113.400801603206	0.000952009146800089\\
    113.947895791583	0.000930958566163025\\
    114.49498997996	0.000910398976012537\\
    115.042084168337	0.000890318489717939\\
    115.589178356713	0.000870705512005592\\
    116.13627254509	0.000851548732060575\\
    116.683366733467	0.0008328371167677\\
    117.230460921844	0.000814559904090769\\
    117.77755511022	0.000796706596588887\\
    118.324649298597	0.00077926695506844\\
    118.871743486974	0.000762230992369359\\
    119.418837675351	0.000745588967284099\\
    119.965931863727	0.000729331378607736\\
    120.513026052104	0.000713448959317502\\
    121.060120240481	0.000697932670880018\\
    121.607214428858	0.000682773697684388\\
    122.154308617234	0.00066796344159936\\
    122.701402805611	0.000653493516652608\\
    123.248496993988	0.000639355743830246\\
    123.795591182365	0.000625542145994597\\
    124.342685370741	0.000612044942918267\\
    124.889779559118	0.000598856546432485\\
    125.436873747495	0.000585969555687752\\
    125.983967935872	0.000573376752524738\\
    126.531062124248	0.000561071096953431\\
    127.078156312625	0.000549045722738498\\
    127.625250501002	0.000537293933088847\\
    128.172344689379	0.000525809196449363\\
    128.719438877756	0.000514585142392817\\
    129.266533066132	0.00050361555760994\\
    129.813627254509	0.000492894381995682\\
    130.360721442886	0.000482415704829682\\
    130.907815631263	0.000472173761048995\\
    131.454909819639	0.00046216292761113\\
    132.002004008016	0.000452377719945501\\
    132.549098196393	0.000442812788491371\\
    133.09619238477	0.000433462915320439\\
    133.643286573146	0.000424323010842202\\
    134.190380761523	0.000415388110590281\\
    134.7374749499	0.000406653372087901\\
    135.284569138277	0.000398114071790753\\
    135.831663326653	0.000389765602105505\\
    136.37875751503	0.000381603468482218\\
    136.925851703407	0.000373623286578998\\
    137.472945891784	0.000365820779497205\\
    138.02004008016	0.000358191775085606\\
    138.567134268537	0.000350732203311824\\
    139.114228456914	0.000343438093699558\\
    139.661322645291	0.000336305572829986\\
    140.208416833667	0.00032933086190584\\
    140.755511022044	0.000322510274376683\\
    141.302605210421	0.000315840213623893\\
    141.849699398798	0.000309317170703947\\
    142.396793587174	0.000302937722148588\\
    142.943887775551	0.000296698527820502\\
    143.490981963928	0.000290596328823159\\
    144.038076152305	0.000284627945463493\\
    144.585170340681	0.000278790275266137\\
    145.132264529058	0.000273080291037942\\
    145.679358717435	0.000267495038981544\\
    146.226452905812	0.000262031636856769\\
    146.773547094188	0.000256687272188698\\
    147.320641282565	0.000251459200521222\\
    147.867735470942	0.000246344743714972\\
    148.414829659319	0.000241341288288495\\
    148.961923847695	0.000236446283801623\\
    149.509018036072	0.000231657241279952\\
    150.056112224449	0.00022697173167942\\
    150.603206412826	0.000222387384389961\\
    151.150300601202	0.000217901885777253\\
    151.697394789579	0.000213512977761614\\
    152.244488977956	0.00020921845643308\\
    152.791583166333	0.000205016170701768\\
    153.338677354709	0.00020090402098263\\
    153.885771543086	0.000196879957913709\\
    154.432865731463	0.000192941981107062\\
    154.97995991984	0.0001890881379315\\
    155.527054108216	0.000185316522326361\\
    156.074148296593	0.000181625273645488\\
    156.62124248497	0.000178012575530672\\
    157.168336673347	0.00017447665481379\\
    157.715430861723	0.000171015780446899\\
    158.2625250501	0.00016762826245959\\
    158.809619238477	0.00016431245094288\\
    159.356713426854	0.000161066735058969\\
    159.90380761523	0.000157889542076208\\
    160.450901803607	0.000154779336428612\\
    160.997995991984	0.000151734618799299\\
    161.545090180361	0.000148753925227237\\
    162.092184368737	0.00014583582623669\\
    162.639278557114	0.000142978925988795\\
    163.186372745491	0.000140181861454683\\
    163.733466933868	0.000137443301609595\\
    164.280561122244	0.000134761946647461\\
    164.827655310621	0.000132136527215388\\
    165.374749498998	0.000129565803667576\\
    165.921843687375	0.000127048565338137\\
    166.468937875751	0.00012458362983233\\
    167.016032064128	0.000122169842335758\\
    167.563126252505	0.000119806074941028\\
    168.110220440882	0.000117491225991461\\
    168.657314629259	0.000115224219441375\\
    169.204408817635	0.000113004004232545\\
    169.751503006012	0.000110829553686398\\
    170.298597194389	0.000108699864911553\\
    170.845691382766	0.000106613958226293\\
    171.392785571142	0.000104570876595604\\
    171.939879759519	0.000102569685082384\\
    172.486973947896	0.00010060947031247\\
    173.034068136273	9.8689339953122e-05\\
    173.581162324649	9.68084222046131e-05\\
    174.128256513026	9.49658653045932e-05\\
    174.675350701403	9.31608370448921e-05\\
    175.22244488978	9.1392524300444e-05\\
    175.769539078156	8.96601325700183e-05\\
    176.316633266533	8.79628855284539e-05\\
    176.86372745491	8.63000245901014e-05\\
    177.410821643287	8.46708084831796e-05\\
    177.957915831663	8.30745128347726e-05\\
    178.50501002004	8.1510429766186e-05\\
    179.052104208417	7.9977867498401e-05\\
    179.599198396794	7.84761499673634e-05\\
    180.14629258517	7.70046164488558e-05\\
    180.693386773547	7.55626211927073e-05\\
    181.240480961924	7.4149533066099e-05\\
    181.787575150301	7.27647352057321e-05\\
    182.334669338677	7.14076246786335e-05\\
    182.881763527054	7.00776121513734e-05\\
    183.428857715431	6.87741215674826e-05\\
    183.975951903808	6.74965898328565e-05\\
    184.523046092184	6.62444665089436e-05\\
    185.070140280561	6.50172135135176e-05\\
    185.617234468938	6.38143048288404e-05\\
    186.164328657315	6.26352262170251e-05\\
    186.711422845691	6.14794749424164e-05\\
    187.258517034068	6.03465595008102e-05\\
    187.805611222445	5.92359993553344e-05\\
    188.352705410822	5.81473246788244e-05\\
    188.899799599198	5.70800761025265e-05\\
    189.446893787575	5.60338044709674e-05\\
    189.993987975952	5.50080706028331e-05\\
    190.541082164329	5.40024450577062e-05\\
    191.088176352705	5.30165079085085e-05\\
    191.635270541082	5.2049848519509e-05\\
    192.182364729459	5.11020653297518e-05\\
    192.729458917836	5.01727656417694e-05\\
    193.276553106212	4.9261565415445e-05\\
    193.823647294589	4.83680890668955e-05\\
    194.370741482966	4.7491969272247e-05\\
    194.917835671343	4.66328467761786e-05\\
    195.464929859719	4.57903702051161e-05\\
    196.012024048096	4.49641958849545e-05\\
    196.559118236473	4.41539876631995e-05\\
    197.10621242485	4.33594167354138e-05\\
    197.653306613226	4.25801614758589e-05\\
    198.200400801603	4.18159072722302e-05\\
    198.74749498998	4.10663463643792e-05\\
    199.294589178357	4.03311776869248e-05\\
    199.841683366733	3.96101067156543e-05\\
    200.38877755511	3.89028453176202e-05\\
    200.935871743487	3.82091116048404e-05\\
    201.482965931864	3.75286297915103e-05\\
    202.03006012024	3.68611300546401e-05\\
    202.577154308617	3.62063483980305e-05\\
    203.124248496994	3.55640265195056e-05\\
    203.671342685371	3.49339116813185e-05\\
    204.218436873747	3.43157565836536e-05\\
    204.765531062124	3.37093192411463e-05\\
    205.312625250501	3.31143628623462e-05\\
    205.859719438878	3.25306557320499e-05\\
    206.406813627255	3.19579710964329e-05\\
    206.953907815631	3.13960870509101e-05\\
    207.501002004008	3.08447864306583e-05\\
    208.048096192385	3.03038567037335e-05\\
    208.595190380762	2.97730898667206e-05\\
    209.142284569138	2.92522823428505e-05\\
    209.689378757515	2.87412348825258e-05\\
    210.236472945892	2.82397524661946e-05\\
    210.783567134269	2.77476442095146e-05\\
    211.330661322645	2.7264723270751e-05\\
    211.877755511022	2.67908067603536e-05\\
    212.424849699399	2.63257156526592e-05\\
    212.971943887776	2.58692746996668e-05\\
    213.519038076152	2.54213123468363e-05\\
    214.066132264529	2.49816606508582e-05\\
    214.613226452906	2.45501551993491e-05\\
    215.160320641283	2.41266350324239e-05\\
    215.707414829659	2.37109425660995e-05\\
    216.254509018036	2.3302923517485e-05\\
    216.801603206413	2.29024268317159e-05\\
    217.34869739479	2.25093046105876e-05\\
    217.895791583166	2.21234120428488e-05\\
    218.442885771543	2.17446073361142e-05\\
    218.98997995992	2.13727516503559e-05\\
    219.537074148297	2.10077090329364e-05\\
    220.084168336673	2.06493463551454e-05\\
    220.63126252505	2.02975332502038e-05\\
    221.178356713427	1.99521420526996e-05\\
    221.725450901804	1.96130477394215e-05\\
    222.27254509018	1.92801278715549e-05\\
    222.819639278557	1.89532625382099e-05\\
    223.366733466934	1.86323343012466e-05\\
    223.913827655311	1.83172281413683e-05\\
    224.460921843687	1.80078314054514e-05\\
    225.008016032064	1.77040337550821e-05\\
    225.555110220441	1.74057271162717e-05\\
    226.102204408818	1.71128056303208e-05\\
    226.649298597194	1.68251656058069e-05\\
    227.196392785571	1.65427054716667e-05\\
    227.743486973948	1.62653257313476e-05\\
    228.290581162325	1.59929289180035e-05\\
    228.837675350701	1.57254195507086e-05\\
    229.384769539078	1.54627040916667e-05\\
    229.931863727455	1.52046909043903e-05\\
    230.478957915832	1.49512902128282e-05\\
    231.026052104208	1.47024140614181e-05\\
    231.573146292585	1.44579762760424e-05\\
    232.120240480962	1.42178924258663e-05\\
    232.667334669339	1.39820797860361e-05\\
    233.214428857715	1.37504573012191e-05\\
    233.761523046092	1.35229455499632e-05\\
    234.308617234469	1.32994667098591e-05\\
    234.855711422846	1.30799445234837e-05\\
    235.402805611222	1.28643042651087e-05\\
    235.949899799599	1.26524727081544e-05\\
    236.496993987976	1.24443780933725e-05\\
    237.044088176353	1.22399500977403e-05\\
    237.591182364729	1.20391198040495e-05\\
    238.138276553106	1.18418196711734e-05\\
    238.685370741483	1.16479835049973e-05\\
    239.23246492986	1.14575464299948e-05\\
    239.779559118236	1.12704448614371e-05\\
    240.326653306613	1.10866164782194e-05\\
    240.87374749499	1.09060001962896e-05\\
    241.420841683367	1.07285361426665e-05\\
    241.967935871743	1.05541656300324e-05\\
    242.51503006012	1.0382831131888e-05\\
    243.062124248497	1.02144762582559e-05\\
    243.609218436874	1.004904573192e-05\\
    244.156312625251	9.88648536518828e-06\\
    244.703406813627	9.72674203716783e-06\\
    245.250501002004	9.56976367153836e-06\\
    245.797595190381	9.41549921481488e-06\\
    246.344689378758	9.26389861508668e-06\\
    246.891783567134	9.1149128012223e-06\\
    247.438877755511	8.96849366252981e-06\\
    247.985971943888	8.82459402886151e-06\\
    248.533066132265	8.68316765115346e-06\\
    249.080160320641	8.54416918238922e-06\\
    249.627254509018	8.40755415897854e-06\\
    250.174348697395	8.2732789825414e-06\\
    250.721442885772	8.14130090208799e-06\\
    251.268537074148	8.0115779965857e-06\\
    251.815631262525	7.88406915790437e-06\\
    252.362725450902	7.75873407413096e-06\\
    252.909819639279	7.63553321324549e-06\\
    253.456913827655	7.51442780714971e-06\\
    254.004008016032	7.39537983604102e-06\\
    254.551102204409	7.2783520131231e-06\\
    255.098196392786	7.16330776964624e-06\\
    255.645290581162	7.05021124026954e-06\\
    256.192384769539	6.9390272487379e-06\\
    256.739478957916	6.82972129386651e-06\\
    257.286573146293	6.72225953582612e-06\\
    257.833667334669	6.61660878272198e-06\\
    258.380761523046	6.51273647746038e-06\\
    258.927855711423	6.41061068489563e-06\\
    259.4749498998	6.31020007925172e-06\\
    260.022044088176	6.21147393181235e-06\\
    260.569138276553	6.11440209887314e-06\\
    261.11623246493	6.01895500995032e-06\\
    261.663326653307	5.92510365624029e-06\\
    262.210420841683	5.83281957932395e-06\\
    262.75751503006	5.74207486011095e-06\\
    263.304609218437	5.65284210801813e-06\\
    263.851703406814	5.56509445037698e-06\\
    264.39879759519	5.47880552206508e-06\\
    264.945891783567	5.39394945535662e-06\\
    265.492985971944	5.31050086998696e-06\\
    266.040080160321	5.22843486342655e-06\\
    266.587174348697	5.14772700135971e-06\\
    267.134268537074	5.06835330836344e-06\\
    267.681362725451	4.99029025878226e-06\\
    268.228456913828	4.91351476779421e-06\\
    268.775551102204	4.83800418266434e-06\\
    269.322645290581	4.76373627418118e-06\\
    269.869739478958	4.69068922827231e-06\\
    270.416833667335	4.61884163779509e-06\\
    270.963927855711	4.54817249449867e-06\\
    271.511022044088	4.47866118115355e-06\\
    272.058116232465	4.41028746384497e-06\\
    272.605210420842	4.34303148442662e-06\\
    273.152304609218	4.27687375313108e-06\\
    273.699398797595	4.21179514133364e-06\\
    274.246492985972	4.14777687446611e-06\\
    274.793587174349	4.08480052507733e-06\\
    275.340681362725	4.02284800603722e-06\\
    275.887775551102	3.96190156388136e-06\\
    276.434869739479	3.90194377229263e-06\\
    276.981963927856	3.84295752571765e-06\\
    277.529058116232	3.78492603311429e-06\\
    278.076152304609	3.72783281182805e-06\\
    278.623246492986	3.67166168159414e-06\\
    279.170340681363	3.61639675866277e-06\\
    279.717434869739	3.56202245004482e-06\\
    280.264529058116	3.50852344787538e-06\\
    280.811623246493	3.45588472389267e-06\\
    281.35871743487	3.4040915240297e-06\\
    281.905811623247	3.35312936311645e-06\\
    282.452905811623	3.30298401968994e-06\\
    283	3.25364153091023e-06\\
    };
    \addlegendentry{Fittagio lognormale}
        % This file was created by matlab2tikz.
%
\definecolor{mycolor1}{rgb}{0.00000,0.44700,0.74100}%
%
\nextgroupplot[
    % width=4.521in,
    % height=3.566in,
    % at={(0.758in,0.481in)},
    scale only axis,
    xmin=0,
    xmax=300,
    xlabel style={font=\color{white!15!black}},
    xlabel={$k$},
    ymin=40,
    ymax=95,
    ylabel style={font=\color{white!15!black}},
    ylabel={$k_{nn}(k)$},
    axis background/.style={fill=white},
    % axis x line*=bottom,
    % axis y line*=left,
    xmajorgrids,
    ymajorgrids,
    grid style={dashed},
    legend style=plotLegend
]
    \addplot[
        only marks,
        mark=*,
        mark options={},
        mark size=1.5000pt,
        color=black,
        fill=black,
        forget plot
    ] table[row sep=crcr]{%
        x	y\\
        10	89.3\\
        11	87\\
        12	89.75\\
        13	76.0512820512821\\
        14	75.3285714285714\\
        15	71.3666666666667\\
        16	80.9\\
        17	73.3921568627451\\
        18	74.75\\
        19	69.2842105263158\\
        20	70.07\\
        21	66.7619047619048\\
        22	60.5454545454545\\
        23	73.3188405797101\\
        24	68.65\\
        25	74\\
        26	65.2211538461538\\
        27	67.0940170940171\\
        28	70.0178571428571\\
        29	68.367816091954\\
        30	72.2388888888889\\
        31	66.6589861751152\\
        32	64.1741071428571\\
        33	65.7212121212121\\
        34	70.1617647058823\\
        35	68.7285714285714\\
        36	71.9027777777778\\
        37	70.7837837837838\\
        38	66.4368421052632\\
        39	67.2582417582418\\
        40	74.4583333333333\\
        41	72.8024390243902\\
        42	69.7047619047619\\
        43	71.4249471458774\\
        44	68.2318181818182\\
        45	74.2395061728395\\
        46	73.2217391304348\\
        47	72.5471124620061\\
        48	68.2625\\
        49	69.0979591836735\\
        50	73.9366666666667\\
        51	67.7303921568627\\
        52	75.3205128205128\\
        53	70.2924528301887\\
        54	66.7453703703704\\
        55	67.6\\
        56	75.8154761904762\\
        57	69.7280701754386\\
        58	50.6896551724138\\
        59	54.7966101694915\\
        61	76.1803278688525\\
        62	74.4596774193548\\
        63	64.5714285714286\\
        64	69.76953125\\
        65	71.2307692307692\\
        67	72.1940298507463\\
        68	75.6911764705882\\
        69	50.7971014492754\\
        70	63.0714285714286\\
        71	65.6056338028169\\
        72	73.5972222222222\\
        73	73.1872146118722\\
        76	51.5131578947368\\
        77	63.6883116883117\\
        80	64.85\\
        82	61.1341463414634\\
        83	69.2048192771084\\
        84	57.6547619047619\\
        85	75.5058823529412\\
        86	52.1860465116279\\
        87	62.5632183908046\\
        88	72.0397727272727\\
        89	62.9101123595506\\
        90	60.9\\
        91	62.2747252747253\\
        93	66.5161290322581\\
        94	71.6063829787234\\
        95	58.1684210526316\\
        96	60.0347222222222\\
        97	44.7216494845361\\
        98	68.7908163265306\\
        99	67.4949494949495\\
        101	68.2772277227723\\
        103	66.3300970873786\\
        108	63.75\\
        109	57.0917431192661\\
        111	66.045045045045\\
        112	65.7589285714286\\
        117	63.8034188034188\\
        120	63.125\\
        127	53.8740157480315\\
        129	62.2286821705426\\
        132	63.0606060606061\\
        135	48.3925925925926\\
        140	61.7071428571429\\
        149	62.1744966442953\\
        154	47.7207792207792\\
        182	55.7967032967033\\
        200	54.56\\
        206	54.3155339805825\\
        283	50.3462897526502\\
    };
        % This file was created by matlab2tikz.
%
\nextgroupplot[
    % width=4.521in,
    % height=3.566in,
    % at={(0.758in,0.481in)},
    scale only axis,
    xmode=log,
    xmin=0.1,
    xmax=100000,
    xminorticks=true,
    xlabel style={font=\color{white!15!black}},
    xlabel={$w$},
    ymode=log,
    ymin=1e-08,
    ymax=1,
    yminorticks=true,
    ytickten={-8,-6,-4,-2,0},
    ylabel style={font=\color{white!15!black}},
    ylabel={$P(w)$},
    axis background/.style={fill=white},
    % axis x line*=bottom,
    % axis y line*=left,
    xmajorgrids,
    ymajorgrids,
    grid style={dashed},
    % legend style={
    %     legend cell align=left,
    %     align=left,
    %     draw=white!15!black
    % },
    legend style=plotLegend,
    log origin=infty
]
    \addplot[
        ybar interval,
        fill=colore,
        area legend,
        draw=none
    ] table[
        row sep=crcr,
        x=Lower,
        y=Count
    ] {%
        Lower	Upper	Count\\
        1	1.65709146891364	0.645588797757302\\
        1.65709146891364	2.74595213634636	0.134535967904517\\
        2.74595213634636	4.55029385918474	0.0705509116756377\\
        4.55029385918474	7.54025313510515	0.0254058774758624\\
        7.54025313510515	12.4948891436321	0.0118545769628516\\
        12.4948891436321	20.7051742049343	0.00525272541319737\\
        20.7051742049343	34.3103675373674	0.00279592349281125\\
        34.3103675373674	56.8554173414631	0.00121543363316469\\
        56.8554173414631	94.214627038063	0.000529215472189377\\
        94.214627038063	156.122254711654	0.000235320892045568\\
        156.122254711654	258.708856390245	9.46722601860312e-05\\
        258.708856390245	428.704238856679	4.1488412073954e-05\\
        428.704238856679	710.402136896518	1.19027824068337e-05\\
        710.402136896518	1177.20132054924	4.95374914842227e-06\\
        1177.20132054924	1950.73026547601	7.47356021280746e-07\\
        1950.73026547601	3232.53848107194	6.31406563500675e-07\\
        3232.53848107194	5356.61193991936	1.63299864856744e-07\\
        5356.61193991936	8876.39594792132	3.28486926079384e-08\\
        8876.39594792132	14709	1.98231016357072e-08\\
        14709	14709	1.98231016357072e-08\\
    };
    \addlegendentry{Istogramma}

    \addplot [color=blue]
    table[row sep=crcr]{%
        1.28728064885387	0.62520801005048\\
        2.13314178131336	0.245418093491598\\
        3.53481104779761	0.0963360028099894\\
        5.85750523152711	0.0378155714005007\\
        9.70642194828058	0.0148440603578591\\
        16.0844290041719	0.00582686231484118\\
        26.6533700851603	0.00228726666542692\\
        44.1670721859171	0.000897839783419662\\
        73.1888785261761	0.000352436508115071\\
        121.280666225083	0.000138344830053366\\
        200.973157345748	5.43056453057517e-05\\
        333.030904518277	2.13170460431121e-05\\
        551.862670761829	8.36775715389635e-06\\
        914.486923731325	3.28466522260985e-06\\
        1515.38847974825	1.28935692398755e-06\\
        2511.13732188084	5.06121983449424e-07\\
        4161.18423335939	1.98672266278731e-07\\
        6895.46289367779	7.79864749587012e-08\\
        11426.412735324	3.06126788122075e-08\\
    };
    \addlegendentry{Regressione}
        % This file was created by matlab2tikz.
%
\definecolor{mycolor1}{rgb}{0.00000,0.44700,0.74100}%
%
\nextgroupplot[%
    % width=4.521in,
    % height=3.566in,
    % at={(0.758in,0.481in)},
    height=\halfHeightPlot cm,
    scale only axis,
    xmode=log,
    xmin=10,
    xmax=283,
    % xminorticks=true,
    xtickten={1,2,3},
    xlabel style={font=\color{white!15!black}},
    xticklabels=\empty,
    % xlabel={$k$},
    % ymode=log,
    ymin=0.2,
    ymax=1,
    ytick={0.5,1},
    ylabel style={font=\color{white!15!black}},
    ylabel={$C^w(k)$},
    axis background/.style={fill=white},
    % axis x line*=bottom,
    % axis y line*=left,
    xmajorgrids,
    ymajorgrids,
    grid style={dashed}
]
    \addplot[
        only marks,
        mark=*,
        mark options={},
        mark size=1.5000pt,
        color=black,
        fill=black,
        forget plot
    ] table[row sep=crcr]{%
        x	y\\
        10	0.894327894327894\\
        11	0.896875\\
        12	0.830909090909091\\
        13	0.800453303527074\\
        14	0.815850170144056\\
        15	0.813130799972905\\
        16	0.834878759270644\\
        17	0.775269541778976\\
        18	0.757456305734169\\
        19	0.770403781959471\\
        20	0.83479343860568\\
        21	0.816423991450708\\
        22	0.847839502331492\\
        23	0.729939824999801\\
        24	0.810997209910532\\
        25	0.78459333853098\\
        26	0.801518547409401\\
        27	0.776575852318383\\
        28	0.734995863363864\\
        29	0.749257653419528\\
        30	0.775776268651412\\
        31	0.762293497616241\\
        32	0.787049482034579\\
        33	0.778817140903613\\
        34	0.784012737306352\\
        35	0.721792410868428\\
        36	0.750934488976285\\
        37	0.731722001441343\\
        38	0.773717163822543\\
        39	0.747765922770739\\
        40	0.748744286264932\\
        41	0.741600052528491\\
        42	0.749792400268859\\
        43	0.768847661947213\\
        44	0.765081107651386\\
        45	0.766256093035422\\
        46	0.722138909086021\\
        47	0.751204822423547\\
        48	0.6547134119275\\
        49	0.780532499792838\\
        50	0.775736585299164\\
        51	0.697561305096584\\
        52	0.819252193888351\\
        53	0.722607000952232\\
        54	0.729430834357906\\
        55	0.722550842209151\\
        56	0.843210995241948\\
        57	0.805505769117992\\
        58	0.88051919923265\\
        59	0.533710344827586\\
        61	0.771145313366611\\
        62	0.632787499133979\\
        63	0.456271962727524\\
        64	0.529447076518452\\
        65	0.766292869922502\\
        67	0.742633028853125\\
        68	0.835663014131182\\
        69	0.574723798148701\\
        70	0.713903911612884\\
        71	0.708923132055187\\
        72	0.756661579712847\\
        73	0.805549028209804\\
        76	0.493916159082766\\
        77	0.501117659012396\\
        80	0.611005383300957\\
        82	0.467625344862132\\
        83	0.822259065390164\\
        84	0.43300425421362\\
        85	0.811411555130148\\
        86	0.504297798688477\\
        87	0.485174637010117\\
        88	0.732616441115305\\
        89	0.398602675163464\\
        90	0.481058796652059\\
        91	0.403352490421456\\
        93	0.546489872775994\\
        94	0.694512907930786\\
        95	0.613204482702615\\
        96	0.782246320183604\\
        97	0.746396568491248\\
        98	0.734958906667555\\
        99	0.917936104201023\\
        101	0.871418955304254\\
        103	0.553052938347056\\
        108	0.880113487920137\\
        109	0.469670557324036\\
        111	0.870154491465592\\
        112	0.563468856504336\\
        117	0.57590193342635\\
        120	0.641274333680985\\
        127	0.436843340963785\\
        129	0.602637264165786\\
        132	0.62281237907623\\
        135	0.33888893378263\\
        140	0.780628406299617\\
        149	0.879238140141167\\
        154	0.377818602505703\\
        182	0.29470083659524\\
        200	0.3072496314974\\
        206	0.27787132519668\\
        283	0.369798951169732\\
    };
        % This file was created by matlab2tikz.
%
\definecolor{mycolor1}{rgb}{0.00000,0.44700,0.74100}%
%
\nextgroupplot[%
    % width=4.521in,
    % height=3.566in,
    % at={(0.758in,0.481in)},
    height=\halfHeightPlot cm,
    yshift=\yPlotSepPartial em,
    scale only axis,
    xmode=log,
    xmin=10,
    xmax=283,
    xminorticks=true,
    xtickten={1,2,3},
    xlabel style={font=\color{white!15!black}},
    xlabel={$k$},
    ymode=log,
    % ymin=0.128303426794971,
    % ymax=1.9673950820926,
    ymin=1,
    ymax=1000,
    ytickten={0,1,2,3},
    yminorticks=true,
    ylabel style={font=\color{white!15!black}},
    ylabel={$C^w_{\text{rel}}(k) [\%]$},
    axis background/.style={fill=white},
    % axis x line*=bottom,
    % axis y line*=left,
    xmajorgrids,
    ymajorgrids,
    grid style={dashed}
]
    \addplot[
        only marks,
        mark=*,
        mark options={},
        mark size=1.5000pt,
        color=black,
        fill=black,
        forget plot
    ] table[
        row sep=crcr,
        y expr=\thisrow{y}*100
    ]{%
        x	y\\
        10	0.166514644775514\\
        11	0.147165697674419\\
        12	0.246363636363636\\
        13	0.142110201373996\\
        14	0.128303426794971\\
        15	0.130844158902716\\
        16	0.175885576437527\\
        17	0.207289973457336\\
        18	0.15123325937743\\
        19	0.224340582853806\\
        20	0.330627125294288\\
        21	0.293955005318103\\
        22	0.222540106358143\\
        23	0.314411215124197\\
        24	0.273237940473873\\
        25	0.282325674354508\\
        26	0.368018527307074\\
        27	0.365516614307815\\
        28	0.300086272117644\\
        29	0.379585520581988\\
        30	0.32251865524506\\
        31	0.407410853511551\\
        32	0.432950079509207\\
        33	0.373465098186732\\
        34	0.372328067484753\\
        35	0.3496746840563\\
        36	0.453979955605254\\
        37	0.448026845639999\\
        38	0.470857669462541\\
        39	0.407862737354564\\
        40	0.415235565314976\\
        41	0.439318445144054\\
        42	0.538539696452544\\
        43	0.433364081479291\\
        44	0.464521909830457\\
        45	0.445245933307707\\
        46	0.50870765220838\\
        47	0.50419870105292\\
        48	0.608968907743398\\
        49	0.467710616815442\\
        50	0.516803379076578\\
        51	0.522277559260838\\
        52	0.515102383676365\\
        53	0.746934118091535\\
        54	0.687656465587977\\
        55	0.497192559089194\\
        56	0.597881377365792\\
        57	0.765916493835598\\
        58	1.05288890878924\\
        59	0.381510438729198\\
        61	0.681997524983192\\
        62	0.434773574175485\\
        63	0.291448033633122\\
        64	0.511848875724079\\
        65	0.961709747001604\\
        67	0.631357801087192\\
        68	0.601043184348892\\
        69	0.608952303647796\\
        70	0.932100799266099\\
        71	0.438101210740523\\
        72	0.514508220631196\\
        73	0.656481100262413\\
        76	0.339353999415683\\
        77	0.64380075142407\\
        80	0.900371074046283\\
        82	0.507751233288487\\
        83	1.05594974248547\\
        84	0.331087151841869\\
        85	0.776051043417921\\
        86	0.530904031732876\\
        87	0.570102350393465\\
        88	0.73489374363711\\
        89	0.247744265339827\\
        90	0.605533733826248\\
        91	0.316118285478774\\
        93	0.30316815815814\\
        94	0.659768135902387\\
        95	1.10450270197323\\
        96	1.0425901240908\\
        97	1.49119886945896\\
        98	0.828931771408844\\
        99	1.39403658144041\\
        101	1.33580983242382\\
        103	0.517861590980713\\
        108	1.47098918037053\\
        109	0.775517598207628\\
        111	1.26827206250958\\
        112	0.555294143885859\\
        117	0.708072779821335\\
        120	0.926251048583185\\
        127	0.910980629333647\\
        129	1.077400105617\\
        132	0.931433224351897\\
        135	0.423050327791963\\
        140	1.52594426115573\\
        149	1.9673950820926\\
        154	0.813067599234089\\
        182	0.515459718876113\\
        200	0.450253241650442\\
        206	0.52515025514112\\
        283	1.29523838054531\\
    };
    \end{groupplot}

    %\begingroup Impostazioni pgfplots - Seconda colonna
        \pgfplotsset{
            /pgfplots/group/plotGroup/.append style={
                group name=Col2,
                y descriptions at=edge right,
            },
            plotColumn/.append style={
                xmax=50,
                % xlabel={},
                axis y line*=right,
                y tick scale label style={
                    xshift=.4cm,
                    at={(1,1)},
                    anchor=south west,
                },
                % ytick distance=0.01,
            },
            % plotLegend/.append style={
            %     at={(1,.5)},
            %     anchor=east,
            %     xshift=-2*\xLegendSep, % Le traslazioni vengono sommate; quindi se prima si aveva \xLegendSep ora si ha «\xLegendSep-2*\xLegendSepcm=-\xLegendSep», come desiderato
            % },
        }
    %\endgroup

    \begin{groupplot}[group style=plotGroup,plotColumn,clip=false]
        % This file was created by matlab2tikz.
%
\definecolor{mycolor1}{rgb}{0.00000,0.44700,0.74100}%
%
\nextgroupplot[%
    at={($(Col1 c1r1.south east)+(\xPlotSep em,0)$)},
    % at={(0,0)},
    anchor=south west,
    % title style={font=\dimTesto{\titleSize}},
    % title={Prova},
    %
    % width=4.521in,
    % height=3.566in,
    % at={(0.758in,0.481in)},
    scale only axis,
    xmin=0,
    xmax=300,
    xlabel style={font=\color{white!15!black}},
    xlabel={$k$},
    ymin=0.1,
    ymax=0.8,
    ylabel style={font=\color{white!15!black}},
    ylabel={$C(k)$},
    axis background/.style={fill=white},
    % axis x line*=bottom,
    % axis y line*=right,
    xmajorgrids,
    ymajorgrids,
    grid style={dashed}
]
    \addplot[
        only marks,
        mark=*,
        mark options={},
        mark size=1.5000pt,
        color=black,
        fill=black,
        forget plot
    ] table[row sep=crcr]{%
        x	y\\
        10	0.766666666666667\\
        11	0.781818181818182\\
        12	0.666666666666667\\
        13	0.700854700854701\\
        14	0.723076923076923\\
        15	0.719047619047619\\
        16	0.71\\
        17	0.642156862745098\\
        18	0.657952069716776\\
        19	0.629239766081871\\
        20	0.627368421052632\\
        21	0.630952380952381\\
        22	0.693506493506494\\
        23	0.555335968379447\\
        24	0.63695652173913\\
        25	0.611851851851852\\
        26	0.585897435897436\\
        27	0.56870479947403\\
        28	0.565343915343915\\
        29	0.543103448275862\\
        30	0.586590038314176\\
        31	0.541628264208909\\
        32	0.549251152073733\\
        33	0.567045454545455\\
        34	0.571301247771836\\
        35	0.534789915966387\\
        36	0.516468253968254\\
        37	0.505323505323505\\
        38	0.526031294452347\\
        39	0.531135531135531\\
        40	0.529059829059829\\
        41	0.515243902439024\\
        42	0.487340301974448\\
        43	0.53639383871942\\
        44	0.522410147991543\\
        45	0.530190796857464\\
        46	0.478647342995169\\
        47	0.499405312541298\\
        48	0.406914893617021\\
        49	0.531802721088435\\
        50	0.511428571428571\\
        51	0.458235294117647\\
        52	0.540723981900452\\
        53	0.413642960812772\\
        54	0.432215234102027\\
        55	0.482603815937149\\
        56	0.527705627705628\\
        57	0.456140350877193\\
        58	0.428917120387175\\
        59	0.386323787258913\\
        61	0.458469945355191\\
        62	0.441036488630354\\
        63	0.353302611367127\\
        64	0.350198412698413\\
        65	0.390625\\
        67	0.455223880597015\\
        68	0.521949078138718\\
        69	0.357203751065644\\
        70	0.369496204278813\\
        71	0.492957746478873\\
        72	0.499608763693271\\
        73	0.486301369863014\\
        76	0.368771929824561\\
        77	0.304853041695147\\
        80	0.321518987341772\\
        82	0.310147545919904\\
        83	0.399941228327946\\
        84	0.325301204819277\\
        85	0.456862745098039\\
        86	0.329411764705882\\
        87	0.309008286554397\\
        88	0.422283176593521\\
        89	0.319458631256384\\
        90	0.299625468164794\\
        91	0.306471306471306\\
        93	0.419354838709677\\
        94	0.418439716312057\\
        95	0.291377379619261\\
        96	0.38296783625731\\
        97	0.299613402061856\\
        98	0.401851462234378\\
        99	0.383426097711812\\
        101	0.373069306930693\\
        103	0.364363221016562\\
        108	0.356178608515057\\
        109	0.264525993883792\\
        111	0.383619983619984\\
        112	0.362290862290862\\
        117	0.337164750957854\\
        120	0.332913165266106\\
        127	0.228596425446819\\
        129	0.290092054263566\\
        132	0.322461253758964\\
        135	0.238142620232172\\
        140	0.309044193216855\\
        149	0.296299655360058\\
        154	0.208386384856973\\
        182	0.194462995567968\\
        200	0.211859296482412\\
        206	0.182192753966375\\
        283	0.161115705586046\\
    };
        % This file was created by matlab2tikz.
%
\definecolor{mycolor1}{rgb}{0.00000,0.44700,0.74100}%
%
\nextgroupplot[%
    % width=4.521in,
    % height=3.566in,
    % at={(0.758in,0.481in)},
    scale only axis,
    xmode=log,
    xmin=10,
    xmax=283,
    xminorticks=true,
    xtickten={1,2,3},
    xlabel style={font=\color{white!15!black}},
    xlabel={$k$},
    ymode=log,
    ymin=0.1,
    ymax=100000,
    yminorticks=true,
    ylabel style={font=\color{white!15!black}},
    ylabel={$g(i)$},
    axis background/.style={fill=white},
    % axis x line*=bottom,
    % axis y line*=left,
    xmajorgrids,
    ymajorgrids,
    grid style={dashed}
]
    \addplot[
        only marks,
        mark=*,
        mark options={},
        mark size=1.5000pt,
        color=black,
        fill=black,
        forget plot
    ] table[row sep=crcr]{%
        x	y\\
        34	24.4077595983695\\
        29	24.0548838676952\\
        96	910.816660309013\\
        31	25.2996229232657\\
        27	13.9840093551101\\
        70	554.608345889032\\
        24	14.8482406741793\\
        37	40.2826018794221\\
        42	246.671576922462\\
        16	1.32168783345254\\
        29	29.4747553886748\\
        59	195.572145500989\\
        69	270.874479765742\\
        28	18.7726637358274\\
        11	3.37575686673779\\
        20	5.29951330297145\\
        45	81.287143222938\\
        27	20.2192077351399\\
        22	5.53956779308469\\
        27	10.8248169060855\\
        44	67.7241071465653\\
        16	8.09628111142582\\
        39	35.7214069499101\\
        19	4.41152544194396\\
        33	18.8581472042029\\
        37	149.092921666421\\
        33	39.684293605326\\
        18	4.87634613863792\\
        29	32.2821736541342\\
        22	8.93167057828336\\
        33	44.623208877497\\
        22	9.68002451218361\\
        49	93.8802330926963\\
        25	5.47032516352949\\
        43	149.160960658111\\
        26	11.4727075628203\\
        31	45.4979602836858\\
        26	94.8645864265104\\
        18	17.9909172382737\\
        13	1.16726108273929\\
        32	30.8667101637254\\
        38	43.719369594331\\
        32	55.6734563269646\\
        29	28.6298860183576\\
        35	48.4885214667306\\
        35	24.2038859532622\\
        135	2523.77533612426\\
        27	25.7647196168196\\
        42	61.5692480434759\\
        39	32.4530959423884\\
        51	98.7921520282781\\
        86	799.134065872299\\
        15	6.62365182766655\\
        39	51.6490650231878\\
        42	52.964020421646\\
        43	34.5077581608904\\
        58	186.898874230204\\
        97	891.35557363391\\
        50	124.195428418057\\
        20	6.17627374991726\\
        28	32.8537941255416\\
        22	4.9634136481218\\
        47	107.151810331161\\
        154	4102.09786983822\\
        30	23.9189212447411\\
        10	3.67655526782783\\
        37	67.8857830398076\\
        31	90.6606211853056\\
        54	292.203711602299\\
        76	584.256949076862\\
        55	202.948768295517\\
        26	15.5371884724382\\
        33	29.0888330131769\\
        34	36.9250648178486\\
        25	8.20116141300345\\
        31	87.8908619235439\\
        34	45.3020561970372\\
        26	27.7785523028594\\
        39	64.0148005251858\\
        27	12.7663630780348\\
        32	13.7207539004658\\
        26	7.91316099030401\\
        29	236.506588680164\\
        23	16.2512132708547\\
        26	12.786842760483\\
        14	5.67042257379387\\
        33	40.3118703371952\\
        24	25.6318730925871\\
        20	4.78292497766301\\
        51	145.08881801201\\
        35	54.2674991608661\\
        29	25.6468745604937\\
        46	73.8796585934957\\
        44	117.753574596781\\
        45	122.239044968079\\
        29	25.2761563957573\\
        28	28.800768897232\\
        77	553.786673567684\\
        70	233.169294565903\\
        70	305.858498707502\\
        40	134.109563749234\\
        90	552.938195550776\\
        53	216.086799826985\\
        62	245.435720195646\\
        47	78.7214824837711\\
        34	69.0789759577431\\
        14	0.689398918075389\\
        47	99.6366900263505\\
        43	134.256195850689\\
        20	5.51092800048796\\
        28	28.8896516270878\\
        80	576.507141396327\\
        25	10.543199036633\\
        36	45.9067819055063\\
        27	17.6588538860079\\
        82	430.283635099357\\
        31	25.7159892924415\\
        39	71.3563966359726\\
        23	9.98588642384381\\
        27	19.8025243042393\\
        37	73.9726704345446\\
        129	1161.47704583538\\
        63	226.175921872939\\
        64	346.339609070301\\
        91	523.055435625984\\
        27	25.5013993028184\\
        26	20.9404570310581\\
        17	8.74641658422946\\
        33	54.234936330109\\
        30	22.2482078041211\\
        33	53.8873400217068\\
        182	4849.33892917923\\
        26	37.6618254049445\\
        48	102.731744079356\\
        46	194.956390934928\\
        10	0.640807656395892\\
        25	33.2825476098488\\
        26	22.4177475617209\\
        206	5474.49939142422\\
        37	88.5311567063907\\
        43	55.7579270228685\\
        51	181.727873021351\\
        57	185.607597733343\\
        36	70.3632175138736\\
        38	61.3889586894383\\
        28	28.7156195342204\\
        19	10.2781895725273\\
        36	28.5866769223062\\
        45	137.953976107289\\
        53	125.345111019604\\
        48	85.5688453009621\\
        64	263.899912886093\\
        65	277.529004087262\\
        45	137.945953194533\\
        54	210.791057031158\\
        16	11.1174192026491\\
        24	22.1801333980861\\
        127	1535.78662133912\\
        48	108.291963026947\\
        64	240.243409190148\\
        42	102.933931123422\\
        42	64.3778371990855\\
        17	14.26170366487\\
        48	165.657946273305\\
        42	76.2141116197296\\
        20	10.2620776202165\\
        34	26.9103761127203\\
        47	56.2109594941505\\
        42	49.9744088269208\\
        50	151.019622709178\\
        87	560.979858655156\\
        64	236.218660446638\\
        32	37.5871782528139\\
        39	79.0270632587659\\
        35	99.9860073120595\\
        18	4.75274834573202\\
        28	18.6411194641975\\
        27	43.6029363491292\\
        37	49.4242313582483\\
        30	93.3958683325663\\
        89	502.962905812225\\
        22	6.84463605137342\\
        28	30.8436118137574\\
        25	13.6441680736215\\
        15	3.04232528064302\\
        41	78.2646177322482\\
        32	46.0425989008112\\
        34	35.1536316868496\\
        16	15.9773860926892\\
        88	366.425107836358\\
        39	51.1132869888248\\
        140	1104.30349177925\\
        39	38.3647357762183\\
        49	24.0678484897534\\
        72	148.30540362676\\
        37	17.1166748784462\\
        41	27.0856456061036\\
        283	10694.3669159702\\
        31	22.184716658744\\
        99	398.400067111576\\
        103	620.533173276366\\
        39	24.9600766207824\\
        35	23.6287318728375\\
        98	299.755019783444\\
        45	28.42260984773\\
        73	193.469054674577\\
        34	17.2463209717962\\
        56	48.9711786684488\\
        39	17.9267216934751\\
        37	31.3709034109002\\
        55	46.9290721312557\\
        25	10.6891832911646\\
        29	11.1210569127706\\
        42	57.8470412126995\\
        35	13.0588815772823\\
        26	8.75749615207151\\
        45	65.3911643929573\\
        56	60.244837870085\\
        39	11.9391742844899\\
        62	86.8097755661464\\
        94	272.407258219325\\
        112	654.92039746093\\
        25	7.64124645574208\\
        57	72.974182335944\\
        50	65.9173891830578\\
        47	97.6263436667008\\
        96	244.272398585917\\
        54	120.971005579926\\
        28	4.90039069187048\\
        38	13.4288226015146\\
        47	21.6920288022092\\
        35	16.0800042344566\\
        45	26.1211356264126\\
        50	28.0813027165556\\
        43	38.5213576543652\\
        29	22.2122156877576\\
        42	15.1864424930664\\
        71	88.9198022966985\\
        68	94.4818574789457\\
        149	1036.46247873234\\
        67	124.772689033249\\
        51	21.735938738428\\
        34	12.1224815492461\\
        120	595.834430302694\\
        43	29.1233744717856\\
        117	719.031481968108\\
        39	23.9393254811616\\
        72	77.8621887415518\\
        40	23.171367308371\\
        43	36.71688770357\\
        32	10.9318195658633\\
        49	107.652890376015\\
        41	74.3921372703194\\
        98	374.207992108966\\
        96	228.83979973588\\
        45	18.4819221847899\\
        129	807.609002869321\\
        43	14.9034003743462\\
        93	251.62431099296\\
        41	13.3592819657656\\
        88	349.419702261447\\
        73	131.694504438992\\
        108	436.50120231456\\
        52	53.006432846204\\
        18	6.38694965822758\\
        36	22.4163744097497\\
        50	96.6718932249648\\
        37	17.2054923949993\\
        83	188.809920049495\\
        40	30.9956451862742\\
        26	5.96268343702886\\
        40	53.190362004783\\
        49	55.6267185366156\\
        27	36.5237292245088\\
        41	44.2278979726284\\
        36	37.9515457793217\\
        52	34.4759336267436\\
        37	25.8927310519444\\
        68	135.588943829043\\
        43	37.2681119412443\\
        132	742.281665109493\\
        73	116.283195237917\\
        46	68.4397666376253\\
        47	74.4679059428968\\
        44	38.0646810375424\\
        41	34.9696703174062\\
        30	15.4734135468493\\
        25	7.29853437929289\\
        24	2.70935242453087\\
        85	356.220437102301\\
        44	19.7586897127959\\
        29	8.70484721509163\\
        34	10.3995326130367\\
        101	449.731434113423\\
        18	2.48120402291706\\
        21	3.15093837703114\\
        111	397.902079005116\\
        95	455.542060611277\\
        23	8.83517979995395\\
        31	33.76769608053\\
        84	270.458388401051\\
        23	23.1147829453057\\
        55	83.5762879569899\\
        32	37.5458931078993\\
        27	19.3157856930691\\
        48	86.8689480907807\\
        18	4.23863553931953\\
        28	13.948787065896\\
        40	38.2364604747105\\
        30	18.6822344176352\\
        13	3.67083565344704\\
        33	24.4293991543148\\
        14	7.36327809129481\\
        52	105.753770341713\\
        65	123.367074026716\\
        49	98.9285902033178\\
        35	24.9689466322811\\
        109	697.750975293326\\
        26	12.7344395275036\\
        41	52.7850002477668\\
        39	29.5426837260549\\
        61	120.367420881037\\
        40	36.5211666004454\\
        36	34.1516678713418\\
        21	11.5050456284073\\
        80	217.807477965195\\
        36	27.0134092145702\\
        36	30.5561456009748\\
        38	48.9779446108512\\
        42	65.3986151883739\\
        27	19.9446489737459\\
        46	59.7218389608714\\
        23	11.8081415469247\\
        34	23.4918250127662\\
        200	2892.58375117005\\
        37	29.0624500327908\\
        16	3.91776965369167\\
        50	63.9900173007261\\
        19	5.4616470126747\\
        41	33.0292560090971\\
        25	12.9936442154897\\
        39	61.3637445566803\\
        45	37.6788090221551\\
        83	258.780256886106\\
        27	18.0547299202938\\
        67	163.784228182039\\
        46	49.5574913034486\\
        35	57.6704215975226\\
        44	61.850023991591\\
        30	21.8430252023888\\
        29	28.9195568516962\\
        23	12.4133892981056\\
        43	55.745805795836\\
        28	22.1483646199189\\
        24	12.03401314593\\
        56	122.028257659249\\
        33	34.0181633009693\\
        13	3.52899452344483\\
        41	37.8438854466842\\
        34	41.7679009572519\\
        14	2.69923144260421\\
        83	248.062575837943\\
        33	16.8326173996442\\
        35	63.2582441272547\\
        29	18.3768100152577\\
        54	72.8450547804046\\
        41	56.7771829502471\\
        19	4.77886246913444\\
        34	26.4497727315071\\
        19	11.118254822446\\
        38	32.8774733594555\\
        43	67.842890904133\\
        14	1.2763225550872\\
        17	5.36823748040343\\
        12	2.60795038370574\\
    };
        % This file was created by matlab2tikz.
%
\nextgroupplot[
    % width=4.521in,
    % height=3.566in,
    % at={(0.758in,0.481in)},
    scale only axis,
    xmode=log,
    xmin=10,
    xmax=100000,
    xminorticks=true,
    xlabel style={font=\color{white!15!black}},
    xlabel={$s$},
    ymode=log,
    ymin=1e-07,
    ymax=0.01,
    yminorticks=true,
    ylabel style={font=\color{white!15!black}},
    ylabel={$P(s)$},
    axis background/.style={fill=white},
    % axis x line*=bottom,
    % axis y line*=left,
    xmajorgrids,
    ymajorgrids,
    grid style={dashed},
    % legend style={
    %     legend cell align=left,
    %     align=left,
    %     draw=white!15!black
    % },
    legend style=plotLegend,
    log origin=infty
]
    \addplot[
        ybar interval,
        fill=colore,
        % fill=gray,
        % fill={rgb,1:red,0.50196;green,0.50196;blue,0.50196},
        area legend,
        draw=none
    ] table[
        row sep=crcr,
        x=Lower,
        y expr=\thisrow{Count}
        % y expr={(\thisrow{Count}==0) ? 1e-15 : \thisrow{Count}}
    ] {%
        Lower	Upper	Count\\
        32	46.9815074680496	0.000894758138715566\\
        46.9815074680496	68.9769388740751	0.00134076100066422\\
        68.9769388740751	101.270017776111	0.00149435650903862\\
        101.270017776111	148.681815513684	0.00203566938850381\\
        148.681815513684	218.290494559979	0.00192574057156727\\
        218.290494559979	320.488015636684	0.00141659128751084\\
        320.488015636684	470.531565626727	0.000964867320001916\\
        470.531565626727	690.82132076391	0.000657189517834413\\
        690.82132076391	1014.24459501742	0.000223812151887666\\
        1014.24459501742	1489.08562547625	8.46904008546565e-05\\
        1489.08562547625	2186.23398230868	7.30667114142712e-05\\
        2186.23398230868	3209.7677552106	2.09545808330449e-05\\
        3209.7677552106	4712.49149256661	1.78407053858758e-05\\
        4712.49149256661	6918.74857034808	4.86065776187599e-06\\
        6918.74857034808	10157.9136758552	3.3106866246429e-06\\
        10157.9136758552	14913.5655381873	1.12748587367721e-06\\
        14913.5655381873	21895.6809596123	3.83976056773051e-07\\
        21895.6809596123	32146.6280788143	2.6153340918432e-07\\
        32146.6280788143	47196.7827236696	0\\
        47196.7827236696	69293	0\\
        69293	69293	0\\
    };
    \addlegendentry{Istogramma}

    \addplot [color=blue]
    table[row sep=crcr]{%
        388.329406254705	0.00114686983396336\\
        570.134403125584	0.000533495034511199\\
        837.055428694895	0.000248168487320418\\
        1228.94143357502	0.000115441745686222\\
        1804.29753560314	5.37005999068522e-05\\
        2649.01931698507	2.49801699828243e-05\\
        3889.21627574818	1.16201475114466e-05\\
        5710.0388594976	5.4054006950561e-06\\
        8383.31979126063	2.51445660610857e-06\\
        12308.15629313	1.16966204370117e-06\\
        18070.4917751128	5.44097397883717e-07\\
        26530.5920088684	2.53100440403337e-07\\
    };
    \addlegendentry{Regressione}
        % This file was created by matlab2tikz.
%
\definecolor{mycolor1}{rgb}{0.00000,0.44700,0.74100}%
%
\nextgroupplot[%
    % width=4.521in,
    % height=3.566in,
    % at={(0.758in,0.481in)},
    height=\halfHeightPlot cm,
    scale only axis,
    xmode=log,
    xmin=10,
    xmax=283,
    % xminorticks=true,
    xtickten={1,2,3},
    xlabel style={font=\color{white!15!black}},
    xticklabels=\empty,
    % xlabel={$k$},
    ymode=log,
    % ymin=51.7058402006449,
    % ymax=239.173547251027,
    ymin=10,
    ymax=1000,
    yminorticks=true,
    ylabel style={font=\color{white!15!black}},
    ylabel={$k_{nn}^w(k)$},
    axis background/.style={fill=white},
    % axis x line*=bottom,
    % axis y line*=left,
    xmajorgrids,
    ymajorgrids,
    grid style={dashed}
]
    \addplot[
        only marks,
        mark=*,
        mark options={},
        mark size=1.5000pt,
        color=black,
        fill=black,
        forget plot
    ] table[row sep=crcr]{%
        x	y\\
        10	90.7659574468085\\
        11	123.125\\
        12	103.16\\
        13	97.0215827338129\\
        14	87.472972972973\\
        15	83.6382978723404\\
        16	101.102521008403\\
        17	89.434456928839\\
        18	81.0640243902439\\
        19	99.7971014492754\\
        20	112.584337349398\\
        21	89.0989010989011\\
        22	78.7355072463768\\
        23	103.068376068376\\
        24	126.060240963855\\
        25	118.303880597015\\
        26	108.964928615767\\
        27	105.287221905305\\
        28	103.751041184637\\
        29	99.5552387740556\\
        30	101.399585921325\\
        31	94.8460921843687\\
        32	103.071215510813\\
        33	108.202391118702\\
        34	116.456153846154\\
        35	92.389226519337\\
        36	120.750543478261\\
        37	120.831153388823\\
        38	108.214364640884\\
        39	114.794029352902\\
        40	119.897660818713\\
        41	127.982328686432\\
        42	111.902226524685\\
        43	110.648392821929\\
        44	105.732180663373\\
        45	121.52722513089\\
        46	129.118884316415\\
        47	167.834926375614\\
        48	115.055004955401\\
        49	109.794285714286\\
        50	143.99156626506\\
        51	129.527538403897\\
        52	205.90963060686\\
        53	131.846755277561\\
        54	124.982682512733\\
        55	110.335412026726\\
        56	135.259582863585\\
        57	168.340934371524\\
        58	126.459435626102\\
        59	85.1824\\
        61	147.031613976706\\
        62	127.914043583535\\
        63	68.3402692778458\\
        64	93.5602379771939\\
        65	168.014519906323\\
        67	158.128321451717\\
        68	181.765517241379\\
        69	93.7681895093063\\
        70	111.147068906411\\
        71	84.9118437555993\\
        72	185.472670250896\\
        73	186.912613122172\\
        76	51.7058402006449\\
        77	100.899613899614\\
        80	127.672243346008\\
        82	85.605493133583\\
        83	190.114580389941\\
        84	75.9946091644205\\
        85	175.970965940815\\
        86	67.1205967675093\\
        87	86.8542449286251\\
        88	162.978809869376\\
        89	55.5067542898868\\
        90	94.8354898336414\\
        91	54.9750453720508\\
        93	99.1987784564131\\
        94	125.994614607876\\
        95	114.96401308615\\
        96	146.162619737135\\
        97	112.637260350097\\
        98	155.530179084678\\
        99	239.173547251027\\
        101	217.956919763059\\
        103	87.030303030303\\
        108	224.160127513487\\
        109	88.3833560709413\\
        111	212.979899497487\\
        112	102.97824499927\\
        117	113.29633911368\\
        120	129.969928922909\\
        127	90.932382461701\\
        129	209.865220759101\\
        132	127.887096774194\\
        135	53.8669915177766\\
        140	195.225798615329\\
        149	237.026167265264\\
        154	62.2550379481811\\
        182	71.588102621934\\
        200	70.003553099092\\
        206	67.7991107518189\\
        283	106.299813834009\\
    };
        % This file was created by matlab2tikz.
%
\definecolor{mycolor1}{rgb}{0.00000,0.44700,0.74100}%
%
\nextgroupplot[%
    % width=4.521in,
    % height=3.566in,
    % at={(0.758in,0.481in)},
    height=\halfHeightPlot cm,
    yshift=\yPlotSepPartial em,
    scale only axis,
    xmode=log,
    xmin=10,
    xmax=283,
    xminorticks=true,
    xtickten={1,2,3},
    xlabel style={font=\color{white!15!black}},
    xlabel={$k$},
    ymode=log,
    % ymin=0.001,
    % ymax=10,
    ymin=1,
    ymax=1000,
    ytickten={0,1,2,3},
    yminorticks=true,
    ylabel style={font=\color{white!15!black}},
    ylabel={$k_{nn,rel}^w(k)[\%]$},
    axis background/.style={fill=white},
    % axis x line*=bottom,
    % axis y line*=left,
    xmajorgrids,
    ymajorgrids,
    grid style={dashed}
]
    \addplot[
        only marks,
        mark=*,
        mark options={},
        mark size=1.5000pt,
        color=black,
        fill=black,
        forget plot
    ] table[
        row sep=crcr,
        y expr=\thisrow{y}*100
    ]{%
        x	y\\
        10	0.0164160968287628\\
        11	0.415229885057471\\
        12	0.14941504178273\\
        13	0.275738950309745\\
        14	0.161219060896664\\
        15	0.171951861826349\\
        16	0.249722138546395\\
        17	0.218583303064597\\
        18	0.0844685537156375\\
        19	0.440401798493035\\
        20	0.606740935484481\\
        21	0.334576977943597\\
        22	0.300436305871088\\
        23	0.405755672804497\\
        24	0.836274449582745\\
        25	0.59870108914885\\
        26	0.670699185616944\\
        27	0.569249039862511\\
        28	0.481779726176909\\
        29	0.456171111859925\\
        30	0.403670342677729\\
        31	0.422855306187903\\
        32	0.606118419090231\\
        33	0.646384593746387\\
        34	0.65982361382068\\
        35	0.344262285669007\\
        36	0.679358533983926\\
        37	0.70704569506928\\
        38	0.628830648955712\\
        39	0.706765243217723\\
        40	0.610265170480763\\
        41	0.757940124005403\\
        42	0.605374202089352\\
        43	0.549156103622195\\
        44	0.549602274728008\\
        45	0.636961658230301\\
        46	0.763395486774864\\
        47	1.31346115206875\\
        48	0.685478922620786\\
        49	0.588965680193751\\
        50	0.947498754768409\\
        51	0.912399061619375\\
        52	1.73377892550385\\
        53	0.875688640373295\\
        54	0.872529612454077\\
        55	0.63218065128293\\
        56	0.784062959965635\\
        57	1.41424915314552\\
        58	1.49477798173943\\
        59	0.554519517476029\\
        61	0.930047009377886\\
        62	0.717896827072279\\
        63	0.058367002090532\\
        64	0.340989917818802\\
        65	1.35873516067192\\
        67	1.19032407220696\\
        68	1.40140959246431\\
        69	0.845935827715302\\
        70	0.762241183114099\\
        71	0.294276708168217\\
        72	1.52010421929883\\
        73	1.55389707223332\\
        76	0.00374044833946731\\
        77	0.584272077950708\\
        80	0.968731585906054\\
        82	0.400289335119451\\
        83	1.74712920828083\\
        84	0.318097701798745\\
        85	1.33055969226695\\
        86	0.286178993316801\\
        87	0.388263698105894\\
        88	1.26234486450116\\
        89	-0.11768152673693\\
        90	0.55723300219444\\
        91	-0.117217376238464\\
        93	0.491349239645396\\
        94	0.759544461913587\\
        95	0.976399066808595\\
        96	1.43463473015008\\
        97	1.5186293808113\\
        98	1.26091486320529\\
        99	2.54357694969346\\
        101	2.19223446868749\\
        103	0.312078631750763\\
        108	2.51623729432921\\
        109	0.548093493770304\\
        111	2.22476726834281\\
        112	0.565996393743141\\
        117	0.77570953466853\\
        120	1.05892956709558\\
        127	0.687870881706523\\
        129	2.37248377177503\\
        132	1.02800297623661\\
        135	0.113124729052478\\
        140	2.16374717052275\\
        149	2.81227319975436\\
        154	0.304568763644018\\
        182	0.283016708733825\\
        200	0.283056325129985\\
        206	0.248245313689758\\
        283	1.11137333766315\\
    };
    \end{groupplot}

    %\begingroup Didascalie
        \tikzset{SubCaption/.style={
            text width=\widthPlot cm,
            anchor=north,
            align=center,
            yshift=-3em
        }}
        \node[SubCaption] at (Col1 c1r1.south)
            {\dimTesto{\captionSize}\sottoDidascalia{Distribuzione di probabilità dei gradi.\label{figDeMontisPk}}};
        \node[SubCaption] at (Col2 c1r1.south)
            {\dimTesto{\captionSize}\sottoDidascalia{Coefficiente d'aggregrazione \(C(k)\).\label{figDeMontisCk}}};

        \node[SubCaption] at (Col1 c1r2.south)
            {\dimTesto{\captionSize}\sottoDidascalia{Assortatività \(k_{nn}(k)\).\label{figDeMontisKnnk}}};
        \node[SubCaption] at (Col2 c1r2.south)
            {\dimTesto{\captionSize}\sottoDidascalia{Centralità per intermediazione \(g(i)\).\label{figDeMontisGi}}};
    
        \node[SubCaption] at (Col1 c1r3.south)
            {\dimTesto{\captionSize}\sottoDidascalia{Distribuzione di probabilità dei pesi.\label{figDeMontisWk}}};
        \node[SubCaption] at (Col2 c1r3.south)
            {\dimTesto{\captionSize}\sottoDidascalia{Distribuzione di probabilità delle forze.\label{figDeMontisSk}}};

        
        \pgfmathparse{1.5*\widthPlot}\node[SubCaption,text width=\pgfmathresult cm] at (Col1 c1r5.south)
            {\dimTesto{\captionSize}\sottoDidascalia{Coefficiente [relativo] d'aggregazione pesato \(C^w(k)\).\label{figDeMontisCwk}}};
        \node[SubCaption] at (Col2 c1r5.south)
            {\dimTesto{\captionSize}\sottoDidascalia{Assortatività [relativa] pesata \(k_{nn}^w(k)\).\label{figDeMontisKnnwk}}};
    
        % https://tex.stackexchange.com/questions/632045/get-subcaption-for-every-plot-using-groupplot-pgf-tikz-after-2020-update
    %\endgroup

    \redefineTikZbounds{Col1 c1r1}{Col2 c1r1}
\end{tikzpicture}%}
            % {Studio della configurazione di riferimento della \ref{eqLeggeEmigrazioneTagliaForza} con parametri dalla \cref{tabParametriRegolaEmigrazioneTagliaForza}; per la spiegazione dei grafici si veda il \cref{secDefinizioneGrafici}.}
            % {.7\linewidth}{figProva}

            I risultati nella \daRivedere{}, ottenuti tramite i parametri nella \cref{tabParametriRegolaEmigrazioneTaglia}, non sono purtroppo corretti dato che non solo mostrano dei centri abitati spopolarsi, ma che questi sono piú per la maggior parte i le città piú popolate.

            \begin{table}[htb]
                \centering
                \begin{tabular}{ccc}
                    \hline
                    \(\lambda\)&\(\alpha\)&\(N_t\)\\
                    \hline
                    \(\num{1}\)&\(\num{1}\)&\(\num{1}\)\\
                    \hline
                \end{tabular}
                \caption{Dati e parametri della \ref{eqLeggeEmigrazioneTaglia}.}
                \label{tabParametriRegolaEmigrazioneTaglia}
            \end{table}

        \subsection{Regola taglia-gradi}

            \eqtag{\([\text{RE}]_{TD}\)}{
                \label{eqLeggeEmigrazioneTagliaGradi}
                E_{TD}(s,s_*,i,i_*)\equiv\lambda\frac{[(s_*/s)(k_{i_*}/k_i)]^\alpha}{1+[(s_*/s)(k_{i_*}/k_i)]^\alpha},
            }%

            \begin{table}[htb]
                \centering
                \begin{tabular}{ccc}
                    \hline
                    \(\lambda\)&\(\alpha\)&\(N_t\)\\
                    \hline
                    \(\num{1}\)&\(\num{1}\)&\(\num{1}\)\\
                    \hline
                \end{tabular}
                \caption{Dati e parametri della \ref{eqLeggeEmigrazioneTagliaGradi}.}
                \label{tabParametriRegolaEmigrazioneTagliaGradi}
            \end{table}

        \subsection{Regola frazionata}
            
            \eqtag{\([\text{RE}]^f_{TD}\)}{
                \label{eqLeggeEmigrazioneTagliaGradiFrazionata}
                E^f_{TD}(s,s_*,i,i_*)\equiv (1-\zeta)E_{TD}(s,s_*,k_i,k_{i_*})+\zeta E_{TD}(s_*,s,k_{i_*},k_i)
            }%

            \begin{table}[htb]
                \centering
                \begin{tabular}{cccc}
                    \hline
                    \(\lambda\)&\(\alpha\)&\(\zeta\)&\(N_t\)\\
                    \hline
                    \(\num{1}\)&\(\num{1}\)&\(\num{1e-1}\)&\(\num{1}\)\\
                    \hline
                \end{tabular}
                \caption{Dati e parametri della \ref{eqLeggeEmigrazioneTagliaGradiFrazionata}.}
                \label{tabParametriRegolaEmigrazioneTagliaFrazionata}
            \end{table}

        \subsection{Regola taglia-forza}

            \eqtag{\([\text{RE}]_{TF}\)}{
                \label{eqLeggeEmigrazioneTagliaForza}
                E_{TD}(s,s_*,i,i_*)\equiv\lambda\frac{[(s_*/s)(w_{i_*}/w_i)]^\alpha}{1+[(s_*/s)(w_{i_*}/w_i)]^\alpha},
            }%
            
            \begin{table}[htb]
                \centering
                \begin{tabular}{ccc}
                    \hline
                    \(\lambda\)&\(\alpha\)&\(N_t\)\\
                    \hline
                    \(\num{1}\)&\(\num{1}\)&\(\num{1}\)\\
                    \hline
                \end{tabular}
                \caption{Dati e parametri della \ref{eqLeggeEmigrazioneTagliaForza}.}
                \label{tabParametriRegolaEmigrazioneTagliaForza}
            \end{table}

            \tikzfigure{p}
            {
% !TEX root = ../../../../Esperimenti/.tex/MEF.tex
% LTeX: language=it

\begin{tikzpicture}%
    %\begingroup Impostazioni
        %\begingroup Testo
            \fontSizeInfopage
            % % \renewcommand\titleSize{12}
            % \renewcommand\subCaptionSize8
            % \renewcommand\ParameterSize7
            % \renewcommand\labelSize7
            % \renewcommand\tickSize6
            % \renewcommand\legendSize6
        %\endgroup

        %\begingroup Dimensioni
            \pgfmathsetmacro\plotWidth7
            \pgfmathsetmacro\plotHeight{.618*\plotWidth}
            %
            \pgfmathparse{0.825*\plotWidth}
            \pgfmathsetmacro\plotWidth\pgfmathresult
            \pgfmathsetmacro\halfWidth{\plotWidth/2}
            \pgfmathsetmacro\plotLastWidth{1.25*\plotWidth}

            \pgfmathsetmacro\xPlotSep{1}
            \pgfmathsetmacro\yPlotSep{1.25}
            \pgfmathsetmacro\yGroupSep{2.75}
        %\endgroup

        %\begingroup Gruppi
            \Ifthispageodd{%
                \rectoGroupOptions
                % \pgfplotsset{/pgfplots/group/yDescPos/.style={y descriptions at=edge right}}
            }{%
                \versoGroupOptions
                % \pgfplotsset{/pgfplots/group/yDescPos/.style={y descriptions at=edge left}}
            }
            % \pgfplotsset{
            %     /pgfplots/group/plotGroup/.style={
            %         horizontal sep=\xPlotSep em,
            %         vertical sep=\yPlotSep em,
            %         yDescPos,
            %         group size=3 by 1,
            %     },
            %     /pgfplots/group/plotLastGroup/.style={
            %         plotGroup,
            %         x descriptions at=edge bottom,
            %     },
            % }
        %\endgroup

        %\begingroup Stili
            %\begingroup Selezione «recto»/«verso»
                \Ifthispageodd{%
                    \rectoPlotOptions
                    % \pgfplotsset{
                    %     yLineAxis/.style={axis y line*=right},
                    %     /tikz/yLabelShift/.style={
                    %         yshift=-2.5em,
                    %         at={(axis description cs:1,0.5)},
                    %         anchor=south,
                    %         % rotate=180,
                    %     },
                    %     /tikz/yScaleLabelShift/.style={
                    %         yshift=.25em,
                    %         at={(1,1)},
                    %         anchor=south east,
                    %     },
                    %     /tikz/yTickLabelShift/.style={xshift=-0.15em},
                    % }
                }{%
                    \versoPlotOptions
                    % \pgfplotsset{
                    %     yLineAxis/.style={axis y line*=left},
                    %     /tikz/yLabelShift/.style={
                    %         yshift=1.25em,
                    %         at={(axis description cs:0,0.5)},
                    %         anchor=south,
                    %         % rotate=180,
                    %     },
                    %     /tikz/yScaleLabelShift/.style={
                    %         yshift=.25em,
                    %         at={(0,1)},
                    %         anchor=south west,
                    %     },
                    %     /tikz/yTickLabelShift/.style={xshift=+0.15em},
                    % }
                }
            %\endgroup
            %
            %\begingroup Gruppi
                % \pgfplotsset{
                %     plotRow/.style={
                %         scale only axis,
                %         width=\plotWidth cm,height=\plotHeight cm,
                %         xlabel style={font=\color{white!15!black}\dimTesto{\labelSize}},
                %         ylabel style={font=\color{white!15!black}\dimTesto{\labelSize},yLabelShift},
                %         xticklabel style={font=\dimTesto{\tickSize}},
                %         yticklabel style={font=\dimTesto{\tickSize},yTickLabelShift},
                %         xlabel={\(s\)},ylabel={\(\underhat{\bar f}_N(s)\)},
                %         % axis x line*=bottom,yLineAxis,
                %         enlarge x limits=0.1,enlarge y limits=0.1,
                %         xmajorgrids,ymajorgrids,
                %         xminorticks=true,yminorticks=true,
                %         axis background/.style={fill=white},
                %         grid style={dashed}
                %     },
                % }
            %\endgroup
            %
            %\begingroup Estetica
                % \newcommand\smallminus{\scalebox{0.4}[1]{$-$}\hspace{-.125em}}
                % \pgfplotsset{
                %     /pgfplots/log number format basis/.code 2 args={%
                %         \pgfmathsetmacro\exponent{#2}%
                %         \pgfmathparse{sign(\exponent)}%
                %         \ifnum\pgfmathresult<0%
                %             \pgfmathparse{abs(\exponent)}%
                %             $#1^{\smash{\smallminus\pgfmathprintnumber{\pgfmathresult}}}$%
                %         \else%
                %             $#1^{\smash{\pgfmathprintnumber{\exponent}}}$%
                %         \fi%
                %     },
                %     ytick scale label code/.code={%
                %         \pgfmathsetmacro\exponent{#1}%
                %         \pgfmathparse{sign(\exponent)}%
                %         \ifnum\pgfmathresult<0
                %             \pgfmathparse{abs(\exponent)}%
                %             $\cdot 10^{\smash{\smallminus\pgfmathprintnumber{\pgfmathresult}}}$%
                %         \else
                %             $\cdot 10^{\smash{\pgfmathprintnumber{\exponent}}}$%
                %         \fi
                %     },
                % }
            %\endgroup
        %\endgroup

        %\begingroup Legende
            % \pgfplotsset{
            %     plotLegend/.style={
            %         font=\dimTesto{\legendSize},
            %         legend cell align=left,
            %         align=left,
            %         % draw=white!15!black,
            %         draw=none, % Bordo invisibile
            %         fill=white,
            %         fill opacity=0.5,
            %         text opacity=1,
            %     },
            %     plotLegendNW/.style={legend style={plotLegend,legend pos=north west}},
            %     plotLegendNE/.style={legend style={plotLegend,legend pos=north east}},
            %     plotLegendSW/.style={legend style={plotLegend,legend pos=south west}},
            %     plotLegendSE/.style={legend style={plotLegend,legend pos=south east}},
            % }
        %\endgroup
    %\endgroup

    %\begingroup Righe
        \begin{groupplot}[
            group style={
                group name=Row1,
                plotGroup,
            },
            plotRow,
            plotLegendNE,
            %
            enlarge x limits={lower,value=0.05},
            enlarge y limits={upper,value=.025},
            xmode=log,
            y tick scale label style={yScaleLabelShift},
            log origin=infty,
            % xmin=10,
            xmax=1e+06,
            ymin=0,ymax=0.0004,
        ]
            \nextgroupplot[%
                % xmin=10,xmax=1e+06,
                % ymin=0,ymax=0.00035,
            ] % This file was created by matlab2tikz.
%
\definecolor{mycolor1}{rgb}{0.00000,0.44706,0.69804}%
\definecolor{mycolor2}{rgb}{0.83529,0.36863,0.00000}%
%
\addplot[ybar interval, fill=mycolor1, fill opacity=0.35, area legend, draw=none] table[row sep=crcr, x=Lower, y=Count] {%
Lower	Upper	Count\\
27.667	39.83	2.1925e-06\\
39.83	57.339	7.6151e-06\\
57.339	82.545	2.4333e-05\\
82.545	118.83	3.8949e-05\\
118.83	171.07	8.3719e-05\\
171.07	246.27	0.000139\\
246.27	354.53	0.00020937\\
354.53	510.38	0.00027496\\
510.38	734.75	0.0003303\\
734.75	1057.7	0.00032909\\
1057.7	1522.7	0.00028549\\
1522.7	2192.1	0.00021791\\
2192.1	3155.8	0.00013787\\
3155.8	4543	7.8543e-05\\
4543	6540.1	3.7174e-05\\
6540.1	9415.1	1.7271e-05\\
9415.1	13554	8.5112e-06\\
13554	19512	3.934e-06\\
19512	28090	1.7783e-06\\
28090	40438	6.8458e-07\\
40438	58215	2.6702e-07\\
58215	83806	1.0212e-07\\
83806	1.2065e+05	1.5201e-08\\
1.2065e+05	1.7368e+05	3.9722e-08\\
1.7368e+05	2.5003e+05	4.5405e-09\\
2.5003e+05	2.5003e+05	4.5405e-09\\
};
\addlegendentry{Istogramma esatto}

\addplot [color=mycolor1, only marks, every error bar/.append style={opacity=0.45}, mark=*, mark size=0pt, draw=none, forget plot]
 plot [error bars/.cd, y dir=both, y explicit, error bar style={line width=1pt, color=mycolor1}, error mark options={mark=none,mark size=0pt}]
 table[row sep=crcr, y error plus index=2, y error minus index=3]{%
33.196	2.1925e-06	4.3504e-06	2.1925e-06\\
47.789	7.6151e-06	6.6194e-06	6.6194e-06\\
68.797	2.4333e-05	9.8302e-06	9.8302e-06\\
99.041	3.8949e-05	1.1624e-05	1.1624e-05\\
142.58	8.3719e-05	1.1107e-05	1.1107e-05\\
205.26	0.000139	1.2478e-05	1.2478e-05\\
295.49	0.00020937	1.4397e-05	1.4397e-05\\
425.38	0.00027496	1.2896e-05	1.2896e-05\\
612.38	0.0003303	1.3472e-05	1.3472e-05\\
881.58	0.00032909	1.0604e-05	1.0604e-05\\
1269.1	0.00028549	7.4112e-06	7.4112e-06\\
1827	0.00021791	4.8252e-06	4.8252e-06\\
2630.2	0.00013787	3.6233e-06	3.6233e-06\\
3786.4	7.8543e-05	2.1968e-06	2.1968e-06\\
5450.9	3.7174e-05	1.306e-06	1.306e-06\\
7847	1.7271e-05	6.401e-07	6.401e-07\\
11297	8.5112e-06	4.1607e-07	4.1607e-07\\
16263	3.934e-06	2.0325e-07	2.0325e-07\\
23411	1.7783e-06	1.1733e-07	1.1733e-07\\
33703	6.8458e-07	6.3602e-08	6.3602e-08\\
48519	2.6702e-07	3.0968e-08	3.0968e-08\\
69848	1.0212e-07	1.5545e-08	1.5545e-08\\
1.0055e+05	1.5201e-08	6.8607e-09	6.8607e-09\\
1.4476e+05	3.9722e-08	4.0841e-09	4.0841e-09\\
2.0839e+05	4.5405e-09	2.3424e-09	2.3424e-09\\
};
\addplot [color=mycolor1]
  table[row sep=crcr]{%
27.667	5.7757e-06\\
31.435	7.1447e-06\\
35.391	8.6843e-06\\
39.483	1.0382e-05\\
43.648	1.2217e-05\\
47.815	1.4156e-05\\
51.905	1.6158e-05\\
56.344	1.8439e-05\\
60.608	2.0732e-05\\
65.194	2.3307e-05\\
69.491	2.5817e-05\\
74.071	2.8594e-05\\
78.953	3.1662e-05\\
83.393	3.4544e-05\\
88.082	3.7679e-05\\
93.035	4.1085e-05\\
98.267	4.4781e-05\\
103.79	4.8786e-05\\
108.63	5.2373e-05\\
113.7	5.6199e-05\\
119.01	6.0274e-05\\
124.56	6.4608e-05\\
130.37	6.9211e-05\\
136.45	7.4092e-05\\
142.81	7.9257e-05\\
149.47	8.4714e-05\\
156.45	9.0467e-05\\
163.74	9.6521e-05\\
171.38	0.00010288\\
179.38	0.00010954\\
187.74	0.00011649\\
196.5	0.00012375\\
205.67	0.00013129\\
215.26	0.00013911\\
225.3	0.00014719\\
235.81	0.00015552\\
249.07	0.00016582\\
263.08	0.00017641\\
280.42	0.00018907\\
301.64	0.00020383\\
382.34	0.00025209\\
403.84	0.00026281\\
422.67	0.0002715\\
442.39	0.00027989\\
458.82	0.00028636\\
475.87	0.00029258\\
493.54	0.00029852\\
511.88	0.00030416\\
530.89	0.00030946\\
545.61	0.0003132\\
560.74	0.00031673\\
576.29	0.00032004\\
592.28	0.00032311\\
608.7	0.00032594\\
625.58	0.00032852\\
642.93	0.00033084\\
660.76	0.00033288\\
679.09	0.00033465\\
697.92	0.00033614\\
717.27	0.00033734\\
737.16	0.00033824\\
757.61	0.00033884\\
778.62	0.00033915\\
800.21	0.00033914\\
822.4	0.00033883\\
845.21	0.00033821\\
868.65	0.00033729\\
892.74	0.00033606\\
917.5	0.00033452\\
942.94	0.00033268\\
969.09	0.00033055\\
995.96	0.00032812\\
1023.6	0.00032541\\
1052	0.00032242\\
1081.1	0.00031915\\
1111.1	0.00031563\\
1141.9	0.00031184\\
1173.6	0.00030781\\
1206.2	0.00030355\\
1239.6	0.00029906\\
1274	0.00029436\\
1321.3	0.00028778\\
1370.4	0.00028088\\
1421.3	0.00027369\\
1474.1	0.00026624\\
1542.8	0.0002566\\
1614.8	0.00024668\\
1705.6	0.00023448\\
1834.7	0.00021789\\
2142.3	0.00018255\\
2262.8	0.00017036\\
2390	0.00015845\\
2501.5	0.00014879\\
2618.2	0.00013941\\
2740.3	0.00013034\\
2842.1	0.00012333\\
2947.7	0.00011654\\
3057.2	0.00010998\\
3170.8	0.00010367\\
3288.5	9.7602e-05\\
3410.7	9.1784e-05\\
3537.4	8.6216e-05\\
3668.8	8.0899e-05\\
3805.1	7.5832e-05\\
3946.4	7.1011e-05\\
4093	6.6434e-05\\
4245	6.2096e-05\\
4402.7	5.7991e-05\\
4566.3	5.4113e-05\\
4735.9	5.0455e-05\\
4956.8	4.618e-05\\
5188	4.2222e-05\\
5430	3.8566e-05\\
5683.3	3.5194e-05\\
5948.4	3.209e-05\\
6225.9	2.9238e-05\\
6516.3	2.6622e-05\\
6882.7	2.377e-05\\
7269.8	2.1207e-05\\
7678.6	1.8909e-05\\
8110.4	1.685e-05\\
8644.9	1.4721e-05\\
9214.7	1.2854e-05\\
9912	1.1004e-05\\
10662	9.4154e-06\\
11574	7.8976e-06\\
12679	6.4932e-06\\
14017	5.2318e-06\\
15637	4.1294e-06\\
17766	3.1276e-06\\
20557	2.2689e-06\\
24446	1.5395e-06\\
29876	9.7051e-07\\
38565	5.2623e-07\\
54040	2.224e-07\\
90874	5.1572e-08\\
2.5003e+05	1.842e-09\\
};
\addlegendentry{Adatt. BLN esatto ($\mathit{ML}$)}


\addplot[area legend, draw=none, fill=mycolor1, fill opacity=0.15, forget plot]
table[row sep=crcr] {%
x	y\\
27.667	7.3893e-06\\
27.921	7.4978e-06\\
28.177	7.6077e-06\\
28.435	7.7191e-06\\
28.695	7.832e-06\\
28.958	7.9463e-06\\
29.223	8.0622e-06\\
29.491	8.1796e-06\\
29.761	8.2986e-06\\
30.034	8.4191e-06\\
30.309	8.5412e-06\\
30.586	8.6649e-06\\
30.867	8.7902e-06\\
31.149	8.9172e-06\\
31.435	9.0459e-06\\
31.723	9.1762e-06\\
32.013	9.3083e-06\\
32.306	9.4421e-06\\
32.602	9.5776e-06\\
32.901	9.715e-06\\
33.202	9.8541e-06\\
33.506	9.995e-06\\
33.813	1.0138e-05\\
34.123	1.0283e-05\\
34.436	1.0429e-05\\
34.751	1.0578e-05\\
35.069	1.0728e-05\\
35.391	1.088e-05\\
35.715	1.1035e-05\\
36.042	1.1191e-05\\
36.372	1.135e-05\\
36.705	1.151e-05\\
37.042	1.1673e-05\\
37.381	1.1838e-05\\
37.723	1.2005e-05\\
38.069	1.2174e-05\\
38.417	1.2345e-05\\
38.769	1.2518e-05\\
39.124	1.2694e-05\\
39.483	1.2872e-05\\
39.845	1.3053e-05\\
40.209	1.3236e-05\\
40.578	1.3421e-05\\
40.949	1.3608e-05\\
41.325	1.3799e-05\\
41.703	1.3991e-05\\
42.085	1.4186e-05\\
42.471	1.4384e-05\\
42.86	1.4584e-05\\
43.252	1.4787e-05\\
43.648	1.4992e-05\\
44.048	1.5201e-05\\
44.452	1.5412e-05\\
44.859	1.5625e-05\\
45.27	1.5842e-05\\
45.684	1.6061e-05\\
46.103	1.6283e-05\\
46.525	1.6508e-05\\
46.951	1.6737e-05\\
47.381	1.6968e-05\\
47.815	1.7202e-05\\
48.253	1.7439e-05\\
48.695	1.7679e-05\\
49.141	1.7923e-05\\
49.592	1.8169e-05\\
50.046	1.8419e-05\\
50.504	1.8673e-05\\
50.967	1.8929e-05\\
51.434	1.9189e-05\\
51.905	1.9452e-05\\
52.38	1.9719e-05\\
52.86	1.9989e-05\\
53.344	2.0263e-05\\
53.833	2.0541e-05\\
54.326	2.0822e-05\\
54.824	2.1107e-05\\
55.326	2.1395e-05\\
55.833	2.1688e-05\\
56.344	2.1984e-05\\
56.86	2.2284e-05\\
57.381	2.2588e-05\\
57.907	2.2896e-05\\
58.437	2.3209e-05\\
58.972	2.3525e-05\\
59.512	2.3846e-05\\
60.058	2.417e-05\\
60.608	2.4499e-05\\
61.163	2.4833e-05\\
61.723	2.517e-05\\
62.289	2.5513e-05\\
62.859	2.5859e-05\\
63.435	2.6211e-05\\
64.016	2.6567e-05\\
64.602	2.6928e-05\\
65.194	2.7293e-05\\
65.791	2.7663e-05\\
66.394	2.8038e-05\\
67.002	2.8419e-05\\
67.616	2.8804e-05\\
68.235	2.9194e-05\\
68.86	2.9589e-05\\
69.491	2.999e-05\\
70.127	3.0396e-05\\
70.77	3.0807e-05\\
71.418	3.1223e-05\\
72.072	3.1646e-05\\
72.732	3.2073e-05\\
73.399	3.2506e-05\\
74.071	3.2945e-05\\
74.749	3.339e-05\\
75.434	3.3841e-05\\
76.125	3.4297e-05\\
76.822	3.476e-05\\
77.526	3.5228e-05\\
78.236	3.5703e-05\\
78.953	3.6184e-05\\
79.676	3.6671e-05\\
80.406	3.7164e-05\\
81.142	3.7664e-05\\
81.886	3.8171e-05\\
82.636	3.8684e-05\\
83.393	3.9203e-05\\
84.156	3.973e-05\\
84.927	4.0263e-05\\
85.705	4.0803e-05\\
86.49	4.135e-05\\
87.283	4.1904e-05\\
88.082	4.2466e-05\\
88.889	4.3034e-05\\
89.703	4.361e-05\\
90.525	4.4193e-05\\
91.354	4.4784e-05\\
92.191	4.5382e-05\\
93.035	4.5988e-05\\
93.887	4.6601e-05\\
94.747	4.7223e-05\\
95.615	4.7852e-05\\
96.491	4.8489e-05\\
97.375	4.9134e-05\\
98.267	4.9788e-05\\
99.167	5.0449e-05\\
100.08	5.1119e-05\\
100.99	5.1798e-05\\
101.92	5.2484e-05\\
102.85	5.318e-05\\
103.79	5.3883e-05\\
104.74	5.4596e-05\\
105.7	5.5318e-05\\
106.67	5.6048e-05\\
107.65	5.6787e-05\\
108.63	5.7535e-05\\
109.63	5.8293e-05\\
110.63	5.906e-05\\
111.65	5.9836e-05\\
112.67	6.0621e-05\\
113.7	6.1416e-05\\
114.74	6.222e-05\\
115.79	6.3034e-05\\
116.85	6.3858e-05\\
117.93	6.4691e-05\\
119.01	6.5535e-05\\
120.1	6.6388e-05\\
121.2	6.7251e-05\\
122.31	6.8124e-05\\
123.43	6.9008e-05\\
124.56	6.9902e-05\\
125.7	7.0806e-05\\
126.85	7.172e-05\\
128.01	7.2645e-05\\
129.18	7.3581e-05\\
130.37	7.4527e-05\\
131.56	7.5484e-05\\
132.77	7.6451e-05\\
133.98	7.743e-05\\
135.21	7.8419e-05\\
136.45	7.9419e-05\\
137.7	8.043e-05\\
138.96	8.1452e-05\\
140.23	8.2486e-05\\
141.52	8.353e-05\\
142.81	8.4586e-05\\
144.12	8.5653e-05\\
145.44	8.6731e-05\\
146.77	8.7821e-05\\
148.12	8.8922e-05\\
149.47	9.0035e-05\\
150.84	9.1159e-05\\
152.23	9.2295e-05\\
153.62	9.3442e-05\\
155.03	9.4601e-05\\
156.45	9.5772e-05\\
157.88	9.6955e-05\\
159.33	9.8149e-05\\
160.79	9.9355e-05\\
162.26	0.00010057\\
163.74	0.0001018\\
165.24	0.00010304\\
166.76	0.0001043\\
168.29	0.00010556\\
169.83	0.00010684\\
171.38	0.00010813\\
172.95	0.00010943\\
174.54	0.00011074\\
176.14	0.00011207\\
177.75	0.0001134\\
179.38	0.00011475\\
181.02	0.00011611\\
182.68	0.00011749\\
184.35	0.00011887\\
186.04	0.00012027\\
187.74	0.00012167\\
189.46	0.00012309\\
191.2	0.00012453\\
192.95	0.00012597\\
194.72	0.00012742\\
196.5	0.00012889\\
198.3	0.00013037\\
200.12	0.00013186\\
201.95	0.00013336\\
203.8	0.00013487\\
205.67	0.00013639\\
207.55	0.00013793\\
209.45	0.00013947\\
211.37	0.00014103\\
213.31	0.0001426\\
215.26	0.00014418\\
217.23	0.00014577\\
219.22	0.00014737\\
221.23	0.00014898\\
223.26	0.0001506\\
225.3	0.00015223\\
227.37	0.00015388\\
229.45	0.00015553\\
231.55	0.00015719\\
233.67	0.00015886\\
235.81	0.00016054\\
237.97	0.00016223\\
240.15	0.00016393\\
242.35	0.00016564\\
244.57	0.00016736\\
246.81	0.00016909\\
249.07	0.00017083\\
251.35	0.00017257\\
253.66	0.00017432\\
255.98	0.00017608\\
258.32	0.00017785\\
260.69	0.00017963\\
263.08	0.00018141\\
265.49	0.0001832\\
267.92	0.000185\\
270.37	0.0001868\\
272.85	0.00018861\\
275.35	0.00019043\\
277.87	0.00019225\\
280.42	0.00019408\\
282.99	0.00019591\\
285.58	0.00019775\\
288.19	0.00019959\\
290.83	0.00020144\\
293.5	0.00020329\\
296.19	0.00020515\\
298.9	0.00020701\\
301.64	0.00020887\\
304.4	0.00021074\\
307.19	0.0002126\\
310	0.00021448\\
312.84	0.00021635\\
315.71	0.00021822\\
318.6	0.0002201\\
321.52	0.00022197\\
324.46	0.00022385\\
327.44	0.00022572\\
330.43	0.0002276\\
333.46	0.00022948\\
336.52	0.00023135\\
339.6	0.00023322\\
342.71	0.00023509\\
345.85	0.00023696\\
349.02	0.00023883\\
352.21	0.00024069\\
355.44	0.00024255\\
358.7	0.00024441\\
361.98	0.00024626\\
365.3	0.00024811\\
368.64	0.00024995\\
372.02	0.00025179\\
375.43	0.00025362\\
378.87	0.00025544\\
382.34	0.00025726\\
385.84	0.00025907\\
389.37	0.00026087\\
392.94	0.00026267\\
396.54	0.00026445\\
400.17	0.00026623\\
403.84	0.000268\\
407.54	0.00026976\\
411.27	0.0002715\\
415.04	0.00027324\\
418.84	0.00027497\\
422.67	0.00027668\\
426.55	0.00027838\\
430.45	0.00028007\\
434.4	0.00028175\\
438.37	0.00028341\\
442.39	0.00028506\\
446.44	0.00028669\\
450.53	0.00028831\\
454.66	0.00028992\\
458.82	0.00029151\\
463.03	0.00029308\\
467.27	0.00029464\\
471.55	0.00029617\\
475.87	0.0002977\\
480.23	0.0002992\\
484.62	0.00030068\\
489.06	0.00030215\\
493.54	0.0003036\\
498.06	0.00030503\\
502.63	0.00030643\\
507.23	0.00030782\\
511.88	0.00030919\\
516.56	0.00031053\\
521.3	0.00031185\\
526.07	0.00031316\\
530.89	0.00031443\\
535.75	0.00031569\\
540.66	0.00031692\\
545.61	0.00031813\\
550.61	0.00031931\\
555.65	0.00032047\\
560.74	0.00032161\\
565.88	0.00032272\\
571.06	0.0003238\\
576.29	0.00032486\\
581.57	0.00032589\\
586.9	0.0003269\\
592.28	0.00032788\\
597.7	0.00032883\\
603.18	0.00032975\\
608.7	0.00033065\\
614.28	0.00033151\\
619.9	0.00033235\\
625.58	0.00033316\\
631.31	0.00033394\\
637.1	0.00033469\\
642.93	0.00033542\\
648.82	0.00033611\\
654.76	0.00033677\\
660.76	0.0003374\\
666.81	0.000338\\
672.92	0.00033857\\
679.09	0.0003391\\
685.31	0.00033961\\
691.58	0.00034008\\
697.92	0.00034052\\
704.31	0.00034093\\
710.76	0.00034131\\
717.27	0.00034165\\
723.84	0.00034197\\
730.47	0.00034224\\
737.16	0.00034249\\
743.92	0.0003427\\
750.73	0.00034288\\
757.61	0.00034303\\
764.55	0.00034314\\
771.55	0.00034322\\
778.62	0.00034326\\
785.75	0.00034327\\
792.95	0.00034325\\
800.21	0.00034319\\
807.54	0.0003431\\
814.94	0.00034297\\
822.4	0.00034281\\
829.94	0.00034261\\
837.54	0.00034239\\
845.21	0.00034212\\
852.95	0.00034182\\
860.76	0.00034149\\
868.65	0.00034112\\
876.61	0.00034072\\
884.64	0.00034029\\
892.74	0.00033982\\
900.92	0.00033932\\
909.17	0.00033878\\
917.5	0.00033821\\
925.9	0.0003376\\
934.38	0.00033697\\
942.94	0.00033629\\
951.58	0.00033559\\
960.29	0.00033485\\
969.09	0.00033408\\
977.97	0.00033328\\
986.92	0.00033244\\
995.96	0.00033157\\
1005.1	0.00033067\\
1014.3	0.00032974\\
1023.6	0.00032877\\
1033	0.00032778\\
1042.4	0.00032675\\
1052	0.00032569\\
1061.6	0.00032461\\
1071.3	0.00032349\\
1081.1	0.00032234\\
1091	0.00032116\\
1101	0.00031996\\
1111.1	0.00031872\\
1121.3	0.00031746\\
1131.6	0.00031617\\
1141.9	0.00031485\\
1152.4	0.0003135\\
1163	0.00031213\\
1173.6	0.00031073\\
1184.4	0.00030931\\
1195.2	0.00030786\\
1206.2	0.00030638\\
1217.2	0.00030488\\
1228.4	0.00030336\\
1239.6	0.00030181\\
1251	0.00030024\\
1262.4	0.00029864\\
1274	0.00029702\\
1285.7	0.00029539\\
1297.4	0.00029373\\
1309.3	0.00029205\\
1321.3	0.00029035\\
1333.4	0.00028862\\
1345.6	0.00028688\\
1357.9	0.00028513\\
1370.4	0.00028335\\
1382.9	0.00028155\\
1395.6	0.00027974\\
1408.4	0.00027791\\
1421.3	0.00027607\\
1434.3	0.00027421\\
1447.4	0.00027233\\
1460.7	0.00027044\\
1474.1	0.00026853\\
1487.6	0.00026661\\
1501.2	0.00026468\\
1515	0.00026274\\
1528.8	0.00026078\\
1542.8	0.00025881\\
1557	0.00025683\\
1571.2	0.00025484\\
1585.6	0.00025284\\
1600.2	0.00025083\\
1614.8	0.00024881\\
1629.6	0.00024678\\
1644.5	0.00024474\\
1659.6	0.0002427\\
1674.8	0.00024065\\
1690.1	0.00023859\\
1705.6	0.00023653\\
1721.2	0.00023446\\
1737	0.00023238\\
1752.9	0.0002303\\
1769	0.00022821\\
1785.2	0.00022613\\
1801.5	0.00022403\\
1818	0.00022194\\
1834.7	0.00021984\\
1851.5	0.00021774\\
1868.5	0.00021564\\
1885.6	0.00021354\\
1902.8	0.00021144\\
1920.3	0.00020934\\
1937.9	0.00020723\\
1955.6	0.00020513\\
1973.5	0.00020303\\
1991.6	0.00020093\\
2009.8	0.00019884\\
2028.3	0.00019675\\
2046.8	0.00019465\\
2065.6	0.00019257\\
2084.5	0.00019048\\
2103.6	0.0001884\\
2122.9	0.00018633\\
2142.3	0.00018426\\
2161.9	0.0001822\\
2181.7	0.00018014\\
2201.7	0.00017809\\
2221.9	0.00017604\\
2242.2	0.000174\\
2262.8	0.00017197\\
2283.5	0.00016995\\
2304.4	0.00016793\\
2325.5	0.00016592\\
2346.8	0.00016393\\
2368.3	0.00016194\\
2390	0.00015996\\
2411.9	0.00015799\\
2434	0.00015603\\
2456.3	0.00015408\\
2478.8	0.00015214\\
2501.5	0.00015021\\
2524.4	0.0001483\\
2547.6	0.00014639\\
2570.9	0.0001445\\
2594.4	0.00014262\\
2618.2	0.00014075\\
2642.2	0.00013889\\
2666.4	0.00013705\\
2690.8	0.00013522\\
2715.5	0.0001334\\
2740.3	0.0001316\\
2765.4	0.00012981\\
2790.8	0.00012803\\
2816.3	0.00012627\\
2842.1	0.00012452\\
2868.2	0.00012278\\
2894.4	0.00012106\\
2920.9	0.00011936\\
2947.7	0.00011767\\
2974.7	0.00011599\\
3001.9	0.00011433\\
3029.4	0.00011269\\
3057.2	0.00011106\\
3085.2	0.00010944\\
3113.5	0.00010784\\
3142	0.00010626\\
3170.8	0.00010469\\
3199.8	0.00010314\\
3229.1	0.0001016\\
3258.7	0.00010008\\
3288.5	9.8574e-05\\
3318.7	9.7084e-05\\
3349.1	9.561e-05\\
3379.7	9.4152e-05\\
3410.7	9.271e-05\\
3441.9	9.1283e-05\\
3473.5	8.9873e-05\\
3505.3	8.8478e-05\\
3537.4	8.7099e-05\\
3569.8	8.5736e-05\\
3602.5	8.4389e-05\\
3635.5	8.3058e-05\\
3668.8	8.1742e-05\\
3702.4	8.0442e-05\\
3736.3	7.9158e-05\\
3770.5	7.789e-05\\
3805.1	7.6637e-05\\
3839.9	7.54e-05\\
3875.1	7.4178e-05\\
3910.6	7.2972e-05\\
3946.4	7.1781e-05\\
3982.6	7.0605e-05\\
4019	6.9444e-05\\
4055.9	6.8299e-05\\
4093	6.7169e-05\\
4130.5	6.6054e-05\\
4168.3	6.4953e-05\\
4206.5	6.3868e-05\\
4245	6.2797e-05\\
4283.9	6.1741e-05\\
4323.2	6.0699e-05\\
4362.8	5.9672e-05\\
4402.7	5.8659e-05\\
4443.1	5.766e-05\\
4483.8	5.6675e-05\\
4524.8	5.5705e-05\\
4566.3	5.4748e-05\\
4608.1	5.3805e-05\\
4650.3	5.2876e-05\\
4692.9	5.196e-05\\
4735.9	5.1058e-05\\
4779.3	5.0169e-05\\
4823.1	4.9293e-05\\
4867.2	4.8431e-05\\
4911.8	4.7581e-05\\
4956.8	4.6744e-05\\
5002.2	4.592e-05\\
5048	4.5108e-05\\
5094.3	4.4309e-05\\
5140.9	4.3523e-05\\
5188	4.2748e-05\\
5235.5	4.1986e-05\\
5283.5	4.1235e-05\\
5331.9	4.0496e-05\\
5380.7	3.9769e-05\\
5430	3.9054e-05\\
5479.8	3.835e-05\\
5530	3.7657e-05\\
5580.6	3.6976e-05\\
5631.7	3.6306e-05\\
5683.3	3.5646e-05\\
5735.4	3.4997e-05\\
5787.9	3.4359e-05\\
5840.9	3.3732e-05\\
5894.4	3.3114e-05\\
5948.4	3.2507e-05\\
6002.9	3.1911e-05\\
6057.9	3.1324e-05\\
6113.4	3.0747e-05\\
6169.4	3.018e-05\\
6225.9	2.9622e-05\\
6282.9	2.9074e-05\\
6340.5	2.8536e-05\\
6398.5	2.8006e-05\\
6457.2	2.7486e-05\\
6516.3	2.6974e-05\\
6576	2.6472e-05\\
6636.2	2.5978e-05\\
6697	2.5493e-05\\
6758.4	2.5016e-05\\
6820.3	2.4548e-05\\
6882.7	2.4087e-05\\
6945.8	2.3635e-05\\
7009.4	2.3191e-05\\
7073.6	2.2755e-05\\
7138.4	2.2327e-05\\
7203.8	2.1906e-05\\
7269.8	2.1493e-05\\
7336.4	2.1087e-05\\
7403.6	2.0689e-05\\
7471.4	2.0297e-05\\
7539.8	1.9913e-05\\
7608.9	1.9536e-05\\
7678.6	1.9165e-05\\
7748.9	1.8801e-05\\
7819.9	1.8444e-05\\
7891.5	1.8094e-05\\
7963.8	1.775e-05\\
8036.8	1.7412e-05\\
8110.4	1.708e-05\\
8184.7	1.6755e-05\\
8259.6	1.6435e-05\\
8335.3	1.6122e-05\\
8411.6	1.5814e-05\\
8488.7	1.5512e-05\\
8566.5	1.5215e-05\\
8644.9	1.4924e-05\\
8724.1	1.4639e-05\\
8804	1.4359e-05\\
8884.7	1.4084e-05\\
8966	1.3814e-05\\
9048.2	1.3549e-05\\
9131.1	1.3289e-05\\
9214.7	1.3035e-05\\
9299.1	1.2784e-05\\
9384.3	1.2539e-05\\
9470.2	1.2298e-05\\
9557	1.2062e-05\\
9644.5	1.183e-05\\
9732.9	1.1603e-05\\
9822	1.138e-05\\
9912	1.1161e-05\\
10003	1.0946e-05\\
10094	1.0736e-05\\
10187	1.0529e-05\\
10280	1.0326e-05\\
10374	1.0128e-05\\
10469	9.9326e-06\\
10565	9.7412e-06\\
10662	9.5535e-06\\
10760	9.3694e-06\\
10858	9.1888e-06\\
10958	9.0116e-06\\
11058	8.8378e-06\\
11159	8.6674e-06\\
11262	8.5002e-06\\
11365	8.3361e-06\\
11469	8.1752e-06\\
11574	8.0174e-06\\
11680	7.8626e-06\\
11787	7.7108e-06\\
11895	7.5618e-06\\
12004	7.4157e-06\\
12114	7.2724e-06\\
12225	7.1318e-06\\
12337	6.994e-06\\
12450	6.8587e-06\\
12564	6.7261e-06\\
12679	6.5959e-06\\
12795	6.4683e-06\\
12912	6.3431e-06\\
13030	6.2203e-06\\
13150	6.0998e-06\\
13270	5.9817e-06\\
13392	5.8657e-06\\
13515	5.752e-06\\
13638	5.6405e-06\\
13763	5.5311e-06\\
13889	5.4238e-06\\
14017	5.3186e-06\\
14145	5.2153e-06\\
14274	5.114e-06\\
14405	5.0147e-06\\
14537	4.9172e-06\\
14670	4.8216e-06\\
14805	4.7278e-06\\
14940	4.6358e-06\\
15077	4.5456e-06\\
15215	4.457e-06\\
15355	4.3702e-06\\
15495	4.285e-06\\
15637	4.2014e-06\\
15780	4.1194e-06\\
15925	4.039e-06\\
16071	3.9601e-06\\
16218	3.8827e-06\\
16367	3.8068e-06\\
16517	3.7323e-06\\
16668	3.6592e-06\\
16821	3.5875e-06\\
16975	3.5172e-06\\
17130	3.4482e-06\\
17287	3.3805e-06\\
17445	3.3141e-06\\
17605	3.2489e-06\\
17766	3.185e-06\\
17929	3.1223e-06\\
18093	3.0608e-06\\
18259	3.0005e-06\\
18426	2.9412e-06\\
18595	2.8832e-06\\
18765	2.8262e-06\\
18937	2.7703e-06\\
19111	2.7154e-06\\
19286	2.6616e-06\\
19463	2.6088e-06\\
19641	2.557e-06\\
19821	2.5062e-06\\
20002	2.4564e-06\\
20186	2.4075e-06\\
20370	2.3595e-06\\
20557	2.3124e-06\\
20745	2.2662e-06\\
20935	2.2209e-06\\
21127	2.1765e-06\\
21321	2.1329e-06\\
21516	2.0901e-06\\
21713	2.0481e-06\\
21912	2.0069e-06\\
22113	1.9666e-06\\
22315	1.9269e-06\\
22520	1.8881e-06\\
22726	1.8499e-06\\
22934	1.8125e-06\\
23144	1.7758e-06\\
23356	1.7398e-06\\
23570	1.7045e-06\\
23786	1.6698e-06\\
24004	1.6358e-06\\
24224	1.6025e-06\\
24446	1.5698e-06\\
24669	1.5377e-06\\
24895	1.5063e-06\\
25123	1.4754e-06\\
25354	1.4451e-06\\
25586	1.4154e-06\\
25820	1.3863e-06\\
26057	1.3577e-06\\
26295	1.3297e-06\\
26536	1.3022e-06\\
26779	1.2752e-06\\
27025	1.2488e-06\\
27272	1.2229e-06\\
27522	1.1974e-06\\
27774	1.1725e-06\\
28028	1.148e-06\\
28285	1.124e-06\\
28544	1.1005e-06\\
28806	1.0774e-06\\
29070	1.0548e-06\\
29336	1.0326e-06\\
29605	1.0109e-06\\
29876	9.8953e-07\\
30149	9.6861e-07\\
30426	9.481e-07\\
30704	9.28e-07\\
30986	9.0828e-07\\
31269	8.8895e-07\\
31556	8.7e-07\\
31845	8.5143e-07\\
32137	8.3321e-07\\
32431	8.1536e-07\\
32728	7.9786e-07\\
33028	7.807e-07\\
33330	7.6388e-07\\
33636	7.474e-07\\
33944	7.3124e-07\\
34255	7.154e-07\\
34568	6.9988e-07\\
34885	6.8467e-07\\
35205	6.6976e-07\\
35527	6.5515e-07\\
35852	6.4084e-07\\
36181	6.2681e-07\\
36512	6.1306e-07\\
36847	5.9959e-07\\
37184	5.8639e-07\\
37525	5.7346e-07\\
37869	5.6079e-07\\
38215	5.4837e-07\\
38565	5.3621e-07\\
38919	5.243e-07\\
39275	5.1263e-07\\
39635	5.012e-07\\
39998	4.9e-07\\
40364	4.7903e-07\\
40734	4.6829e-07\\
41107	4.5777e-07\\
41484	4.4747e-07\\
41864	4.3738e-07\\
42247	4.275e-07\\
42634	4.1782e-07\\
43025	4.0835e-07\\
43419	3.9907e-07\\
43817	3.8999e-07\\
44218	3.8109e-07\\
44623	3.7239e-07\\
45032	3.6386e-07\\
45444	3.5552e-07\\
45860	3.4735e-07\\
46281	3.3936e-07\\
46704	3.3153e-07\\
47132	3.2387e-07\\
47564	3.1638e-07\\
48000	3.0904e-07\\
48439	3.0186e-07\\
48883	2.9484e-07\\
49331	2.8796e-07\\
49783	2.8123e-07\\
50239	2.7465e-07\\
50699	2.6821e-07\\
51163	2.6191e-07\\
51632	2.5575e-07\\
52105	2.4972e-07\\
52582	2.4382e-07\\
53064	2.3805e-07\\
53550	2.324e-07\\
54040	2.2688e-07\\
54535	2.2148e-07\\
55035	2.162e-07\\
55539	2.1104e-07\\
56048	2.0599e-07\\
56561	2.0105e-07\\
57079	1.9622e-07\\
57602	1.915e-07\\
58130	1.8688e-07\\
58662	1.8237e-07\\
59200	1.7795e-07\\
59742	1.7364e-07\\
60289	1.6942e-07\\
60841	1.653e-07\\
61399	1.6127e-07\\
61961	1.5733e-07\\
62529	1.5348e-07\\
63101	1.4972e-07\\
63679	1.4604e-07\\
64263	1.4244e-07\\
64851	1.3893e-07\\
65445	1.355e-07\\
66045	1.3215e-07\\
66650	1.2887e-07\\
67260	1.2567e-07\\
67876	1.2254e-07\\
68498	1.1949e-07\\
69125	1.165e-07\\
69759	1.1359e-07\\
70398	1.1074e-07\\
71042	1.0796e-07\\
71693	1.0524e-07\\
72350	1.0259e-07\\
73013	9.9998e-08\\
73681	9.7467e-08\\
74356	9.4996e-08\\
75037	9.2583e-08\\
75725	9.0227e-08\\
76418	8.7926e-08\\
77118	8.568e-08\\
77825	8.3487e-08\\
78538	8.1347e-08\\
79257	7.9257e-08\\
79983	7.7217e-08\\
80716	7.5226e-08\\
81455	7.3283e-08\\
82201	7.1386e-08\\
82954	6.9535e-08\\
83714	6.7729e-08\\
84481	6.5966e-08\\
85255	6.4246e-08\\
86036	6.2568e-08\\
86824	6.093e-08\\
87619	5.9333e-08\\
88421	5.7774e-08\\
89231	5.6254e-08\\
90049	5.477e-08\\
90874	5.3324e-08\\
91706	5.1912e-08\\
92546	5.0536e-08\\
93394	4.9194e-08\\
94249	4.7885e-08\\
95113	4.6608e-08\\
95984	4.5363e-08\\
96863	4.4149e-08\\
97750	4.2966e-08\\
98646	4.1812e-08\\
99549	4.0687e-08\\
1.0046e+05	3.959e-08\\
1.0138e+05	3.8521e-08\\
1.0231e+05	3.7479e-08\\
1.0325e+05	3.6463e-08\\
1.0419e+05	3.5473e-08\\
1.0515e+05	3.4508e-08\\
1.0611e+05	3.3568e-08\\
1.0708e+05	3.2652e-08\\
1.0806e+05	3.1759e-08\\
1.0905e+05	3.0889e-08\\
1.1005e+05	3.0041e-08\\
1.1106e+05	2.9215e-08\\
1.1208e+05	2.841e-08\\
1.131e+05	2.7626e-08\\
1.1414e+05	2.6863e-08\\
1.1519e+05	2.6119e-08\\
1.1624e+05	2.5394e-08\\
1.1731e+05	2.4689e-08\\
1.1838e+05	2.4001e-08\\
1.1946e+05	2.3332e-08\\
1.2056e+05	2.268e-08\\
1.2166e+05	2.2045e-08\\
1.2278e+05	2.1427e-08\\
1.239e+05	2.0826e-08\\
1.2504e+05	2.024e-08\\
1.2618e+05	1.9669e-08\\
1.2734e+05	1.9114e-08\\
1.285e+05	1.8574e-08\\
1.2968e+05	1.8048e-08\\
1.3087e+05	1.7535e-08\\
1.3207e+05	1.7037e-08\\
1.3328e+05	1.6552e-08\\
1.345e+05	1.608e-08\\
1.3573e+05	1.562e-08\\
1.3697e+05	1.5173e-08\\
1.3823e+05	1.4738e-08\\
1.3949e+05	1.4315e-08\\
1.4077e+05	1.3903e-08\\
1.4206e+05	1.3503e-08\\
1.4336e+05	1.3113e-08\\
1.4468e+05	1.2734e-08\\
1.46e+05	1.2365e-08\\
1.4734e+05	1.2007e-08\\
1.4869e+05	1.1658e-08\\
1.5005e+05	1.1319e-08\\
1.5142e+05	1.0989e-08\\
1.5281e+05	1.0668e-08\\
1.5421e+05	1.0356e-08\\
1.5562e+05	1.0053e-08\\
1.5705e+05	9.7577e-09\\
1.5849e+05	9.4709e-09\\
1.5994e+05	9.1921e-09\\
1.614e+05	8.9211e-09\\
1.6288e+05	8.6577e-09\\
1.6438e+05	8.4016e-09\\
1.6588e+05	8.1527e-09\\
1.674e+05	7.9108e-09\\
1.6893e+05	7.6757e-09\\
1.7048e+05	7.4473e-09\\
1.7204e+05	7.2252e-09\\
1.7362e+05	7.0095e-09\\
1.7521e+05	6.7999e-09\\
1.7681e+05	6.5962e-09\\
1.7843e+05	6.3983e-09\\
1.8007e+05	6.206e-09\\
1.8172e+05	6.0193e-09\\
1.8338e+05	5.8379e-09\\
1.8506e+05	5.6616e-09\\
1.8676e+05	5.4905e-09\\
1.8847e+05	5.3242e-09\\
1.9019e+05	5.1627e-09\\
1.9194e+05	5.0059e-09\\
1.9369e+05	4.8537e-09\\
1.9547e+05	4.7058e-09\\
1.9726e+05	4.5622e-09\\
1.9907e+05	4.4228e-09\\
2.0089e+05	4.2875e-09\\
2.0273e+05	4.1561e-09\\
2.0459e+05	4.0285e-09\\
2.0646e+05	3.9047e-09\\
2.0835e+05	3.7845e-09\\
2.1026e+05	3.6678e-09\\
2.1219e+05	3.5545e-09\\
2.1413e+05	3.4446e-09\\
2.1609e+05	3.338e-09\\
2.1807e+05	3.2344e-09\\
2.2007e+05	3.134e-09\\
2.2208e+05	3.0365e-09\\
2.2412e+05	2.9419e-09\\
2.2617e+05	2.8502e-09\\
2.2824e+05	2.7611e-09\\
2.3033e+05	2.6748e-09\\
2.3244e+05	2.591e-09\\
2.3457e+05	2.5097e-09\\
2.3672e+05	2.4309e-09\\
2.3889e+05	2.3544e-09\\
2.4108e+05	2.2802e-09\\
2.4329e+05	2.2083e-09\\
2.4551e+05	2.1385e-09\\
2.4776e+05	2.0709e-09\\
2.5003e+05	2.0052e-09\\
2.5003e+05	1.6788e-09\\
2.4776e+05	1.7359e-09\\
2.4551e+05	1.7947e-09\\
2.4329e+05	1.8555e-09\\
2.4108e+05	1.9183e-09\\
2.3889e+05	1.983e-09\\
2.3672e+05	2.0499e-09\\
2.3457e+05	2.1189e-09\\
2.3244e+05	2.1901e-09\\
2.3033e+05	2.2636e-09\\
2.2824e+05	2.3395e-09\\
2.2617e+05	2.4178e-09\\
2.2412e+05	2.4985e-09\\
2.2208e+05	2.5819e-09\\
2.2007e+05	2.6678e-09\\
2.1807e+05	2.7566e-09\\
2.1609e+05	2.8481e-09\\
2.1413e+05	2.9425e-09\\
2.1219e+05	3.0399e-09\\
2.1026e+05	3.1403e-09\\
2.0835e+05	3.2439e-09\\
2.0646e+05	3.3507e-09\\
2.0459e+05	3.4609e-09\\
2.0273e+05	3.5746e-09\\
2.0089e+05	3.6917e-09\\
1.9907e+05	3.8125e-09\\
1.9726e+05	3.9371e-09\\
1.9547e+05	4.0655e-09\\
1.9369e+05	4.198e-09\\
1.9194e+05	4.3344e-09\\
1.9019e+05	4.4752e-09\\
1.8847e+05	4.6202e-09\\
1.8676e+05	4.7697e-09\\
1.8506e+05	4.9237e-09\\
1.8338e+05	5.0825e-09\\
1.8172e+05	5.2461e-09\\
1.8007e+05	5.4148e-09\\
1.7843e+05	5.5885e-09\\
1.7681e+05	5.7675e-09\\
1.7521e+05	5.952e-09\\
1.7362e+05	6.142e-09\\
1.7204e+05	6.3377e-09\\
1.7048e+05	6.5394e-09\\
1.6893e+05	6.7471e-09\\
1.674e+05	6.961e-09\\
1.6588e+05	7.1814e-09\\
1.6438e+05	7.4083e-09\\
1.6288e+05	7.642e-09\\
1.614e+05	7.8826e-09\\
1.5994e+05	8.1304e-09\\
1.5849e+05	8.3856e-09\\
1.5705e+05	8.6482e-09\\
1.5562e+05	8.9187e-09\\
1.5421e+05	9.1971e-09\\
1.5281e+05	9.4837e-09\\
1.5142e+05	9.7787e-09\\
1.5005e+05	1.0082e-08\\
1.4869e+05	1.0395e-08\\
1.4734e+05	1.0716e-08\\
1.46e+05	1.1047e-08\\
1.4468e+05	1.1388e-08\\
1.4336e+05	1.1738e-08\\
1.4206e+05	1.2099e-08\\
1.4077e+05	1.247e-08\\
1.3949e+05	1.2852e-08\\
1.3823e+05	1.3244e-08\\
1.3697e+05	1.3648e-08\\
1.3573e+05	1.4064e-08\\
1.345e+05	1.4491e-08\\
1.3328e+05	1.4931e-08\\
1.3207e+05	1.5383e-08\\
1.3087e+05	1.5847e-08\\
1.2968e+05	1.6325e-08\\
1.285e+05	1.6817e-08\\
1.2734e+05	1.7322e-08\\
1.2618e+05	1.7842e-08\\
1.2504e+05	1.8376e-08\\
1.239e+05	1.8925e-08\\
1.2278e+05	1.9489e-08\\
1.2166e+05	2.0069e-08\\
1.2056e+05	2.0665e-08\\
1.1946e+05	2.1278e-08\\
1.1838e+05	2.1908e-08\\
1.1731e+05	2.2555e-08\\
1.1624e+05	2.322e-08\\
1.1519e+05	2.3903e-08\\
1.1414e+05	2.4605e-08\\
1.131e+05	2.5326e-08\\
1.1208e+05	2.6067e-08\\
1.1106e+05	2.6828e-08\\
1.1005e+05	2.761e-08\\
1.0905e+05	2.8413e-08\\
1.0806e+05	2.9237e-08\\
1.0708e+05	3.0084e-08\\
1.0611e+05	3.0954e-08\\
1.0515e+05	3.1847e-08\\
1.0419e+05	3.2764e-08\\
1.0325e+05	3.3706e-08\\
1.0231e+05	3.4672e-08\\
1.0138e+05	3.5665e-08\\
1.0046e+05	3.6683e-08\\
99549	3.7729e-08\\
98646	3.8802e-08\\
97750	3.9904e-08\\
96863	4.1034e-08\\
95984	4.2195e-08\\
95113	4.3385e-08\\
94249	4.4607e-08\\
93394	4.5861e-08\\
92546	4.7147e-08\\
91706	4.8466e-08\\
90874	4.9819e-08\\
90049	5.1208e-08\\
89231	5.2632e-08\\
88421	5.4093e-08\\
87619	5.5591e-08\\
86824	5.7127e-08\\
86036	5.8702e-08\\
85255	6.0318e-08\\
84481	6.1974e-08\\
83714	6.3672e-08\\
82954	6.5413e-08\\
82201	6.7198e-08\\
81455	6.9027e-08\\
80716	7.0902e-08\\
79983	7.2825e-08\\
79257	7.4794e-08\\
78538	7.6813e-08\\
77825	7.8882e-08\\
77118	8.1002e-08\\
76418	8.3174e-08\\
75725	8.5399e-08\\
75037	8.7679e-08\\
74356	9.0015e-08\\
73681	9.2407e-08\\
73013	9.4858e-08\\
72350	9.7368e-08\\
71693	9.9939e-08\\
71042	1.0257e-07\\
70398	1.0527e-07\\
69759	1.0803e-07\\
69125	1.1086e-07\\
68498	1.1375e-07\\
67876	1.1671e-07\\
67260	1.1975e-07\\
66650	1.2285e-07\\
66045	1.2603e-07\\
65445	1.2928e-07\\
64851	1.3261e-07\\
64263	1.3602e-07\\
63679	1.3951e-07\\
63101	1.4308e-07\\
62529	1.4673e-07\\
61961	1.5047e-07\\
61399	1.5429e-07\\
60841	1.582e-07\\
60289	1.622e-07\\
59742	1.6629e-07\\
59200	1.7048e-07\\
58662	1.7476e-07\\
58130	1.7914e-07\\
57602	1.8362e-07\\
57079	1.882e-07\\
56561	1.9288e-07\\
56048	1.9767e-07\\
55539	2.0257e-07\\
55035	2.0757e-07\\
54535	2.1269e-07\\
54040	2.1792e-07\\
53550	2.2327e-07\\
53064	2.2874e-07\\
52582	2.3433e-07\\
52105	2.4004e-07\\
51632	2.4588e-07\\
51163	2.5184e-07\\
50699	2.5794e-07\\
50239	2.6417e-07\\
49783	2.7054e-07\\
49331	2.7704e-07\\
48883	2.8369e-07\\
48439	2.9048e-07\\
48000	2.9742e-07\\
47564	3.045e-07\\
47132	3.1174e-07\\
46704	3.1914e-07\\
46281	3.2669e-07\\
45860	3.3441e-07\\
45444	3.4229e-07\\
45032	3.5034e-07\\
44623	3.5856e-07\\
44218	3.6695e-07\\
43817	3.7553e-07\\
43419	3.8428e-07\\
43025	3.9322e-07\\
42634	4.0235e-07\\
42247	4.1166e-07\\
41864	4.2118e-07\\
41484	4.3089e-07\\
41107	4.4081e-07\\
40734	4.5093e-07\\
40364	4.6127e-07\\
39998	4.7182e-07\\
39635	4.8259e-07\\
39275	4.9358e-07\\
38919	5.048e-07\\
38565	5.1625e-07\\
38215	5.2794e-07\\
37869	5.3987e-07\\
37525	5.5205e-07\\
37184	5.6447e-07\\
36847	5.7715e-07\\
36512	5.9009e-07\\
36181	6.033e-07\\
35852	6.1677e-07\\
35527	6.3052e-07\\
35205	6.4455e-07\\
34885	6.5886e-07\\
34568	6.7346e-07\\
34255	6.8836e-07\\
33944	7.0357e-07\\
33636	7.1908e-07\\
33330	7.349e-07\\
33028	7.5104e-07\\
32728	7.6751e-07\\
32431	7.8431e-07\\
32137	8.0144e-07\\
31845	8.1893e-07\\
31556	8.3676e-07\\
31269	8.5495e-07\\
30986	8.735e-07\\
30704	8.9242e-07\\
30426	9.1172e-07\\
30149	9.3141e-07\\
29876	9.5149e-07\\
29605	9.7197e-07\\
29336	9.9285e-07\\
29070	1.0142e-06\\
28806	1.0359e-06\\
28544	1.058e-06\\
28285	1.0806e-06\\
28028	1.1037e-06\\
27774	1.1272e-06\\
27522	1.1511e-06\\
27272	1.1756e-06\\
27025	1.2005e-06\\
26779	1.2259e-06\\
26536	1.2518e-06\\
26295	1.2782e-06\\
26057	1.3052e-06\\
25820	1.3327e-06\\
25586	1.3607e-06\\
25354	1.3892e-06\\
25123	1.4184e-06\\
24895	1.4481e-06\\
24669	1.4783e-06\\
24446	1.5092e-06\\
24224	1.5407e-06\\
24004	1.5728e-06\\
23786	1.6055e-06\\
23570	1.6389e-06\\
23356	1.6729e-06\\
23144	1.7076e-06\\
22934	1.743e-06\\
22726	1.779e-06\\
22520	1.8158e-06\\
22315	1.8533e-06\\
22113	1.8915e-06\\
21912	1.9305e-06\\
21713	1.9702e-06\\
21516	2.0107e-06\\
21321	2.052e-06\\
21127	2.0941e-06\\
20935	2.137e-06\\
20745	2.1808e-06\\
20557	2.2254e-06\\
20370	2.2709e-06\\
20186	2.3173e-06\\
20002	2.3646e-06\\
19821	2.4128e-06\\
19641	2.4619e-06\\
19463	2.512e-06\\
19286	2.5631e-06\\
19111	2.6152e-06\\
18937	2.6683e-06\\
18765	2.7224e-06\\
18595	2.7776e-06\\
18426	2.8339e-06\\
18259	2.8913e-06\\
18093	2.9497e-06\\
17929	3.0094e-06\\
17766	3.0702e-06\\
17605	3.1322e-06\\
17445	3.1954e-06\\
17287	3.2598e-06\\
17130	3.3255e-06\\
16975	3.3925e-06\\
16821	3.4607e-06\\
16668	3.5304e-06\\
16517	3.6013e-06\\
16367	3.6737e-06\\
16218	3.7475e-06\\
16071	3.8227e-06\\
15925	3.8994e-06\\
15780	3.9776e-06\\
15637	4.0573e-06\\
15495	4.1386e-06\\
15355	4.2215e-06\\
15215	4.306e-06\\
15077	4.3922e-06\\
14940	4.48e-06\\
14805	4.5696e-06\\
14670	4.6609e-06\\
14537	4.754e-06\\
14405	4.8489e-06\\
14274	4.9457e-06\\
14145	5.0444e-06\\
14017	5.145e-06\\
13889	5.2476e-06\\
13763	5.3521e-06\\
13638	5.4588e-06\\
13515	5.5675e-06\\
13392	5.6784e-06\\
13270	5.7914e-06\\
13150	5.9066e-06\\
13030	6.0241e-06\\
12912	6.1439e-06\\
12795	6.266e-06\\
12679	6.3905e-06\\
12564	6.5175e-06\\
12450	6.6469e-06\\
12337	6.7789e-06\\
12225	6.9135e-06\\
12114	7.0506e-06\\
12004	7.1905e-06\\
11895	7.3331e-06\\
11787	7.4785e-06\\
11680	7.6267e-06\\
11574	7.7778e-06\\
11469	7.9318e-06\\
11365	8.0889e-06\\
11262	8.249e-06\\
11159	8.4122e-06\\
11058	8.5786e-06\\
10958	8.7482e-06\\
10858	8.9212e-06\\
10760	9.0975e-06\\
10662	9.2772e-06\\
10565	9.4604e-06\\
10469	9.6471e-06\\
10374	9.8375e-06\\
10280	1.0032e-05\\
10187	1.0229e-05\\
10094	1.0431e-05\\
10003	1.0637e-05\\
9912	1.0846e-05\\
9822	1.106e-05\\
9732.9	1.1277e-05\\
9644.5	1.1499e-05\\
9557	1.1725e-05\\
9470.2	1.1956e-05\\
9384.3	1.2191e-05\\
9299.1	1.243e-05\\
9214.7	1.2674e-05\\
9131.1	1.2923e-05\\
9048.2	1.3176e-05\\
8966	1.3434e-05\\
8884.7	1.3697e-05\\
8804	1.3966e-05\\
8724.1	1.4239e-05\\
8644.9	1.4517e-05\\
8566.5	1.4801e-05\\
8488.7	1.509e-05\\
8411.6	1.5385e-05\\
8335.3	1.5685e-05\\
8259.6	1.599e-05\\
8184.7	1.6302e-05\\
8110.4	1.6619e-05\\
8036.8	1.6943e-05\\
7963.8	1.7272e-05\\
7891.5	1.7608e-05\\
7819.9	1.7949e-05\\
7748.9	1.8297e-05\\
7678.6	1.8652e-05\\
7608.9	1.9013e-05\\
7539.8	1.9381e-05\\
7471.4	1.9756e-05\\
7403.6	2.0137e-05\\
7336.4	2.0526e-05\\
7269.8	2.0922e-05\\
7203.8	2.1325e-05\\
7138.4	2.1735e-05\\
7073.6	2.2153e-05\\
7009.4	2.2578e-05\\
6945.8	2.3011e-05\\
6882.7	2.3452e-05\\
6820.3	2.3901e-05\\
6758.4	2.4358e-05\\
6697	2.4823e-05\\
6636.2	2.5296e-05\\
6576	2.5778e-05\\
6516.3	2.6269e-05\\
6457.2	2.6768e-05\\
6398.5	2.7276e-05\\
6340.5	2.7793e-05\\
6282.9	2.8319e-05\\
6225.9	2.8854e-05\\
6169.4	2.9399e-05\\
6113.4	2.9953e-05\\
6057.9	3.0516e-05\\
6002.9	3.109e-05\\
5948.4	3.1673e-05\\
5894.4	3.2266e-05\\
5840.9	3.287e-05\\
5787.9	3.3483e-05\\
5735.4	3.4107e-05\\
5683.3	3.4742e-05\\
5631.7	3.5387e-05\\
5580.6	3.6043e-05\\
5530	3.671e-05\\
5479.8	3.7388e-05\\
5430	3.8078e-05\\
5380.7	3.8778e-05\\
5331.9	3.949e-05\\
5283.5	4.0214e-05\\
5235.5	4.095e-05\\
5188	4.1697e-05\\
5140.9	4.2456e-05\\
5094.3	4.3228e-05\\
5048	4.4011e-05\\
5002.2	4.4808e-05\\
4956.8	4.5616e-05\\
4911.8	4.6437e-05\\
4867.2	4.7271e-05\\
4823.1	4.8118e-05\\
4779.3	4.8978e-05\\
4735.9	4.9851e-05\\
4692.9	5.0738e-05\\
4650.3	5.1638e-05\\
4608.1	5.2551e-05\\
4566.3	5.3478e-05\\
4524.8	5.4418e-05\\
4483.8	5.5373e-05\\
4443.1	5.6341e-05\\
4402.7	5.7323e-05\\
4362.8	5.832e-05\\
4323.2	5.9331e-05\\
4283.9	6.0356e-05\\
4245	6.1396e-05\\
4206.5	6.245e-05\\
4168.3	6.3518e-05\\
4130.5	6.4602e-05\\
4093	6.57e-05\\
4055.9	6.6813e-05\\
4019	6.7941e-05\\
3982.6	6.9084e-05\\
3946.4	7.0242e-05\\
3910.6	7.1415e-05\\
3875.1	7.2604e-05\\
3839.9	7.3807e-05\\
3805.1	7.5026e-05\\
3770.5	7.6261e-05\\
3736.3	7.751e-05\\
3702.4	7.8775e-05\\
3668.8	8.0056e-05\\
3635.5	8.1352e-05\\
3602.5	8.2663e-05\\
3569.8	8.399e-05\\
3537.4	8.5333e-05\\
3505.3	8.6691e-05\\
3473.5	8.8064e-05\\
3441.9	8.9453e-05\\
3410.7	9.0858e-05\\
3379.7	9.2278e-05\\
3349.1	9.3713e-05\\
3318.7	9.5164e-05\\
3288.5	9.663e-05\\
3258.7	9.8112e-05\\
3229.1	9.9609e-05\\
3199.8	0.00010112\\
3170.8	0.00010265\\
3142	0.00010419\\
3113.5	0.00010575\\
3085.2	0.00010732\\
3057.2	0.00010891\\
3029.4	0.00011051\\
3001.9	0.00011213\\
2974.7	0.00011376\\
2947.7	0.00011541\\
2920.9	0.00011707\\
2894.4	0.00011874\\
2868.2	0.00012043\\
2842.1	0.00012214\\
2816.3	0.00012385\\
2790.8	0.00012558\\
2765.4	0.00012733\\
2740.3	0.00012909\\
2715.5	0.00013086\\
2690.8	0.00013264\\
2666.4	0.00013444\\
2642.2	0.00013625\\
2618.2	0.00013807\\
2594.4	0.00013991\\
2570.9	0.00014175\\
2547.6	0.00014361\\
2524.4	0.00014548\\
2501.5	0.00014736\\
2478.8	0.00014926\\
2456.3	0.00015116\\
2434	0.00015308\\
2411.9	0.000155\\
2390	0.00015694\\
2368.3	0.00015888\\
2346.8	0.00016084\\
2325.5	0.0001628\\
2304.4	0.00016477\\
2283.5	0.00016676\\
2262.8	0.00016875\\
2242.2	0.00017074\\
2221.9	0.00017275\\
2201.7	0.00017476\\
2181.7	0.00017678\\
2161.9	0.00017881\\
2142.3	0.00018084\\
2122.9	0.00018288\\
2103.6	0.00018492\\
2084.5	0.00018697\\
2065.6	0.00018903\\
2046.8	0.00019108\\
2028.3	0.00019315\\
2009.8	0.00019521\\
1991.6	0.00019728\\
1973.5	0.00019935\\
1955.6	0.00020142\\
1937.9	0.0002035\\
1920.3	0.00020557\\
1902.8	0.00020765\\
1885.6	0.00020972\\
1868.5	0.0002118\\
1851.5	0.00021387\\
1834.7	0.00021595\\
1818	0.00021802\\
1801.5	0.00022009\\
1785.2	0.00022215\\
1769	0.00022422\\
1752.9	0.00022628\\
1737	0.00022833\\
1721.2	0.00023038\\
1705.6	0.00023243\\
1690.1	0.00023446\\
1674.8	0.0002365\\
1659.6	0.00023852\\
1644.5	0.00024054\\
1629.6	0.00024255\\
1614.8	0.00024455\\
1600.2	0.00024654\\
1585.6	0.00024852\\
1571.2	0.00025049\\
1557	0.00025245\\
1542.8	0.00025439\\
1528.8	0.00025633\\
1515	0.00025825\\
1501.2	0.00026016\\
1487.6	0.00026206\\
1474.1	0.00026394\\
1460.7	0.00026581\\
1447.4	0.00026766\\
1434.3	0.00026949\\
1421.3	0.00027131\\
1408.4	0.00027312\\
1395.6	0.0002749\\
1382.9	0.00027667\\
1370.4	0.00027842\\
1357.9	0.00028015\\
1345.6	0.00028186\\
1333.4	0.00028355\\
1321.3	0.00028522\\
1309.3	0.00028687\\
1297.4	0.0002885\\
1285.7	0.00029011\\
1274	0.00029169\\
1262.4	0.00029326\\
1251	0.00029479\\
1239.6	0.00029631\\
1228.4	0.0002978\\
1217.2	0.00029927\\
1206.2	0.00030071\\
1195.2	0.00030213\\
1184.4	0.00030352\\
1173.6	0.00030489\\
1163	0.00030623\\
1152.4	0.00030754\\
1141.9	0.00030883\\
1131.6	0.00031009\\
1121.3	0.00031132\\
1111.1	0.00031253\\
1101	0.0003137\\
1091	0.00031485\\
1081.1	0.00031597\\
1071.3	0.00031706\\
1061.6	0.00031812\\
1052	0.00031915\\
1042.4	0.00032014\\
1033	0.00032111\\
1023.6	0.00032205\\
1014.3	0.00032296\\
1005.1	0.00032383\\
995.96	0.00032468\\
986.92	0.00032549\\
977.97	0.00032627\\
969.09	0.00032702\\
960.29	0.00032774\\
951.58	0.00032842\\
942.94	0.00032908\\
934.38	0.0003297\\
925.9	0.00033028\\
917.5	0.00033083\\
909.17	0.00033135\\
900.92	0.00033184\\
892.74	0.00033229\\
884.64	0.00033271\\
876.61	0.0003331\\
868.65	0.00033345\\
860.76	0.00033377\\
852.95	0.00033405\\
845.21	0.0003343\\
837.54	0.00033452\\
829.94	0.0003347\\
822.4	0.00033485\\
814.94	0.00033497\\
807.54	0.00033505\\
800.21	0.00033509\\
792.95	0.00033511\\
785.75	0.00033508\\
778.62	0.00033503\\
771.55	0.00033494\\
764.55	0.00033482\\
757.61	0.00033466\\
750.73	0.00033447\\
743.92	0.00033425\\
737.16	0.00033399\\
730.47	0.0003337\\
723.84	0.00033338\\
717.27	0.00033302\\
710.76	0.00033263\\
704.31	0.00033221\\
697.92	0.00033176\\
691.58	0.00033127\\
685.31	0.00033075\\
679.09	0.00033021\\
672.92	0.00032962\\
666.81	0.00032901\\
660.76	0.00032837\\
654.76	0.0003277\\
648.82	0.00032699\\
642.93	0.00032626\\
637.1	0.0003255\\
631.31	0.0003247\\
625.58	0.00032388\\
619.9	0.00032303\\
614.28	0.00032215\\
608.7	0.00032124\\
603.18	0.00032031\\
597.7	0.00031934\\
592.28	0.00031835\\
586.9	0.00031733\\
581.57	0.00031629\\
576.29	0.00031522\\
571.06	0.00031413\\
565.88	0.00031301\\
560.74	0.00031186\\
555.65	0.00031069\\
550.61	0.0003095\\
545.61	0.00030828\\
540.66	0.00030704\\
535.75	0.00030577\\
530.89	0.00030449\\
526.07	0.00030318\\
521.3	0.00030185\\
516.56	0.0003005\\
511.88	0.00029913\\
507.23	0.00029773\\
502.63	0.00029632\\
498.06	0.00029489\\
493.54	0.00029344\\
489.06	0.00029197\\
484.62	0.00029049\\
480.23	0.00028898\\
475.87	0.00028746\\
471.55	0.00028592\\
467.27	0.00028437\\
463.03	0.0002828\\
458.82	0.00028121\\
454.66	0.00027961\\
450.53	0.00027799\\
446.44	0.00027636\\
442.39	0.00027472\\
438.37	0.00027306\\
434.4	0.00027139\\
430.45	0.00026971\\
426.55	0.00026802\\
422.67	0.00026631\\
418.84	0.0002646\\
415.04	0.00026287\\
411.27	0.00026113\\
407.54	0.00025939\\
403.84	0.00025763\\
400.17	0.00025586\\
396.54	0.00025409\\
392.94	0.00025231\\
389.37	0.00025052\\
385.84	0.00024872\\
382.34	0.00024691\\
378.87	0.0002451\\
375.43	0.00024329\\
372.02	0.00024146\\
368.64	0.00023964\\
365.3	0.0002378\\
361.98	0.00023597\\
358.7	0.00023412\\
355.44	0.00023228\\
352.21	0.00023043\\
349.02	0.00022858\\
345.85	0.00022672\\
342.71	0.00022486\\
339.6	0.000223\\
336.52	0.00022114\\
333.46	0.00021928\\
330.43	0.00021742\\
327.44	0.00021555\\
324.46	0.00021369\\
321.52	0.00021182\\
318.6	0.00020996\\
315.71	0.00020809\\
312.84	0.00020623\\
310	0.00020437\\
307.19	0.00020251\\
304.4	0.00020065\\
301.64	0.0001988\\
298.9	0.00019694\\
296.19	0.00019509\\
293.5	0.00019325\\
290.83	0.0001914\\
288.19	0.00018956\\
285.58	0.00018772\\
282.99	0.00018589\\
280.42	0.00018406\\
277.87	0.00018224\\
275.35	0.00018042\\
272.85	0.00017861\\
270.37	0.0001768\\
267.92	0.000175\\
265.49	0.00017321\\
263.08	0.00017142\\
260.69	0.00016963\\
258.32	0.00016786\\
255.98	0.00016609\\
253.66	0.00016432\\
251.35	0.00016257\\
249.07	0.00016082\\
246.81	0.00015908\\
244.57	0.00015735\\
242.35	0.00015563\\
240.15	0.00015391\\
237.97	0.0001522\\
235.81	0.00015051\\
233.67	0.00014882\\
231.55	0.00014714\\
229.45	0.00014547\\
227.37	0.0001438\\
225.3	0.00014215\\
223.26	0.00014051\\
221.23	0.00013888\\
219.22	0.00013725\\
217.23	0.00013564\\
215.26	0.00013404\\
213.31	0.00013245\\
211.37	0.00013087\\
209.45	0.00012929\\
207.55	0.00012773\\
205.67	0.00012619\\
203.8	0.00012465\\
201.95	0.00012312\\
200.12	0.0001216\\
198.3	0.0001201\\
196.5	0.00011861\\
194.72	0.00011713\\
192.95	0.00011566\\
191.2	0.0001142\\
189.46	0.00011275\\
187.74	0.00011132\\
186.04	0.00010989\\
184.35	0.00010848\\
182.68	0.00010708\\
181.02	0.00010569\\
179.38	0.00010432\\
177.75	0.00010296\\
176.14	0.00010161\\
174.54	0.00010027\\
172.95	9.8941e-05\\
171.38	9.7627e-05\\
169.83	9.6325e-05\\
168.29	9.5035e-05\\
166.76	9.3758e-05\\
165.24	9.2493e-05\\
163.74	9.1241e-05\\
162.26	9e-05\\
160.79	8.8772e-05\\
159.33	8.7557e-05\\
157.88	8.6354e-05\\
156.45	8.5163e-05\\
155.03	8.3984e-05\\
153.62	8.2818e-05\\
152.23	8.1664e-05\\
150.84	8.0522e-05\\
149.47	7.9393e-05\\
148.12	7.8275e-05\\
146.77	7.717e-05\\
145.44	7.6077e-05\\
144.12	7.4997e-05\\
142.81	7.3928e-05\\
141.52	7.2871e-05\\
140.23	7.1827e-05\\
138.96	7.0794e-05\\
137.7	6.9773e-05\\
136.45	6.8764e-05\\
135.21	6.7767e-05\\
133.98	6.6782e-05\\
132.77	6.5808e-05\\
131.56	6.4846e-05\\
130.37	6.3896e-05\\
129.18	6.2957e-05\\
128.01	6.2029e-05\\
126.85	6.1113e-05\\
125.7	6.0208e-05\\
124.56	5.9315e-05\\
123.43	5.8433e-05\\
122.31	5.7561e-05\\
121.2	5.6701e-05\\
120.1	5.5852e-05\\
119.01	5.5014e-05\\
117.93	5.4186e-05\\
116.85	5.3369e-05\\
115.79	5.2563e-05\\
114.74	5.1768e-05\\
113.7	5.0982e-05\\
112.67	5.0208e-05\\
111.65	4.9443e-05\\
110.63	4.8689e-05\\
109.63	4.7945e-05\\
108.63	4.7211e-05\\
107.65	4.6487e-05\\
106.67	4.5773e-05\\
105.7	4.5068e-05\\
104.74	4.4374e-05\\
103.79	4.3688e-05\\
102.85	4.3013e-05\\
101.92	4.2347e-05\\
100.99	4.169e-05\\
100.08	4.1042e-05\\
99.167	4.0404e-05\\
98.267	3.9774e-05\\
97.375	3.9154e-05\\
96.491	3.8542e-05\\
95.615	3.7939e-05\\
94.747	3.7345e-05\\
93.887	3.6759e-05\\
93.035	3.6182e-05\\
92.191	3.5614e-05\\
91.354	3.5053e-05\\
90.525	3.4501e-05\\
89.703	3.3957e-05\\
88.889	3.3421e-05\\
88.082	3.2893e-05\\
87.283	3.2372e-05\\
86.49	3.186e-05\\
85.705	3.1355e-05\\
84.927	3.0857e-05\\
84.156	3.0368e-05\\
83.393	2.9885e-05\\
82.636	2.941e-05\\
81.886	2.8942e-05\\
81.142	2.8481e-05\\
80.406	2.8027e-05\\
79.676	2.758e-05\\
78.953	2.714e-05\\
78.236	2.6706e-05\\
77.526	2.6279e-05\\
76.822	2.5859e-05\\
76.125	2.5445e-05\\
75.434	2.5038e-05\\
74.749	2.4637e-05\\
74.071	2.4242e-05\\
73.399	2.3853e-05\\
72.732	2.3471e-05\\
72.072	2.3094e-05\\
71.418	2.2723e-05\\
70.77	2.2358e-05\\
70.127	2.1999e-05\\
69.491	2.1645e-05\\
68.86	2.1297e-05\\
68.235	2.0954e-05\\
67.616	2.0617e-05\\
67.002	2.0285e-05\\
66.394	1.9959e-05\\
65.791	1.9637e-05\\
65.194	1.9321e-05\\
64.602	1.901e-05\\
64.016	1.8703e-05\\
63.435	1.8402e-05\\
62.859	1.8105e-05\\
62.289	1.7813e-05\\
61.723	1.7526e-05\\
61.163	1.7243e-05\\
60.608	1.6965e-05\\
60.058	1.6691e-05\\
59.512	1.6422e-05\\
58.972	1.6157e-05\\
58.437	1.5896e-05\\
57.907	1.564e-05\\
57.381	1.5387e-05\\
56.86	1.5139e-05\\
56.344	1.4894e-05\\
55.833	1.4654e-05\\
55.326	1.4417e-05\\
54.824	1.4185e-05\\
54.326	1.3956e-05\\
53.833	1.373e-05\\
53.344	1.3509e-05\\
52.86	1.329e-05\\
52.38	1.3076e-05\\
51.905	1.2865e-05\\
51.434	1.2657e-05\\
50.967	1.2452e-05\\
50.504	1.2251e-05\\
50.046	1.2053e-05\\
49.592	1.1859e-05\\
49.141	1.1667e-05\\
48.695	1.1479e-05\\
48.253	1.1294e-05\\
47.815	1.1111e-05\\
47.381	1.0932e-05\\
46.951	1.0755e-05\\
46.525	1.0581e-05\\
46.103	1.0411e-05\\
45.684	1.0242e-05\\
45.27	1.0077e-05\\
44.859	9.9142e-06\\
44.452	9.754e-06\\
44.048	9.5965e-06\\
43.648	9.4414e-06\\
43.252	9.2889e-06\\
42.86	9.1388e-06\\
42.471	8.9912e-06\\
42.085	8.8459e-06\\
41.703	8.703e-06\\
41.325	8.5623e-06\\
40.949	8.4239e-06\\
40.578	8.2877e-06\\
40.209	8.1538e-06\\
39.845	8.0219e-06\\
39.483	7.8922e-06\\
39.124	7.7646e-06\\
38.769	7.639e-06\\
38.417	7.5154e-06\\
38.069	7.3938e-06\\
37.723	7.2741e-06\\
37.381	7.1563e-06\\
37.042	7.0405e-06\\
36.705	6.9265e-06\\
36.372	6.8142e-06\\
36.042	6.7038e-06\\
35.715	6.5952e-06\\
35.391	6.4883e-06\\
35.069	6.383e-06\\
34.751	6.2795e-06\\
34.436	6.1776e-06\\
34.123	6.0773e-06\\
33.813	5.9786e-06\\
33.506	5.8815e-06\\
33.202	5.7859e-06\\
32.901	5.6919e-06\\
32.602	5.5993e-06\\
32.306	5.5082e-06\\
32.013	5.4186e-06\\
31.723	5.3303e-06\\
31.435	5.2435e-06\\
31.149	5.158e-06\\
30.867	5.0739e-06\\
30.586	4.9911e-06\\
30.309	4.9096e-06\\
30.034	4.8294e-06\\
29.761	4.7505e-06\\
29.491	4.6728e-06\\
29.223	4.5963e-06\\
28.958	4.5211e-06\\
28.695	4.447e-06\\
28.435	4.3741e-06\\
28.177	4.3023e-06\\
27.921	4.2317e-06\\
27.667	4.1622e-06\\
}--cycle;
\addplot[ybar interval, fill=mycolor2, fill opacity=0.35, area legend, draw=none] table[row sep=crcr, x=Lower, y=Count] {%
Lower	Upper	Count\\
27.667	39.83	6.5776e-06\\
39.83	57.339	6.092e-06\\
57.339	82.545	2.6449e-05\\
82.545	118.83	5.2179e-05\\
118.83	171.07	8.3727e-05\\
171.07	246.27	0.00014504\\
246.27	354.53	0.00022145\\
354.53	510.38	0.00027069\\
510.38	734.75	0.00031378\\
734.75	1057.7	0.00032455\\
1057.7	1522.7	0.00029226\\
1522.7	2192.1	0.0002176\\
2192.1	3155.8	0.00014089\\
3155.8	4543	7.693e-05\\
4543	6540.1	3.7509e-05\\
6540.1	9415.1	1.7828e-05\\
9415.1	13554	8.1764e-06\\
13554	19512	3.8043e-06\\
19512	28090	1.8001e-06\\
28090	40438	6.9973e-07\\
40438	58215	2.3402e-07\\
58215	83806	8.9618e-08\\
83806	1.2065e+05	2.461e-08\\
1.2065e+05	1.7368e+05	3.57e-08\\
1.7368e+05	2.5003e+05	6.9854e-09\\
2.5003e+05	2.5003e+05	6.9854e-09\\
};
\addlegendentry{Istogramma appr.}

\addplot [color=mycolor2, only marks, every error bar/.append style={opacity=0.45}, mark=*, mark size=0pt, draw=none, forget plot]
 plot [error bars/.cd, y dir=both, y explicit, error bar style={line width=1pt, color=mycolor2}, error mark options={mark=none,mark size=0pt}]
 table[row sep=crcr, y error plus index=2, y error minus index=3]{%
33.196	6.5776e-06	7.4587e-06	6.5776e-06\\
47.789	6.092e-06	5.9517e-06	5.9517e-06\\
68.797	2.6449e-05	9.1355e-06	9.1355e-06\\
99.041	5.2179e-05	1.1778e-05	1.1778e-05\\
142.58	8.3727e-05	1.3241e-05	1.3241e-05\\
205.26	0.00014504	1.2811e-05	1.2811e-05\\
295.49	0.00022145	1.2949e-05	1.2949e-05\\
425.38	0.00027069	1.1259e-05	1.1259e-05\\
612.38	0.00031378	1.1163e-05	1.1163e-05\\
881.58	0.00032455	8.5579e-06	8.5579e-06\\
1269.1	0.00029226	6.0378e-06	6.0378e-06\\
1827	0.0002176	5.8867e-06	5.8867e-06\\
2630.2	0.00014089	3.4833e-06	3.4833e-06\\
3786.4	7.693e-05	2.2862e-06	2.2862e-06\\
5450.9	3.7509e-05	1.2892e-06	1.2892e-06\\
7847	1.7828e-05	6.6986e-07	6.6986e-07\\
11297	8.1764e-06	3.7143e-07	3.7143e-07\\
16263	3.8043e-06	2.3187e-07	2.3187e-07\\
23411	1.8001e-06	1.3391e-07	1.3391e-07\\
33703	6.9973e-07	6.7317e-08	6.7317e-08\\
48519	2.3402e-07	3.2309e-08	3.2309e-08\\
69848	8.9618e-08	1.47e-08	1.47e-08\\
1.0055e+05	2.461e-08	7.9641e-09	7.9641e-09\\
1.4476e+05	3.57e-08	4.55e-09	4.55e-09\\
2.0839e+05	6.9854e-09	2.7861e-09	2.7861e-09\\
};
\addplot [color=mycolor2]
  table[row sep=crcr]{%
27.667	6.8918e-06\\
31.149	8.4524e-06\\
34.751	1.0169e-05\\
38.417	1.2013e-05\\
42.085	1.3949e-05\\
46.103	1.6166e-05\\
50.046	1.8432e-05\\
53.833	2.0688e-05\\
57.907	2.3195e-05\\
62.289	2.598e-05\\
66.394	2.8666e-05\\
70.77	3.1604e-05\\
75.434	3.4817e-05\\
80.406	3.8326e-05\\
84.927	4.1586e-05\\
89.703	4.5095e-05\\
94.747	4.8868e-05\\
100.08	5.2918e-05\\
105.7	5.7262e-05\\
111.65	6.1914e-05\\
116.85	6.6035e-05\\
122.31	7.0386e-05\\
128.01	7.4976e-05\\
133.98	7.9809e-05\\
140.23	8.4892e-05\\
146.77	9.0228e-05\\
153.62	9.5822e-05\\
160.79	0.00010168\\
168.29	0.00010779\\
176.14	0.00011416\\
184.35	0.00012079\\
192.95	0.00012767\\
201.95	0.00013479\\
211.37	0.00014215\\
223.26	0.00015128\\
235.81	0.0001607\\
249.07	0.00017039\\
265.49	0.00018198\\
285.58	0.00019553\\
315.71	0.00021448\\
361.98	0.00024029\\
385.84	0.00025206\\
407.54	0.0002619\\
426.55	0.00026985\\
446.44	0.00027753\\
463.03	0.00028345\\
480.23	0.00028915\\
498.06	0.00029459\\
516.56	0.00029976\\
535.75	0.00030463\\
555.65	0.00030918\\
576.29	0.00031339\\
592.28	0.0003163\\
608.7	0.00031901\\
625.58	0.00032149\\
642.93	0.00032374\\
660.76	0.00032575\\
679.09	0.00032751\\
697.92	0.00032902\\
717.27	0.00033028\\
737.16	0.00033127\\
757.61	0.00033199\\
778.62	0.00033244\\
800.21	0.00033261\\
822.4	0.00033251\\
845.21	0.00033213\\
868.65	0.00033146\\
892.74	0.00033052\\
917.5	0.0003293\\
942.94	0.0003278\\
969.09	0.00032603\\
995.96	0.00032399\\
1023.6	0.00032169\\
1052	0.00031912\\
1081.1	0.0003163\\
1111.1	0.00031322\\
1141.9	0.0003099\\
1173.6	0.00030634\\
1206.2	0.00030255\\
1239.6	0.00029853\\
1274	0.00029428\\
1309.3	0.00028982\\
1345.6	0.00028515\\
1395.6	0.00027862\\
1447.4	0.00027174\\
1501.2	0.00026455\\
1557	0.00025706\\
1614.8	0.00024932\\
1690.1	0.00023931\\
1769	0.00022903\\
1868.5	0.00021642\\
2262.8	0.00017198\\
2368.3	0.00016175\\
2478.8	0.00015177\\
2594.4	0.0001421\\
2690.8	0.00013461\\
2790.8	0.00012735\\
2894.4	0.00012034\\
3001.9	0.00011358\\
3113.5	0.00010707\\
3229.1	0.00010083\\
3349.1	9.4848e-05\\
3473.5	8.9131e-05\\
3602.5	8.3676e-05\\
3736.3	7.8479e-05\\
3875.1	7.3536e-05\\
4019	6.8843e-05\\
4168.3	6.4394e-05\\
4323.2	6.0181e-05\\
4483.8	5.6199e-05\\
4650.3	5.2439e-05\\
4867.2	4.8041e-05\\
5094.3	4.3964e-05\\
5331.9	4.019e-05\\
5580.6	3.6704e-05\\
5840.9	3.349e-05\\
6113.4	3.0531e-05\\
6398.5	2.7811e-05\\
6758.4	2.4841e-05\\
7138.4	2.2167e-05\\
7539.8	1.9764e-05\\
7963.8	1.7608e-05\\
8488.7	1.5374e-05\\
9048.2	1.3414e-05\\
9732.9	1.1468e-05\\
10469	9.7966e-06\\
11365	8.1998e-06\\
12450	6.7234e-06\\
13763	5.3996e-06\\
15355	4.2462e-06\\
17287	3.268e-06\\
19821	2.4098e-06\\
23356	1.6641e-06\\
28285	1.0709e-06\\
35852	6.0983e-07\\
49331	2.7489e-07\\
78538	7.825e-08\\
2.0089e+05	4.2816e-09\\
2.5003e+05	2.0329e-09\\
};
\addlegendentry{Adatt. BLN appr. ($\mathit{ML}$)}


\addplot[area legend, draw=none, fill=mycolor2, fill opacity=0.15, forget plot]
table[row sep=crcr] {%
x	y\\
27.667	8.2361e-06\\
27.921	8.3634e-06\\
28.177	8.4924e-06\\
28.435	8.6231e-06\\
28.695	8.7557e-06\\
28.958	8.89e-06\\
29.223	9.0261e-06\\
29.491	9.1641e-06\\
29.761	9.304e-06\\
30.034	9.4457e-06\\
30.309	9.5893e-06\\
30.586	9.7349e-06\\
30.867	9.8824e-06\\
31.149	1.0032e-05\\
31.435	1.0183e-05\\
31.723	1.0337e-05\\
32.013	1.0492e-05\\
32.306	1.065e-05\\
32.602	1.081e-05\\
32.901	1.0972e-05\\
33.202	1.1136e-05\\
33.506	1.1302e-05\\
33.813	1.147e-05\\
34.123	1.1641e-05\\
34.436	1.1813e-05\\
34.751	1.1988e-05\\
35.069	1.2166e-05\\
35.391	1.2345e-05\\
35.715	1.2527e-05\\
36.042	1.2712e-05\\
36.372	1.2899e-05\\
36.705	1.3088e-05\\
37.042	1.328e-05\\
37.381	1.3474e-05\\
37.723	1.3671e-05\\
38.069	1.387e-05\\
38.417	1.4072e-05\\
38.769	1.4276e-05\\
39.124	1.4483e-05\\
39.483	1.4693e-05\\
39.845	1.4905e-05\\
40.209	1.5121e-05\\
40.578	1.5338e-05\\
40.949	1.5559e-05\\
41.325	1.5783e-05\\
41.703	1.6009e-05\\
42.085	1.6238e-05\\
42.471	1.6471e-05\\
42.86	1.6706e-05\\
43.252	1.6944e-05\\
43.648	1.7185e-05\\
44.048	1.7429e-05\\
44.452	1.7677e-05\\
44.859	1.7927e-05\\
45.27	1.8181e-05\\
45.684	1.8438e-05\\
46.103	1.8698e-05\\
46.525	1.8961e-05\\
46.951	1.9228e-05\\
47.381	1.9498e-05\\
47.815	1.9771e-05\\
48.253	2.0048e-05\\
48.695	2.0329e-05\\
49.141	2.0612e-05\\
49.592	2.09e-05\\
50.046	2.1191e-05\\
50.504	2.1485e-05\\
50.967	2.1784e-05\\
51.434	2.2086e-05\\
51.905	2.2391e-05\\
52.38	2.2701e-05\\
52.86	2.3014e-05\\
53.344	2.3331e-05\\
53.833	2.3653e-05\\
54.326	2.3978e-05\\
54.824	2.4307e-05\\
55.326	2.464e-05\\
55.833	2.4977e-05\\
56.344	2.5319e-05\\
56.86	2.5665e-05\\
57.381	2.6014e-05\\
57.907	2.6369e-05\\
58.437	2.6727e-05\\
58.972	2.709e-05\\
59.512	2.7458e-05\\
60.058	2.7829e-05\\
60.608	2.8206e-05\\
61.163	2.8587e-05\\
61.723	2.8972e-05\\
62.289	2.9363e-05\\
62.859	2.9758e-05\\
63.435	3.0158e-05\\
64.016	3.0562e-05\\
64.602	3.0972e-05\\
65.194	3.1386e-05\\
65.791	3.1806e-05\\
66.394	3.2231e-05\\
67.002	3.266e-05\\
67.616	3.3095e-05\\
68.235	3.3535e-05\\
68.86	3.3981e-05\\
69.491	3.4431e-05\\
70.127	3.4888e-05\\
70.77	3.5349e-05\\
71.418	3.5816e-05\\
72.072	3.6289e-05\\
72.732	3.6767e-05\\
73.399	3.7251e-05\\
74.071	3.7741e-05\\
74.749	3.8236e-05\\
75.434	3.8737e-05\\
76.125	3.9245e-05\\
76.822	3.9758e-05\\
77.526	4.0277e-05\\
78.236	4.0802e-05\\
78.953	4.1334e-05\\
79.676	4.1872e-05\\
80.406	4.2416e-05\\
81.142	4.2966e-05\\
81.886	4.3523e-05\\
82.636	4.4086e-05\\
83.393	4.4656e-05\\
84.156	4.5232e-05\\
84.927	4.5816e-05\\
85.705	4.6405e-05\\
86.49	4.7002e-05\\
87.283	4.7605e-05\\
88.082	4.8216e-05\\
88.889	4.8833e-05\\
89.703	4.9458e-05\\
90.525	5.0089e-05\\
91.354	5.0728e-05\\
92.191	5.1374e-05\\
93.035	5.2027e-05\\
93.887	5.2688e-05\\
94.747	5.3356e-05\\
95.615	5.4032e-05\\
96.491	5.4715e-05\\
97.375	5.5406e-05\\
98.267	5.6104e-05\\
99.167	5.6811e-05\\
100.08	5.7525e-05\\
100.99	5.8247e-05\\
101.92	5.8977e-05\\
102.85	5.9715e-05\\
103.79	6.0461e-05\\
104.74	6.1216e-05\\
105.7	6.1978e-05\\
106.67	6.2749e-05\\
107.65	6.3528e-05\\
108.63	6.4316e-05\\
109.63	6.5112e-05\\
110.63	6.5917e-05\\
111.65	6.673e-05\\
112.67	6.7552e-05\\
113.7	6.8383e-05\\
114.74	6.9222e-05\\
115.79	7.007e-05\\
116.85	7.0928e-05\\
117.93	7.1794e-05\\
119.01	7.2669e-05\\
120.1	7.3553e-05\\
121.2	7.4447e-05\\
122.31	7.5349e-05\\
123.43	7.6261e-05\\
124.56	7.7182e-05\\
125.7	7.8113e-05\\
126.85	7.9053e-05\\
128.01	8.0002e-05\\
129.18	8.0961e-05\\
130.37	8.1929e-05\\
131.56	8.2907e-05\\
132.77	8.3895e-05\\
133.98	8.4892e-05\\
135.21	8.5899e-05\\
136.45	8.6916e-05\\
137.7	8.7943e-05\\
138.96	8.898e-05\\
140.23	9.0026e-05\\
141.52	9.1082e-05\\
142.81	9.2149e-05\\
144.12	9.3225e-05\\
145.44	9.4312e-05\\
146.77	9.5408e-05\\
148.12	9.6515e-05\\
149.47	9.7631e-05\\
150.84	9.8758e-05\\
152.23	9.9895e-05\\
153.62	0.00010104\\
155.03	0.0001022\\
156.45	0.00010337\\
157.88	0.00010455\\
159.33	0.00010573\\
160.79	0.00010693\\
162.26	0.00010814\\
163.74	0.00010936\\
165.24	0.00011059\\
166.76	0.00011183\\
168.29	0.00011308\\
169.83	0.00011434\\
171.38	0.00011561\\
172.95	0.00011689\\
174.54	0.00011818\\
176.14	0.00011948\\
177.75	0.00012079\\
179.38	0.00012211\\
181.02	0.00012344\\
182.68	0.00012478\\
184.35	0.00012614\\
186.04	0.0001275\\
187.74	0.00012887\\
189.46	0.00013025\\
191.2	0.00013164\\
192.95	0.00013304\\
194.72	0.00013445\\
196.5	0.00013588\\
198.3	0.00013731\\
200.12	0.00013875\\
201.95	0.0001402\\
203.8	0.00014165\\
205.67	0.00014312\\
207.55	0.0001446\\
209.45	0.00014609\\
211.37	0.00014758\\
213.31	0.00014909\\
215.26	0.0001506\\
217.23	0.00015213\\
219.22	0.00015366\\
221.23	0.0001552\\
223.26	0.00015675\\
225.3	0.00015831\\
227.37	0.00015987\\
229.45	0.00016144\\
231.55	0.00016303\\
233.67	0.00016462\\
235.81	0.00016621\\
237.97	0.00016782\\
240.15	0.00016943\\
242.35	0.00017105\\
244.57	0.00017268\\
246.81	0.00017431\\
249.07	0.00017595\\
251.35	0.0001776\\
253.66	0.00017925\\
255.98	0.00018091\\
258.32	0.00018258\\
260.69	0.00018425\\
263.08	0.00018593\\
265.49	0.00018761\\
267.92	0.0001893\\
270.37	0.00019099\\
272.85	0.00019269\\
275.35	0.00019439\\
277.87	0.0001961\\
280.42	0.00019781\\
282.99	0.00019952\\
285.58	0.00020124\\
288.19	0.00020296\\
290.83	0.00020469\\
293.5	0.00020641\\
296.19	0.00020814\\
298.9	0.00020988\\
301.64	0.00021161\\
304.4	0.00021335\\
307.19	0.00021509\\
310	0.00021683\\
312.84	0.00021857\\
315.71	0.00022031\\
318.6	0.00022205\\
321.52	0.00022379\\
324.46	0.00022553\\
327.44	0.00022727\\
330.43	0.00022901\\
333.46	0.00023075\\
336.52	0.00023248\\
339.6	0.00023422\\
342.71	0.00023595\\
345.85	0.00023768\\
349.02	0.00023941\\
352.21	0.00024113\\
355.44	0.00024285\\
358.7	0.00024457\\
361.98	0.00024628\\
365.3	0.00024798\\
368.64	0.00024969\\
372.02	0.00025138\\
375.43	0.00025307\\
378.87	0.00025476\\
382.34	0.00025644\\
385.84	0.00025811\\
389.37	0.00025977\\
392.94	0.00026143\\
396.54	0.00026308\\
400.17	0.00026472\\
403.84	0.00026635\\
407.54	0.00026797\\
411.27	0.00026959\\
415.04	0.00027119\\
418.84	0.00027278\\
422.67	0.00027437\\
426.55	0.00027594\\
430.45	0.0002775\\
434.4	0.00027905\\
438.37	0.00028058\\
442.39	0.0002821\\
446.44	0.00028361\\
450.53	0.00028511\\
454.66	0.00028659\\
458.82	0.00028806\\
463.03	0.00028952\\
467.27	0.00029096\\
471.55	0.00029238\\
475.87	0.00029379\\
480.23	0.00029518\\
484.62	0.00029656\\
489.06	0.00029791\\
493.54	0.00029926\\
498.06	0.00030058\\
502.63	0.00030189\\
507.23	0.00030317\\
511.88	0.00030444\\
516.56	0.00030569\\
521.3	0.00030692\\
526.07	0.00030813\\
530.89	0.00030932\\
535.75	0.00031049\\
540.66	0.00031164\\
545.61	0.00031277\\
550.61	0.00031387\\
555.65	0.00031496\\
560.74	0.00031602\\
565.88	0.00031706\\
571.06	0.00031807\\
576.29	0.00031906\\
581.57	0.00032003\\
586.9	0.00032098\\
592.28	0.0003219\\
597.7	0.0003228\\
603.18	0.00032367\\
608.7	0.00032451\\
614.28	0.00032534\\
619.9	0.00032613\\
625.58	0.0003269\\
631.31	0.00032764\\
637.1	0.00032836\\
642.93	0.00032905\\
648.82	0.00032972\\
654.76	0.00033035\\
660.76	0.00033096\\
666.81	0.00033154\\
672.92	0.0003321\\
679.09	0.00033262\\
685.31	0.00033312\\
691.58	0.00033359\\
697.92	0.00033403\\
704.31	0.00033444\\
710.76	0.00033482\\
717.27	0.00033518\\
723.84	0.0003355\\
730.47	0.0003358\\
737.16	0.00033606\\
743.92	0.0003363\\
750.73	0.0003365\\
757.61	0.00033668\\
764.55	0.00033683\\
771.55	0.00033694\\
778.62	0.00033703\\
785.75	0.00033708\\
792.95	0.00033711\\
800.21	0.0003371\\
807.54	0.00033707\\
814.94	0.000337\\
822.4	0.0003369\\
829.94	0.00033678\\
837.54	0.00033662\\
845.21	0.00033643\\
852.95	0.00033621\\
860.76	0.00033596\\
868.65	0.00033567\\
876.61	0.00033536\\
884.64	0.00033502\\
892.74	0.00033464\\
900.92	0.00033423\\
909.17	0.0003338\\
917.5	0.00033333\\
925.9	0.00033283\\
934.38	0.0003323\\
942.94	0.00033174\\
951.58	0.00033115\\
960.29	0.00033053\\
969.09	0.00032988\\
977.97	0.0003292\\
986.92	0.00032849\\
995.96	0.00032774\\
1005.1	0.00032697\\
1014.3	0.00032617\\
1023.6	0.00032534\\
1033	0.00032448\\
1042.4	0.00032359\\
1052	0.00032267\\
1061.6	0.00032173\\
1071.3	0.00032075\\
1081.1	0.00031975\\
1091	0.00031872\\
1101	0.00031766\\
1111.1	0.00031658\\
1121.3	0.00031546\\
1131.6	0.00031433\\
1141.9	0.00031316\\
1152.4	0.00031197\\
1163	0.00031076\\
1173.6	0.00030952\\
1184.4	0.00030826\\
1195.2	0.00030697\\
1206.2	0.00030565\\
1217.2	0.00030432\\
1228.4	0.00030296\\
1239.6	0.00030158\\
1251	0.00030017\\
1262.4	0.00029875\\
1274	0.0002973\\
1285.7	0.00029583\\
1297.4	0.00029433\\
1309.3	0.00029282\\
1321.3	0.00029128\\
1333.4	0.00028973\\
1345.6	0.00028815\\
1357.9	0.00028655\\
1370.4	0.00028493\\
1382.9	0.00028329\\
1395.6	0.00028163\\
1408.4	0.00027995\\
1421.3	0.00027825\\
1434.3	0.00027653\\
1447.4	0.00027478\\
1460.7	0.00027302\\
1474.1	0.00027124\\
1487.6	0.00026943\\
1501.2	0.00026761\\
1515	0.00026577\\
1528.8	0.00026391\\
1542.8	0.00026203\\
1557	0.00026013\\
1571.2	0.00025821\\
1585.6	0.00025628\\
1600.2	0.00025433\\
1614.8	0.00025236\\
1629.6	0.00025037\\
1644.5	0.00024837\\
1659.6	0.00024636\\
1674.8	0.00024432\\
1690.1	0.00024228\\
1705.6	0.00024022\\
1721.2	0.00023815\\
1737	0.00023606\\
1752.9	0.00023397\\
1769	0.00023186\\
1785.2	0.00022974\\
1801.5	0.00022762\\
1818	0.00022548\\
1834.7	0.00022334\\
1851.5	0.00022119\\
1868.5	0.00021904\\
1885.6	0.00021688\\
1902.8	0.00021471\\
1920.3	0.00021254\\
1937.9	0.00021037\\
1955.6	0.0002082\\
1973.5	0.00020602\\
1991.6	0.00020384\\
2009.8	0.00020167\\
2028.3	0.00019949\\
2046.8	0.00019732\\
2065.6	0.00019514\\
2084.5	0.00019297\\
2103.6	0.00019081\\
2122.9	0.00018865\\
2142.3	0.00018649\\
2161.9	0.00018434\\
2181.7	0.00018219\\
2201.7	0.00018006\\
2221.9	0.00017793\\
2242.2	0.0001758\\
2262.8	0.00017369\\
2283.5	0.00017159\\
2304.4	0.00016949\\
2325.5	0.00016741\\
2346.8	0.00016533\\
2368.3	0.00016327\\
2390	0.00016122\\
2411.9	0.00015918\\
2434	0.00015715\\
2456.3	0.00015514\\
2478.8	0.00015314\\
2501.5	0.00015115\\
2524.4	0.00014918\\
2547.6	0.00014722\\
2570.9	0.00014527\\
2594.4	0.00014334\\
2618.2	0.00014143\\
2642.2	0.00013953\\
2666.4	0.00013764\\
2690.8	0.00013577\\
2715.5	0.00013392\\
2740.3	0.00013208\\
2765.4	0.00013026\\
2790.8	0.00012846\\
2816.3	0.00012667\\
2842.1	0.00012489\\
2868.2	0.00012314\\
2894.4	0.0001214\\
2920.9	0.00011967\\
2947.7	0.00011797\\
2974.7	0.00011628\\
3001.9	0.0001146\\
3029.4	0.00011294\\
3057.2	0.0001113\\
3085.2	0.00010968\\
3113.5	0.00010807\\
3142	0.00010648\\
3170.8	0.0001049\\
3199.8	0.00010335\\
3229.1	0.0001018\\
3258.7	0.00010028\\
3288.5	9.8769e-05\\
3318.7	9.7276e-05\\
3349.1	9.58e-05\\
3379.7	9.434e-05\\
3410.7	9.2896e-05\\
3441.9	9.1469e-05\\
3473.5	9.0058e-05\\
3505.3	8.8663e-05\\
3537.4	8.7284e-05\\
3569.8	8.5921e-05\\
3602.5	8.4574e-05\\
3635.5	8.3243e-05\\
3668.8	8.1928e-05\\
3702.4	8.0629e-05\\
3736.3	7.9346e-05\\
3770.5	7.8078e-05\\
3805.1	7.6826e-05\\
3839.9	7.559e-05\\
3875.1	7.4369e-05\\
3910.6	7.3163e-05\\
3946.4	7.1973e-05\\
3982.6	7.0798e-05\\
4019	6.9639e-05\\
4055.9	6.8494e-05\\
4093	6.7364e-05\\
4130.5	6.625e-05\\
4168.3	6.515e-05\\
4206.5	6.4065e-05\\
4245	6.2994e-05\\
4283.9	6.1939e-05\\
4323.2	6.0897e-05\\
4362.8	5.987e-05\\
4402.7	5.8857e-05\\
4443.1	5.7858e-05\\
4483.8	5.6874e-05\\
4524.8	5.5903e-05\\
4566.3	5.4946e-05\\
4608.1	5.4002e-05\\
4650.3	5.3073e-05\\
4692.9	5.2156e-05\\
4735.9	5.1253e-05\\
4779.3	5.0363e-05\\
4823.1	4.9487e-05\\
4867.2	4.8623e-05\\
4911.8	4.7772e-05\\
4956.8	4.6934e-05\\
5002.2	4.6109e-05\\
5048	4.5296e-05\\
5094.3	4.4496e-05\\
5140.9	4.3707e-05\\
5188	4.2931e-05\\
5235.5	4.2167e-05\\
5283.5	4.1415e-05\\
5331.9	4.0674e-05\\
5380.7	3.9945e-05\\
5430	3.9228e-05\\
5479.8	3.8522e-05\\
5530	3.7827e-05\\
5580.6	3.7144e-05\\
5631.7	3.6471e-05\\
5683.3	3.5809e-05\\
5735.4	3.5158e-05\\
5787.9	3.4517e-05\\
5840.9	3.3887e-05\\
5894.4	3.3268e-05\\
5948.4	3.2658e-05\\
6002.9	3.2059e-05\\
6057.9	3.147e-05\\
6113.4	3.089e-05\\
6169.4	3.032e-05\\
6225.9	2.976e-05\\
6282.9	2.9209e-05\\
6340.5	2.8668e-05\\
6398.5	2.8135e-05\\
6457.2	2.7612e-05\\
6516.3	2.7098e-05\\
6576	2.6593e-05\\
6636.2	2.6096e-05\\
6697	2.5608e-05\\
6758.4	2.5128e-05\\
6820.3	2.4657e-05\\
6882.7	2.4194e-05\\
6945.8	2.3739e-05\\
7009.4	2.3292e-05\\
7073.6	2.2853e-05\\
7138.4	2.2422e-05\\
7203.8	2.1998e-05\\
7269.8	2.1582e-05\\
7336.4	2.1174e-05\\
7403.6	2.0772e-05\\
7471.4	2.0378e-05\\
7539.8	1.9991e-05\\
7608.9	1.9611e-05\\
7678.6	1.9238e-05\\
7748.9	1.8871e-05\\
7819.9	1.8511e-05\\
7891.5	1.8158e-05\\
7963.8	1.7811e-05\\
8036.8	1.7471e-05\\
8110.4	1.7137e-05\\
8184.7	1.6808e-05\\
8259.6	1.6486e-05\\
8335.3	1.617e-05\\
8411.6	1.586e-05\\
8488.7	1.5555e-05\\
8566.5	1.5256e-05\\
8644.9	1.4963e-05\\
8724.1	1.4675e-05\\
8804	1.4392e-05\\
8884.7	1.4115e-05\\
8966	1.3842e-05\\
9048.2	1.3575e-05\\
9131.1	1.3313e-05\\
9214.7	1.3056e-05\\
9299.1	1.2804e-05\\
9384.3	1.2556e-05\\
9470.2	1.2313e-05\\
9557	1.2075e-05\\
9644.5	1.1841e-05\\
9732.9	1.1611e-05\\
9822	1.1386e-05\\
9912	1.1165e-05\\
10003	1.0949e-05\\
10094	1.0736e-05\\
10187	1.0527e-05\\
10280	1.0323e-05\\
10374	1.0122e-05\\
10469	9.9251e-06\\
10565	9.732e-06\\
10662	9.5425e-06\\
10760	9.3566e-06\\
10858	9.1743e-06\\
10958	8.9955e-06\\
11058	8.8201e-06\\
11159	8.648e-06\\
11262	8.4792e-06\\
11365	8.3137e-06\\
11469	8.1513e-06\\
11574	7.992e-06\\
11680	7.8358e-06\\
11787	7.6826e-06\\
11895	7.5323e-06\\
12004	7.3849e-06\\
12114	7.2403e-06\\
12225	7.0986e-06\\
12337	6.9595e-06\\
12450	6.8231e-06\\
12564	6.6893e-06\\
12679	6.5581e-06\\
12795	6.4295e-06\\
12912	6.3033e-06\\
13030	6.1795e-06\\
13150	6.0581e-06\\
13270	5.9391e-06\\
13392	5.8223e-06\\
13515	5.7078e-06\\
13638	5.5955e-06\\
13763	5.4854e-06\\
13889	5.3774e-06\\
14017	5.2714e-06\\
14145	5.1676e-06\\
14274	5.0657e-06\\
14405	4.9658e-06\\
14537	4.8678e-06\\
14670	4.7717e-06\\
14805	4.6775e-06\\
14940	4.585e-06\\
15077	4.4944e-06\\
15215	4.4055e-06\\
15355	4.3183e-06\\
15495	4.2329e-06\\
15637	4.149e-06\\
15780	4.0668e-06\\
15925	3.9862e-06\\
16071	3.9071e-06\\
16218	3.8296e-06\\
16367	3.7535e-06\\
16517	3.6789e-06\\
16668	3.6058e-06\\
16821	3.5341e-06\\
16975	3.4637e-06\\
17130	3.3948e-06\\
17287	3.3271e-06\\
17445	3.2608e-06\\
17605	3.1957e-06\\
17766	3.1319e-06\\
17929	3.0694e-06\\
18093	3.008e-06\\
18259	2.9478e-06\\
18426	2.8888e-06\\
18595	2.831e-06\\
18765	2.7742e-06\\
18937	2.7186e-06\\
19111	2.664e-06\\
19286	2.6105e-06\\
19463	2.558e-06\\
19641	2.5065e-06\\
19821	2.4561e-06\\
20002	2.4066e-06\\
20186	2.358e-06\\
20370	2.3105e-06\\
20557	2.2638e-06\\
20745	2.218e-06\\
20935	2.1731e-06\\
21127	2.1291e-06\\
21321	2.086e-06\\
21516	2.0436e-06\\
21713	2.0021e-06\\
21912	1.9614e-06\\
22113	1.9215e-06\\
22315	1.8824e-06\\
22520	1.844e-06\\
22726	1.8064e-06\\
22934	1.7695e-06\\
23144	1.7333e-06\\
23356	1.6979e-06\\
23570	1.6631e-06\\
23786	1.629e-06\\
24004	1.5955e-06\\
24224	1.5627e-06\\
24446	1.5306e-06\\
24669	1.499e-06\\
24895	1.4681e-06\\
25123	1.4378e-06\\
25354	1.4081e-06\\
25586	1.379e-06\\
25820	1.3504e-06\\
26057	1.3224e-06\\
26295	1.2949e-06\\
26536	1.268e-06\\
26779	1.2416e-06\\
27025	1.2157e-06\\
27272	1.1903e-06\\
27522	1.1655e-06\\
27774	1.1411e-06\\
28028	1.1172e-06\\
28285	1.0937e-06\\
28544	1.0707e-06\\
28806	1.0482e-06\\
29070	1.0261e-06\\
29336	1.0045e-06\\
29605	9.8328e-07\\
29876	9.6248e-07\\
30149	9.421e-07\\
30426	9.2211e-07\\
30704	9.0253e-07\\
30986	8.8333e-07\\
31269	8.6451e-07\\
31556	8.4607e-07\\
31845	8.2799e-07\\
32137	8.1027e-07\\
32431	7.9291e-07\\
32728	7.7589e-07\\
33028	7.5922e-07\\
33330	7.4287e-07\\
33636	7.2686e-07\\
33944	7.1116e-07\\
34255	6.9578e-07\\
34568	6.8071e-07\\
34885	6.6594e-07\\
35205	6.5147e-07\\
35527	6.3729e-07\\
35852	6.234e-07\\
36181	6.0979e-07\\
36512	5.9646e-07\\
36847	5.8339e-07\\
37184	5.7059e-07\\
37525	5.5805e-07\\
37869	5.4577e-07\\
38215	5.3374e-07\\
38565	5.2195e-07\\
38919	5.104e-07\\
39275	4.9909e-07\\
39635	4.8802e-07\\
39998	4.7717e-07\\
40364	4.6654e-07\\
40734	4.5613e-07\\
41107	4.4594e-07\\
41484	4.3596e-07\\
41864	4.2619e-07\\
42247	4.1662e-07\\
42634	4.0724e-07\\
43025	3.9807e-07\\
43419	3.8908e-07\\
43817	3.8028e-07\\
44218	3.7167e-07\\
44623	3.6323e-07\\
45032	3.5498e-07\\
45444	3.469e-07\\
45860	3.3898e-07\\
46281	3.3124e-07\\
46704	3.2366e-07\\
47132	3.1624e-07\\
47564	3.0897e-07\\
48000	3.0187e-07\\
48439	2.9491e-07\\
48883	2.881e-07\\
49331	2.8144e-07\\
49783	2.7492e-07\\
50239	2.6854e-07\\
50699	2.623e-07\\
51163	2.5619e-07\\
51632	2.5021e-07\\
52105	2.4437e-07\\
52582	2.3865e-07\\
53064	2.3305e-07\\
53550	2.2758e-07\\
54040	2.2222e-07\\
54535	2.1699e-07\\
55035	2.1187e-07\\
55539	2.0685e-07\\
56048	2.0195e-07\\
56561	1.9716e-07\\
57079	1.9248e-07\\
57602	1.8789e-07\\
58130	1.8341e-07\\
58662	1.7903e-07\\
59200	1.7475e-07\\
59742	1.7056e-07\\
60289	1.6646e-07\\
60841	1.6246e-07\\
61399	1.5854e-07\\
61961	1.5472e-07\\
62529	1.5098e-07\\
63101	1.4732e-07\\
63679	1.4375e-07\\
64263	1.4025e-07\\
64851	1.3684e-07\\
65445	1.335e-07\\
66045	1.3024e-07\\
66650	1.2706e-07\\
67260	1.2395e-07\\
67876	1.209e-07\\
68498	1.1793e-07\\
69125	1.1503e-07\\
69759	1.1219e-07\\
70398	1.0942e-07\\
71042	1.0671e-07\\
71693	1.0407e-07\\
72350	1.0148e-07\\
73013	9.8958e-08\\
73681	9.6493e-08\\
74356	9.4086e-08\\
75037	9.1734e-08\\
75725	8.9438e-08\\
76418	8.7195e-08\\
77118	8.5005e-08\\
77825	8.2867e-08\\
78538	8.0779e-08\\
79257	7.874e-08\\
79983	7.6749e-08\\
80716	7.4805e-08\\
81455	7.2908e-08\\
82201	7.1056e-08\\
82954	6.9248e-08\\
83714	6.7482e-08\\
84481	6.576e-08\\
85255	6.4078e-08\\
86036	6.2437e-08\\
86824	6.0835e-08\\
87619	5.9271e-08\\
88421	5.7746e-08\\
89231	5.6257e-08\\
90049	5.4804e-08\\
90874	5.3387e-08\\
91706	5.2003e-08\\
92546	5.0654e-08\\
93394	4.9337e-08\\
94249	4.8053e-08\\
95113	4.68e-08\\
95984	4.5577e-08\\
96863	4.4385e-08\\
97750	4.3222e-08\\
98646	4.2088e-08\\
99549	4.0981e-08\\
1.0046e+05	3.9902e-08\\
1.0138e+05	3.885e-08\\
1.0231e+05	3.7824e-08\\
1.0325e+05	3.6823e-08\\
1.0419e+05	3.5847e-08\\
1.0515e+05	3.4895e-08\\
1.0611e+05	3.3968e-08\\
1.0708e+05	3.3063e-08\\
1.0806e+05	3.2181e-08\\
1.0905e+05	3.1322e-08\\
1.1005e+05	3.0483e-08\\
1.1106e+05	2.9667e-08\\
1.1208e+05	2.887e-08\\
1.131e+05	2.8094e-08\\
1.1414e+05	2.7337e-08\\
1.1519e+05	2.66e-08\\
1.1624e+05	2.5882e-08\\
1.1731e+05	2.5181e-08\\
1.1838e+05	2.4499e-08\\
1.1946e+05	2.3834e-08\\
1.2056e+05	2.3186e-08\\
1.2166e+05	2.2555e-08\\
1.2278e+05	2.1939e-08\\
1.239e+05	2.134e-08\\
1.2504e+05	2.0756e-08\\
1.2618e+05	2.0188e-08\\
1.2734e+05	1.9634e-08\\
1.285e+05	1.9094e-08\\
1.2968e+05	1.8568e-08\\
1.3087e+05	1.8056e-08\\
1.3207e+05	1.7558e-08\\
1.3328e+05	1.7072e-08\\
1.345e+05	1.6599e-08\\
1.3573e+05	1.6139e-08\\
1.3697e+05	1.569e-08\\
1.3823e+05	1.5253e-08\\
1.3949e+05	1.4828e-08\\
1.4077e+05	1.4414e-08\\
1.4206e+05	1.4011e-08\\
1.4336e+05	1.3619e-08\\
1.4468e+05	1.3237e-08\\
1.46e+05	1.2865e-08\\
1.4734e+05	1.2503e-08\\
1.4869e+05	1.2151e-08\\
1.5005e+05	1.1808e-08\\
1.5142e+05	1.1475e-08\\
1.5281e+05	1.115e-08\\
1.5421e+05	1.0834e-08\\
1.5562e+05	1.0526e-08\\
1.5705e+05	1.0227e-08\\
1.5849e+05	9.9358e-09\\
1.5994e+05	9.6525e-09\\
1.614e+05	9.3768e-09\\
1.6288e+05	9.1086e-09\\
1.6438e+05	8.8477e-09\\
1.6588e+05	8.5939e-09\\
1.674e+05	8.3471e-09\\
1.6893e+05	8.1069e-09\\
1.7048e+05	7.8734e-09\\
1.7204e+05	7.6462e-09\\
1.7362e+05	7.4252e-09\\
1.7521e+05	7.2104e-09\\
1.7681e+05	7.0014e-09\\
1.7843e+05	6.7982e-09\\
1.8007e+05	6.6006e-09\\
1.8172e+05	6.4085e-09\\
1.8338e+05	6.2217e-09\\
1.8506e+05	6.0401e-09\\
1.8676e+05	5.8635e-09\\
1.8847e+05	5.6919e-09\\
1.9019e+05	5.525e-09\\
1.9194e+05	5.3628e-09\\
1.9369e+05	5.2052e-09\\
1.9547e+05	5.0519e-09\\
1.9726e+05	4.903e-09\\
1.9907e+05	4.7583e-09\\
2.0089e+05	4.6176e-09\\
2.0273e+05	4.4809e-09\\
2.0459e+05	4.348e-09\\
2.0646e+05	4.2189e-09\\
2.0835e+05	4.0935e-09\\
2.1026e+05	3.9717e-09\\
2.1219e+05	3.8533e-09\\
2.1413e+05	3.7382e-09\\
2.1609e+05	3.6265e-09\\
2.1807e+05	3.5179e-09\\
2.2007e+05	3.4125e-09\\
2.2208e+05	3.3101e-09\\
2.2412e+05	3.2106e-09\\
2.2617e+05	3.114e-09\\
2.2824e+05	3.0201e-09\\
2.3033e+05	2.929e-09\\
2.3244e+05	2.8405e-09\\
2.3457e+05	2.7545e-09\\
2.3672e+05	2.6711e-09\\
2.3889e+05	2.5901e-09\\
2.4108e+05	2.5114e-09\\
2.4329e+05	2.435e-09\\
2.4551e+05	2.3608e-09\\
2.4776e+05	2.2888e-09\\
2.5003e+05	2.2189e-09\\
2.5003e+05	1.8469e-09\\
2.4776e+05	1.9072e-09\\
2.4551e+05	1.9694e-09\\
2.4329e+05	2.0335e-09\\
2.4108e+05	2.0996e-09\\
2.3889e+05	2.1678e-09\\
2.3672e+05	2.2381e-09\\
2.3457e+05	2.3106e-09\\
2.3244e+05	2.3853e-09\\
2.3033e+05	2.4624e-09\\
2.2824e+05	2.5418e-09\\
2.2617e+05	2.6236e-09\\
2.2412e+05	2.708e-09\\
2.2208e+05	2.7949e-09\\
2.2007e+05	2.8846e-09\\
2.1807e+05	2.9769e-09\\
2.1609e+05	3.0721e-09\\
2.1413e+05	3.1702e-09\\
2.1219e+05	3.2713e-09\\
2.1026e+05	3.3754e-09\\
2.0835e+05	3.4828e-09\\
2.0646e+05	3.5933e-09\\
2.0459e+05	3.7073e-09\\
2.0273e+05	3.8246e-09\\
2.0089e+05	3.9455e-09\\
1.9907e+05	4.0701e-09\\
1.9726e+05	4.1984e-09\\
1.9547e+05	4.3305e-09\\
1.9369e+05	4.4666e-09\\
1.9194e+05	4.6068e-09\\
1.9019e+05	4.7511e-09\\
1.8847e+05	4.8998e-09\\
1.8676e+05	5.0529e-09\\
1.8506e+05	5.2105e-09\\
1.8338e+05	5.3728e-09\\
1.8172e+05	5.5399e-09\\
1.8007e+05	5.712e-09\\
1.7843e+05	5.8891e-09\\
1.7681e+05	6.0715e-09\\
1.7521e+05	6.2592e-09\\
1.7362e+05	6.4525e-09\\
1.7204e+05	6.6514e-09\\
1.7048e+05	6.8561e-09\\
1.6893e+05	7.0668e-09\\
1.674e+05	7.2837e-09\\
1.6588e+05	7.5068e-09\\
1.6438e+05	7.7365e-09\\
1.6288e+05	7.9728e-09\\
1.614e+05	8.2159e-09\\
1.5994e+05	8.4661e-09\\
1.5849e+05	8.7235e-09\\
1.5705e+05	8.9883e-09\\
1.5562e+05	9.2607e-09\\
1.5421e+05	9.5409e-09\\
1.5281e+05	9.8292e-09\\
1.5142e+05	1.0126e-08\\
1.5005e+05	1.0431e-08\\
1.4869e+05	1.0744e-08\\
1.4734e+05	1.1067e-08\\
1.46e+05	1.1399e-08\\
1.4468e+05	1.174e-08\\
1.4336e+05	1.209e-08\\
1.4206e+05	1.2451e-08\\
1.4077e+05	1.2822e-08\\
1.3949e+05	1.3203e-08\\
1.3823e+05	1.3595e-08\\
1.3697e+05	1.3998e-08\\
1.3573e+05	1.4412e-08\\
1.345e+05	1.4838e-08\\
1.3328e+05	1.5275e-08\\
1.3207e+05	1.5725e-08\\
1.3087e+05	1.6187e-08\\
1.2968e+05	1.6662e-08\\
1.285e+05	1.715e-08\\
1.2734e+05	1.7651e-08\\
1.2618e+05	1.8167e-08\\
1.2504e+05	1.8696e-08\\
1.239e+05	1.924e-08\\
1.2278e+05	1.9799e-08\\
1.2166e+05	2.0373e-08\\
1.2056e+05	2.0962e-08\\
1.1946e+05	2.1568e-08\\
1.1838e+05	2.219e-08\\
1.1731e+05	2.2829e-08\\
1.1624e+05	2.3485e-08\\
1.1519e+05	2.4159e-08\\
1.1414e+05	2.4851e-08\\
1.131e+05	2.5561e-08\\
1.1208e+05	2.6291e-08\\
1.1106e+05	2.704e-08\\
1.1005e+05	2.7809e-08\\
1.0905e+05	2.8598e-08\\
1.0806e+05	2.9408e-08\\
1.0708e+05	3.024e-08\\
1.0611e+05	3.1094e-08\\
1.0515e+05	3.197e-08\\
1.0419e+05	3.2869e-08\\
1.0325e+05	3.3792e-08\\
1.0231e+05	3.4739e-08\\
1.0138e+05	3.5711e-08\\
1.0046e+05	3.6708e-08\\
99549	3.7731e-08\\
98646	3.878e-08\\
97750	3.9857e-08\\
96863	4.0961e-08\\
95984	4.2094e-08\\
95113	4.3256e-08\\
94249	4.4448e-08\\
93394	4.567e-08\\
92546	4.6924e-08\\
91706	4.8209e-08\\
90874	4.9527e-08\\
90049	5.0879e-08\\
89231	5.2265e-08\\
88421	5.3685e-08\\
87619	5.5142e-08\\
86824	5.6635e-08\\
86036	5.8165e-08\\
85255	5.9734e-08\\
84481	6.1342e-08\\
83714	6.299e-08\\
82954	6.4678e-08\\
82201	6.6409e-08\\
81455	6.8182e-08\\
80716	6.9999e-08\\
79983	7.186e-08\\
79257	7.3768e-08\\
78538	7.5721e-08\\
77825	7.7723e-08\\
77118	7.9773e-08\\
76418	8.1873e-08\\
75725	8.4024e-08\\
75037	8.6227e-08\\
74356	8.8483e-08\\
73681	9.0794e-08\\
73013	9.3159e-08\\
72350	9.5582e-08\\
71693	9.8062e-08\\
71042	1.006e-07\\
70398	1.032e-07\\
69759	1.0586e-07\\
69125	1.0859e-07\\
68498	1.1138e-07\\
67876	1.1423e-07\\
67260	1.1715e-07\\
66650	1.2014e-07\\
66045	1.232e-07\\
65445	1.2633e-07\\
64851	1.2954e-07\\
64263	1.3282e-07\\
63679	1.3617e-07\\
63101	1.396e-07\\
62529	1.4312e-07\\
61961	1.4671e-07\\
61399	1.5038e-07\\
60841	1.5414e-07\\
60289	1.5799e-07\\
59742	1.6192e-07\\
59200	1.6594e-07\\
58662	1.7006e-07\\
58130	1.7427e-07\\
57602	1.7857e-07\\
57079	1.8297e-07\\
56561	1.8747e-07\\
56048	1.9207e-07\\
55539	1.9677e-07\\
55035	2.0158e-07\\
54535	2.065e-07\\
54040	2.1152e-07\\
53550	2.1666e-07\\
53064	2.2191e-07\\
52582	2.2728e-07\\
52105	2.3277e-07\\
51632	2.3838e-07\\
51163	2.4411e-07\\
50699	2.4997e-07\\
50239	2.5596e-07\\
49783	2.6208e-07\\
49331	2.6833e-07\\
48883	2.7472e-07\\
48439	2.8125e-07\\
48000	2.8792e-07\\
47564	2.9474e-07\\
47132	3.017e-07\\
46704	3.0882e-07\\
46281	3.1609e-07\\
45860	3.2351e-07\\
45444	3.311e-07\\
45032	3.3885e-07\\
44623	3.4676e-07\\
44218	3.5485e-07\\
43817	3.6311e-07\\
43419	3.7154e-07\\
43025	3.8016e-07\\
42634	3.8896e-07\\
42247	3.9794e-07\\
41864	4.0712e-07\\
41484	4.1649e-07\\
41107	4.2606e-07\\
40734	4.3583e-07\\
40364	4.4581e-07\\
39998	4.56e-07\\
39635	4.664e-07\\
39275	4.7703e-07\\
38919	4.8787e-07\\
38565	4.9894e-07\\
38215	5.1025e-07\\
37869	5.2179e-07\\
37525	5.3357e-07\\
37184	5.456e-07\\
36847	5.5788e-07\\
36512	5.7041e-07\\
36181	5.832e-07\\
35852	5.9626e-07\\
35527	6.0959e-07\\
35205	6.232e-07\\
34885	6.3709e-07\\
34568	6.5126e-07\\
34255	6.6572e-07\\
33944	6.8049e-07\\
33636	6.9556e-07\\
33330	7.1093e-07\\
33028	7.2663e-07\\
32728	7.4264e-07\\
32431	7.5898e-07\\
32137	7.7566e-07\\
31845	7.9268e-07\\
31556	8.1005e-07\\
31269	8.2777e-07\\
30986	8.4585e-07\\
30704	8.643e-07\\
30426	8.8312e-07\\
30149	9.0233e-07\\
29876	9.2193e-07\\
29605	9.4193e-07\\
29336	9.6233e-07\\
29070	9.8315e-07\\
28806	1.0044e-06\\
28544	1.0261e-06\\
28285	1.0482e-06\\
28028	1.0707e-06\\
27774	1.0937e-06\\
27522	1.1172e-06\\
27272	1.1411e-06\\
27025	1.1656e-06\\
26779	1.1905e-06\\
26536	1.2159e-06\\
26295	1.2418e-06\\
26057	1.2683e-06\\
25820	1.2953e-06\\
25586	1.3228e-06\\
25354	1.3509e-06\\
25123	1.3795e-06\\
24895	1.4088e-06\\
24669	1.4386e-06\\
24446	1.469e-06\\
24224	1.5e-06\\
24004	1.5316e-06\\
23786	1.5639e-06\\
23570	1.5968e-06\\
23356	1.6304e-06\\
23144	1.6646e-06\\
22934	1.6995e-06\\
22726	1.7352e-06\\
22520	1.7715e-06\\
22315	1.8086e-06\\
22113	1.8464e-06\\
21912	1.885e-06\\
21713	1.9243e-06\\
21516	1.9644e-06\\
21321	2.0053e-06\\
21127	2.0471e-06\\
20935	2.0897e-06\\
20745	2.1331e-06\\
20557	2.1774e-06\\
20370	2.2225e-06\\
20186	2.2686e-06\\
20002	2.3156e-06\\
19821	2.3636e-06\\
19641	2.4124e-06\\
19463	2.4623e-06\\
19286	2.5132e-06\\
19111	2.5651e-06\\
18937	2.618e-06\\
18765	2.6719e-06\\
18595	2.727e-06\\
18426	2.7831e-06\\
18259	2.8404e-06\\
18093	2.8988e-06\\
17929	2.9584e-06\\
17766	3.0192e-06\\
17605	3.0811e-06\\
17445	3.1443e-06\\
17287	3.2088e-06\\
17130	3.2746e-06\\
16975	3.3417e-06\\
16821	3.4101e-06\\
16668	3.4799e-06\\
16517	3.551e-06\\
16367	3.6236e-06\\
16218	3.6977e-06\\
16071	3.7732e-06\\
15925	3.8502e-06\\
15780	3.9288e-06\\
15637	4.0089e-06\\
15495	4.0906e-06\\
15355	4.174e-06\\
15215	4.259e-06\\
15077	4.3457e-06\\
14940	4.4342e-06\\
14805	4.5244e-06\\
14670	4.6164e-06\\
14537	4.7102e-06\\
14405	4.8059e-06\\
14274	4.9036e-06\\
14145	5.0031e-06\\
14017	5.1047e-06\\
13889	5.2082e-06\\
13763	5.3138e-06\\
13638	5.4216e-06\\
13515	5.5314e-06\\
13392	5.6435e-06\\
13270	5.7577e-06\\
13150	5.8743e-06\\
13030	5.9931e-06\\
12912	6.1144e-06\\
12795	6.238e-06\\
12679	6.364e-06\\
12564	6.4926e-06\\
12450	6.6237e-06\\
12337	6.7574e-06\\
12225	6.8938e-06\\
12114	7.0328e-06\\
12004	7.1746e-06\\
11895	7.3191e-06\\
11787	7.4666e-06\\
11680	7.6169e-06\\
11574	7.7702e-06\\
11469	7.9265e-06\\
11365	8.0859e-06\\
11262	8.2484e-06\\
11159	8.414e-06\\
11058	8.583e-06\\
10958	8.7552e-06\\
10858	8.9308e-06\\
10760	9.1098e-06\\
10662	9.2923e-06\\
10565	9.4784e-06\\
10469	9.6681e-06\\
10374	9.8615e-06\\
10280	1.0059e-05\\
10187	1.026e-05\\
10094	1.0464e-05\\
10003	1.0673e-05\\
9912	1.0886e-05\\
9822	1.1103e-05\\
9732.9	1.1324e-05\\
9644.5	1.1549e-05\\
9557	1.1779e-05\\
9470.2	1.2013e-05\\
9384.3	1.2252e-05\\
9299.1	1.2495e-05\\
9214.7	1.2742e-05\\
9131.1	1.2995e-05\\
9048.2	1.3252e-05\\
8966	1.3514e-05\\
8884.7	1.3781e-05\\
8804	1.4053e-05\\
8724.1	1.433e-05\\
8644.9	1.4613e-05\\
8566.5	1.49e-05\\
8488.7	1.5194e-05\\
8411.6	1.5492e-05\\
8335.3	1.5796e-05\\
8259.6	1.6106e-05\\
8184.7	1.6422e-05\\
8110.4	1.6743e-05\\
8036.8	1.707e-05\\
7963.8	1.7404e-05\\
7891.5	1.7743e-05\\
7819.9	1.8089e-05\\
7748.9	1.8441e-05\\
7678.6	1.88e-05\\
7608.9	1.9165e-05\\
7539.8	1.9536e-05\\
7471.4	1.9915e-05\\
7403.6	2.03e-05\\
7336.4	2.0692e-05\\
7269.8	2.1092e-05\\
7203.8	2.1498e-05\\
7138.4	2.1912e-05\\
7073.6	2.2333e-05\\
7009.4	2.2762e-05\\
6945.8	2.3198e-05\\
6882.7	2.3642e-05\\
6820.3	2.4094e-05\\
6758.4	2.4554e-05\\
6697	2.5022e-05\\
6636.2	2.5498e-05\\
6576	2.5982e-05\\
6516.3	2.6475e-05\\
6457.2	2.6977e-05\\
6398.5	2.7487e-05\\
6340.5	2.8006e-05\\
6282.9	2.8533e-05\\
6225.9	2.907e-05\\
6169.4	2.9616e-05\\
6113.4	3.0172e-05\\
6057.9	3.0736e-05\\
6002.9	3.1311e-05\\
5948.4	3.1895e-05\\
5894.4	3.2489e-05\\
5840.9	3.3092e-05\\
5787.9	3.3706e-05\\
5735.4	3.433e-05\\
5683.3	3.4965e-05\\
5631.7	3.561e-05\\
5580.6	3.6265e-05\\
5530	3.6931e-05\\
5479.8	3.7608e-05\\
5430	3.8296e-05\\
5380.7	3.8995e-05\\
5331.9	3.9706e-05\\
5283.5	4.0428e-05\\
5235.5	4.1161e-05\\
5188	4.1906e-05\\
5140.9	4.2663e-05\\
5094.3	4.3432e-05\\
5048	4.4213e-05\\
5002.2	4.5006e-05\\
4956.8	4.5811e-05\\
4911.8	4.6629e-05\\
4867.2	4.746e-05\\
4823.1	4.8303e-05\\
4779.3	4.9159e-05\\
4735.9	5.0028e-05\\
4692.9	5.0911e-05\\
4650.3	5.1806e-05\\
4608.1	5.2715e-05\\
4566.3	5.3638e-05\\
4524.8	5.4574e-05\\
4483.8	5.5524e-05\\
4443.1	5.6488e-05\\
4402.7	5.7466e-05\\
4362.8	5.8459e-05\\
4323.2	5.9465e-05\\
4283.9	6.0486e-05\\
4245	6.1522e-05\\
4206.5	6.2572e-05\\
4168.3	6.3637e-05\\
4130.5	6.4717e-05\\
4093	6.5812e-05\\
4055.9	6.6922e-05\\
4019	6.8048e-05\\
3982.6	6.9188e-05\\
3946.4	7.0344e-05\\
3910.6	7.1516e-05\\
3875.1	7.2704e-05\\
3839.9	7.3907e-05\\
3805.1	7.5126e-05\\
3770.5	7.6361e-05\\
3736.3	7.7612e-05\\
3702.4	7.8879e-05\\
3668.8	8.0162e-05\\
3635.5	8.1461e-05\\
3602.5	8.2777e-05\\
3569.8	8.4109e-05\\
3537.4	8.5458e-05\\
3505.3	8.6823e-05\\
3473.5	8.8204e-05\\
3441.9	8.9602e-05\\
3410.7	9.1017e-05\\
3379.7	9.2448e-05\\
3349.1	9.3896e-05\\
3318.7	9.536e-05\\
3288.5	9.6841e-05\\
3258.7	9.8338e-05\\
3229.1	9.9852e-05\\
3199.8	0.00010138\\
3170.8	0.00010293\\
3142	0.00010449\\
3113.5	0.00010607\\
3085.2	0.00010767\\
3057.2	0.00010928\\
3029.4	0.00011091\\
3001.9	0.00011255\\
2974.7	0.00011421\\
2947.7	0.00011588\\
2920.9	0.00011757\\
2894.4	0.00011928\\
2868.2	0.000121\\
2842.1	0.00012274\\
2816.3	0.00012449\\
2790.8	0.00012625\\
2765.4	0.00012803\\
2740.3	0.00012982\\
2715.5	0.00013163\\
2690.8	0.00013345\\
2666.4	0.00013529\\
2642.2	0.00013713\\
2618.2	0.00013899\\
2594.4	0.00014087\\
2570.9	0.00014275\\
2547.6	0.00014465\\
2524.4	0.00014656\\
2501.5	0.00014848\\
2478.8	0.00015041\\
2456.3	0.00015235\\
2434	0.00015431\\
2411.9	0.00015627\\
2390	0.00015824\\
2368.3	0.00016023\\
2346.8	0.00016222\\
2325.5	0.00016422\\
2304.4	0.00016623\\
2283.5	0.00016825\\
2262.8	0.00017027\\
2242.2	0.00017231\\
2221.9	0.00017434\\
2201.7	0.00017639\\
2181.7	0.00017844\\
2161.9	0.0001805\\
2142.3	0.00018256\\
2122.9	0.00018463\\
2103.6	0.0001867\\
2084.5	0.00018878\\
2065.6	0.00019086\\
2046.8	0.00019294\\
2028.3	0.00019503\\
2009.8	0.00019711\\
1991.6	0.0001992\\
1973.5	0.00020129\\
1955.6	0.00020337\\
1937.9	0.00020546\\
1920.3	0.00020755\\
1902.8	0.00020963\\
1885.6	0.00021172\\
1868.5	0.0002138\\
1851.5	0.00021587\\
1834.7	0.00021795\\
1818	0.00022002\\
1801.5	0.00022208\\
1785.2	0.00022414\\
1769	0.00022619\\
1752.9	0.00022824\\
1737	0.00023028\\
1721.2	0.00023231\\
1705.6	0.00023433\\
1690.1	0.00023635\\
1674.8	0.00023836\\
1659.6	0.00024035\\
1644.5	0.00024234\\
1629.6	0.00024431\\
1614.8	0.00024627\\
1600.2	0.00024822\\
1585.6	0.00025016\\
1571.2	0.00025209\\
1557	0.000254\\
1542.8	0.00025589\\
1528.8	0.00025777\\
1515	0.00025964\\
1501.2	0.00026148\\
1487.6	0.00026332\\
1474.1	0.00026513\\
1460.7	0.00026692\\
1447.4	0.0002687\\
1434.3	0.00027046\\
1421.3	0.00027219\\
1408.4	0.00027391\\
1395.6	0.0002756\\
1382.9	0.00027728\\
1370.4	0.00027893\\
1357.9	0.00028055\\
1345.6	0.00028216\\
1333.4	0.00028374\\
1321.3	0.00028529\\
1309.3	0.00028683\\
1297.4	0.00028833\\
1285.7	0.00028981\\
1274	0.00029127\\
1262.4	0.0002927\\
1251	0.0002941\\
1239.6	0.00029547\\
1228.4	0.00029682\\
1217.2	0.00029815\\
1206.2	0.00029944\\
1195.2	0.00030071\\
1184.4	0.00030195\\
1173.6	0.00030316\\
1163	0.00030435\\
1152.4	0.00030551\\
1141.9	0.00030664\\
1131.6	0.00030774\\
1121.3	0.00030882\\
1111.1	0.00030987\\
1101	0.00031089\\
1091	0.00031188\\
1081.1	0.00031284\\
1071.3	0.00031378\\
1061.6	0.00031469\\
1052	0.00031557\\
1042.4	0.00031642\\
1033	0.00031724\\
1023.6	0.00031804\\
1014.3	0.0003188\\
1005.1	0.00031954\\
995.96	0.00032024\\
986.92	0.00032092\\
977.97	0.00032157\\
969.09	0.00032219\\
960.29	0.00032278\\
951.58	0.00032334\\
942.94	0.00032387\\
934.38	0.00032437\\
925.9	0.00032484\\
917.5	0.00032527\\
909.17	0.00032568\\
900.92	0.00032606\\
892.74	0.0003264\\
884.64	0.00032672\\
876.61	0.000327\\
868.65	0.00032726\\
860.76	0.00032748\\
852.95	0.00032767\\
845.21	0.00032783\\
837.54	0.00032795\\
829.94	0.00032805\\
822.4	0.00032812\\
814.94	0.00032815\\
807.54	0.00032815\\
800.21	0.00032812\\
792.95	0.00032806\\
785.75	0.00032797\\
778.62	0.00032785\\
771.55	0.0003277\\
764.55	0.00032751\\
757.61	0.0003273\\
750.73	0.00032706\\
743.92	0.00032678\\
737.16	0.00032647\\
730.47	0.00032614\\
723.84	0.00032577\\
717.27	0.00032538\\
710.76	0.00032496\\
704.31	0.0003245\\
697.92	0.00032402\\
691.58	0.00032351\\
685.31	0.00032297\\
679.09	0.0003224\\
672.92	0.00032181\\
666.81	0.00032118\\
660.76	0.00032053\\
654.76	0.00031986\\
648.82	0.00031915\\
642.93	0.00031842\\
637.1	0.00031766\\
631.31	0.00031688\\
625.58	0.00031607\\
619.9	0.00031524\\
614.28	0.00031438\\
608.7	0.0003135\\
603.18	0.00031259\\
597.7	0.00031166\\
592.28	0.00031071\\
586.9	0.00030973\\
581.57	0.00030873\\
576.29	0.00030771\\
571.06	0.00030667\\
565.88	0.0003056\\
560.74	0.00030451\\
555.65	0.00030341\\
550.61	0.00030228\\
545.61	0.00030113\\
540.66	0.00029996\\
535.75	0.00029877\\
530.89	0.00029756\\
526.07	0.00029634\\
521.3	0.00029509\\
516.56	0.00029383\\
511.88	0.00029255\\
507.23	0.00029125\\
502.63	0.00028993\\
498.06	0.0002886\\
493.54	0.00028725\\
489.06	0.00028589\\
484.62	0.00028451\\
480.23	0.00028311\\
475.87	0.0002817\\
471.55	0.00028028\\
467.27	0.00027884\\
463.03	0.00027739\\
458.82	0.00027592\\
454.66	0.00027444\\
450.53	0.00027295\\
446.44	0.00027145\\
442.39	0.00026994\\
438.37	0.00026841\\
434.4	0.00026687\\
430.45	0.00026532\\
426.55	0.00026376\\
422.67	0.00026219\\
418.84	0.00026061\\
415.04	0.00025903\\
411.27	0.00025743\\
407.54	0.00025582\\
403.84	0.00025421\\
400.17	0.00025258\\
396.54	0.00025095\\
392.94	0.00024932\\
389.37	0.00024767\\
385.84	0.00024602\\
382.34	0.00024436\\
378.87	0.0002427\\
375.43	0.00024103\\
372.02	0.00023935\\
368.64	0.00023767\\
365.3	0.00023599\\
361.98	0.0002343\\
358.7	0.00023261\\
355.44	0.00023091\\
352.21	0.00022921\\
349.02	0.00022751\\
345.85	0.0002258\\
342.71	0.00022409\\
339.6	0.00022238\\
336.52	0.00022067\\
333.46	0.00021895\\
330.43	0.00021724\\
327.44	0.00021552\\
324.46	0.0002138\\
321.52	0.00021208\\
318.6	0.00021037\\
315.71	0.00020865\\
312.84	0.00020693\\
310	0.00020521\\
307.19	0.00020349\\
304.4	0.00020178\\
301.64	0.00020007\\
298.9	0.00019835\\
296.19	0.00019664\\
293.5	0.00019493\\
290.83	0.00019323\\
288.19	0.00019153\\
285.58	0.00018983\\
282.99	0.00018813\\
280.42	0.00018644\\
277.87	0.00018475\\
275.35	0.00018306\\
272.85	0.00018138\\
270.37	0.0001797\\
267.92	0.00017803\\
265.49	0.00017636\\
263.08	0.00017469\\
260.69	0.00017304\\
258.32	0.00017138\\
255.98	0.00016974\\
253.66	0.00016809\\
251.35	0.00016646\\
249.07	0.00016483\\
246.81	0.0001632\\
244.57	0.00016159\\
242.35	0.00015998\\
240.15	0.00015837\\
237.97	0.00015678\\
235.81	0.00015519\\
233.67	0.0001536\\
231.55	0.00015203\\
229.45	0.00015046\\
227.37	0.0001489\\
225.3	0.00014735\\
223.26	0.00014581\\
221.23	0.00014427\\
219.22	0.00014274\\
217.23	0.00014123\\
215.26	0.00013972\\
213.31	0.00013821\\
211.37	0.00013672\\
209.45	0.00013524\\
207.55	0.00013376\\
205.67	0.00013229\\
203.8	0.00013084\\
201.95	0.00012939\\
200.12	0.00012795\\
198.3	0.00012652\\
196.5	0.0001251\\
194.72	0.00012369\\
192.95	0.00012229\\
191.2	0.0001209\\
189.46	0.00011952\\
187.74	0.00011815\\
186.04	0.00011679\\
184.35	0.00011544\\
182.68	0.0001141\\
181.02	0.00011277\\
179.38	0.00011145\\
177.75	0.00011014\\
176.14	0.00010884\\
174.54	0.00010755\\
172.95	0.00010627\\
171.38	0.00010501\\
169.83	0.00010375\\
168.29	0.0001025\\
166.76	0.00010126\\
165.24	0.00010004\\
163.74	9.882e-05\\
162.26	9.7615e-05\\
160.79	9.6419e-05\\
159.33	9.5235e-05\\
157.88	9.4061e-05\\
156.45	9.2897e-05\\
155.03	9.1745e-05\\
153.62	9.0602e-05\\
152.23	8.9471e-05\\
150.84	8.8349e-05\\
149.47	8.7239e-05\\
148.12	8.6138e-05\\
146.77	8.5049e-05\\
145.44	8.397e-05\\
144.12	8.2901e-05\\
142.81	8.1843e-05\\
141.52	8.0795e-05\\
140.23	7.9758e-05\\
138.96	7.8731e-05\\
137.7	7.7714e-05\\
136.45	7.6708e-05\\
135.21	7.5712e-05\\
133.98	7.4726e-05\\
132.77	7.375e-05\\
131.56	7.2785e-05\\
130.37	7.183e-05\\
129.18	7.0885e-05\\
128.01	6.995e-05\\
126.85	6.9025e-05\\
125.7	6.811e-05\\
124.56	6.7204e-05\\
123.43	6.6309e-05\\
122.31	6.5424e-05\\
121.2	6.4548e-05\\
120.1	6.3682e-05\\
119.01	6.2826e-05\\
117.93	6.1979e-05\\
116.85	6.1142e-05\\
115.79	6.0314e-05\\
114.74	5.9496e-05\\
113.7	5.8687e-05\\
112.67	5.7888e-05\\
111.65	5.7097e-05\\
110.63	5.6316e-05\\
109.63	5.5544e-05\\
108.63	5.4781e-05\\
107.65	5.4027e-05\\
106.67	5.3282e-05\\
105.7	5.2546e-05\\
104.74	5.1819e-05\\
103.79	5.11e-05\\
102.85	5.039e-05\\
101.92	4.9689e-05\\
100.99	4.8996e-05\\
100.08	4.8312e-05\\
99.167	4.7636e-05\\
98.267	4.6968e-05\\
97.375	4.6309e-05\\
96.491	4.5657e-05\\
95.615	4.5014e-05\\
94.747	4.4379e-05\\
93.887	4.3752e-05\\
93.035	4.3133e-05\\
92.191	4.2521e-05\\
91.354	4.1917e-05\\
90.525	4.1321e-05\\
89.703	4.0733e-05\\
88.889	4.0152e-05\\
88.082	3.9578e-05\\
87.283	3.9012e-05\\
86.49	3.8453e-05\\
85.705	3.7902e-05\\
84.927	3.7357e-05\\
84.156	3.682e-05\\
83.393	3.6289e-05\\
82.636	3.5766e-05\\
81.886	3.5249e-05\\
81.142	3.474e-05\\
80.406	3.4237e-05\\
79.676	3.374e-05\\
78.953	3.325e-05\\
78.236	3.2767e-05\\
77.526	3.229e-05\\
76.822	3.182e-05\\
76.125	3.1355e-05\\
75.434	3.0897e-05\\
74.749	3.0445e-05\\
74.071	3e-05\\
73.399	2.956e-05\\
72.732	2.9126e-05\\
72.072	2.8698e-05\\
71.418	2.8276e-05\\
70.77	2.786e-05\\
70.127	2.7449e-05\\
69.491	2.7044e-05\\
68.86	2.6644e-05\\
68.235	2.625e-05\\
67.616	2.5862e-05\\
67.002	2.5478e-05\\
66.394	2.51e-05\\
65.791	2.4728e-05\\
65.194	2.436e-05\\
64.602	2.3997e-05\\
64.016	2.364e-05\\
63.435	2.3287e-05\\
62.859	2.294e-05\\
62.289	2.2597e-05\\
61.723	2.2259e-05\\
61.163	2.1926e-05\\
60.608	2.1597e-05\\
60.058	2.1273e-05\\
59.512	2.0953e-05\\
58.972	2.0638e-05\\
58.437	2.0328e-05\\
57.907	2.0022e-05\\
57.381	1.972e-05\\
56.86	1.9422e-05\\
56.344	1.9129e-05\\
55.833	1.884e-05\\
55.326	1.8554e-05\\
54.824	1.8273e-05\\
54.326	1.7996e-05\\
53.833	1.7723e-05\\
53.344	1.7454e-05\\
52.86	1.7188e-05\\
52.38	1.6927e-05\\
51.905	1.6669e-05\\
51.434	1.6414e-05\\
50.967	1.6164e-05\\
50.504	1.5917e-05\\
50.046	1.5673e-05\\
49.592	1.5433e-05\\
49.141	1.5197e-05\\
48.695	1.4964e-05\\
48.253	1.4734e-05\\
47.815	1.4507e-05\\
47.381	1.4284e-05\\
46.951	1.4064e-05\\
46.525	1.3847e-05\\
46.103	1.3634e-05\\
45.684	1.3423e-05\\
45.27	1.3216e-05\\
44.859	1.3011e-05\\
44.452	1.2809e-05\\
44.048	1.2611e-05\\
43.648	1.2415e-05\\
43.252	1.2222e-05\\
42.86	1.2032e-05\\
42.471	1.1844e-05\\
42.085	1.166e-05\\
41.703	1.1478e-05\\
41.325	1.1298e-05\\
40.949	1.1122e-05\\
40.578	1.0948e-05\\
40.209	1.0776e-05\\
39.845	1.0607e-05\\
39.483	1.044e-05\\
39.124	1.0276e-05\\
38.769	1.0114e-05\\
38.417	9.9551e-06\\
38.069	9.7981e-06\\
37.723	9.6433e-06\\
37.381	9.4909e-06\\
37.042	9.3406e-06\\
36.705	9.1926e-06\\
36.372	9.0468e-06\\
36.042	8.9031e-06\\
35.715	8.7615e-06\\
35.391	8.622e-06\\
35.069	8.4846e-06\\
34.751	8.3492e-06\\
34.436	8.2158e-06\\
34.123	8.0844e-06\\
33.813	7.9549e-06\\
33.506	7.8274e-06\\
33.202	7.7017e-06\\
32.901	7.5779e-06\\
32.602	7.456e-06\\
32.306	7.3358e-06\\
32.013	7.2175e-06\\
31.723	7.1009e-06\\
31.435	6.986e-06\\
31.149	6.8729e-06\\
30.867	6.7614e-06\\
30.586	6.6516e-06\\
30.309	6.5435e-06\\
30.034	6.437e-06\\
29.761	6.332e-06\\
29.491	6.2287e-06\\
29.223	6.1269e-06\\
28.958	6.0266e-06\\
28.695	5.9278e-06\\
28.435	5.8305e-06\\
28.177	5.7347e-06\\
27.921	5.6403e-06\\
27.667	5.5474e-06\\
}--cycle;
            % \input{../../../.tex/20/KS/SizeDistributionFittingsfig1.tex}
    
            \nextgroupplot[%
                % xmin=10,xmax=1e+06,
                % ymin=0,ymax=0.0004,
            ] % This file was created by matlab2tikz.
%
\definecolor{mycolor1}{rgb}{0.00000,0.44706,0.69804}%
%
\addplot[ybar interval, fill=mycolor1, fill opacity=0.35, area legend, draw=none] table[row sep=crcr, x=Lower, y=Count] {%
Lower	Upper	Count\\
29.871	42.524	0\\
42.524	60.539	1.0362e-05\\
60.539	86.185	2.4955e-05\\
86.185	122.69	4.3095e-05\\
122.69	174.67	8.7222e-05\\
174.67	248.67	0.00013947\\
248.67	354.01	0.00021139\\
354.01	503.97	0.00027402\\
503.97	717.47	0.00032876\\
717.47	1021.4	0.00033069\\
1021.4	1454.1	0.00029176\\
1454.1	2070.1	0.00022676\\
2070.1	2947	0.00015299\\
2947	4195.5	9.0099e-05\\
4195.5	5972.8	4.3348e-05\\
5972.8	8503	2.1469e-05\\
8503	12105	1.0661e-05\\
12105	17233	4.9246e-06\\
17233	24534	2.4584e-06\\
24534	34927	9.9816e-07\\
34927	49722	3.6228e-07\\
49722	70786	1.6712e-07\\
70786	1.0077e+05	4.0018e-08\\
1.0077e+05	1.4346e+05	2.4988e-08\\
1.4346e+05	2.0424e+05	2.6766e-08\\
2.0424e+05	2.0424e+05	2.6766e-08\\
};
\addlegendentry{Istogramma esatto}

\addplot [color=mycolor1, only marks, every error bar/.append style={opacity=0.45}, mark=*, mark size=0pt, draw=none, forget plot]
 plot [error bars/.cd, y dir=both, y explicit, error bar style={line width=1pt, color=mycolor1}, error mark options={mark=none,mark size=0pt}]
 table[row sep=crcr, y error plus index=2, y error minus index=3]{%
35.64	0	0	0\\
50.738	1.0362e-05	7.532e-06	7.532e-06\\
72.232	2.4955e-05	9.7789e-06	9.7789e-06\\
102.83	4.3095e-05	1.1487e-05	1.1487e-05\\
146.39	8.7222e-05	1.1439e-05	1.1439e-05\\
208.41	0.00013947	1.2922e-05	1.2922e-05\\
296.7	0.00021139	1.4379e-05	1.4379e-05\\
422.39	0.00027402	1.3278e-05	1.3278e-05\\
601.32	0.00032876	1.3347e-05	1.3347e-05\\
856.06	0.00033069	1.058e-05	1.058e-05\\
1218.7	0.00029176	7.3766e-06	7.3766e-06\\
1735	0.00022676	5.0392e-06	5.0392e-06\\
2470	0.00015299	4.1912e-06	4.1912e-06\\
3516.3	9.0099e-05	2.3474e-06	2.3474e-06\\
5005.9	4.3348e-05	1.5021e-06	1.5021e-06\\
7126.5	2.1469e-05	7.8454e-07	7.8454e-07\\
10145	1.0661e-05	5.1317e-07	5.1317e-07\\
14443	4.9246e-06	2.8187e-07	2.8187e-07\\
20562	2.4584e-06	1.5887e-07	1.5887e-07\\
29272	9.9816e-07	8.5309e-08	8.5309e-08\\
41673	3.6228e-07	4.1762e-08	4.1762e-08\\
59327	1.6712e-07	2.3919e-08	2.3919e-08\\
84459	4.0018e-08	1.0751e-08	1.0751e-08\\
1.2024e+05	2.4988e-08	6.3523e-09	6.3523e-09\\
1.7117e+05	2.6766e-08	4.2679e-09	4.2679e-09\\
};
\addplot [color=mycolor1]
  table[row sep=crcr]{%
29.871	6.5645e-06\\
34.105	8.173e-06\\
38.258	9.8633e-06\\
42.539	1.1718e-05\\
46.883	1.3714e-05\\
51.216	1.5814e-05\\
55.949	1.8232e-05\\
60.582	2.0718e-05\\
65.021	2.3208e-05\\
69.785	2.5993e-05\\
74.898	2.9106e-05\\
79.679	3.2127e-05\\
84.764	3.5452e-05\\
90.175	3.9107e-05\\
95.086	4.2522e-05\\
100.26	4.6218e-05\\
105.73	5.021e-05\\
111.48	5.4516e-05\\
117.56	5.9154e-05\\
123.96	6.4139e-05\\
129.56	6.8569e-05\\
135.41	7.3258e-05\\
141.53	7.8215e-05\\
147.93	8.3444e-05\\
154.61	8.8951e-05\\
161.6	9.474e-05\\
168.9	0.00010081\\
176.53	0.00010717\\
184.51	0.00011381\\
192.85	0.00012073\\
201.56	0.00012792\\
210.67	0.00013538\\
222.14	0.00014466\\
234.24	0.00015429\\
247	0.00016423\\
260.45	0.00017445\\
277.08	0.00018667\\
300.02	0.00020274\\
384.27	0.00025309\\
405.2	0.00026346\\
427.27	0.00027352\\
446.58	0.00028158\\
466.75	0.00028931\\
483.55	0.00029522\\
500.95	0.00030086\\
518.98	0.0003062\\
537.66	0.00031122\\
557.01	0.00031589\\
577.06	0.0003202\\
597.82	0.00032411\\
613.89	0.00032677\\
630.38	0.0003292\\
647.32	0.00033137\\
664.72	0.00033329\\
682.58	0.00033495\\
700.92	0.00033635\\
719.76	0.00033747\\
739.1	0.00033831\\
758.96	0.00033887\\
779.35	0.00033915\\
800.3	0.00033914\\
821.8	0.00033884\\
843.88	0.00033826\\
866.56	0.00033738\\
889.85	0.00033622\\
913.76	0.00033477\\
938.31	0.00033304\\
963.53	0.00033102\\
989.42	0.00032874\\
1016	0.00032618\\
1043.3	0.00032335\\
1071.3	0.00032027\\
1100.1	0.00031694\\
1129.7	0.00031336\\
1160.1	0.00030955\\
1201.8	0.00030412\\
1245.1	0.00029832\\
1289.9	0.00029216\\
1336.3	0.00028569\\
1384.4	0.00027891\\
1434.2	0.00027187\\
1499	0.00026273\\
1566.7	0.00025329\\
1652.1	0.00024162\\
1757.5	0.0002277\\
1971.5	0.00020142\\
2153.7	0.00018136\\
2291.2	0.00016762\\
2416	0.00015614\\
2547.6	0.000145\\
2662.7	0.00013602\\
2783	0.00012734\\
2908.7	0.00011899\\
3040.2	0.00011097\\
3177.6	0.00010331\\
3321.1	9.6004e-05\\
3471.2	8.9068e-05\\
3628.1	8.25e-05\\
3792	7.6299e-05\\
3963.3	7.0462e-05\\
4142.4	6.4981e-05\\
4329.6	5.9848e-05\\
4525.2	5.5052e-05\\
4729.7	5.0582e-05\\
4943.4	4.6424e-05\\
5166.8	4.2566e-05\\
5400.3	3.8991e-05\\
5644.3	3.5686e-05\\
5951.7	3.2054e-05\\
6275.9	2.8763e-05\\
6617.7	2.5786e-05\\
6978.1	2.3099e-05\\
7423.5	2.0298e-05\\
7897.4	1.7823e-05\\
8401.4	1.5639e-05\\
9017	1.3461e-05\\
9677.7	1.158e-05\\
10479	9.7706e-06\\
11447	8.0859e-06\\
12616	6.5626e-06\\
14028	5.2225e-06\\
15736	4.0732e-06\\
17967	3.0516e-06\\
21066	2.1491e-06\\
25363	1.416e-06\\
31915	8.3081e-07\\
42725	4.079e-07\\
64729	1.3648e-07\\
1.3481e+05	1.5173e-08\\
2.0424e+05	3.767e-09\\
};
\addlegendentry{Adatt. BLN esatto ($\mathit{ML}$)}


\addplot[area legend, draw=none, fill=mycolor1, fill opacity=0.15, forget plot]
table[row sep=crcr] {%
x	y\\
29.871	8.3469e-06\\
30.136	8.4643e-06\\
30.403	8.5832e-06\\
30.673	8.7037e-06\\
30.946	8.8256e-06\\
31.22	8.9492e-06\\
31.497	9.0743e-06\\
31.777	9.2009e-06\\
32.059	9.3293e-06\\
32.344	9.4592e-06\\
32.631	9.5908e-06\\
32.921	9.7241e-06\\
33.213	9.859e-06\\
33.508	9.9957e-06\\
33.805	1.0134e-05\\
34.105	1.0274e-05\\
34.408	1.0416e-05\\
34.714	1.056e-05\\
35.022	1.0706e-05\\
35.333	1.0853e-05\\
35.647	1.1002e-05\\
35.963	1.1153e-05\\
36.282	1.1307e-05\\
36.604	1.1462e-05\\
36.929	1.1619e-05\\
37.257	1.1778e-05\\
37.588	1.1939e-05\\
37.922	1.2102e-05\\
38.258	1.2267e-05\\
38.598	1.2434e-05\\
38.941	1.2603e-05\\
39.287	1.2775e-05\\
39.635	1.2948e-05\\
39.987	1.3124e-05\\
40.342	1.3302e-05\\
40.7	1.3483e-05\\
41.062	1.3665e-05\\
41.426	1.385e-05\\
41.794	1.4038e-05\\
42.165	1.4227e-05\\
42.539	1.4419e-05\\
42.917	1.4614e-05\\
43.298	1.4811e-05\\
43.683	1.501e-05\\
44.07	1.5212e-05\\
44.462	1.5417e-05\\
44.856	1.5624e-05\\
45.255	1.5834e-05\\
45.656	1.6046e-05\\
46.062	1.6261e-05\\
46.471	1.6479e-05\\
46.883	1.67e-05\\
47.3	1.6924e-05\\
47.719	1.715e-05\\
48.143	1.7379e-05\\
48.571	1.7611e-05\\
49.002	1.7846e-05\\
49.437	1.8084e-05\\
49.876	1.8326e-05\\
50.319	1.857e-05\\
50.765	1.8817e-05\\
51.216	1.9068e-05\\
51.671	1.9321e-05\\
52.129	1.9578e-05\\
52.592	1.9838e-05\\
53.059	2.0102e-05\\
53.53	2.0369e-05\\
54.005	2.0639e-05\\
54.485	2.0913e-05\\
54.969	2.119e-05\\
55.457	2.1471e-05\\
55.949	2.1755e-05\\
56.446	2.2043e-05\\
56.947	2.2335e-05\\
57.453	2.263e-05\\
57.963	2.2929e-05\\
58.477	2.3232e-05\\
58.996	2.3539e-05\\
59.52	2.385e-05\\
60.049	2.4165e-05\\
60.582	2.4484e-05\\
61.12	2.4807e-05\\
61.662	2.5134e-05\\
62.21	2.5465e-05\\
62.762	2.58e-05\\
63.319	2.614e-05\\
63.881	2.6484e-05\\
64.448	2.6833e-05\\
65.021	2.7186e-05\\
65.598	2.7543e-05\\
66.18	2.7905e-05\\
66.768	2.8272e-05\\
67.361	2.8643e-05\\
67.959	2.902e-05\\
68.562	2.94e-05\\
69.171	2.9786e-05\\
69.785	3.0177e-05\\
70.404	3.0573e-05\\
71.029	3.0974e-05\\
71.66	3.1379e-05\\
72.296	3.1791e-05\\
72.938	3.2207e-05\\
73.586	3.2628e-05\\
74.239	3.3055e-05\\
74.898	3.3488e-05\\
75.563	3.3926e-05\\
76.234	3.4369e-05\\
76.911	3.4818e-05\\
77.594	3.5273e-05\\
78.282	3.5734e-05\\
78.977	3.62e-05\\
79.679	3.6673e-05\\
80.386	3.7151e-05\\
81.1	3.7635e-05\\
81.82	3.8126e-05\\
82.546	3.8622e-05\\
83.279	3.9125e-05\\
84.018	3.9634e-05\\
84.764	4.015e-05\\
85.517	4.0672e-05\\
86.276	4.1201e-05\\
87.042	4.1736e-05\\
87.815	4.2278e-05\\
88.595	4.2826e-05\\
89.381	4.3382e-05\\
90.175	4.3944e-05\\
90.975	4.4514e-05\\
91.783	4.509e-05\\
92.598	4.5674e-05\\
93.42	4.6264e-05\\
94.249	4.6863e-05\\
95.086	4.7468e-05\\
95.93	4.8081e-05\\
96.782	4.8701e-05\\
97.641	4.9329e-05\\
98.508	4.9965e-05\\
99.383	5.0608e-05\\
100.26	5.1259e-05\\
101.16	5.1918e-05\\
102.05	5.2585e-05\\
102.96	5.3261e-05\\
103.87	5.3944e-05\\
104.8	5.4635e-05\\
105.73	5.5335e-05\\
106.66	5.6043e-05\\
107.61	5.6759e-05\\
108.57	5.7484e-05\\
109.53	5.8218e-05\\
110.5	5.896e-05\\
111.48	5.9711e-05\\
112.47	6.0471e-05\\
113.47	6.1239e-05\\
114.48	6.2017e-05\\
115.5	6.2803e-05\\
116.52	6.3599e-05\\
117.56	6.4404e-05\\
118.6	6.5218e-05\\
119.65	6.6041e-05\\
120.72	6.6874e-05\\
121.79	6.7716e-05\\
122.87	6.8568e-05\\
123.96	6.9429e-05\\
125.06	7.03e-05\\
126.17	7.1181e-05\\
127.29	7.2071e-05\\
128.42	7.2972e-05\\
129.56	7.3882e-05\\
130.71	7.4802e-05\\
131.87	7.5733e-05\\
133.04	7.6673e-05\\
134.22	7.7624e-05\\
135.41	7.8584e-05\\
136.62	7.9556e-05\\
137.83	8.0537e-05\\
139.05	8.1529e-05\\
140.29	8.2531e-05\\
141.53	8.3544e-05\\
142.79	8.4567e-05\\
144.06	8.5601e-05\\
145.34	8.6646e-05\\
146.63	8.7701e-05\\
147.93	8.8768e-05\\
149.24	8.9844e-05\\
150.57	9.0932e-05\\
151.9	9.2031e-05\\
153.25	9.314e-05\\
154.61	9.4261e-05\\
155.99	9.5392e-05\\
157.37	9.6535e-05\\
158.77	9.7688e-05\\
160.18	9.8853e-05\\
161.6	0.00010003\\
163.03	0.00010121\\
164.48	0.00010241\\
165.94	0.00010362\\
167.42	0.00010484\\
168.9	0.00010607\\
170.4	0.00010732\\
171.91	0.00010857\\
173.44	0.00010983\\
174.98	0.00011111\\
176.53	0.0001124\\
178.1	0.0001137\\
179.68	0.00011501\\
181.28	0.00011633\\
182.89	0.00011766\\
184.51	0.000119\\
186.15	0.00012036\\
187.8	0.00012172\\
189.47	0.0001231\\
191.15	0.00012449\\
192.85	0.00012588\\
194.56	0.00012729\\
196.29	0.00012871\\
198.03	0.00013014\\
199.79	0.00013159\\
201.56	0.00013304\\
203.35	0.0001345\\
205.16	0.00013598\\
206.98	0.00013746\\
208.82	0.00013896\\
210.67	0.00014046\\
212.54	0.00014198\\
214.43	0.00014351\\
216.33	0.00014504\\
218.25	0.00014659\\
220.19	0.00014814\\
222.14	0.00014971\\
224.12	0.00015129\\
226.11	0.00015287\\
228.11	0.00015447\\
230.14	0.00015607\\
232.18	0.00015769\\
234.24	0.00015931\\
236.32	0.00016094\\
238.42	0.00016258\\
240.54	0.00016423\\
242.67	0.00016589\\
244.83	0.00016756\\
247	0.00016924\\
249.19	0.00017092\\
251.41	0.00017261\\
253.64	0.00017431\\
255.89	0.00017602\\
258.16	0.00017773\\
260.45	0.00017945\\
262.77	0.00018118\\
265.1	0.00018291\\
267.45	0.00018465\\
269.83	0.0001864\\
272.22	0.00018815\\
274.64	0.00018991\\
277.08	0.00019168\\
279.54	0.00019345\\
282.02	0.00019522\\
284.52	0.000197\\
287.05	0.00019879\\
289.6	0.00020058\\
292.17	0.00020237\\
294.76	0.00020417\\
297.38	0.00020597\\
300.02	0.00020777\\
302.68	0.00020958\\
305.37	0.00021139\\
308.08	0.0002132\\
310.82	0.00021501\\
313.58	0.00021683\\
316.36	0.00021865\\
319.17	0.00022046\\
322	0.00022228\\
324.86	0.0002241\\
327.75	0.00022592\\
330.66	0.00022774\\
333.59	0.00022956\\
336.55	0.00023137\\
339.54	0.00023319\\
342.56	0.000235\\
345.6	0.00023681\\
348.67	0.00023862\\
351.76	0.00024043\\
354.88	0.00024223\\
358.04	0.00024403\\
361.21	0.00024583\\
364.42	0.00024762\\
367.66	0.00024941\\
370.92	0.00025119\\
374.21	0.00025297\\
377.54	0.00025474\\
380.89	0.0002565\\
384.27	0.00025826\\
387.68	0.00026001\\
391.12	0.00026176\\
394.6	0.00026349\\
398.1	0.00026522\\
401.63	0.00026694\\
405.2	0.00026865\\
408.8	0.00027035\\
412.43	0.00027204\\
416.09	0.00027372\\
419.78	0.00027539\\
423.51	0.00027705\\
427.27	0.0002787\\
431.06	0.00028033\\
434.89	0.00028196\\
438.75	0.00028357\\
442.65	0.00028516\\
446.58	0.00028675\\
450.54	0.00028832\\
454.54	0.00028987\\
458.58	0.00029141\\
462.65	0.00029294\\
466.75	0.00029445\\
470.9	0.00029594\\
475.08	0.00029742\\
479.3	0.00029888\\
483.55	0.00030033\\
487.84	0.00030175\\
492.18	0.00030316\\
496.55	0.00030455\\
500.95	0.00030592\\
505.4	0.00030727\\
509.89	0.00030861\\
514.42	0.00030992\\
518.98	0.00031121\\
523.59	0.00031248\\
528.24	0.00031373\\
532.93	0.00031496\\
537.66	0.00031617\\
542.43	0.00031736\\
547.25	0.00031852\\
552.11	0.00031966\\
557.01	0.00032078\\
561.95	0.00032187\\
566.94	0.00032294\\
571.98	0.00032399\\
577.06	0.00032501\\
582.18	0.00032601\\
587.35	0.00032698\\
592.56	0.00032793\\
597.82	0.00032885\\
603.13	0.00032974\\
608.49	0.00033061\\
613.89	0.00033146\\
619.34	0.00033227\\
624.84	0.00033306\\
630.38	0.00033382\\
635.98	0.00033455\\
641.63	0.00033526\\
647.32	0.00033593\\
653.07	0.00033658\\
658.87	0.0003372\\
664.72	0.00033779\\
670.62	0.00033836\\
676.57	0.00033889\\
682.58	0.00033939\\
688.64	0.00033986\\
694.75	0.00034031\\
700.92	0.00034072\\
707.14	0.0003411\\
713.42	0.00034146\\
719.76	0.00034178\\
726.15	0.00034207\\
732.59	0.00034233\\
739.1	0.00034256\\
745.66	0.00034275\\
752.28	0.00034292\\
758.96	0.00034305\\
765.7	0.00034315\\
772.5	0.00034323\\
779.35	0.00034326\\
786.27	0.00034327\\
793.25	0.00034325\\
800.3	0.00034319\\
807.4	0.0003431\\
814.57	0.00034298\\
821.8	0.00034282\\
829.1	0.00034264\\
836.46	0.00034242\\
843.88	0.00034217\\
851.38	0.00034189\\
858.94	0.00034157\\
866.56	0.00034122\\
874.25	0.00034085\\
882.02	0.00034043\\
889.85	0.00033999\\
897.75	0.00033951\\
905.72	0.00033901\\
913.76	0.00033847\\
921.87	0.0003379\\
930.06	0.00033729\\
938.31	0.00033666\\
946.64	0.00033599\\
955.05	0.0003353\\
963.53	0.00033457\\
972.08	0.00033381\\
980.71	0.00033302\\
989.42	0.0003322\\
998.2	0.00033135\\
1007.1	0.00033047\\
1016	0.00032956\\
1025	0.00032862\\
1034.1	0.00032765\\
1043.3	0.00032665\\
1052.6	0.00032563\\
1061.9	0.00032457\\
1071.3	0.00032349\\
1080.9	0.00032238\\
1090.5	0.00032124\\
1100.1	0.00032007\\
1109.9	0.00031887\\
1119.8	0.00031765\\
1129.7	0.00031641\\
1139.7	0.00031513\\
1149.8	0.00031383\\
1160.1	0.00031251\\
1170.4	0.00031116\\
1180.7	0.00030979\\
1191.2	0.00030839\\
1201.8	0.00030697\\
1212.5	0.00030552\\
1223.2	0.00030406\\
1234.1	0.00030257\\
1245.1	0.00030105\\
1256.1	0.00029952\\
1267.3	0.00029797\\
1278.5	0.00029639\\
1289.9	0.00029479\\
1301.3	0.00029318\\
1312.9	0.00029154\\
1324.5	0.00028989\\
1336.3	0.00028822\\
1348.1	0.00028653\\
1360.1	0.00028482\\
1372.2	0.00028309\\
1384.4	0.00028135\\
1396.7	0.00027959\\
1409.1	0.00027782\\
1421.6	0.00027603\\
1434.2	0.00027422\\
1446.9	0.00027241\\
1459.8	0.00027057\\
1472.7	0.00026873\\
1485.8	0.00026687\\
1499	0.000265\\
1512.3	0.00026311\\
1525.7	0.00026122\\
1539.3	0.00025931\\
1552.9	0.0002574\\
1566.7	0.00025547\\
1580.6	0.00025353\\
1594.7	0.00025159\\
1608.8	0.00024963\\
1623.1	0.00024767\\
1637.5	0.0002457\\
1652.1	0.00024372\\
1666.7	0.00024174\\
1681.5	0.00023974\\
1696.5	0.00023775\\
1711.5	0.00023574\\
1726.7	0.00023373\\
1742	0.00023172\\
1757.5	0.0002297\\
1773.1	0.00022768\\
1788.9	0.00022566\\
1804.7	0.00022363\\
1820.8	0.0002216\\
1836.9	0.00021956\\
1853.2	0.00021753\\
1869.7	0.00021549\\
1886.3	0.00021345\\
1903	0.00021142\\
1919.9	0.00020938\\
1937	0.00020734\\
1954.2	0.0002053\\
1971.5	0.00020327\\
1989	0.00020123\\
2006.7	0.0001992\\
2024.5	0.00019717\\
2042.5	0.00019514\\
2060.6	0.00019312\\
2078.9	0.0001911\\
2097.4	0.00018908\\
2116	0.00018707\\
2134.8	0.00018506\\
2153.7	0.00018306\\
2172.8	0.00018106\\
2192.1	0.00017907\\
2211.6	0.00017708\\
2231.2	0.0001751\\
2251	0.00017313\\
2271	0.00017116\\
2291.2	0.0001692\\
2311.5	0.00016725\\
2332	0.00016531\\
2352.7	0.00016338\\
2373.6	0.00016145\\
2394.7	0.00015953\\
2416	0.00015763\\
2437.4	0.00015573\\
2459.1	0.00015384\\
2480.9	0.00015196\\
2502.9	0.00015009\\
2525.1	0.00014824\\
2547.6	0.00014639\\
2570.2	0.00014456\\
2593	0.00014273\\
2616	0.00014092\\
2639.2	0.00013912\\
2662.7	0.00013733\\
2686.3	0.00013555\\
2710.2	0.00013379\\
2734.2	0.00013204\\
2758.5	0.0001303\\
2783	0.00012857\\
2807.7	0.00012686\\
2832.6	0.00012516\\
2857.8	0.00012347\\
2883.1	0.0001218\\
2908.7	0.00012014\\
2934.6	0.00011849\\
2960.6	0.00011686\\
2986.9	0.00011525\\
3013.4	0.00011364\\
3040.2	0.00011205\\
3067.2	0.00011048\\
3094.4	0.00010892\\
3121.9	0.00010737\\
3149.6	0.00010584\\
3177.6	0.00010432\\
3205.8	0.00010282\\
3234.2	0.00010134\\
3262.9	9.9863e-05\\
3291.9	9.8406e-05\\
3321.1	9.6963e-05\\
3350.6	9.5535e-05\\
3380.4	9.4122e-05\\
3410.4	9.2724e-05\\
3440.7	9.1341e-05\\
3471.2	8.9973e-05\\
3502	8.8619e-05\\
3533.1	8.7281e-05\\
3564.5	8.5957e-05\\
3596.1	8.4649e-05\\
3628.1	8.3355e-05\\
3660.3	8.2076e-05\\
3692.8	8.0812e-05\\
3725.5	7.9563e-05\\
3758.6	7.8328e-05\\
3792	7.7108e-05\\
3825.7	7.5903e-05\\
3859.6	7.4712e-05\\
3893.9	7.3536e-05\\
3928.5	7.2374e-05\\
3963.3	7.1227e-05\\
3998.5	7.0094e-05\\
4034	6.8975e-05\\
4069.8	6.7871e-05\\
4106	6.678e-05\\
4142.4	6.5704e-05\\
4179.2	6.4642e-05\\
4216.3	6.3593e-05\\
4253.7	6.2558e-05\\
4291.5	6.1538e-05\\
4329.6	6.053e-05\\
4368	5.9537e-05\\
4406.8	5.8556e-05\\
4445.9	5.7589e-05\\
4485.4	5.6636e-05\\
4525.2	5.5695e-05\\
4565.4	5.4768e-05\\
4605.9	5.3853e-05\\
4646.8	5.2951e-05\\
4688.1	5.2062e-05\\
4729.7	5.1186e-05\\
4771.7	5.0322e-05\\
4814.1	4.9471e-05\\
4856.8	4.8632e-05\\
4899.9	4.7805e-05\\
4943.4	4.6991e-05\\
4987.3	4.6188e-05\\
5031.6	4.5397e-05\\
5076.3	4.4618e-05\\
5121.3	4.385e-05\\
5166.8	4.3094e-05\\
5212.7	4.235e-05\\
5259	4.1617e-05\\
5305.7	4.0895e-05\\
5352.8	4.0184e-05\\
5400.3	3.9484e-05\\
5448.2	3.8794e-05\\
5496.6	3.8116e-05\\
5545.4	3.7448e-05\\
5594.6	3.679e-05\\
5644.3	3.6143e-05\\
5694.4	3.5506e-05\\
5745	3.4879e-05\\
5796	3.4263e-05\\
5847.4	3.3656e-05\\
5899.3	3.3058e-05\\
5951.7	3.2471e-05\\
6004.6	3.1893e-05\\
6057.9	3.1324e-05\\
6111.7	3.0765e-05\\
6165.9	3.0215e-05\\
6220.7	2.9673e-05\\
6275.9	2.9141e-05\\
6331.6	2.8618e-05\\
6387.8	2.8103e-05\\
6444.5	2.7597e-05\\
6501.7	2.7099e-05\\
6559.5	2.661e-05\\
6617.7	2.6128e-05\\
6676.5	2.5655e-05\\
6735.7	2.519e-05\\
6795.5	2.4733e-05\\
6855.9	2.4284e-05\\
6916.7	2.3842e-05\\
6978.1	2.3408e-05\\
7040.1	2.2981e-05\\
7102.6	2.2562e-05\\
7165.7	2.215e-05\\
7229.3	2.1745e-05\\
7293.5	2.1347e-05\\
7358.2	2.0956e-05\\
7423.5	2.0572e-05\\
7489.4	2.0195e-05\\
7555.9	1.9824e-05\\
7623	1.946e-05\\
7690.7	1.9102e-05\\
7759	1.875e-05\\
7827.9	1.8405e-05\\
7897.4	1.8066e-05\\
7967.5	1.7733e-05\\
8038.2	1.7405e-05\\
8109.6	1.7084e-05\\
8181.6	1.6768e-05\\
8254.2	1.6458e-05\\
8327.5	1.6154e-05\\
8401.4	1.5855e-05\\
8476	1.5561e-05\\
8551.3	1.5273e-05\\
8627.2	1.499e-05\\
8703.8	1.4711e-05\\
8781.1	1.4438e-05\\
8859	1.417e-05\\
8937.7	1.3907e-05\\
9017	1.3649e-05\\
9097.1	1.3395e-05\\
9177.8	1.3146e-05\\
9259.3	1.2901e-05\\
9341.5	1.2661e-05\\
9424.5	1.2426e-05\\
9508.1	1.2194e-05\\
9592.5	1.1967e-05\\
9677.7	1.1744e-05\\
9763.6	1.1525e-05\\
9850.3	1.131e-05\\
9937.8	1.11e-05\\
10026	1.0892e-05\\
10115	1.0689e-05\\
10205	1.049e-05\\
10295	1.0294e-05\\
10387	1.0102e-05\\
10479	9.9131e-06\\
10572	9.7279e-06\\
10666	9.5461e-06\\
10761	9.3677e-06\\
10856	9.1926e-06\\
10953	9.0208e-06\\
11050	8.8521e-06\\
11148	8.6865e-06\\
11247	8.524e-06\\
11347	8.3645e-06\\
11447	8.208e-06\\
11549	8.0543e-06\\
11652	7.9036e-06\\
11755	7.7556e-06\\
11859	7.6103e-06\\
11965	7.4677e-06\\
12071	7.3278e-06\\
12178	7.1905e-06\\
12286	7.0557e-06\\
12395	6.9234e-06\\
12505	6.7936e-06\\
12616	6.6661e-06\\
12728	6.5411e-06\\
12841	6.4183e-06\\
12955	6.2978e-06\\
13070	6.1796e-06\\
13186	6.0635e-06\\
13304	5.9496e-06\\
13422	5.8378e-06\\
13541	5.7281e-06\\
13661	5.6204e-06\\
13782	5.5147e-06\\
13905	5.4109e-06\\
14028	5.3091e-06\\
14153	5.2092e-06\\
14278	5.1111e-06\\
14405	5.0148e-06\\
14533	4.9203e-06\\
14662	4.8275e-06\\
14792	4.7365e-06\\
14923	4.6471e-06\\
15056	4.5594e-06\\
15190	4.4733e-06\\
15325	4.3888e-06\\
15461	4.3058e-06\\
15598	4.2244e-06\\
15736	4.1445e-06\\
15876	4.066e-06\\
16017	3.989e-06\\
16159	3.9134e-06\\
16303	3.8392e-06\\
16447	3.7664e-06\\
16593	3.6949e-06\\
16741	3.6247e-06\\
16889	3.5559e-06\\
17039	3.4882e-06\\
17191	3.4219e-06\\
17343	3.3567e-06\\
17497	3.2928e-06\\
17652	3.23e-06\\
17809	3.1684e-06\\
17967	3.1079e-06\\
18127	3.0485e-06\\
18288	2.9902e-06\\
18450	2.933e-06\\
18614	2.8768e-06\\
18779	2.8217e-06\\
18946	2.7675e-06\\
19114	2.7144e-06\\
19284	2.6622e-06\\
19455	2.611e-06\\
19628	2.5608e-06\\
19802	2.5114e-06\\
19978	2.463e-06\\
20155	2.4155e-06\\
20334	2.3688e-06\\
20515	2.323e-06\\
20697	2.278e-06\\
20881	2.2338e-06\\
21066	2.1905e-06\\
21253	2.148e-06\\
21442	2.1062e-06\\
21632	2.0652e-06\\
21824	2.025e-06\\
22018	1.9855e-06\\
22213	1.9467e-06\\
22411	1.9087e-06\\
22609	1.8713e-06\\
22810	1.8346e-06\\
23013	1.7986e-06\\
23217	1.7633e-06\\
23423	1.7286e-06\\
23631	1.6946e-06\\
23841	1.6612e-06\\
24053	1.6284e-06\\
24266	1.5962e-06\\
24482	1.5646e-06\\
24699	1.5336e-06\\
24918	1.5031e-06\\
25139	1.4733e-06\\
25363	1.4439e-06\\
25588	1.4152e-06\\
25815	1.3869e-06\\
26044	1.3592e-06\\
26275	1.332e-06\\
26509	1.3053e-06\\
26744	1.2791e-06\\
26981	1.2534e-06\\
27221	1.2282e-06\\
27463	1.2034e-06\\
27706	1.1791e-06\\
27952	1.1552e-06\\
28201	1.1318e-06\\
28451	1.1089e-06\\
28704	1.0864e-06\\
28958	1.0643e-06\\
29216	1.0426e-06\\
29475	1.0213e-06\\
29737	1.0004e-06\\
30001	9.7991e-07\\
30267	9.5981e-07\\
30536	9.4009e-07\\
30807	9.2075e-07\\
31080	9.0177e-07\\
31356	8.8315e-07\\
31635	8.6489e-07\\
31915	8.4697e-07\\
32199	8.2939e-07\\
32485	8.1215e-07\\
32773	7.9524e-07\\
33064	7.7866e-07\\
33358	7.6239e-07\\
33654	7.4643e-07\\
33953	7.3078e-07\\
34254	7.1544e-07\\
34558	7.0038e-07\\
34865	6.8562e-07\\
35174	6.7115e-07\\
35487	6.5696e-07\\
35802	6.4304e-07\\
36120	6.2939e-07\\
36440	6.1601e-07\\
36764	6.0289e-07\\
37090	5.9003e-07\\
37420	5.7742e-07\\
37752	5.6505e-07\\
38087	5.5293e-07\\
38425	5.4105e-07\\
38766	5.294e-07\\
39110	5.1798e-07\\
39458	5.0679e-07\\
39808	4.9582e-07\\
40161	4.8507e-07\\
40518	4.7454e-07\\
40878	4.6421e-07\\
41241	4.5409e-07\\
41607	4.4417e-07\\
41976	4.3446e-07\\
42349	4.2493e-07\\
42725	4.156e-07\\
43104	4.0646e-07\\
43487	3.975e-07\\
43873	3.8873e-07\\
44262	3.8013e-07\\
44655	3.7171e-07\\
45052	3.6345e-07\\
45452	3.5537e-07\\
45855	3.4745e-07\\
46262	3.397e-07\\
46673	3.3211e-07\\
47087	3.2467e-07\\
47506	3.1738e-07\\
47927	3.1025e-07\\
48353	3.0326e-07\\
48782	2.9642e-07\\
49215	2.8972e-07\\
49652	2.8316e-07\\
50093	2.7673e-07\\
50538	2.7045e-07\\
50986	2.6429e-07\\
51439	2.5826e-07\\
51896	2.5236e-07\\
52356	2.4658e-07\\
52821	2.4093e-07\\
53290	2.354e-07\\
53763	2.2998e-07\\
54241	2.2468e-07\\
54722	2.1949e-07\\
55208	2.1441e-07\\
55698	2.0944e-07\\
56193	2.0458e-07\\
56692	1.9982e-07\\
57195	1.9516e-07\\
57703	1.9061e-07\\
58215	1.8615e-07\\
58732	1.8179e-07\\
59253	1.7752e-07\\
59779	1.7334e-07\\
60310	1.6926e-07\\
60845	1.6527e-07\\
61386	1.6136e-07\\
61931	1.5754e-07\\
62481	1.538e-07\\
63035	1.5014e-07\\
63595	1.4657e-07\\
64159	1.4307e-07\\
64729	1.3965e-07\\
65304	1.3631e-07\\
65884	1.3304e-07\\
66468	1.2984e-07\\
67059	1.2672e-07\\
67654	1.2366e-07\\
68255	1.2067e-07\\
68861	1.1775e-07\\
69472	1.149e-07\\
70089	1.1211e-07\\
70711	1.0938e-07\\
71339	1.0671e-07\\
71972	1.041e-07\\
72611	1.0156e-07\\
73256	9.9068e-08\\
73906	9.6635e-08\\
74562	9.4258e-08\\
75224	9.1935e-08\\
75892	8.9665e-08\\
76566	8.7447e-08\\
77246	8.528e-08\\
77931	8.3162e-08\\
78623	8.1094e-08\\
79321	7.9073e-08\\
80026	7.71e-08\\
80736	7.5172e-08\\
81453	7.3288e-08\\
82176	7.1449e-08\\
82906	6.9653e-08\\
83642	6.7898e-08\\
84384	6.6185e-08\\
85133	6.4512e-08\\
85889	6.2878e-08\\
86652	6.1283e-08\\
87421	5.9725e-08\\
88197	5.8204e-08\\
88980	5.672e-08\\
89770	5.527e-08\\
90567	5.3855e-08\\
91371	5.2474e-08\\
92183	5.1126e-08\\
93001	4.981e-08\\
93827	4.8526e-08\\
94660	4.7272e-08\\
95500	4.6049e-08\\
96348	4.4855e-08\\
97203	4.369e-08\\
98066	4.2554e-08\\
98937	4.1445e-08\\
99815	4.0363e-08\\
1.007e+05	3.9307e-08\\
1.016e+05	3.8277e-08\\
1.025e+05	3.7273e-08\\
1.0341e+05	3.6293e-08\\
1.0433e+05	3.5337e-08\\
1.0525e+05	3.4405e-08\\
1.0619e+05	3.3495e-08\\
1.0713e+05	3.2608e-08\\
1.0808e+05	3.1744e-08\\
1.0904e+05	3.09e-08\\
1.1001e+05	3.0078e-08\\
1.1098e+05	2.9276e-08\\
1.1197e+05	2.8494e-08\\
1.1296e+05	2.7732e-08\\
1.1397e+05	2.6988e-08\\
1.1498e+05	2.6264e-08\\
1.16e+05	2.5558e-08\\
1.1703e+05	2.4869e-08\\
1.1807e+05	2.4198e-08\\
1.1912e+05	2.3544e-08\\
1.2017e+05	2.2906e-08\\
1.2124e+05	2.2285e-08\\
1.2232e+05	2.168e-08\\
1.234e+05	2.109e-08\\
1.245e+05	2.0515e-08\\
1.256e+05	1.9955e-08\\
1.2672e+05	1.9409e-08\\
1.2784e+05	1.8877e-08\\
1.2898e+05	1.8359e-08\\
1.3012e+05	1.7854e-08\\
1.3128e+05	1.7363e-08\\
1.3245e+05	1.6884e-08\\
1.3362e+05	1.6417e-08\\
1.3481e+05	1.5963e-08\\
1.36e+05	1.5521e-08\\
1.3721e+05	1.509e-08\\
1.3843e+05	1.467e-08\\
1.3966e+05	1.4261e-08\\
1.409e+05	1.3863e-08\\
1.4215e+05	1.3476e-08\\
1.4341e+05	1.3099e-08\\
1.4469e+05	1.2732e-08\\
1.4597e+05	1.2374e-08\\
1.4727e+05	1.2026e-08\\
1.4857e+05	1.1687e-08\\
1.4989e+05	1.1357e-08\\
1.5122e+05	1.1036e-08\\
1.5257e+05	1.0724e-08\\
1.5392e+05	1.042e-08\\
1.5529e+05	1.0124e-08\\
1.5667e+05	9.8361e-09\\
1.5806e+05	9.5559e-09\\
1.5946e+05	9.2833e-09\\
1.6088e+05	9.018e-09\\
1.623e+05	8.7599e-09\\
1.6374e+05	8.5088e-09\\
1.652e+05	8.2645e-09\\
1.6666e+05	8.0268e-09\\
1.6814e+05	7.7957e-09\\
1.6964e+05	7.5708e-09\\
1.7114e+05	7.3521e-09\\
1.7266e+05	7.1394e-09\\
1.742e+05	6.9325e-09\\
1.7574e+05	6.7314e-09\\
1.773e+05	6.5357e-09\\
1.7888e+05	6.3455e-09\\
1.8046e+05	6.1605e-09\\
1.8207e+05	5.9806e-09\\
1.8368e+05	5.8057e-09\\
1.8531e+05	5.6357e-09\\
1.8696e+05	5.4705e-09\\
1.8862e+05	5.3098e-09\\
1.9029e+05	5.1536e-09\\
1.9198e+05	5.0018e-09\\
1.9369e+05	4.8542e-09\\
1.9541e+05	4.7108e-09\\
1.9714e+05	4.5714e-09\\
1.9889e+05	4.436e-09\\
2.0066e+05	4.3043e-09\\
2.0244e+05	4.1764e-09\\
2.0424e+05	4.0521e-09\\
2.0424e+05	3.482e-09\\
2.0244e+05	3.5927e-09\\
2.0066e+05	3.7068e-09\\
1.9889e+05	3.8243e-09\\
1.9714e+05	3.9453e-09\\
1.9541e+05	4.07e-09\\
1.9369e+05	4.1985e-09\\
1.9198e+05	4.3307e-09\\
1.9029e+05	4.4669e-09\\
1.8862e+05	4.6072e-09\\
1.8696e+05	4.7517e-09\\
1.8531e+05	4.9004e-09\\
1.8368e+05	5.0536e-09\\
1.8207e+05	5.2113e-09\\
1.8046e+05	5.3736e-09\\
1.7888e+05	5.5407e-09\\
1.773e+05	5.7128e-09\\
1.7574e+05	5.8899e-09\\
1.742e+05	6.0722e-09\\
1.7266e+05	6.2598e-09\\
1.7114e+05	6.4529e-09\\
1.6964e+05	6.6517e-09\\
1.6814e+05	6.8562e-09\\
1.6666e+05	7.0667e-09\\
1.652e+05	7.2832e-09\\
1.6374e+05	7.506e-09\\
1.623e+05	7.7353e-09\\
1.6088e+05	7.9712e-09\\
1.5946e+05	8.2138e-09\\
1.5806e+05	8.4634e-09\\
1.5667e+05	8.7202e-09\\
1.5529e+05	8.9843e-09\\
1.5392e+05	9.2559e-09\\
1.5257e+05	9.5353e-09\\
1.5122e+05	9.8226e-09\\
1.4989e+05	1.0118e-08\\
1.4857e+05	1.0422e-08\\
1.4727e+05	1.0734e-08\\
1.4597e+05	1.1055e-08\\
1.4469e+05	1.1386e-08\\
1.4341e+05	1.1725e-08\\
1.4215e+05	1.2074e-08\\
1.409e+05	1.2433e-08\\
1.3966e+05	1.2802e-08\\
1.3843e+05	1.3181e-08\\
1.3721e+05	1.357e-08\\
1.36e+05	1.3971e-08\\
1.3481e+05	1.4382e-08\\
1.3362e+05	1.4805e-08\\
1.3245e+05	1.524e-08\\
1.3128e+05	1.5686e-08\\
1.3012e+05	1.6145e-08\\
1.2898e+05	1.6616e-08\\
1.2784e+05	1.71e-08\\
1.2672e+05	1.7598e-08\\
1.256e+05	1.8108e-08\\
1.245e+05	1.8633e-08\\
1.234e+05	1.9172e-08\\
1.2232e+05	1.9726e-08\\
1.2124e+05	2.0294e-08\\
1.2017e+05	2.0878e-08\\
1.1912e+05	2.1477e-08\\
1.1807e+05	2.2093e-08\\
1.1703e+05	2.2725e-08\\
1.16e+05	2.3374e-08\\
1.1498e+05	2.404e-08\\
1.1397e+05	2.4724e-08\\
1.1296e+05	2.5426e-08\\
1.1197e+05	2.6146e-08\\
1.1098e+05	2.6886e-08\\
1.1001e+05	2.7645e-08\\
1.0904e+05	2.8424e-08\\
1.0808e+05	2.9223e-08\\
1.0713e+05	3.0043e-08\\
1.0619e+05	3.0885e-08\\
1.0525e+05	3.1748e-08\\
1.0433e+05	3.2634e-08\\
1.0341e+05	3.3543e-08\\
1.025e+05	3.4476e-08\\
1.016e+05	3.5432e-08\\
1.007e+05	3.6414e-08\\
99815	3.742e-08\\
98937	3.8452e-08\\
98066	3.9511e-08\\
97203	4.0596e-08\\
96348	4.1709e-08\\
95500	4.2851e-08\\
94660	4.4021e-08\\
93827	4.5221e-08\\
93001	4.6451e-08\\
92183	4.7712e-08\\
91371	4.9004e-08\\
90567	5.0329e-08\\
89770	5.1688e-08\\
88980	5.3079e-08\\
88197	5.4506e-08\\
87421	5.5968e-08\\
86652	5.7466e-08\\
85889	5.9001e-08\\
85133	6.0573e-08\\
84384	6.2184e-08\\
83642	6.3835e-08\\
82906	6.5526e-08\\
82176	6.7258e-08\\
81453	6.9033e-08\\
80736	7.085e-08\\
80026	7.2711e-08\\
79321	7.4617e-08\\
78623	7.6569e-08\\
77931	7.8568e-08\\
77246	8.0615e-08\\
76566	8.271e-08\\
75892	8.4856e-08\\
75224	8.7052e-08\\
74562	8.93e-08\\
73906	9.1602e-08\\
73256	9.3958e-08\\
72611	9.6369e-08\\
71972	9.8837e-08\\
71339	1.0136e-07\\
70711	1.0395e-07\\
70089	1.0659e-07\\
69472	1.093e-07\\
68861	1.1207e-07\\
68255	1.149e-07\\
67654	1.178e-07\\
67059	1.2076e-07\\
66468	1.2379e-07\\
65884	1.269e-07\\
65304	1.3007e-07\\
64729	1.3331e-07\\
64159	1.3663e-07\\
63595	1.4002e-07\\
63035	1.4349e-07\\
62481	1.4704e-07\\
61931	1.5067e-07\\
61386	1.5438e-07\\
60845	1.5817e-07\\
60310	1.6205e-07\\
59779	1.6601e-07\\
59253	1.7006e-07\\
58732	1.742e-07\\
58215	1.7843e-07\\
57703	1.8275e-07\\
57195	1.8717e-07\\
56692	1.9169e-07\\
56193	1.963e-07\\
55698	2.0102e-07\\
55208	2.0584e-07\\
54722	2.1076e-07\\
54241	2.1579e-07\\
53763	2.2093e-07\\
53290	2.2617e-07\\
52821	2.3153e-07\\
52356	2.3701e-07\\
51896	2.426e-07\\
51439	2.4831e-07\\
50986	2.5415e-07\\
50538	2.601e-07\\
50093	2.6619e-07\\
49652	2.724e-07\\
49215	2.7874e-07\\
48782	2.8522e-07\\
48353	2.9183e-07\\
47927	2.9858e-07\\
47506	3.0547e-07\\
47087	3.1251e-07\\
46673	3.1969e-07\\
46262	3.2702e-07\\
45855	3.3451e-07\\
45452	3.4215e-07\\
45052	3.4994e-07\\
44655	3.579e-07\\
44262	3.6602e-07\\
43873	3.7431e-07\\
43487	3.8277e-07\\
43104	3.914e-07\\
42725	4.0021e-07\\
42349	4.0919e-07\\
41976	4.1836e-07\\
41607	4.2772e-07\\
41241	4.3727e-07\\
40878	4.4701e-07\\
40518	4.5694e-07\\
40161	4.6708e-07\\
39808	4.7742e-07\\
39458	4.8797e-07\\
39110	4.9873e-07\\
38766	5.0971e-07\\
38425	5.209e-07\\
38087	5.3232e-07\\
37752	5.4397e-07\\
37420	5.5585e-07\\
37090	5.6797e-07\\
36764	5.8032e-07\\
36440	5.9293e-07\\
36120	6.0578e-07\\
35802	6.1888e-07\\
35487	6.3225e-07\\
35174	6.4588e-07\\
34865	6.5978e-07\\
34558	6.7395e-07\\
34254	6.884e-07\\
33953	7.0313e-07\\
33654	7.1815e-07\\
33358	7.3347e-07\\
33064	7.4908e-07\\
32773	7.65e-07\\
32485	7.8123e-07\\
32199	7.9778e-07\\
31915	8.1465e-07\\
31635	8.3184e-07\\
31356	8.4938e-07\\
31080	8.6725e-07\\
30807	8.8546e-07\\
30536	9.0403e-07\\
30267	9.2296e-07\\
30001	9.4226e-07\\
29737	9.6193e-07\\
29475	9.8197e-07\\
29216	1.0024e-06\\
28958	1.0232e-06\\
28704	1.0445e-06\\
28451	1.0661e-06\\
28201	1.0881e-06\\
27952	1.1106e-06\\
27706	1.1335e-06\\
27463	1.1569e-06\\
27221	1.1807e-06\\
26981	1.2049e-06\\
26744	1.2296e-06\\
26509	1.2548e-06\\
26275	1.2805e-06\\
26044	1.3066e-06\\
25815	1.3333e-06\\
25588	1.3604e-06\\
25363	1.3881e-06\\
25139	1.4163e-06\\
24918	1.4451e-06\\
24699	1.4744e-06\\
24482	1.5042e-06\\
24266	1.5346e-06\\
24053	1.5656e-06\\
23841	1.5972e-06\\
23631	1.6294e-06\\
23423	1.6622e-06\\
23217	1.6956e-06\\
23013	1.7296e-06\\
22810	1.7643e-06\\
22609	1.7997e-06\\
22411	1.8357e-06\\
22213	1.8724e-06\\
22018	1.9098e-06\\
21824	1.9479e-06\\
21632	1.9867e-06\\
21442	2.0263e-06\\
21253	2.0666e-06\\
21066	2.1077e-06\\
20881	2.1495e-06\\
20697	2.1922e-06\\
20515	2.2356e-06\\
20334	2.2799e-06\\
20155	2.325e-06\\
19978	2.371e-06\\
19802	2.4178e-06\\
19628	2.4655e-06\\
19455	2.5142e-06\\
19284	2.5637e-06\\
19114	2.6142e-06\\
18946	2.6656e-06\\
18779	2.718e-06\\
18614	2.7715e-06\\
18450	2.8259e-06\\
18288	2.8813e-06\\
18127	2.9378e-06\\
17967	2.9954e-06\\
17809	3.054e-06\\
17652	3.1138e-06\\
17497	3.1747e-06\\
17343	3.2367e-06\\
17191	3.3e-06\\
17039	3.3644e-06\\
16889	3.43e-06\\
16741	3.4969e-06\\
16593	3.5651e-06\\
16447	3.6345e-06\\
16303	3.7053e-06\\
16159	3.7774e-06\\
16017	3.8508e-06\\
15876	3.9257e-06\\
15736	4.002e-06\\
15598	4.0797e-06\\
15461	4.1589e-06\\
15325	4.2396e-06\\
15190	4.3218e-06\\
15056	4.4056e-06\\
14923	4.491e-06\\
14792	4.578e-06\\
14662	4.6666e-06\\
14533	4.757e-06\\
14405	4.849e-06\\
14278	4.9428e-06\\
14153	5.0384e-06\\
14028	5.1358e-06\\
13905	5.235e-06\\
13782	5.3361e-06\\
13661	5.4392e-06\\
13541	5.5442e-06\\
13422	5.6511e-06\\
13304	5.7602e-06\\
13186	5.8712e-06\\
13070	5.9844e-06\\
12955	6.0998e-06\\
12841	6.2173e-06\\
12728	6.337e-06\\
12616	6.459e-06\\
12505	6.5834e-06\\
12395	6.7101e-06\\
12286	6.8391e-06\\
12178	6.9707e-06\\
12071	7.1047e-06\\
11965	7.2413e-06\\
11859	7.3804e-06\\
11755	7.5222e-06\\
11652	7.6666e-06\\
11549	7.8138e-06\\
11447	7.9638e-06\\
11347	8.1166e-06\\
11247	8.2723e-06\\
11148	8.4309e-06\\
11050	8.5925e-06\\
10953	8.7571e-06\\
10856	8.9249e-06\\
10761	9.0958e-06\\
10666	9.27e-06\\
10572	9.4474e-06\\
10479	9.6281e-06\\
10387	9.8122e-06\\
10295	9.9998e-06\\
10205	1.0191e-05\\
10115	1.0386e-05\\
10026	1.0584e-05\\
9937.8	1.0786e-05\\
9850.3	1.0992e-05\\
9763.6	1.1202e-05\\
9677.7	1.1415e-05\\
9592.5	1.1633e-05\\
9508.1	1.1854e-05\\
9424.5	1.208e-05\\
9341.5	1.231e-05\\
9259.3	1.2544e-05\\
9177.8	1.2783e-05\\
9097.1	1.3026e-05\\
9017	1.3273e-05\\
8937.7	1.3525e-05\\
8859	1.3782e-05\\
8781.1	1.4043e-05\\
8703.8	1.431e-05\\
8627.2	1.4581e-05\\
8551.3	1.4857e-05\\
8476	1.5138e-05\\
8401.4	1.5424e-05\\
8327.5	1.5716e-05\\
8254.2	1.6013e-05\\
8181.6	1.6315e-05\\
8109.6	1.6623e-05\\
8038.2	1.6936e-05\\
7967.5	1.7255e-05\\
7897.4	1.758e-05\\
7827.9	1.7911e-05\\
7759	1.8248e-05\\
7690.7	1.859e-05\\
7623	1.8939e-05\\
7555.9	1.9294e-05\\
7489.4	1.9656e-05\\
7423.5	2.0024e-05\\
7358.2	2.0398e-05\\
7293.5	2.078e-05\\
7229.3	2.1168e-05\\
7165.7	2.1563e-05\\
7102.6	2.1964e-05\\
7040.1	2.2373e-05\\
6978.1	2.2789e-05\\
6916.7	2.3213e-05\\
6855.9	2.3644e-05\\
6795.5	2.4082e-05\\
6735.7	2.4528e-05\\
6676.5	2.4982e-05\\
6617.7	2.5443e-05\\
6559.5	2.5913e-05\\
6501.7	2.639e-05\\
6444.5	2.6876e-05\\
6387.8	2.737e-05\\
6331.6	2.7873e-05\\
6275.9	2.8384e-05\\
6220.7	2.8904e-05\\
6165.9	2.9433e-05\\
6111.7	2.997e-05\\
6057.9	3.0517e-05\\
6004.6	3.1072e-05\\
5951.7	3.1637e-05\\
5899.3	3.2212e-05\\
5847.4	3.2796e-05\\
5796	3.3389e-05\\
5745	3.3992e-05\\
5694.4	3.4605e-05\\
5644.3	3.5229e-05\\
5594.6	3.5862e-05\\
5545.4	3.6505e-05\\
5496.6	3.7159e-05\\
5448.2	3.7823e-05\\
5400.3	3.8498e-05\\
5352.8	3.9184e-05\\
5305.7	3.9881e-05\\
5259	4.0588e-05\\
5212.7	4.1307e-05\\
5166.8	4.2037e-05\\
5121.3	4.2778e-05\\
5076.3	4.353e-05\\
5031.6	4.4295e-05\\
4987.3	4.507e-05\\
4943.4	4.5858e-05\\
4899.9	4.6658e-05\\
4856.8	4.7469e-05\\
4814.1	4.8293e-05\\
4771.7	4.9129e-05\\
4729.7	4.9977e-05\\
4688.1	5.0838e-05\\
4646.8	5.1712e-05\\
4605.9	5.2598e-05\\
4565.4	5.3497e-05\\
4525.2	5.4409e-05\\
4485.4	5.5334e-05\\
4445.9	5.6272e-05\\
4406.8	5.7223e-05\\
4368	5.8187e-05\\
4329.6	5.9165e-05\\
4291.5	6.0156e-05\\
4253.7	6.1161e-05\\
4216.3	6.2179e-05\\
4179.2	6.3212e-05\\
4142.4	6.4258e-05\\
4106	6.5317e-05\\
4069.8	6.6391e-05\\
4034	6.7479e-05\\
3998.5	6.8581e-05\\
3963.3	6.9697e-05\\
3928.5	7.0827e-05\\
3893.9	7.1971e-05\\
3859.6	7.313e-05\\
3825.7	7.4303e-05\\
3792	7.5491e-05\\
3758.6	7.6692e-05\\
3725.5	7.7909e-05\\
3692.8	7.914e-05\\
3660.3	8.0385e-05\\
3628.1	8.1645e-05\\
3596.1	8.2919e-05\\
3564.5	8.4208e-05\\
3533.1	8.5512e-05\\
3502	8.683e-05\\
3471.2	8.8163e-05\\
3440.7	8.951e-05\\
3410.4	9.0872e-05\\
3380.4	9.2248e-05\\
3350.6	9.3639e-05\\
3321.1	9.5045e-05\\
3291.9	9.6465e-05\\
3262.9	9.7899e-05\\
3234.2	9.9348e-05\\
3205.8	0.00010081\\
3177.6	0.00010229\\
3149.6	0.00010378\\
3121.9	0.00010529\\
3094.4	0.00010681\\
3067.2	0.00010834\\
3040.2	0.00010989\\
3013.4	0.00011145\\
2986.9	0.00011303\\
2960.6	0.00011462\\
2934.6	0.00011622\\
2908.7	0.00011783\\
2883.1	0.00011946\\
2857.8	0.00012111\\
2832.6	0.00012276\\
2807.7	0.00012443\\
2783	0.00012612\\
2758.5	0.00012781\\
2734.2	0.00012952\\
2710.2	0.00013124\\
2686.3	0.00013297\\
2662.7	0.00013471\\
2639.2	0.00013647\\
2616	0.00013824\\
2593	0.00014002\\
2570.2	0.00014181\\
2547.6	0.00014361\\
2525.1	0.00014542\\
2502.9	0.00014725\\
2480.9	0.00014908\\
2459.1	0.00015093\\
2437.4	0.00015278\\
2416	0.00015465\\
2394.7	0.00015652\\
2373.6	0.0001584\\
2352.7	0.0001603\\
2332	0.0001622\\
2311.5	0.00016411\\
2291.2	0.00016603\\
2271	0.00016795\\
2251	0.00016989\\
2231.2	0.00017183\\
2211.6	0.00017377\\
2192.1	0.00017573\\
2172.8	0.00017769\\
2153.7	0.00017966\\
2134.8	0.00018163\\
2116	0.00018361\\
2097.4	0.00018559\\
2078.9	0.00018758\\
2060.6	0.00018957\\
2042.5	0.00019157\\
2024.5	0.00019357\\
2006.7	0.00019557\\
1989	0.00019757\\
1971.5	0.00019958\\
1954.2	0.00020159\\
1937	0.0002036\\
1919.9	0.00020561\\
1903	0.00020762\\
1886.3	0.00020964\\
1869.7	0.00021165\\
1853.2	0.00021366\\
1836.9	0.00021567\\
1820.8	0.00021768\\
1804.7	0.00021969\\
1788.9	0.00022169\\
1773.1	0.00022369\\
1757.5	0.00022569\\
1742	0.00022768\\
1726.7	0.00022967\\
1711.5	0.00023165\\
1696.5	0.00023363\\
1681.5	0.0002356\\
1666.7	0.00023757\\
1652.1	0.00023953\\
1637.5	0.00024148\\
1623.1	0.00024342\\
1608.8	0.00024536\\
1594.7	0.00024728\\
1580.6	0.0002492\\
1566.7	0.00025111\\
1552.9	0.000253\\
1539.3	0.00025489\\
1525.7	0.00025676\\
1512.3	0.00025862\\
1499	0.00026047\\
1485.8	0.00026231\\
1472.7	0.00026413\\
1459.8	0.00026594\\
1446.9	0.00026773\\
1434.2	0.00026951\\
1421.6	0.00027128\\
1409.1	0.00027302\\
1396.7	0.00027476\\
1384.4	0.00027647\\
1372.2	0.00027817\\
1360.1	0.00027985\\
1348.1	0.00028151\\
1336.3	0.00028315\\
1324.5	0.00028478\\
1312.9	0.00028638\\
1301.3	0.00028797\\
1289.9	0.00028953\\
1278.5	0.00029108\\
1267.3	0.0002926\\
1256.1	0.0002941\\
1245.1	0.00029558\\
1234.1	0.00029704\\
1223.2	0.00029848\\
1212.5	0.00029989\\
1201.8	0.00030128\\
1191.2	0.00030264\\
1180.7	0.00030399\\
1170.4	0.0003053\\
1160.1	0.00030659\\
1149.8	0.00030786\\
1139.7	0.0003091\\
1129.7	0.00031032\\
1119.8	0.00031151\\
1109.9	0.00031267\\
1100.1	0.00031381\\
1090.5	0.00031492\\
1080.9	0.000316\\
1071.3	0.00031706\\
1061.9	0.00031808\\
1052.6	0.00031908\\
1043.3	0.00032005\\
1034.1	0.00032099\\
1025	0.00032191\\
1016	0.00032279\\
1007.1	0.00032365\\
998.2	0.00032447\\
989.42	0.00032527\\
980.71	0.00032604\\
972.08	0.00032677\\
963.53	0.00032748\\
955.05	0.00032815\\
946.64	0.0003288\\
938.31	0.00032941\\
930.06	0.00033\\
921.87	0.00033055\\
913.76	0.00033107\\
905.72	0.00033156\\
897.75	0.00033202\\
889.85	0.00033245\\
882.02	0.00033284\\
874.25	0.00033321\\
866.56	0.00033354\\
858.94	0.00033384\\
851.38	0.00033411\\
843.88	0.00033434\\
836.46	0.00033455\\
829.1	0.00033472\\
821.8	0.00033486\\
814.57	0.00033497\\
807.4	0.00033505\\
800.3	0.00033509\\
793.25	0.00033511\\
786.27	0.00033509\\
779.35	0.00033504\\
772.5	0.00033495\\
765.7	0.00033484\\
758.96	0.00033469\\
752.28	0.00033452\\
745.66	0.00033431\\
739.1	0.00033407\\
732.59	0.0003338\\
726.15	0.00033349\\
719.76	0.00033316\\
713.42	0.0003328\\
707.14	0.0003324\\
700.92	0.00033198\\
694.75	0.00033152\\
688.64	0.00033103\\
682.58	0.00033052\\
676.57	0.00032997\\
670.62	0.0003294\\
664.72	0.00032879\\
658.87	0.00032816\\
653.07	0.0003275\\
647.32	0.00032681\\
641.63	0.00032609\\
635.98	0.00032535\\
630.38	0.00032457\\
624.84	0.00032377\\
619.34	0.00032294\\
613.89	0.00032209\\
608.49	0.00032121\\
603.13	0.0003203\\
597.82	0.00031936\\
592.56	0.0003184\\
587.35	0.00031742\\
582.18	0.00031641\\
577.06	0.00031538\\
571.98	0.00031432\\
566.94	0.00031324\\
561.95	0.00031213\\
557.01	0.000311\\
552.11	0.00030985\\
547.25	0.00030868\\
542.43	0.00030749\\
537.66	0.00030627\\
532.93	0.00030503\\
528.24	0.00030377\\
523.59	0.00030249\\
518.98	0.00030119\\
514.42	0.00029987\\
509.89	0.00029853\\
505.4	0.00029718\\
500.95	0.0002958\\
496.55	0.00029441\\
492.18	0.000293\\
487.84	0.00029157\\
483.55	0.00029012\\
479.3	0.00028866\\
475.08	0.00028718\\
470.9	0.00028569\\
466.75	0.00028418\\
462.65	0.00028265\\
458.58	0.00028112\\
454.54	0.00027956\\
450.54	0.000278\\
446.58	0.00027642\\
442.65	0.00027482\\
438.75	0.00027322\\
434.89	0.0002716\\
431.06	0.00026997\\
427.27	0.00026833\\
423.51	0.00026668\\
419.78	0.00026502\\
416.09	0.00026335\\
412.43	0.00026167\\
408.8	0.00025998\\
405.2	0.00025828\\
401.63	0.00025657\\
398.1	0.00025485\\
394.6	0.00025313\\
391.12	0.0002514\\
387.68	0.00024966\\
384.27	0.00024791\\
380.89	0.00024616\\
377.54	0.0002444\\
374.21	0.00024264\\
370.92	0.00024087\\
367.66	0.0002391\\
364.42	0.00023732\\
361.21	0.00023554\\
358.04	0.00023375\\
354.88	0.00023196\\
351.76	0.00023017\\
348.67	0.00022837\\
345.6	0.00022657\\
342.56	0.00022477\\
339.54	0.00022297\\
336.55	0.00022117\\
333.59	0.00021936\\
330.66	0.00021755\\
327.75	0.00021575\\
324.86	0.00021394\\
322	0.00021213\\
319.17	0.00021032\\
316.36	0.00020852\\
313.58	0.00020671\\
310.82	0.00020491\\
308.08	0.0002031\\
305.37	0.0002013\\
302.68	0.0001995\\
300.02	0.0001977\\
297.38	0.00019591\\
294.76	0.00019412\\
292.17	0.00019233\\
289.6	0.00019054\\
287.05	0.00018876\\
284.52	0.00018698\\
282.02	0.00018521\\
279.54	0.00018344\\
277.08	0.00018167\\
274.64	0.00017991\\
272.22	0.00017815\\
269.83	0.0001764\\
267.45	0.00017466\\
265.1	0.00017292\\
262.77	0.00017118\\
260.45	0.00016946\\
258.16	0.00016773\\
255.89	0.00016602\\
253.64	0.00016431\\
251.41	0.00016261\\
249.19	0.00016092\\
247	0.00015923\\
244.83	0.00015755\\
242.67	0.00015588\\
240.54	0.00015421\\
238.42	0.00015256\\
236.32	0.00015091\\
234.24	0.00014927\\
232.18	0.00014764\\
230.14	0.00014601\\
228.11	0.0001444\\
226.11	0.00014279\\
224.12	0.0001412\\
222.14	0.00013961\\
220.19	0.00013803\\
218.25	0.00013647\\
216.33	0.00013491\\
214.43	0.00013336\\
212.54	0.00013182\\
210.67	0.00013029\\
208.82	0.00012877\\
206.98	0.00012726\\
205.16	0.00012576\\
203.35	0.00012428\\
201.56	0.0001228\\
199.79	0.00012133\\
198.03	0.00011987\\
196.29	0.00011843\\
194.56	0.00011699\\
192.85	0.00011557\\
191.15	0.00011416\\
189.47	0.00011275\\
187.8	0.00011136\\
186.15	0.00010998\\
184.51	0.00010861\\
182.89	0.00010726\\
181.28	0.00010591\\
179.68	0.00010458\\
178.1	0.00010325\\
176.53	0.00010194\\
174.98	0.00010064\\
173.44	9.935e-05\\
171.91	9.8072e-05\\
170.4	9.6806e-05\\
168.9	9.5551e-05\\
167.42	9.4308e-05\\
165.94	9.3076e-05\\
164.48	9.1856e-05\\
163.03	9.0648e-05\\
161.6	8.9451e-05\\
160.18	8.8266e-05\\
158.77	8.7092e-05\\
157.37	8.593e-05\\
155.99	8.478e-05\\
154.61	8.3641e-05\\
153.25	8.2514e-05\\
151.9	8.1398e-05\\
150.57	8.0294e-05\\
149.24	7.9201e-05\\
147.93	7.812e-05\\
146.63	7.705e-05\\
145.34	7.5992e-05\\
144.06	7.4945e-05\\
142.79	7.3909e-05\\
141.53	7.2885e-05\\
140.29	7.1872e-05\\
139.05	7.087e-05\\
137.83	6.988e-05\\
136.62	6.89e-05\\
135.41	6.7932e-05\\
134.22	6.6975e-05\\
133.04	6.6029e-05\\
131.87	6.5093e-05\\
130.71	6.4169e-05\\
129.56	6.3255e-05\\
128.42	6.2353e-05\\
127.29	6.1461e-05\\
126.17	6.0579e-05\\
125.06	5.9709e-05\\
123.96	5.8848e-05\\
122.87	5.7998e-05\\
121.79	5.7159e-05\\
120.72	5.633e-05\\
119.65	5.5511e-05\\
118.6	5.4703e-05\\
117.56	5.3904e-05\\
116.52	5.3116e-05\\
115.5	5.2337e-05\\
114.48	5.1569e-05\\
113.47	5.081e-05\\
112.47	5.0061e-05\\
111.48	4.9322e-05\\
110.5	4.8592e-05\\
109.53	4.7872e-05\\
108.57	4.7161e-05\\
107.61	4.646e-05\\
106.66	4.5768e-05\\
105.73	4.5085e-05\\
104.8	4.4411e-05\\
103.87	4.3746e-05\\
102.96	4.309e-05\\
102.05	4.2443e-05\\
101.16	4.1805e-05\\
100.26	4.1176e-05\\
99.383	4.0555e-05\\
98.508	3.9942e-05\\
97.641	3.9339e-05\\
96.782	3.8743e-05\\
95.93	3.8156e-05\\
95.086	3.7577e-05\\
94.249	3.7005e-05\\
93.42	3.6442e-05\\
92.598	3.5887e-05\\
91.783	3.534e-05\\
90.975	3.4801e-05\\
90.175	3.4269e-05\\
89.381	3.3745e-05\\
88.595	3.3228e-05\\
87.815	3.2718e-05\\
87.042	3.2216e-05\\
86.276	3.1722e-05\\
85.517	3.1234e-05\\
84.764	3.0754e-05\\
84.018	3.028e-05\\
83.279	2.9814e-05\\
82.546	2.9354e-05\\
81.82	2.8901e-05\\
81.1	2.8454e-05\\
80.386	2.8015e-05\\
79.679	2.7581e-05\\
78.977	2.7154e-05\\
78.282	2.6734e-05\\
77.594	2.632e-05\\
76.911	2.5912e-05\\
76.234	2.551e-05\\
75.563	2.5114e-05\\
74.898	2.4724e-05\\
74.239	2.4339e-05\\
73.586	2.3961e-05\\
72.938	2.3588e-05\\
72.296	2.3221e-05\\
71.66	2.286e-05\\
71.029	2.2504e-05\\
70.404	2.2153e-05\\
69.785	2.1808e-05\\
69.171	2.1468e-05\\
68.562	2.1133e-05\\
67.959	2.0804e-05\\
67.361	2.0479e-05\\
66.768	2.0159e-05\\
66.18	1.9845e-05\\
65.598	1.9535e-05\\
65.021	1.923e-05\\
64.448	1.8929e-05\\
63.881	1.8633e-05\\
63.319	1.8342e-05\\
62.762	1.8055e-05\\
62.21	1.7773e-05\\
61.662	1.7495e-05\\
61.12	1.7222e-05\\
60.582	1.6952e-05\\
60.049	1.6687e-05\\
59.52	1.6426e-05\\
58.996	1.6169e-05\\
58.477	1.5916e-05\\
57.963	1.5667e-05\\
57.453	1.5422e-05\\
56.947	1.518e-05\\
56.446	1.4943e-05\\
55.949	1.4709e-05\\
55.457	1.4478e-05\\
54.969	1.4252e-05\\
54.485	1.4029e-05\\
54.005	1.3809e-05\\
53.53	1.3593e-05\\
53.059	1.338e-05\\
52.592	1.317e-05\\
52.129	1.2964e-05\\
51.671	1.2761e-05\\
51.216	1.2561e-05\\
50.765	1.2365e-05\\
50.319	1.2171e-05\\
49.876	1.198e-05\\
49.437	1.1793e-05\\
49.002	1.1608e-05\\
48.571	1.1426e-05\\
48.143	1.1247e-05\\
47.719	1.1071e-05\\
47.3	1.0898e-05\\
46.883	1.0727e-05\\
46.471	1.0559e-05\\
46.062	1.0394e-05\\
45.656	1.0231e-05\\
45.255	1.0071e-05\\
44.856	9.9132e-06\\
44.462	9.7579e-06\\
44.07	9.6051e-06\\
43.683	9.4547e-06\\
43.298	9.3066e-06\\
42.917	9.1608e-06\\
42.539	9.0173e-06\\
42.165	8.876e-06\\
41.794	8.7369e-06\\
41.426	8.6e-06\\
41.062	8.4652e-06\\
40.7	8.3325e-06\\
40.342	8.2019e-06\\
39.987	8.0734e-06\\
39.635	7.9468e-06\\
39.287	7.8222e-06\\
38.941	7.6995e-06\\
38.598	7.5787e-06\\
38.258	7.4598e-06\\
37.922	7.3428e-06\\
37.588	7.2275e-06\\
37.257	7.1141e-06\\
36.929	7.0024e-06\\
36.604	6.8924e-06\\
36.282	6.7841e-06\\
35.963	6.6775e-06\\
35.647	6.5726e-06\\
35.333	6.4693e-06\\
35.022	6.3676e-06\\
34.714	6.2674e-06\\
34.408	6.1688e-06\\
34.105	6.0717e-06\\
33.805	5.9761e-06\\
33.508	5.882e-06\\
33.213	5.7893e-06\\
32.921	5.698e-06\\
32.631	5.6082e-06\\
32.344	5.5197e-06\\
32.059	5.4326e-06\\
31.777	5.3468e-06\\
31.497	5.2624e-06\\
31.22	5.1792e-06\\
30.946	5.0973e-06\\
30.673	5.0167e-06\\
30.403	4.9373e-06\\
30.136	4.8591e-06\\
29.871	4.7821e-06\\
}--cycle;
\addplot[ybar interval, fill=black, fill opacity=0.35, area legend, draw=none] table[row sep=crcr, x=Lower, y=Count] {%
Lower	Upper	Count\\
29.871	42.524	0\\
42.524	60.539	0\\
60.539	86.185	0\\
86.185	122.69	7.3039e-05\\
122.69	174.67	5.1305e-05\\
174.67	248.67	0.00025227\\
248.67	354.01	0.00027846\\
354.01	503.97	0.00017782\\
503.97	717.47	0.00039969\\
717.47	1021.4	0.00032463\\
1021.4	1454.1	0.00033896\\
1454.1	2070.1	0.00024676\\
2070.1	2947	0.0001338\\
2947	4195.5	9.3984e-05\\
4195.5	5972.8	3.751e-05\\
5972.8	8503	1.8971e-05\\
8503	12105	9.624e-06\\
12105	17233	4.1601e-06\\
17233	24534	1.0958e-06\\
24534	34927	7.6975e-07\\
34927	49722	5.4069e-07\\
49722	70786	1.266e-07\\
70786	1.0077e+05	0\\
1.0077e+05	1.4346e+05	6.2466e-08\\
1.4346e+05	2.0424e+05	4.3878e-08\\
2.0424e+05	2.0424e+05	4.3878e-08\\
};
\addlegendentry{Istogramma reale}

\addplot [color=black]
  table[row sep=crcr]{%
29.871	1.2966e-06\\
34.105	2.0618e-06\\
38.258	3.0331e-06\\
42.165	4.1559e-06\\
46.062	5.4827e-06\\
49.876	6.9826e-06\\
53.53	8.6051e-06\\
57.453	1.0545e-05\\
61.12	1.2539e-05\\
65.021	1.4847e-05\\
68.562	1.7101e-05\\
72.296	1.9634e-05\\
76.234	2.2471e-05\\
80.386	2.5636e-05\\
84.764	2.9153e-05\\
88.595	3.2372e-05\\
92.598	3.5867e-05\\
96.782	3.965e-05\\
101.16	4.3737e-05\\
105.73	4.8137e-05\\
110.5	5.2864e-05\\
115.5	5.7926e-05\\
120.72	6.3332e-05\\
126.17	6.909e-05\\
131.87	7.5205e-05\\
137.83	8.168e-05\\
144.06	8.8516e-05\\
150.57	9.5713e-05\\
157.37	0.00010327\\
164.48	0.00011117\\
171.91	0.00011941\\
179.68	0.00012798\\
187.8	0.00013686\\
196.29	0.00014603\\
205.16	0.00015548\\
216.33	0.00016713\\
228.11	0.00017909\\
242.67	0.00019334\\
262.77	0.00021199\\
310.82	0.00025149\\
330.66	0.00026568\\
348.67	0.00027752\\
364.42	0.00028709\\
380.89	0.00029633\\
394.6	0.00030345\\
408.8	0.00031031\\
423.51	0.00031687\\
438.75	0.00032312\\
454.54	0.00032902\\
470.9	0.00033455\\
487.84	0.0003397\\
500.95	0.0003433\\
514.42	0.00034665\\
528.24	0.00034977\\
542.43	0.00035262\\
557.01	0.00035522\\
571.98	0.00035756\\
587.35	0.00035962\\
603.13	0.00036141\\
619.34	0.00036292\\
635.98	0.00036415\\
653.07	0.00036509\\
670.62	0.00036574\\
688.64	0.00036611\\
707.14	0.00036618\\
726.15	0.00036596\\
745.66	0.00036546\\
765.7	0.00036466\\
786.27	0.00036358\\
807.4	0.00036221\\
829.1	0.00036057\\
851.38	0.00035865\\
874.25	0.00035645\\
897.75	0.00035399\\
921.87	0.00035127\\
946.64	0.00034829\\
972.08	0.00034507\\
998.2	0.0003416\\
1034.1	0.00033663\\
1071.3	0.00033125\\
1109.9	0.00032551\\
1149.8	0.00031942\\
1191.2	0.00031301\\
1234.1	0.0003063\\
1278.5	0.00029931\\
1336.3	0.00029023\\
1396.7	0.00028081\\
1459.8	0.00027111\\
1539.3	0.00025917\\
1637.5	0.00024493\\
1820.8	0.00022015\\
2024.5	0.00019544\\
2153.7	0.00018129\\
2271	0.00016943\\
2394.7	0.00015786\\
2525.1	0.00014664\\
2639.2	0.00013758\\
2758.5	0.00012882\\
2883.1	0.00012036\\
3013.4	0.00011223\\
3149.6	0.00010444\\
3291.9	9.6998e-05\\
3440.7	8.9903e-05\\
3596.1	8.3164e-05\\
3758.6	7.678e-05\\
3928.5	7.0751e-05\\
4106	6.5071e-05\\
4291.5	5.9736e-05\\
4485.4	5.4738e-05\\
4688.1	5.0069e-05\\
4899.9	4.5718e-05\\
5121.3	4.1673e-05\\
5352.8	3.7923e-05\\
5594.6	3.4456e-05\\
5847.4	3.1256e-05\\
6165.9	2.7753e-05\\
6501.7	2.4592e-05\\
6855.9	2.1749e-05\\
7229.3	1.9202e-05\\
7690.7	1.6571e-05\\
8181.6	1.4273e-05\\
8703.8	1.2274e-05\\
9341.5	1.0313e-05\\
10026	8.6558e-06\\
10856	7.102e-06\\
11859	5.7006e-06\\
13070	4.4823e-06\\
14533	3.4596e-06\\
16447	2.5743e-06\\
19114	1.8202e-06\\
23013	1.2112e-06\\
29475	7.2692e-07\\
41976	3.63e-07\\
73906	1.1579e-07\\
2.0424e+05	1.0036e-08\\
};
\addlegendentry{Adatt. BLN reale ($\mathit{ML}$)}

            % \input{../../../.tex/20/KS/SizeDistributionFittingsfig2.tex}
    
            \nextgroupplot[%
                % xmin=10,xmax=1e+06,
                % ymin=0,ymax=0.0004,
            ] % This file was created by matlab2tikz.
%
\definecolor{mycolor1}{rgb}{0.83529,0.36863,0.00000}%
%
\addplot[ybar interval, fill=mycolor1, fill opacity=0.35, area legend, draw=none] table[row sep=crcr, x=Lower, y=Count] {%
Lower	Upper	Count\\
27.667	40.439	6.2638e-06\\
40.439	59.107	8.571e-06\\
59.107	86.392	2.7366e-05\\
86.392	126.27	5.7507e-05\\
126.27	184.56	9.5623e-05\\
184.56	269.76	0.00015525\\
269.76	394.29	0.00023942\\
394.29	576.29	0.00028922\\
576.29	842.32	0.00032027\\
842.32	1231.2	0.00031651\\
1231.2	1799.5	0.00026051\\
1799.5	2630.2	0.00017936\\
2630.2	3844.3	0.00010532\\
3844.3	5618.9	5.243e-05\\
5618.9	8212.7	2.3852e-05\\
8212.7	12004	1.0931e-05\\
12004	17545	4.7114e-06\\
17545	25644	2.262e-06\\
25644	37482	8.8534e-07\\
37482	54785	2.8205e-07\\
54785	80074	1.065e-07\\
80074	1.1704e+05	2.6693e-08\\
1.1704e+05	1.7107e+05	3.4058e-08\\
1.7107e+05	2.5003e+05	7.7669e-09\\
2.5003e+05	2.5003e+05	7.7669e-09\\
};
\addlegendentry{Istogramma appr.}

\addplot [color=mycolor1, only marks, every error bar/.append style={opacity=0.45}, mark=*, mark size=0pt, draw=none, forget plot]
 plot [error bars/.cd, y dir=both, y explicit, error bar style={line width=1pt, color=mycolor1}, error mark options={mark=none,mark size=0pt}]
 table[row sep=crcr, y error plus index=2, y error minus index=3]{%
33.449	6.2638e-06	7.1029e-06	6.2638e-06\\
48.89	8.571e-06	6.7654e-06	6.7654e-06\\
71.459	2.7366e-05	9.1749e-06	9.1749e-06\\
104.45	5.7507e-05	1.1318e-05	1.1318e-05\\
152.66	9.5623e-05	1.3543e-05	1.3543e-05\\
223.13	0.00015525	1.3268e-05	1.3268e-05\\
326.13	0.00023942	1.2325e-05	1.2325e-05\\
476.68	0.00028922	1.1596e-05	1.1596e-05\\
696.73	0.00032027	1.0521e-05	1.0521e-05\\
1018.3	0.00031651	7.8553e-06	7.8553e-06\\
1488.4	0.00026051	5.6374e-06	5.6374e-06\\
2175.5	0.00017936	4.9436e-06	4.9436e-06\\
3179.8	0.00010532	2.6398e-06	2.6398e-06\\
4647.7	5.243e-05	1.4394e-06	1.4394e-06\\
6793.1	2.3852e-05	7.583e-07	7.583e-07\\
9929	1.0931e-05	4.8087e-07	4.8087e-07\\
14512	4.7114e-06	2.5254e-07	2.5254e-07\\
21212	2.262e-06	1.4991e-07	1.4991e-07\\
31003	8.8534e-07	8.0187e-08	8.0187e-08\\
45315	2.8205e-07	3.7656e-08	3.7656e-08\\
66233	1.065e-07	1.4104e-08	1.4104e-08\\
96808	2.6693e-08	8.0502e-09	8.0502e-09\\
1.415e+05	3.4058e-08	4.5525e-09	4.5525e-09\\
2.0681e+05	7.7669e-09	2.834e-09	2.834e-09\\
};
\addplot [color=mycolor1]
  table[row sep=crcr]{%
27.667	6.8918e-06\\
31.435	8.5847e-06\\
35.069	1.0325e-05\\
39.124	1.238e-05\\
43.252	1.4583e-05\\
47.381	1.6891e-05\\
51.434	1.925e-05\\
55.833	2.1909e-05\\
60.058	2.4551e-05\\
64.602	2.7485e-05\\
69.491	3.0738e-05\\
74.071	3.387e-05\\
78.953	3.7292e-05\\
84.156	4.1026e-05\\
89.703	4.5095e-05\\
94.747	4.8868e-05\\
100.08	5.2918e-05\\
105.7	5.7262e-05\\
111.65	6.1914e-05\\
117.93	6.6886e-05\\
124.56	7.2193e-05\\
131.56	7.7846e-05\\
138.96	8.3855e-05\\
145.44	8.9141e-05\\
152.23	9.4683e-05\\
159.33	0.00010048\\
166.76	0.00010655\\
174.54	0.00011287\\
182.68	0.00011944\\
192.95	0.00012767\\
203.8	0.00013625\\
215.26	0.00014516\\
227.37	0.00015439\\
240.15	0.0001639\\
255.98	0.00017532\\
275.35	0.00018872\\
301.64	0.00020584\\
372.02	0.00024537\\
396.54	0.00025702\\
418.84	0.0002667\\
438.37	0.00027449\\
458.82	0.00028199\\
480.23	0.00028915\\
498.06	0.00029459\\
516.56	0.00029976\\
535.75	0.00030463\\
555.65	0.00030918\\
576.29	0.00031339\\
597.7	0.00031723\\
619.9	0.00032069\\
637.1	0.00032301\\
654.76	0.0003251\\
672.92	0.00032695\\
691.58	0.00032855\\
710.76	0.00032989\\
730.47	0.00033097\\
750.73	0.00033178\\
771.55	0.00033232\\
792.95	0.00033259\\
814.94	0.00033258\\
837.54	0.00033229\\
860.76	0.00033172\\
884.64	0.00033087\\
909.17	0.00032974\\
934.38	0.00032833\\
960.29	0.00032665\\
986.92	0.0003247\\
1014.3	0.00032249\\
1042.4	0.00032\\
1071.3	0.00031727\\
1101	0.00031427\\
1131.6	0.00031104\\
1163	0.00030755\\
1195.2	0.00030384\\
1239.6	0.00029853\\
1285.7	0.00029282\\
1333.4	0.00028673\\
1382.9	0.00028028\\
1434.3	0.00027349\\
1487.6	0.00026637\\
1542.8	0.00025896\\
1600.2	0.00025128\\
1674.8	0.00024134\\
1752.9	0.0002311\\
1868.5	0.00021642\\
2283.5	0.00016992\\
2411.9	0.00015772\\
2524.4	0.00014787\\
2642.2	0.00013833\\
2765.4	0.00012915\\
2868.2	0.00012207\\
2974.7	0.00011524\\
3085.2	0.00010867\\
3199.8	0.00010236\\
3318.7	9.6318e-05\\
3441.9	9.0536e-05\\
3569.8	8.5015e-05\\
3702.4	7.9754e-05\\
3839.9	7.4748e-05\\
3982.6	6.9993e-05\\
4168.3	6.4394e-05\\
4362.8	5.9164e-05\\
4566.3	5.4292e-05\\
4779.3	4.9761e-05\\
5002.2	4.5557e-05\\
5235.5	4.1664e-05\\
5479.8	3.8065e-05\\
5735.4	3.4744e-05\\
6002.9	3.1685e-05\\
6340.5	2.8337e-05\\
6697	2.5315e-05\\
7073.6	2.2593e-05\\
7471.4	2.0147e-05\\
7963.8	1.7608e-05\\
8488.7	1.5374e-05\\
9131.1	1.3154e-05\\
9822	1.1245e-05\\
10662	9.4174e-06\\
11680	7.7264e-06\\
12912	6.2088e-06\\
14405	4.8859e-06\\
16218	3.7636e-06\\
18595	2.779e-06\\
21912	1.9232e-06\\
26536	1.2419e-06\\
33636	7.1121e-07\\
46281	3.2366e-07\\
73681	9.3644e-08\\
1.8506e+05	5.6253e-09\\
2.5003e+05	2.0329e-09\\
};
\addlegendentry{Adatt. BLN appr. ($\mathit{ML}$)}


\addplot[area legend, draw=none, fill=mycolor1, fill opacity=0.15, forget plot]
table[row sep=crcr] {%
x	y\\
27.667	8.2361e-06\\
27.921	8.3634e-06\\
28.177	8.4924e-06\\
28.435	8.6231e-06\\
28.695	8.7557e-06\\
28.958	8.89e-06\\
29.223	9.0261e-06\\
29.491	9.1641e-06\\
29.761	9.304e-06\\
30.034	9.4457e-06\\
30.309	9.5893e-06\\
30.586	9.7349e-06\\
30.867	9.8824e-06\\
31.149	1.0032e-05\\
31.435	1.0183e-05\\
31.723	1.0337e-05\\
32.013	1.0492e-05\\
32.306	1.065e-05\\
32.602	1.081e-05\\
32.901	1.0972e-05\\
33.202	1.1136e-05\\
33.506	1.1302e-05\\
33.813	1.147e-05\\
34.123	1.1641e-05\\
34.436	1.1813e-05\\
34.751	1.1988e-05\\
35.069	1.2166e-05\\
35.391	1.2345e-05\\
35.715	1.2527e-05\\
36.042	1.2712e-05\\
36.372	1.2899e-05\\
36.705	1.3088e-05\\
37.042	1.328e-05\\
37.381	1.3474e-05\\
37.723	1.3671e-05\\
38.069	1.387e-05\\
38.417	1.4072e-05\\
38.769	1.4276e-05\\
39.124	1.4483e-05\\
39.483	1.4693e-05\\
39.845	1.4905e-05\\
40.209	1.5121e-05\\
40.578	1.5338e-05\\
40.949	1.5559e-05\\
41.325	1.5783e-05\\
41.703	1.6009e-05\\
42.085	1.6238e-05\\
42.471	1.6471e-05\\
42.86	1.6706e-05\\
43.252	1.6944e-05\\
43.648	1.7185e-05\\
44.048	1.7429e-05\\
44.452	1.7677e-05\\
44.859	1.7927e-05\\
45.27	1.8181e-05\\
45.684	1.8438e-05\\
46.103	1.8698e-05\\
46.525	1.8961e-05\\
46.951	1.9228e-05\\
47.381	1.9498e-05\\
47.815	1.9771e-05\\
48.253	2.0048e-05\\
48.695	2.0329e-05\\
49.141	2.0612e-05\\
49.592	2.09e-05\\
50.046	2.1191e-05\\
50.504	2.1485e-05\\
50.967	2.1784e-05\\
51.434	2.2086e-05\\
51.905	2.2391e-05\\
52.38	2.2701e-05\\
52.86	2.3014e-05\\
53.344	2.3331e-05\\
53.833	2.3653e-05\\
54.326	2.3978e-05\\
54.824	2.4307e-05\\
55.326	2.464e-05\\
55.833	2.4977e-05\\
56.344	2.5319e-05\\
56.86	2.5665e-05\\
57.381	2.6014e-05\\
57.907	2.6369e-05\\
58.437	2.6727e-05\\
58.972	2.709e-05\\
59.512	2.7458e-05\\
60.058	2.7829e-05\\
60.608	2.8206e-05\\
61.163	2.8587e-05\\
61.723	2.8972e-05\\
62.289	2.9363e-05\\
62.859	2.9758e-05\\
63.435	3.0158e-05\\
64.016	3.0562e-05\\
64.602	3.0972e-05\\
65.194	3.1386e-05\\
65.791	3.1806e-05\\
66.394	3.2231e-05\\
67.002	3.266e-05\\
67.616	3.3095e-05\\
68.235	3.3535e-05\\
68.86	3.3981e-05\\
69.491	3.4431e-05\\
70.127	3.4888e-05\\
70.77	3.5349e-05\\
71.418	3.5816e-05\\
72.072	3.6289e-05\\
72.732	3.6767e-05\\
73.399	3.7251e-05\\
74.071	3.7741e-05\\
74.749	3.8236e-05\\
75.434	3.8737e-05\\
76.125	3.9245e-05\\
76.822	3.9758e-05\\
77.526	4.0277e-05\\
78.236	4.0802e-05\\
78.953	4.1334e-05\\
79.676	4.1872e-05\\
80.406	4.2416e-05\\
81.142	4.2966e-05\\
81.886	4.3523e-05\\
82.636	4.4086e-05\\
83.393	4.4656e-05\\
84.156	4.5232e-05\\
84.927	4.5816e-05\\
85.705	4.6405e-05\\
86.49	4.7002e-05\\
87.283	4.7605e-05\\
88.082	4.8216e-05\\
88.889	4.8833e-05\\
89.703	4.9458e-05\\
90.525	5.0089e-05\\
91.354	5.0728e-05\\
92.191	5.1374e-05\\
93.035	5.2027e-05\\
93.887	5.2688e-05\\
94.747	5.3356e-05\\
95.615	5.4032e-05\\
96.491	5.4715e-05\\
97.375	5.5406e-05\\
98.267	5.6104e-05\\
99.167	5.6811e-05\\
100.08	5.7525e-05\\
100.99	5.8247e-05\\
101.92	5.8977e-05\\
102.85	5.9715e-05\\
103.79	6.0461e-05\\
104.74	6.1216e-05\\
105.7	6.1978e-05\\
106.67	6.2749e-05\\
107.65	6.3528e-05\\
108.63	6.4316e-05\\
109.63	6.5112e-05\\
110.63	6.5917e-05\\
111.65	6.673e-05\\
112.67	6.7552e-05\\
113.7	6.8383e-05\\
114.74	6.9222e-05\\
115.79	7.007e-05\\
116.85	7.0928e-05\\
117.93	7.1794e-05\\
119.01	7.2669e-05\\
120.1	7.3553e-05\\
121.2	7.4447e-05\\
122.31	7.5349e-05\\
123.43	7.6261e-05\\
124.56	7.7182e-05\\
125.7	7.8113e-05\\
126.85	7.9053e-05\\
128.01	8.0002e-05\\
129.18	8.0961e-05\\
130.37	8.1929e-05\\
131.56	8.2907e-05\\
132.77	8.3895e-05\\
133.98	8.4892e-05\\
135.21	8.5899e-05\\
136.45	8.6916e-05\\
137.7	8.7943e-05\\
138.96	8.898e-05\\
140.23	9.0026e-05\\
141.52	9.1082e-05\\
142.81	9.2149e-05\\
144.12	9.3225e-05\\
145.44	9.4312e-05\\
146.77	9.5408e-05\\
148.12	9.6515e-05\\
149.47	9.7631e-05\\
150.84	9.8758e-05\\
152.23	9.9895e-05\\
153.62	0.00010104\\
155.03	0.0001022\\
156.45	0.00010337\\
157.88	0.00010455\\
159.33	0.00010573\\
160.79	0.00010693\\
162.26	0.00010814\\
163.74	0.00010936\\
165.24	0.00011059\\
166.76	0.00011183\\
168.29	0.00011308\\
169.83	0.00011434\\
171.38	0.00011561\\
172.95	0.00011689\\
174.54	0.00011818\\
176.14	0.00011948\\
177.75	0.00012079\\
179.38	0.00012211\\
181.02	0.00012344\\
182.68	0.00012478\\
184.35	0.00012614\\
186.04	0.0001275\\
187.74	0.00012887\\
189.46	0.00013025\\
191.2	0.00013164\\
192.95	0.00013304\\
194.72	0.00013445\\
196.5	0.00013588\\
198.3	0.00013731\\
200.12	0.00013875\\
201.95	0.0001402\\
203.8	0.00014165\\
205.67	0.00014312\\
207.55	0.0001446\\
209.45	0.00014609\\
211.37	0.00014758\\
213.31	0.00014909\\
215.26	0.0001506\\
217.23	0.00015213\\
219.22	0.00015366\\
221.23	0.0001552\\
223.26	0.00015675\\
225.3	0.00015831\\
227.37	0.00015987\\
229.45	0.00016144\\
231.55	0.00016303\\
233.67	0.00016462\\
235.81	0.00016621\\
237.97	0.00016782\\
240.15	0.00016943\\
242.35	0.00017105\\
244.57	0.00017268\\
246.81	0.00017431\\
249.07	0.00017595\\
251.35	0.0001776\\
253.66	0.00017925\\
255.98	0.00018091\\
258.32	0.00018258\\
260.69	0.00018425\\
263.08	0.00018593\\
265.49	0.00018761\\
267.92	0.0001893\\
270.37	0.00019099\\
272.85	0.00019269\\
275.35	0.00019439\\
277.87	0.0001961\\
280.42	0.00019781\\
282.99	0.00019952\\
285.58	0.00020124\\
288.19	0.00020296\\
290.83	0.00020469\\
293.5	0.00020641\\
296.19	0.00020814\\
298.9	0.00020988\\
301.64	0.00021161\\
304.4	0.00021335\\
307.19	0.00021509\\
310	0.00021683\\
312.84	0.00021857\\
315.71	0.00022031\\
318.6	0.00022205\\
321.52	0.00022379\\
324.46	0.00022553\\
327.44	0.00022727\\
330.43	0.00022901\\
333.46	0.00023075\\
336.52	0.00023248\\
339.6	0.00023422\\
342.71	0.00023595\\
345.85	0.00023768\\
349.02	0.00023941\\
352.21	0.00024113\\
355.44	0.00024285\\
358.7	0.00024457\\
361.98	0.00024628\\
365.3	0.00024798\\
368.64	0.00024969\\
372.02	0.00025138\\
375.43	0.00025307\\
378.87	0.00025476\\
382.34	0.00025644\\
385.84	0.00025811\\
389.37	0.00025977\\
392.94	0.00026143\\
396.54	0.00026308\\
400.17	0.00026472\\
403.84	0.00026635\\
407.54	0.00026797\\
411.27	0.00026959\\
415.04	0.00027119\\
418.84	0.00027278\\
422.67	0.00027437\\
426.55	0.00027594\\
430.45	0.0002775\\
434.4	0.00027905\\
438.37	0.00028058\\
442.39	0.0002821\\
446.44	0.00028361\\
450.53	0.00028511\\
454.66	0.00028659\\
458.82	0.00028806\\
463.03	0.00028952\\
467.27	0.00029096\\
471.55	0.00029238\\
475.87	0.00029379\\
480.23	0.00029518\\
484.62	0.00029656\\
489.06	0.00029791\\
493.54	0.00029926\\
498.06	0.00030058\\
502.63	0.00030189\\
507.23	0.00030317\\
511.88	0.00030444\\
516.56	0.00030569\\
521.3	0.00030692\\
526.07	0.00030813\\
530.89	0.00030932\\
535.75	0.00031049\\
540.66	0.00031164\\
545.61	0.00031277\\
550.61	0.00031387\\
555.65	0.00031496\\
560.74	0.00031602\\
565.88	0.00031706\\
571.06	0.00031807\\
576.29	0.00031906\\
581.57	0.00032003\\
586.9	0.00032098\\
592.28	0.0003219\\
597.7	0.0003228\\
603.18	0.00032367\\
608.7	0.00032451\\
614.28	0.00032534\\
619.9	0.00032613\\
625.58	0.0003269\\
631.31	0.00032764\\
637.1	0.00032836\\
642.93	0.00032905\\
648.82	0.00032972\\
654.76	0.00033035\\
660.76	0.00033096\\
666.81	0.00033154\\
672.92	0.0003321\\
679.09	0.00033262\\
685.31	0.00033312\\
691.58	0.00033359\\
697.92	0.00033403\\
704.31	0.00033444\\
710.76	0.00033482\\
717.27	0.00033518\\
723.84	0.0003355\\
730.47	0.0003358\\
737.16	0.00033606\\
743.92	0.0003363\\
750.73	0.0003365\\
757.61	0.00033668\\
764.55	0.00033683\\
771.55	0.00033694\\
778.62	0.00033703\\
785.75	0.00033708\\
792.95	0.00033711\\
800.21	0.0003371\\
807.54	0.00033707\\
814.94	0.000337\\
822.4	0.0003369\\
829.94	0.00033678\\
837.54	0.00033662\\
845.21	0.00033643\\
852.95	0.00033621\\
860.76	0.00033596\\
868.65	0.00033567\\
876.61	0.00033536\\
884.64	0.00033502\\
892.74	0.00033464\\
900.92	0.00033423\\
909.17	0.0003338\\
917.5	0.00033333\\
925.9	0.00033283\\
934.38	0.0003323\\
942.94	0.00033174\\
951.58	0.00033115\\
960.29	0.00033053\\
969.09	0.00032988\\
977.97	0.0003292\\
986.92	0.00032849\\
995.96	0.00032774\\
1005.1	0.00032697\\
1014.3	0.00032617\\
1023.6	0.00032534\\
1033	0.00032448\\
1042.4	0.00032359\\
1052	0.00032267\\
1061.6	0.00032173\\
1071.3	0.00032075\\
1081.1	0.00031975\\
1091	0.00031872\\
1101	0.00031766\\
1111.1	0.00031658\\
1121.3	0.00031546\\
1131.6	0.00031433\\
1141.9	0.00031316\\
1152.4	0.00031197\\
1163	0.00031076\\
1173.6	0.00030952\\
1184.4	0.00030826\\
1195.2	0.00030697\\
1206.2	0.00030565\\
1217.2	0.00030432\\
1228.4	0.00030296\\
1239.6	0.00030158\\
1251	0.00030017\\
1262.4	0.00029875\\
1274	0.0002973\\
1285.7	0.00029583\\
1297.4	0.00029433\\
1309.3	0.00029282\\
1321.3	0.00029128\\
1333.4	0.00028973\\
1345.6	0.00028815\\
1357.9	0.00028655\\
1370.4	0.00028493\\
1382.9	0.00028329\\
1395.6	0.00028163\\
1408.4	0.00027995\\
1421.3	0.00027825\\
1434.3	0.00027653\\
1447.4	0.00027478\\
1460.7	0.00027302\\
1474.1	0.00027124\\
1487.6	0.00026943\\
1501.2	0.00026761\\
1515	0.00026577\\
1528.8	0.00026391\\
1542.8	0.00026203\\
1557	0.00026013\\
1571.2	0.00025821\\
1585.6	0.00025628\\
1600.2	0.00025433\\
1614.8	0.00025236\\
1629.6	0.00025037\\
1644.5	0.00024837\\
1659.6	0.00024636\\
1674.8	0.00024432\\
1690.1	0.00024228\\
1705.6	0.00024022\\
1721.2	0.00023815\\
1737	0.00023606\\
1752.9	0.00023397\\
1769	0.00023186\\
1785.2	0.00022974\\
1801.5	0.00022762\\
1818	0.00022548\\
1834.7	0.00022334\\
1851.5	0.00022119\\
1868.5	0.00021904\\
1885.6	0.00021688\\
1902.8	0.00021471\\
1920.3	0.00021254\\
1937.9	0.00021037\\
1955.6	0.0002082\\
1973.5	0.00020602\\
1991.6	0.00020384\\
2009.8	0.00020167\\
2028.3	0.00019949\\
2046.8	0.00019732\\
2065.6	0.00019514\\
2084.5	0.00019297\\
2103.6	0.00019081\\
2122.9	0.00018865\\
2142.3	0.00018649\\
2161.9	0.00018434\\
2181.7	0.00018219\\
2201.7	0.00018006\\
2221.9	0.00017793\\
2242.2	0.0001758\\
2262.8	0.00017369\\
2283.5	0.00017159\\
2304.4	0.00016949\\
2325.5	0.00016741\\
2346.8	0.00016533\\
2368.3	0.00016327\\
2390	0.00016122\\
2411.9	0.00015918\\
2434	0.00015715\\
2456.3	0.00015514\\
2478.8	0.00015314\\
2501.5	0.00015115\\
2524.4	0.00014918\\
2547.6	0.00014722\\
2570.9	0.00014527\\
2594.4	0.00014334\\
2618.2	0.00014143\\
2642.2	0.00013953\\
2666.4	0.00013764\\
2690.8	0.00013577\\
2715.5	0.00013392\\
2740.3	0.00013208\\
2765.4	0.00013026\\
2790.8	0.00012846\\
2816.3	0.00012667\\
2842.1	0.00012489\\
2868.2	0.00012314\\
2894.4	0.0001214\\
2920.9	0.00011967\\
2947.7	0.00011797\\
2974.7	0.00011628\\
3001.9	0.0001146\\
3029.4	0.00011294\\
3057.2	0.0001113\\
3085.2	0.00010968\\
3113.5	0.00010807\\
3142	0.00010648\\
3170.8	0.0001049\\
3199.8	0.00010335\\
3229.1	0.0001018\\
3258.7	0.00010028\\
3288.5	9.8769e-05\\
3318.7	9.7276e-05\\
3349.1	9.58e-05\\
3379.7	9.434e-05\\
3410.7	9.2896e-05\\
3441.9	9.1469e-05\\
3473.5	9.0058e-05\\
3505.3	8.8663e-05\\
3537.4	8.7284e-05\\
3569.8	8.5921e-05\\
3602.5	8.4574e-05\\
3635.5	8.3243e-05\\
3668.8	8.1928e-05\\
3702.4	8.0629e-05\\
3736.3	7.9346e-05\\
3770.5	7.8078e-05\\
3805.1	7.6826e-05\\
3839.9	7.559e-05\\
3875.1	7.4369e-05\\
3910.6	7.3163e-05\\
3946.4	7.1973e-05\\
3982.6	7.0798e-05\\
4019	6.9639e-05\\
4055.9	6.8494e-05\\
4093	6.7364e-05\\
4130.5	6.625e-05\\
4168.3	6.515e-05\\
4206.5	6.4065e-05\\
4245	6.2994e-05\\
4283.9	6.1939e-05\\
4323.2	6.0897e-05\\
4362.8	5.987e-05\\
4402.7	5.8857e-05\\
4443.1	5.7858e-05\\
4483.8	5.6874e-05\\
4524.8	5.5903e-05\\
4566.3	5.4946e-05\\
4608.1	5.4002e-05\\
4650.3	5.3073e-05\\
4692.9	5.2156e-05\\
4735.9	5.1253e-05\\
4779.3	5.0363e-05\\
4823.1	4.9487e-05\\
4867.2	4.8623e-05\\
4911.8	4.7772e-05\\
4956.8	4.6934e-05\\
5002.2	4.6109e-05\\
5048	4.5296e-05\\
5094.3	4.4496e-05\\
5140.9	4.3707e-05\\
5188	4.2931e-05\\
5235.5	4.2167e-05\\
5283.5	4.1415e-05\\
5331.9	4.0674e-05\\
5380.7	3.9945e-05\\
5430	3.9228e-05\\
5479.8	3.8522e-05\\
5530	3.7827e-05\\
5580.6	3.7144e-05\\
5631.7	3.6471e-05\\
5683.3	3.5809e-05\\
5735.4	3.5158e-05\\
5787.9	3.4517e-05\\
5840.9	3.3887e-05\\
5894.4	3.3268e-05\\
5948.4	3.2658e-05\\
6002.9	3.2059e-05\\
6057.9	3.147e-05\\
6113.4	3.089e-05\\
6169.4	3.032e-05\\
6225.9	2.976e-05\\
6282.9	2.9209e-05\\
6340.5	2.8668e-05\\
6398.5	2.8135e-05\\
6457.2	2.7612e-05\\
6516.3	2.7098e-05\\
6576	2.6593e-05\\
6636.2	2.6096e-05\\
6697	2.5608e-05\\
6758.4	2.5128e-05\\
6820.3	2.4657e-05\\
6882.7	2.4194e-05\\
6945.8	2.3739e-05\\
7009.4	2.3292e-05\\
7073.6	2.2853e-05\\
7138.4	2.2422e-05\\
7203.8	2.1998e-05\\
7269.8	2.1582e-05\\
7336.4	2.1174e-05\\
7403.6	2.0772e-05\\
7471.4	2.0378e-05\\
7539.8	1.9991e-05\\
7608.9	1.9611e-05\\
7678.6	1.9238e-05\\
7748.9	1.8871e-05\\
7819.9	1.8511e-05\\
7891.5	1.8158e-05\\
7963.8	1.7811e-05\\
8036.8	1.7471e-05\\
8110.4	1.7137e-05\\
8184.7	1.6808e-05\\
8259.6	1.6486e-05\\
8335.3	1.617e-05\\
8411.6	1.586e-05\\
8488.7	1.5555e-05\\
8566.5	1.5256e-05\\
8644.9	1.4963e-05\\
8724.1	1.4675e-05\\
8804	1.4392e-05\\
8884.7	1.4115e-05\\
8966	1.3842e-05\\
9048.2	1.3575e-05\\
9131.1	1.3313e-05\\
9214.7	1.3056e-05\\
9299.1	1.2804e-05\\
9384.3	1.2556e-05\\
9470.2	1.2313e-05\\
9557	1.2075e-05\\
9644.5	1.1841e-05\\
9732.9	1.1611e-05\\
9822	1.1386e-05\\
9912	1.1165e-05\\
10003	1.0949e-05\\
10094	1.0736e-05\\
10187	1.0527e-05\\
10280	1.0323e-05\\
10374	1.0122e-05\\
10469	9.9251e-06\\
10565	9.732e-06\\
10662	9.5425e-06\\
10760	9.3566e-06\\
10858	9.1743e-06\\
10958	8.9955e-06\\
11058	8.8201e-06\\
11159	8.648e-06\\
11262	8.4792e-06\\
11365	8.3137e-06\\
11469	8.1513e-06\\
11574	7.992e-06\\
11680	7.8358e-06\\
11787	7.6826e-06\\
11895	7.5323e-06\\
12004	7.3849e-06\\
12114	7.2403e-06\\
12225	7.0986e-06\\
12337	6.9595e-06\\
12450	6.8231e-06\\
12564	6.6893e-06\\
12679	6.5581e-06\\
12795	6.4295e-06\\
12912	6.3033e-06\\
13030	6.1795e-06\\
13150	6.0581e-06\\
13270	5.9391e-06\\
13392	5.8223e-06\\
13515	5.7078e-06\\
13638	5.5955e-06\\
13763	5.4854e-06\\
13889	5.3774e-06\\
14017	5.2714e-06\\
14145	5.1676e-06\\
14274	5.0657e-06\\
14405	4.9658e-06\\
14537	4.8678e-06\\
14670	4.7717e-06\\
14805	4.6775e-06\\
14940	4.585e-06\\
15077	4.4944e-06\\
15215	4.4055e-06\\
15355	4.3183e-06\\
15495	4.2329e-06\\
15637	4.149e-06\\
15780	4.0668e-06\\
15925	3.9862e-06\\
16071	3.9071e-06\\
16218	3.8296e-06\\
16367	3.7535e-06\\
16517	3.6789e-06\\
16668	3.6058e-06\\
16821	3.5341e-06\\
16975	3.4637e-06\\
17130	3.3948e-06\\
17287	3.3271e-06\\
17445	3.2608e-06\\
17605	3.1957e-06\\
17766	3.1319e-06\\
17929	3.0694e-06\\
18093	3.008e-06\\
18259	2.9478e-06\\
18426	2.8888e-06\\
18595	2.831e-06\\
18765	2.7742e-06\\
18937	2.7186e-06\\
19111	2.664e-06\\
19286	2.6105e-06\\
19463	2.558e-06\\
19641	2.5065e-06\\
19821	2.4561e-06\\
20002	2.4066e-06\\
20186	2.358e-06\\
20370	2.3105e-06\\
20557	2.2638e-06\\
20745	2.218e-06\\
20935	2.1731e-06\\
21127	2.1291e-06\\
21321	2.086e-06\\
21516	2.0436e-06\\
21713	2.0021e-06\\
21912	1.9614e-06\\
22113	1.9215e-06\\
22315	1.8824e-06\\
22520	1.844e-06\\
22726	1.8064e-06\\
22934	1.7695e-06\\
23144	1.7333e-06\\
23356	1.6979e-06\\
23570	1.6631e-06\\
23786	1.629e-06\\
24004	1.5955e-06\\
24224	1.5627e-06\\
24446	1.5306e-06\\
24669	1.499e-06\\
24895	1.4681e-06\\
25123	1.4378e-06\\
25354	1.4081e-06\\
25586	1.379e-06\\
25820	1.3504e-06\\
26057	1.3224e-06\\
26295	1.2949e-06\\
26536	1.268e-06\\
26779	1.2416e-06\\
27025	1.2157e-06\\
27272	1.1903e-06\\
27522	1.1655e-06\\
27774	1.1411e-06\\
28028	1.1172e-06\\
28285	1.0937e-06\\
28544	1.0707e-06\\
28806	1.0482e-06\\
29070	1.0261e-06\\
29336	1.0045e-06\\
29605	9.8328e-07\\
29876	9.6248e-07\\
30149	9.421e-07\\
30426	9.2211e-07\\
30704	9.0253e-07\\
30986	8.8333e-07\\
31269	8.6451e-07\\
31556	8.4607e-07\\
31845	8.2799e-07\\
32137	8.1027e-07\\
32431	7.9291e-07\\
32728	7.7589e-07\\
33028	7.5922e-07\\
33330	7.4287e-07\\
33636	7.2686e-07\\
33944	7.1116e-07\\
34255	6.9578e-07\\
34568	6.8071e-07\\
34885	6.6594e-07\\
35205	6.5147e-07\\
35527	6.3729e-07\\
35852	6.234e-07\\
36181	6.0979e-07\\
36512	5.9646e-07\\
36847	5.8339e-07\\
37184	5.7059e-07\\
37525	5.5805e-07\\
37869	5.4577e-07\\
38215	5.3374e-07\\
38565	5.2195e-07\\
38919	5.104e-07\\
39275	4.9909e-07\\
39635	4.8802e-07\\
39998	4.7717e-07\\
40364	4.6654e-07\\
40734	4.5613e-07\\
41107	4.4594e-07\\
41484	4.3596e-07\\
41864	4.2619e-07\\
42247	4.1662e-07\\
42634	4.0724e-07\\
43025	3.9807e-07\\
43419	3.8908e-07\\
43817	3.8028e-07\\
44218	3.7167e-07\\
44623	3.6323e-07\\
45032	3.5498e-07\\
45444	3.469e-07\\
45860	3.3898e-07\\
46281	3.3124e-07\\
46704	3.2366e-07\\
47132	3.1624e-07\\
47564	3.0897e-07\\
48000	3.0187e-07\\
48439	2.9491e-07\\
48883	2.881e-07\\
49331	2.8144e-07\\
49783	2.7492e-07\\
50239	2.6854e-07\\
50699	2.623e-07\\
51163	2.5619e-07\\
51632	2.5021e-07\\
52105	2.4437e-07\\
52582	2.3865e-07\\
53064	2.3305e-07\\
53550	2.2758e-07\\
54040	2.2222e-07\\
54535	2.1699e-07\\
55035	2.1187e-07\\
55539	2.0685e-07\\
56048	2.0195e-07\\
56561	1.9716e-07\\
57079	1.9248e-07\\
57602	1.8789e-07\\
58130	1.8341e-07\\
58662	1.7903e-07\\
59200	1.7475e-07\\
59742	1.7056e-07\\
60289	1.6646e-07\\
60841	1.6246e-07\\
61399	1.5854e-07\\
61961	1.5472e-07\\
62529	1.5098e-07\\
63101	1.4732e-07\\
63679	1.4375e-07\\
64263	1.4025e-07\\
64851	1.3684e-07\\
65445	1.335e-07\\
66045	1.3024e-07\\
66650	1.2706e-07\\
67260	1.2395e-07\\
67876	1.209e-07\\
68498	1.1793e-07\\
69125	1.1503e-07\\
69759	1.1219e-07\\
70398	1.0942e-07\\
71042	1.0671e-07\\
71693	1.0407e-07\\
72350	1.0148e-07\\
73013	9.8958e-08\\
73681	9.6493e-08\\
74356	9.4086e-08\\
75037	9.1734e-08\\
75725	8.9438e-08\\
76418	8.7195e-08\\
77118	8.5005e-08\\
77825	8.2867e-08\\
78538	8.0779e-08\\
79257	7.874e-08\\
79983	7.6749e-08\\
80716	7.4805e-08\\
81455	7.2908e-08\\
82201	7.1056e-08\\
82954	6.9248e-08\\
83714	6.7482e-08\\
84481	6.576e-08\\
85255	6.4078e-08\\
86036	6.2437e-08\\
86824	6.0835e-08\\
87619	5.9271e-08\\
88421	5.7746e-08\\
89231	5.6257e-08\\
90049	5.4804e-08\\
90874	5.3387e-08\\
91706	5.2003e-08\\
92546	5.0654e-08\\
93394	4.9337e-08\\
94249	4.8053e-08\\
95113	4.68e-08\\
95984	4.5577e-08\\
96863	4.4385e-08\\
97750	4.3222e-08\\
98646	4.2088e-08\\
99549	4.0981e-08\\
1.0046e+05	3.9902e-08\\
1.0138e+05	3.885e-08\\
1.0231e+05	3.7824e-08\\
1.0325e+05	3.6823e-08\\
1.0419e+05	3.5847e-08\\
1.0515e+05	3.4895e-08\\
1.0611e+05	3.3968e-08\\
1.0708e+05	3.3063e-08\\
1.0806e+05	3.2181e-08\\
1.0905e+05	3.1322e-08\\
1.1005e+05	3.0483e-08\\
1.1106e+05	2.9667e-08\\
1.1208e+05	2.887e-08\\
1.131e+05	2.8094e-08\\
1.1414e+05	2.7337e-08\\
1.1519e+05	2.66e-08\\
1.1624e+05	2.5882e-08\\
1.1731e+05	2.5181e-08\\
1.1838e+05	2.4499e-08\\
1.1946e+05	2.3834e-08\\
1.2056e+05	2.3186e-08\\
1.2166e+05	2.2555e-08\\
1.2278e+05	2.1939e-08\\
1.239e+05	2.134e-08\\
1.2504e+05	2.0756e-08\\
1.2618e+05	2.0188e-08\\
1.2734e+05	1.9634e-08\\
1.285e+05	1.9094e-08\\
1.2968e+05	1.8568e-08\\
1.3087e+05	1.8056e-08\\
1.3207e+05	1.7558e-08\\
1.3328e+05	1.7072e-08\\
1.345e+05	1.6599e-08\\
1.3573e+05	1.6139e-08\\
1.3697e+05	1.569e-08\\
1.3823e+05	1.5253e-08\\
1.3949e+05	1.4828e-08\\
1.4077e+05	1.4414e-08\\
1.4206e+05	1.4011e-08\\
1.4336e+05	1.3619e-08\\
1.4468e+05	1.3237e-08\\
1.46e+05	1.2865e-08\\
1.4734e+05	1.2503e-08\\
1.4869e+05	1.2151e-08\\
1.5005e+05	1.1808e-08\\
1.5142e+05	1.1475e-08\\
1.5281e+05	1.115e-08\\
1.5421e+05	1.0834e-08\\
1.5562e+05	1.0526e-08\\
1.5705e+05	1.0227e-08\\
1.5849e+05	9.9358e-09\\
1.5994e+05	9.6525e-09\\
1.614e+05	9.3768e-09\\
1.6288e+05	9.1086e-09\\
1.6438e+05	8.8477e-09\\
1.6588e+05	8.5939e-09\\
1.674e+05	8.3471e-09\\
1.6893e+05	8.1069e-09\\
1.7048e+05	7.8734e-09\\
1.7204e+05	7.6462e-09\\
1.7362e+05	7.4252e-09\\
1.7521e+05	7.2104e-09\\
1.7681e+05	7.0014e-09\\
1.7843e+05	6.7982e-09\\
1.8007e+05	6.6006e-09\\
1.8172e+05	6.4085e-09\\
1.8338e+05	6.2217e-09\\
1.8506e+05	6.0401e-09\\
1.8676e+05	5.8635e-09\\
1.8847e+05	5.6919e-09\\
1.9019e+05	5.525e-09\\
1.9194e+05	5.3628e-09\\
1.9369e+05	5.2052e-09\\
1.9547e+05	5.0519e-09\\
1.9726e+05	4.903e-09\\
1.9907e+05	4.7583e-09\\
2.0089e+05	4.6176e-09\\
2.0273e+05	4.4809e-09\\
2.0459e+05	4.348e-09\\
2.0646e+05	4.2189e-09\\
2.0835e+05	4.0935e-09\\
2.1026e+05	3.9717e-09\\
2.1219e+05	3.8533e-09\\
2.1413e+05	3.7382e-09\\
2.1609e+05	3.6265e-09\\
2.1807e+05	3.5179e-09\\
2.2007e+05	3.4125e-09\\
2.2208e+05	3.3101e-09\\
2.2412e+05	3.2106e-09\\
2.2617e+05	3.114e-09\\
2.2824e+05	3.0201e-09\\
2.3033e+05	2.929e-09\\
2.3244e+05	2.8405e-09\\
2.3457e+05	2.7545e-09\\
2.3672e+05	2.6711e-09\\
2.3889e+05	2.5901e-09\\
2.4108e+05	2.5114e-09\\
2.4329e+05	2.435e-09\\
2.4551e+05	2.3608e-09\\
2.4776e+05	2.2888e-09\\
2.5003e+05	2.2189e-09\\
2.5003e+05	1.8469e-09\\
2.4776e+05	1.9072e-09\\
2.4551e+05	1.9694e-09\\
2.4329e+05	2.0335e-09\\
2.4108e+05	2.0996e-09\\
2.3889e+05	2.1678e-09\\
2.3672e+05	2.2381e-09\\
2.3457e+05	2.3106e-09\\
2.3244e+05	2.3853e-09\\
2.3033e+05	2.4624e-09\\
2.2824e+05	2.5418e-09\\
2.2617e+05	2.6236e-09\\
2.2412e+05	2.708e-09\\
2.2208e+05	2.7949e-09\\
2.2007e+05	2.8846e-09\\
2.1807e+05	2.9769e-09\\
2.1609e+05	3.0721e-09\\
2.1413e+05	3.1702e-09\\
2.1219e+05	3.2713e-09\\
2.1026e+05	3.3754e-09\\
2.0835e+05	3.4828e-09\\
2.0646e+05	3.5933e-09\\
2.0459e+05	3.7073e-09\\
2.0273e+05	3.8246e-09\\
2.0089e+05	3.9455e-09\\
1.9907e+05	4.0701e-09\\
1.9726e+05	4.1984e-09\\
1.9547e+05	4.3305e-09\\
1.9369e+05	4.4666e-09\\
1.9194e+05	4.6068e-09\\
1.9019e+05	4.7511e-09\\
1.8847e+05	4.8998e-09\\
1.8676e+05	5.0529e-09\\
1.8506e+05	5.2105e-09\\
1.8338e+05	5.3728e-09\\
1.8172e+05	5.5399e-09\\
1.8007e+05	5.712e-09\\
1.7843e+05	5.8891e-09\\
1.7681e+05	6.0715e-09\\
1.7521e+05	6.2592e-09\\
1.7362e+05	6.4525e-09\\
1.7204e+05	6.6514e-09\\
1.7048e+05	6.8561e-09\\
1.6893e+05	7.0668e-09\\
1.674e+05	7.2837e-09\\
1.6588e+05	7.5068e-09\\
1.6438e+05	7.7365e-09\\
1.6288e+05	7.9728e-09\\
1.614e+05	8.2159e-09\\
1.5994e+05	8.4661e-09\\
1.5849e+05	8.7235e-09\\
1.5705e+05	8.9883e-09\\
1.5562e+05	9.2607e-09\\
1.5421e+05	9.5409e-09\\
1.5281e+05	9.8292e-09\\
1.5142e+05	1.0126e-08\\
1.5005e+05	1.0431e-08\\
1.4869e+05	1.0744e-08\\
1.4734e+05	1.1067e-08\\
1.46e+05	1.1399e-08\\
1.4468e+05	1.174e-08\\
1.4336e+05	1.209e-08\\
1.4206e+05	1.2451e-08\\
1.4077e+05	1.2822e-08\\
1.3949e+05	1.3203e-08\\
1.3823e+05	1.3595e-08\\
1.3697e+05	1.3998e-08\\
1.3573e+05	1.4412e-08\\
1.345e+05	1.4838e-08\\
1.3328e+05	1.5275e-08\\
1.3207e+05	1.5725e-08\\
1.3087e+05	1.6187e-08\\
1.2968e+05	1.6662e-08\\
1.285e+05	1.715e-08\\
1.2734e+05	1.7651e-08\\
1.2618e+05	1.8167e-08\\
1.2504e+05	1.8696e-08\\
1.239e+05	1.924e-08\\
1.2278e+05	1.9799e-08\\
1.2166e+05	2.0373e-08\\
1.2056e+05	2.0962e-08\\
1.1946e+05	2.1568e-08\\
1.1838e+05	2.219e-08\\
1.1731e+05	2.2829e-08\\
1.1624e+05	2.3485e-08\\
1.1519e+05	2.4159e-08\\
1.1414e+05	2.4851e-08\\
1.131e+05	2.5561e-08\\
1.1208e+05	2.6291e-08\\
1.1106e+05	2.704e-08\\
1.1005e+05	2.7809e-08\\
1.0905e+05	2.8598e-08\\
1.0806e+05	2.9408e-08\\
1.0708e+05	3.024e-08\\
1.0611e+05	3.1094e-08\\
1.0515e+05	3.197e-08\\
1.0419e+05	3.2869e-08\\
1.0325e+05	3.3792e-08\\
1.0231e+05	3.4739e-08\\
1.0138e+05	3.5711e-08\\
1.0046e+05	3.6708e-08\\
99549	3.7731e-08\\
98646	3.878e-08\\
97750	3.9857e-08\\
96863	4.0961e-08\\
95984	4.2094e-08\\
95113	4.3256e-08\\
94249	4.4448e-08\\
93394	4.567e-08\\
92546	4.6924e-08\\
91706	4.8209e-08\\
90874	4.9527e-08\\
90049	5.0879e-08\\
89231	5.2265e-08\\
88421	5.3685e-08\\
87619	5.5142e-08\\
86824	5.6635e-08\\
86036	5.8165e-08\\
85255	5.9734e-08\\
84481	6.1342e-08\\
83714	6.299e-08\\
82954	6.4678e-08\\
82201	6.6409e-08\\
81455	6.8182e-08\\
80716	6.9999e-08\\
79983	7.186e-08\\
79257	7.3768e-08\\
78538	7.5721e-08\\
77825	7.7723e-08\\
77118	7.9773e-08\\
76418	8.1873e-08\\
75725	8.4024e-08\\
75037	8.6227e-08\\
74356	8.8483e-08\\
73681	9.0794e-08\\
73013	9.3159e-08\\
72350	9.5582e-08\\
71693	9.8062e-08\\
71042	1.006e-07\\
70398	1.032e-07\\
69759	1.0586e-07\\
69125	1.0859e-07\\
68498	1.1138e-07\\
67876	1.1423e-07\\
67260	1.1715e-07\\
66650	1.2014e-07\\
66045	1.232e-07\\
65445	1.2633e-07\\
64851	1.2954e-07\\
64263	1.3282e-07\\
63679	1.3617e-07\\
63101	1.396e-07\\
62529	1.4312e-07\\
61961	1.4671e-07\\
61399	1.5038e-07\\
60841	1.5414e-07\\
60289	1.5799e-07\\
59742	1.6192e-07\\
59200	1.6594e-07\\
58662	1.7006e-07\\
58130	1.7427e-07\\
57602	1.7857e-07\\
57079	1.8297e-07\\
56561	1.8747e-07\\
56048	1.9207e-07\\
55539	1.9677e-07\\
55035	2.0158e-07\\
54535	2.065e-07\\
54040	2.1152e-07\\
53550	2.1666e-07\\
53064	2.2191e-07\\
52582	2.2728e-07\\
52105	2.3277e-07\\
51632	2.3838e-07\\
51163	2.4411e-07\\
50699	2.4997e-07\\
50239	2.5596e-07\\
49783	2.6208e-07\\
49331	2.6833e-07\\
48883	2.7472e-07\\
48439	2.8125e-07\\
48000	2.8792e-07\\
47564	2.9474e-07\\
47132	3.017e-07\\
46704	3.0882e-07\\
46281	3.1609e-07\\
45860	3.2351e-07\\
45444	3.311e-07\\
45032	3.3885e-07\\
44623	3.4676e-07\\
44218	3.5485e-07\\
43817	3.6311e-07\\
43419	3.7154e-07\\
43025	3.8016e-07\\
42634	3.8896e-07\\
42247	3.9794e-07\\
41864	4.0712e-07\\
41484	4.1649e-07\\
41107	4.2606e-07\\
40734	4.3583e-07\\
40364	4.4581e-07\\
39998	4.56e-07\\
39635	4.664e-07\\
39275	4.7703e-07\\
38919	4.8787e-07\\
38565	4.9894e-07\\
38215	5.1025e-07\\
37869	5.2179e-07\\
37525	5.3357e-07\\
37184	5.456e-07\\
36847	5.5788e-07\\
36512	5.7041e-07\\
36181	5.832e-07\\
35852	5.9626e-07\\
35527	6.0959e-07\\
35205	6.232e-07\\
34885	6.3709e-07\\
34568	6.5126e-07\\
34255	6.6572e-07\\
33944	6.8049e-07\\
33636	6.9556e-07\\
33330	7.1093e-07\\
33028	7.2663e-07\\
32728	7.4264e-07\\
32431	7.5898e-07\\
32137	7.7566e-07\\
31845	7.9268e-07\\
31556	8.1005e-07\\
31269	8.2777e-07\\
30986	8.4585e-07\\
30704	8.643e-07\\
30426	8.8312e-07\\
30149	9.0233e-07\\
29876	9.2193e-07\\
29605	9.4193e-07\\
29336	9.6233e-07\\
29070	9.8315e-07\\
28806	1.0044e-06\\
28544	1.0261e-06\\
28285	1.0482e-06\\
28028	1.0707e-06\\
27774	1.0937e-06\\
27522	1.1172e-06\\
27272	1.1411e-06\\
27025	1.1656e-06\\
26779	1.1905e-06\\
26536	1.2159e-06\\
26295	1.2418e-06\\
26057	1.2683e-06\\
25820	1.2953e-06\\
25586	1.3228e-06\\
25354	1.3509e-06\\
25123	1.3795e-06\\
24895	1.4088e-06\\
24669	1.4386e-06\\
24446	1.469e-06\\
24224	1.5e-06\\
24004	1.5316e-06\\
23786	1.5639e-06\\
23570	1.5968e-06\\
23356	1.6304e-06\\
23144	1.6646e-06\\
22934	1.6995e-06\\
22726	1.7352e-06\\
22520	1.7715e-06\\
22315	1.8086e-06\\
22113	1.8464e-06\\
21912	1.885e-06\\
21713	1.9243e-06\\
21516	1.9644e-06\\
21321	2.0053e-06\\
21127	2.0471e-06\\
20935	2.0897e-06\\
20745	2.1331e-06\\
20557	2.1774e-06\\
20370	2.2225e-06\\
20186	2.2686e-06\\
20002	2.3156e-06\\
19821	2.3636e-06\\
19641	2.4124e-06\\
19463	2.4623e-06\\
19286	2.5132e-06\\
19111	2.5651e-06\\
18937	2.618e-06\\
18765	2.6719e-06\\
18595	2.727e-06\\
18426	2.7831e-06\\
18259	2.8404e-06\\
18093	2.8988e-06\\
17929	2.9584e-06\\
17766	3.0192e-06\\
17605	3.0811e-06\\
17445	3.1443e-06\\
17287	3.2088e-06\\
17130	3.2746e-06\\
16975	3.3417e-06\\
16821	3.4101e-06\\
16668	3.4799e-06\\
16517	3.551e-06\\
16367	3.6236e-06\\
16218	3.6977e-06\\
16071	3.7732e-06\\
15925	3.8502e-06\\
15780	3.9288e-06\\
15637	4.0089e-06\\
15495	4.0906e-06\\
15355	4.174e-06\\
15215	4.259e-06\\
15077	4.3457e-06\\
14940	4.4342e-06\\
14805	4.5244e-06\\
14670	4.6164e-06\\
14537	4.7102e-06\\
14405	4.8059e-06\\
14274	4.9036e-06\\
14145	5.0031e-06\\
14017	5.1047e-06\\
13889	5.2082e-06\\
13763	5.3138e-06\\
13638	5.4216e-06\\
13515	5.5314e-06\\
13392	5.6435e-06\\
13270	5.7577e-06\\
13150	5.8743e-06\\
13030	5.9931e-06\\
12912	6.1144e-06\\
12795	6.238e-06\\
12679	6.364e-06\\
12564	6.4926e-06\\
12450	6.6237e-06\\
12337	6.7574e-06\\
12225	6.8938e-06\\
12114	7.0328e-06\\
12004	7.1746e-06\\
11895	7.3191e-06\\
11787	7.4666e-06\\
11680	7.6169e-06\\
11574	7.7702e-06\\
11469	7.9265e-06\\
11365	8.0859e-06\\
11262	8.2484e-06\\
11159	8.414e-06\\
11058	8.583e-06\\
10958	8.7552e-06\\
10858	8.9308e-06\\
10760	9.1098e-06\\
10662	9.2923e-06\\
10565	9.4784e-06\\
10469	9.6681e-06\\
10374	9.8615e-06\\
10280	1.0059e-05\\
10187	1.026e-05\\
10094	1.0464e-05\\
10003	1.0673e-05\\
9912	1.0886e-05\\
9822	1.1103e-05\\
9732.9	1.1324e-05\\
9644.5	1.1549e-05\\
9557	1.1779e-05\\
9470.2	1.2013e-05\\
9384.3	1.2252e-05\\
9299.1	1.2495e-05\\
9214.7	1.2742e-05\\
9131.1	1.2995e-05\\
9048.2	1.3252e-05\\
8966	1.3514e-05\\
8884.7	1.3781e-05\\
8804	1.4053e-05\\
8724.1	1.433e-05\\
8644.9	1.4613e-05\\
8566.5	1.49e-05\\
8488.7	1.5194e-05\\
8411.6	1.5492e-05\\
8335.3	1.5796e-05\\
8259.6	1.6106e-05\\
8184.7	1.6422e-05\\
8110.4	1.6743e-05\\
8036.8	1.707e-05\\
7963.8	1.7404e-05\\
7891.5	1.7743e-05\\
7819.9	1.8089e-05\\
7748.9	1.8441e-05\\
7678.6	1.88e-05\\
7608.9	1.9165e-05\\
7539.8	1.9536e-05\\
7471.4	1.9915e-05\\
7403.6	2.03e-05\\
7336.4	2.0692e-05\\
7269.8	2.1092e-05\\
7203.8	2.1498e-05\\
7138.4	2.1912e-05\\
7073.6	2.2333e-05\\
7009.4	2.2762e-05\\
6945.8	2.3198e-05\\
6882.7	2.3642e-05\\
6820.3	2.4094e-05\\
6758.4	2.4554e-05\\
6697	2.5022e-05\\
6636.2	2.5498e-05\\
6576	2.5982e-05\\
6516.3	2.6475e-05\\
6457.2	2.6977e-05\\
6398.5	2.7487e-05\\
6340.5	2.8006e-05\\
6282.9	2.8533e-05\\
6225.9	2.907e-05\\
6169.4	2.9616e-05\\
6113.4	3.0172e-05\\
6057.9	3.0736e-05\\
6002.9	3.1311e-05\\
5948.4	3.1895e-05\\
5894.4	3.2489e-05\\
5840.9	3.3092e-05\\
5787.9	3.3706e-05\\
5735.4	3.433e-05\\
5683.3	3.4965e-05\\
5631.7	3.561e-05\\
5580.6	3.6265e-05\\
5530	3.6931e-05\\
5479.8	3.7608e-05\\
5430	3.8296e-05\\
5380.7	3.8995e-05\\
5331.9	3.9706e-05\\
5283.5	4.0428e-05\\
5235.5	4.1161e-05\\
5188	4.1906e-05\\
5140.9	4.2663e-05\\
5094.3	4.3432e-05\\
5048	4.4213e-05\\
5002.2	4.5006e-05\\
4956.8	4.5811e-05\\
4911.8	4.6629e-05\\
4867.2	4.746e-05\\
4823.1	4.8303e-05\\
4779.3	4.9159e-05\\
4735.9	5.0028e-05\\
4692.9	5.0911e-05\\
4650.3	5.1806e-05\\
4608.1	5.2715e-05\\
4566.3	5.3638e-05\\
4524.8	5.4574e-05\\
4483.8	5.5524e-05\\
4443.1	5.6488e-05\\
4402.7	5.7466e-05\\
4362.8	5.8459e-05\\
4323.2	5.9465e-05\\
4283.9	6.0486e-05\\
4245	6.1522e-05\\
4206.5	6.2572e-05\\
4168.3	6.3637e-05\\
4130.5	6.4717e-05\\
4093	6.5812e-05\\
4055.9	6.6922e-05\\
4019	6.8048e-05\\
3982.6	6.9188e-05\\
3946.4	7.0344e-05\\
3910.6	7.1516e-05\\
3875.1	7.2704e-05\\
3839.9	7.3907e-05\\
3805.1	7.5126e-05\\
3770.5	7.6361e-05\\
3736.3	7.7612e-05\\
3702.4	7.8879e-05\\
3668.8	8.0162e-05\\
3635.5	8.1461e-05\\
3602.5	8.2777e-05\\
3569.8	8.4109e-05\\
3537.4	8.5458e-05\\
3505.3	8.6823e-05\\
3473.5	8.8204e-05\\
3441.9	8.9602e-05\\
3410.7	9.1017e-05\\
3379.7	9.2448e-05\\
3349.1	9.3896e-05\\
3318.7	9.536e-05\\
3288.5	9.6841e-05\\
3258.7	9.8338e-05\\
3229.1	9.9852e-05\\
3199.8	0.00010138\\
3170.8	0.00010293\\
3142	0.00010449\\
3113.5	0.00010607\\
3085.2	0.00010767\\
3057.2	0.00010928\\
3029.4	0.00011091\\
3001.9	0.00011255\\
2974.7	0.00011421\\
2947.7	0.00011588\\
2920.9	0.00011757\\
2894.4	0.00011928\\
2868.2	0.000121\\
2842.1	0.00012274\\
2816.3	0.00012449\\
2790.8	0.00012625\\
2765.4	0.00012803\\
2740.3	0.00012982\\
2715.5	0.00013163\\
2690.8	0.00013345\\
2666.4	0.00013529\\
2642.2	0.00013713\\
2618.2	0.00013899\\
2594.4	0.00014087\\
2570.9	0.00014275\\
2547.6	0.00014465\\
2524.4	0.00014656\\
2501.5	0.00014848\\
2478.8	0.00015041\\
2456.3	0.00015235\\
2434	0.00015431\\
2411.9	0.00015627\\
2390	0.00015824\\
2368.3	0.00016023\\
2346.8	0.00016222\\
2325.5	0.00016422\\
2304.4	0.00016623\\
2283.5	0.00016825\\
2262.8	0.00017027\\
2242.2	0.00017231\\
2221.9	0.00017434\\
2201.7	0.00017639\\
2181.7	0.00017844\\
2161.9	0.0001805\\
2142.3	0.00018256\\
2122.9	0.00018463\\
2103.6	0.0001867\\
2084.5	0.00018878\\
2065.6	0.00019086\\
2046.8	0.00019294\\
2028.3	0.00019503\\
2009.8	0.00019711\\
1991.6	0.0001992\\
1973.5	0.00020129\\
1955.6	0.00020337\\
1937.9	0.00020546\\
1920.3	0.00020755\\
1902.8	0.00020963\\
1885.6	0.00021172\\
1868.5	0.0002138\\
1851.5	0.00021587\\
1834.7	0.00021795\\
1818	0.00022002\\
1801.5	0.00022208\\
1785.2	0.00022414\\
1769	0.00022619\\
1752.9	0.00022824\\
1737	0.00023028\\
1721.2	0.00023231\\
1705.6	0.00023433\\
1690.1	0.00023635\\
1674.8	0.00023836\\
1659.6	0.00024035\\
1644.5	0.00024234\\
1629.6	0.00024431\\
1614.8	0.00024627\\
1600.2	0.00024822\\
1585.6	0.00025016\\
1571.2	0.00025209\\
1557	0.000254\\
1542.8	0.00025589\\
1528.8	0.00025777\\
1515	0.00025964\\
1501.2	0.00026148\\
1487.6	0.00026332\\
1474.1	0.00026513\\
1460.7	0.00026692\\
1447.4	0.0002687\\
1434.3	0.00027046\\
1421.3	0.00027219\\
1408.4	0.00027391\\
1395.6	0.0002756\\
1382.9	0.00027728\\
1370.4	0.00027893\\
1357.9	0.00028055\\
1345.6	0.00028216\\
1333.4	0.00028374\\
1321.3	0.00028529\\
1309.3	0.00028683\\
1297.4	0.00028833\\
1285.7	0.00028981\\
1274	0.00029127\\
1262.4	0.0002927\\
1251	0.0002941\\
1239.6	0.00029547\\
1228.4	0.00029682\\
1217.2	0.00029815\\
1206.2	0.00029944\\
1195.2	0.00030071\\
1184.4	0.00030195\\
1173.6	0.00030316\\
1163	0.00030435\\
1152.4	0.00030551\\
1141.9	0.00030664\\
1131.6	0.00030774\\
1121.3	0.00030882\\
1111.1	0.00030987\\
1101	0.00031089\\
1091	0.00031188\\
1081.1	0.00031284\\
1071.3	0.00031378\\
1061.6	0.00031469\\
1052	0.00031557\\
1042.4	0.00031642\\
1033	0.00031724\\
1023.6	0.00031804\\
1014.3	0.0003188\\
1005.1	0.00031954\\
995.96	0.00032024\\
986.92	0.00032092\\
977.97	0.00032157\\
969.09	0.00032219\\
960.29	0.00032278\\
951.58	0.00032334\\
942.94	0.00032387\\
934.38	0.00032437\\
925.9	0.00032484\\
917.5	0.00032527\\
909.17	0.00032568\\
900.92	0.00032606\\
892.74	0.0003264\\
884.64	0.00032672\\
876.61	0.000327\\
868.65	0.00032726\\
860.76	0.00032748\\
852.95	0.00032767\\
845.21	0.00032783\\
837.54	0.00032795\\
829.94	0.00032805\\
822.4	0.00032812\\
814.94	0.00032815\\
807.54	0.00032815\\
800.21	0.00032812\\
792.95	0.00032806\\
785.75	0.00032797\\
778.62	0.00032785\\
771.55	0.0003277\\
764.55	0.00032751\\
757.61	0.0003273\\
750.73	0.00032706\\
743.92	0.00032678\\
737.16	0.00032647\\
730.47	0.00032614\\
723.84	0.00032577\\
717.27	0.00032538\\
710.76	0.00032496\\
704.31	0.0003245\\
697.92	0.00032402\\
691.58	0.00032351\\
685.31	0.00032297\\
679.09	0.0003224\\
672.92	0.00032181\\
666.81	0.00032118\\
660.76	0.00032053\\
654.76	0.00031986\\
648.82	0.00031915\\
642.93	0.00031842\\
637.1	0.00031766\\
631.31	0.00031688\\
625.58	0.00031607\\
619.9	0.00031524\\
614.28	0.00031438\\
608.7	0.0003135\\
603.18	0.00031259\\
597.7	0.00031166\\
592.28	0.00031071\\
586.9	0.00030973\\
581.57	0.00030873\\
576.29	0.00030771\\
571.06	0.00030667\\
565.88	0.0003056\\
560.74	0.00030451\\
555.65	0.00030341\\
550.61	0.00030228\\
545.61	0.00030113\\
540.66	0.00029996\\
535.75	0.00029877\\
530.89	0.00029756\\
526.07	0.00029634\\
521.3	0.00029509\\
516.56	0.00029383\\
511.88	0.00029255\\
507.23	0.00029125\\
502.63	0.00028993\\
498.06	0.0002886\\
493.54	0.00028725\\
489.06	0.00028589\\
484.62	0.00028451\\
480.23	0.00028311\\
475.87	0.0002817\\
471.55	0.00028028\\
467.27	0.00027884\\
463.03	0.00027739\\
458.82	0.00027592\\
454.66	0.00027444\\
450.53	0.00027295\\
446.44	0.00027145\\
442.39	0.00026994\\
438.37	0.00026841\\
434.4	0.00026687\\
430.45	0.00026532\\
426.55	0.00026376\\
422.67	0.00026219\\
418.84	0.00026061\\
415.04	0.00025903\\
411.27	0.00025743\\
407.54	0.00025582\\
403.84	0.00025421\\
400.17	0.00025258\\
396.54	0.00025095\\
392.94	0.00024932\\
389.37	0.00024767\\
385.84	0.00024602\\
382.34	0.00024436\\
378.87	0.0002427\\
375.43	0.00024103\\
372.02	0.00023935\\
368.64	0.00023767\\
365.3	0.00023599\\
361.98	0.0002343\\
358.7	0.00023261\\
355.44	0.00023091\\
352.21	0.00022921\\
349.02	0.00022751\\
345.85	0.0002258\\
342.71	0.00022409\\
339.6	0.00022238\\
336.52	0.00022067\\
333.46	0.00021895\\
330.43	0.00021724\\
327.44	0.00021552\\
324.46	0.0002138\\
321.52	0.00021208\\
318.6	0.00021037\\
315.71	0.00020865\\
312.84	0.00020693\\
310	0.00020521\\
307.19	0.00020349\\
304.4	0.00020178\\
301.64	0.00020007\\
298.9	0.00019835\\
296.19	0.00019664\\
293.5	0.00019493\\
290.83	0.00019323\\
288.19	0.00019153\\
285.58	0.00018983\\
282.99	0.00018813\\
280.42	0.00018644\\
277.87	0.00018475\\
275.35	0.00018306\\
272.85	0.00018138\\
270.37	0.0001797\\
267.92	0.00017803\\
265.49	0.00017636\\
263.08	0.00017469\\
260.69	0.00017304\\
258.32	0.00017138\\
255.98	0.00016974\\
253.66	0.00016809\\
251.35	0.00016646\\
249.07	0.00016483\\
246.81	0.0001632\\
244.57	0.00016159\\
242.35	0.00015998\\
240.15	0.00015837\\
237.97	0.00015678\\
235.81	0.00015519\\
233.67	0.0001536\\
231.55	0.00015203\\
229.45	0.00015046\\
227.37	0.0001489\\
225.3	0.00014735\\
223.26	0.00014581\\
221.23	0.00014427\\
219.22	0.00014274\\
217.23	0.00014123\\
215.26	0.00013972\\
213.31	0.00013821\\
211.37	0.00013672\\
209.45	0.00013524\\
207.55	0.00013376\\
205.67	0.00013229\\
203.8	0.00013084\\
201.95	0.00012939\\
200.12	0.00012795\\
198.3	0.00012652\\
196.5	0.0001251\\
194.72	0.00012369\\
192.95	0.00012229\\
191.2	0.0001209\\
189.46	0.00011952\\
187.74	0.00011815\\
186.04	0.00011679\\
184.35	0.00011544\\
182.68	0.0001141\\
181.02	0.00011277\\
179.38	0.00011145\\
177.75	0.00011014\\
176.14	0.00010884\\
174.54	0.00010755\\
172.95	0.00010627\\
171.38	0.00010501\\
169.83	0.00010375\\
168.29	0.0001025\\
166.76	0.00010126\\
165.24	0.00010004\\
163.74	9.882e-05\\
162.26	9.7615e-05\\
160.79	9.6419e-05\\
159.33	9.5235e-05\\
157.88	9.4061e-05\\
156.45	9.2897e-05\\
155.03	9.1745e-05\\
153.62	9.0602e-05\\
152.23	8.9471e-05\\
150.84	8.8349e-05\\
149.47	8.7239e-05\\
148.12	8.6138e-05\\
146.77	8.5049e-05\\
145.44	8.397e-05\\
144.12	8.2901e-05\\
142.81	8.1843e-05\\
141.52	8.0795e-05\\
140.23	7.9758e-05\\
138.96	7.8731e-05\\
137.7	7.7714e-05\\
136.45	7.6708e-05\\
135.21	7.5712e-05\\
133.98	7.4726e-05\\
132.77	7.375e-05\\
131.56	7.2785e-05\\
130.37	7.183e-05\\
129.18	7.0885e-05\\
128.01	6.995e-05\\
126.85	6.9025e-05\\
125.7	6.811e-05\\
124.56	6.7204e-05\\
123.43	6.6309e-05\\
122.31	6.5424e-05\\
121.2	6.4548e-05\\
120.1	6.3682e-05\\
119.01	6.2826e-05\\
117.93	6.1979e-05\\
116.85	6.1142e-05\\
115.79	6.0314e-05\\
114.74	5.9496e-05\\
113.7	5.8687e-05\\
112.67	5.7888e-05\\
111.65	5.7097e-05\\
110.63	5.6316e-05\\
109.63	5.5544e-05\\
108.63	5.4781e-05\\
107.65	5.4027e-05\\
106.67	5.3282e-05\\
105.7	5.2546e-05\\
104.74	5.1819e-05\\
103.79	5.11e-05\\
102.85	5.039e-05\\
101.92	4.9689e-05\\
100.99	4.8996e-05\\
100.08	4.8312e-05\\
99.167	4.7636e-05\\
98.267	4.6968e-05\\
97.375	4.6309e-05\\
96.491	4.5657e-05\\
95.615	4.5014e-05\\
94.747	4.4379e-05\\
93.887	4.3752e-05\\
93.035	4.3133e-05\\
92.191	4.2521e-05\\
91.354	4.1917e-05\\
90.525	4.1321e-05\\
89.703	4.0733e-05\\
88.889	4.0152e-05\\
88.082	3.9578e-05\\
87.283	3.9012e-05\\
86.49	3.8453e-05\\
85.705	3.7902e-05\\
84.927	3.7357e-05\\
84.156	3.682e-05\\
83.393	3.6289e-05\\
82.636	3.5766e-05\\
81.886	3.5249e-05\\
81.142	3.474e-05\\
80.406	3.4237e-05\\
79.676	3.374e-05\\
78.953	3.325e-05\\
78.236	3.2767e-05\\
77.526	3.229e-05\\
76.822	3.182e-05\\
76.125	3.1355e-05\\
75.434	3.0897e-05\\
74.749	3.0445e-05\\
74.071	3e-05\\
73.399	2.956e-05\\
72.732	2.9126e-05\\
72.072	2.8698e-05\\
71.418	2.8276e-05\\
70.77	2.786e-05\\
70.127	2.7449e-05\\
69.491	2.7044e-05\\
68.86	2.6644e-05\\
68.235	2.625e-05\\
67.616	2.5862e-05\\
67.002	2.5478e-05\\
66.394	2.51e-05\\
65.791	2.4728e-05\\
65.194	2.436e-05\\
64.602	2.3997e-05\\
64.016	2.364e-05\\
63.435	2.3287e-05\\
62.859	2.294e-05\\
62.289	2.2597e-05\\
61.723	2.2259e-05\\
61.163	2.1926e-05\\
60.608	2.1597e-05\\
60.058	2.1273e-05\\
59.512	2.0953e-05\\
58.972	2.0638e-05\\
58.437	2.0328e-05\\
57.907	2.0022e-05\\
57.381	1.972e-05\\
56.86	1.9422e-05\\
56.344	1.9129e-05\\
55.833	1.884e-05\\
55.326	1.8554e-05\\
54.824	1.8273e-05\\
54.326	1.7996e-05\\
53.833	1.7723e-05\\
53.344	1.7454e-05\\
52.86	1.7188e-05\\
52.38	1.6927e-05\\
51.905	1.6669e-05\\
51.434	1.6414e-05\\
50.967	1.6164e-05\\
50.504	1.5917e-05\\
50.046	1.5673e-05\\
49.592	1.5433e-05\\
49.141	1.5197e-05\\
48.695	1.4964e-05\\
48.253	1.4734e-05\\
47.815	1.4507e-05\\
47.381	1.4284e-05\\
46.951	1.4064e-05\\
46.525	1.3847e-05\\
46.103	1.3634e-05\\
45.684	1.3423e-05\\
45.27	1.3216e-05\\
44.859	1.3011e-05\\
44.452	1.2809e-05\\
44.048	1.2611e-05\\
43.648	1.2415e-05\\
43.252	1.2222e-05\\
42.86	1.2032e-05\\
42.471	1.1844e-05\\
42.085	1.166e-05\\
41.703	1.1478e-05\\
41.325	1.1298e-05\\
40.949	1.1122e-05\\
40.578	1.0948e-05\\
40.209	1.0776e-05\\
39.845	1.0607e-05\\
39.483	1.044e-05\\
39.124	1.0276e-05\\
38.769	1.0114e-05\\
38.417	9.9551e-06\\
38.069	9.7981e-06\\
37.723	9.6433e-06\\
37.381	9.4909e-06\\
37.042	9.3406e-06\\
36.705	9.1926e-06\\
36.372	9.0468e-06\\
36.042	8.9031e-06\\
35.715	8.7615e-06\\
35.391	8.622e-06\\
35.069	8.4846e-06\\
34.751	8.3492e-06\\
34.436	8.2158e-06\\
34.123	8.0844e-06\\
33.813	7.9549e-06\\
33.506	7.8274e-06\\
33.202	7.7017e-06\\
32.901	7.5779e-06\\
32.602	7.456e-06\\
32.306	7.3358e-06\\
32.013	7.2175e-06\\
31.723	7.1009e-06\\
31.435	6.986e-06\\
31.149	6.8729e-06\\
30.867	6.7614e-06\\
30.586	6.6516e-06\\
30.309	6.5435e-06\\
30.034	6.437e-06\\
29.761	6.332e-06\\
29.491	6.2287e-06\\
29.223	6.1269e-06\\
28.958	6.0266e-06\\
28.695	5.9278e-06\\
28.435	5.8305e-06\\
28.177	5.7347e-06\\
27.921	5.6403e-06\\
27.667	5.5474e-06\\
}--cycle;
\addplot[ybar interval, fill=black, fill opacity=0.35, area legend, draw=none] table[row sep=crcr, x=Lower, y=Count] {%
Lower	Upper	Count\\
27.667	40.439	0\\
40.439	59.107	0\\
59.107	86.392	0\\
86.392	126.27	6.6867e-05\\
126.27	184.56	9.1497e-05\\
184.56	269.76	0.0002817\\
269.76	394.29	0.00019273\\
394.29	576.29	0.00033698\\
576.29	842.32	0.00035084\\
842.32	1231.2	0.00038405\\
1231.2	1799.5	0.00025807\\
1799.5	2630.2	0.00016051\\
2630.2	3844.3	0.00010542\\
3844.3	5618.9	5.1091e-05\\
5618.9	8212.7	1.9534e-05\\
8212.7	12004	9.8475e-06\\
12004	17545	3.8499e-06\\
17545	25644	9.8776e-07\\
25644	37482	6.758e-07\\
37482	54785	4.6236e-07\\
54785	80074	1.0544e-07\\
80074	1.1704e+05	0\\
1.1704e+05	1.7107e+05	4.9358e-08\\
1.7107e+05	2.5003e+05	3.3769e-08\\
2.5003e+05	2.5003e+05	3.3769e-08\\
};
\addlegendentry{Istogramma reale}

\addplot [color=black]
  table[row sep=crcr]{%
27.667	9.8285e-07\\
32.013	1.6563e-06\\
36.042	2.4867e-06\\
39.845	3.4641e-06\\
43.648	4.6365e-06\\
47.381	5.9792e-06\\
50.967	7.4482e-06\\
54.824	9.2223e-06\\
58.437	1.1063e-05\\
62.289	1.3211e-05\\
65.791	1.5325e-05\\
69.491	1.7716e-05\\
73.399	2.0411e-05\\
77.526	2.3437e-05\\
81.142	2.623e-05\\
84.927	2.9288e-05\\
88.889	3.2625e-05\\
93.035	3.6256e-05\\
97.375	4.0197e-05\\
101.92	4.4462e-05\\
106.67	4.9063e-05\\
111.65	5.4013e-05\\
116.85	5.9322e-05\\
122.31	6.5001e-05\\
128.01	7.1055e-05\\
132.77	7.6172e-05\\
137.7	8.1536e-05\\
142.81	8.7145e-05\\
148.12	9.3e-05\\
155.03	0.00010066\\
162.26	0.0001087\\
169.83	0.0001171\\
177.75	0.00012585\\
186.04	0.00013494\\
194.72	0.00014435\\
203.8	0.00015405\\
215.26	0.00016603\\
227.37	0.00017834\\
242.35	0.00019303\\
263.08	0.00021227\\
310	0.00025088\\
330.43	0.00026552\\
349.02	0.00027774\\
365.3	0.0002876\\
382.34	0.00029711\\
396.54	0.00030442\\
411.27	0.00031145\\
426.55	0.00031816\\
442.39	0.00032453\\
458.82	0.00033052\\
475.87	0.00033612\\
489.06	0.00034005\\
502.63	0.00034373\\
516.56	0.00034716\\
530.89	0.00035033\\
545.61	0.00035322\\
560.74	0.00035584\\
576.29	0.00035817\\
592.28	0.00036022\\
608.7	0.00036197\\
625.58	0.00036342\\
642.93	0.00036457\\
660.76	0.00036541\\
679.09	0.00036595\\
697.92	0.00036618\\
717.27	0.0003661\\
737.16	0.00036571\\
757.61	0.00036501\\
778.62	0.00036401\\
800.21	0.00036271\\
822.4	0.0003611\\
845.21	0.0003592\\
868.65	0.00035701\\
892.74	0.00035453\\
917.5	0.00035178\\
942.94	0.00034875\\
969.09	0.00034545\\
995.96	0.00034191\\
1023.6	0.00033811\\
1052	0.00033407\\
1091	0.00032834\\
1131.6	0.00032222\\
1173.6	0.00031575\\
1217.2	0.00030895\\
1262.4	0.00030184\\
1309.3	0.00029446\\
1370.4	0.0002849\\
1434.3	0.000275\\
1515	0.00026278\\
1614.8	0.00024817\\
1769	0.00022691\\
2028.3	0.00019501\\
2161.9	0.00018043\\
2283.5	0.00016822\\
2411.9	0.00015632\\
2524.4	0.0001467\\
2642.2	0.00013736\\
2765.4	0.00012833\\
2894.4	0.00011963\\
3029.4	0.00011128\\
3170.8	0.00010329\\
3318.7	9.5672e-05\\
3473.5	8.8426e-05\\
3635.5	8.1559e-05\\
3805.1	7.5069e-05\\
3982.6	6.8956e-05\\
4168.3	6.3213e-05\\
4362.8	5.7835e-05\\
4566.3	5.2812e-05\\
4779.3	4.8134e-05\\
5002.2	4.379e-05\\
5235.5	3.9766e-05\\
5479.8	3.6049e-05\\
5735.4	3.2625e-05\\
6002.9	2.9478e-05\\
6282.9	2.6594e-05\\
6636.2	2.3458e-05\\
7009.4	2.0651e-05\\
7403.6	1.8147e-05\\
7819.9	1.5921e-05\\
8335.3	1.3642e-05\\
8884.7	1.167e-05\\
9557	9.7479e-06\\
10280	8.1337e-06\\
11159	6.6314e-06\\
12225	5.2874e-06\\
13515	4.1293e-06\\
15077	3.1661e-06\\
17130	2.3404e-06\\
20002	1.644e-06\\
24446	1.0661e-06\\
32137	6.1208e-07\\
48000	2.7914e-07\\
92546	7.0864e-08\\
2.5003e+05	5.7352e-09\\
};
\addlegendentry{Adatt. BLN reale ($\mathit{ML}$)}

            % \input{../../../.tex/20/KS/SizeDistributionFittingsfig3.tex}
        \end{groupplot}
    
        \begin{groupplot}[
            group style={
                group name=Row2,
                plotGroup,
            },
            plotRow,
            plotLegendSW,
            %
            xticklabels={},
            xmode=log,ymode=log,
            xmin=3637.5,xmax=2.5003e+05,
            ymin=0.0045909,ymax=1,
        ]
            \nextgroupplot[%
                at={($(Row1 c1r1.south west)-(0,\yPlotSep em)-(0,\plotHeight cm)$)},
                % xmin=3968.2,xmax=2.5003e+05,
                % ymin=0.0051726,ymax=1,
            ] % This file was created by matlab2tikz.
%
\definecolor{mycolor1}{rgb}{0.00000,0.44706,0.69804}%
\definecolor{mycolor2}{rgb}{0.00000,0.44700,0.74100}%
\definecolor{mycolor3}{rgb}{0.83529,0.36863,0.00000}%
\definecolor{mycolor4}{rgb}{0.49400,0.18400,0.55600}%
%
\addplot[only marks, mark=*, mark size=1.1180pt, color=mycolor1, fill=mycolor1, opacity=0.60, draw=none, mark options={draw=none,line width=0pt}] table[row sep=crcr]{%
x	y\\
4448.8	0.99468\\
4495.8	0.98404\\
4542.6	0.9734\\
4584	0.96277\\
4638.1	0.95213\\
4680.6	0.94149\\
4729.4	0.93085\\
4791.3	0.92021\\
4843.1	0.90957\\
4888.5	0.89894\\
4949.8	0.8883\\
5011.5	0.87766\\
5078.7	0.86702\\
5140.6	0.85638\\
5199.1	0.84574\\
5259.8	0.83511\\
5324.2	0.82447\\
5392	0.81383\\
5451.4	0.80319\\
5530.6	0.79255\\
5604.7	0.78191\\
5676.5	0.77128\\
5735.5	0.76064\\
5813.4	0.75\\
5888	0.73936\\
5972.3	0.72872\\
6040.5	0.71809\\
6135.5	0.70745\\
6219	0.69681\\
6301.4	0.68617\\
6379.4	0.67553\\
6489.5	0.66489\\
6580	0.65426\\
6680.1	0.64362\\
6784.4	0.63298\\
6879.3	0.62234\\
6999.8	0.6117\\
7088.5	0.60106\\
7211.9	0.59043\\
7315.2	0.57979\\
7430.2	0.56915\\
7594.1	0.55851\\
7737	0.54787\\
7880	0.53723\\
8036.3	0.5266\\
8163.1	0.51596\\
8296.5	0.50532\\
8423.4	0.49468\\
8580.8	0.48404\\
8734.8	0.4734\\
8903.6	0.46277\\
9095.9	0.45213\\
9259.6	0.44149\\
9466.5	0.43085\\
9627.4	0.42021\\
9822	0.40957\\
10012	0.39894\\
10230	0.3883\\
10440	0.37766\\
10691	0.36702\\
10922	0.35638\\
11164	0.34574\\
11407	0.33511\\
11670	0.32447\\
11947	0.31383\\
12239	0.30319\\
12557	0.29255\\
12855	0.28191\\
13179	0.27128\\
13570	0.26064\\
13919	0.25\\
14359	0.23936\\
14747	0.22872\\
15113	0.21809\\
15556	0.20745\\
16118	0.19681\\
16666	0.18617\\
17343	0.17553\\
17930	0.16489\\
18664	0.15426\\
19408	0.14362\\
20220	0.13298\\
21201	0.12234\\
22218	0.1117\\
23579	0.10106\\
24805	0.090426\\
26355	0.079787\\
28038	0.069149\\
31068	0.058511\\
34651	0.047872\\
39510	0.037234\\
46003	0.026596\\
56575	0.015957\\
81621	0.0053191\\
};
\addlegendentry{FRC empirica esatta}

\addplot [color=mycolor1, only marks, every error bar/.append style={opacity=0.30}, mark=*, mark size=0pt, draw=none, forget plot]
 plot [error bars/.cd, x dir=both, x explicit, error bar style={line width=1pt, color=mycolor1}, error mark options={mark=none,mark size=0pt}]
 table[row sep=crcr, x error plus index=2, x error minus index=3]{%
4448.8	0.99468	36.556	36.556\\
4495.8	0.98404	37.517	37.517\\
4542.6	0.9734	36.289	36.289\\
4584	0.96277	39.036	39.036\\
4638.1	0.95213	39.975	39.975\\
4680.6	0.94149	38.656	38.656\\
4729.4	0.93085	37.188	37.188\\
4791.3	0.92021	39.039	39.039\\
4843.1	0.90957	39.99	39.99\\
4888.5	0.89894	39.87	39.87\\
4949.8	0.8883	40.836	40.836\\
5011.5	0.87766	42.304	42.304\\
5078.7	0.86702	44.043	44.043\\
5140.6	0.85638	45.555	45.555\\
5199.1	0.84574	46.546	46.546\\
5259.8	0.83511	48.558	48.558\\
5324.2	0.82447	51.85	51.85\\
5392	0.81383	54.309	54.309\\
5451.4	0.80319	54.052	54.052\\
5530.6	0.79255	57.171	57.171\\
5604.7	0.78191	56.471	56.471\\
5676.5	0.77128	59.211	59.211\\
5735.5	0.76064	59.446	59.446\\
5813.4	0.75	63.624	63.624\\
5888	0.73936	63.978	63.978\\
5972.3	0.72872	68.81	68.81\\
6040.5	0.71809	72.094	72.094\\
6135.5	0.70745	75.986	75.986\\
6219	0.69681	76.447	76.447\\
6301.4	0.68617	78.64	78.64\\
6379.4	0.67553	81.78	81.78\\
6489.5	0.66489	84.094	84.094\\
6580	0.65426	85.688	85.688\\
6680.1	0.64362	87.716	87.716\\
6784.4	0.63298	89.416	89.416\\
6879.3	0.62234	88.855	88.855\\
6999.8	0.6117	93.364	93.364\\
7088.5	0.60106	93.401	93.401\\
7211.9	0.59043	95.147	95.147\\
7315.2	0.57979	96.001	96.001\\
7430.2	0.56915	96.668	96.668\\
7594.1	0.55851	98.875	98.875\\
7737	0.54787	100.12	100.12\\
7880	0.53723	102.08	102.08\\
8036.3	0.5266	110.95	110.95\\
8163.1	0.51596	110.26	110.26\\
8296.5	0.50532	113.92	113.92\\
8423.4	0.49468	113.62	113.62\\
8580.8	0.48404	111.99	111.99\\
8734.8	0.4734	110.02	110.02\\
8903.6	0.46277	114.81	114.81\\
9095.9	0.45213	111.92	111.92\\
9259.6	0.44149	109.77	109.77\\
9466.5	0.43085	116.35	116.35\\
9627.4	0.42021	115.04	115.04\\
9822	0.40957	119.95	119.95\\
10012	0.39894	123.64	123.64\\
10230	0.3883	130.29	130.29\\
10440	0.37766	131.46	131.46\\
10691	0.36702	138.43	138.43\\
10922	0.35638	129.17	129.17\\
11164	0.34574	134.78	134.78\\
11407	0.33511	126.73	126.73\\
11670	0.32447	140.53	140.53\\
11947	0.31383	150.71	150.71\\
12239	0.30319	152.52	152.52\\
12557	0.29255	153.84	153.84\\
12855	0.28191	167.86	167.86\\
13179	0.27128	176.34	176.34\\
13570	0.26064	189.33	189.33\\
13919	0.25	194.07	194.07\\
14359	0.23936	203.53	203.53\\
14747	0.22872	210.77	210.77\\
15113	0.21809	209.39	209.39\\
15556	0.20745	211.76	211.76\\
16118	0.19681	229.46	229.46\\
16666	0.18617	238.49	238.49\\
17343	0.17553	252.97	252.97\\
17930	0.16489	267.54	267.54\\
18664	0.15426	315.04	315.04\\
19408	0.14362	336.88	336.88\\
20220	0.13298	345.81	345.81\\
21201	0.12234	344.35	344.35\\
22218	0.1117	390.31	390.31\\
23579	0.10106	426.29	426.29\\
24805	0.090426	466.46	466.46\\
26355	0.079787	513.58	513.58\\
28038	0.069149	566.7	566.7\\
31068	0.058511	684.02	684.02\\
34651	0.047872	836.31	836.31\\
39510	0.037234	1174.9	1174.9\\
46003	0.026596	1412.7	1412.7\\
56575	0.015957	1860.4	1860.4\\
81621	0.0053191	3216.4	3216.4\\
};
\addplot [color=mycolor1]
  table[row sep=crcr]{%
4432.9	0.97921\\
4591.6	0.95142\\
4755.9	0.91405\\
7001.9	0.5555\\
16283	0.18779\\
37864	0.063695\\
88052	0.021674\\
1.4406e+05	0.011574\\
};
\addlegendentry{Adatt. Pareto esatto}% ($\mathit{ML}$)
% \addlegendentry{Adatt. Pareto esatto ($\mathit{ML}$)}


\addplot[area legend, draw=none, fill=mycolor1, fill opacity=0.15, forget plot]
table[row sep=crcr] {%
x	y\\
4432.9	0.98491\\
4591.6	0.95938\\
4755.9	0.9227\\
4926.1	0.8817\\
5102.4	0.84232\\
5285	0.80472\\
5474.1	0.76881\\
5670.1	0.73451\\
5873	0.70176\\
6083.2	0.67049\\
6300.9	0.64063\\
6526.4	0.61211\\
6759.9	0.58488\\
7001.9	0.55889\\
7252.5	0.53407\\
7512	0.51037\\
7780.9	0.48774\\
8059.3	0.46614\\
8347.8	0.44552\\
8646.5	0.42582\\
8956	0.40702\\
9276.5	0.38906\\
9608.5	0.37192\\
9952.4	0.35554\\
10309	0.3399\\
10677	0.32495\\
11060	0.31067\\
11455	0.29703\\
11865	0.284\\
12290	0.27154\\
12730	0.25963\\
13185	0.24825\\
13657	0.23738\\
14146	0.22698\\
14652	0.21705\\
15177	0.20755\\
15720	0.19846\\
16283	0.18978\\
16865	0.18148\\
17469	0.17355\\
18094	0.16596\\
18742	0.15871\\
19412	0.15177\\
20107	0.14514\\
20827	0.1388\\
21572	0.13274\\
22344	0.12695\\
23144	0.1214\\
23972	0.1161\\
24830	0.11104\\
25719	0.10619\\
26639	0.10156\\
27592	0.097129\\
28580	0.092892\\
29603	0.088841\\
30662	0.084967\\
31760	0.081263\\
32896	0.07772\\
34074	0.074333\\
35293	0.071093\\
36556	0.067995\\
37864	0.065033\\
39220	0.0622\\
40623	0.05949\\
42077	0.0569\\
43583	0.054422\\
45143	0.052052\\
46758	0.049786\\
48432	0.047619\\
50165	0.045546\\
51960	0.043564\\
53820	0.041668\\
55746	0.039855\\
57741	0.038122\\
59808	0.036463\\
61948	0.034878\\
64165	0.033361\\
66461	0.03191\\
68840	0.030523\\
71304	0.029196\\
73856	0.027927\\
76499	0.026713\\
79237	0.025552\\
82072	0.024442\\
85010	0.02338\\
88052	0.022365\\
91203	0.021393\\
94467	0.020464\\
97848	0.019575\\
1.0135e+05	0.018725\\
1.0498e+05	0.017913\\
1.0873e+05	0.017135\\
1.1263e+05	0.016391\\
1.1666e+05	0.01568\\
1.2083e+05	0.014999\\
1.2516e+05	0.014349\\
1.2964e+05	0.013726\\
1.3427e+05	0.013131\\
1.3908e+05	0.012561\\
1.4406e+05	0.012016\\
1.4406e+05	0.011131\\
1.3908e+05	0.011646\\
1.3427e+05	0.012186\\
1.2964e+05	0.01275\\
1.2516e+05	0.01334\\
1.2083e+05	0.013958\\
1.1666e+05	0.014605\\
1.1263e+05	0.015281\\
1.0873e+05	0.015989\\
1.0498e+05	0.01673\\
1.0135e+05	0.017505\\
97848	0.018316\\
94467	0.019165\\
91203	0.020054\\
88052	0.020983\\
85010	0.021956\\
82072	0.022974\\
79237	0.02404\\
76499	0.025155\\
73856	0.026322\\
71304	0.027543\\
68840	0.028821\\
66461	0.030158\\
64165	0.031558\\
61948	0.033022\\
59808	0.034555\\
57741	0.036159\\
55746	0.037838\\
53820	0.039595\\
51960	0.041434\\
50165	0.043359\\
48432	0.045373\\
46758	0.047481\\
45143	0.049687\\
43583	0.051995\\
42077	0.054412\\
40623	0.056941\\
39220	0.059588\\
37864	0.062358\\
36556	0.065257\\
35293	0.068291\\
34074	0.071467\\
32896	0.074791\\
31760	0.078269\\
30662	0.08191\\
29603	0.085721\\
28580	0.089709\\
27592	0.093883\\
26639	0.098252\\
25719	0.10283\\
24830	0.10761\\
23972	0.11262\\
23144	0.11786\\
22344	0.12335\\
21572	0.12909\\
20827	0.1351\\
20107	0.1414\\
19412	0.14798\\
18742	0.15487\\
18094	0.16208\\
17469	0.16963\\
16865	0.17753\\
16283	0.1858\\
15720	0.19446\\
15177	0.20351\\
14652	0.21299\\
14146	0.22291\\
13657	0.23329\\
13185	0.24416\\
12730	0.25552\\
12290	0.26742\\
11865	0.27986\\
11455	0.29288\\
11060	0.3065\\
10677	0.32075\\
10309	0.33566\\
9952.4	0.35125\\
9608.5	0.36755\\
9276.5	0.3846\\
8956	0.40243\\
8646.5	0.42107\\
8347.8	0.44056\\
8059.3	0.46093\\
7780.9	0.48223\\
7512	0.5045\\
7252.5	0.52778\\
7001.9	0.55212\\
6759.9	0.57756\\
6526.4	0.60416\\
6300.9	0.63198\\
6083.2	0.66106\\
5873	0.69148\\
5670.1	0.72328\\
5474.1	0.75653\\
5285	0.79131\\
5102.4	0.82768\\
4926.1	0.86572\\
4755.9	0.9054\\
4591.6	0.94347\\
4432.9	0.97351\\
}--cycle;
\addplot[only marks, mark=*, mark size=1.1180pt, color=mycolor3, fill=mycolor3, opacity=0.60, draw=none, mark options={draw=none,line width=0pt}] table[row sep=crcr]{%
x	y\\
4422.1	0.99468\\
4464.8	0.98404\\
4510.7	0.9734\\
4562.7	0.96277\\
4609.4	0.95213\\
4660.3	0.94149\\
4703.8	0.93085\\
4764.5	0.92021\\
4823.4	0.90957\\
4878.8	0.89894\\
4931.1	0.8883\\
4981.5	0.87766\\
5032.7	0.86702\\
5090.6	0.85638\\
5147.3	0.84574\\
5204.1	0.83511\\
5268.9	0.82447\\
5327.5	0.81383\\
5396.1	0.80319\\
5465.4	0.79255\\
5528.8	0.78191\\
5611.7	0.77128\\
5684.8	0.76064\\
5757.3	0.75\\
5826.3	0.73936\\
5907.5	0.72872\\
5985.8	0.71809\\
6067.1	0.70745\\
6146.4	0.69681\\
6223.3	0.68617\\
6317	0.67553\\
6420.1	0.66489\\
6512	0.65426\\
6617.6	0.64362\\
6721.5	0.63298\\
6820	0.62234\\
6914.7	0.6117\\
7008.4	0.60106\\
7109.9	0.59043\\
7218	0.57979\\
7331.2	0.56915\\
7454.3	0.55851\\
7588.2	0.54787\\
7729.2	0.53723\\
7865.4	0.5266\\
7997.5	0.51596\\
8145.7	0.50532\\
8288	0.49468\\
8465.9	0.48404\\
8612.1	0.4734\\
8772.6	0.46277\\
8971.5	0.45213\\
9149.8	0.44149\\
9376.5	0.43085\\
9583	0.42021\\
9775.6	0.40957\\
9973.9	0.39894\\
10173	0.3883\\
10378	0.37766\\
10624	0.36702\\
10863	0.35638\\
11083	0.34574\\
11390	0.33511\\
11621	0.32447\\
11887	0.31383\\
12143	0.30319\\
12426	0.29255\\
12689	0.28191\\
13007	0.27128\\
13315	0.26064\\
13699	0.25\\
14073	0.23936\\
14500	0.22872\\
14961	0.21809\\
15478	0.20745\\
16061	0.19681\\
16604	0.18617\\
17140	0.17553\\
17638	0.16489\\
18376	0.15426\\
19009	0.14362\\
20015	0.13298\\
21053	0.12234\\
22307	0.1117\\
23541	0.10106\\
25138	0.090426\\
26542	0.079787\\
28800	0.069149\\
31358	0.058511\\
34949	0.047872\\
39886	0.037234\\
46309	0.026596\\
57859	0.015957\\
86479	0.0053191\\
};
\addlegendentry{FRC empirica appr.}

\addplot [color=mycolor3, only marks, every error bar/.append style={opacity=0.30}, mark=*, mark size=0pt, draw=none, forget plot]
 plot [error bars/.cd, x dir=both, x explicit, error bar style={line width=1pt, color=mycolor3}, error mark options={mark=none,mark size=0pt}]
 table[row sep=crcr, x error plus index=2, x error minus index=3]{%
4422.1	0.99468	48.059	48.059\\
4464.8	0.98404	49.114	49.114\\
4510.7	0.9734	49.637	49.637\\
4562.7	0.96277	51.66	51.66\\
4609.4	0.95213	51.566	51.566\\
4660.3	0.94149	50.283	50.283\\
4703.8	0.93085	51.069	51.069\\
4764.5	0.92021	52.715	52.715\\
4823.4	0.90957	53.485	53.485\\
4878.8	0.89894	54.86	54.86\\
4931.1	0.8883	55.763	55.763\\
4981.5	0.87766	54.275	54.275\\
5032.7	0.86702	54.846	54.846\\
5090.6	0.85638	56.572	56.572\\
5147.3	0.84574	56.516	56.516\\
5204.1	0.83511	56.688	56.688\\
5268.9	0.82447	56.043	56.043\\
5327.5	0.81383	55.478	55.478\\
5396.1	0.80319	59.489	59.489\\
5465.4	0.79255	60.152	60.152\\
5528.8	0.78191	59.725	59.725\\
5611.7	0.77128	62.487	62.487\\
5684.8	0.76064	65.931	65.931\\
5757.3	0.75	67.565	67.565\\
5826.3	0.73936	68.645	68.645\\
5907.5	0.72872	69.665	69.665\\
5985.8	0.71809	70.821	70.821\\
6067.1	0.70745	70.517	70.517\\
6146.4	0.69681	71.213	71.213\\
6223.3	0.68617	71.633	71.633\\
6317	0.67553	70.716	70.716\\
6420.1	0.66489	75.234	75.234\\
6512	0.65426	75.997	75.997\\
6617.6	0.64362	76.277	76.277\\
6721.5	0.63298	80.82	80.82\\
6820	0.62234	82.177	82.177\\
6914.7	0.6117	86.711	86.711\\
7008.4	0.60106	88.348	88.348\\
7109.9	0.59043	88.354	88.354\\
7218	0.57979	89.402	89.402\\
7331.2	0.56915	89.672	89.672\\
7454.3	0.55851	86.852	86.852\\
7588.2	0.54787	90.66	90.66\\
7729.2	0.53723	92.844	92.844\\
7865.4	0.5266	93.134	93.134\\
7997.5	0.51596	95.207	95.207\\
8145.7	0.50532	97.355	97.355\\
8288	0.49468	101.71	101.71\\
8465.9	0.48404	105.09	105.09\\
8612.1	0.4734	106.08	106.08\\
8772.6	0.46277	109.61	109.61\\
8971.5	0.45213	116.49	116.49\\
9149.8	0.44149	117.94	117.94\\
9376.5	0.43085	116.97	116.97\\
9583	0.42021	122.95	122.95\\
9775.6	0.40957	122.73	122.73\\
9973.9	0.39894	122.87	122.87\\
10173	0.3883	128.39	128.39\\
10378	0.37766	133.59	133.59\\
10624	0.36702	142.75	142.75\\
10863	0.35638	150.6	150.6\\
11083	0.34574	145.08	145.08\\
11390	0.33511	149.72	149.72\\
11621	0.32447	147.49	147.49\\
11887	0.31383	154.6	154.6\\
12143	0.30319	155.93	155.93\\
12426	0.29255	166.44	166.44\\
12689	0.28191	178.08	178.08\\
13007	0.27128	180.81	180.81\\
13315	0.26064	181.05	181.05\\
13699	0.25	191.07	191.07\\
14073	0.23936	183.77	183.77\\
14500	0.22872	188.27	188.27\\
14961	0.21809	198.08	198.08\\
15478	0.20745	236.2	236.2\\
16061	0.19681	222.1	222.1\\
16604	0.18617	235.41	235.41\\
17140	0.17553	260.77	260.77\\
17638	0.16489	294.49	294.49\\
18376	0.15426	304.14	304.14\\
19009	0.14362	331.39	331.39\\
20015	0.13298	344.51	344.51\\
21053	0.12234	391.85	391.85\\
22307	0.1117	433.47	433.47\\
23541	0.10106	465.71	465.71\\
25138	0.090426	519.35	519.35\\
26542	0.079787	530.6	530.6\\
28800	0.069149	660.25	660.25\\
31358	0.058511	794.32	794.32\\
34949	0.047872	1018.4	1018.4\\
39886	0.037234	1187.3	1187.3\\
46309	0.026596	1605	1605\\
57859	0.015957	2254	2254\\
86479	0.0053191	4199	4199\\
};
\addplot [color=mycolor3]
  table[row sep=crcr]{%
4415	0.97229\\
4573.2	0.94455\\
4737.1	0.90915\\
4906.8	0.87129\\
5648.8	0.727\\
11422	0.2931\\
23095	0.11856\\
48370	0.045986\\
1.0131e+05	0.017897\\
1.4406e+05	0.011431\\
};
\addlegendentry{Adatt. Pareto appr.}% ($\mathit{ML}$)
% \addlegendentry{Adatt. Pareto appr. ($\mathit{ML}$)}


\addplot[area legend, draw=none, fill=mycolor3, fill opacity=0.15, forget plot]
table[row sep=crcr] {%
x	y\\
4415	0.97971\\
4573.2	0.95407\\
4737.1	0.91967\\
4906.8	0.88194\\
5082.6	0.84311\\
5264.7	0.80569\\
5453.4	0.76947\\
5648.8	0.73488\\
5851.2	0.70187\\
6060.8	0.67035\\
6278	0.64026\\
6503	0.61154\\
6736	0.58412\\
6977.3	0.55795\\
7227.3	0.53297\\
7486.3	0.50912\\
7754.5	0.48636\\
8032.4	0.46464\\
8320.2	0.44391\\
8618.3	0.42413\\
8927.1	0.40525\\
9247	0.38723\\
9578.3	0.37004\\
9921.5	0.35363\\
10277	0.33797\\
10645	0.32302\\
11027	0.30875\\
11422	0.29512\\
11831	0.28211\\
12255	0.26968\\
12694	0.25781\\
13149	0.24647\\
13620	0.23564\\
14108	0.22529\\
14614	0.2154\\
15137	0.20595\\
15680	0.19692\\
16241	0.18829\\
16823	0.18004\\
17426	0.17215\\
18051	0.16461\\
18697	0.1574\\
19367	0.15052\\
20061	0.14393\\
20780	0.13763\\
21525	0.13162\\
22296	0.12586\\
23095	0.12036\\
23922	0.1151\\
24779	0.11007\\
25667	0.10526\\
26587	0.10067\\
27540	0.096272\\
28526	0.092069\\
29548	0.088051\\
30607	0.084208\\
31704	0.080534\\
32840	0.077021\\
34017	0.073662\\
35235	0.07045\\
36498	0.067378\\
37806	0.064441\\
39160	0.061632\\
40563	0.058947\\
42017	0.056378\\
43522	0.053922\\
45082	0.051574\\
46697	0.049328\\
48370	0.04718\\
50104	0.045126\\
51899	0.043161\\
53758	0.041283\\
55685	0.039487\\
57680	0.037769\\
59747	0.036126\\
61887	0.034554\\
64105	0.033052\\
66402	0.031615\\
68781	0.03024\\
71246	0.028926\\
73798	0.027669\\
76443	0.026466\\
79182	0.025316\\
82019	0.024216\\
84958	0.023165\\
88002	0.022159\\
91155	0.021196\\
94421	0.020276\\
97804	0.019396\\
1.0131e+05	0.018554\\
1.0494e+05	0.017749\\
1.087e+05	0.016979\\
1.1259e+05	0.016242\\
1.1663e+05	0.015538\\
1.2081e+05	0.014864\\
1.2514e+05	0.014219\\
1.2962e+05	0.013603\\
1.3426e+05	0.013013\\
1.3907e+05	0.012449\\
1.4406e+05	0.011909\\
1.4406e+05	0.010953\\
1.3907e+05	0.011461\\
1.3426e+05	0.011992\\
1.2962e+05	0.012548\\
1.2514e+05	0.013131\\
1.2081e+05	0.01374\\
1.1663e+05	0.014377\\
1.1259e+05	0.015045\\
1.087e+05	0.015743\\
1.0494e+05	0.016474\\
1.0131e+05	0.017239\\
97804	0.01804\\
94421	0.018878\\
91155	0.019755\\
88002	0.020672\\
84958	0.021633\\
82019	0.022639\\
79182	0.023691\\
76443	0.024793\\
73798	0.025946\\
71246	0.027153\\
68781	0.028416\\
66402	0.029738\\
64105	0.031122\\
61887	0.03257\\
59747	0.034086\\
57680	0.035673\\
55685	0.037334\\
53758	0.039073\\
51899	0.040893\\
50104	0.042798\\
48370	0.044792\\
46697	0.04688\\
45082	0.049065\\
43522	0.051352\\
42017	0.053747\\
40563	0.056253\\
39160	0.058877\\
37806	0.061623\\
36498	0.064499\\
35235	0.067508\\
34017	0.070659\\
32840	0.073957\\
31704	0.07741\\
30607	0.081024\\
29548	0.084808\\
28526	0.088769\\
27540	0.092915\\
26587	0.097256\\
25667	0.1018\\
24779	0.10656\\
23922	0.11154\\
23095	0.11675\\
22296	0.12221\\
21525	0.12792\\
20780	0.1339\\
20061	0.14016\\
19367	0.14672\\
18697	0.15358\\
18051	0.16076\\
17426	0.16828\\
16823	0.17615\\
16241	0.18439\\
15680	0.19301\\
15137	0.20204\\
14614	0.21148\\
14108	0.22137\\
13620	0.23171\\
13149	0.24254\\
12694	0.25387\\
12255	0.26572\\
11831	0.27812\\
11422	0.29109\\
11027	0.30466\\
10645	0.31884\\
10277	0.33367\\
9921.5	0.34918\\
9578.3	0.3654\\
9247	0.38235\\
8927.1	0.40007\\
8618.3	0.41859\\
8320.2	0.43796\\
8032.4	0.4582\\
7754.5	0.47936\\
7486.3	0.50149\\
7227.3	0.52463\\
6977.3	0.54882\\
6736	0.57412\\
6503	0.60058\\
6278	0.62825\\
6060.8	0.65719\\
5851.2	0.68745\\
5648.8	0.71911\\
5453.4	0.75222\\
5264.7	0.78686\\
5082.6	0.82299\\
4906.8	0.86065\\
4737.1	0.89863\\
4573.2	0.93502\\
4415	0.96486\\
}--cycle;
            % \input{../../../.tex/20/KS/SizeDistributionFittingsfig8.tex}
    
            \nextgroupplot[%
                % xmin=3637.5,xmax=2.0424e+05,
                % ymin=0.0053191,ymax=1,
            ] % This file was created by matlab2tikz.
%
\definecolor{mycolor1}{rgb}{0.00000,0.44706,0.69804}%
\definecolor{mycolor2}{rgb}{0.00000,0.44700,0.74100}%
\definecolor{mycolor3}{rgb}{0.49400,0.18400,0.55600}%
%
\addplot[only marks, mark=*, mark size=1.2500pt, color=mycolor1, fill=mycolor1, opacity=0.60, draw=none, mark options={draw=none,line width=0pt}] table[row sep=crcr]{%
x	y\\
4009.8	0.99468\\
4042.8	0.98404\\
4083.2	0.9734\\
4118.9	0.96277\\
4158.9	0.95213\\
4199.9	0.94149\\
4241.8	0.93085\\
4293.5	0.92021\\
4346.1	0.90957\\
4392.3	0.89894\\
4440.5	0.8883\\
4485	0.87766\\
4529	0.86702\\
4574.2	0.85638\\
4618.5	0.84574\\
4670.1	0.83511\\
4730.3	0.82447\\
4787.1	0.81383\\
4841	0.80319\\
4903.8	0.79255\\
4960.2	0.78191\\
5017.2	0.77128\\
5087.5	0.76064\\
5155.4	0.75\\
5233.1	0.73936\\
5291.8	0.72872\\
5360.5	0.71809\\
5431.2	0.70745\\
5501.1	0.69681\\
5574	0.68617\\
5661.2	0.67553\\
5733.2	0.66489\\
5815.2	0.65426\\
5904.3	0.64362\\
5987.8	0.63298\\
6075.5	0.62234\\
6160.2	0.6117\\
6238.2	0.60106\\
6339.4	0.59043\\
6443.4	0.57979\\
6550.4	0.56915\\
6655.6	0.55851\\
6768.6	0.54787\\
6863	0.53723\\
6984.1	0.5266\\
7105.1	0.51596\\
7243	0.50532\\
7363.9	0.49468\\
7529.2	0.48404\\
7674.5	0.4734\\
7798.1	0.46277\\
7951.1	0.45213\\
8117.5	0.44149\\
8265.8	0.43085\\
8464.1	0.42021\\
8676.5	0.40957\\
8871.5	0.39894\\
9049.7	0.3883\\
9256	0.37766\\
9441.2	0.36702\\
9666.5	0.35638\\
9881.2	0.34574\\
10150	0.33511\\
10431	0.32447\\
10718	0.31383\\
11062	0.30319\\
11293	0.29255\\
11688	0.28191\\
12013	0.27128\\
12316	0.26064\\
12705	0.25\\
13150	0.23936\\
13536	0.22872\\
13995	0.21809\\
14547	0.20745\\
15079	0.19681\\
15757	0.18617\\
16494	0.17553\\
17200	0.16489\\
17972	0.15426\\
18737	0.14362\\
19593	0.13298\\
20816	0.12234\\
22198	0.1117\\
23437	0.10106\\
25016	0.090426\\
27001	0.079787\\
29590	0.069149\\
32563	0.058511\\
36426	0.047872\\
43570	0.037234\\
54044	0.026596\\
68767	0.015957\\
1.4931e+05	0.0053191\\
};
\addlegendentry{FRC empirica esatta}

\addplot [color=mycolor1, only marks, every error bar/.append style={opacity=0.30}, mark=*, mark size=0pt, draw=none, forget plot]
 plot [error bars/.cd, x dir=both, x explicit, error bar style={line width=1pt, color=mycolor1}, error mark options={mark=none,mark size=0pt}]
 table[row sep=crcr, x error plus index=2, x error minus index=3]{%
4009.8	0.99468	41.683	41.683\\
4042.8	0.98404	42.729	42.729\\
4083.2	0.9734	42.64	42.64\\
4118.9	0.96277	42.259	42.259\\
4158.9	0.95213	42.401	42.401\\
4199.9	0.94149	43.247	43.247\\
4241.8	0.93085	44.025	44.025\\
4293.5	0.92021	45.653	45.653\\
4346.1	0.90957	47.232	47.232\\
4392.3	0.89894	47.443	47.443\\
4440.5	0.8883	47.604	47.604\\
4485	0.87766	47.648	47.648\\
4529	0.86702	48.056	48.056\\
4574.2	0.85638	48.712	48.712\\
4618.5	0.84574	48.794	48.794\\
4670.1	0.83511	49.967	49.967\\
4730.3	0.82447	48.825	48.825\\
4787.1	0.81383	49.432	49.432\\
4841	0.80319	50.68	50.68\\
4903.8	0.79255	49.872	49.872\\
4960.2	0.78191	50.875	50.875\\
5017.2	0.77128	51.172	51.172\\
5087.5	0.76064	52.783	52.783\\
5155.4	0.75	53.548	53.548\\
5233.1	0.73936	54.7	54.7\\
5291.8	0.72872	56.659	56.659\\
5360.5	0.71809	57.557	57.557\\
5431.2	0.70745	56.553	56.553\\
5501.1	0.69681	55.696	55.696\\
5574	0.68617	59.561	59.561\\
5661.2	0.67553	62.198	62.198\\
5733.2	0.66489	60.853	60.853\\
5815.2	0.65426	65.634	65.634\\
5904.3	0.64362	63.226	63.226\\
5987.8	0.63298	65.228	65.228\\
6075.5	0.62234	68.996	68.996\\
6160.2	0.6117	69.293	69.293\\
6238.2	0.60106	72.946	72.946\\
6339.4	0.59043	75.51	75.51\\
6443.4	0.57979	74.961	74.961\\
6550.4	0.56915	74.817	74.817\\
6655.6	0.55851	79.725	79.725\\
6768.6	0.54787	82.852	82.852\\
6863	0.53723	81.443	81.443\\
6984.1	0.5266	81.923	81.923\\
7105.1	0.51596	88.89	88.89\\
7243	0.50532	93.199	93.199\\
7363.9	0.49468	97.499	97.499\\
7529.2	0.48404	96.758	96.758\\
7674.5	0.4734	95.477	95.477\\
7798.1	0.46277	93.925	93.925\\
7951.1	0.45213	98.413	98.413\\
8117.5	0.44149	97.624	97.624\\
8265.8	0.43085	101.94	101.94\\
8464.1	0.42021	108.66	108.66\\
8676.5	0.40957	112.81	112.81\\
8871.5	0.39894	110.39	110.39\\
9049.7	0.3883	117.1	117.1\\
9256	0.37766	118.93	118.93\\
9441.2	0.36702	117.23	117.23\\
9666.5	0.35638	114.85	114.85\\
9881.2	0.34574	119.48	119.48\\
10150	0.33511	123.96	123.96\\
10431	0.32447	124.98	124.98\\
10718	0.31383	133.42	133.42\\
11062	0.30319	143.38	143.38\\
11293	0.29255	150.62	150.62\\
11688	0.28191	166.57	166.57\\
12013	0.27128	171.22	171.22\\
12316	0.26064	181.72	181.72\\
12705	0.25	192.57	192.57\\
13150	0.23936	208.87	208.87\\
13536	0.22872	200.05	200.05\\
13995	0.21809	213.59	213.59\\
14547	0.20745	224.95	224.95\\
15079	0.19681	224.48	224.48\\
15757	0.18617	254.54	254.54\\
16494	0.17553	273.77	273.77\\
17200	0.16489	282.21	282.21\\
17972	0.15426	309.96	309.96\\
18737	0.14362	340.73	340.73\\
19593	0.13298	351.38	351.38\\
20816	0.12234	385	385\\
22198	0.1117	402.42	402.42\\
23437	0.10106	424.8	424.8\\
25016	0.090426	456.58	456.58\\
27001	0.079787	560.02	560.02\\
29590	0.069149	636.25	636.25\\
32563	0.058511	797.84	797.84\\
36426	0.047872	902.21	902.21\\
43570	0.037234	1186.4	1186.4\\
54044	0.026596	1844.3	1844.3\\
68767	0.015957	2565.5	2565.5\\
1.4931e+05	0.0053191	4008.6	4008.6\\
};
\addplot [color=mycolor1]
  table[row sep=crcr]{%
3997.8	0.97494\\
4159.8	0.94331\\
4328.4	0.90397\\
4686.4	0.81967\\
10374	0.30231\\
22965	0.1119\\
52898	0.039557\\
1.2185e+05	0.014037\\
2.0424e+05	0.0074052\\
};
\addlegendentry{Adatt. Pareto esatto ($\mathit{ML}$)}


\addplot[area legend, draw=none, fill=mycolor1, fill opacity=0.15, forget plot]
table[row sep=crcr] {%
x	y\\
3997.8	0.98138\\
4159.8	0.95214\\
4328.4	0.91395\\
4503.8	0.87112\\
4686.4	0.82863\\
4876.4	0.78787\\
5074	0.74913\\
5279.7	0.7123\\
5493.7	0.67731\\
5716.3	0.64405\\
5948	0.61244\\
6189.1	0.58241\\
6440	0.55387\\
6701	0.52675\\
6972.6	0.50098\\
7255.3	0.4765\\
7549.3	0.45324\\
7855.3	0.43114\\
8173.7	0.41014\\
8505	0.39019\\
8849.8	0.37124\\
9208.5	0.35323\\
9581.7	0.33612\\
9970.1	0.31985\\
10374	0.30439\\
10795	0.28969\\
11232	0.27571\\
11687	0.26242\\
12161	0.24977\\
12654	0.23775\\
13167	0.22631\\
13701	0.21542\\
14256	0.20507\\
14834	0.19522\\
15435	0.18584\\
16061	0.17692\\
16712	0.16843\\
17389	0.16034\\
18094	0.15265\\
18827	0.14533\\
19591	0.13837\\
20385	0.13173\\
21211	0.12542\\
22071	0.11941\\
22965	0.1137\\
23896	0.10825\\
24865	0.10307\\
25872	0.098136\\
26921	0.09344\\
28012	0.088969\\
29148	0.084713\\
30329	0.080662\\
31558	0.076805\\
32838	0.073134\\
34169	0.069638\\
35553	0.06631\\
36995	0.063142\\
38494	0.060125\\
40054	0.057254\\
41678	0.05452\\
43367	0.051917\\
45125	0.049438\\
46954	0.047079\\
48857	0.044832\\
50837	0.042693\\
52898	0.040656\\
55042	0.038717\\
57273	0.036871\\
59595	0.035113\\
62010	0.033439\\
64524	0.031845\\
67139	0.030327\\
69860	0.028882\\
72692	0.027506\\
75638	0.026196\\
78704	0.024948\\
81894	0.02376\\
85213	0.022629\\
88667	0.021552\\
92261	0.020526\\
96001	0.019549\\
99892	0.018619\\
1.0394e+05	0.017734\\
1.0815e+05	0.01689\\
1.1254e+05	0.016087\\
1.171e+05	0.015322\\
1.2185e+05	0.014594\\
1.2678e+05	0.0139\\
1.3192e+05	0.013239\\
1.3727e+05	0.01261\\
1.4283e+05	0.012011\\
1.4862e+05	0.01144\\
1.5465e+05	0.010897\\
1.6092e+05	0.01038\\
1.6744e+05	0.0098868\\
1.7423e+05	0.0094174\\
1.8129e+05	0.0089704\\
1.8864e+05	0.0085447\\
1.9628e+05	0.0081392\\
2.0424e+05	0.0077531\\
2.0424e+05	0.0070573\\
1.9628e+05	0.0074172\\
1.8864e+05	0.0077955\\
1.8129e+05	0.0081931\\
1.7423e+05	0.0086111\\
1.6744e+05	0.0090505\\
1.6092e+05	0.0095124\\
1.5465e+05	0.0099979\\
1.4862e+05	0.010508\\
1.4283e+05	0.011045\\
1.3727e+05	0.011609\\
1.3192e+05	0.012202\\
1.2678e+05	0.012825\\
1.2185e+05	0.013481\\
1.171e+05	0.014169\\
1.1254e+05	0.014894\\
1.0815e+05	0.015655\\
1.0394e+05	0.016455\\
99892	0.017297\\
96001	0.018182\\
92261	0.019112\\
88667	0.020089\\
85213	0.021117\\
81894	0.022198\\
78704	0.023334\\
75638	0.024529\\
72692	0.025785\\
69860	0.027106\\
67139	0.028494\\
64524	0.029954\\
62010	0.031489\\
59595	0.033102\\
57273	0.034799\\
55042	0.036583\\
52898	0.038459\\
50837	0.040431\\
48857	0.042505\\
46954	0.044685\\
45125	0.046978\\
43367	0.049388\\
41678	0.051923\\
40054	0.054588\\
38494	0.057391\\
36995	0.060338\\
35553	0.063436\\
34169	0.066694\\
32838	0.070121\\
31558	0.073723\\
30329	0.077512\\
29148	0.081495\\
28012	0.085684\\
26921	0.090089\\
25872	0.094721\\
24865	0.099591\\
23896	0.10471\\
22965	0.1101\\
22071	0.11576\\
21211	0.12172\\
20385	0.12798\\
19591	0.13457\\
18827	0.14149\\
18094	0.14877\\
17389	0.15643\\
16712	0.16448\\
16061	0.17294\\
15435	0.18185\\
14834	0.1912\\
14256	0.20104\\
13701	0.21139\\
13167	0.22227\\
12654	0.2337\\
12161	0.24572\\
11687	0.25835\\
11232	0.27163\\
10795	0.28558\\
10374	0.30024\\
9970.1	0.31564\\
9581.7	0.33182\\
9208.5	0.34882\\
8849.8	0.36666\\
8505	0.38541\\
8173.7	0.40509\\
7855.3	0.42575\\
7549.3	0.44745\\
7255.3	0.47023\\
6972.6	0.49416\\
6701	0.51928\\
6440	0.54566\\
6189.1	0.57338\\
5948	0.60248\\
5716.3	0.63305\\
5493.7	0.66517\\
5279.7	0.69891\\
5074	0.73435\\
4876.4	0.77158\\
4686.4	0.8107\\
4503.8	0.8517\\
4328.4	0.89399\\
4159.8	0.93448\\
3997.8	0.96849\\
}--cycle;
\addplot[only marks, mark=*, mark size=1.2500pt, color=black, fill=black, opacity=0.60, draw=none, mark options={draw=none,line width=0pt}] table[row sep=crcr]{%
x	y\\
3650	0.99468\\
3735	0.98404\\
3741	0.9734\\
3761	0.96277\\
3772	0.95213\\
3785	0.94149\\
3825	0.93085\\
3847	0.92021\\
3900	0.90957\\
3928	0.89894\\
3945	0.8883\\
3996	0.87766\\
4009	0.86702\\
4024	0.85638\\
4061	0.84574\\
4061	0.83511\\
4071	0.82447\\
4113	0.81383\\
4228	0.80319\\
4348	0.79255\\
4430	0.78191\\
4503	0.77128\\
4539	0.76064\\
4632	0.75\\
4654	0.73936\\
4663	0.72872\\
4679	0.71809\\
4702	0.70745\\
4779	0.69681\\
4861	0.68617\\
4953	0.67553\\
5029	0.66489\\
5048	0.65426\\
5236	0.64362\\
5264	0.63298\\
5327	0.62234\\
5379	0.6117\\
5458	0.60106\\
5463	0.59043\\
5524	0.57979\\
5607	0.56915\\
5857	0.55851\\
5868	0.54787\\
5982	0.53723\\
6317	0.5266\\
6332	0.51596\\
6342	0.50532\\
6356	0.49468\\
6367	0.48404\\
6468	0.4734\\
6629	0.46277\\
6926	0.45213\\
7252	0.44149\\
7294	0.43085\\
7320	0.42021\\
7348	0.40957\\
7596	0.39894\\
7724	0.3883\\
7877	0.37766\\
8035	0.36702\\
8499	0.35638\\
8518	0.34574\\
8994	0.33511\\
9128	0.32447\\
9267	0.31383\\
9435	0.30319\\
9599	0.29255\\
9837	0.28191\\
10119	0.27128\\
10336	0.26064\\
10377	0.25\\
11048	0.23936\\
11424	0.22872\\
11830	0.21809\\
12182	0.20745\\
12313	0.19681\\
13086	0.18617\\
13380	0.17553\\
13398	0.16489\\
13899	0.15426\\
14984	0.14362\\
16428	0.13298\\
20491	0.12234\\
21264	0.1117\\
23237	0.10106\\
30134	0.090426\\
30990	0.079787\\
32887	0.069149\\
37527	0.058511\\
39026	0.047872\\
41095	0.037234\\
61636	0.026596\\
1.2234e+05	0.015957\\
2.0424e+05	0.0053191\\
};
\addlegendentry{FRC empirica esatta}

\addplot [color=black]
  table[row sep=crcr]{%
3637.5	1\\
2.0424e+05	0.005939\\
};
\addlegendentry{Adatt. Pareto esatto ($\mathit{ML}$)}

            % \input{../../../.tex/20/KS/SizeDistributionFittingsfig9.tex}
    
            \nextgroupplot[%
                % xmin=3637.5,xmax=2.5003e+05,
                % ymin=0.0045909,ymax=1,
            ] % This file was created by matlab2tikz.
%
\definecolor{mycolor1}{rgb}{0.83529,0.36863,0.00000}%
\definecolor{mycolor2}{rgb}{0.00000,0.44700,0.74100}%
\definecolor{mycolor3}{rgb}{0.49400,0.18400,0.55600}%
%
\addplot[only marks, mark=*, mark size=1.2500pt, color=mycolor1, fill=mycolor1, opacity=0.60, draw=none, mark options={draw=none,line width=0pt}] table[row sep=crcr]{%
x	y\\
4026.3	0.99468\\
4058.6	0.98404\\
4099.8	0.9734\\
4144	0.96277\\
4185.7	0.95213\\
4222.6	0.94149\\
4258.8	0.93085\\
4294	0.92021\\
4338.8	0.90957\\
4387.5	0.89894\\
4434.4	0.8883\\
4477.8	0.87766\\
4531.1	0.86702\\
4583.2	0.85638\\
4627.9	0.84574\\
4689.2	0.83511\\
4733	0.82447\\
4787	0.81383\\
4843.5	0.80319\\
4896.6	0.79255\\
4951.8	0.78191\\
5010.6	0.77128\\
5072.2	0.76064\\
5133.1	0.75\\
5190.8	0.73936\\
5257.5	0.72872\\
5330.8	0.71809\\
5395.5	0.70745\\
5462.9	0.69681\\
5544.5	0.68617\\
5620.8	0.67553\\
5702.2	0.66489\\
5790.5	0.65426\\
5881.8	0.64362\\
5964	0.63298\\
6028.3	0.62234\\
6133.8	0.6117\\
6237	0.60106\\
6320	0.59043\\
6406.8	0.57979\\
6506.9	0.56915\\
6616.8	0.55851\\
6722.3	0.54787\\
6828.5	0.53723\\
6953.2	0.5266\\
7088.7	0.51596\\
7213.5	0.50532\\
7325.9	0.49468\\
7446.2	0.48404\\
7581.1	0.4734\\
7724.5	0.46277\\
7861.5	0.45213\\
8007	0.44149\\
8164.3	0.43085\\
8339.4	0.42021\\
8505.8	0.40957\\
8697.7	0.39894\\
8854.2	0.3883\\
9063	0.37766\\
9293.7	0.36702\\
9542.3	0.35638\\
9795.1	0.34574\\
10050	0.33511\\
10298	0.32447\\
10572	0.31383\\
10877	0.30319\\
11188	0.29255\\
11456	0.28191\\
11809	0.27128\\
12134	0.26064\\
12539	0.25\\
12911	0.23936\\
13427	0.22872\\
13974	0.21809\\
14460	0.20745\\
15013	0.19681\\
15718	0.18617\\
16449	0.17553\\
17281	0.16489\\
18072	0.15426\\
18837	0.14362\\
19847	0.13298\\
20821	0.12234\\
22193	0.1117\\
23455	0.10106\\
24972	0.090426\\
26751	0.079787\\
29026	0.069149\\
31699	0.058511\\
35825	0.047872\\
42313	0.037234\\
53000	0.026596\\
72389	0.015957\\
1.5403e+05	0.0053191\\
};
\addlegendentry{FRC empirica appr.}

\addplot [color=mycolor1, only marks, every error bar/.append style={opacity=0.30}, mark=*, mark size=0pt, draw=none, forget plot]
 plot [error bars/.cd, x dir=both, x explicit, error bar style={line width=1pt, color=mycolor1}, error mark options={mark=none,mark size=0pt}]
 table[row sep=crcr, x error plus index=2, x error minus index=3]{%
4026.3	0.99468	36.588	36.588\\
4058.6	0.98404	35.959	35.959\\
4099.8	0.9734	36.838	36.838\\
4144	0.96277	37.465	37.465\\
4185.7	0.95213	38.893	38.893\\
4222.6	0.94149	39.784	39.784\\
4258.8	0.93085	40.702	40.702\\
4294	0.92021	40.402	40.402\\
4338.8	0.90957	39.037	39.037\\
4387.5	0.89894	40.059	40.059\\
4434.4	0.8883	39.704	39.704\\
4477.8	0.87766	40.479	40.479\\
4531.1	0.86702	41.553	41.553\\
4583.2	0.85638	42.897	42.897\\
4627.9	0.84574	42.524	42.524\\
4689.2	0.83511	46.655	46.655\\
4733	0.82447	48.383	48.383\\
4787	0.81383	48.919	48.919\\
4843.5	0.80319	48.945	48.945\\
4896.6	0.79255	49.791	49.791\\
4951.8	0.78191	49.148	49.148\\
5010.6	0.77128	47.503	47.503\\
5072.2	0.76064	48.508	48.508\\
5133.1	0.75	50.07	50.07\\
5190.8	0.73936	49.206	49.206\\
5257.5	0.72872	50.721	50.721\\
5330.8	0.71809	50.264	50.264\\
5395.5	0.70745	50.129	50.129\\
5462.9	0.69681	50.129	50.129\\
5544.5	0.68617	52.551	52.551\\
5620.8	0.67553	53.404	53.404\\
5702.2	0.66489	54.765	54.765\\
5790.5	0.65426	55.651	55.651\\
5881.8	0.64362	58.065	58.065\\
5964	0.63298	61.673	61.673\\
6028.3	0.62234	61.598	61.598\\
6133.8	0.6117	62.325	62.325\\
6237	0.60106	65.704	65.704\\
6320	0.59043	68.439	68.439\\
6406.8	0.57979	67.477	67.477\\
6506.9	0.56915	74.772	74.772\\
6616.8	0.55851	75.741	75.741\\
6722.3	0.54787	78.334	78.334\\
6828.5	0.53723	80.276	80.276\\
6953.2	0.5266	85.433	85.433\\
7088.7	0.51596	85.646	85.646\\
7213.5	0.50532	87.296	87.296\\
7325.9	0.49468	90.363	90.363\\
7446.2	0.48404	89.834	89.834\\
7581.1	0.4734	87.181	87.181\\
7724.5	0.46277	85.579	85.579\\
7861.5	0.45213	91.022	91.022\\
8007	0.44149	89.981	89.981\\
8164.3	0.43085	95.299	95.299\\
8339.4	0.42021	90.558	90.558\\
8505.8	0.40957	97.967	97.967\\
8697.7	0.39894	97.569	97.569\\
8854.2	0.3883	97.36	97.36\\
9063	0.37766	101.82	101.82\\
9293.7	0.36702	108.12	108.12\\
9542.3	0.35638	116.55	116.55\\
9795.1	0.34574	116.34	116.34\\
10050	0.33511	122.98	122.98\\
10298	0.32447	127.97	127.97\\
10572	0.31383	139.11	139.11\\
10877	0.30319	147.36	147.36\\
11188	0.29255	147.95	147.95\\
11456	0.28191	158.48	158.48\\
11809	0.27128	170.86	170.86\\
12134	0.26064	180.39	180.39\\
12539	0.25	194.84	194.84\\
12911	0.23936	200.17	200.17\\
13427	0.22872	209.38	209.38\\
13974	0.21809	224.5	224.5\\
14460	0.20745	234.84	234.84\\
15013	0.19681	233.15	233.15\\
15718	0.18617	258.63	258.63\\
16449	0.17553	271.52	271.52\\
17281	0.16489	294.36	294.36\\
18072	0.15426	345.21	345.21\\
18837	0.14362	336.44	336.44\\
19847	0.13298	382.5	382.5\\
20821	0.12234	385.99	385.99\\
22193	0.1117	410.26	410.26\\
23455	0.10106	437.73	437.73\\
24972	0.090426	505.73	505.73\\
26751	0.079787	587.47	587.47\\
29026	0.069149	684.85	684.85\\
31699	0.058511	813.07	813.07\\
35825	0.047872	967.25	967.25\\
42313	0.037234	1232	1232\\
53000	0.026596	1956	1956\\
72389	0.015957	2969.7	2969.7\\
1.5403e+05	0.0053191	5124.6	5124.6\\
};
\addplot [color=mycolor1]
  table[row sep=crcr]{%
4007.8	0.97865\\
4178.6	0.94382\\
4356.8	0.90082\\
4938.1	0.76874\\
11867	0.25265\\
29734	0.079066\\
74501	0.024844\\
1.8667e+05	0.0078379\\
2.5003e+05	0.0054343\\
};
\addlegendentry{Adatt. Pareto appr.}% ($\mathit{ML}$)
% \addlegendentry{Adatt. Pareto appr. ($\mathit{ML}$)}


\addplot[area legend, draw=none, fill=mycolor1, fill opacity=0.15, forget plot]
table[row sep=crcr] {%
x	y\\
4007.8	0.98443\\
4178.6	0.95203\\
4356.8	0.91008\\
4542.5	0.86353\\
4736.2	0.8185\\
4938.1	0.77583\\
5148.7	0.73541\\
5368.2	0.69711\\
5597.1	0.66082\\
5835.7	0.62645\\
6084.5	0.59388\\
6343.9	0.56303\\
6614.4	0.5338\\
6896.4	0.50612\\
7190.4	0.4799\\
7497	0.45506\\
7816.6	0.43154\\
8149.9	0.40926\\
8497.3	0.38816\\
8859.6	0.36816\\
9237.4	0.34922\\
9631.2	0.33128\\
10042	0.31427\\
10470	0.29815\\
10916	0.28287\\
11382	0.26838\\
11867	0.25465\\
12373	0.24163\\
12900	0.22927\\
13450	0.21756\\
14024	0.20645\\
14622	0.19591\\
15245	0.18591\\
15895	0.17643\\
16573	0.16743\\
17280	0.15889\\
18016	0.1508\\
18784	0.14311\\
19585	0.13582\\
20420	0.1289\\
21291	0.12234\\
22199	0.11611\\
23145	0.1102\\
24132	0.10459\\
25161	0.099274\\
26233	0.094224\\
27352	0.089432\\
28518	0.084885\\
29734	0.080569\\
31002	0.076474\\
32323	0.072588\\
33701	0.0689\\
35138	0.0654\\
36636	0.062078\\
38198	0.058925\\
39827	0.055933\\
41525	0.053094\\
43295	0.050399\\
45141	0.047841\\
47066	0.045414\\
49072	0.04311\\
51165	0.040923\\
53346	0.038848\\
55621	0.036878\\
57992	0.035008\\
60464	0.033233\\
63042	0.031549\\
65730	0.02995\\
68532	0.028433\\
71454	0.026992\\
74501	0.025625\\
77677	0.024327\\
80989	0.023096\\
84442	0.021926\\
88042	0.020816\\
91796	0.019763\\
95709	0.018763\\
99790	0.017814\\
1.0404e+05	0.016912\\
1.0848e+05	0.016057\\
1.1311e+05	0.015245\\
1.1793e+05	0.014474\\
1.2296e+05	0.013742\\
1.282e+05	0.013048\\
1.3366e+05	0.012388\\
1.3936e+05	0.011762\\
1.453e+05	0.011168\\
1.515e+05	0.010604\\
1.5796e+05	0.010068\\
1.6469e+05	0.0095597\\
1.7171e+05	0.009077\\
1.7904e+05	0.0086187\\
1.8667e+05	0.0081837\\
1.9463e+05	0.0077706\\
2.0292e+05	0.0073785\\
2.1158e+05	0.0070062\\
2.206e+05	0.0066528\\
2.3e+05	0.0063172\\
2.3981e+05	0.0059986\\
2.5003e+05	0.0056961\\
2.5003e+05	0.0051726\\
2.3981e+05	0.0054536\\
2.3e+05	0.0057499\\
2.206e+05	0.0060624\\
2.1158e+05	0.0063919\\
2.0292e+05	0.0067393\\
1.9463e+05	0.0071057\\
1.8667e+05	0.0074921\\
1.7904e+05	0.0078996\\
1.7171e+05	0.0083292\\
1.6469e+05	0.0087824\\
1.5796e+05	0.0092602\\
1.515e+05	0.0097641\\
1.453e+05	0.010295\\
1.3936e+05	0.010856\\
1.3366e+05	0.011447\\
1.282e+05	0.01207\\
1.2296e+05	0.012727\\
1.1793e+05	0.013421\\
1.1311e+05	0.014152\\
1.0848e+05	0.014923\\
1.0404e+05	0.015736\\
99790	0.016593\\
95709	0.017498\\
91796	0.018452\\
88042	0.019458\\
84442	0.020519\\
80989	0.021638\\
77677	0.022818\\
74501	0.024063\\
71454	0.025376\\
68532	0.02676\\
65730	0.028221\\
63042	0.029761\\
60464	0.031386\\
57992	0.033099\\
55621	0.034907\\
53346	0.036813\\
51165	0.038824\\
49072	0.040945\\
47066	0.043183\\
45141	0.045542\\
43295	0.048031\\
41525	0.050657\\
39827	0.053426\\
38198	0.056347\\
36636	0.059429\\
35138	0.062679\\
33701	0.066107\\
32323	0.069724\\
31002	0.073539\\
29734	0.077563\\
28518	0.081807\\
27352	0.086285\\
26233	0.091008\\
25161	0.09599\\
24132	0.10125\\
23145	0.10679\\
22199	0.11264\\
21291	0.11881\\
20420	0.12531\\
19585	0.13218\\
18784	0.13942\\
18016	0.14706\\
17280	0.15511\\
16573	0.16361\\
15895	0.17257\\
15245	0.18203\\
14622	0.192\\
14024	0.20252\\
13450	0.21361\\
12900	0.22531\\
12373	0.23765\\
11867	0.25066\\
11382	0.26437\\
10916	0.27884\\
10470	0.29408\\
10042	0.31015\\
9631.2	0.32709\\
9237.4	0.34494\\
8859.6	0.36375\\
8497.3	0.38357\\
8149.9	0.40444\\
7816.6	0.42643\\
7497	0.4496\\
7190.4	0.474\\
6896.4	0.49971\\
6614.4	0.52678\\
6343.9	0.55531\\
6084.5	0.58537\\
5835.7	0.61703\\
5597.1	0.6504\\
5368.2	0.68556\\
5148.7	0.72261\\
4938.1	0.76166\\
4736.2	0.80281\\
4542.5	0.84618\\
4356.8	0.89157\\
4178.6	0.93561\\
4007.8	0.97287\\
}--cycle;
\addplot[only marks, mark=*, mark size=1.2500pt, color=black, fill=black, opacity=0.60, draw=none, mark options={draw=none,line width=0pt}] table[row sep=crcr]{%
x	y\\
3650	0.99468\\
3735	0.98404\\
3741	0.9734\\
3761	0.96277\\
3772	0.95213\\
3785	0.94149\\
3825	0.93085\\
3847	0.92021\\
3900	0.90957\\
3928	0.89894\\
3945	0.8883\\
3996	0.87766\\
4009	0.86702\\
4024	0.85638\\
4061	0.84574\\
4061	0.83511\\
4071	0.82447\\
4113	0.81383\\
4228	0.80319\\
4348	0.79255\\
4430	0.78191\\
4503	0.77128\\
4539	0.76064\\
4632	0.75\\
4654	0.73936\\
4663	0.72872\\
4679	0.71809\\
4702	0.70745\\
4779	0.69681\\
4861	0.68617\\
4953	0.67553\\
5029	0.66489\\
5048	0.65426\\
5236	0.64362\\
5264	0.63298\\
5327	0.62234\\
5379	0.6117\\
5458	0.60106\\
5463	0.59043\\
5524	0.57979\\
5607	0.56915\\
5857	0.55851\\
5868	0.54787\\
5982	0.53723\\
6317	0.5266\\
6332	0.51596\\
6342	0.50532\\
6356	0.49468\\
6367	0.48404\\
6468	0.4734\\
6629	0.46277\\
6926	0.45213\\
7252	0.44149\\
7294	0.43085\\
7320	0.42021\\
7348	0.40957\\
7596	0.39894\\
7724	0.3883\\
7877	0.37766\\
8035	0.36702\\
8499	0.35638\\
8518	0.34574\\
8994	0.33511\\
9128	0.32447\\
9267	0.31383\\
9435	0.30319\\
9599	0.29255\\
9837	0.28191\\
10119	0.27128\\
10336	0.26064\\
10377	0.25\\
11048	0.23936\\
11424	0.22872\\
11830	0.21809\\
12182	0.20745\\
12313	0.19681\\
13086	0.18617\\
13380	0.17553\\
13398	0.16489\\
13899	0.15426\\
14984	0.14362\\
16428	0.13298\\
20491	0.12234\\
21264	0.1117\\
23237	0.10106\\
30134	0.090426\\
30990	0.079787\\
32887	0.069149\\
37527	0.058511\\
39026	0.047872\\
41095	0.037234\\
61636	0.026596\\
1.2234e+05	0.015957\\
2.0424e+05	0.0053191\\
};
\addlegendentry{FRC empirica reale}

\addplot [color=black]
  table[row sep=crcr]{%
3637.5	1\\
2.5003e+05	0.0045909\\
};
\addlegendentry{Adatt. Pareto reale}% ($\mathit{ML}$)
% \addlegendentry{Adatt. Pareto reale ($\mathit{ML}$)}

            % \input{../../../.tex/20/KS/SizeDistributionFittingsfig10.tex}
        \end{groupplot}
    
        \begin{groupplot}[
            group style={
                group name=Row3,
                plotLastGroup,
            },
            plotRow,
            plotLegendSW,
            %
            xmode=log,ymode=log,
            xmin=3637.5,xmax=2.5003e+05,
            ymin=0.00068239,ymax=1.0175,
        ]
            \nextgroupplot[%
                at={($(Row2 c1r1.south west)-(0,.5*\yPlotSep em)-(0,\plotHeight cm)$)},
                % xmin=3637.5,xmax=2.0424e+05,
                % ymin=0.0041301,ymax=1.0005,
            ] % This file was created by matlab2tikz.
%
\definecolor{mycolor1}{rgb}{0.00000,0.44700,0.74100}%

\addplot[only marks, mark=*, mark size=1.3693pt, color=black, fill=black, opacity=0.60, draw=none, mark options={draw=none,line width=0pt}] table[row sep=crcr]{%
x	y\\
3650	0.99468085106383\\
3735	0.984042553191489\\
3741	0.973404255319149\\
3761	0.962765957446808\\
3772	0.952127659574468\\
3785	0.941489361702128\\
3825	0.930851063829787\\
3847	0.920212765957447\\
3900	0.909574468085106\\
3928	0.898936170212766\\
3945	0.888297872340426\\
3996	0.877659574468085\\
4009	0.867021276595745\\
4024	0.856382978723404\\
4061	0.845744680851064\\
4061	0.835106382978723\\
4071	0.824468085106383\\
4113	0.813829787234043\\
4228	0.803191489361702\\
4348	0.792553191489362\\
4430	0.781914893617021\\
4503	0.771276595744681\\
4539	0.76063829787234\\
4632	0.75\\
4654	0.73936170212766\\
4663	0.728723404255319\\
4679	0.718085106382979\\
4702	0.707446808510638\\
4779	0.696808510638298\\
4861	0.686170212765957\\
4953	0.675531914893617\\
5029	0.664893617021277\\
5048	0.654255319148936\\
5236	0.643617021276596\\
5264	0.632978723404255\\
5327	0.622340425531915\\
5379	0.611702127659574\\
5458	0.601063829787234\\
5463	0.590425531914894\\
5524	0.579787234042553\\
5607	0.569148936170213\\
5857	0.558510638297872\\
5868	0.547872340425532\\
5982	0.537234042553192\\
6317	0.526595744680851\\
6332	0.515957446808511\\
6342	0.50531914893617\\
6356	0.49468085106383\\
6367	0.484042553191489\\
6468	0.473404255319149\\
6629	0.462765957446808\\
6926	0.452127659574468\\
7252	0.441489361702128\\
7294	0.430851063829787\\
7320	0.420212765957447\\
7348	0.409574468085106\\
7596	0.398936170212766\\
7724	0.388297872340426\\
7877	0.377659574468085\\
8035	0.367021276595745\\
8499	0.356382978723404\\
8518	0.345744680851064\\
8994	0.335106382978723\\
9128	0.324468085106383\\
9267	0.313829787234043\\
9435	0.303191489361702\\
9599	0.292553191489362\\
9837	0.281914893617021\\
10119	0.271276595744681\\
10336	0.26063829787234\\
10377	0.25\\
11048	0.23936170212766\\
11424	0.228723404255319\\
11830	0.218085106382979\\
12182	0.207446808510638\\
12313	0.196808510638298\\
13086	0.186170212765957\\
13380	0.175531914893617\\
13398	0.164893617021277\\
13899	0.154255319148936\\
14984	0.143617021276596\\
16428	0.132978723404255\\
20491	0.122340425531915\\
21264	0.111702127659574\\
23237	0.101063829787234\\
30134	0.0904255319148936\\
30990	0.0797872340425532\\
32887	0.0691489361702128\\
37527	0.0585106382978723\\
39026	0.0478723404255319\\
41095	0.0372340425531915\\
61636	0.0265957446808511\\
122339	0.0159574468085106\\
204237	0.00531914893617021\\
};
\addlegendentry{FRC empirica reale}

\addplot [color=black, mark repeat=9, mark phase=1, mark=+, mark options={solid, black}]
  table[row sep=crcr]{%
3637.5	1\\
3788.54993935839	0.949537816810858\\
3945.87234172165	0.901622065553929\\
4109.72767586131	0.856124247714574\\
4280.38722671172	0.812922349093734\\
4458.13354451935	0.771900512595218\\
4643.26091264341	0.732948727524845\\
4836.07583478224	0.695962534568237\\
5036.89754243213	0.660842745656075\\
5246.05852341874	0.627495177965562\\
5463.90507237627	0.59583040134476\\
5690.79786408554	0.565763498482441\\
5927.11255062052	0.53721383718029\\
6173.24038329176	0.510104854116755\\
6429.58886041643	0.484363849522645\\
6696.58240198765	0.459921792217835\\
6974.66305235981	0.436713134486259\\
7264.29121211354	0.414675636292709\\
7565.94640031188	0.393750198370032\\
7880.12804840974	0.373880703729123\\
8207.35632713103	0.355013867166658\\
8548.17300768248	0.337099092367008\\
8903.14235873003	0.32008833621509\\
9272.85208062291	0.303935979956297\\
9657.9142784119	0.28859870685797\\
10058.9664752731	0.274035386044354\\
10476.6726680149	0.260206962193476\\
10911.7244264153	0.247076350800179\\
11364.8420382106	0.234608338724395\\
11836.7757016304	0.222769489757984\\
12328.3067674531	0.211528054956865\\
12840.2490326395	0.200853887497989\\
13373.4500876847	0.190718361832814\\
13928.7927199204	0.181094296920473\\
14507.19637509	0.171955883334763\\
15109.6186796172	0.163278614049474\\
15737.0570260872	0.15503921871644\\
16390.550224567	0.147215601260069\\
17071.1802224973	0.139786780620984\\
17780.0738960051	0.132732834489867\\
18518.4049156008	0.126034845880625\\
19287.3956893508	0.119674852399582\\
20088.3193867412	0.113635798074661\\
20922.5020465844	0.107901487615373\\
21791.3247724572	0.102456542980945\\
22696.2260193077	0.0972863621401141\\
23638.7039750138	0.0923770799119943\\
24620.3190408384	0.0877155307829971\\
25642.6964148873	0.0832892136000928\\
26707.5287828473	0.0790862580457252\\
27816.5791204588	0.0750953928044779\\
28971.6836123634	0.0713059153361178\\
30174.7546921595	0.0677076631739571\\
31427.7842086969	0.0642909866715642\\
32732.8467238564	0.061046723124733\\
34092.1029472698	0.0579661721993158\\
35507.8033136712	0.0550410725990206\\
36982.2917087997	0.0522635799106019\\
38518.0093500227	0.0496262455670326\\
40117.4988281057	0.0471219968722397\\
41783.4083168193	0.0447441180338345\\
43518.4959573534	0.0424862321529745\\
45325.6344247971	0.0403422841230547\\
47207.8156842464	0.0383065243913687\\
49168.1559444106	0.0363734935401921\\
51209.900816924	0.0345380076459378\\
53336.4306898988	0.0327951443771205\\
55551.2663246212	0.0311402297938478\\
57858.074684653	0.0295688258134387\\
60260.6750069927	0.0280767183085531\\
62763.0451253437	0.026659905805917\\
65369.3280559641	0.0253145887553335\\
68083.8388569957	0.024037159340204\\
70911.0717726344	0.022824191802232\\
73855.7076739664	0.0216724332543637\\
76922.6218087908	0.0205787949573275\\
80116.89187326	0.0195403440363791\\
83443.8064187003	0.0185542956160364\\
86908.8736075329	0.0176180053517145\\
90517.8303327903	0.016728962338229\\
94276.651716329	0.015884782376153\\
98191.5610014598	0.0150832015779679\\
102269.039856381	0.0143220702968617\\
106515.839105467	0.0135993473618937\\
110938.989906178	0.012913094604065\\
115545.815390112	0.012261471658616\\
120343.942787444	0.0116427310296104\\
125341.316054851	0.0110552134035723\\
130546.209027823	0.0104973431996062\\
135967.239119128	0.00996762434406833\\
141613.381586117	0.00946463625845741\\
147493.984390494	0.00898703004976452\\
153618.783675143	0.00853352489306699\\
159997.919883649	0.00810290459666392\\
166641.954549187	0.00769401434054292\\
173561.887780587	0.00730575757943056\\
180769.176474522	0.00693709310212186\\
188275.753283964	0.00658703223920245\\
196094.046374327	0.00625463621167503\\
204237	0.00593901361338004\\
};
\addlegendentry{Adatt. Pareto reale ($\mathit{ML}$)}

\addplot [color=black, mark repeat=9, mark phase=1, mark=x, mark options={solid, black}]
  table[row sep=crcr]{%
3637.5	1.0004869525303\\
3788.54993935839	0.953165189609478\\
3945.87234172165	0.907427082756923\\
4109.72767586131	0.863290892405921\\
4280.38722671172	0.820768189461985\\
4458.13354451935	0.77986406626895\\
4643.26091264341	0.740577387414794\\
4836.07583478224	0.702901076404548\\
5036.89754243213	0.666822433995031\\
5246.05852341874	0.632323483817515\\
5463.90507237627	0.59938134080958\\
5690.79786408554	0.567968597935033\\
5927.11255062052	0.538053726688371\\
6173.24038329176	0.509601486955019\\
6429.58886041643	0.482573341926024\\
6696.58240198765	0.456927873941986\\
6974.66305235981	0.432621197360071\\
7264.29121211354	0.409607364794705\\
7565.94640031188	0.387838763371018\\
7880.12804840974	0.367266497944131\\
8207.35632713103	0.347840758571035\\
8548.17300768248	0.329511169868802\\
8903.14235873003	0.312227120247536\\
9272.85208062291	0.295938069362992\\
9657.9142784119	0.280593832487039\\
10058.9664752731	0.266144840839093\\
10476.6726680149	0.252542377254078\\
10911.7244264153	0.239738786878372\\
11364.8420382106	0.227687662881319\\
11836.7757016304	0.21634400744349\\
12328.3067674531	0.205664368531773\\
12840.2490326395	0.195606953193889\\
13373.4500876847	0.186131718300219\\
13928.7927199204	0.177200439828105\\
14507.19637509	0.168776761923201\\
15109.6186796172	0.160826227084399\\
15737.0570260872	0.153316288904044\\
16390.550224567	0.146216308854908\\
17071.1802224973	0.139497538651068\\
17780.0738960051	0.133133089723205\\
18518.4049156008	0.127097891341835\\
19287.3956893508	0.121368638896574\\
20088.3193867412	0.115923733797975\\
20922.5020465844	0.110743216412846\\
21791.3247724572	0.105808693376563\\
22696.2260193077	0.101103260548696\\
23638.7039750138	0.096611422793629\\
24620.3190408384	0.0923190116774575\\
25642.6964148873	0.0882131020783547\\
26707.5287828473	0.0842819286114441\\
27816.5791204588	0.0805148026725548\\
28971.6836123634	0.0769020308094709\\
30174.7546921595	0.0734348350356383\\
31427.7842086969	0.0701052756108015\\
32732.8467238564	0.0669061767265721\\
34092.1029472698	0.063831055453225\\
35507.8033136712	0.0608740542275787\\
36982.2917087997	0.0580298770910952\\
38518.0093500227	0.0552937298225691\\
40117.4988281057	0.0526612640511152\\
41783.4083168193	0.0501285253826495\\
43518.4959573534	0.0476919055266147\\
45325.6344247971	0.0453480983691843\\
47207.8156842464	0.0430940599043728\\
49168.1559444106	0.0409269719051086\\
51209.900816924	0.0388442091920599\\
53336.4306898988	0.0368433103385172\\
55551.2663246212	0.0349219516345141\\
57858.074684653	0.0330779241222635\\
60260.6750069927	0.0313091135074718\\
62763.0451253437	0.0296134827468033\\
65369.3280559641	0.0279890571102966\\
68083.8388569957	0.0264339115185369\\
70911.0717726344	0.0249461599574891\\
73855.7076739664	0.0235239467787836\\
76922.6218087908	0.0221654396996028\\
80116.89187326	0.0208688243238502\\
83443.8064187003	0.0196323000147655\\
86908.8736075329	0.0184540769582996\\
90517.8303327903	0.0173323742662265\\
94276.651716329	0.0162654189779162\\
98191.5610014598	0.0152514458297972\\
102269.039856381	0.0142886976716394\\
106515.839105467	0.0133754264187857\\
110938.989906178	0.0125098944392443\\
115545.815390112	0.0116903762840562\\
120343.942787444	0.0109151606784948\\
125341.316054851	0.0101825527003985\\
130546.209027823	0.00949087608024179\\
135967.239119128	0.00883847556538394\\
141613.381586117	0.00822371929828939\\
147493.984390494	0.00764500116537325\\
153618.783675143	0.00710074307949289\\
159997.919883649	0.00658939716498238\\
166641.954549187	0.00610944781952394\\
173561.887780587	0.00565941363207689\\
180769.176474522	0.00523784914056129\\
188275.753283964	0.00484334641703407\\
196094.046374327	0.00447453647172286\\
204237	0.00413009047051692\\
};
\addlegendentry{Adatt. BLN reale ($\mathit{ML}$)}
            % \input{../../../.tex/20/KS/SizeDistributionFittingsfig5.tex}
    
            \nextgroupplot[%
                % xmin=3968.2,xmax=2.5003e+05,
                % ymin=0.0001,ymax=1.0175,
            ] % This file was created by matlab2tikz.
%
\definecolor{mycolor1}{rgb}{0.00000,0.44706,0.69804}%
\definecolor{mycolor2}{rgb}{0.00000,0.44700,0.74100}%
%
\addplot[only marks, mark=*, mark size=1.2500pt, color=mycolor1, fill=mycolor1, opacity=0.60, draw=none, mark options={draw=none,line width=0pt}] table[row sep=crcr]{%
x	y\\
4009.8	0.99468\\
4042.8	0.98404\\
4083.2	0.9734\\
4118.9	0.96277\\
4158.9	0.95213\\
4199.9	0.94149\\
4241.8	0.93085\\
4293.5	0.92021\\
4346.1	0.90957\\
4392.3	0.89894\\
4440.5	0.8883\\
4485	0.87766\\
4529	0.86702\\
4574.2	0.85638\\
4618.5	0.84574\\
4670.1	0.83511\\
4730.3	0.82447\\
4787.1	0.81383\\
4841	0.80319\\
4903.8	0.79255\\
4960.2	0.78191\\
5017.2	0.77128\\
5087.5	0.76064\\
5155.4	0.75\\
5233.1	0.73936\\
5291.8	0.72872\\
5360.5	0.71809\\
5431.2	0.70745\\
5501.1	0.69681\\
5574	0.68617\\
5661.2	0.67553\\
5733.2	0.66489\\
5815.2	0.65426\\
5904.3	0.64362\\
5987.8	0.63298\\
6075.5	0.62234\\
6160.2	0.6117\\
6238.2	0.60106\\
6339.4	0.59043\\
6443.4	0.57979\\
6550.4	0.56915\\
6655.6	0.55851\\
6768.6	0.54787\\
6863	0.53723\\
6984.1	0.5266\\
7105.1	0.51596\\
7243	0.50532\\
7363.9	0.49468\\
7529.2	0.48404\\
7674.5	0.4734\\
7798.1	0.46277\\
7951.1	0.45213\\
8117.5	0.44149\\
8265.8	0.43085\\
8464.1	0.42021\\
8676.5	0.40957\\
8871.5	0.39894\\
9049.7	0.3883\\
9256	0.37766\\
9441.2	0.36702\\
9666.5	0.35638\\
9881.2	0.34574\\
10150	0.33511\\
10431	0.32447\\
10718	0.31383\\
11062	0.30319\\
11293	0.29255\\
11688	0.28191\\
12013	0.27128\\
12316	0.26064\\
12705	0.25\\
13150	0.23936\\
13536	0.22872\\
13995	0.21809\\
14547	0.20745\\
15079	0.19681\\
15757	0.18617\\
16494	0.17553\\
17200	0.16489\\
17972	0.15426\\
18737	0.14362\\
19593	0.13298\\
20816	0.12234\\
22198	0.1117\\
23437	0.10106\\
25016	0.090426\\
27001	0.079787\\
29590	0.069149\\
32563	0.058511\\
36426	0.047872\\
43570	0.037234\\
54044	0.026596\\
68767	0.015957\\
1.4931e+05	0.0053191\\
};
\addlegendentry{FRC empirica esatta}

\addplot [color=mycolor1, only marks, every error bar/.append style={opacity=0.30}, mark=*, mark size=0pt, draw=none, forget plot]
 plot [error bars/.cd, x dir=both, x explicit, error bar style={line width=1pt, color=mycolor1}, error mark options={mark=none,mark size=0pt}]
 table[row sep=crcr, x error plus index=2, x error minus index=3]{%
4009.8	0.99468	41.683	41.683\\
4042.8	0.98404	42.729	42.729\\
4083.2	0.9734	42.64	42.64\\
4118.9	0.96277	42.259	42.259\\
4158.9	0.95213	42.401	42.401\\
4199.9	0.94149	43.247	43.247\\
4241.8	0.93085	44.025	44.025\\
4293.5	0.92021	45.653	45.653\\
4346.1	0.90957	47.232	47.232\\
4392.3	0.89894	47.443	47.443\\
4440.5	0.8883	47.604	47.604\\
4485	0.87766	47.648	47.648\\
4529	0.86702	48.056	48.056\\
4574.2	0.85638	48.712	48.712\\
4618.5	0.84574	48.794	48.794\\
4670.1	0.83511	49.967	49.967\\
4730.3	0.82447	48.825	48.825\\
4787.1	0.81383	49.432	49.432\\
4841	0.80319	50.68	50.68\\
4903.8	0.79255	49.872	49.872\\
4960.2	0.78191	50.875	50.875\\
5017.2	0.77128	51.172	51.172\\
5087.5	0.76064	52.783	52.783\\
5155.4	0.75	53.548	53.548\\
5233.1	0.73936	54.7	54.7\\
5291.8	0.72872	56.659	56.659\\
5360.5	0.71809	57.557	57.557\\
5431.2	0.70745	56.553	56.553\\
5501.1	0.69681	55.696	55.696\\
5574	0.68617	59.561	59.561\\
5661.2	0.67553	62.198	62.198\\
5733.2	0.66489	60.853	60.853\\
5815.2	0.65426	65.634	65.634\\
5904.3	0.64362	63.226	63.226\\
5987.8	0.63298	65.228	65.228\\
6075.5	0.62234	68.996	68.996\\
6160.2	0.6117	69.293	69.293\\
6238.2	0.60106	72.946	72.946\\
6339.4	0.59043	75.51	75.51\\
6443.4	0.57979	74.961	74.961\\
6550.4	0.56915	74.817	74.817\\
6655.6	0.55851	79.725	79.725\\
6768.6	0.54787	82.852	82.852\\
6863	0.53723	81.443	81.443\\
6984.1	0.5266	81.923	81.923\\
7105.1	0.51596	88.89	88.89\\
7243	0.50532	93.199	93.199\\
7363.9	0.49468	97.499	97.499\\
7529.2	0.48404	96.758	96.758\\
7674.5	0.4734	95.477	95.477\\
7798.1	0.46277	93.925	93.925\\
7951.1	0.45213	98.413	98.413\\
8117.5	0.44149	97.624	97.624\\
8265.8	0.43085	101.94	101.94\\
8464.1	0.42021	108.66	108.66\\
8676.5	0.40957	112.81	112.81\\
8871.5	0.39894	110.39	110.39\\
9049.7	0.3883	117.1	117.1\\
9256	0.37766	118.93	118.93\\
9441.2	0.36702	117.23	117.23\\
9666.5	0.35638	114.85	114.85\\
9881.2	0.34574	119.48	119.48\\
10150	0.33511	123.96	123.96\\
10431	0.32447	124.98	124.98\\
10718	0.31383	133.42	133.42\\
11062	0.30319	143.38	143.38\\
11293	0.29255	150.62	150.62\\
11688	0.28191	166.57	166.57\\
12013	0.27128	171.22	171.22\\
12316	0.26064	181.72	181.72\\
12705	0.25	192.57	192.57\\
13150	0.23936	208.87	208.87\\
13536	0.22872	200.05	200.05\\
13995	0.21809	213.59	213.59\\
14547	0.20745	224.95	224.95\\
15079	0.19681	224.48	224.48\\
15757	0.18617	254.54	254.54\\
16494	0.17553	273.77	273.77\\
17200	0.16489	282.21	282.21\\
17972	0.15426	309.96	309.96\\
18737	0.14362	340.73	340.73\\
19593	0.13298	351.38	351.38\\
20816	0.12234	385	385\\
22198	0.1117	402.42	402.42\\
23437	0.10106	424.8	424.8\\
25016	0.090426	456.58	456.58\\
27001	0.079787	560.02	560.02\\
29590	0.069149	636.25	636.25\\
32563	0.058511	797.84	797.84\\
36426	0.047872	902.21	902.21\\
43570	0.037234	1186.4	1186.4\\
54044	0.026596	1844.3	1844.3\\
68767	0.015957	2565.5	2565.5\\
1.4931e+05	0.0053191	4008.6	4008.6\\
};
\addplot [color=mycolor1, mark repeat=9, mark phase=1, mark=+, mark options={solid, mycolor1}]
  table[row sep=crcr]{%
3997.8	0.97494\\
4168.3	0.94141\\
4346.1	0.89973\\
4531.5	0.85488\\
4724.9	0.81129\\
4926.4	0.76977\\
5136.6	0.73039\\
5355.7	0.69303\\
5584.2	0.65759\\
5822.4	0.62397\\
6070.8	0.59207\\
6329.8	0.56181\\
6599.9	0.5331\\
6881.4	0.50586\\
7175	0.48002\\
7481.1	0.4555\\
7800.2	0.43224\\
8133	0.41017\\
8480	0.38924\\
8841.7	0.36937\\
9218.9	0.35052\\
9612.2	0.33264\\
10022	0.31567\\
10450	0.29957\\
10896	0.2843\\
11360	0.2698\\
11845	0.25605\\
12350	0.243\\
12877	0.23062\\
13427	0.21887\\
14000	0.20772\\
14597	0.19714\\
15219	0.18711\\
15869	0.17758\\
16546	0.16854\\
17252	0.15997\\
17988	0.15183\\
18755	0.1441\\
19555	0.13677\\
20389	0.12982\\
21259	0.12322\\
22166	0.11696\\
23112	0.11101\\
24098	0.10537\\
25126	0.10002\\
26198	0.094937\\
27315	0.090115\\
28481	0.085539\\
29696	0.081196\\
30962	0.077075\\
32283	0.073163\\
33661	0.069451\\
35097	0.065927\\
36594	0.062583\\
38155	0.059409\\
39783	0.056397\\
41480	0.053538\\
43249	0.050824\\
45095	0.048248\\
47018	0.045804\\
49024	0.043483\\
51116	0.041281\\
53296	0.03919\\
55570	0.037206\\
57941	0.035322\\
60412	0.033534\\
62990	0.031837\\
65677	0.030226\\
68479	0.028697\\
71400	0.027246\\
74446	0.025868\\
77622	0.02456\\
80934	0.023318\\
84386	0.02214\\
87986	0.021021\\
91740	0.019959\\
95654	0.01895\\
99734	0.017993\\
1.0399e+05	0.017085\\
1.0843e+05	0.016222\\
1.1305e+05	0.015403\\
1.1787e+05	0.014626\\
1.229e+05	0.013888\\
1.2815e+05	0.013187\\
1.3361e+05	0.012522\\
1.3931e+05	0.01189\\
1.4526e+05	0.01129\\
1.5145e+05	0.010721\\
1.5791e+05	0.010181\\
1.6465e+05	0.0096675\\
1.7168e+05	0.0091803\\
1.79e+05	0.0087177\\
1.8664e+05	0.0082785\\
1.946e+05	0.0078615\\
2.029e+05	0.0074656\\
2.1156e+05	0.0070897\\
2.2058e+05	0.0067328\\
2.2999e+05	0.0063939\\
2.398e+05	0.0060721\\
2.5003e+05	0.0057666\\
};
\addlegendentry{Adatt. Pareto esatto ($\mathit{ML}$)}


\addplot[area legend, draw=none, fill=mycolor1, fill opacity=0.15, postaction={pattern={Lines[angle=90,distance={5pt},line width={0.5pt}]}, pattern color=mycolor1!60}, forget plot]
table[row sep=crcr] {%
x	y\\
3997.8	0.98138\\
4168.3	0.95033\\
4346.1	0.90976\\
4531.5	0.8645\\
4724.9	0.82008\\
4926.4	0.77772\\
5136.6	0.73756\\
5355.7	0.6995\\
5584.2	0.66342\\
5822.4	0.62922\\
6070.8	0.59681\\
6329.8	0.56608\\
6599.9	0.53697\\
6881.4	0.50937\\
7175	0.48323\\
7481.1	0.45845\\
7800.2	0.43497\\
8133	0.41272\\
8480	0.39164\\
8841.7	0.37166\\
9218.9	0.35273\\
9612.2	0.33478\\
10022	0.31777\\
10450	0.30164\\
10896	0.28635\\
11360	0.27184\\
11845	0.25808\\
12350	0.24503\\
12877	0.23264\\
13427	0.22089\\
14000	0.20974\\
14597	0.19915\\
15219	0.18911\\
15869	0.17957\\
16546	0.17052\\
17252	0.16193\\
17988	0.15377\\
18755	0.14603\\
19555	0.13868\\
20389	0.1317\\
21259	0.12507\\
22166	0.11878\\
23112	0.11281\\
24098	0.10713\\
25126	0.10175\\
26198	0.096634\\
27315	0.091778\\
28481	0.087168\\
29696	0.08279\\
30962	0.078632\\
32283	0.074684\\
33661	0.070936\\
35097	0.067376\\
36594	0.063995\\
38155	0.060784\\
39783	0.057736\\
41480	0.05484\\
43249	0.052091\\
45095	0.049479\\
47018	0.046999\\
49024	0.044644\\
51116	0.042407\\
53296	0.040283\\
55570	0.038265\\
57941	0.036349\\
60412	0.034529\\
62990	0.032801\\
65677	0.031159\\
68479	0.0296\\
71400	0.028119\\
74446	0.026712\\
77622	0.025376\\
80934	0.024107\\
84386	0.022902\\
87986	0.021757\\
91740	0.020669\\
95654	0.019637\\
99734	0.018655\\
1.0399e+05	0.017723\\
1.0843e+05	0.016838\\
1.1305e+05	0.015997\\
1.1787e+05	0.015199\\
1.229e+05	0.01444\\
1.2815e+05	0.013719\\
1.3361e+05	0.013034\\
1.3931e+05	0.012384\\
1.4526e+05	0.011766\\
1.5145e+05	0.011179\\
1.5791e+05	0.010622\\
1.6465e+05	0.010092\\
1.7168e+05	0.0095889\\
1.79e+05	0.0091109\\
1.8664e+05	0.0086568\\
1.946e+05	0.0082255\\
2.029e+05	0.0078156\\
2.1156e+05	0.0074263\\
2.2058e+05	0.0070564\\
2.2999e+05	0.006705\\
2.398e+05	0.0063712\\
2.5003e+05	0.006054\\
2.5003e+05	0.0054792\\
2.398e+05	0.0057731\\
2.2999e+05	0.0060828\\
2.2058e+05	0.0064092\\
2.1156e+05	0.0067531\\
2.029e+05	0.0071156\\
1.946e+05	0.0074976\\
1.8664e+05	0.0079002\\
1.79e+05	0.0083245\\
1.7168e+05	0.0087716\\
1.6465e+05	0.0092429\\
1.5791e+05	0.0097395\\
1.5145e+05	0.010263\\
1.4526e+05	0.010815\\
1.3931e+05	0.011396\\
1.3361e+05	0.012009\\
1.2815e+05	0.012655\\
1.229e+05	0.013335\\
1.1787e+05	0.014053\\
1.1305e+05	0.014809\\
1.0843e+05	0.015606\\
1.0399e+05	0.016446\\
99734	0.017331\\
95654	0.018264\\
91740	0.019248\\
87986	0.020285\\
84386	0.021378\\
80934	0.02253\\
77622	0.023744\\
74446	0.025024\\
71400	0.026373\\
68479	0.027795\\
65677	0.029294\\
62990	0.030874\\
60412	0.03254\\
57941	0.034295\\
55570	0.036146\\
53296	0.038097\\
51116	0.040154\\
49024	0.042322\\
47018	0.044608\\
45095	0.047018\\
43249	0.049558\\
41480	0.052235\\
39783	0.055058\\
38155	0.058034\\
36594	0.061172\\
35097	0.064479\\
33661	0.067966\\
32283	0.071642\\
30962	0.075517\\
29696	0.079603\\
28481	0.08391\\
27315	0.088452\\
26198	0.093239\\
25126	0.098286\\
24098	0.10361\\
23112	0.10922\\
22166	0.11513\\
21259	0.12137\\
20389	0.12794\\
19555	0.13487\\
18755	0.14218\\
17988	0.14988\\
17252	0.15801\\
16546	0.16657\\
15869	0.17559\\
15219	0.18511\\
14597	0.19514\\
14000	0.20571\\
13427	0.21685\\
12877	0.2286\\
12350	0.24097\\
11845	0.25402\\
11360	0.26776\\
10896	0.28225\\
10450	0.2975\\
10022	0.31357\\
9612.2	0.3305\\
9218.9	0.34832\\
8841.7	0.36708\\
8480	0.38683\\
8133	0.40763\\
7800.2	0.42952\\
7481.1	0.45256\\
7175	0.47681\\
6881.4	0.50235\\
6599.9	0.52923\\
6329.8	0.55753\\
6070.8	0.58733\\
5822.4	0.61872\\
5584.2	0.65176\\
5355.7	0.68657\\
5136.6	0.72322\\
4926.4	0.76183\\
4724.9	0.8025\\
4531.5	0.84527\\
4346.1	0.88969\\
4168.3	0.9325\\
3997.8	0.96849\\
}--cycle;
\addplot [color=mycolor1, mark repeat=9, mark phase=1, mark=x, mark options={solid, mycolor1}]
  table[row sep=crcr]{%
3997.8	1.0099\\
4168.3	0.96444\\
4346.1	0.92061\\
4531.5	0.87836\\
4724.9	0.83767\\
4926.4	0.79853\\
5136.6	0.76091\\
5355.7	0.72479\\
5584.2	0.69014\\
5822.4	0.65691\\
6070.8	0.62508\\
6329.8	0.5946\\
6599.9	0.56544\\
6881.4	0.53755\\
7175	0.51088\\
7481.1	0.4854\\
7800.2	0.46107\\
8133	0.43783\\
8480	0.41564\\
8841.7	0.39447\\
9218.9	0.37427\\
9612.2	0.35499\\
10022	0.33661\\
10450	0.31908\\
10896	0.30236\\
11360	0.28642\\
11845	0.27123\\
12350	0.25674\\
12877	0.24294\\
13427	0.22978\\
14000	0.21724\\
14597	0.20529\\
15219	0.19391\\
15869	0.18307\\
16546	0.17275\\
17252	0.16292\\
17988	0.15356\\
18755	0.14466\\
19555	0.13619\\
20389	0.12814\\
21259	0.12048\\
22166	0.11321\\
23112	0.1063\\
24098	0.099745\\
25126	0.093524\\
26198	0.087626\\
27315	0.082036\\
28481	0.076742\\
29696	0.071733\\
30962	0.066995\\
32283	0.062519\\
33661	0.058291\\
35097	0.054303\\
36594	0.050543\\
38155	0.047002\\
39783	0.043669\\
41480	0.040536\\
43249	0.037592\\
45095	0.03483\\
47018	0.032241\\
49024	0.029816\\
51116	0.027547\\
53296	0.025426\\
55570	0.023445\\
57941	0.021598\\
60412	0.019877\\
62990	0.018276\\
65677	0.016787\\
68479	0.015404\\
71400	0.014121\\
74446	0.012932\\
77622	0.011831\\
80934	0.010814\\
84386	0.0098738\\
87986	0.0090067\\
91740	0.0082077\\
95654	0.0074722\\
99734	0.0067959\\
1.0399e+05	0.0061748\\
1.0843e+05	0.005605\\
1.1305e+05	0.0050828\\
1.1787e+05	0.0046047\\
1.229e+05	0.0041676\\
1.2815e+05	0.0037683\\
1.3361e+05	0.0034041\\
1.3931e+05	0.0030721\\
1.4526e+05	0.0027698\\
1.5145e+05	0.0024949\\
1.5791e+05	0.0022452\\
1.6465e+05	0.0020186\\
1.7168e+05	0.0018132\\
1.79e+05	0.0016272\\
1.8664e+05	0.0014589\\
1.946e+05	0.0013068\\
2.029e+05	0.0011696\\
2.1156e+05	0.0010458\\
2.2058e+05	0.00093423\\
2.2999e+05	0.00083385\\
2.398e+05	0.00074361\\
2.5003e+05	0.00066255\\
};
\addlegendentry{Adatt. BLN esatto ($\mathit{ML}$)}


\addplot[area legend, draw=none, fill=mycolor1, fill opacity=0.15, postaction={pattern={Lines[angle=45,distance={5pt},line width={0.5pt}]}, pattern color=mycolor1!60}, forget plot]
table[row sep=crcr] {%
x	y\\
3997.8	1.0175\\
4168.3	0.97182\\
4346.1	0.92776\\
4531.5	0.88527\\
4724.9	0.84436\\
4926.4	0.805\\
5136.6	0.76718\\
5355.7	0.73085\\
5584.2	0.696\\
5822.4	0.66258\\
6070.8	0.63056\\
6329.8	0.59991\\
6599.9	0.57058\\
6881.4	0.54252\\
7175	0.5157\\
7481.1	0.49007\\
7800.2	0.46559\\
8133	0.44221\\
8480	0.41988\\
8841.7	0.39857\\
9218.9	0.37824\\
9612.2	0.35883\\
10022	0.34031\\
10450	0.32265\\
10896	0.3058\\
11360	0.28973\\
11845	0.27441\\
12350	0.25979\\
12877	0.24586\\
13427	0.23258\\
14000	0.21991\\
14597	0.20785\\
15219	0.19635\\
15869	0.18539\\
16546	0.17496\\
17252	0.16502\\
17988	0.15556\\
18755	0.14656\\
19555	0.138\\
20389	0.12986\\
21259	0.12212\\
22166	0.11477\\
23112	0.1078\\
24098	0.10117\\
25126	0.094891\\
26198	0.088937\\
27315	0.083297\\
28481	0.077958\\
29696	0.072906\\
30962	0.06813\\
32283	0.063618\\
33661	0.059358\\
35097	0.055339\\
36594	0.05155\\
38155	0.047981\\
39783	0.044622\\
41480	0.041462\\
43249	0.038494\\
45095	0.035706\\
47018	0.033091\\
49024	0.030641\\
51116	0.028346\\
53296	0.0262\\
55570	0.024193\\
57941	0.02232\\
60412	0.020573\\
62990	0.018945\\
65677	0.017429\\
68479	0.016019\\
71400	0.01471\\
74446	0.013495\\
77622	0.012368\\
80934	0.011325\\
84386	0.010359\\
87986	0.0094673\\
91740	0.0086439\\
95654	0.0078846\\
99734	0.0071852\\
1.0399e+05	0.0065416\\
1.0843e+05	0.00595\\
1.1305e+05	0.0054068\\
1.1787e+05	0.0049086\\
1.229e+05	0.0044521\\
1.2815e+05	0.0040343\\
1.3361e+05	0.0036523\\
1.3931e+05	0.0033034\\
1.4526e+05	0.0029851\\
1.5145e+05	0.002695\\
1.5791e+05	0.0024308\\
1.6465e+05	0.0021906\\
1.7168e+05	0.0019723\\
1.79e+05	0.0017742\\
1.8664e+05	0.0015946\\
1.946e+05	0.0014319\\
2.029e+05	0.0012846\\
2.1156e+05	0.0011515\\
2.2058e+05	0.0010313\\
2.2999e+05	0.00092285\\
2.398e+05	0.0008251\\
2.5003e+05	0.00073707\\
2.5003e+05	0.00058803\\
2.398e+05	0.00066212\\
2.2999e+05	0.00074485\\
2.2058e+05	0.00083715\\
2.1156e+05	0.00094\\
2.029e+05	0.0010545\\
1.946e+05	0.0011818\\
1.8664e+05	0.0013232\\
1.79e+05	0.0014801\\
1.7168e+05	0.0016541\\
1.6465e+05	0.0018466\\
1.5791e+05	0.0020596\\
1.5145e+05	0.0022949\\
1.4526e+05	0.0025545\\
1.3931e+05	0.0028407\\
1.3361e+05	0.0031559\\
1.2815e+05	0.0035024\\
1.229e+05	0.0038831\\
1.1787e+05	0.0043009\\
1.1305e+05	0.0047587\\
1.0843e+05	0.00526\\
1.0399e+05	0.0058081\\
99734	0.0064067\\
95654	0.0070599\\
91740	0.0077715\\
87986	0.0085461\\
84386	0.0093882\\
80934	0.010303\\
77622	0.011294\\
74446	0.012369\\
71400	0.013531\\
68479	0.014788\\
65677	0.016144\\
62990	0.017607\\
60412	0.019182\\
57941	0.020877\\
55570	0.022697\\
53296	0.024652\\
51116	0.026747\\
49024	0.02899\\
47018	0.03139\\
45095	0.033954\\
43249	0.036691\\
41480	0.039609\\
39783	0.042716\\
38155	0.046022\\
36594	0.049536\\
35097	0.053267\\
33661	0.057225\\
32283	0.061419\\
30962	0.065861\\
29696	0.07056\\
28481	0.075527\\
27315	0.080775\\
26198	0.086314\\
25126	0.092157\\
24098	0.098317\\
23112	0.10481\\
22166	0.11165\\
21259	0.11884\\
20389	0.12641\\
19555	0.13438\\
18755	0.14276\\
17988	0.15156\\
17252	0.16081\\
16546	0.17054\\
15869	0.18075\\
15219	0.19148\\
14597	0.20274\\
14000	0.21457\\
13427	0.22698\\
12877	0.24001\\
12350	0.25369\\
11845	0.26805\\
11360	0.28312\\
10896	0.29893\\
10450	0.31551\\
10022	0.33291\\
9612.2	0.35116\\
9218.9	0.3703\\
8841.7	0.39036\\
8480	0.4114\\
8133	0.43344\\
7800.2	0.45654\\
7481.1	0.48073\\
7175	0.50606\\
6881.4	0.53257\\
6599.9	0.5603\\
6329.8	0.5893\\
6070.8	0.6196\\
5822.4	0.65125\\
5584.2	0.68428\\
5355.7	0.71874\\
5136.6	0.75465\\
4926.4	0.79206\\
4724.9	0.83098\\
4531.5	0.87144\\
4346.1	0.91346\\
4168.3	0.95706\\
3997.8	1.0023\\
}--cycle;
            % \input{../../../.tex/20/KS/SizeDistributionFittingsfig6.tex}
    
            \nextgroupplot[%
                % xmin=3989.7,xmax=2.5003e+05,
                % ymin=0.00068239,ymax=1.0119,
            ] % This file was created by matlab2tikz.
%
\definecolor{mycolor1}{rgb}{0.83529,0.36863,0.00000}%
\definecolor{mycolor2}{rgb}{0.00000,0.44700,0.74100}%
%
\addplot[only marks, mark=*, mark size=1.2500pt, color=mycolor1, fill=mycolor1, opacity=0.60, draw=none, mark options={draw=none,line width=0pt}] table[row sep=crcr]{%
x	y\\
4026.3	0.99468\\
4058.6	0.98404\\
4099.8	0.9734\\
4144	0.96277\\
4185.7	0.95213\\
4222.6	0.94149\\
4258.8	0.93085\\
4294	0.92021\\
4338.8	0.90957\\
4387.5	0.89894\\
4434.4	0.8883\\
4477.8	0.87766\\
4531.1	0.86702\\
4583.2	0.85638\\
4627.9	0.84574\\
4689.2	0.83511\\
4733	0.82447\\
4787	0.81383\\
4843.5	0.80319\\
4896.6	0.79255\\
4951.8	0.78191\\
5010.6	0.77128\\
5072.2	0.76064\\
5133.1	0.75\\
5190.8	0.73936\\
5257.5	0.72872\\
5330.8	0.71809\\
5395.5	0.70745\\
5462.9	0.69681\\
5544.5	0.68617\\
5620.8	0.67553\\
5702.2	0.66489\\
5790.5	0.65426\\
5881.8	0.64362\\
5964	0.63298\\
6028.3	0.62234\\
6133.8	0.6117\\
6237	0.60106\\
6320	0.59043\\
6406.8	0.57979\\
6506.9	0.56915\\
6616.8	0.55851\\
6722.3	0.54787\\
6828.5	0.53723\\
6953.2	0.5266\\
7088.7	0.51596\\
7213.5	0.50532\\
7325.9	0.49468\\
7446.2	0.48404\\
7581.1	0.4734\\
7724.5	0.46277\\
7861.5	0.45213\\
8007	0.44149\\
8164.3	0.43085\\
8339.4	0.42021\\
8505.8	0.40957\\
8697.7	0.39894\\
8854.2	0.3883\\
9063	0.37766\\
9293.7	0.36702\\
9542.3	0.35638\\
9795.1	0.34574\\
10050	0.33511\\
10298	0.32447\\
10572	0.31383\\
10877	0.30319\\
11188	0.29255\\
11456	0.28191\\
11809	0.27128\\
12134	0.26064\\
12539	0.25\\
12911	0.23936\\
13427	0.22872\\
13974	0.21809\\
14460	0.20745\\
15013	0.19681\\
15718	0.18617\\
16449	0.17553\\
17281	0.16489\\
18072	0.15426\\
18837	0.14362\\
19847	0.13298\\
20821	0.12234\\
22193	0.1117\\
23455	0.10106\\
24972	0.090426\\
26751	0.079787\\
29026	0.069149\\
31699	0.058511\\
35825	0.047872\\
42313	0.037234\\
53000	0.026596\\
72389	0.015957\\
1.5403e+05	0.0053191\\
};
\addlegendentry{FRC empirica appr.}

\addplot [color=mycolor1, only marks, every error bar/.append style={opacity=0.30}, mark=*, mark size=0pt, draw=none, forget plot]
 plot [error bars/.cd, x dir=both, x explicit, error bar style={line width=1pt, color=mycolor1}, error mark options={mark=none,mark size=0pt}]
 table[row sep=crcr, x error plus index=2, x error minus index=3]{%
4026.3	0.99468	36.588	36.588\\
4058.6	0.98404	35.959	35.959\\
4099.8	0.9734	36.838	36.838\\
4144	0.96277	37.465	37.465\\
4185.7	0.95213	38.893	38.893\\
4222.6	0.94149	39.784	39.784\\
4258.8	0.93085	40.702	40.702\\
4294	0.92021	40.402	40.402\\
4338.8	0.90957	39.037	39.037\\
4387.5	0.89894	40.059	40.059\\
4434.4	0.8883	39.704	39.704\\
4477.8	0.87766	40.479	40.479\\
4531.1	0.86702	41.553	41.553\\
4583.2	0.85638	42.897	42.897\\
4627.9	0.84574	42.524	42.524\\
4689.2	0.83511	46.655	46.655\\
4733	0.82447	48.383	48.383\\
4787	0.81383	48.919	48.919\\
4843.5	0.80319	48.945	48.945\\
4896.6	0.79255	49.791	49.791\\
4951.8	0.78191	49.148	49.148\\
5010.6	0.77128	47.503	47.503\\
5072.2	0.76064	48.508	48.508\\
5133.1	0.75	50.07	50.07\\
5190.8	0.73936	49.206	49.206\\
5257.5	0.72872	50.721	50.721\\
5330.8	0.71809	50.264	50.264\\
5395.5	0.70745	50.129	50.129\\
5462.9	0.69681	50.129	50.129\\
5544.5	0.68617	52.551	52.551\\
5620.8	0.67553	53.404	53.404\\
5702.2	0.66489	54.765	54.765\\
5790.5	0.65426	55.651	55.651\\
5881.8	0.64362	58.065	58.065\\
5964	0.63298	61.673	61.673\\
6028.3	0.62234	61.598	61.598\\
6133.8	0.6117	62.325	62.325\\
6237	0.60106	65.704	65.704\\
6320	0.59043	68.439	68.439\\
6406.8	0.57979	67.477	67.477\\
6506.9	0.56915	74.772	74.772\\
6616.8	0.55851	75.741	75.741\\
6722.3	0.54787	78.334	78.334\\
6828.5	0.53723	80.276	80.276\\
6953.2	0.5266	85.433	85.433\\
7088.7	0.51596	85.646	85.646\\
7213.5	0.50532	87.296	87.296\\
7325.9	0.49468	90.363	90.363\\
7446.2	0.48404	89.834	89.834\\
7581.1	0.4734	87.181	87.181\\
7724.5	0.46277	85.579	85.579\\
7861.5	0.45213	91.022	91.022\\
8007	0.44149	89.981	89.981\\
8164.3	0.43085	95.299	95.299\\
8339.4	0.42021	90.558	90.558\\
8505.8	0.40957	97.967	97.967\\
8697.7	0.39894	97.569	97.569\\
8854.2	0.3883	97.36	97.36\\
9063	0.37766	101.82	101.82\\
9293.7	0.36702	108.12	108.12\\
9542.3	0.35638	116.55	116.55\\
9795.1	0.34574	116.34	116.34\\
10050	0.33511	122.98	122.98\\
10298	0.32447	127.97	127.97\\
10572	0.31383	139.11	139.11\\
10877	0.30319	147.36	147.36\\
11188	0.29255	147.95	147.95\\
11456	0.28191	158.48	158.48\\
11809	0.27128	170.86	170.86\\
12134	0.26064	180.39	180.39\\
12539	0.25	194.84	194.84\\
12911	0.23936	200.17	200.17\\
13427	0.22872	209.38	209.38\\
13974	0.21809	224.5	224.5\\
14460	0.20745	234.84	234.84\\
15013	0.19681	233.15	233.15\\
15718	0.18617	258.63	258.63\\
16449	0.17553	271.52	271.52\\
17281	0.16489	294.36	294.36\\
18072	0.15426	345.21	345.21\\
18837	0.14362	336.44	336.44\\
19847	0.13298	382.5	382.5\\
20821	0.12234	385.99	385.99\\
22193	0.1117	410.26	410.26\\
23455	0.10106	437.73	437.73\\
24972	0.090426	505.73	505.73\\
26751	0.079787	587.47	587.47\\
29026	0.069149	684.85	684.85\\
31699	0.058511	813.07	813.07\\
35825	0.047872	967.25	967.25\\
42313	0.037234	1232	1232\\
53000	0.026596	1956	1956\\
72389	0.015957	2969.7	2969.7\\
1.5403e+05	0.0053191	5124.6	5124.6\\
};
\addplot [color=mycolor1, mark repeat=9, mark phase=1, mark=+, mark options={solid, mycolor1}]
  table[row sep=crcr]{%
4007.8	0.97865\\
4178.6	0.94382\\
4356.8	0.90082\\
4542.5	0.85485\\
4736.2	0.81065\\
4938.1	0.76874\\
5148.7	0.72901\\
5368.2	0.69133\\
5597.1	0.65561\\
5835.7	0.62174\\
6084.5	0.58962\\
6343.9	0.55917\\
6614.4	0.53029\\
6896.4	0.50291\\
7190.4	0.47695\\
7497	0.45233\\
7816.6	0.42899\\
8149.9	0.40685\\
8497.3	0.38586\\
8859.6	0.36596\\
9237.4	0.34708\\
9631.2	0.32918\\
10042	0.31221\\
10470	0.29612\\
10916	0.28085\\
11382	0.26638\\
11867	0.25265\\
12373	0.23964\\
12900	0.22729\\
13450	0.21559\\
14024	0.20448\\
14622	0.19396\\
15245	0.18397\\
15895	0.1745\\
16573	0.16552\\
17280	0.157\\
18016	0.14893\\
18784	0.14127\\
19585	0.134\\
20420	0.12711\\
21291	0.12057\\
22199	0.11438\\
23145	0.1085\\
24132	0.10292\\
25161	0.097632\\
26233	0.092616\\
27352	0.087859\\
28518	0.083346\\
29734	0.079066\\
31002	0.075006\\
32323	0.071156\\
33701	0.067504\\
35138	0.064039\\
36636	0.060753\\
38198	0.057636\\
39827	0.05468\\
41525	0.051875\\
43295	0.049215\\
45141	0.046692\\
47066	0.044298\\
49072	0.042028\\
51165	0.039874\\
53346	0.037831\\
55621	0.035892\\
57992	0.034054\\
60464	0.03231\\
63042	0.030655\\
65730	0.029085\\
68532	0.027596\\
71454	0.026184\\
74501	0.024844\\
77677	0.023573\\
80989	0.022367\\
84442	0.021222\\
88042	0.020137\\
91796	0.019107\\
95709	0.01813\\
99790	0.017203\\
1.0404e+05	0.016324\\
1.0848e+05	0.01549\\
1.1311e+05	0.014698\\
1.1793e+05	0.013947\\
1.2296e+05	0.013235\\
1.282e+05	0.012559\\
1.3366e+05	0.011918\\
1.3936e+05	0.011309\\
1.453e+05	0.010732\\
1.515e+05	0.010184\\
1.5796e+05	0.0096642\\
1.6469e+05	0.009171\\
1.7171e+05	0.0087031\\
1.7904e+05	0.0082591\\
1.8667e+05	0.0078379\\
1.9463e+05	0.0074382\\
2.0292e+05	0.0070589\\
2.1158e+05	0.006699\\
2.206e+05	0.0063576\\
2.3e+05	0.0060336\\
2.3981e+05	0.0057261\\
2.5003e+05	0.0054343\\
};
\addlegendentry{Adatt. Pareto appr. ($\mathit{ML}$)}


\addplot[area legend, draw=none, fill=mycolor1, fill opacity=0.15, postaction={pattern={Lines[angle=90,distance={5pt},line width={0.5pt}]}, pattern color=mycolor1!60}, forget plot]
table[row sep=crcr] {%
x	y\\
4007.8	0.98443\\
4178.6	0.95203\\
4356.8	0.91008\\
4542.5	0.86353\\
4736.2	0.8185\\
4938.1	0.77583\\
5148.7	0.73541\\
5368.2	0.69711\\
5597.1	0.66082\\
5835.7	0.62645\\
6084.5	0.59388\\
6343.9	0.56303\\
6614.4	0.5338\\
6896.4	0.50612\\
7190.4	0.4799\\
7497	0.45506\\
7816.6	0.43154\\
8149.9	0.40926\\
8497.3	0.38816\\
8859.6	0.36816\\
9237.4	0.34922\\
9631.2	0.33128\\
10042	0.31427\\
10470	0.29815\\
10916	0.28287\\
11382	0.26838\\
11867	0.25465\\
12373	0.24163\\
12900	0.22927\\
13450	0.21756\\
14024	0.20645\\
14622	0.19591\\
15245	0.18591\\
15895	0.17643\\
16573	0.16743\\
17280	0.15889\\
18016	0.1508\\
18784	0.14311\\
19585	0.13582\\
20420	0.1289\\
21291	0.12234\\
22199	0.11611\\
23145	0.1102\\
24132	0.10459\\
25161	0.099274\\
26233	0.094224\\
27352	0.089432\\
28518	0.084885\\
29734	0.080569\\
31002	0.076474\\
32323	0.072588\\
33701	0.0689\\
35138	0.0654\\
36636	0.062078\\
38198	0.058925\\
39827	0.055933\\
41525	0.053094\\
43295	0.050399\\
45141	0.047841\\
47066	0.045414\\
49072	0.04311\\
51165	0.040923\\
53346	0.038848\\
55621	0.036878\\
57992	0.035008\\
60464	0.033233\\
63042	0.031549\\
65730	0.02995\\
68532	0.028433\\
71454	0.026992\\
74501	0.025625\\
77677	0.024327\\
80989	0.023096\\
84442	0.021926\\
88042	0.020816\\
91796	0.019763\\
95709	0.018763\\
99790	0.017814\\
1.0404e+05	0.016912\\
1.0848e+05	0.016057\\
1.1311e+05	0.015245\\
1.1793e+05	0.014474\\
1.2296e+05	0.013742\\
1.282e+05	0.013048\\
1.3366e+05	0.012388\\
1.3936e+05	0.011762\\
1.453e+05	0.011168\\
1.515e+05	0.010604\\
1.5796e+05	0.010068\\
1.6469e+05	0.0095597\\
1.7171e+05	0.009077\\
1.7904e+05	0.0086187\\
1.8667e+05	0.0081837\\
1.9463e+05	0.0077706\\
2.0292e+05	0.0073785\\
2.1158e+05	0.0070062\\
2.206e+05	0.0066528\\
2.3e+05	0.0063172\\
2.3981e+05	0.0059986\\
2.5003e+05	0.0056961\\
2.5003e+05	0.0051726\\
2.3981e+05	0.0054536\\
2.3e+05	0.0057499\\
2.206e+05	0.0060624\\
2.1158e+05	0.0063919\\
2.0292e+05	0.0067393\\
1.9463e+05	0.0071057\\
1.8667e+05	0.0074921\\
1.7904e+05	0.0078996\\
1.7171e+05	0.0083292\\
1.6469e+05	0.0087824\\
1.5796e+05	0.0092602\\
1.515e+05	0.0097641\\
1.453e+05	0.010295\\
1.3936e+05	0.010856\\
1.3366e+05	0.011447\\
1.282e+05	0.01207\\
1.2296e+05	0.012727\\
1.1793e+05	0.013421\\
1.1311e+05	0.014152\\
1.0848e+05	0.014923\\
1.0404e+05	0.015736\\
99790	0.016593\\
95709	0.017498\\
91796	0.018452\\
88042	0.019458\\
84442	0.020519\\
80989	0.021638\\
77677	0.022818\\
74501	0.024063\\
71454	0.025376\\
68532	0.02676\\
65730	0.028221\\
63042	0.029761\\
60464	0.031386\\
57992	0.033099\\
55621	0.034907\\
53346	0.036813\\
51165	0.038824\\
49072	0.040945\\
47066	0.043183\\
45141	0.045542\\
43295	0.048031\\
41525	0.050657\\
39827	0.053426\\
38198	0.056347\\
36636	0.059429\\
35138	0.062679\\
33701	0.066107\\
32323	0.069724\\
31002	0.073539\\
29734	0.077563\\
28518	0.081807\\
27352	0.086285\\
26233	0.091008\\
25161	0.09599\\
24132	0.10125\\
23145	0.10679\\
22199	0.11264\\
21291	0.11881\\
20420	0.12531\\
19585	0.13218\\
18784	0.13942\\
18016	0.14706\\
17280	0.15511\\
16573	0.16361\\
15895	0.17257\\
15245	0.18203\\
14622	0.192\\
14024	0.20252\\
13450	0.21361\\
12900	0.22531\\
12373	0.23765\\
11867	0.25066\\
11382	0.26437\\
10916	0.27884\\
10470	0.29408\\
10042	0.31015\\
9631.2	0.32709\\
9237.4	0.34494\\
8859.6	0.36375\\
8497.3	0.38357\\
8149.9	0.40444\\
7816.6	0.42643\\
7497	0.4496\\
7190.4	0.474\\
6896.4	0.49971\\
6614.4	0.52678\\
6343.9	0.55531\\
6084.5	0.58537\\
5835.7	0.61703\\
5597.1	0.6504\\
5368.2	0.68556\\
5148.7	0.72261\\
4938.1	0.76166\\
4736.2	0.80281\\
4542.5	0.84618\\
4356.8	0.89157\\
4178.6	0.93561\\
4007.8	0.97287\\
}--cycle;
\addplot [color=mycolor1, mark repeat=9, mark phase=1, mark=x, mark options={solid, mycolor1}]
  table[row sep=crcr]{%
4007.8	1.0058\\
4178.6	0.96036\\
4356.8	0.91653\\
4542.5	0.87427\\
4736.2	0.83356\\
4938.1	0.79438\\
5148.7	0.7567\\
5368.2	0.72052\\
5597.1	0.68578\\
5835.7	0.65247\\
6084.5	0.62055\\
6343.9	0.58998\\
6614.4	0.56072\\
6896.4	0.53274\\
7190.4	0.50598\\
7497	0.48043\\
7816.6	0.45602\\
8149.9	0.43272\\
8497.3	0.41049\\
8859.6	0.38928\\
9237.4	0.36906\\
9631.2	0.34978\\
10042	0.33141\\
10470	0.31391\\
10916	0.29724\\
11382	0.28136\\
11867	0.26624\\
12373	0.25184\\
12900	0.23814\\
13450	0.22509\\
14024	0.21268\\
14622	0.20088\\
15245	0.18964\\
15895	0.17896\\
16573	0.1688\\
17280	0.15915\\
18016	0.14997\\
18784	0.14125\\
19585	0.13296\\
20420	0.12509\\
21291	0.11762\\
22199	0.11053\\
23145	0.1038\\
24132	0.097427\\
25161	0.091382\\
26233	0.085656\\
27352	0.080233\\
28518	0.075101\\
29734	0.070248\\
31002	0.06566\\
32323	0.061326\\
33701	0.057235\\
35138	0.053375\\
36636	0.049737\\
38198	0.04631\\
39827	0.043084\\
41525	0.040051\\
43295	0.0372\\
45141	0.034523\\
47066	0.032012\\
49072	0.029659\\
51165	0.027455\\
53346	0.025393\\
55621	0.023466\\
57992	0.021666\\
60464	0.019987\\
63042	0.018422\\
65730	0.016964\\
68532	0.015609\\
71454	0.014349\\
74501	0.013179\\
77677	0.012094\\
80989	0.011088\\
84442	0.010157\\
88042	0.0092967\\
91796	0.0085014\\
95709	0.0077674\\
99790	0.0070907\\
1.0404e+05	0.0064673\\
1.0848e+05	0.0058936\\
1.1311e+05	0.0053662\\
1.1793e+05	0.0048818\\
1.2296e+05	0.0044373\\
1.282e+05	0.0040299\\
1.3366e+05	0.0036568\\
1.3936e+05	0.0033154\\
1.453e+05	0.0030034\\
1.515e+05	0.0027184\\
1.5796e+05	0.0024585\\
1.6469e+05	0.0022216\\
1.7171e+05	0.0020058\\
1.7904e+05	0.0018095\\
1.8667e+05	0.0016311\\
1.9463e+05	0.0014691\\
2.0292e+05	0.0013221\\
2.1158e+05	0.0011889\\
2.206e+05	0.0010682\\
2.3e+05	0.00095904\\
2.3981e+05	0.00086034\\
2.5003e+05	0.00077119\\
};
\addlegendentry{Adatt. BLN appr. ($\mathit{ML}$)}


\addplot[area legend, draw=none, fill=mycolor1, fill opacity=0.15, postaction={pattern={Lines[angle=45,distance={5pt},line width={0.5pt}]}, pattern color=mycolor1!60}, forget plot]
table[row sep=crcr] {%
x	y\\
4007.8	1.0119\\
4178.6	0.9662\\
4356.8	0.92213\\
4542.5	0.87965\\
4736.2	0.83875\\
4938.1	0.7994\\
5148.7	0.76158\\
5368.2	0.72527\\
5597.1	0.69043\\
5835.7	0.65702\\
6084.5	0.62501\\
6343.9	0.59437\\
6614.4	0.56504\\
6896.4	0.53699\\
7190.4	0.51017\\
7497	0.48455\\
7816.6	0.46007\\
8149.9	0.4367\\
8497.3	0.4144\\
8859.6	0.39311\\
9237.4	0.37281\\
9631.2	0.35345\\
10042	0.33499\\
10470	0.31739\\
10916	0.30062\\
11382	0.28465\\
11867	0.26943\\
12373	0.25493\\
12900	0.24113\\
13450	0.22799\\
14024	0.21548\\
14622	0.20357\\
15245	0.19224\\
15895	0.18146\\
16573	0.17121\\
17280	0.16146\\
18016	0.1522\\
18784	0.14339\\
19585	0.13502\\
20420	0.12707\\
21291	0.11953\\
22199	0.11237\\
23145	0.10557\\
24132	0.099129\\
25161	0.093021\\
26233	0.087234\\
27352	0.081755\\
28518	0.076569\\
29734	0.071664\\
31002	0.067027\\
32323	0.062646\\
33701	0.05851\\
35138	0.054608\\
36636	0.050929\\
38198	0.047462\\
39827	0.044198\\
41525	0.041128\\
43295	0.038241\\
45141	0.03553\\
47066	0.032985\\
49072	0.030598\\
51165	0.028361\\
53346	0.026267\\
55621	0.024308\\
57992	0.022477\\
60464	0.020767\\
63042	0.019171\\
65730	0.017683\\
68532	0.016297\\
71454	0.015007\\
74501	0.013808\\
77677	0.012694\\
80989	0.01166\\
84442	0.010701\\
88042	0.0098132\\
91796	0.0089912\\
95709	0.0082311\\
99790	0.007529\\
1.0404e+05	0.0068809\\
1.0848e+05	0.0062833\\
1.1311e+05	0.0057329\\
1.1793e+05	0.0052262\\
1.2296e+05	0.0047604\\
1.282e+05	0.0043325\\
1.3366e+05	0.0039397\\
1.3936e+05	0.0035796\\
1.453e+05	0.0032498\\
1.515e+05	0.0029479\\
1.5796e+05	0.0026719\\
1.6469e+05	0.0024198\\
1.7171e+05	0.0021897\\
1.7904e+05	0.0019799\\
1.8667e+05	0.0017888\\
1.9463e+05	0.0016148\\
2.0292e+05	0.0014567\\
2.1158e+05	0.001313\\
2.206e+05	0.0011825\\
2.3e+05	0.0010642\\
2.3981e+05	0.00095703\\
2.5003e+05	0.00085999\\
2.5003e+05	0.00068239\\
2.3981e+05	0.00076365\\
2.3e+05	0.00085386\\
2.206e+05	0.00095392\\
2.1158e+05	0.0010648\\
2.0292e+05	0.0011876\\
1.9463e+05	0.0013234\\
1.8667e+05	0.0014734\\
1.7904e+05	0.0016391\\
1.7171e+05	0.0018219\\
1.6469e+05	0.0020233\\
1.5796e+05	0.002245\\
1.515e+05	0.0024889\\
1.453e+05	0.0027569\\
1.3936e+05	0.0030511\\
1.3366e+05	0.0033738\\
1.282e+05	0.0037273\\
1.2296e+05	0.0041142\\
1.1793e+05	0.0045373\\
1.1311e+05	0.0049995\\
1.0848e+05	0.0055038\\
1.0404e+05	0.0060536\\
99790	0.0066524\\
95709	0.0073038\\
91796	0.0080116\\
88042	0.0087801\\
84442	0.0096135\\
80989	0.010516\\
77677	0.011493\\
74501	0.01255\\
71454	0.01369\\
68532	0.01492\\
65730	0.016246\\
63042	0.017673\\
60464	0.019207\\
57992	0.020855\\
55621	0.022624\\
53346	0.024519\\
51165	0.026549\\
49072	0.02872\\
47066	0.03104\\
45141	0.033517\\
43295	0.036159\\
41525	0.038974\\
39827	0.04197\\
38198	0.045158\\
36636	0.048545\\
35138	0.052142\\
33701	0.055959\\
32323	0.060006\\
31002	0.064293\\
29734	0.068831\\
28518	0.073633\\
27352	0.078711\\
26233	0.084077\\
25161	0.089744\\
24132	0.095726\\
23145	0.10204\\
22199	0.10869\\
21291	0.11571\\
20420	0.12311\\
19585	0.1309\\
18784	0.1391\\
18016	0.14774\\
17280	0.15683\\
16573	0.1664\\
15895	0.17646\\
15245	0.18705\\
14622	0.19818\\
14024	0.20989\\
13450	0.2222\\
12900	0.23515\\
12373	0.24875\\
11867	0.26304\\
11382	0.27807\\
10916	0.29385\\
10470	0.31043\\
10042	0.32784\\
9631.2	0.34612\\
9237.4	0.36531\\
8859.6	0.38545\\
8497.3	0.40658\\
8149.9	0.42873\\
7816.6	0.45196\\
7497	0.4763\\
7190.4	0.5018\\
6896.4	0.52848\\
6614.4	0.5564\\
6343.9	0.58559\\
6084.5	0.61609\\
5835.7	0.64792\\
5597.1	0.68114\\
5368.2	0.71576\\
5148.7	0.75183\\
4938.1	0.78935\\
4736.2	0.82836\\
4542.5	0.86889\\
4356.8	0.91094\\
4178.6	0.95453\\
4007.8	0.99967\\
}--cycle;
            % \input{../../../.tex/20/KS/SizeDistributionFittingsfig7.tex}
        \end{groupplot}
    
        \begin{groupplot}[
            group style={
                group name=Row4,
                plotLastGroup
            },
            plotRow,
            plotLegendNW,
            %
            xlabel={\(k\)},ylabel={\(s(k)\)},
            xmode=log,ymode=log,
            xmin=10,xmax=283,
            ymin=100,ymax=2.0424e+05,
        ]
            \nextgroupplot[%
                at={($(Row3 c1r1.south west)-(0,\yGroupSep em)-(0,\plotHeight cm)$)},
                % xmin=10,xmax=283,
                % ymin=100,ymax=1.5403e+05,
            ] % This file was created by matlab2tikz.
%
\definecolor{mycolor1}{rgb}{0.00000,0.44706,0.69804}%
\definecolor{mycolor2}{rgb}{0.00000,0.44700,0.74100}%
\definecolor{mycolor3}{rgb}{0.83529,0.36863,0.00000}%
\definecolor{mycolor4}{rgb}{0.85000,0.32500,0.09800}%
%
\addplot[only marks, mark=*, mark size=1.2748pt, color=mycolor1, fill=mycolor1, opacity=0.70, draw=none, mark options={draw=none,line width=0pt}] table[row sep=crcr]{%
x	y\\
34	2204\\
29	1419.2\\
96	13522\\
31	1700.6\\
27	1321.7\\
70	7005.4\\
24	1130.8\\
37	2255.1\\
42	3197\\
16	494.22\\
29	1550.6\\
59	5347.6\\
69	7665.8\\
28	1380.2\\
11	251.04\\
20	826.08\\
45	3483.9\\
27	1356.1\\
22	896.32\\
27	1343.6\\
44	3154.9\\
16	416.75\\
39	2322.3\\
19	624.81\\
33	1754.9\\
37	2247.3\\
33	1736.9\\
18	698.43\\
29	1501.8\\
22	938.97\\
33	1900.6\\
22	820.12\\
49	3829.8\\
25	1137.7\\
43	2759.8\\
26	1142.4\\
31	1697.6\\
26	1234.6\\
18	612.08\\
13	363.43\\
32	1739.9\\
38	2481.6\\
32	1621.8\\
29	1528.5\\
35	2216.9\\
35	2056.4\\
135	22776\\
27	1169.5\\
42	2986.3\\
39	2542.7\\
51	4544.7\\
86	10360\\
15	467.95\\
39	2461.5\\
42	2697.7\\
43	3066.5\\
58	4720.8\\
97	12588\\
50	3905.6\\
20	734.78\\
28	1468.7\\
22	900.81\\
47	3358.1\\
154	28590\\
30	1629.8\\
10	201.96\\
37	2315.8\\
31	1764.1\\
54	4527\\
76	8282\\
55	4833\\
26	1175\\
33	1932.5\\
34	1895.7\\
25	1011.6\\
31	1534.5\\
34	1813.7\\
26	1337.6\\
39	2574.3\\
27	1258.7\\
32	1732.5\\
26	1099.6\\
29	1286\\
23	883.83\\
26	1343.9\\
14	417.2\\
33	1913.8\\
24	1009.8\\
20	804.89\\
51	4413.8\\
35	2120.5\\
29	1456\\
46	3451.1\\
44	3127.2\\
45	3174.9\\
29	1499.8\\
28	1370.2\\
77	8511.8\\
70	7002.3\\
70	7370.1\\
40	2639.4\\
90	11114\\
53	4510.9\\
62	6255.4\\
47	3579.9\\
34	1798.9\\
14	414.32\\
47	4092.3\\
43	2858.7\\
20	739.45\\
28	1395.1\\
80	9350.5\\
25	1077\\
36	1979.8\\
27	1223\\
82	9235.5\\
31	1524.3\\
39	2356.7\\
23	973.44\\
27	1260.4\\
37	2235.3\\
129	22339\\
63	5688.1\\
64	6257.3\\
91	11253\\
27	1190.2\\
26	1180.4\\
17	513.08\\
33	1938.7\\
30	1439.1\\
33	1680.6\\
182	39382\\
26	1092.2\\
48	3546.8\\
46	3210.5\\
10	236.63\\
25	1130.8\\
26	1097.1\\
206	50199\\
37	2271\\
43	3111.3\\
51	3929.2\\
57	5458.4\\
36	1971.6\\
38	2460.5\\
28	1316.5\\
19	602.37\\
36	2189.7\\
45	3167.3\\
53	4265.1\\
48	3379.8\\
64	5260.2\\
65	5828.1\\
45	3153\\
54	4367.4\\
16	531.23\\
24	1004.1\\
127	20709\\
48	3775.8\\
64	6498.4\\
42	2714.7\\
42	2934.6\\
17	615.17\\
48	3830.5\\
42	2739.2\\
20	798.7\\
34	1906.7\\
47	3562.6\\
42	2781.9\\
50	4176.2\\
87	11700\\
64	6034.2\\
32	1867.8\\
39	2505.3\\
35	2189.5\\
18	589.6\\
28	1228.2\\
27	1220.2\\
37	2366.2\\
30	1620.1\\
89	10471\\
22	916.34\\
28	1356\\
25	1016.5\\
15	431.33\\
41	3018.9\\
32	1619\\
34	2049\\
16	483.33\\
88	10508\\
39	2483.4\\
140	23991\\
39	2390\\
49	3731.4\\
72	7646\\
37	2322.2\\
41	2913.5\\
283	80140\\
31	1640.8\\
99	13329\\
103	14584\\
39	2592.4\\
35	2037.3\\
98	12986\\
45	3170.6\\
73	7099.8\\
34	1911.3\\
56	4827.9\\
39	2280.5\\
37	2148.8\\
55	4621.3\\
25	1189.4\\
29	1267.2\\
42	3041.6\\
35	2123.9\\
26	1198\\
45	3467.1\\
56	5112.6\\
39	2671.2\\
62	6339.9\\
94	11820\\
112	18004\\
25	954.26\\
57	4707.1\\
50	4079.6\\
47	3385.9\\
96	13325\\
54	4694.2\\
28	1475.3\\
38	2312.8\\
47	3642.3\\
35	2133.5\\
45	3270.8\\
50	4296.1\\
43	2851.4\\
29	1359.2\\
42	2928.4\\
71	7221.5\\
68	6679.9\\
149	26699\\
67	6692.6\\
51	4006.8\\
34	2098.3\\
120	19892\\
43	2973.8\\
117	19956\\
39	2370.6\\
72	7598.6\\
40	2675.5\\
43	3050.8\\
32	1717.2\\
49	4366.9\\
41	2715.3\\
98	13157\\
96	12755\\
45	3325.8\\
129	21231\\
43	2743\\
93	11214\\
41	2752.9\\
88	10775\\
73	7742.3\\
108	15642\\
52	3938.6\\
18	638.71\\
36	2315.5\\
50	4083.8\\
37	2234.8\\
83	9856.5\\
40	2577.9\\
26	1366\\
40	2448.3\\
49	3565.3\\
27	1199.4\\
41	2650\\
36	2217.7\\
52	4125.4\\
37	1998.5\\
68	6450.2\\
43	2786.6\\
132	22326\\
73	7705.9\\
46	3505.5\\
47	3546.1\\
44	2703.3\\
41	3139.2\\
30	1619.1\\
25	1128.5\\
24	1001.4\\
85	9409.3\\
44	3072.3\\
29	1408.1\\
34	1983\\
101	13874\\
18	608.97\\
21	861.32\\
111	17246\\
95	12600\\
23	1060.4\\
31	1536.5\\
84	10151\\
23	1014.6\\
55	5079.2\\
32	1711.9\\
27	1208.7\\
48	3429\\
18	569.1\\
28	1299.4\\
40	2830.5\\
30	1621.1\\
13	303.77\\
33	1772.6\\
14	470.46\\
52	4267.8\\
65	6344.5\\
49	3912.1\\
35	2308.6\\
109	15767\\
26	993.47\\
41	2571.4\\
39	2735.8\\
61	6253.4\\
40	2329.2\\
36	2197.9\\
21	780.94\\
80	9303.2\\
36	2201.7\\
36	2231.3\\
38	2683.5\\
42	2915.2\\
27	1208.2\\
46	3370.7\\
23	991.23\\
34	2086\\
200	44668\\
37	2388.7\\
16	461.99\\
50	3705.7\\
19	621.95\\
41	2513.9\\
25	1120.5\\
39	2604.5\\
45	2968.5\\
83	9881.1\\
27	1268.4\\
67	7175\\
46	3112.4\\
35	2143\\
44	3300.2\\
30	1615.9\\
29	1402.8\\
23	932.95\\
43	2922.5\\
28	1323.6\\
24	1061\\
56	5089.4\\
33	1872.8\\
13	305.09\\
41	2935.3\\
34	1925.6\\
14	421.81\\
83	9860.6\\
33	1650.9\\
35	2109\\
29	1494.5\\
54	3824.1\\
41	2736\\
19	696.87\\
34	1766.9\\
19	658.17\\
38	2509\\
43	2900.4\\
14	434.27\\
17	538.13\\
12	275.34\\
};
\addlegendentry{Dispersione esatta}

\addplot[only marks, mark=*, mark size=1.2748pt, color=mycolor3, fill=mycolor3, opacity=0.70, draw=none, mark options={draw=none,line width=0pt}] table[row sep=crcr]{%
x	y\\
34	2135.5\\
29	1342.2\\
96	12881\\
31	1551.9\\
27	1296.3\\
70	7487.3\\
24	1072\\
37	2707.5\\
42	2846.8\\
16	431.79\\
29	1471.7\\
59	5225.5\\
69	6778.8\\
28	1313\\
11	250.13\\
20	779.89\\
45	3310.4\\
27	1280.7\\
22	979.64\\
27	1278.3\\
44	3074.2\\
16	497.41\\
39	2743.9\\
19	731.07\\
33	1932.4\\
37	2263.7\\
33	1949.8\\
18	646.58\\
29	1397.2\\
22	913.85\\
33	1857.3\\
22	916.36\\
49	3869.7\\
25	1177.4\\
43	2873.5\\
26	1123.2\\
31	1552.8\\
26	1234.9\\
18	539.74\\
13	369.03\\
32	1749.9\\
38	2509.2\\
32	1667.5\\
29	1508.2\\
35	2028.2\\
35	1918.7\\
135	24100\\
27	1282\\
42	2856.7\\
39	2431.9\\
51	4022\\
86	10068\\
15	490.06\\
39	2473.4\\
42	2811.5\\
43	3389.5\\
58	5089.2\\
97	12304\\
50	4085.2\\
20	774.32\\
28	1460.2\\
22	837.8\\
47	3368.6\\
154	27794\\
30	1490.7\\
10	200.01\\
37	2316\\
31	1573.5\\
54	4657.7\\
76	8212.4\\
55	4633.9\\
26	1251.3\\
33	1994\\
34	1984.8\\
25	1071.2\\
31	1678.3\\
34	1980.4\\
26	1144.6\\
39	2758\\
27	1167.7\\
32	1903.2\\
26	1088.7\\
29	1559.3\\
23	1025.4\\
26	1158.4\\
14	425.08\\
33	1843.9\\
24	1057\\
20	711.58\\
51	3655.8\\
35	2163.5\\
29	1336.4\\
46	3606.3\\
44	3048\\
45	3116.4\\
29	1359.3\\
28	1371.2\\
77	7989.7\\
70	7233.8\\
70	7502.7\\
40	2445.5\\
90	10805\\
53	4494\\
62	5832.6\\
47	3798.6\\
34	2026.1\\
14	351.78\\
47	3486.6\\
43	3045.3\\
20	725.51\\
28	1399\\
80	9010\\
25	1187.3\\
36	2217.3\\
27	1231.9\\
82	9897.3\\
31	1618.5\\
39	2899.5\\
23	821.99\\
27	1226.8\\
37	2341.9\\
129	21861\\
63	5980.2\\
64	5525.2\\
91	11909\\
27	1186.8\\
26	1243.4\\
17	579.27\\
33	2015.8\\
30	1555.7\\
33	1714.6\\
182	42427\\
26	1198\\
48	3857.6\\
46	3271.2\\
10	214.41\\
25	1143.9\\
26	1195.4\\
206	49461\\
37	2356\\
43	2799\\
51	3998.8\\
57	5250.3\\
36	2049.7\\
38	2310.4\\
28	1480.1\\
19	677.65\\
36	2223.9\\
45	3143.7\\
53	4262.9\\
48	3906.7\\
64	5807.9\\
65	6115.7\\
45	3078.5\\
54	4177.6\\
16	495.92\\
24	982.13\\
127	20514\\
48	3669\\
64	6356.3\\
42	2840.7\\
42	2785.4\\
17	532.26\\
48	3649.1\\
42	2449.1\\
20	767.31\\
34	2024\\
47	3400.9\\
42	2914.1\\
50	3853.9\\
87	10973\\
64	6165.9\\
32	1581.6\\
39	2311.6\\
35	2115.6\\
18	640.13\\
28	1395\\
27	1314.7\\
37	1977.8\\
30	1508.4\\
89	11897\\
22	868.13\\
28	1436\\
25	1194\\
15	443.08\\
41	2669.8\\
32	1924.7\\
34	1853.5\\
16	571.46\\
88	10377\\
39	2187.3\\
140	23859\\
39	2718.4\\
49	3448.4\\
72	6859\\
37	2170.9\\
41	2654.9\\
283	84255\\
31	1594.6\\
99	13494\\
103	15522\\
39	2369.1\\
35	1998.8\\
98	12600\\
45	3318.8\\
73	7311.9\\
34	1914.1\\
56	4944.2\\
39	2639.5\\
37	2345.9\\
55	4357\\
25	1106.5\\
29	1498.3\\
42	2454.8\\
35	2051.9\\
26	1182.4\\
45	3179.4\\
56	4816.1\\
39	2328.7\\
62	6282.3\\
94	12206\\
112	16351\\
25	1139.3\\
57	4888.9\\
50	3925.4\\
47	3691.8\\
96	12936\\
54	4575.7\\
28	1545.5\\
38	2500\\
47	3751.2\\
35	2098.4\\
45	3455.9\\
50	3835.7\\
43	3023.9\\
29	1430.7\\
42	2835\\
71	7520.2\\
68	6848.1\\
149	26776\\
67	6901.1\\
51	3717.3\\
34	1926.8\\
120	19897\\
43	3000.1\\
117	18050\\
39	2637.4\\
72	7382.1\\
40	2514.7\\
43	3225.1\\
32	1900.9\\
49	3757.7\\
41	2869.6\\
98	12090\\
96	12557\\
45	3641.2\\
129	23130\\
43	2889.8\\
93	11218\\
41	2584.7\\
88	10506\\
73	7758\\
108	14978\\
52	4557.2\\
18	632.81\\
36	2082.3\\
50	4182.8\\
37	2169.4\\
83	10037\\
40	2538.1\\
26	1159.7\\
40	2687.8\\
49	4139.4\\
27	1396.9\\
41	2725.2\\
36	2137.8\\
52	4286.7\\
37	2151.3\\
68	6743.6\\
43	2741\\
132	23340\\
73	8146.9\\
46	3439.8\\
47	3400\\
44	3203\\
41	2568.3\\
30	1505\\
25	1023.8\\
24	1065.6\\
85	10803\\
44	3146.6\\
29	1454\\
34	1908.4\\
101	13076\\
18	668.45\\
21	731.34\\
111	15602\\
95	11531\\
23	824.24\\
31	1677.2\\
84	10359\\
23	1001.8\\
55	5332\\
32	1743.7\\
27	1213.9\\
48	3322.1\\
18	581.63\\
28	1330.2\\
40	2418.6\\
30	1610.4\\
13	324.58\\
33	2055.2\\
14	385.93\\
52	4220.3\\
65	6696.1\\
49	3730.8\\
35	2070.9\\
109	16476\\
26	1166.6\\
41	2728.9\\
39	2654.9\\
61	5632\\
40	2864.4\\
36	2108.4\\
21	789.92\\
80	9178.8\\
36	2283.6\\
36	2109.8\\
38	2294.8\\
42	2768.9\\
27	1159.4\\
46	3227.2\\
23	890.85\\
34	1956.6\\
200	44048\\
37	2282.4\\
16	453.34\\
50	3864.2\\
19	735.41\\
41	2977.6\\
25	1134.2\\
39	2378.7\\
45	3377\\
83	9862\\
27	1254.9\\
67	6476\\
46	3415\\
35	1931.4\\
44	3043.5\\
30	1600\\
29	1521.7\\
23	1082.8\\
43	2902\\
28	1384.6\\
24	992.03\\
56	4589.5\\
33	1832.8\\
13	306.02\\
41	2836.1\\
34	1873.4\\
14	392.34\\
83	9650.2\\
33	1945.1\\
35	1961.1\\
29	1434.7\\
54	4522.3\\
41	2762\\
19	690.3\\
34	1843.6\\
19	644.05\\
38	2458\\
43	2764.6\\
14	392.9\\
17	575.66\\
12	285.16\\
};
\addlegendentry{Dispersione appr.}

            % \input{../../../.tex/20/KS/SizeVsDegreefig1.tex}
    
            \nextgroupplot[%
                % xmin=10,xmax=283,
                % ymin=100,ymax=1e+06,
            ] % This file was created by matlab2tikz.
%
\definecolor{mycolor1}{rgb}{0.00000,0.44706,0.69804}%
\definecolor{mycolor2}{rgb}{0.00000,0.44700,0.74100}%
\definecolor{mycolor3}{rgb}{0.85000,0.32500,0.09800}%
%
\addplot[only marks, mark=*, mark size=1.3919pt, color=mycolor1, fill=mycolor1, opacity=0.60, draw=none, mark options={draw=none,line width=0pt}] table[row sep=crcr]{%
x	y\\
34	2792.9\\
29	1461.2\\
96	17740\\
31	1841\\
27	1284.2\\
70	7413.7\\
24	1296.8\\
37	2086.3\\
42	2822.2\\
16	1121.2\\
29	1448.3\\
59	3517.6\\
69	3084.8\\
28	1666.2\\
11	316.43\\
20	1049.1\\
45	3669.5\\
27	1456\\
22	926.17\\
27	1339.8\\
44	5307.2\\
16	1660.6\\
39	4632.6\\
19	1253.3\\
33	2123.8\\
37	2536.8\\
33	1427.6\\
18	559.44\\
29	2346.3\\
22	1056.1\\
33	1455.6\\
22	853.89\\
49	8060.2\\
25	1557.4\\
43	3728.7\\
26	2498.1\\
31	3183.8\\
26	1283.1\\
18	928.69\\
13	531.65\\
32	4009.8\\
38	2341.5\\
32	3526.7\\
29	1847.1\\
35	1522.8\\
35	2959.3\\
135	21037\\
27	4250\\
42	4002.8\\
39	4585.9\\
51	7450.7\\
86	10546\\
15	958.4\\
39	4667.8\\
42	3524.4\\
43	3228.3\\
58	5888.3\\
97	25809\\
50	2364.1\\
20	1551.5\\
28	1180\\
22	1123.4\\
47	3198.2\\
154	61690\\
30	1800.9\\
10	476.02\\
37	7865.5\\
31	1732.5\\
54	14568\\
76	12463\\
55	4780.6\\
26	2590.9\\
33	1632.8\\
34	2090.4\\
25	1707.6\\
31	3693.4\\
34	4483.4\\
26	2189.6\\
39	3281\\
27	2696.9\\
32	1911.2\\
26	2267.6\\
29	2191.6\\
23	3101.7\\
26	2424.1\\
14	749.54\\
33	1965.1\\
24	969.75\\
20	1584.6\\
51	2047.9\\
35	3238\\
29	1268.3\\
46	1012.5\\
44	3591.5\\
45	3179.7\\
29	876.9\\
28	1678.3\\
77	3058.1\\
70	3893.3\\
70	3527.1\\
40	2561.9\\
90	5380.7\\
53	3920.4\\
62	1729.1\\
47	3658.2\\
34	1287.7\\
14	1023.3\\
47	2152.9\\
43	1029.3\\
20	773.59\\
28	1035.6\\
80	3048.4\\
25	559.24\\
36	1782.7\\
27	2168\\
82	4324.6\\
31	1099.3\\
39	1321.7\\
23	2157.3\\
27	2729.2\\
37	2212.2\\
129	8349.8\\
63	4553.7\\
64	1883\\
91	9154.3\\
27	1235.3\\
26	1276.6\\
17	753.82\\
33	1498.6\\
30	3090.4\\
33	1373.3\\
182	13735\\
26	1143.7\\
48	3081.7\\
46	1721.4\\
10	341.85\\
25	1199.2\\
26	825.11\\
206	25437\\
37	1207\\
43	1644.7\\
51	1967.5\\
57	6553.8\\
36	1815.8\\
38	1496\\
28	1388.6\\
19	1119.3\\
36	1547.1\\
45	3206.7\\
53	3502.1\\
48	3538.3\\
64	3162.7\\
65	2906\\
45	1333.3\\
54	2937.4\\
16	447.02\\
24	1107.1\\
127	8840.3\\
48	1831.9\\
64	2648.7\\
42	2839.5\\
42	1277.9\\
17	548.47\\
48	2816.9\\
42	2354.5\\
20	822.18\\
34	1128.5\\
47	989.88\\
42	2987\\
50	2293.3\\
87	6890.7\\
64	4044.4\\
32	1848.4\\
39	1189.2\\
35	2259.7\\
18	1096.8\\
28	964.82\\
27	892.5\\
37	2182\\
30	2481.7\\
89	11365\\
22	1876.3\\
28	1709.8\\
25	1245.4\\
15	603.61\\
41	3353.8\\
32	965.73\\
34	1941.2\\
16	777.61\\
88	6155.1\\
39	971.74\\
140	28355\\
39	1419.8\\
49	1914.8\\
72	2341.1\\
37	1657.1\\
41	3235.4\\
283	1.4911e+05\\
31	3074.5\\
99	18404\\
103	25946\\
39	3634.2\\
35	1688.7\\
98	13675\\
45	4322.5\\
73	8846.5\\
34	2226\\
56	8632.5\\
39	3146.1\\
37	2833.3\\
55	2350.2\\
25	727.72\\
29	1057.7\\
42	1232.3\\
35	3687.2\\
26	704.59\\
45	8540.1\\
56	6196.3\\
39	1517.5\\
62	3177.4\\
94	11954\\
112	24603\\
25	763.44\\
57	2394.6\\
50	2472.1\\
47	6309.8\\
96	6803\\
54	6535.5\\
28	2492.1\\
38	3660.2\\
47	3766\\
35	1989.3\\
45	1582.7\\
50	3799.9\\
43	1022.3\\
29	2097.1\\
42	1715.3\\
71	21267\\
68	6977.2\\
149	56314\\
67	5440.5\\
51	2841.7\\
34	1911.6\\
120	15030\\
43	7027.6\\
117	11737\\
39	1344.2\\
72	8082.9\\
40	3640.7\\
43	2307.2\\
32	2692.1\\
49	8621.3\\
41	3736.3\\
98	4900.1\\
96	11706\\
45	1674.9\\
129	34122\\
43	1978.4\\
93	8298.9\\
41	3445.7\\
88	10586\\
73	5801.6\\
108	16371\\
52	7741.2\\
18	413.78\\
36	1080.4\\
50	5419.3\\
37	2342.3\\
83	14389\\
40	2113.4\\
26	2404.1\\
40	1716.3\\
49	4315.8\\
27	1348\\
41	1402.6\\
36	918.64\\
52	5235.2\\
37	1030.1\\
68	7059\\
43	2343.2\\
132	14126\\
73	2991.4\\
46	4674.1\\
47	1179\\
44	1658.9\\
41	4812.7\\
30	1525.1\\
25	2873.9\\
24	1863.3\\
85	7690.3\\
44	2933.9\\
29	1795.1\\
34	1587.8\\
101	14161\\
18	1641.5\\
21	1386.1\\
111	16407\\
95	4706.6\\
23	1086.5\\
31	631.56\\
84	4251.8\\
23	847.99\\
55	4700\\
32	1171\\
27	835.69\\
48	880.94\\
18	365.55\\
28	1830.6\\
40	1133.6\\
30	1237.5\\
13	364.11\\
33	1634.7\\
14	555.67\\
52	2371.1\\
65	7456.3\\
49	2469.2\\
35	1344.7\\
109	6966.6\\
26	793.34\\
41	1319.3\\
39	1253.4\\
61	5936.9\\
40	1510.6\\
36	2086.5\\
21	594.12\\
80	4147.5\\
36	1272.1\\
36	2002.8\\
38	1140.6\\
42	2199.9\\
27	767.59\\
46	2351.8\\
23	631.19\\
34	1761.8\\
200	41377\\
37	2540.5\\
16	606.82\\
50	2176.7\\
19	661.74\\
41	2790.1\\
25	823.13\\
39	2005.5\\
45	2965.9\\
83	7188\\
27	595.27\\
67	2573.5\\
46	3338\\
35	1627\\
44	2367.6\\
30	1746.4\\
29	1119.9\\
23	511.76\\
43	2186.9\\
28	798.17\\
24	1210.4\\
56	3217.9\\
33	988.27\\
13	430.76\\
41	2929.4\\
34	900.86\\
14	434\\
83	8213.4\\
33	1569.8\\
35	1624.4\\
29	1040.3\\
54	3609.5\\
41	1429.5\\
19	678.71\\
34	2139.5\\
19	723.93\\
38	2002.5\\
43	1751.1\\
14	620.02\\
17	763.96\\
12	451.63\\
};
\addlegendentry{Dispersione esatta}

\addplot[only marks, mark=*, mark size=1.3919pt, color=black, fill=black, opacity=0.60, draw=none, mark options={draw=none,line width=0pt}] table[row sep=crcr]{%
x	y\\
34	1787\\
29	2052\\
96	39026\\
31	939\\
27	839\\
70	9435\\
24	756\\
37	2292\\
42	3353\\
16	509\\
29	1198\\
59	4061\\
69	4632\\
28	987\\
11	377\\
20	825\\
45	6367\\
27	1370\\
22	689\\
27	1107\\
44	4679\\
16	622\\
39	5236\\
19	558\\
33	1997\\
37	1291\\
33	1077\\
18	485\\
29	1640\\
22	758\\
33	1231\\
22	643\\
49	9267\\
25	1115\\
43	11048\\
26	1875\\
31	2762\\
26	927\\
18	665\\
13	140\\
32	2698\\
38	2211\\
32	775\\
29	1173\\
35	1719\\
35	3078\\
135	41095\\
27	2591\\
42	3900\\
39	3847\\
51	5607\\
86	11830\\
15	918\\
39	3169\\
42	3772\\
43	2489\\
58	4861\\
97	21264\\
50	3266\\
20	736\\
28	686\\
22	1102\\
47	4024\\
154	1.2234e+05\\
30	1543\\
10	296\\
37	7252\\
31	1142\\
54	13398\\
76	13899\\
55	3344\\
26	1499\\
33	1121\\
34	1971\\
25	1709\\
31	2961\\
34	3625\\
26	2774\\
39	3551\\
27	1922\\
32	1860\\
26	1781\\
29	1942\\
23	2014\\
26	1636\\
14	530\\
33	1466\\
24	787\\
20	1114\\
51	1692\\
35	2940\\
29	1332\\
46	1054\\
44	3996\\
45	4071\\
29	803\\
28	557\\
77	3928\\
70	3625\\
70	2531\\
40	1723\\
90	8518\\
53	3650\\
62	3213\\
47	8035\\
34	817\\
14	526\\
47	2742\\
43	759\\
20	950\\
28	533\\
80	4654\\
25	1153\\
36	1716\\
27	2299\\
82	3021\\
31	1103\\
39	1587\\
23	810\\
27	2388\\
37	2269\\
129	3241\\
63	3568\\
64	2459\\
91	6356\\
27	696\\
26	1459\\
17	558\\
33	2465\\
30	2046\\
33	1791\\
182	11424\\
26	663\\
48	2633\\
46	2157\\
10	222\\
25	802\\
26	424\\
206	37527\\
37	1069\\
43	1448\\
51	2715\\
57	7724\\
36	1800\\
38	1162\\
28	539\\
19	746\\
36	1003\\
45	3212\\
53	4779\\
48	5264\\
64	2484\\
65	3084\\
45	1597\\
54	3279\\
16	299\\
24	1131\\
127	2601\\
48	1783\\
64	2544\\
42	2057\\
42	1206\\
17	254\\
48	2507\\
42	2016\\
20	816\\
34	1805\\
47	1076\\
42	2516\\
50	2233\\
87	10377\\
64	2072\\
32	1299\\
39	1215\\
35	3735\\
18	866\\
28	626\\
27	253\\
37	2518\\
30	2667\\
89	9128\\
22	1204\\
28	1732\\
25	1513\\
15	873\\
41	3761\\
32	1140\\
34	1426\\
16	352\\
88	7596\\
39	668\\
140	20491\\
39	1086\\
49	1025\\
72	1475\\
37	1233\\
41	3002\\
283	2.0424e+05\\
31	2681\\
99	16428\\
103	32887\\
39	6629\\
35	1076\\
98	6332\\
45	4113\\
73	7877\\
34	1444\\
56	6926\\
39	2045\\
37	3241\\
55	1793\\
25	444\\
29	1020\\
42	1480\\
35	2286\\
26	591\\
45	5458\\
56	7320\\
39	1112\\
62	3063\\
94	13380\\
112	30134\\
25	317\\
57	1896\\
50	2648\\
47	5982\\
96	4539\\
54	4348\\
28	1451\\
38	3579\\
47	3124\\
35	1834\\
45	921\\
50	3106\\
43	719\\
29	1516\\
42	1249\\
71	5868\\
68	5857\\
149	61636\\
67	5463\\
51	1708\\
34	1459\\
120	10119\\
43	6342\\
117	8499\\
39	1041\\
72	6468\\
40	4061\\
43	1854\\
32	2516\\
49	12313\\
41	4009\\
98	4503\\
96	5379\\
45	1383\\
129	23237\\
43	1506\\
93	4228\\
41	2160\\
88	9837\\
73	5327\\
108	12182\\
52	5524\\
18	189\\
36	878\\
50	4430\\
37	1383\\
83	13086\\
40	2202\\
26	1338\\
40	1198\\
49	4702\\
27	1189\\
41	1263\\
36	572\\
52	3610\\
37	656\\
68	6317\\
43	2092\\
132	14984\\
73	3147\\
46	3825\\
47	739\\
44	1624\\
41	5048\\
30	1539\\
25	1574\\
24	2598\\
85	7294\\
44	1753\\
29	1546\\
34	1160\\
101	9599\\
18	1226\\
21	992\\
111	7348\\
95	2640\\
23	617\\
31	310\\
84	1691\\
23	466\\
55	3785\\
32	1352\\
27	544\\
48	517\\
18	119\\
28	1203\\
40	962\\
30	685\\
13	188\\
33	1762\\
14	209\\
52	1836\\
65	8994\\
49	3401\\
35	1176\\
109	4663\\
26	550\\
41	1006\\
39	973\\
61	4953\\
40	1238\\
36	1694\\
21	551\\
80	5029\\
36	1037\\
36	1691\\
38	832\\
42	1192\\
27	590\\
46	1533\\
23	486\\
34	1267\\
200	30990\\
37	1240\\
16	413\\
50	2688\\
19	324\\
41	2143\\
25	934\\
39	3741\\
45	2869\\
83	3945\\
27	521\\
67	2908\\
46	2419\\
35	1912\\
44	2676\\
30	2051\\
29	674\\
23	204\\
43	936\\
28	859\\
24	467\\
56	2173\\
33	679\\
13	261\\
41	2629\\
34	588\\
14	216\\
83	10336\\
33	1005\\
35	1380\\
29	687\\
54	3377\\
41	993\\
19	351\\
34	1937\\
19	459\\
38	1126\\
43	1196\\
14	412\\
17	325\\
12	184\\
};
\addlegendentry{Dispersione reale}

            % \input{../../../.tex/20/KS/SizeVsDegreefig2.tex}
    
            \nextgroupplot[%
                % xmin=10,xmax=283,
                % ymin=100,ymax=1e+06,
            ] % This file was created by matlab2tikz.
%
\definecolor{mycolor1}{rgb}{0.83529,0.36863,0.00000}%
\definecolor{mycolor2}{rgb}{0.00000,0.44700,0.74100}%
\definecolor{mycolor3}{rgb}{0.85000,0.32500,0.09800}%
%
\addplot[only marks, mark=*, mark size=1.2748pt, color=mycolor1, fill=mycolor1, opacity=0.60, draw=none, mark options={draw=none,line width=0pt}] table[row sep=crcr]{%
x	y\\
34	1181.2\\
29	195.19\\
96	1176.9\\
31	1213.6\\
27	581.44\\
70	3.1994e-05\\
24	458.2\\
37	280.3\\
42	85.348\\
16	1368.2\\
29	73.346\\
59	896.27\\
69	3138.3\\
28	305.75\\
11	1680.2\\
20	630.02\\
45	1079.4\\
27	43.186\\
22	131.02\\
27	815.82\\
44	0.35379\\
16	1404.1\\
39	578.34\\
19	1956\\
33	308.89\\
37	501.48\\
33	175.8\\
18	177.57\\
29	823.04\\
22	3054.5\\
33	125.76\\
22	1363\\
49	109.54\\
25	1422\\
43	9.9387\\
26	215.41\\
31	415.4\\
26	810.41\\
18	459.63\\
13	1107.2\\
32	984.98\\
38	2178.1\\
32	532.51\\
29	12.558\\
35	103.58\\
35	118.63\\
135	87359\\
27	148.1\\
42	1381.3\\
39	36.64\\
51	18.392\\
86	133.31\\
15	1116.7\\
39	489.1\\
42	189.17\\
43	831.81\\
58	0.08224\\
97	0.07288\\
50	374.99\\
20	1652.1\\
28	226.97\\
22	61.147\\
47	178.16\\
154	86610\\
30	312.63\\
10	1134\\
37	113.98\\
31	541.06\\
54	3.0622\\
76	4.5027e-08\\
55	5026.2\\
26	752.17\\
33	116.44\\
34	4409\\
25	835.86\\
31	588.99\\
34	1.9694\\
26	823.56\\
39	782.2\\
27	396.59\\
32	1201\\
26	182.32\\
29	759.85\\
23	737.33\\
26	243.05\\
14	1311.3\\
33	1720.9\\
24	1467\\
20	1998.8\\
51	415.38\\
35	9.5976\\
29	125.4\\
46	58.071\\
44	3917.9\\
45	1.1202\\
29	368.5\\
28	416.25\\
77	13516\\
70	1.4534e-08\\
70	392.48\\
40	17.843\\
90	2456.3\\
53	2.4597\\
62	1464.2\\
47	1256.5\\
34	545.86\\
14	893.36\\
47	6.7683\\
43	60.621\\
20	1093.9\\
28	328.06\\
80	53.553\\
25	830.37\\
36	2192.7\\
27	1714.2\\
82	6876.4\\
31	2537.4\\
39	0.44564\\
23	1412.2\\
27	397.05\\
37	1152.6\\
129	71760\\
63	2343.9\\
64	1395.1\\
91	17880\\
27	137.9\\
26	228.12\\
17	1865.6\\
33	212.93\\
30	159.61\\
33	6.5491\\
182	34862\\
26	648.23\\
48	0.036691\\
46	21.586\\
10	2122.1\\
25	49.548\\
26	1245.6\\
206	1.4774e+05\\
37	324.38\\
43	403.07\\
51	3.3213\\
57	879.85\\
36	505.3\\
38	204.28\\
28	55.571\\
19	812.89\\
36	86.713\\
45	3637.2\\
53	1457.3\\
48	158.7\\
64	15055\\
65	2785.8\\
45	11.749\\
54	32.631\\
16	1245.5\\
24	2076.3\\
127	51116\\
48	2163.3\\
64	1052.9\\
42	176.19\\
42	425.97\\
17	768.53\\
48	150.67\\
42	492.37\\
20	1031.8\\
34	76.142\\
47	1891.8\\
42	525.39\\
50	265.14\\
87	14638\\
64	2.4447\\
32	132.02\\
39	989.09\\
35	2352.5\\
18	866.32\\
28	595.75\\
27	2768\\
37	162.16\\
30	322.08\\
89	227\\
22	301.09\\
28	769.06\\
25	744.28\\
15	787.04\\
41	0.69486\\
32	262.98\\
34	11.335\\
16	794.28\\
88	3975.2\\
39	7.3249\\
140	63102\\
39	456.3\\
49	59.401\\
72	8602.3\\
37	0.0108\\
41	704.35\\
283	2.7648e+05\\
31	138.99\\
99	3115.9\\
103	2301.6\\
39	3543.5\\
35	348.94\\
98	3374.2\\
45	0.091589\\
73	5.9721\\
34	41.234\\
56	13.169\\
39	1060.3\\
37	1121.3\\
55	1849.2\\
25	397.8\\
29	2163\\
42	632.75\\
35	1055.9\\
26	2011.1\\
45	1821.5\\
56	225\\
39	647.68\\
62	8510.8\\
94	16255\\
112	1058.7\\
25	151.36\\
57	1208.1\\
50	925.94\\
47	18.72\\
96	8007.6\\
54	1374\\
28	1322.3\\
38	222.96\\
47	3.1997\\
35	0.032899\\
45	755.58\\
50	3.0507e-06\\
43	1489.7\\
29	290.69\\
42	2782.7\\
71	5512.6\\
68	3.7236e-09\\
149	45416\\
67	68.103\\
51	6735.5\\
34	27.974\\
120	745.68\\
43	200.13\\
117	27500\\
39	0.013448\\
72	72.469\\
40	0.54523\\
43	669.77\\
32	152.04\\
49	225.83\\
41	622.71\\
98	14402\\
96	28670\\
45	326.76\\
129	4407.9\\
43	1935.4\\
93	3397.7\\
41	5624.1\\
88	551.54\\
73	831.05\\
108	1894.9\\
52	5197.3\\
18	698.64\\
36	648.35\\
50	104.79\\
37	13.378\\
83	12835\\
40	9.7425\\
26	79.957\\
40	49.923\\
49	6.2296\\
27	297.95\\
41	2364.4\\
36	70.463\\
52	477.05\\
37	40.655\\
68	81.061\\
43	1612\\
132	18.154\\
73	4189.6\\
46	175.32\\
47	42.367\\
44	4336.2\\
41	286.22\\
30	96.768\\
25	1024\\
24	635.43\\
85	14451\\
44	206.56\\
29	597.51\\
34	118.05\\
101	23703\\
18	1005.6\\
21	755.62\\
111	17084\\
95	14246\\
23	1300\\
31	1260.1\\
84	4433.8\\
23	370.56\\
55	4530.3\\
32	1653.9\\
27	660.23\\
48	49.592\\
18	1071.4\\
28	515\\
40	293.07\\
30	174.92\\
13	672.62\\
33	1596.4\\
14	971.92\\
52	39.254\\
65	1206.5\\
49	14.105\\
35	51.535\\
109	18387\\
26	37.288\\
41	332.69\\
39	543.1\\
61	0.00041025\\
40	379.36\\
36	1804\\
21	574.01\\
80	11350\\
36	499.34\\
36	1415\\
38	116.83\\
42	5406.2\\
27	1.3599\\
46	26.817\\
23	593.55\\
34	48.534\\
200	1.9122e+05\\
37	1010.1\\
16	1613.3\\
50	900.68\\
19	232.1\\
41	725.69\\
25	408.71\\
39	1908.8\\
45	671.98\\
83	159.06\\
27	398.22\\
67	1304.5\\
46	1557\\
35	12.362\\
44	471.02\\
30	262.69\\
29	832.8\\
23	190.38\\
43	317.99\\
28	2044\\
24	1702.2\\
56	5.2612\\
33	169.68\\
13	1622.4\\
41	566.78\\
34	386.21\\
14	1496.2\\
83	412.96\\
33	46.589\\
35	428.22\\
29	26.538\\
54	1659.2\\
41	170.62\\
19	1763.1\\
34	2.5268\\
19	588.44\\
38	1.595\\
43	100.76\\
14	1391.5\\
17	676.93\\
12	889.5\\
};
\addlegendentry{Dispersione appr.}

\addplot[only marks, mark=*, mark size=1.2748pt, color=black, fill=black, opacity=0.60, draw=none, mark options={draw=none,line width=0pt}] table[row sep=crcr]{%
x	y\\
34	1787\\
29	2052\\
96	39026\\
31	939\\
27	839\\
70	9435\\
24	756\\
37	2292\\
42	3353\\
16	509\\
29	1198\\
59	4061\\
69	4632\\
28	987\\
11	377\\
20	825\\
45	6367\\
27	1370\\
22	689\\
27	1107\\
44	4679\\
16	622\\
39	5236\\
19	558\\
33	1997\\
37	1291\\
33	1077\\
18	485\\
29	1640\\
22	758\\
33	1231\\
22	643\\
49	9267\\
25	1115\\
43	11048\\
26	1875\\
31	2762\\
26	927\\
18	665\\
13	140\\
32	2698\\
38	2211\\
32	775\\
29	1173\\
35	1719\\
35	3078\\
135	41095\\
27	2591\\
42	3900\\
39	3847\\
51	5607\\
86	11830\\
15	918\\
39	3169\\
42	3772\\
43	2489\\
58	4861\\
97	21264\\
50	3266\\
20	736\\
28	686\\
22	1102\\
47	4024\\
154	1.2234e+05\\
30	1543\\
10	296\\
37	7252\\
31	1142\\
54	13398\\
76	13899\\
55	3344\\
26	1499\\
33	1121\\
34	1971\\
25	1709\\
31	2961\\
34	3625\\
26	2774\\
39	3551\\
27	1922\\
32	1860\\
26	1781\\
29	1942\\
23	2014\\
26	1636\\
14	530\\
33	1466\\
24	787\\
20	1114\\
51	1692\\
35	2940\\
29	1332\\
46	1054\\
44	3996\\
45	4071\\
29	803\\
28	557\\
77	3928\\
70	3625\\
70	2531\\
40	1723\\
90	8518\\
53	3650\\
62	3213\\
47	8035\\
34	817\\
14	526\\
47	2742\\
43	759\\
20	950\\
28	533\\
80	4654\\
25	1153\\
36	1716\\
27	2299\\
82	3021\\
31	1103\\
39	1587\\
23	810\\
27	2388\\
37	2269\\
129	3241\\
63	3568\\
64	2459\\
91	6356\\
27	696\\
26	1459\\
17	558\\
33	2465\\
30	2046\\
33	1791\\
182	11424\\
26	663\\
48	2633\\
46	2157\\
10	222\\
25	802\\
26	424\\
206	37527\\
37	1069\\
43	1448\\
51	2715\\
57	7724\\
36	1800\\
38	1162\\
28	539\\
19	746\\
36	1003\\
45	3212\\
53	4779\\
48	5264\\
64	2484\\
65	3084\\
45	1597\\
54	3279\\
16	299\\
24	1131\\
127	2601\\
48	1783\\
64	2544\\
42	2057\\
42	1206\\
17	254\\
48	2507\\
42	2016\\
20	816\\
34	1805\\
47	1076\\
42	2516\\
50	2233\\
87	10377\\
64	2072\\
32	1299\\
39	1215\\
35	3735\\
18	866\\
28	626\\
27	253\\
37	2518\\
30	2667\\
89	9128\\
22	1204\\
28	1732\\
25	1513\\
15	873\\
41	3761\\
32	1140\\
34	1426\\
16	352\\
88	7596\\
39	668\\
140	20491\\
39	1086\\
49	1025\\
72	1475\\
37	1233\\
41	3002\\
283	2.0424e+05\\
31	2681\\
99	16428\\
103	32887\\
39	6629\\
35	1076\\
98	6332\\
45	4113\\
73	7877\\
34	1444\\
56	6926\\
39	2045\\
37	3241\\
55	1793\\
25	444\\
29	1020\\
42	1480\\
35	2286\\
26	591\\
45	5458\\
56	7320\\
39	1112\\
62	3063\\
94	13380\\
112	30134\\
25	317\\
57	1896\\
50	2648\\
47	5982\\
96	4539\\
54	4348\\
28	1451\\
38	3579\\
47	3124\\
35	1834\\
45	921\\
50	3106\\
43	719\\
29	1516\\
42	1249\\
71	5868\\
68	5857\\
149	61636\\
67	5463\\
51	1708\\
34	1459\\
120	10119\\
43	6342\\
117	8499\\
39	1041\\
72	6468\\
40	4061\\
43	1854\\
32	2516\\
49	12313\\
41	4009\\
98	4503\\
96	5379\\
45	1383\\
129	23237\\
43	1506\\
93	4228\\
41	2160\\
88	9837\\
73	5327\\
108	12182\\
52	5524\\
18	189\\
36	878\\
50	4430\\
37	1383\\
83	13086\\
40	2202\\
26	1338\\
40	1198\\
49	4702\\
27	1189\\
41	1263\\
36	572\\
52	3610\\
37	656\\
68	6317\\
43	2092\\
132	14984\\
73	3147\\
46	3825\\
47	739\\
44	1624\\
41	5048\\
30	1539\\
25	1574\\
24	2598\\
85	7294\\
44	1753\\
29	1546\\
34	1160\\
101	9599\\
18	1226\\
21	992\\
111	7348\\
95	2640\\
23	617\\
31	310\\
84	1691\\
23	466\\
55	3785\\
32	1352\\
27	544\\
48	517\\
18	119\\
28	1203\\
40	962\\
30	685\\
13	188\\
33	1762\\
14	209\\
52	1836\\
65	8994\\
49	3401\\
35	1176\\
109	4663\\
26	550\\
41	1006\\
39	973\\
61	4953\\
40	1238\\
36	1694\\
21	551\\
80	5029\\
36	1037\\
36	1691\\
38	832\\
42	1192\\
27	590\\
46	1533\\
23	486\\
34	1267\\
200	30990\\
37	1240\\
16	413\\
50	2688\\
19	324\\
41	2143\\
25	934\\
39	3741\\
45	2869\\
83	3945\\
27	521\\
67	2908\\
46	2419\\
35	1912\\
44	2676\\
30	2051\\
29	674\\
23	204\\
43	936\\
28	859\\
24	467\\
56	2173\\
33	679\\
13	261\\
41	2629\\
34	588\\
14	216\\
83	10336\\
33	1005\\
35	1380\\
29	687\\
54	3377\\
41	993\\
19	351\\
34	1937\\
19	459\\
38	1126\\
43	1196\\
14	412\\
17	325\\
12	184\\
};
\addlegendentry{Dispersione reale}

            % \input{../../../.tex/20/KS/SizeVsDegreefig3.tex}
        \end{groupplot}

        \begin{groupplot}[
            group style={
                group name=Row5,
                plotLastGroup,
                group size=2 by 1,
            },
            plotRow,
            plotLegendSE,
            width=\plotLastWidth cm,
            xlabel={\(t\)},ylabel={\(s(t)\)},
            enlarge x limits=0,enlarge y limits=0.01,
            ymode=log,
            x tick scale label style={
                yshift=-1em,
                at={(.95,0)},
                anchor=north,
            },
            xmin=0,xmax=30000,
            ymin=100,ymax=1.5461e+05,
        ]
            \nextgroupplot[%
                at={($
                    (Row4 c2r1.south west)+(\halfWidth cm,0)
                    -(0,\yGroupSep em)-(0,\plotHeight cm)
                    -(\xPlotSep em,0)-(\plotLastWidth cm,0)
                $)},
                % xmin=0,xmax=30000,
                % ymin=100,ymax=1.5393e+05,
            ] % This file was created by matlab2tikz.
%
\definecolor{mycolor1}{rgb}{0.00146,0.00047,0.01387}%
\definecolor{mycolor2}{rgb}{0.01166,0.00942,0.06346}%
\definecolor{mycolor3}{rgb}{0.02943,0.02150,0.11462}%
\definecolor{mycolor4}{rgb}{0.06134,0.03659,0.17764}%
\definecolor{mycolor5}{rgb}{0.08741,0.04456,0.22481}%
\definecolor{mycolor6}{rgb}{0.12291,0.04754,0.28162}%
\definecolor{mycolor7}{rgb}{0.14907,0.04547,0.31709}%
\definecolor{mycolor8}{rgb}{0.18343,0.04033,0.35497}%
\definecolor{mycolor9}{rgb}{0.21795,0.03662,0.38352}%
\definecolor{mycolor10}{rgb}{0.24497,0.03705,0.40001}%
\definecolor{mycolor11}{rgb}{0.27135,0.04092,0.41198}%
\definecolor{mycolor12}{rgb}{0.29076,0.04564,0.41864}%
\definecolor{mycolor13}{rgb}{0.31628,0.05349,0.42512}%
\definecolor{mycolor14}{rgb}{0.33522,0.06006,0.42852}%
\definecolor{mycolor15}{rgb}{0.36028,0.06925,0.43150}%
\definecolor{mycolor16}{rgb}{0.37900,0.07625,0.43272}%
\definecolor{mycolor17}{rgb}{0.39767,0.08326,0.43318}%
\definecolor{mycolor18}{rgb}{0.41633,0.09020,0.43294}%
\definecolor{mycolor19}{rgb}{0.42877,0.09479,0.43241}%
\definecolor{mycolor20}{rgb}{0.44743,0.10160,0.43108}%
\definecolor{mycolor21}{rgb}{0.46610,0.10832,0.42912}%
\definecolor{mycolor22}{rgb}{0.47856,0.11276,0.42747}%
\definecolor{mycolor23}{rgb}{0.49726,0.11938,0.42449}%
\definecolor{mycolor24}{rgb}{0.50973,0.12377,0.42216}%
\definecolor{mycolor25}{rgb}{0.52221,0.12815,0.41955}%
\definecolor{mycolor26}{rgb}{0.53468,0.13253,0.41667}%
\definecolor{mycolor27}{rgb}{0.55339,0.13913,0.41183}%
\definecolor{mycolor28}{rgb}{0.56585,0.14357,0.40826}%
\definecolor{mycolor29}{rgb}{0.57830,0.14804,0.40441}%
\definecolor{mycolor30}{rgb}{0.59073,0.15256,0.40029}%
\definecolor{mycolor31}{rgb}{0.60314,0.15715,0.39589}%
\definecolor{mycolor32}{rgb}{0.61551,0.16182,0.39122}%
\definecolor{mycolor33}{rgb}{0.62169,0.16418,0.38878}%
\definecolor{mycolor34}{rgb}{0.63400,0.16899,0.38370}%
\definecolor{mycolor35}{rgb}{0.64626,0.17391,0.37836}%
\definecolor{mycolor36}{rgb}{0.65846,0.17896,0.37275}%
\definecolor{mycolor37}{rgb}{0.66454,0.18154,0.36985}%
\definecolor{mycolor38}{rgb}{0.67664,0.18681,0.36385}%
\definecolor{mycolor39}{rgb}{0.68865,0.19224,0.35760}%
\definecolor{mycolor40}{rgb}{0.69463,0.19502,0.35439}%
\definecolor{mycolor41}{rgb}{0.70650,0.20073,0.34778}%
\definecolor{mycolor42}{rgb}{0.71240,0.20366,0.34438}%
\definecolor{mycolor43}{rgb}{0.72410,0.20967,0.33742}%
\definecolor{mycolor44}{rgb}{0.72991,0.21276,0.33386}%
\definecolor{mycolor45}{rgb}{0.74142,0.21911,0.32658}%
\definecolor{mycolor46}{rgb}{0.74713,0.22238,0.32286}%
\definecolor{mycolor47}{rgb}{0.75279,0.22571,0.31909}%
\definecolor{mycolor48}{rgb}{0.76401,0.23255,0.31140}%
\definecolor{mycolor49}{rgb}{0.76956,0.23608,0.30749}%
\definecolor{mycolor50}{rgb}{0.77506,0.23967,0.30353}%
\definecolor{mycolor51}{rgb}{0.79129,0.25086,0.29139}%
\definecolor{mycolor52}{rgb}{0.79661,0.25473,0.28726}%
\definecolor{mycolor53}{rgb}{0.80187,0.25867,0.28310}%
\definecolor{mycolor54}{rgb}{0.81224,0.26679,0.27466}%
\definecolor{mycolor55}{rgb}{0.81734,0.27095,0.27039}%
\definecolor{mycolor56}{rgb}{0.82737,0.27952,0.26175}%
\definecolor{mycolor57}{rgb}{0.83230,0.28391,0.25738}%
\definecolor{mycolor58}{rgb}{0.83717,0.28839,0.25299}%
\definecolor{mycolor59}{rgb}{0.84671,0.29756,0.24411}%
\definecolor{mycolor60}{rgb}{0.85138,0.30226,0.23964}%
\definecolor{mycolor61}{rgb}{0.85599,0.30704,0.23513}%
\definecolor{mycolor62}{rgb}{0.86053,0.31189,0.23061}%
\definecolor{mycolor63}{rgb}{0.87374,0.32691,0.21689}%
\definecolor{mycolor64}{rgb}{0.87800,0.33206,0.21227}%
\definecolor{mycolor65}{rgb}{0.89034,0.34796,0.19829}%
\definecolor{mycolor66}{rgb}{0.89819,0.35891,0.18886}%
\definecolor{mycolor67}{rgb}{0.90200,0.36449,0.18412}%
\definecolor{mycolor68}{rgb}{0.90573,0.37014,0.17935}%
\definecolor{mycolor69}{rgb}{0.90939,0.37586,0.17456}%
\definecolor{mycolor70}{rgb}{0.91297,0.38164,0.16975}%
\definecolor{mycolor71}{rgb}{0.91646,0.38748,0.16492}%
\definecolor{mycolor72}{rgb}{0.91988,0.39339,0.16007}%
\definecolor{mycolor73}{rgb}{0.92322,0.39936,0.15519}%
\definecolor{mycolor74}{rgb}{0.92647,0.40539,0.15029}%
\definecolor{mycolor75}{rgb}{0.92964,0.41148,0.14537}%
\definecolor{mycolor76}{rgb}{0.93274,0.41763,0.14042}%
\definecolor{mycolor77}{rgb}{0.93575,0.42383,0.13544}%
\definecolor{mycolor78}{rgb}{0.93868,0.43009,0.13044}%
\definecolor{mycolor79}{rgb}{0.94152,0.43640,0.12541}%
\definecolor{mycolor80}{rgb}{0.94429,0.44277,0.12035}%
\definecolor{mycolor81}{rgb}{0.94956,0.45566,0.11016}%
\definecolor{mycolor82}{rgb}{0.95208,0.46218,0.10503}%
\definecolor{mycolor83}{rgb}{0.96129,0.48872,0.08429}%
\definecolor{mycolor84}{rgb}{0.96540,0.50225,0.07386}%
\definecolor{mycolor85}{rgb}{0.96732,0.50908,0.06866}%
\definecolor{mycolor86}{rgb}{0.97259,0.52980,0.05332}%
\definecolor{mycolor87}{rgb}{0.97568,0.54380,0.04362}%
\definecolor{mycolor88}{rgb}{0.98082,0.57221,0.02851}%
\definecolor{mycolor89}{rgb}{0.98189,0.57939,0.02625}%
\definecolor{mycolor90}{rgb}{0.98378,0.59385,0.02377}%
\definecolor{mycolor91}{rgb}{0.98532,0.60842,0.02420}%
\definecolor{mycolor92}{rgb}{0.98696,0.63048,0.03091}%
\definecolor{mycolor93}{rgb}{0.98793,0.66025,0.05175}%
\definecolor{mycolor94}{rgb}{0.98771,0.68281,0.07249}%
\definecolor{mycolor95}{rgb}{0.97750,0.78226,0.18592}%
\definecolor{mycolor96}{rgb}{0.96624,0.83619,0.26153}%
\definecolor{mycolor97}{rgb}{0.96252,0.85148,0.28555}%
\definecolor{mycolor98}{rgb}{0.98836,0.99836,0.64492}%
%
\addplot [color=mycolor1]
  table[row sep=crcr]{%
0	4395.3\\
600	53946\\
1200	1.0517e+05\\
1800	1.2691e+05\\
2400	1.353e+05\\
3000	1.4152e+05\\
3600	1.433e+05\\
4200	1.4622e+05\\
4800	1.4508e+05\\
5400	1.48e+05\\
6000	1.4861e+05\\
7800	1.49e+05\\
8400	1.485e+05\\
9000	1.5137e+05\\
9600	1.5206e+05\\
10200	1.5147e+05\\
10800	1.5198e+05\\
11400	1.509e+05\\
12000	1.4901e+05\\
12600	1.5156e+05\\
13800	1.4876e+05\\
14400	1.4823e+05\\
15000	1.4547e+05\\
15600	1.4782e+05\\
16200	1.4729e+05\\
16800	1.4791e+05\\
17400	1.511e+05\\
18000	1.5176e+05\\
19200	1.5037e+05\\
19800	1.5161e+05\\
20400	1.4935e+05\\
21000	1.5235e+05\\
21600	1.5247e+05\\
22800	1.4864e+05\\
23400	1.475e+05\\
24000	1.5017e+05\\
25200	1.5393e+05\\
26400	1.539e+05\\
27000	1.5068e+05\\
27600	1.5231e+05\\
28200	1.5174e+05\\
28800	1.4895e+05\\
29400	1.493e+05\\
30000	1.4922e+05\\
};
\addlegendentry{Classe k=283}

\addplot [color=mycolor2, forget plot]
  table[row sep=crcr]{%
0	4395.3\\
600	19979\\
1200	27583\\
1800	26508\\
2400	27049\\
3000	25425\\
3600	26065\\
4200	25587\\
4800	25728\\
5400	25118\\
6000	25903\\
6600	26593\\
7200	25615\\
7800	25121\\
8400	25825\\
9000	25462\\
9600	25637\\
10800	25495\\
11400	27060\\
12000	26213\\
12600	24511\\
13200	25830\\
13800	25088\\
14400	25686\\
15000	25993\\
15600	26046\\
16200	27306\\
16800	25908\\
17400	24273\\
18000	24818\\
18600	24990\\
19200	25662\\
19800	25103\\
20400	25535\\
21000	25641\\
21600	25611\\
22200	25958\\
22800	25455\\
23400	25808\\
24000	26529\\
24600	25551\\
25200	25547\\
25800	26133\\
26400	25325\\
27000	25714\\
28200	25927\\
28800	26332\\
29400	26081\\
30000	25612\\
};
\addplot [color=mycolor3, forget plot]
  table[row sep=crcr]{%
0	4395.3\\
600	25350\\
1200	38094\\
1800	40162\\
2400	39207\\
3000	38855\\
3600	37158\\
4800	39069\\
5400	39390\\
6000	40329\\
6600	39211\\
7200	40159\\
7800	40599\\
8400	41160\\
9600	38507\\
10200	38259\\
10800	39042\\
11400	39398\\
12000	38610\\
12600	38178\\
13200	38674\\
13800	38818\\
14400	38654\\
15000	39790\\
15600	40665\\
16200	39662\\
16800	39489\\
17400	38123\\
18000	38324\\
18600	39904\\
19200	39028\\
19800	38505\\
20400	38130\\
21600	40517\\
22200	40102\\
22800	38154\\
23400	39830\\
24000	38440\\
24600	38299\\
25200	38935\\
25800	39855\\
27000	40261\\
27600	39524\\
28200	40365\\
28800	42213\\
29400	42275\\
30000	41728\\
};
\addplot [color=mycolor4, forget plot]
  table[row sep=crcr]{%
0	4395.3\\
600	12787\\
1200	15158\\
1800	15397\\
2400	14866\\
3000	13847\\
3600	14393\\
4200	14332\\
4800	14457\\
5400	14035\\
6000	14151\\
6600	13835\\
7200	13767\\
7800	13908\\
8400	13839\\
9000	14838\\
9600	14095\\
10200	14272\\
10800	14588\\
11400	14111\\
12000	13823\\
12600	14213\\
13200	14457\\
13800	13794\\
14400	13956\\
15000	14442\\
15600	14710\\
16200	14446\\
16800	14527\\
17400	14218\\
18000	14074\\
18600	13690\\
19200	14573\\
19800	14925\\
20400	14177\\
21000	14231\\
21600	14200\\
22200	14538\\
22800	14101\\
23400	13927\\
24000	14404\\
24600	14217\\
25200	13991\\
25800	13892\\
26400	14308\\
27000	14456\\
27600	13910\\
28200	14477\\
28800	14743\\
29400	14186\\
30000	13814\\
};
\addplot [color=mycolor5, forget plot]
  table[row sep=crcr]{%
0	4395.3\\
600	27337\\
1200	50398\\
1800	58996\\
2400	63516\\
3000	65437\\
3600	64651\\
4200	62233\\
4800	62560\\
5400	63691\\
6000	63737\\
6600	66077\\
7200	65983\\
7800	66944\\
8400	65921\\
9000	64235\\
9600	64389\\
10200	62541\\
10800	62506\\
11400	63971\\
12600	61603\\
13800	62580\\
14400	60385\\
15000	60956\\
15600	62204\\
16200	62097\\
16800	60883\\
17400	61717\\
18000	61677\\
18600	63384\\
19200	62494\\
19800	61233\\
21000	59468\\
21600	58391\\
22200	59551\\
22800	62376\\
23400	62082\\
24600	61917\\
25200	62845\\
25800	62408\\
26400	63085\\
27000	63011\\
27600	61802\\
28200	61330\\
28800	62715\\
29400	62914\\
30000	62671\\
};
\addplot [color=mycolor6, forget plot]
  table[row sep=crcr]{%
0	4395.3\\
600	24164\\
1200	41211\\
1800	48880\\
2400	53459\\
3000	54608\\
3600	54502\\
4200	54925\\
5400	55122\\
6000	54262\\
6600	55392\\
7200	58157\\
7800	56518\\
8400	53527\\
9000	52441\\
9600	52873\\
10200	53593\\
10800	54773\\
11400	55494\\
12000	55746\\
12600	56741\\
13200	57061\\
13800	57819\\
14400	57825\\
15600	57163\\
16200	55274\\
17400	55696\\
18000	56663\\
18600	57971\\
19200	56981\\
19800	54080\\
20400	54765\\
21000	55099\\
21600	57070\\
22200	56523\\
22800	56544\\
23400	55337\\
24000	54384\\
24600	54684\\
25200	56661\\
26400	54573\\
27000	56428\\
27600	56501\\
28200	54788\\
28800	53425\\
29400	54887\\
30000	56149\\
};
\addplot [color=mycolor7, forget plot]
  table[row sep=crcr]{%
0	4395.3\\
600	16024\\
1200	23738\\
1800	26500\\
2400	26482\\
3000	27844\\
3600	27955\\
4200	27253\\
4800	28103\\
5400	27695\\
6000	28739\\
7200	26904\\
7800	27067\\
8400	27754\\
9000	28158\\
9600	28993\\
10200	29357\\
10800	28154\\
11400	27915\\
12000	28764\\
12600	28193\\
13200	28894\\
13800	28348\\
14400	27651\\
15000	27948\\
15600	27774\\
16200	27379\\
16800	28654\\
17400	28362\\
18000	28905\\
18600	27844\\
19800	29328\\
20400	29115\\
21000	27438\\
21600	26814\\
22200	27838\\
22800	30133\\
23400	28204\\
24000	27451\\
24600	27227\\
25200	27563\\
25800	28255\\
26400	28175\\
27000	26878\\
27600	27467\\
28200	28731\\
28800	28690\\
29400	29868\\
30000	28652\\
};
\addplot [color=mycolor8, forget plot]
  table[row sep=crcr]{%
0	4395.3\\
600	15183\\
1200	20901\\
1800	22010\\
2400	21121\\
3000	20675\\
3600	21167\\
4200	21562\\
4800	21415\\
5400	21016\\
6000	20498\\
6600	21221\\
7200	20160\\
7800	20320\\
8400	20840\\
9000	20517\\
9600	21308\\
10200	20275\\
10800	20637\\
11400	20408\\
12000	20490\\
12600	20407\\
13200	20177\\
13800	19663\\
14400	20525\\
15000	20824\\
15600	20913\\
16200	20220\\
16800	19925\\
17400	21284\\
18000	20433\\
18600	19914\\
19200	20746\\
19800	20631\\
20400	20578\\
21000	21089\\
21600	21064\\
22200	20605\\
22800	20616\\
23400	20454\\
24000	20945\\
24600	20117\\
25200	20121\\
25800	20644\\
26400	20367\\
27600	20080\\
28200	19605\\
28800	20094\\
29400	20353\\
30000	20290\\
};
\addplot [color=mycolor9, forget plot]
  table[row sep=crcr]{%
0	4395.3\\
600	10868\\
1200	13001\\
1800	13469\\
2400	13194\\
3000	13154\\
3600	13919\\
4200	13466\\
4800	14107\\
5400	13630\\
6000	14074\\
6600	14295\\
7200	13685\\
8400	14027\\
9000	13802\\
9600	14007\\
10200	14751\\
10800	13866\\
12000	13855\\
12600	13792\\
13200	14215\\
13800	14348\\
14400	14058\\
15000	14086\\
15600	14202\\
16200	14389\\
16800	13901\\
17400	13318\\
18000	14254\\
18600	14449\\
19200	13778\\
19800	14334\\
20400	14163\\
21000	13797\\
21600	13694\\
22200	13985\\
22800	14480\\
23400	14236\\
24000	14234\\
24600	14438\\
25200	14034\\
25800	13697\\
26400	14165\\
27000	14349\\
27600	13541\\
28200	14051\\
28800	14046\\
29400	13665\\
30000	13941\\
};
\addplot [color=mycolor10, forget plot]
  table[row sep=crcr]{%
0	4395.3\\
600	13121\\
1200	18391\\
1800	19567\\
2400	20105\\
3000	21590\\
3600	22024\\
4200	22292\\
4800	22823\\
5400	22393\\
6000	22722\\
6600	22258\\
7200	21658\\
7800	21634\\
8400	21405\\
9000	21616\\
9600	22099\\
10200	21997\\
10800	22065\\
11400	22614\\
12000	22363\\
12600	22391\\
13200	22317\\
13800	22657\\
14400	22560\\
15000	22116\\
15600	22016\\
16200	22617\\
16800	22536\\
17400	22283\\
18000	21306\\
18600	21194\\
19200	21475\\
19800	21578\\
20400	21744\\
21000	21545\\
21600	22151\\
22200	21866\\
22800	21831\\
23400	22786\\
24000	22506\\
24600	21954\\
25200	22119\\
25800	22208\\
26400	21813\\
27000	22501\\
27600	22247\\
28200	21852\\
28800	21764\\
29400	22026\\
30000	21736\\
};
\addplot [color=mycolor11, forget plot]
  table[row sep=crcr]{%
0	4395.3\\
600	9348.3\\
1200	10286\\
1800	9540.7\\
2400	9284\\
3000	8721.2\\
3600	8393.6\\
4200	8555\\
4800	8589.1\\
5400	8426.8\\
6600	8840.6\\
7200	8484.2\\
7800	8775.4\\
8400	8816.9\\
9600	8495.3\\
10200	8802.2\\
10800	8677.2\\
11400	8872.6\\
12000	8690.7\\
13200	8647.1\\
13800	8554.5\\
14400	7864.2\\
15000	8551.7\\
15600	8399.8\\
16200	8759.3\\
16800	8764.8\\
17400	8902.8\\
18000	8515.3\\
18600	8509\\
19200	8678\\
19800	9191.5\\
20400	8956.1\\
21000	8666.2\\
21600	8530.7\\
22200	8201.8\\
22800	8356\\
24000	8937.6\\
24600	8920.7\\
25200	8688.3\\
25800	8527.6\\
26400	8881.9\\
27000	8888.4\\
27600	8395.9\\
28200	8306.8\\
28800	8628.4\\
29400	8213.8\\
30000	8486\\
};
\addplot [color=mycolor12, forget plot]
  table[row sep=crcr]{%
0	4395.3\\
600	10421\\
1200	13955\\
1800	13726\\
2400	14418\\
3000	13941\\
3600	14228\\
4200	14591\\
4800	14517\\
5400	14633\\
6000	15403\\
6600	14321\\
7800	14986\\
8400	14627\\
9000	14950\\
9600	15032\\
10200	14953\\
10800	14471\\
11400	15318\\
12000	15148\\
12600	14015\\
13200	14829\\
13800	15076\\
14400	14555\\
15000	14116\\
15600	14528\\
16200	14656\\
16800	14479\\
17400	14664\\
18000	14643\\
18600	14438\\
19200	14551\\
19800	14600\\
20400	14934\\
21000	14561\\
21600	14265\\
22200	14333\\
22800	14479\\
23400	14689\\
24000	14996\\
24600	14153\\
25200	14193\\
25800	13989\\
26400	14215\\
27000	14207\\
27600	14925\\
28800	14608\\
29400	13924\\
30000	14962\\
};
\addplot [color=mycolor13, forget plot]
  table[row sep=crcr]{%
0	4395.3\\
600	9144.7\\
1200	10184\\
1800	10836\\
2400	11002\\
3000	11117\\
3600	11595\\
4200	11103\\
4800	10865\\
5400	11033\\
6000	10859\\
6600	11256\\
7200	10564\\
7800	10442\\
8400	10497\\
9000	10865\\
9600	11345\\
10200	11517\\
10800	11183\\
11400	10622\\
12000	11325\\
12600	11505\\
13200	10482\\
13800	10679\\
14400	10776\\
15000	11343\\
15600	11079\\
16200	11011\\
16800	11356\\
17400	11151\\
18000	11320\\
18600	11445\\
19200	10922\\
19800	11123\\
20400	11657\\
21000	11366\\
21600	11446\\
22200	11366\\
23400	11573\\
24000	11276\\
24600	10955\\
25200	11276\\
25800	11122\\
26400	10910\\
27000	11005\\
27600	10849\\
28200	10887\\
28800	10868\\
29400	11039\\
30000	11059\\
};
\addplot [color=mycolor14, forget plot]
  table[row sep=crcr]{%
0	4395.3\\
600	13611\\
1200	20034\\
1800	22697\\
2400	23320\\
3000	23523\\
3600	23798\\
4200	24151\\
4800	24581\\
5400	23736\\
6000	23497\\
6600	24009\\
7800	24633\\
8400	24024\\
9600	24358\\
10200	23820\\
11400	24164\\
12000	25502\\
12600	25455\\
13200	24922\\
13800	24044\\
14400	24678\\
15000	24528\\
15600	23227\\
16200	24655\\
16800	25199\\
17400	25171\\
18000	24185\\
18600	23964\\
19200	24310\\
19800	23994\\
20400	24619\\
21000	23582\\
21600	24052\\
22200	24922\\
22800	24126\\
23400	23561\\
24000	23464\\
24600	23901\\
25200	23030\\
25800	23889\\
26400	24657\\
27000	24240\\
27600	23394\\
28800	24660\\
29400	22807\\
30000	23345\\
};
\addplot [color=mycolor15, forget plot]
  table[row sep=crcr]{%
0	4395.3\\
600	10913\\
1200	15720\\
1800	15723\\
2400	16043\\
3000	16444\\
3600	16725\\
4200	16133\\
4800	16275\\
5400	16637\\
6000	16866\\
6600	17603\\
7200	17042\\
7800	16342\\
8400	16593\\
9000	17277\\
9600	16979\\
10200	17196\\
10800	16366\\
11400	16838\\
12000	17509\\
12600	17598\\
13200	16448\\
13800	17334\\
14400	17201\\
15000	17474\\
15600	17277\\
16200	17154\\
16800	16345\\
18000	16489\\
18600	17208\\
19200	16128\\
19800	16873\\
20400	16296\\
21000	16202\\
21600	16377\\
22200	16700\\
22800	17172\\
23400	17333\\
24000	16441\\
24600	16988\\
25800	17145\\
26400	16867\\
27000	16741\\
27600	17464\\
28200	16555\\
28800	16352\\
29400	16947\\
30000	16773\\
};
\addplot [color=mycolor16, forget plot]
  table[row sep=crcr]{%
0	4395.3\\
600	7812.1\\
1200	8670.8\\
1800	7775.2\\
2400	7428.3\\
3000	6975.7\\
3600	6982.4\\
4200	6893.8\\
4800	6645\\
5400	6703.8\\
6600	6781.9\\
7200	7001.4\\
7800	7099.3\\
8400	7330\\
9000	7151.7\\
9600	7395.9\\
10200	7281.8\\
10800	6960\\
11400	6824.9\\
12000	7033.2\\
12600	7343.1\\
13200	6994.3\\
13800	6960.5\\
14400	7091.9\\
15600	6795.7\\
16200	7264.1\\
16800	7210\\
17400	7211.9\\
18000	6949\\
18600	7004.6\\
19200	6866\\
19800	6827.7\\
20400	7121.8\\
21000	6744.7\\
21600	6928.9\\
22200	6878.2\\
22800	6942.5\\
23400	7125.5\\
24000	7525.3\\
24600	7013.5\\
25200	7195.1\\
25800	6842.2\\
26400	7444.2\\
27000	7087.1\\
27600	7302.7\\
28200	7184.9\\
28800	7047.1\\
29400	7202.8\\
30000	7199.8\\
};
\addplot [color=mycolor17, forget plot]
  table[row sep=crcr]{%
0	4395.3\\
600	10899\\
1200	14514\\
1800	14953\\
2400	15730\\
3000	15105\\
3600	15500\\
4200	16541\\
4800	15819\\
5400	15335\\
6000	15482\\
6600	16084\\
7200	15437\\
7800	16400\\
8400	16763\\
9000	16279\\
9600	16394\\
10200	16407\\
10800	16127\\
11400	15713\\
12000	15735\\
12600	16509\\
13800	16140\\
15000	16479\\
15600	16345\\
16200	16001\\
17400	16205\\
18000	16021\\
18600	15256\\
19200	16512\\
19800	15836\\
20400	16156\\
21600	15905\\
22200	16068\\
22800	16776\\
23400	15884\\
24000	16425\\
24600	15984\\
25200	16269\\
25800	16329\\
26400	15689\\
27000	15786\\
27600	16456\\
28200	15973\\
28800	15286\\
29400	15325\\
30000	15853\\
};
\addplot [color=mycolor18, forget plot]
  table[row sep=crcr]{%
0	4395.3\\
600	13378\\
1200	19956\\
1800	22504\\
2400	23460\\
3000	24680\\
3600	26060\\
4200	24961\\
4800	24346\\
5400	23486\\
6000	24243\\
6600	24303\\
7200	24548\\
7800	24619\\
8400	25395\\
9000	25323\\
9600	24491\\
10800	23396\\
11400	24537\\
12000	24256\\
12600	24546\\
13800	25336\\
14400	26003\\
15000	25819\\
15600	25160\\
16200	24726\\
16800	25624\\
17400	25328\\
18000	24873\\
18600	24514\\
19200	23666\\
19800	24178\\
20400	24509\\
21000	24938\\
21600	24374\\
22200	23351\\
22800	24238\\
24000	23610\\
24600	22834\\
25200	24057\\
25800	24256\\
26400	25363\\
27000	25076\\
27600	25426\\
28200	24426\\
28800	24272\\
29400	25319\\
30000	25716\\
};
\addplot [color=mycolor19, forget plot]
  table[row sep=crcr]{%
0	4395.3\\
600	9806.4\\
1200	13990\\
1800	14017\\
2400	14921\\
3000	15195\\
3600	14766\\
4200	14244\\
4800	13942\\
5400	14899\\
6000	14639\\
6600	14545\\
7200	15332\\
8400	15094\\
9000	14881\\
9600	14506\\
10200	13859\\
10800	14859\\
11400	15333\\
12000	14854\\
12600	14947\\
13200	15486\\
13800	15750\\
14400	14747\\
15000	15063\\
15600	14916\\
16200	14853\\
16800	14161\\
17400	14706\\
18000	14487\\
18600	14422\\
19200	14516\\
19800	14872\\
20400	14241\\
21000	14708\\
21600	14752\\
22200	14960\\
22800	15345\\
23400	14606\\
24000	14601\\
24600	15014\\
25200	15110\\
25800	14938\\
26400	15285\\
27000	14438\\
27600	14264\\
28200	15212\\
28800	15544\\
29400	15260\\
30000	15412\\
};
\addplot [color=mycolor20, forget plot]
  table[row sep=crcr]{%
0	4395.3\\
600	11323\\
1200	16011\\
1800	17145\\
2400	18809\\
3000	19710\\
3600	19917\\
4200	19197\\
4800	19284\\
5400	19767\\
6000	19392\\
6600	18717\\
7200	18968\\
7800	19022\\
8400	18685\\
9000	19125\\
9600	18779\\
10200	18944\\
10800	18571\\
11400	18637\\
12000	18418\\
12600	18692\\
13200	19672\\
13800	18934\\
14400	18758\\
15000	18929\\
16200	18988\\
16800	18492\\
17400	18983\\
18000	18847\\
19200	19571\\
19800	19007\\
20400	18578\\
21600	19610\\
22200	19413\\
22800	18209\\
23400	18666\\
24000	19277\\
24600	19088\\
25200	18318\\
25800	18774\\
26400	18900\\
27000	19198\\
27600	19682\\
28200	19409\\
28800	19869\\
29400	19495\\
30000	18397\\
};
\addplot [color=mycolor21, forget plot]
  table[row sep=crcr]{%
0	4395.3\\
600	7058.8\\
1200	8716.4\\
1800	9317.7\\
2400	9261\\
3000	9591.6\\
3600	9558.6\\
4200	9217.9\\
4800	9718\\
5400	9898.1\\
6000	9506.5\\
6600	9459\\
7200	9465.7\\
7800	9699.2\\
8400	9400.1\\
9000	9783.7\\
9600	9663.6\\
10200	9678\\
10800	9814.4\\
11400	9518.9\\
12000	9978\\
12600	9841.7\\
13200	9572\\
13800	9622.4\\
14400	9704.7\\
15000	9616.1\\
15600	9295.1\\
16200	9374\\
16800	9777.4\\
17400	9659.3\\
18600	9477.8\\
19200	9572.6\\
19800	9611.6\\
20400	9758\\
21000	9641.6\\
21600	9351.3\\
22200	9734.6\\
22800	9905.1\\
23400	9856\\
24000	9983.3\\
24600	9738.7\\
25200	9306.1\\
25800	9326\\
26400	9182.4\\
27000	9357.5\\
27600	9749\\
28200	9397\\
29400	9889.9\\
30000	9798.5\\
};
\addplot [color=mycolor22, forget plot]
  table[row sep=crcr]{%
0	4395.3\\
600	15451\\
1200	24500\\
1800	26310\\
2400	26900\\
3000	27173\\
3600	26155\\
4200	25934\\
5400	26064\\
6000	25936\\
6600	25470\\
7200	25862\\
7800	25691\\
8400	25281\\
9600	24018\\
10200	24376\\
10800	24448\\
11400	25192\\
12000	25152\\
12600	24287\\
13200	24030\\
14400	24907\\
15000	25028\\
15600	25657\\
16200	25162\\
16800	24817\\
17400	24819\\
18000	25548\\
18600	25181\\
19800	25003\\
20400	25601\\
21600	25234\\
22200	25317\\
22800	25243\\
23400	25373\\
24000	25593\\
24600	24972\\
25200	24921\\
25800	24305\\
26400	25023\\
27000	25215\\
27600	24624\\
28200	24704\\
28800	24589\\
29400	23788\\
30000	24494\\
};
\addplot [color=mycolor23, forget plot]
  table[row sep=crcr]{%
0	4395.3\\
600	8840.5\\
1200	11588\\
1800	12207\\
2400	12163\\
3000	12343\\
3600	12278\\
4200	12476\\
4800	12288\\
5400	12151\\
6000	12352\\
6600	12365\\
7800	12272\\
8400	12100\\
9000	12272\\
9600	11980\\
10200	12293\\
10800	12320\\
11400	12166\\
12000	12235\\
12600	12368\\
13200	12323\\
13800	12453\\
14400	12257\\
15000	12366\\
15600	12557\\
16200	12399\\
16800	12112\\
17400	11889\\
18000	12160\\
18600	12530\\
19200	12395\\
19800	12305\\
20400	12392\\
21000	12301\\
21600	12391\\
22200	12375\\
22800	12059\\
23400	12250\\
24000	12245\\
24600	12387\\
25200	12257\\
25800	12162\\
26400	12335\\
27600	11995\\
28200	12187\\
28800	12308\\
30000	12216\\
};
\addplot [color=mycolor24, forget plot]
  table[row sep=crcr]{%
0	4395.3\\
600	5691.1\\
1200	5733.1\\
1800	5171.2\\
2400	4783.3\\
3000	4809.8\\
3600	4598.9\\
4200	4492.5\\
4800	4666\\
5400	4589.4\\
6000	4779.9\\
6600	4535.9\\
7200	4776\\
7800	4729.7\\
8400	4624.9\\
9000	4900.8\\
9600	4758.8\\
10200	4780.9\\
10800	4950.3\\
11400	4875.3\\
12000	4659\\
12600	4935.8\\
13200	4753.2\\
13800	5090.1\\
14400	5111.9\\
15000	4845.7\\
15600	4768.6\\
16200	4760.2\\
16800	4958.6\\
17400	4655.1\\
18600	4959.5\\
19200	4972.5\\
19800	4907.5\\
21000	4926.8\\
21600	4794.2\\
22200	4583.2\\
22800	4717.6\\
23400	4658.3\\
24000	4705.2\\
24600	5025.3\\
25200	4746\\
25800	4668.2\\
26400	4667.3\\
27000	4747\\
27600	4798.6\\
28200	4677.6\\
28800	4765.6\\
29400	4675.4\\
30000	4667.9\\
};
\addplot [color=mycolor25, forget plot]
  table[row sep=crcr]{%
0	4395.3\\
600	8198.3\\
1200	11266\\
1800	11596\\
2400	12664\\
3000	12513\\
3600	12034\\
4200	12326\\
4800	12587\\
5400	12386\\
6000	12620\\
6600	12131\\
7200	12025\\
7800	12755\\
8400	12506\\
9000	12142\\
9600	12458\\
10200	12525\\
10800	12140\\
11400	12051\\
12000	12286\\
12600	12130\\
13200	12032\\
13800	12683\\
14400	12643\\
15000	12331\\
15600	12573\\
16200	12863\\
16800	12441\\
17400	11656\\
18000	11953\\
18600	12703\\
19200	12062\\
19800	11836\\
20400	12476\\
21000	12687\\
21600	12823\\
22200	12232\\
22800	12542\\
23400	12946\\
24000	12510\\
24600	13074\\
25200	12030\\
25800	12591\\
26400	12857\\
27000	12353\\
27600	12718\\
28200	12105\\
29400	12692\\
30000	11918\\
};
\addplot [color=mycolor26, forget plot]
  table[row sep=crcr]{%
0	4395.3\\
600	6770.7\\
1200	7605.9\\
1800	7598.8\\
2400	7749.5\\
3000	7541.5\\
3600	7213.2\\
4200	7699.5\\
4800	8406.4\\
5400	7996.8\\
6000	8588.6\\
6600	8476.7\\
7200	8157.4\\
7800	7681.8\\
8400	8361.3\\
9000	8609.2\\
9600	9058.8\\
10200	8566\\
10800	8258.9\\
11400	8495.1\\
12000	8570.5\\
12600	8376.6\\
13200	7871.1\\
13800	8099.7\\
14400	8493.5\\
15000	8111.4\\
15600	7858.1\\
16200	7673.2\\
16800	8140.2\\
17400	8078.7\\
18000	8724\\
18600	8608.1\\
19200	8189.2\\
19800	8146.3\\
20400	8273\\
21000	8474.2\\
21600	8487.7\\
22800	8016.2\\
23400	8165.5\\
24000	7821.8\\
24600	8018.3\\
25200	8684.4\\
25800	8378.1\\
26400	8570.7\\
27000	8431.7\\
27600	8224.3\\
28200	8390.6\\
28800	7611\\
29400	8015.2\\
30000	8052.2\\
};
\addplot [color=mycolor27, forget plot]
  table[row sep=crcr]{%
0	4395.3\\
600	8104.7\\
1200	9942.7\\
1800	10068\\
3000	10090\\
3600	9664.5\\
4200	10016\\
4800	10003\\
5400	10250\\
6000	9521.8\\
6600	9673.8\\
7200	9567.2\\
7800	9402.7\\
8400	9605.2\\
9000	9649.9\\
9600	9289.7\\
10200	10197\\
10800	9981.7\\
11400	9439.8\\
12000	9383.4\\
12600	10001\\
13200	9610.3\\
13800	10083\\
14400	9872.1\\
15000	9901.5\\
15600	9387.9\\
16200	9529.7\\
16800	9708.9\\
17400	10124\\
18000	9792.2\\
18600	9504.9\\
19200	9612.6\\
19800	9881.2\\
20400	10069\\
21000	9585.2\\
22200	9918\\
22800	9759.7\\
23400	10022\\
24000	10115\\
24600	9985.7\\
25200	9482.9\\
25800	9490.6\\
26400	9712.1\\
27000	10092\\
27600	9743.4\\
28200	9655.3\\
28800	9610.4\\
29400	9865.9\\
30000	9405.4\\
};
\addplot [color=mycolor28, forget plot]
  table[row sep=crcr]{%
0	4395.3\\
600	6442.6\\
1200	6315.7\\
1800	5991.8\\
2400	5726.7\\
3000	5812.4\\
3600	5815.6\\
4200	5372.9\\
4800	5660.1\\
5400	5752.9\\
6000	5751.1\\
6600	5367.8\\
7200	5460.3\\
7800	5427.8\\
8400	5151\\
9000	5488.5\\
9600	5716.6\\
10200	5444.6\\
10800	5523.2\\
11400	5347.9\\
12000	5127.6\\
12600	5489.8\\
13200	5576\\
13800	5562.9\\
14400	5406.1\\
15000	5838.1\\
15600	5390.2\\
16200	5624.2\\
16800	5569.6\\
17400	5567.4\\
18000	5443.8\\
18600	5351.2\\
19200	5400\\
19800	5382.9\\
20400	5591.4\\
21000	5824.4\\
21600	5891.6\\
22200	5547.3\\
22800	5257.6\\
23400	5579.7\\
24000	5352.5\\
24600	5636.8\\
25200	5769.1\\
25800	5875.9\\
26400	5128.7\\
27000	5419.3\\
27600	5494.4\\
28200	5428.9\\
28800	5596.4\\
29400	5721.2\\
30000	5571.7\\
};
\addplot [color=mycolor29, forget plot]
  table[row sep=crcr]{%
0	4395.3\\
600	9498.5\\
1200	11553\\
1800	11569\\
2400	12057\\
3000	11674\\
3600	11701\\
4200	10925\\
4800	11255\\
5400	12134\\
6000	11193\\
6600	11526\\
7200	11560\\
7800	11234\\
8400	12047\\
9000	11767\\
9600	11825\\
10200	12037\\
10800	11566\\
11400	11492\\
12000	11721\\
12600	11806\\
13200	11849\\
13800	11829\\
14400	11590\\
15000	11692\\
15600	10986\\
16200	11345\\
16800	10936\\
17400	11253\\
18000	11110\\
18600	11254\\
19200	11559\\
19800	11839\\
20400	11530\\
21000	11135\\
21600	11283\\
22200	11278\\
22800	10951\\
23400	11387\\
24000	11878\\
24600	12040\\
25200	11385\\
25800	11641\\
26400	11292\\
27000	11107\\
27600	11746\\
28200	11762\\
28800	11356\\
29400	11304\\
30000	11087\\
};
\addplot [color=mycolor30, forget plot]
  table[row sep=crcr]{%
0	4395.3\\
600	6089.2\\
1200	7214.9\\
1800	7696\\
3000	7508.5\\
3600	7356.9\\
4200	8033.8\\
4800	7826.6\\
5400	7991.2\\
6000	7343.9\\
6600	7416.1\\
7200	7881.7\\
7800	8096.8\\
8400	7946.7\\
9000	7658.7\\
10200	7927.6\\
10800	8087.3\\
11400	7747.8\\
12000	8047.3\\
12600	7917.1\\
13200	7946.4\\
13800	7873.9\\
14400	7775.1\\
15000	8288\\
15600	8085.7\\
16200	7816.1\\
16800	7821.6\\
17400	7617.4\\
18000	7792.6\\
18600	7745.8\\
19200	7592.2\\
19800	8054.5\\
20400	8041.8\\
21000	7764.7\\
21600	7797.3\\
22200	8140.7\\
22800	8276.1\\
24000	7812\\
24600	7668.9\\
25200	8028.1\\
25800	7836.2\\
26400	7693.3\\
27000	7914.3\\
27600	7971.9\\
28200	8147.1\\
28800	7648.9\\
29400	7786.3\\
30000	8305.7\\
};
\addplot [color=mycolor31, forget plot]
  table[row sep=crcr]{%
0	4395.3\\
600	6796.8\\
1200	7188.6\\
1800	7278.7\\
2400	6611.8\\
3000	6435.7\\
3600	6693.3\\
4200	6549.3\\
4800	6854.9\\
6000	6937.9\\
6600	6442.3\\
7200	6408.2\\
7800	6270.9\\
8400	6551.1\\
9000	6757.6\\
9600	6612.2\\
10200	6488.3\\
10800	6768\\
11400	6591.6\\
12000	6498.8\\
12600	6852.7\\
13200	6739.9\\
13800	6501\\
14400	6543\\
15000	6449.1\\
15600	6464.4\\
16200	6307.5\\
16800	6484.7\\
17400	6808.4\\
18000	6196.5\\
18600	6588.3\\
19200	6585.2\\
19800	6923.9\\
20400	6783.9\\
21000	6840\\
21600	6653.1\\
22200	6614\\
22800	6607.7\\
23400	6307.7\\
24000	6361.1\\
24600	6154.2\\
25200	6084.4\\
25800	6479.8\\
26400	6596.5\\
27000	6339.7\\
27600	6426.2\\
28800	6418.2\\
29400	6463.3\\
30000	7091.5\\
};
\addplot [color=mycolor32, forget plot]
  table[row sep=crcr]{%
0	4395.3\\
600	9208\\
1200	11439\\
1800	11783\\
2400	10860\\
3000	10787\\
3600	11436\\
4200	10565\\
4800	10901\\
5400	10616\\
6000	10840\\
6600	10816\\
7200	10498\\
7800	10221\\
8400	10549\\
9000	9998.5\\
9600	10314\\
10200	10759\\
10800	10713\\
11400	10358\\
12000	10588\\
12600	10249\\
13200	10308\\
13800	10413\\
14400	10175\\
15000	10274\\
15600	10595\\
16200	10021\\
16800	10558\\
17400	10104\\
18000	10125\\
18600	10745\\
19800	10471\\
20400	10606\\
21000	10420\\
21600	10793\\
22800	10680\\
23400	10173\\
24000	10644\\
24600	10545\\
25200	10541\\
25800	10102\\
26400	10093\\
27000	10013\\
27600	10502\\
28200	10587\\
28800	10061\\
29400	10060\\
30000	10546\\
};
\addplot [color=mycolor33, forget plot]
  table[row sep=crcr]{%
0	4395.3\\
600	6043.2\\
1200	7117.2\\
1800	7588.7\\
2400	7838.1\\
3000	8218.3\\
3600	8092.6\\
4200	8294.3\\
4800	8363\\
5400	8367.2\\
6000	8094.5\\
6600	8180\\
7200	8062.5\\
7800	8023.7\\
9600	8203.8\\
10200	8380.4\\
10800	8447.4\\
11400	8158\\
12000	8050.9\\
12600	8181.8\\
13200	7922.5\\
14400	8036.8\\
15000	7929.1\\
15600	8143.8\\
16200	8505.9\\
16800	8260.7\\
17400	8208.3\\
18000	8436.7\\
18600	8428.1\\
19200	8553\\
19800	7934.6\\
20400	8297.5\\
21000	8086.2\\
21600	8157.5\\
22200	8048.3\\
23400	7964.3\\
24000	7777.7\\
24600	8322.1\\
25200	7990.5\\
25800	8469.1\\
27000	8066.4\\
27600	8046.5\\
28200	8229.8\\
28800	8519.8\\
29400	7887.3\\
30000	7903\\
};
\addplot [color=mycolor34, forget plot]
  table[row sep=crcr]{%
0	4395.3\\
600	5294.4\\
1200	4942.2\\
1800	4403.2\\
2400	3992\\
3000	3852.5\\
3600	3920.4\\
4200	4035.3\\
4800	4015.8\\
5400	3890.3\\
6000	3922.7\\
6600	4205.3\\
7200	4113\\
7800	4125.6\\
9600	4114.7\\
10200	4074.2\\
10800	4080.9\\
11400	4028.6\\
12000	3927.6\\
12600	4138.7\\
13200	4005.1\\
14400	4052.7\\
15000	4032.6\\
15600	3943.5\\
16200	4192.2\\
16800	4281.7\\
17400	4184\\
18000	3936\\
18600	3918.7\\
19800	3918.5\\
20400	4005.1\\
21000	3935.3\\
21600	4277.1\\
22200	4085.9\\
22800	4148.9\\
23400	4277.3\\
24000	4117.4\\
24600	4136\\
25200	4037.5\\
25800	4008.5\\
26400	3912.4\\
27000	4031.6\\
27600	4183.5\\
28200	4077.2\\
28800	4013.1\\
29400	3934.8\\
30000	4165.1\\
};
\addplot [color=mycolor35]
  table[row sep=crcr]{%
0	4395.3\\
600	7636.6\\
1200	9386.7\\
1800	9502.5\\
2400	9967.2\\
3000	9779.7\\
3600	9819.1\\
4200	9807.3\\
4800	9963.2\\
5400	9914.2\\
6000	10047\\
6600	9687.9\\
7200	10058\\
7800	10286\\
9000	10081\\
9600	10249\\
10200	10262\\
10800	10182\\
11400	10005\\
12000	10155\\
12600	9742\\
13200	10098\\
13800	10145\\
14400	10031\\
15000	10132\\
15600	9916.5\\
16200	10113\\
16800	10198\\
17400	10588\\
18000	10285\\
18600	10059\\
19200	10102\\
19800	10478\\
20400	10209\\
21000	9774.2\\
21600	10152\\
22200	10098\\
22800	9980\\
23400	9982.2\\
24000	10049\\
24600	10450\\
25200	9948.5\\
25800	10120\\
26400	10029\\
27000	10269\\
27600	10268\\
28200	10512\\
28800	10019\\
29400	10265\\
30000	9801.2\\
};
\addlegendentry{Classe k=83}

\addplot [color=mycolor36, forget plot]
  table[row sep=crcr]{%
0	4395.3\\
600	5460.2\\
1200	5159.3\\
2400	4246.2\\
3000	4469.4\\
3600	4611.8\\
4200	4392\\
4800	4481.4\\
5400	4355\\
6000	4268.8\\
6600	4266\\
7200	4214.4\\
7800	3966.8\\
8400	4456.5\\
9000	4208.4\\
9600	4223.1\\
10200	4196.9\\
10800	4261.4\\
11400	4367.5\\
12000	4301.9\\
12600	4322.5\\
13200	4079.4\\
13800	4195.7\\
14400	4096\\
15000	4235.7\\
15600	4160.4\\
16200	4213.6\\
16800	4244.9\\
17400	4450.6\\
18000	4516.8\\
18600	4329.1\\
19200	4353.6\\
19800	4246.6\\
20400	4239.3\\
21000	4300.3\\
21600	4178.2\\
22200	4372.1\\
22800	4052.8\\
23400	4357.7\\
24000	4402.5\\
24600	4552.9\\
25200	4218.9\\
25800	4293.8\\
26400	4231.7\\
27000	4611.7\\
27600	4517.8\\
28200	4119\\
28800	4357.8\\
29400	4291.3\\
30000	4396.8\\
};
\addplot [color=mycolor37, forget plot]
  table[row sep=crcr]{%
0	4395.3\\
600	4252.9\\
1200	4130.1\\
1800	3931.4\\
2400	3784\\
3000	3780.5\\
3600	3598.3\\
4200	3553.8\\
4800	3811.8\\
5400	3625.1\\
6000	3630\\
6600	3691.9\\
7200	3684.8\\
7800	3518.7\\
8400	3498.5\\
9000	3604.9\\
9600	3648.9\\
10200	3709.9\\
10800	3724.2\\
11400	3551.7\\
12000	3654.7\\
12600	3773\\
13200	3768.6\\
13800	3663.3\\
14400	3682.7\\
15000	3665.3\\
15600	3520.5\\
16200	3626.3\\
16800	3826.7\\
17400	3722.5\\
18000	3847.5\\
18600	3712.8\\
19200	3736.3\\
19800	3656.2\\
20400	3726.1\\
21000	3775.7\\
21600	3609.4\\
22200	3545.5\\
22800	3617.5\\
23400	3584.7\\
24000	3594.6\\
24600	3667.7\\
25200	3755\\
25800	3651.9\\
26400	3516.5\\
27000	3566.5\\
27600	3573.7\\
28200	3665.2\\
28800	3742.8\\
29400	3807.7\\
30000	3645.5\\
};
\addplot [color=mycolor38, forget plot]
  table[row sep=crcr]{%
0	4395.3\\
600	3843.4\\
1200	3335.2\\
1800	3205.7\\
2400	3146.2\\
3000	2917.4\\
3600	2995.9\\
4200	2857.5\\
4800	3037.6\\
5400	3012.8\\
6000	3131.8\\
6600	3093.3\\
7200	3072.9\\
7800	2995.2\\
8400	3010.5\\
9600	2964.1\\
10200	3044.7\\
10800	2974.6\\
11400	3053.6\\
12000	2952\\
12600	2992.5\\
13200	2858.7\\
13800	3046.5\\
14400	3033.4\\
15000	2921.7\\
15600	3057.2\\
16200	2988.4\\
16800	3135.1\\
17400	3080\\
18000	3065.5\\
18600	3071.1\\
19200	3059.7\\
19800	2884.4\\
20400	2879.9\\
21000	3051.6\\
21600	2972\\
22200	3006.5\\
22800	2878\\
23400	2899.6\\
24000	2981.8\\
24600	3078.9\\
25200	2940.8\\
25800	2975.7\\
26400	2869.3\\
27000	2897.5\\
27600	3171.3\\
28200	3067.3\\
28800	2919.3\\
29400	3024.2\\
30000	3050\\
};
\addplot [color=mycolor39, forget plot]
  table[row sep=crcr]{%
0	4395.3\\
600	9180.2\\
1200	12196\\
1800	12621\\
2400	12587\\
3000	11822\\
3600	12337\\
4200	12454\\
4800	12302\\
5400	11818\\
6000	11621\\
6600	11918\\
7200	11884\\
7800	11692\\
8400	11731\\
9600	11748\\
10200	11622\\
10800	11750\\
11400	11332\\
12000	11499\\
12600	11530\\
13800	11719\\
14400	11472\\
15000	11688\\
15600	12095\\
16200	12208\\
16800	12217\\
17400	11863\\
18000	11721\\
18600	11406\\
19200	11433\\
19800	11886\\
20400	10950\\
21000	11388\\
21600	11627\\
22200	11377\\
22800	11337\\
23400	11003\\
24000	11980\\
24600	11940\\
25200	12058\\
25800	11512\\
26400	12105\\
27000	11860\\
27600	11575\\
28200	11892\\
28800	11737\\
29400	11637\\
30000	11949\\
};
\addplot [color=mycolor40, forget plot]
  table[row sep=crcr]{%
0	4395.3\\
600	4852.5\\
1200	5186.2\\
1800	5360.1\\
2400	5480.5\\
3000	5437.9\\
3600	5605.6\\
4200	5669.7\\
4800	5765.2\\
5400	5735.9\\
6000	5586.4\\
6600	5838.4\\
7200	5884.2\\
7800	5800.2\\
8400	5826.4\\
9000	5763.9\\
9600	5679.9\\
10200	5823.8\\
10800	5825.3\\
11400	5702.4\\
12000	5750.3\\
12600	5932.6\\
13200	5754.7\\
13800	5789.1\\
14400	5950.4\\
15000	5740.8\\
15600	5751.7\\
16200	5640.7\\
16800	5581.1\\
17400	5770.4\\
18000	5733.5\\
18600	5821.2\\
19200	5961.4\\
19800	5804.4\\
20400	5782.5\\
21000	5795.7\\
21600	5629.3\\
22200	5759.4\\
22800	5811.4\\
23400	5923.7\\
24000	5820.6\\
24600	5858.6\\
25200	5624.7\\
25800	5780.7\\
26400	5738.1\\
27000	5733.4\\
27600	5759.5\\
28200	5583.5\\
28800	5601\\
30000	5831.6\\
};
\addplot [color=mycolor41, forget plot]
  table[row sep=crcr]{%
0	4395.3\\
600	4507.4\\
1200	4990.2\\
1800	5005.8\\
2400	5226.5\\
3000	5266.2\\
3600	5330.2\\
4200	5365.1\\
4800	5339.4\\
5400	5519.7\\
6000	5250.7\\
6600	5123\\
7200	5231.9\\
7800	5444.8\\
8400	5501.5\\
9000	5336.5\\
9600	5591.8\\
10200	5322.1\\
10800	5417.3\\
11400	5344.8\\
12600	5453.9\\
13200	5322.1\\
13800	5277.6\\
14400	5332.2\\
15000	5496.6\\
15600	5456.5\\
16200	5574.5\\
16800	5253.8\\
17400	5316.5\\
18000	5537.9\\
18600	5261.5\\
19200	5274.2\\
19800	5145\\
20400	5523.4\\
21000	5664.5\\
21600	5498\\
22200	5677.1\\
22800	5322.3\\
23400	5401.4\\
24000	5231.6\\
25200	5300.3\\
25800	5459.7\\
27600	5619.8\\
28200	5381.8\\
28800	5363.5\\
29400	5049.6\\
30000	5335.2\\
};
\addplot [color=mycolor42, forget plot]
  table[row sep=crcr]{%
0	4395.3\\
600	10133\\
1200	16042\\
1800	17644\\
2400	19744\\
3000	20294\\
3600	19859\\
4200	19787\\
4800	20346\\
5400	20437\\
6000	20393\\
6600	19974\\
8400	20859\\
9000	20741\\
9600	20439\\
10200	20051\\
11400	20605\\
12000	20768\\
12600	19385\\
13200	20716\\
13800	20916\\
14400	20831\\
15000	20907\\
15600	21815\\
16200	21073\\
16800	20787\\
17400	21288\\
18000	21659\\
18600	22195\\
19200	21028\\
19800	20012\\
20400	21089\\
21000	21462\\
21600	20919\\
22200	20974\\
22800	21770\\
23400	21328\\
24000	21865\\
24600	21602\\
25200	21120\\
25800	21106\\
26400	20982\\
27000	20693\\
28200	20297\\
28800	20866\\
29400	20290\\
30000	20963\\
};
\addplot [color=mycolor43, forget plot]
  table[row sep=crcr]{%
0	4395.3\\
600	5118.5\\
1200	5387.2\\
1800	5431.2\\
2400	5106.4\\
3000	4925.4\\
3600	5056.3\\
4200	5060.2\\
4800	4797.5\\
5400	4952.8\\
6000	5032.1\\
7800	4942.2\\
8400	5105.9\\
9000	5131.4\\
9600	5028.6\\
10200	4722.6\\
10800	4755.7\\
11400	5103.1\\
13200	4984.3\\
13800	4879.9\\
14400	4997.1\\
15600	4927.7\\
16200	5090\\
16800	4972.1\\
17400	4811.4\\
18000	5075.3\\
18600	5036\\
19200	4897\\
19800	5028.4\\
21000	5046.1\\
21600	5026.6\\
22200	5122.5\\
22800	5021.8\\
23400	5008.5\\
24000	4866\\
24600	4963.3\\
25200	4920.8\\
25800	4711.1\\
27000	5166.7\\
27600	4965.4\\
28200	5016.6\\
29400	4954.5\\
30000	5070.7\\
};
\addplot [color=mycolor44, forget plot]
  table[row sep=crcr]{%
0	4395.3\\
600	4598.7\\
1200	4090.4\\
1800	3851\\
2400	3653.7\\
3000	3429.4\\
3600	3452.3\\
4200	3596.1\\
4800	3555.3\\
5400	3527.6\\
6000	3338.2\\
6600	3393.7\\
7200	3231.4\\
7800	3323.8\\
8400	3520.4\\
9000	3416.1\\
9600	3530.7\\
10200	3364.5\\
10800	3522.7\\
11400	3432.9\\
12000	3424.4\\
12600	3149.9\\
13200	3018.1\\
13800	3272.3\\
14400	3378.3\\
15000	3512.7\\
15600	3510.3\\
16200	3192.3\\
16800	3310.6\\
17400	3261.2\\
18000	3356.5\\
18600	3270.5\\
19200	3033.5\\
19800	3237.1\\
20400	3324.5\\
21000	3245.7\\
22200	3250.2\\
22800	3354.3\\
23400	3374.3\\
24000	3297.1\\
24600	3393.6\\
25200	3664.9\\
25800	3477\\
26400	3264.9\\
27000	3364.7\\
27600	3240.9\\
28200	3391.6\\
28800	3443\\
30000	3361.7\\
};
\addplot [color=mycolor45, forget plot]
  table[row sep=crcr]{%
0	4395.3\\
600	5133.8\\
1200	6117.1\\
1800	6303.9\\
2400	6662.4\\
3000	6577.1\\
3600	6785.4\\
4200	7183.1\\
5400	6933\\
6000	6959.1\\
6600	7110.6\\
7200	6970.5\\
7800	7318.1\\
8400	7162.5\\
9000	7194.3\\
9600	6857.3\\
10200	6850.6\\
10800	6930.3\\
11400	6965.9\\
12000	6876\\
12600	6822.8\\
13200	6668.4\\
13800	6845.8\\
15000	6878\\
15600	6719.3\\
16200	7005.9\\
16800	6889.2\\
17400	7245.5\\
18000	6991.6\\
18600	6829.1\\
19200	6851\\
19800	6957.1\\
20400	6923.6\\
21000	7119\\
21600	7119.8\\
22200	6821.6\\
22800	6917.3\\
23400	6886.3\\
24600	6932.2\\
25200	7045.7\\
25800	6780.5\\
26400	6833.7\\
27000	6846.1\\
27600	6992.5\\
28200	7209.3\\
28800	7202.7\\
30000	6837.4\\
};
\addplot [color=mycolor46, forget plot]
  table[row sep=crcr]{%
0	4395.3\\
600	4026\\
1200	3962.8\\
1800	4112.2\\
2400	4037.9\\
3000	4144.1\\
3600	4020.2\\
4200	3992\\
4800	4219.1\\
5400	4107.4\\
6000	3912.1\\
6600	4118.4\\
7200	4060.9\\
7800	4212.6\\
8400	4273.4\\
9000	4079.7\\
9600	4058.8\\
10200	4123.5\\
10800	3912.4\\
11400	3974.3\\
12000	4018.4\\
12600	4096.8\\
13200	4227.1\\
13800	3994.5\\
14400	3981.6\\
15000	3957.5\\
15600	4133.9\\
16200	4038.9\\
16800	3968.4\\
17400	4031.9\\
18000	4024.8\\
18600	4165.4\\
19200	4004.9\\
19800	3997.5\\
20400	4082.6\\
21000	4230.1\\
21600	4046.1\\
22200	4151.5\\
22800	4012.9\\
23400	4083\\
24600	3999.7\\
25200	3863.5\\
25800	3943.1\\
26400	4115.4\\
27000	4331.7\\
27600	4093.3\\
28800	3995.3\\
30000	4138.3\\
};
\addplot [color=mycolor47, forget plot]
  table[row sep=crcr]{%
0	4395.3\\
600	5192.2\\
1200	5770.2\\
1800	5545.3\\
2400	5415.1\\
3000	5082.6\\
3600	5163.8\\
4200	5226.2\\
4800	5482\\
5400	5408.9\\
6000	5133\\
6600	5457.6\\
7200	5368.9\\
7800	5105.8\\
8400	5373.9\\
9000	5294.9\\
9600	5405.3\\
10200	5395.8\\
10800	5405.3\\
11400	5489\\
12000	5405.1\\
12600	5231.3\\
13200	5245.4\\
13800	5278.7\\
14400	5464.1\\
16200	5443.2\\
16800	5388.4\\
17400	5162.9\\
18000	5266.6\\
18600	5281\\
19200	5216.1\\
19800	5290.7\\
20400	5449\\
21000	5128.1\\
21600	5290.9\\
22200	5497.7\\
23400	5326.5\\
24000	5285.3\\
24600	5444.1\\
25200	5627.6\\
25800	5359.5\\
26400	5231.9\\
27000	5209.2\\
27600	5163.1\\
28200	5242.8\\
28800	5181.6\\
29400	5237.4\\
30000	5028.6\\
};
\addplot [color=mycolor48, forget plot]
  table[row sep=crcr]{%
0	4395.3\\
600	3731.5\\
1200	3271.5\\
1800	3130.6\\
2400	3009.7\\
3000	2910.6\\
3600	2924.7\\
4200	2984.6\\
4800	2918.9\\
5400	3003.4\\
6000	3019.9\\
6600	2892.1\\
7800	2892.2\\
8400	2858.5\\
9000	2941.9\\
9600	2904.9\\
10200	2933.2\\
10800	3004.9\\
11400	2957.5\\
12000	2946.4\\
12600	2841.5\\
13200	2867.5\\
13800	2846.7\\
14400	2993\\
15000	2892.9\\
15600	2970.5\\
16200	2921.1\\
16800	2900.2\\
17400	2949.9\\
18000	2904.8\\
18600	2870.9\\
19200	2893.6\\
19800	2938.7\\
20400	2909.7\\
21600	2882\\
22200	2886.7\\
22800	2932.3\\
23400	2925.2\\
24000	2930.7\\
24600	2952.1\\
25800	2855.6\\
26400	2862.9\\
27000	2886\\
27600	2896.6\\
28200	2915.6\\
28800	2875.5\\
29400	2900.9\\
30000	2940.2\\
};
\addplot [color=mycolor49, forget plot]
  table[row sep=crcr]{%
0	4395.3\\
600	4739.8\\
1200	4680.2\\
1800	4699.3\\
2400	4516.2\\
3000	4319\\
3600	4343.6\\
4200	4297.6\\
4800	4567.8\\
5400	4464.4\\
6000	4319.5\\
6600	4386.8\\
7200	4512.7\\
7800	4532.4\\
8400	4531.7\\
9000	4395.6\\
9600	4351.4\\
10200	4324\\
10800	4423.6\\
11400	4239.9\\
12000	4373.2\\
12600	4497\\
13200	4346.4\\
14400	4254\\
15000	4220.4\\
15600	4089\\
16200	4130.5\\
16800	4448.8\\
17400	4376.1\\
18000	4575.8\\
18600	4383\\
19200	4408.5\\
19800	4368.5\\
20400	4529.8\\
21000	4632.4\\
21600	4367.2\\
22200	4204.5\\
22800	4390.8\\
23400	4621\\
24000	4220.9\\
24600	4287.5\\
25200	4329.5\\
25800	4459.7\\
26400	4456.7\\
27000	4580.8\\
27600	4222.5\\
28200	4544.7\\
28800	4294.9\\
29400	4444\\
30000	4454.3\\
};
\addplot [color=mycolor50, forget plot]
  table[row sep=crcr]{%
0	4395.3\\
600	3159.5\\
1200	2473.9\\
1800	2427.9\\
2400	2604.6\\
3600	2529.1\\
4200	2515.1\\
4800	2417\\
5400	2437.9\\
6000	2442.9\\
7200	2371.2\\
7800	2387.6\\
8400	2506.3\\
9000	2412\\
9600	2470.2\\
10200	2430.6\\
10800	2501.3\\
11400	2540.9\\
12000	2490.6\\
12600	2574.9\\
13200	2528\\
13800	2518.3\\
14400	2445.9\\
15000	2521.2\\
15600	2477.5\\
16200	2514.3\\
16800	2388\\
17400	2475.4\\
18000	2497.9\\
18600	2434.2\\
19200	2498\\
19800	2508.3\\
21600	2440.6\\
22200	2445.3\\
22800	2415\\
23400	2521.1\\
24000	2469.5\\
24600	2499.6\\
25200	2406\\
26400	2431\\
27000	2583.7\\
27600	2535.4\\
28200	2533.3\\
28800	2427.4\\
29400	2506.2\\
30000	2472\\
};
\addplot [color=mycolor51, forget plot]
  table[row sep=crcr]{%
0	4395.3\\
600	5484.5\\
1800	6186.2\\
2400	6217.1\\
3000	6308.8\\
3600	6104.7\\
4200	5990.9\\
4800	5902.7\\
5400	5765.5\\
6000	5604.5\\
6600	5961.3\\
7200	5823.3\\
7800	5975.9\\
8400	5681.9\\
9000	5964.6\\
9600	5839.4\\
10200	5904.2\\
10800	5870.8\\
11400	6012.3\\
12000	5970.2\\
12600	6032.4\\
13200	6409.5\\
13800	6299.5\\
14400	6034.1\\
15000	5861.4\\
15600	5908.3\\
16200	5914.1\\
16800	5787.7\\
17400	5615.8\\
18000	5821.6\\
18600	5805.2\\
19200	6221.9\\
19800	6331.9\\
20400	6077.7\\
21000	6012.6\\
21600	6076.2\\
22200	6198.9\\
22800	5749.9\\
23400	5604.1\\
24000	5733.1\\
25200	5953.6\\
25800	5985\\
26400	5660.8\\
27000	5626\\
27600	5795.9\\
28200	5821.6\\
28800	5996.6\\
29400	5949.8\\
30000	5925\\
};
\addplot [color=mycolor52, forget plot]
  table[row sep=crcr]{%
0	4395.3\\
600	4306.5\\
1200	4109.7\\
2400	3612.8\\
3000	3800.7\\
3600	3777.3\\
4200	3600.9\\
4800	3590.2\\
5400	3635.9\\
6000	3552.5\\
6600	3736.5\\
7200	3548.5\\
7800	3434.1\\
8400	3416.3\\
9000	3475.3\\
9600	3423.4\\
10800	3513.3\\
11400	3507.6\\
12000	3441.4\\
12600	3413.3\\
13200	3675.1\\
13800	3559.6\\
14400	3641\\
15000	3472.3\\
15600	3456.3\\
16200	3403.7\\
16800	3638.7\\
17400	3423.2\\
18000	3573.5\\
18600	3574.6\\
19200	3442.1\\
19800	3437.1\\
20400	3420.4\\
21000	3266.8\\
21600	3406.3\\
22200	3497.8\\
22800	3444.6\\
23400	3445.8\\
24000	3380.2\\
24600	3557.9\\
25200	3527.6\\
25800	3446.1\\
27000	3468\\
27600	3567.4\\
28800	3556.3\\
29400	3578.7\\
30000	3587.7\\
};
\addplot [color=mycolor53, forget plot]
  table[row sep=crcr]{%
0	4395.3\\
600	5763.9\\
1200	6609.4\\
1800	6343.8\\
2400	6201.9\\
3000	6269.1\\
3600	5585.9\\
4200	5846.8\\
4800	5755.4\\
5400	6103.2\\
6000	5975.9\\
6600	5669.4\\
7200	5635.6\\
7800	5756.7\\
8400	5405.4\\
9000	5474.7\\
9600	5634.8\\
10200	5970.1\\
10800	5642.6\\
11400	5604.5\\
12600	5422.8\\
13200	5614.6\\
13800	5707.5\\
14400	5546.8\\
15000	5592.9\\
15600	5944.6\\
16200	5903.1\\
16800	5548.1\\
17400	5555.9\\
18000	5513.3\\
18600	5663.4\\
19200	5695.2\\
19800	5698.6\\
20400	5777.8\\
21000	5558.9\\
21600	5666.8\\
22200	5504.2\\
22800	5377.1\\
23400	5519.9\\
24000	5851.6\\
25200	5631.8\\
25800	5599\\
26400	5436.4\\
27000	5809.4\\
27600	5466.9\\
28800	5520.6\\
29400	5422.2\\
30000	5655.3\\
};
\addplot [color=mycolor54, forget plot]
  table[row sep=crcr]{%
0	4395.3\\
600	4595.8\\
1200	4823\\
1800	4612.5\\
2400	4777.9\\
3000	4799.3\\
3600	4659.3\\
4200	4680.2\\
4800	4610.1\\
5400	4645\\
6000	4495.8\\
6600	4537.6\\
7200	4656.5\\
8400	4522.1\\
9000	4509.6\\
9600	4417.2\\
10200	4457.1\\
10800	4566.3\\
11400	4432.1\\
12000	4545.9\\
12600	4572.9\\
13200	4663.1\\
13800	4413.1\\
14400	4540.8\\
15000	4487.1\\
15600	4489\\
16200	4427.1\\
16800	4557.2\\
17400	4524.6\\
18000	4526.9\\
18600	4310.7\\
19200	4408.9\\
19800	4776\\
20400	4631.8\\
21000	4515.6\\
21600	4644.8\\
22200	4393.3\\
22800	4649.3\\
23400	4537.6\\
24000	4581.1\\
24600	4594.4\\
25200	4700.9\\
25800	4569.8\\
26400	4683\\
27000	4687.4\\
27600	4706.4\\
28200	4687.9\\
28800	4794.9\\
29400	4727.7\\
30000	4599.3\\
};
\addplot [color=mycolor55, forget plot]
  table[row sep=crcr]{%
0	4395.3\\
600	4829.9\\
1200	5329.2\\
1800	5553.5\\
2400	5680.1\\
4200	5952.6\\
4800	5766.1\\
5400	5607.6\\
6000	5691.6\\
6600	5862.2\\
7200	5704.9\\
7800	5511.7\\
8400	5840.1\\
9000	5957.5\\
9600	5578.2\\
10200	5629.8\\
10800	5831.1\\
11400	5820.1\\
12000	5703.1\\
12600	5768.2\\
13200	5770.8\\
13800	5829\\
14400	5951.4\\
15600	5845.6\\
16200	5744.2\\
16800	5878.3\\
17400	5890.4\\
18000	5699.4\\
18600	5720.1\\
19200	5785.3\\
19800	5797.1\\
20400	5792.6\\
21000	5727.9\\
21600	5716\\
22200	5663.9\\
22800	5586\\
23400	5839.3\\
24000	5792.8\\
24600	5839.6\\
25200	5762.5\\
25800	5798.2\\
26400	5701.6\\
27000	5904.5\\
27600	5773.2\\
28800	5767.6\\
29400	5803.2\\
30000	5980.4\\
};
\addplot [color=mycolor56, forget plot]
  table[row sep=crcr]{%
0	4395.3\\
600	4540.5\\
1200	4469.8\\
2400	4131.5\\
3000	4073.8\\
4800	3850.8\\
6000	3927.4\\
6600	3830.8\\
7800	3891.6\\
8400	4040.5\\
9000	4041\\
9600	4086.2\\
10200	4006.4\\
11400	4062.6\\
12000	4062.8\\
12600	3990.2\\
13200	4072\\
13800	4011\\
14400	3993.5\\
15000	3950.7\\
15600	4062.8\\
16800	4226.5\\
17400	3991.4\\
18000	4028.5\\
18600	3959.1\\
19200	3938.1\\
19800	3938\\
20400	3920.3\\
21000	3995.8\\
21600	3915.8\\
22200	3901.6\\
22800	4080.9\\
23400	4126.2\\
24000	4092.2\\
24600	3843.6\\
25200	3868.9\\
25800	3931.6\\
26400	4087.9\\
27000	3964\\
27600	3983.7\\
28200	4183.7\\
28800	4242.5\\
29400	4166.5\\
30000	4060.2\\
};
\addplot [color=mycolor57, forget plot]
  table[row sep=crcr]{%
0	4395.3\\
600	5455.9\\
1200	6595.3\\
1800	6896.8\\
2400	6874.4\\
3000	6697.1\\
3600	6743.3\\
4200	6893.3\\
4800	6773.9\\
5400	6871.2\\
6000	6872.1\\
6600	6731.8\\
7200	6822.7\\
7800	6674.9\\
8400	6652.1\\
9000	6591.8\\
9600	6738.2\\
10200	6720\\
10800	6758.4\\
11400	6582.8\\
12600	6669\\
13200	6658.8\\
13800	6697.3\\
15000	6895.6\\
15600	6769.7\\
16200	6738.6\\
16800	6731.4\\
17400	6632.5\\
18000	6771\\
18600	6990.1\\
19200	7063.1\\
19800	6926.9\\
20400	7023\\
21000	7071.2\\
21600	7034.1\\
22200	6915.1\\
22800	6654.4\\
23400	6708.6\\
24000	6508.8\\
24600	6548.4\\
25200	6768.2\\
25800	6897.5\\
26400	6757.9\\
28200	6482.6\\
28800	6532\\
29400	6659.6\\
30000	6754.8\\
};
\addplot [color=mycolor58, forget plot]
  table[row sep=crcr]{%
0	4395.3\\
600	4339.7\\
1200	4033.4\\
1800	3772.1\\
2400	3636.6\\
3000	3827.8\\
3600	3724.9\\
4800	3547.3\\
5400	3518.3\\
6000	3515.3\\
6600	3677.9\\
7200	3575.7\\
7800	3644.9\\
8400	3656.2\\
9000	3452\\
9600	3468.1\\
10200	3541.2\\
10800	3670.2\\
11400	3791.5\\
12000	3630.8\\
12600	3564.4\\
13200	3688.8\\
13800	3622.5\\
14400	3624.8\\
15000	3648.3\\
15600	3536.5\\
16200	3530.4\\
16800	3689.6\\
17400	3681.7\\
18000	3568.8\\
18600	3541.5\\
19200	3774.2\\
19800	3502.7\\
20400	3676.1\\
21000	3626.7\\
21600	3510.1\\
22200	3559.6\\
22800	3622.4\\
23400	3525\\
24000	3601.4\\
24600	3584.2\\
25200	3684.1\\
25800	3540.8\\
26400	3663.9\\
27000	3680\\
27600	3520\\
28200	3583.3\\
28800	3521.7\\
29400	3661\\
30000	3776.2\\
};
\addplot [color=mycolor59, forget plot]
  table[row sep=crcr]{%
0	4395.3\\
600	4414.2\\
1200	4686.5\\
1800	4849.3\\
2400	4956.7\\
3000	4975.7\\
3600	4865.4\\
4200	4978.6\\
4800	4999.9\\
5400	5088.1\\
6000	5133.1\\
7800	5084.6\\
8400	5099.8\\
9000	4877\\
9600	4957.2\\
10800	5156.6\\
11400	4937.5\\
12000	4925.4\\
12600	4943.4\\
13200	5043\\
15000	5061.2\\
15600	5045.5\\
16200	4932\\
16800	5155.3\\
17400	5065.6\\
18000	5099.2\\
18600	4969.1\\
19200	5052.7\\
19800	5043.2\\
20400	4918.6\\
21000	4963.8\\
21600	5043.2\\
22800	5059.8\\
23400	4996.9\\
24000	4870.2\\
24600	4942.2\\
25200	5086.9\\
25800	4996.1\\
26400	5095.5\\
27600	5086.8\\
28200	4995\\
28800	5019.1\\
29400	5142.3\\
30000	5038.6\\
};
\addplot [color=mycolor60, forget plot]
  table[row sep=crcr]{%
0	4395.3\\
600	3898.9\\
1200	3677.7\\
1800	3640\\
2400	3620.9\\
3000	3643.9\\
3600	3646.2\\
4200	3583.9\\
4800	3570.5\\
6000	3604\\
6600	3663.8\\
7200	3590.1\\
7800	3646.2\\
8400	3609.4\\
9600	3702.2\\
10200	3511.7\\
10800	3555.7\\
11400	3530.3\\
12000	3546.7\\
12600	3542.2\\
13200	3631.3\\
13800	3623.4\\
14400	3635.9\\
15000	3775.6\\
15600	3664.5\\
16200	3631.3\\
16800	3571.8\\
17400	3599.4\\
18000	3511.9\\
18600	3517.2\\
19200	3432\\
19800	3476.1\\
20400	3540.8\\
21000	3526.6\\
21600	3438\\
22200	3452.8\\
22800	3503.7\\
23400	3568.3\\
24600	3544.4\\
25200	3433.1\\
25800	3466.3\\
26400	3541.6\\
27000	3455.6\\
27600	3441.4\\
28200	3533.2\\
28800	3528.4\\
29400	3643.8\\
30000	3658.6\\
};
\addplot [color=mycolor61, forget plot]
  table[row sep=crcr]{%
0	4395.3\\
600	3562.5\\
1200	3264.2\\
1800	3051.7\\
2400	3110\\
3000	3059.6\\
3600	3120.8\\
4200	3099.7\\
4800	3103\\
5400	3145.4\\
6000	3138.2\\
6600	3098.6\\
7200	3165.7\\
7800	3252.9\\
8400	3124\\
9000	3073.6\\
9600	3050.5\\
10200	3129.1\\
10800	3060.4\\
11400	3099.5\\
12000	3120.8\\
12600	3154\\
13200	3126.7\\
13800	3114.3\\
14400	3144.3\\
15000	3068.1\\
16200	3079.2\\
18000	3145.5\\
18600	3147\\
19200	3113.9\\
19800	3160.4\\
20400	3124.6\\
21000	3237.7\\
21600	3172.4\\
22200	3030.5\\
22800	3185.8\\
23400	3263.6\\
24000	3093.4\\
24600	3103.2\\
25800	3100.8\\
26400	3108.1\\
27000	3069\\
27600	3090.6\\
28200	3152.9\\
28800	3203.5\\
29400	3182.9\\
30000	3072.9\\
};
\addplot [color=mycolor62, forget plot]
  table[row sep=crcr]{%
0	4395.3\\
600	4391.4\\
1200	4881.5\\
1800	5008\\
2400	5000.3\\
3000	5052.9\\
3600	5085.5\\
4200	4940.8\\
4800	4927.8\\
5400	4972\\
6000	4926\\
6600	4903\\
7200	5108.9\\
7800	4918.6\\
8400	5040.9\\
9600	5053.3\\
10200	5252.5\\
10800	4953\\
11400	4906.4\\
12600	5045.6\\
13800	5141.9\\
14400	5026.4\\
15000	5076.1\\
15600	4939.2\\
16200	5098\\
16800	4995\\
17400	4964.3\\
18000	4973.5\\
18600	4936.5\\
19200	5034.6\\
19800	5018.8\\
20400	5023.4\\
21000	5117.6\\
21600	5119.1\\
22200	4991.3\\
22800	4965.9\\
24000	5048.5\\
25800	5023.4\\
26400	4973.6\\
27000	4866.6\\
27600	5013.7\\
28200	5021.1\\
28800	4933.5\\
29400	5045.7\\
30000	4966.5\\
};
\addplot [color=mycolor63, forget plot]
  table[row sep=crcr]{%
0	4395.3\\
600	3446.4\\
1200	2844.5\\
1800	2686.2\\
2400	2544.2\\
3000	2545.7\\
3600	2424.2\\
4200	2499.6\\
4800	2458\\
5400	2522.2\\
6000	2487.6\\
6600	2471.5\\
7200	2366.6\\
7800	2456.2\\
8400	2400.5\\
9000	2390\\
9600	2399.5\\
10200	2356.1\\
10800	2428.1\\
11400	2383.7\\
12000	2390.2\\
12600	2322\\
13200	2352.9\\
13800	2460.9\\
15000	2384.4\\
15600	2401.8\\
16200	2444.9\\
16800	2448.5\\
17400	2397.6\\
18000	2428.7\\
19800	2380.6\\
20400	2468.5\\
21000	2437\\
21600	2429\\
22200	2384.3\\
22800	2423\\
23400	2452.7\\
24000	2468.3\\
24600	2424.1\\
25200	2433\\
25800	2414.1\\
26400	2356.4\\
27600	2473.4\\
28200	2452.5\\
28800	2451.3\\
29400	2381.1\\
30000	2404\\
};
\addplot [color=mycolor64, forget plot]
  table[row sep=crcr]{%
0	4395.3\\
600	3417\\
1200	3022.9\\
1800	3026.1\\
2400	2955.9\\
3000	2946.8\\
3600	3035.6\\
4200	3074.7\\
4800	3020\\
5400	2980.2\\
6000	3075.7\\
6600	3056.5\\
7800	2925.7\\
8400	3011.3\\
9000	3073.5\\
10200	3040\\
10800	3023.6\\
12000	3038.2\\
12600	2996.1\\
13200	3000.4\\
13800	2978.6\\
14400	3112.1\\
15000	2974\\
15600	2963\\
16200	3055.3\\
16800	3121.2\\
17400	3020.1\\
18000	3022.7\\
19200	3131.3\\
19800	3068.2\\
21000	3090.8\\
21600	3036.2\\
22200	3043.9\\
22800	3074.4\\
23400	3033.8\\
24000	3008.6\\
24600	3030.6\\
25200	3012.1\\
27000	2986.9\\
28200	3106.6\\
28800	3122.1\\
29400	3083.1\\
30000	3020.1\\
};
\addplot [color=mycolor65, forget plot]
  table[row sep=crcr]{%
0	4395.3\\
600	3284.9\\
1200	2750.4\\
1800	2631.7\\
2400	2604.6\\
3000	2523.3\\
3600	2515.2\\
4200	2550.3\\
4800	2488.4\\
5400	2484.2\\
6000	2523.7\\
6600	2536.9\\
7200	2513.7\\
7800	2558\\
8400	2562.8\\
9000	2454.3\\
9600	2493.8\\
10200	2341.9\\
10800	2596.5\\
11400	2644.5\\
12000	2581.8\\
12600	2575.5\\
13200	2483.7\\
13800	2466.5\\
15000	2587.2\\
15600	2570.3\\
16200	2580.2\\
16800	2692.2\\
17400	2602.5\\
18000	2507.8\\
18600	2508.7\\
19200	2497.8\\
19800	2567.4\\
21000	2535.8\\
21600	2503.9\\
22200	2569.8\\
22800	2491.2\\
23400	2486.9\\
24000	2453.7\\
24600	2497.5\\
25200	2572.2\\
25800	2621.7\\
26400	2557.6\\
27000	2531.7\\
27600	2557.4\\
28200	2572.1\\
28800	2631.9\\
29400	2553.3\\
30000	2597\\
};
\addplot [color=mycolor66, forget plot]
  table[row sep=crcr]{%
0	4395.3\\
600	3680.8\\
1200	3417.2\\
1800	3429\\
2400	3367.6\\
3000	3452\\
3600	3460.3\\
4200	3510.9\\
4800	3431.3\\
6600	3373.2\\
7200	3310.8\\
7800	3370.4\\
8400	3381.6\\
9000	3266.7\\
9600	3344.8\\
10200	3413.1\\
10800	3459\\
11400	3456\\
12000	3482.3\\
12600	3486.3\\
13200	3455.5\\
13800	3465.7\\
14400	3416\\
15000	3419.5\\
15600	3379.8\\
16200	3435.6\\
16800	3409.2\\
17400	3421.2\\
18000	3394.6\\
19200	3374.7\\
19800	3428.4\\
20400	3464.5\\
21600	3371.1\\
22200	3460.8\\
22800	3488.5\\
23400	3561.7\\
24600	3410.9\\
25200	3363\\
25800	3450.2\\
26400	3492.1\\
27000	3447.6\\
27600	3362\\
28800	3427.2\\
29400	3369.6\\
30000	3360.8\\
};
\addplot [color=mycolor67, forget plot]
  table[row sep=crcr]{%
0	4395.3\\
600	3863.4\\
1200	3545\\
1800	3488.8\\
2400	3283.4\\
3000	3233.1\\
3600	3128\\
4200	3185\\
4800	3383.9\\
5400	3217.3\\
6000	3178.5\\
6600	3159.3\\
7200	3232.4\\
7800	3240.9\\
8400	3281.4\\
9000	3181.8\\
9600	3268.4\\
10200	3227.6\\
10800	3140.1\\
11400	3123\\
12000	3240\\
12600	3262.2\\
13200	3210.3\\
13800	3197\\
14400	3217.7\\
15000	3127.9\\
15600	3218.9\\
16200	3236.8\\
16800	3196.2\\
18000	3221.6\\
18600	3245.5\\
19200	3232.2\\
19800	3261.7\\
20400	3278.8\\
21000	3152.7\\
22200	3248.7\\
22800	3243.5\\
24000	3211.8\\
24600	3238.4\\
25200	3311.2\\
25800	3127\\
26400	3172.5\\
27000	3160.9\\
27600	3224.6\\
28200	3184.9\\
28800	3209\\
29400	3096.7\\
30000	3178.8\\
};
\addplot [color=mycolor67]
  table[row sep=crcr]{%
0	4395.3\\
600	3208.7\\
1200	2690.8\\
1800	2602.1\\
2400	2593\\
3000	2624.9\\
3600	2677.9\\
4800	2592.9\\
5400	2643.5\\
6000	2665.5\\
6600	2642.5\\
7800	2643.5\\
8400	2613.7\\
9000	2612.4\\
9600	2623.2\\
10200	2642.4\\
10800	2651.6\\
11400	2610.2\\
12000	2626.7\\
12600	2586.8\\
13200	2646.8\\
13800	2588.3\\
14400	2605.9\\
15000	2689.3\\
15600	2623.3\\
16200	2573.9\\
16800	2611.3\\
17400	2701.3\\
18000	2713.8\\
18600	2615.5\\
19200	2575.1\\
19800	2586.8\\
20400	2545\\
21000	2532.5\\
21600	2554.3\\
22200	2596.5\\
22800	2612.5\\
24000	2586.3\\
24600	2670.8\\
25200	2669.7\\
25800	2594.7\\
26400	2564.9\\
27000	2667.4\\
27600	2646.6\\
28200	2611.9\\
28800	2672.6\\
30000	2588.2\\
};
\addlegendentry{Classe k=43}

\addplot [color=mycolor68, forget plot]
  table[row sep=crcr]{%
0	4395.3\\
600	3449.5\\
1200	2823.7\\
1800	2709.3\\
2400	2643.3\\
3000	2610.1\\
3600	2516\\
4200	2483.4\\
4800	2494\\
6000	2466.1\\
6600	2449.3\\
7200	2478\\
7800	2523.5\\
8400	2514.3\\
9000	2540.8\\
9600	2464.8\\
10200	2449.8\\
10800	2461.4\\
11400	2453.6\\
12000	2466.2\\
12600	2506.7\\
13800	2383.4\\
14400	2402.2\\
15000	2464.9\\
15600	2505.3\\
16200	2484.4\\
16800	2491.1\\
17400	2448.5\\
18000	2460.5\\
18600	2425.9\\
19200	2464.9\\
19800	2463.3\\
20400	2441.3\\
21000	2465.9\\
21600	2471.6\\
22200	2456.3\\
24600	2455.1\\
25200	2470.4\\
25800	2516.5\\
26400	2463.1\\
27000	2456.2\\
27600	2463.2\\
28200	2492.1\\
28800	2468.2\\
29400	2479\\
30000	2503.6\\
};
\addplot [color=mycolor69, forget plot]
  table[row sep=crcr]{%
0	4395.3\\
600	3627.7\\
1200	3064.1\\
1800	2929.5\\
2400	2858.9\\
3000	2817.1\\
3600	2862.7\\
4200	2870.3\\
4800	2926.5\\
5400	2876.3\\
6600	2877.5\\
7200	2903.2\\
8400	2891.5\\
9000	2912.4\\
9600	2908.8\\
10200	2917.6\\
10800	2885.4\\
11400	2841\\
12000	2898.7\\
12600	2936.5\\
13800	2839.3\\
14400	2882.2\\
15000	2848.6\\
15600	2907.4\\
16200	2907.8\\
16800	2868.3\\
17400	2929.4\\
18000	2967.8\\
18600	2945.4\\
19200	2964.5\\
19800	2945.5\\
20400	2880\\
21000	2911.9\\
21600	2864.9\\
22800	2890.9\\
23400	2858.4\\
24600	2959\\
25200	2920.5\\
25800	2901.6\\
26400	2858.1\\
27000	2912.4\\
27600	2932.8\\
28200	2885.4\\
28800	2858.1\\
29400	2897.1\\
30000	2869.9\\
};
\addplot [color=mycolor70, forget plot]
  table[row sep=crcr]{%
0	4395.3\\
600	3071\\
1200	2203.9\\
1800	2056.7\\
2400	2074.3\\
3000	2027.3\\
3600	2058\\
4200	2129.6\\
4800	2084\\
5400	2070.2\\
6000	2117.1\\
6600	2134.3\\
7200	2122.7\\
7800	2069.2\\
8400	2096.3\\
9600	2099.4\\
10200	2091.6\\
10800	2028.2\\
12000	2083.8\\
12600	2109.4\\
13200	2165.2\\
14400	2187\\
15000	2082.5\\
15600	2142.5\\
16200	2103.2\\
16800	2098.9\\
17400	2137.6\\
18000	2097.7\\
18600	2077.4\\
19200	2067.4\\
19800	2136.8\\
20400	2081.4\\
21000	2113.4\\
21600	2084.5\\
22200	2118.2\\
23400	2091.2\\
24000	2103\\
24600	2071.3\\
25200	2088.8\\
25800	2134.9\\
26400	2095.2\\
27000	2119\\
28200	2027.3\\
28800	2086.3\\
29400	2075.4\\
30000	2157.7\\
};
\addplot [color=mycolor71, forget plot]
  table[row sep=crcr]{%
0	4395.3\\
600	3295.9\\
1200	2686.5\\
1800	2592.3\\
3000	2561.5\\
3600	2514.3\\
4200	2511\\
4800	2520.1\\
5400	2488.1\\
6000	2503.4\\
6600	2488.2\\
7200	2461\\
7800	2476.6\\
8400	2477.8\\
9000	2541.2\\
9600	2498\\
10800	2486.8\\
11400	2489.7\\
12000	2512.3\\
12600	2501.5\\
13200	2428.6\\
13800	2416.8\\
14400	2475.6\\
15000	2552.9\\
16200	2505.6\\
16800	2517.9\\
17400	2513.8\\
18000	2529.1\\
18600	2514\\
19200	2523\\
19800	2498.7\\
20400	2543.3\\
21000	2538.4\\
21600	2509.6\\
22200	2538.4\\
22800	2472.7\\
23400	2520.3\\
24000	2541.9\\
24600	2549.4\\
25200	2518.9\\
26400	2531.8\\
27000	2501.4\\
27600	2482.7\\
28200	2493.8\\
28800	2475.5\\
29400	2466.4\\
30000	2483.1\\
};
\addplot [color=mycolor72, forget plot]
  table[row sep=crcr]{%
0	4395.3\\
600	3311.6\\
1200	2505.6\\
1800	2284.7\\
2400	2224.3\\
3000	2221.7\\
3600	2234.7\\
4200	2304.3\\
4800	2200.1\\
5400	2255.6\\
6000	2276.7\\
6600	2154.5\\
7800	2182\\
8400	2225.9\\
9000	2219.3\\
9600	2153.7\\
10200	2197.6\\
11400	2226.7\\
12000	2199.6\\
12600	2188.7\\
13800	2235.6\\
14400	2232.2\\
15000	2155.6\\
15600	2207.7\\
16200	2175.5\\
16800	2196.9\\
17400	2146.7\\
18000	2160\\
18600	2197.3\\
19200	2183.2\\
19800	2137.6\\
20400	2202.5\\
21000	2136\\
21600	2206.6\\
22800	2196.9\\
23400	2212\\
24000	2279.8\\
24600	2219.1\\
25200	2169.6\\
25800	2176.3\\
26400	2165.9\\
27000	2161.6\\
27600	2195.9\\
28200	2219.5\\
28800	2323.4\\
29400	2313.5\\
30000	2206.9\\
};
\addplot [color=mycolor73, forget plot]
  table[row sep=crcr]{%
0	4395.3\\
600	3348.5\\
1200	2757.8\\
1800	2721.2\\
2400	2648.3\\
3600	2638.9\\
4200	2653.7\\
4800	2696.8\\
5400	2705.3\\
6000	2613.8\\
6600	2584\\
7200	2582.5\\
7800	2603.1\\
8400	2551.4\\
9000	2600\\
9600	2596.6\\
10200	2633.9\\
10800	2637.4\\
11400	2611.9\\
12000	2645\\
12600	2661\\
13200	2595.1\\
13800	2568.6\\
14400	2631.6\\
15000	2653.1\\
15600	2647.6\\
16200	2590.9\\
16800	2552.8\\
17400	2583.1\\
18000	2555.1\\
18600	2610\\
19800	2670.6\\
20400	2628.9\\
21600	2660.5\\
22200	2645.4\\
22800	2654\\
23400	2623.8\\
24000	2654.4\\
24600	2630.9\\
25800	2567.6\\
27000	2569.4\\
28200	2546.8\\
28800	2601.8\\
29400	2550.3\\
30000	2601.2\\
};
\addplot [color=mycolor74, forget plot]
  table[row sep=crcr]{%
0	4395.3\\
600	2993.2\\
1200	2003.3\\
1800	1716.4\\
2400	1598.2\\
3000	1539\\
3600	1553.3\\
4200	1531.1\\
4800	1549.5\\
5400	1526.7\\
6000	1554.1\\
6600	1523.2\\
7200	1518.5\\
7800	1536.6\\
8400	1540.3\\
9000	1550.6\\
9600	1544.6\\
10200	1582.7\\
10800	1598.8\\
11400	1517.8\\
12000	1519.5\\
12600	1545.9\\
13200	1560\\
13800	1525.8\\
14400	1531.2\\
15000	1515.7\\
15600	1515.7\\
16200	1533\\
16800	1582.4\\
17400	1571\\
18000	1538.8\\
18600	1540.4\\
19200	1563.5\\
19800	1533.9\\
20400	1525.5\\
21000	1534.9\\
21600	1533\\
22200	1518.6\\
22800	1492.7\\
23400	1528.6\\
24000	1543.9\\
24600	1546.5\\
25200	1531.4\\
25800	1475.9\\
26400	1482.6\\
27000	1537.2\\
27600	1559\\
28200	1556.4\\
28800	1512.2\\
29400	1541.4\\
30000	1560.9\\
};
\addplot [color=mycolor75, forget plot]
  table[row sep=crcr]{%
0	4395.3\\
600	3295.1\\
1200	2492.4\\
1800	2278.7\\
2400	2248\\
3600	2211.9\\
4200	2225.6\\
4800	2194.2\\
5400	2203.9\\
6600	2160.7\\
7200	2223.6\\
7800	2198.4\\
8400	2244\\
9000	2208.9\\
9600	2201.5\\
10200	2209.5\\
10800	2193.1\\
11400	2199.6\\
12000	2187.7\\
12600	2220.5\\
13200	2204.5\\
13800	2208.9\\
14400	2198.6\\
15000	2227.8\\
16200	2198.7\\
16800	2245.9\\
17400	2211.3\\
18000	2271.1\\
18600	2245\\
19200	2255.1\\
19800	2200.1\\
20400	2207.6\\
21000	2236.8\\
21600	2207.4\\
22200	2169.8\\
22800	2215.7\\
23400	2241.1\\
24000	2217.9\\
24600	2233.2\\
25200	2234.2\\
25800	2132.2\\
26400	2125.6\\
27000	2178.6\\
27600	2190.1\\
28200	2238.1\\
28800	2226.8\\
29400	2201.3\\
30000	2213.6\\
};
\addplot [color=mycolor76, forget plot]
  table[row sep=crcr]{%
0	4395.3\\
600	3120.1\\
1200	2351.2\\
1800	2185.9\\
3000	2058.6\\
3600	2063.7\\
4200	2049.5\\
4800	2009\\
5400	2001.7\\
6000	1976\\
6600	1991.2\\
7200	1988.2\\
7800	2049.1\\
8400	2020.9\\
9000	2036.4\\
9600	2006.9\\
10800	2047.7\\
11400	2008.6\\
12600	1995.8\\
13200	2038.2\\
13800	2064.3\\
14400	2033.3\\
15000	1953.7\\
15600	2004\\
16200	1996\\
16800	2029.5\\
17400	2022.1\\
18000	2027.1\\
18600	2001.1\\
19800	2009.5\\
20400	2025\\
21000	2058.9\\
21600	2048.1\\
22200	2023\\
22800	2044.1\\
23400	2005.7\\
24000	2007.8\\
24600	2028.4\\
25200	2060.4\\
25800	2028.3\\
26400	2034.6\\
27000	2002\\
27600	1990.5\\
28200	1989.5\\
28800	2008.1\\
29400	2050.3\\
30000	2060.1\\
};
\addplot [color=mycolor77, forget plot]
  table[row sep=crcr]{%
0	4395.3\\
600	2955.1\\
1200	2031.9\\
1800	1781.1\\
2400	1701.3\\
3600	1622.5\\
4800	1591.8\\
5400	1613.1\\
6600	1587.1\\
7200	1565.7\\
7800	1530.5\\
8400	1540.4\\
9000	1565.9\\
9600	1576.5\\
10200	1605.5\\
10800	1604.9\\
11400	1556.4\\
12000	1541.4\\
13200	1545.6\\
13800	1597.7\\
14400	1577.1\\
15000	1563.8\\
15600	1597.2\\
16200	1561.5\\
17400	1568.9\\
18000	1578.3\\
19800	1568.1\\
20400	1570.8\\
21000	1583.9\\
21600	1576.2\\
22200	1595.3\\
22800	1596\\
23400	1607.5\\
24000	1597.1\\
24600	1576\\
25200	1535\\
25800	1582\\
26400	1573.7\\
27000	1582\\
27600	1578.3\\
28200	1608.9\\
28800	1580.5\\
29400	1562.6\\
30000	1590\\
};
\addplot [color=mycolor78, forget plot]
  table[row sep=crcr]{%
0	4395.3\\
600	3412.1\\
1200	2684.1\\
1800	2512.5\\
2400	2407.4\\
3000	2389.4\\
3600	2427.8\\
4200	2345.4\\
4800	2296.4\\
5400	2297.9\\
6000	2267.5\\
6600	2272.4\\
7200	2373.1\\
7800	2307\\
8400	2233.6\\
9000	2292.2\\
10200	2325.5\\
10800	2363.9\\
11400	2338.4\\
12000	2276\\
12600	2294.8\\
13800	2257.3\\
14400	2298.1\\
15000	2318.4\\
15600	2251.4\\
16200	2289.3\\
17400	2263.8\\
18000	2219.2\\
18600	2228.8\\
19800	2292.8\\
20400	2291.1\\
21600	2236.5\\
22200	2242.9\\
22800	2294.1\\
23400	2321.4\\
24000	2319.6\\
24600	2293.4\\
25200	2283.4\\
25800	2323.2\\
26400	2263.1\\
27000	2265.1\\
27600	2318.4\\
28200	2354.7\\
28800	2344\\
29400	2274\\
30000	2270.4\\
};
\addplot [color=mycolor79, forget plot]
  table[row sep=crcr]{%
0	4395.3\\
600	3222.4\\
1200	2431.9\\
1800	2267.2\\
2400	2241.1\\
3000	2279.3\\
3600	2251.2\\
4200	2231.5\\
4800	2241.5\\
5400	2152.4\\
6000	2194.4\\
6600	2141.3\\
7200	2147.1\\
7800	2172.6\\
8400	2179\\
9000	2196.7\\
9600	2174.9\\
10200	2166.8\\
10800	2101.6\\
11400	2117.9\\
12000	2143\\
12600	2187\\
13200	2136.4\\
13800	2195\\
14400	2159.1\\
15000	2165.7\\
15600	2123.8\\
16200	2163.2\\
16800	2210.7\\
17400	2158\\
18000	2142.8\\
18600	2172.5\\
19200	2194.9\\
19800	2189.7\\
20400	2139.2\\
21000	2167.5\\
22200	2179.2\\
22800	2205.6\\
24000	2144.2\\
24600	2133.6\\
25200	2111.5\\
25800	2149.4\\
26400	2139.8\\
27000	2157.4\\
27600	2142.9\\
28200	2115.2\\
28800	2192\\
29400	2199.1\\
30000	2185.6\\
};
\addplot [color=mycolor79, forget plot]
  table[row sep=crcr]{%
0	4395.3\\
600	3269.1\\
1200	2410.3\\
1800	2207.9\\
2400	2077.8\\
3000	2040.5\\
3600	2031.3\\
4200	2059.7\\
4800	1991.9\\
5400	2001.1\\
6000	1979.7\\
6600	2007.9\\
7200	1993.9\\
7800	1953.1\\
8400	1935.1\\
9000	1998.1\\
9600	1998.2\\
10200	1972.2\\
10800	2025.7\\
11400	2006.6\\
12000	1958.3\\
12600	1994.6\\
13200	2017.1\\
13800	2023.9\\
14400	2041.3\\
15000	2018.3\\
15600	2048.7\\
16200	1983.2\\
16800	1989.7\\
17400	2043.5\\
18000	2068.5\\
18600	2018.6\\
19200	1995.5\\
19800	1991.5\\
20400	2050.8\\
21000	2051.1\\
21600	2088\\
22200	2052.9\\
22800	2092.4\\
23400	2054\\
24000	2064.4\\
24600	1983.3\\
25200	1984.8\\
25800	2067.8\\
26400	2071.4\\
27000	1985.6\\
27600	1999.1\\
28200	2026.3\\
28800	2035.8\\
29400	2033.3\\
30000	1979.1\\
};
\addplot [color=mycolor80, forget plot]
  table[row sep=crcr]{%
0	4395.3\\
600	2929.7\\
1200	1908.7\\
1800	1675.8\\
2400	1617.2\\
3000	1610.2\\
3600	1563.4\\
4200	1522.5\\
4800	1534.8\\
5400	1537.3\\
6000	1550.8\\
7200	1537\\
8400	1534.5\\
9000	1541.8\\
9600	1525.9\\
10200	1522.8\\
10800	1515.2\\
11400	1521.3\\
12000	1542.7\\
12600	1535\\
13200	1553.5\\
13800	1589.1\\
14400	1534.5\\
15600	1540.2\\
16200	1553.1\\
17400	1587.1\\
18000	1554.8\\
18600	1550.6\\
19200	1529.2\\
19800	1525.9\\
20400	1516.6\\
21000	1545.4\\
21600	1559.7\\
22200	1584.9\\
22800	1580\\
23400	1539.1\\
24600	1554.9\\
25200	1537.6\\
25800	1557.3\\
26400	1562.5\\
27000	1545.2\\
27600	1581.4\\
28200	1569\\
28800	1569.1\\
29400	1550.3\\
30000	1546.2\\
};
\addplot [color=mycolor81, forget plot]
  table[row sep=crcr]{%
0	4395.3\\
600	2913.8\\
1200	1869.7\\
1800	1606.7\\
2400	1501.1\\
3000	1459.4\\
3600	1448.8\\
4800	1489.6\\
5400	1478.5\\
6000	1473.1\\
7200	1479\\
7800	1449.9\\
9000	1452.6\\
9600	1497.9\\
10200	1491.2\\
10800	1466.6\\
11400	1476.7\\
12000	1445.1\\
12600	1476.9\\
13200	1463.4\\
13800	1456.7\\
14400	1465.4\\
15600	1450.1\\
16800	1448.4\\
17400	1469.1\\
18000	1459.4\\
18600	1476.8\\
19200	1450.4\\
19800	1452.7\\
20400	1483.8\\
21000	1451.9\\
21600	1431\\
22200	1480.5\\
22800	1458\\
23400	1445\\
24000	1460.7\\
25200	1464\\
25800	1429.7\\
27000	1416.8\\
27600	1464.3\\
28200	1490.2\\
28800	1478.3\\
29400	1444.9\\
30000	1436\\
};
\addplot [color=mycolor82, forget plot]
  table[row sep=crcr]{%
0	4395.3\\
600	3112.5\\
1200	2180.2\\
1800	1920\\
2400	1798.6\\
3000	1731.9\\
3600	1755.8\\
4200	1728.1\\
4800	1725.7\\
5400	1679.8\\
6000	1673.5\\
6600	1703.3\\
7200	1694.1\\
7800	1656.8\\
8400	1665.6\\
9000	1649.3\\
9600	1663.4\\
10200	1672.1\\
10800	1696.2\\
11400	1746.4\\
12000	1724.6\\
12600	1664.7\\
13800	1683.8\\
14400	1652.9\\
15000	1650.5\\
15600	1667.6\\
16200	1699.5\\
17400	1665\\
18000	1654.3\\
19200	1693.5\\
19800	1700.9\\
20400	1699\\
21000	1703.1\\
21600	1668.4\\
22200	1699.9\\
22800	1702.1\\
23400	1686.2\\
24600	1703.8\\
25200	1655.1\\
25800	1716.1\\
26400	1707.2\\
27000	1726.1\\
27600	1700.4\\
28200	1668.7\\
28800	1676.2\\
29400	1693.3\\
30000	1655\\
};
\addplot [color=mycolor83, forget plot]
  table[row sep=crcr]{%
0	4395.3\\
600	3156.7\\
1200	2210.9\\
1800	1921.4\\
2400	1842.5\\
3000	1796.1\\
3600	1769.2\\
4200	1770\\
4800	1723\\
5400	1732.1\\
6000	1777.7\\
6600	1735.5\\
7200	1712.4\\
7800	1731.7\\
8400	1784.5\\
9000	1803.2\\
9600	1765.5\\
10200	1741.8\\
10800	1711.4\\
11400	1690.8\\
12000	1675.3\\
12600	1704.2\\
13200	1739.1\\
13800	1735.7\\
14400	1748.6\\
15000	1710.9\\
15600	1691.1\\
16200	1748.9\\
16800	1742.9\\
17400	1731.2\\
18000	1703.2\\
18600	1687.2\\
19200	1728.7\\
19800	1755.5\\
20400	1752.3\\
21000	1780.6\\
21600	1765.9\\
22200	1714.7\\
22800	1740.7\\
23400	1723.7\\
24000	1676\\
24600	1660\\
25200	1719.4\\
26400	1711.8\\
27000	1732.7\\
29400	1734.6\\
30000	1703.9\\
};
\addplot [color=mycolor83, forget plot]
  table[row sep=crcr]{%
0	4395.3\\
600	2787.7\\
1200	1658\\
1800	1379\\
2400	1321.8\\
3000	1271.1\\
3600	1267.6\\
4200	1281.2\\
4800	1262.6\\
5400	1260.1\\
6000	1221.8\\
6600	1249.6\\
7200	1270.9\\
7800	1273.3\\
8400	1253.8\\
9000	1273.4\\
9600	1297.9\\
10200	1296.9\\
10800	1244.7\\
11400	1246.2\\
12000	1262\\
12600	1270.6\\
13200	1270.5\\
14400	1303\\
15000	1263.2\\
15600	1239.8\\
16200	1288.4\\
16800	1263.9\\
17400	1231.2\\
18000	1237.3\\
18600	1254.8\\
19200	1237.3\\
19800	1238.2\\
20400	1254.6\\
21000	1228.2\\
21600	1243.4\\
22200	1274.8\\
22800	1212.6\\
23400	1240.6\\
24000	1264.4\\
24600	1279.8\\
25200	1277.2\\
25800	1254.5\\
26400	1278.5\\
27000	1249.2\\
27600	1263.5\\
28200	1270.7\\
29400	1255.5\\
30000	1257.7\\
};
\addplot [color=mycolor84, forget plot]
  table[row sep=crcr]{%
0	4395.3\\
600	2847.4\\
1200	1727.7\\
1800	1477.1\\
2400	1397.3\\
3000	1357.1\\
3600	1302.8\\
4200	1300.4\\
4800	1237.5\\
5400	1236.6\\
6000	1294.3\\
6600	1344.9\\
7200	1321.6\\
7800	1270.5\\
8400	1276.9\\
9000	1279.4\\
9600	1291.8\\
10200	1262.9\\
10800	1230\\
11400	1317\\
12000	1267.9\\
12600	1251.5\\
13200	1219.1\\
13800	1230.4\\
14400	1257.9\\
15000	1274.3\\
15600	1276.8\\
16200	1306.7\\
16800	1265.8\\
17400	1301\\
18000	1301.9\\
18600	1274.2\\
19200	1263\\
19800	1267.4\\
20400	1223.1\\
21000	1217\\
21600	1231.1\\
22200	1282.2\\
22800	1271.3\\
23400	1287.3\\
24000	1292.6\\
24600	1272.6\\
25200	1314.7\\
25800	1294.6\\
26400	1331.9\\
27000	1301.9\\
27600	1252.2\\
28200	1272.8\\
28800	1283.6\\
29400	1305.1\\
30000	1307.3\\
};
\addplot [color=mycolor85, forget plot]
  table[row sep=crcr]{%
0	4395.3\\
600	3126.4\\
1200	1973.7\\
1800	1645.6\\
2400	1551.4\\
3000	1440.6\\
3600	1444.5\\
4200	1409.9\\
4800	1392.6\\
5400	1432.1\\
6600	1430.3\\
7200	1448.6\\
7800	1439.8\\
8400	1407.3\\
9000	1384.8\\
9600	1352.6\\
10200	1368.8\\
10800	1376.6\\
11400	1427.4\\
12600	1405.5\\
13200	1409.6\\
13800	1363.4\\
14400	1408.4\\
15600	1410.4\\
16800	1376.1\\
17400	1424.1\\
18000	1442.3\\
18600	1410.2\\
19200	1467.2\\
19800	1486.6\\
20400	1425\\
21600	1376.1\\
22200	1393.6\\
22800	1346.9\\
23400	1340.8\\
24000	1411.4\\
24600	1414.4\\
25200	1402.2\\
25800	1421.6\\
26400	1431\\
27000	1384.7\\
28200	1386\\
28800	1380.5\\
29400	1390.8\\
30000	1414.2\\
};
\addplot [color=mycolor86, forget plot]
  table[row sep=crcr]{%
0	4395.3\\
600	2858\\
1200	1758\\
1800	1382.2\\
2400	1301.5\\
3000	1206.5\\
3600	1180\\
4200	1223\\
4800	1195.5\\
5400	1247.1\\
6000	1248.6\\
6600	1233.2\\
7200	1209.1\\
7800	1165.2\\
8400	1143.3\\
9000	1193.4\\
9600	1199\\
10200	1152.1\\
10800	1204.7\\
11400	1204.4\\
12000	1167.7\\
12600	1198\\
13200	1174.6\\
13800	1140\\
14400	1133.3\\
15000	1140.8\\
15600	1154.7\\
16200	1189.7\\
16800	1200.2\\
17400	1155.6\\
18000	1127.5\\
18600	1131.9\\
19200	1146.8\\
19800	1149.5\\
20400	1169.2\\
21000	1159.9\\
21600	1189.7\\
22200	1193.8\\
22800	1215.2\\
23400	1193.2\\
24000	1195.9\\
24600	1194.9\\
25800	1172.8\\
26400	1183.8\\
27000	1219\\
27600	1191.1\\
28200	1190\\
28800	1227\\
29400	1205.4\\
30000	1158.5\\
};
\addplot [color=mycolor87, forget plot]
  table[row sep=crcr]{%
0	4395.3\\
600	2781.7\\
1200	1474.5\\
1800	1135.6\\
2400	951.86\\
3000	966.02\\
3600	1027.8\\
4200	1009.4\\
4800	1007.1\\
5400	999.56\\
6000	1009.7\\
6600	1001.2\\
7200	973.42\\
7800	991.51\\
8400	1055.4\\
9000	1023.2\\
10200	985.06\\
10800	1054.3\\
11400	1006.5\\
12000	967.84\\
12600	937.07\\
13200	1000.1\\
13800	940.83\\
15000	1030.5\\
15600	987.29\\
16200	1004.1\\
16800	967.41\\
17400	947.68\\
18000	968.46\\
18600	975.99\\
19200	996.33\\
19800	1053.4\\
20400	1031\\
21000	985.05\\
21600	1004.9\\
22200	978.23\\
22800	962.73\\
23400	1026\\
24000	1020.2\\
24600	1053.3\\
25200	1015.7\\
25800	969.54\\
26400	1022.7\\
27600	956.41\\
28200	951.91\\
28800	936.59\\
29400	967.61\\
30000	983.51\\
};
\addplot [color=mycolor88, forget plot]
  table[row sep=crcr]{%
0	4395.3\\
600	3012.6\\
1200	1809.7\\
1800	1480.1\\
2400	1329\\
3000	1250.9\\
3600	1200.7\\
4200	1222.6\\
4800	1213.9\\
5400	1227.7\\
6600	1240.3\\
7200	1203.3\\
7800	1159.8\\
8400	1195.4\\
9000	1177.6\\
9600	1196.4\\
10200	1183.1\\
10800	1165.9\\
11400	1160.2\\
12000	1171.7\\
12600	1187.3\\
13200	1149.3\\
13800	1150.2\\
14400	1175.2\\
15000	1167.5\\
16200	1222.2\\
16800	1199.3\\
17400	1201.9\\
18600	1188\\
19200	1158.3\\
19800	1158.1\\
20400	1162.5\\
21000	1189.6\\
21600	1175.5\\
22200	1189.9\\
22800	1184.3\\
23400	1203.8\\
24000	1219.1\\
24600	1190.3\\
25200	1152.4\\
25800	1172.1\\
26400	1225.1\\
27000	1163.7\\
28200	1176\\
28800	1134.7\\
29400	1129.9\\
30000	1109.2\\
};
\addplot [color=mycolor89, forget plot]
  table[row sep=crcr]{%
0	4395.3\\
600	2926.4\\
1200	1601.2\\
1800	1197.7\\
2400	996.82\\
3000	927.34\\
3600	905.93\\
4200	872.52\\
4800	853.69\\
5400	839.38\\
6000	848.18\\
6600	859.55\\
7200	855.82\\
7800	842.01\\
8400	817.25\\
9000	811.09\\
9600	829.59\\
10200	844.58\\
10800	852.58\\
11400	853.23\\
12000	878.57\\
12600	870.17\\
13200	864.87\\
13800	855.27\\
14400	836.07\\
15000	849.66\\
15600	829.28\\
16800	868.36\\
17400	843.72\\
18000	849.06\\
18600	825.96\\
19200	860.43\\
19800	851.61\\
20400	840.06\\
21000	853.32\\
21600	851.24\\
22200	869.16\\
22800	861.6\\
23400	848.18\\
24000	854.16\\
24600	844.64\\
25200	868.69\\
25800	860.82\\
26400	849.97\\
27600	882.23\\
28200	891.79\\
28800	891.08\\
29400	896.32\\
30000	888.49\\
};
\addplot [color=mycolor90, forget plot]
  table[row sep=crcr]{%
0	4395.3\\
600	2771.7\\
1200	1400.7\\
1800	995.1\\
2400	884.09\\
3000	864.81\\
3600	857.23\\
4800	812.36\\
5400	824.52\\
6000	790.83\\
6600	809.88\\
7800	782.79\\
8400	791.92\\
9000	793.59\\
10200	831.26\\
10800	844.48\\
11400	877.19\\
12000	867.32\\
12600	821.56\\
13200	809.86\\
14400	823.42\\
15000	816.2\\
15600	824.1\\
16200	801.18\\
16800	815.15\\
17400	850.69\\
18000	825\\
18600	833.64\\
19800	816.25\\
20400	815.59\\
21000	832.32\\
21600	828.26\\
22200	827.14\\
22800	809.81\\
23400	828.54\\
24000	829.09\\
24600	813.5\\
25200	833.54\\
25800	838.48\\
26400	829.41\\
27000	823.41\\
27600	810.6\\
28200	810.1\\
28800	790.04\\
29400	827.12\\
30000	827.68\\
};
\addplot [color=mycolor91, forget plot]
  table[row sep=crcr]{%
0	4395.3\\
600	2895.2\\
1200	1509.3\\
1800	1025.1\\
2400	839.15\\
3000	777.04\\
3600	752.64\\
4200	726.78\\
4800	717.2\\
5400	704.17\\
6600	723.67\\
7200	699.24\\
7800	690.09\\
8400	675.81\\
9000	693.67\\
9600	724.84\\
10800	707.19\\
11400	716.54\\
12600	711.96\\
13200	703.69\\
13800	711.4\\
14400	710.11\\
15000	699.19\\
15600	675.02\\
16200	666.82\\
16800	686.74\\
17400	726.31\\
18000	750.14\\
18600	744.33\\
19200	723.81\\
19800	736.39\\
20400	725.78\\
21000	720.8\\
21600	729.14\\
22200	701.85\\
23400	689.44\\
24000	679.56\\
24600	711.37\\
25200	718.29\\
25800	730.08\\
26400	707.18\\
27000	705.02\\
27600	719.36\\
28200	715.29\\
28800	705.4\\
29400	716.6\\
30000	693.69\\
};
\addplot [color=mycolor92, forget plot]
  table[row sep=crcr]{%
0	4395.3\\
600	2938.1\\
1200	1597.3\\
1800	1196.2\\
2400	1040.3\\
3600	937.62\\
4200	925.1\\
4800	878.38\\
5400	879.39\\
6000	907.9\\
6600	968.59\\
7200	948.08\\
7800	897.5\\
8400	868.25\\
9000	889.56\\
9600	923.12\\
10200	926.04\\
10800	941.95\\
11400	924.14\\
12000	921.09\\
12600	934.84\\
13200	887.74\\
13800	897.9\\
14400	925.37\\
15000	910.98\\
15600	883.93\\
16200	878.2\\
16800	876.09\\
17400	869.09\\
18000	884.4\\
18600	894.92\\
19200	877.18\\
19800	838.84\\
20400	857\\
21000	857.14\\
21600	877.44\\
22200	895.72\\
23400	895.21\\
24000	878.72\\
24600	881.38\\
25200	893.82\\
25800	886.12\\
27000	895.46\\
27600	892.77\\
28200	915.84\\
28800	921.09\\
29400	923.49\\
30000	938.44\\
};
\addplot [color=mycolor93, forget plot]
  table[row sep=crcr]{%
0	4395.3\\
600	3038.5\\
1200	1672.7\\
1800	1197.7\\
2400	948.72\\
3000	815.55\\
3600	807.71\\
4200	784.96\\
4800	799.83\\
5400	759.42\\
6000	746.09\\
6600	754.13\\
7800	751.77\\
8400	747.14\\
9000	735.85\\
9600	740.85\\
10200	767.51\\
10800	725.81\\
11400	720.62\\
12000	720.52\\
13200	726.39\\
13800	750.29\\
14400	712.99\\
15000	700.8\\
15600	705.13\\
16200	714.47\\
16800	748.04\\
17400	737.8\\
18000	743.62\\
18600	709.79\\
19200	725.42\\
19800	764.19\\
20400	754.91\\
21000	781.92\\
21600	782.59\\
22800	714.08\\
23400	733.6\\
24000	710.85\\
24600	720.07\\
25200	771.08\\
25800	773.03\\
26400	782.52\\
27600	749.44\\
28200	707.91\\
28800	735.7\\
29400	726.34\\
30000	752.76\\
};
\addplot [color=mycolor94, forget plot]
  table[row sep=crcr]{%
0	4395.3\\
600	3027.4\\
1200	1645.4\\
1800	1126.7\\
2400	915.28\\
3000	829.86\\
3600	788.4\\
4200	758.18\\
4800	737.41\\
5400	720.88\\
6000	694.24\\
6600	690.58\\
7200	673.45\\
7800	702.77\\
8400	725.72\\
9000	732.57\\
9600	736.15\\
10200	725.09\\
10800	694.03\\
11400	716.37\\
12000	701.68\\
12600	697.61\\
13200	708.63\\
13800	701.32\\
14400	690.27\\
15000	706.84\\
15600	718.08\\
16800	722.26\\
17400	721.34\\
18000	716.87\\
18600	733.29\\
19200	734.53\\
19800	723.04\\
20400	741.31\\
21000	740.37\\
21600	716.4\\
22200	723.85\\
22800	695.4\\
23400	702.16\\
24000	711.94\\
25200	706.48\\
25800	709.1\\
26400	705.73\\
27000	693\\
27600	712.68\\
28200	694.48\\
28800	698.98\\
29400	687.9\\
30000	685.35\\
};
\addplot [color=mycolor95, forget plot]
  table[row sep=crcr]{%
0	4395.3\\
600	2773.5\\
1200	1311.4\\
1800	791.46\\
2400	573.88\\
3000	487.17\\
3600	451.14\\
4200	433.86\\
4800	424.26\\
5400	404.9\\
6000	403.94\\
6600	401.85\\
7200	407.23\\
7800	406.69\\
8400	420.85\\
9000	399.7\\
9600	415.56\\
10200	410.25\\
10800	415.14\\
11400	429.17\\
12000	441.94\\
12600	442.19\\
13200	429.15\\
13800	436.69\\
14400	439.9\\
15000	431.08\\
15600	417.76\\
16200	402.53\\
16800	410.61\\
17400	415.6\\
18000	414.71\\
18600	419.51\\
19200	425.89\\
19800	415.07\\
20400	410.2\\
21000	432.08\\
21600	440.1\\
22200	435.35\\
22800	416.58\\
23400	437.22\\
24600	412.78\\
25200	404.31\\
25800	420.71\\
26400	420.08\\
27000	416.73\\
27600	405.59\\
28200	407.55\\
28800	407.32\\
29400	411.42\\
30000	425.63\\
};
\addplot [color=mycolor96, forget plot]
  table[row sep=crcr]{%
0	4395.3\\
600	2986.3\\
1200	1496.8\\
1800	951.04\\
2400	647.27\\
3000	534.94\\
3600	494.51\\
4200	510.19\\
4800	477.38\\
5400	469.35\\
6000	485.61\\
6600	453.75\\
7200	446.34\\
7800	448.72\\
8400	461.77\\
9000	425.11\\
9600	396.77\\
10200	427.79\\
10800	433.52\\
11400	453.85\\
12000	459.43\\
12600	437.76\\
13200	454.33\\
13800	436.5\\
14400	438.2\\
15000	469.57\\
15600	460.25\\
16200	430.45\\
16800	450.61\\
17400	484.64\\
18000	489.55\\
18600	468.11\\
19200	444.65\\
19800	453.93\\
20400	452.35\\
21000	456.26\\
21600	425.66\\
22200	421.23\\
22800	433.68\\
23400	429.62\\
24000	447.16\\
24600	473.63\\
25200	469.56\\
25800	484.18\\
26400	445.15\\
27600	433.12\\
28200	453.39\\
28800	437.79\\
29400	449.1\\
30000	476.08\\
};
\addplot [color=mycolor97, forget plot]
  table[row sep=crcr]{%
0	4395.3\\
600	2677\\
1200	1140.4\\
1800	695.67\\
2400	468.29\\
3000	381.52\\
3600	355.29\\
4200	320.54\\
4800	304.75\\
5400	295.07\\
6000	306.01\\
6600	325.24\\
7200	318.2\\
7800	309.26\\
8400	327.11\\
9000	310.47\\
9600	319.05\\
10200	339.93\\
10800	336.34\\
11400	326.71\\
12000	319.32\\
12600	336.08\\
13200	351.3\\
13800	320.14\\
14400	326.95\\
15000	331.13\\
15600	330.43\\
16200	334.44\\
16800	362.25\\
17400	357.38\\
18000	341.84\\
18600	330.73\\
19200	346.53\\
19800	352.39\\
20400	349.36\\
21000	321.37\\
21600	329.87\\
22200	329.32\\
22800	316.62\\
23400	297.98\\
24600	296.26\\
25200	297.49\\
25800	301.56\\
26400	325.04\\
27000	313.26\\
28200	325.38\\
28800	316\\
29400	320.97\\
30000	318.19\\
};
\addplot [color=mycolor98]
  table[row sep=crcr]{%
0	4395.3\\
600	3075.9\\
1200	1610.7\\
1800	990.82\\
2400	722.12\\
3000	584.45\\
3600	506.22\\
4200	478.77\\
4800	477.4\\
5400	504.26\\
6000	494.23\\
7200	435.04\\
7800	431.56\\
8400	426.55\\
9000	419.34\\
9600	444.91\\
10200	449.37\\
10800	439.9\\
11400	432.59\\
12000	433.41\\
12600	431.89\\
13200	416.98\\
13800	443.06\\
14400	423.02\\
15000	415.31\\
15600	420.27\\
16200	427.21\\
16800	426.5\\
17400	420.48\\
18000	429.2\\
18600	442.31\\
19200	439.65\\
19800	443.16\\
20400	426.08\\
21000	427.17\\
21600	436.97\\
22200	435.3\\
22800	430.66\\
23400	415.71\\
24000	422.69\\
24600	441.31\\
25200	427.8\\
25800	421.67\\
26400	424.33\\
27000	424.09\\
27600	431.77\\
28200	429.68\\
28800	431.04\\
29400	422.64\\
30000	418.1\\
};
\addlegendentry{Classe k=10}
            % \input{../../../.tex/20/KS/SizeEvolutionsfig1.tex}

            \nextgroupplot[%
                % xmin=0,xmax=30000,
                % ymin=100,ymax=1.5461e+05,
            ] % This file was created by matlab2tikz.
%
\definecolor{mycolor1}{rgb}{0.00146,0.00047,0.01387}%
\definecolor{mycolor2}{rgb}{0.01166,0.00942,0.06346}%
\definecolor{mycolor3}{rgb}{0.02943,0.02150,0.11462}%
\definecolor{mycolor4}{rgb}{0.06134,0.03659,0.17764}%
\definecolor{mycolor5}{rgb}{0.08741,0.04456,0.22481}%
\definecolor{mycolor6}{rgb}{0.12291,0.04754,0.28162}%
\definecolor{mycolor7}{rgb}{0.14907,0.04547,0.31709}%
\definecolor{mycolor8}{rgb}{0.18343,0.04033,0.35497}%
\definecolor{mycolor9}{rgb}{0.21795,0.03662,0.38352}%
\definecolor{mycolor10}{rgb}{0.24497,0.03705,0.40001}%
\definecolor{mycolor11}{rgb}{0.27135,0.04092,0.41198}%
\definecolor{mycolor12}{rgb}{0.29076,0.04564,0.41864}%
\definecolor{mycolor13}{rgb}{0.31628,0.05349,0.42512}%
\definecolor{mycolor14}{rgb}{0.33522,0.06006,0.42852}%
\definecolor{mycolor15}{rgb}{0.36028,0.06925,0.43150}%
\definecolor{mycolor16}{rgb}{0.37900,0.07625,0.43272}%
\definecolor{mycolor17}{rgb}{0.39767,0.08326,0.43318}%
\definecolor{mycolor18}{rgb}{0.41633,0.09020,0.43294}%
\definecolor{mycolor19}{rgb}{0.42877,0.09479,0.43241}%
\definecolor{mycolor20}{rgb}{0.44743,0.10160,0.43108}%
\definecolor{mycolor21}{rgb}{0.46610,0.10832,0.42912}%
\definecolor{mycolor22}{rgb}{0.47856,0.11276,0.42747}%
\definecolor{mycolor23}{rgb}{0.49726,0.11938,0.42449}%
\definecolor{mycolor24}{rgb}{0.50973,0.12377,0.42216}%
\definecolor{mycolor25}{rgb}{0.52221,0.12815,0.41955}%
\definecolor{mycolor26}{rgb}{0.53468,0.13253,0.41667}%
\definecolor{mycolor27}{rgb}{0.55339,0.13913,0.41183}%
\definecolor{mycolor28}{rgb}{0.56585,0.14357,0.40826}%
\definecolor{mycolor29}{rgb}{0.57830,0.14804,0.40441}%
\definecolor{mycolor30}{rgb}{0.59073,0.15256,0.40029}%
\definecolor{mycolor31}{rgb}{0.60314,0.15715,0.39589}%
\definecolor{mycolor32}{rgb}{0.61551,0.16182,0.39122}%
\definecolor{mycolor33}{rgb}{0.62169,0.16418,0.38878}%
\definecolor{mycolor34}{rgb}{0.63400,0.16899,0.38370}%
\definecolor{mycolor35}{rgb}{0.64626,0.17391,0.37836}%
\definecolor{mycolor36}{rgb}{0.65846,0.17896,0.37275}%
\definecolor{mycolor37}{rgb}{0.66454,0.18154,0.36985}%
\definecolor{mycolor38}{rgb}{0.67664,0.18681,0.36385}%
\definecolor{mycolor39}{rgb}{0.68865,0.19224,0.35760}%
\definecolor{mycolor40}{rgb}{0.69463,0.19502,0.35439}%
\definecolor{mycolor41}{rgb}{0.70650,0.20073,0.34778}%
\definecolor{mycolor42}{rgb}{0.71240,0.20366,0.34438}%
\definecolor{mycolor43}{rgb}{0.72410,0.20967,0.33742}%
\definecolor{mycolor44}{rgb}{0.72991,0.21276,0.33386}%
\definecolor{mycolor45}{rgb}{0.74142,0.21911,0.32658}%
\definecolor{mycolor46}{rgb}{0.74713,0.22238,0.32286}%
\definecolor{mycolor47}{rgb}{0.75279,0.22571,0.31909}%
\definecolor{mycolor48}{rgb}{0.76401,0.23255,0.31140}%
\definecolor{mycolor49}{rgb}{0.76956,0.23608,0.30749}%
\definecolor{mycolor50}{rgb}{0.77506,0.23967,0.30353}%
\definecolor{mycolor51}{rgb}{0.79129,0.25086,0.29139}%
\definecolor{mycolor52}{rgb}{0.79661,0.25473,0.28726}%
\definecolor{mycolor53}{rgb}{0.80187,0.25867,0.28310}%
\definecolor{mycolor54}{rgb}{0.81224,0.26679,0.27466}%
\definecolor{mycolor55}{rgb}{0.81734,0.27095,0.27039}%
\definecolor{mycolor56}{rgb}{0.82737,0.27952,0.26175}%
\definecolor{mycolor57}{rgb}{0.83230,0.28391,0.25738}%
\definecolor{mycolor58}{rgb}{0.83717,0.28839,0.25299}%
\definecolor{mycolor59}{rgb}{0.84671,0.29756,0.24411}%
\definecolor{mycolor60}{rgb}{0.85138,0.30226,0.23964}%
\definecolor{mycolor61}{rgb}{0.85599,0.30704,0.23513}%
\definecolor{mycolor62}{rgb}{0.86053,0.31189,0.23061}%
\definecolor{mycolor63}{rgb}{0.87374,0.32691,0.21689}%
\definecolor{mycolor64}{rgb}{0.87800,0.33206,0.21227}%
\definecolor{mycolor65}{rgb}{0.89034,0.34796,0.19829}%
\definecolor{mycolor66}{rgb}{0.89819,0.35891,0.18886}%
\definecolor{mycolor67}{rgb}{0.90200,0.36449,0.18412}%
\definecolor{mycolor68}{rgb}{0.90573,0.37014,0.17935}%
\definecolor{mycolor69}{rgb}{0.90939,0.37586,0.17456}%
\definecolor{mycolor70}{rgb}{0.91297,0.38164,0.16975}%
\definecolor{mycolor71}{rgb}{0.91646,0.38748,0.16492}%
\definecolor{mycolor72}{rgb}{0.91988,0.39339,0.16007}%
\definecolor{mycolor73}{rgb}{0.92322,0.39936,0.15519}%
\definecolor{mycolor74}{rgb}{0.92647,0.40539,0.15029}%
\definecolor{mycolor75}{rgb}{0.92964,0.41148,0.14537}%
\definecolor{mycolor76}{rgb}{0.93274,0.41763,0.14042}%
\definecolor{mycolor77}{rgb}{0.93575,0.42383,0.13544}%
\definecolor{mycolor78}{rgb}{0.93868,0.43009,0.13044}%
\definecolor{mycolor79}{rgb}{0.94152,0.43640,0.12541}%
\definecolor{mycolor80}{rgb}{0.94429,0.44277,0.12035}%
\definecolor{mycolor81}{rgb}{0.94956,0.45566,0.11016}%
\definecolor{mycolor82}{rgb}{0.95208,0.46218,0.10503}%
\definecolor{mycolor83}{rgb}{0.96129,0.48872,0.08429}%
\definecolor{mycolor84}{rgb}{0.96540,0.50225,0.07386}%
\definecolor{mycolor85}{rgb}{0.96732,0.50908,0.06866}%
\definecolor{mycolor86}{rgb}{0.97259,0.52980,0.05332}%
\definecolor{mycolor87}{rgb}{0.97568,0.54380,0.04362}%
\definecolor{mycolor88}{rgb}{0.98082,0.57221,0.02851}%
\definecolor{mycolor89}{rgb}{0.98189,0.57939,0.02625}%
\definecolor{mycolor90}{rgb}{0.98378,0.59385,0.02377}%
\definecolor{mycolor91}{rgb}{0.98532,0.60842,0.02420}%
\definecolor{mycolor92}{rgb}{0.98696,0.63048,0.03091}%
\definecolor{mycolor93}{rgb}{0.98793,0.66025,0.05175}%
\definecolor{mycolor94}{rgb}{0.98771,0.68281,0.07249}%
\definecolor{mycolor95}{rgb}{0.97750,0.78226,0.18592}%
\definecolor{mycolor96}{rgb}{0.96624,0.83619,0.26153}%
\definecolor{mycolor97}{rgb}{0.96252,0.85148,0.28555}%
\definecolor{mycolor98}{rgb}{0.98836,0.99836,0.64492}%
%
\addplot [color=mycolor1]
  table[row sep=crcr]{%
0	4395.3\\
600	56321\\
1200	1.1211e+05\\
1800	1.3055e+05\\
2400	1.4028e+05\\
3000	1.4596e+05\\
3600	1.4884e+05\\
4200	1.5259e+05\\
4800	1.4847e+05\\
5400	1.4898e+05\\
6000	1.4885e+05\\
6600	1.4508e+05\\
7200	1.4818e+05\\
8400	1.5045e+05\\
9000	1.476e+05\\
9600	1.5007e+05\\
10200	1.5066e+05\\
11400	1.5128e+05\\
12000	1.5011e+05\\
12600	1.5143e+05\\
13200	1.543e+05\\
13800	1.5296e+05\\
14400	1.5116e+05\\
15000	1.5034e+05\\
15600	1.4795e+05\\
16200	1.4392e+05\\
16800	1.4678e+05\\
17400	1.4921e+05\\
18000	1.4922e+05\\
18600	1.5171e+05\\
19200	1.5309e+05\\
19800	1.5091e+05\\
21000	1.5175e+05\\
21600	1.5316e+05\\
22200	1.5205e+05\\
22800	1.5166e+05\\
23400	1.499e+05\\
24000	1.5345e+05\\
24600	1.5461e+05\\
25200	1.5324e+05\\
26400	1.5297e+05\\
27000	1.5065e+05\\
28200	1.5056e+05\\
28800	1.5153e+05\\
30000	1.5093e+05\\
};
\addlegendentry{Classe k=283}

\addplot [color=mycolor2, forget plot]
  table[row sep=crcr]{%
0	4395.3\\
600	17749\\
1200	23876\\
1800	25596\\
2400	26479\\
3000	26285\\
3600	25896\\
4200	26146\\
4800	26002\\
5400	27066\\
6000	26518\\
6600	25754\\
7200	25704\\
7800	25344\\
8400	26127\\
9000	26141\\
9600	25038\\
10200	24217\\
10800	25992\\
11400	26105\\
12000	25938\\
12600	25854\\
13200	26217\\
13800	26422\\
14400	26061\\
15000	25590\\
16200	26099\\
16800	25352\\
17400	26076\\
18000	25763\\
18600	25101\\
19200	25690\\
19800	25201\\
20400	26501\\
21000	25626\\
21600	25083\\
22200	25425\\
22800	24338\\
23400	25449\\
24000	26496\\
24600	25833\\
25200	25669\\
25800	26501\\
26400	25302\\
27000	26041\\
27600	25929\\
28200	25417\\
28800	25993\\
29400	25659\\
30000	26139\\
};
\addplot [color=mycolor3, forget plot]
  table[row sep=crcr]{%
0	4395.3\\
600	23996\\
1200	35882\\
1800	39695\\
2400	39498\\
3000	40154\\
3600	40520\\
4200	40722\\
4800	42121\\
5400	39532\\
6000	38995\\
6600	39055\\
7200	38468\\
7800	38147\\
8400	38331\\
9000	38241\\
9600	39887\\
10800	37770\\
11400	38450\\
12000	38747\\
12600	38272\\
13200	39133\\
13800	38678\\
14400	39848\\
15000	39643\\
15600	39931\\
16200	39441\\
16800	38581\\
17400	40304\\
18000	40077\\
18600	39110\\
19800	39164\\
20400	38675\\
21000	38058\\
21600	39978\\
22200	39883\\
22800	38299\\
23400	38853\\
24600	39496\\
25200	39103\\
25800	39322\\
26400	39862\\
27000	39288\\
27600	40162\\
28800	38384\\
29400	38405\\
30000	39872\\
};
\addplot [color=mycolor4, forget plot]
  table[row sep=crcr]{%
0	4395.3\\
600	11262\\
1200	13667\\
1800	13787\\
2400	14251\\
3000	13939\\
3600	14516\\
4200	14228\\
4800	14601\\
5400	13835\\
6000	14063\\
6600	14157\\
7200	14802\\
7800	14565\\
8400	14874\\
9000	14320\\
9600	14552\\
10200	14582\\
11400	13803\\
12000	14543\\
13200	14198\\
13800	13661\\
14400	14534\\
15000	15006\\
15600	14209\\
16200	14151\\
17400	14317\\
18600	13944\\
19200	14044\\
19800	14532\\
20400	14563\\
21000	14381\\
21600	13723\\
22200	14077\\
22800	14260\\
23400	14655\\
24000	13972\\
24600	14570\\
25200	14065\\
25800	14103\\
26400	14431\\
27000	14301\\
27600	14434\\
28800	14024\\
29400	13952\\
30000	14159\\
};
\addplot [color=mycolor5, forget plot]
  table[row sep=crcr]{%
0	4395.3\\
600	25726\\
1200	47906\\
1800	57083\\
2400	60700\\
3000	64041\\
3600	64699\\
4200	62978\\
4800	62626\\
5400	65067\\
6000	63735\\
6600	64581\\
7200	64893\\
7800	61759\\
8400	62772\\
9000	64762\\
9600	62180\\
10200	64362\\
10800	63622\\
11400	63942\\
12000	63257\\
13200	63093\\
13800	63611\\
15000	61704\\
15600	62079\\
16200	63350\\
16800	63872\\
17400	62823\\
18000	62117\\
18600	62133\\
19200	64000\\
19800	62885\\
20400	63067\\
21000	61998\\
21600	60050\\
22200	61381\\
22800	63836\\
23400	65328\\
24000	65489\\
24600	62774\\
25200	62520\\
25800	61341\\
26400	61998\\
27000	60752\\
28800	62923\\
29400	65673\\
30000	65151\\
};
\addplot [color=mycolor6, forget plot]
  table[row sep=crcr]{%
0	4395.3\\
600	24724\\
1200	45466\\
1800	53948\\
2400	55572\\
3000	52182\\
3600	52639\\
4200	56445\\
4800	56554\\
5400	54247\\
6000	56042\\
6600	57371\\
7200	58424\\
8400	55986\\
9000	57445\\
9600	56332\\
10200	57730\\
10800	55997\\
11400	57349\\
12000	57026\\
12600	54956\\
13200	52731\\
13800	54163\\
14400	54371\\
15000	55448\\
15600	57720\\
16200	58793\\
16800	56115\\
17400	56647\\
18000	58633\\
18600	56447\\
19800	53628\\
20400	52843\\
21000	54676\\
21600	56884\\
22200	56156\\
22800	56957\\
23400	56876\\
24000	55470\\
24600	56997\\
25200	55578\\
25800	53983\\
26400	55171\\
27000	56903\\
28200	56329\\
29400	54502\\
30000	55628\\
};
\addplot [color=mycolor7, forget plot]
  table[row sep=crcr]{%
0	4395.3\\
600	16003\\
1200	25496\\
1800	27751\\
2400	26571\\
3000	26544\\
3600	27559\\
4200	28230\\
4800	27795\\
5400	27730\\
6600	28206\\
7200	27715\\
7800	28759\\
8400	28605\\
9000	27799\\
9600	27526\\
10200	27868\\
10800	27626\\
11400	28759\\
12000	28592\\
12600	28561\\
13200	29369\\
13800	27506\\
14400	26870\\
15000	28179\\
15600	27112\\
16200	28191\\
16800	27886\\
17400	27068\\
18000	27154\\
18600	26244\\
19200	27556\\
19800	28227\\
20400	27762\\
21000	28779\\
21600	27848\\
22200	29420\\
22800	29140\\
23400	29264\\
24000	27845\\
24600	28075\\
25200	28143\\
25800	28711\\
26400	27527\\
27000	27480\\
27600	26931\\
28200	26879\\
28800	27099\\
29400	28408\\
30000	28152\\
};
\addplot [color=mycolor8, forget plot]
  table[row sep=crcr]{%
0	4395.3\\
600	13270\\
1200	18655\\
1800	19756\\
2400	20106\\
3000	19965\\
4200	20469\\
4800	20382\\
5400	20569\\
6000	20413\\
6600	19969\\
7200	20437\\
7800	20659\\
8400	20811\\
9000	20522\\
9600	21289\\
10200	21554\\
10800	21385\\
11400	20166\\
12000	20735\\
12600	21132\\
13200	20749\\
13800	21054\\
14400	21025\\
15000	20240\\
16200	20198\\
16800	20974\\
17400	20760\\
18000	19929\\
18600	20404\\
19200	20708\\
19800	20252\\
20400	20437\\
21000	19386\\
21600	19153\\
22200	20224\\
22800	20282\\
23400	20096\\
24000	20052\\
24600	20790\\
25200	20687\\
25800	19884\\
26400	19727\\
27000	19904\\
27600	20306\\
28200	19014\\
28800	20417\\
29400	19927\\
30000	20263\\
};
\addplot [color=mycolor9, forget plot]
  table[row sep=crcr]{%
0	4395.3\\
600	10547\\
1200	13301\\
1800	13850\\
2400	13669\\
3000	14196\\
3600	13482\\
4200	13655\\
4800	13477\\
5400	13679\\
6000	14228\\
6600	14658\\
7200	14073\\
7800	13746\\
8400	13611\\
9000	13702\\
9600	13834\\
10200	13536\\
11400	14138\\
12000	14954\\
12600	14468\\
13200	13896\\
13800	14053\\
14400	13759\\
15000	14297\\
15600	14061\\
16200	13925\\
16800	13504\\
17400	14154\\
18000	14060\\
18600	14496\\
19200	14328\\
19800	14466\\
21000	14003\\
21600	14260\\
22200	13720\\
22800	13988\\
23400	13941\\
24000	14344\\
24600	14273\\
25200	14049\\
25800	14654\\
26400	14027\\
27000	13767\\
27600	14160\\
28200	14009\\
28800	14074\\
29400	13514\\
30000	13493\\
};
\addplot [color=mycolor10, forget plot]
  table[row sep=crcr]{%
0	4395.3\\
600	13055\\
1200	19247\\
1800	21389\\
2400	22521\\
3000	22053\\
4200	22213\\
4800	22374\\
5400	22281\\
6000	22643\\
6600	22740\\
7200	21843\\
7800	22230\\
8400	22290\\
9600	22685\\
10200	23077\\
10800	22523\\
11400	21401\\
12000	21355\\
12600	22119\\
13200	21433\\
13800	21641\\
14400	21482\\
15000	21724\\
15600	22118\\
16200	21977\\
16800	21229\\
17400	21736\\
18000	21111\\
18600	21900\\
19200	21586\\
19800	22595\\
20400	22589\\
21000	22813\\
21600	22556\\
22200	22644\\
22800	21483\\
23400	21766\\
24600	21982\\
25200	21785\\
25800	21279\\
26400	21688\\
27000	21336\\
27600	22341\\
28200	22247\\
28800	21419\\
29400	21685\\
30000	21256\\
};
\addplot [color=mycolor11, forget plot]
  table[row sep=crcr]{%
0	4395.3\\
600	7760\\
1200	9060.4\\
1800	8470.8\\
2400	8752.5\\
3000	8615.7\\
3600	9006\\
4200	8624.4\\
4800	8564.2\\
5400	8855.7\\
6000	8377.4\\
6600	8296.6\\
7200	8614.4\\
7800	8755.8\\
8400	8141.9\\
9000	8324.3\\
9600	8205.8\\
10200	8661.9\\
10800	8851.3\\
11400	8742.8\\
12000	8715.6\\
12600	8824.9\\
13800	9238.7\\
14400	8977.2\\
15000	8777.5\\
15600	8722.4\\
16200	8537.1\\
16800	9112.7\\
17400	8676.1\\
18000	8131\\
18600	8141.9\\
19200	8675.2\\
19800	8870.9\\
20400	9010.2\\
21000	9013\\
21600	8839\\
22200	8617.9\\
22800	8810.6\\
24000	8472.5\\
24600	8193.2\\
25200	8434.2\\
25800	8623.7\\
26400	8438.5\\
27000	8454.7\\
27600	8660.5\\
28200	8845\\
28800	8682.1\\
29400	8630.6\\
30000	8851.4\\
};
\addplot [color=mycolor12, forget plot]
  table[row sep=crcr]{%
0	4395.3\\
600	10556\\
1200	13573\\
1800	14424\\
2400	15255\\
3000	14852\\
3600	15489\\
4200	14783\\
5400	14771\\
6000	15089\\
6600	14524\\
7200	14176\\
7800	14153\\
8400	14457\\
9000	14646\\
9600	13841\\
10800	14848\\
11400	14675\\
12000	14996\\
12600	15080\\
13200	14200\\
13800	14267\\
14400	14796\\
15000	14630\\
15600	14075\\
16200	14611\\
16800	14381\\
17400	15079\\
18000	15142\\
18600	14695\\
19200	14951\\
19800	14703\\
20400	14935\\
21000	15649\\
21600	15035\\
22200	14918\\
22800	14564\\
23400	14584\\
24000	14680\\
24600	14523\\
25200	14468\\
25800	13681\\
26400	14744\\
27000	14987\\
27600	14425\\
28200	14185\\
28800	14558\\
29400	14085\\
30000	15147\\
};
\addplot [color=mycolor13, forget plot]
  table[row sep=crcr]{%
0	4395.3\\
600	9127\\
1200	10955\\
1800	10825\\
2400	10983\\
3000	11225\\
3600	11249\\
4200	11374\\
4800	11153\\
5400	11085\\
6000	10776\\
6600	11156\\
7200	10684\\
7800	10984\\
8400	10894\\
9000	10984\\
9600	11565\\
10200	11396\\
10800	11519\\
11400	11597\\
12000	11346\\
12600	11016\\
13200	10574\\
13800	11013\\
14400	10782\\
15600	11568\\
16200	11874\\
16800	10996\\
17400	11131\\
18000	11572\\
18600	11107\\
19200	11454\\
19800	11579\\
20400	11400\\
21000	10906\\
21600	10705\\
22200	11075\\
22800	10665\\
23400	11355\\
24000	11141\\
24600	11062\\
25200	10761\\
25800	10867\\
26400	10587\\
27000	11130\\
27600	11508\\
28200	11203\\
28800	10737\\
29400	10963\\
30000	11142\\
};
\addplot [color=mycolor14, forget plot]
  table[row sep=crcr]{%
0	4395.3\\
600	13722\\
1200	22264\\
1800	23828\\
2400	23920\\
3000	24723\\
3600	24223\\
4200	23269\\
4800	24402\\
6000	24109\\
7200	24333\\
7800	24876\\
8400	24657\\
9000	23012\\
10200	24174\\
10800	23194\\
11400	23256\\
12000	24473\\
12600	23820\\
13200	24424\\
13800	24394\\
14400	24201\\
15000	24998\\
15600	24639\\
16200	24586\\
16800	25362\\
17400	24051\\
18000	24836\\
18600	23771\\
19200	23780\\
19800	24619\\
20400	23833\\
21000	23962\\
21600	23424\\
22200	23037\\
22800	22988\\
23400	23853\\
24000	23373\\
24600	24137\\
25200	24656\\
25800	24044\\
26400	23773\\
27000	25171\\
27600	24210\\
28200	23850\\
28800	24548\\
29400	24363\\
30000	23559\\
};
\addplot [color=mycolor15, forget plot]
  table[row sep=crcr]{%
0	4395.3\\
600	11547\\
1200	15711\\
1800	16807\\
2400	16869\\
3000	17366\\
3600	17010\\
4200	16493\\
4800	16203\\
5400	16294\\
6000	17281\\
6600	17209\\
7200	17663\\
7800	17556\\
8400	17878\\
9000	17335\\
9600	17546\\
10200	17459\\
10800	17795\\
11400	17205\\
12000	16720\\
12600	16115\\
13200	16859\\
13800	16985\\
14400	16708\\
15600	16673\\
16200	16963\\
16800	17521\\
17400	17240\\
18000	16704\\
18600	17013\\
19200	17453\\
19800	16559\\
20400	17102\\
21000	16223\\
21600	16379\\
22800	17199\\
23400	16718\\
24000	16707\\
24600	16240\\
25800	17520\\
26400	16485\\
27000	16764\\
27600	16460\\
28200	16230\\
28800	16593\\
29400	16456\\
30000	16140\\
};
\addplot [color=mycolor16, forget plot]
  table[row sep=crcr]{%
0	4395.3\\
600	6338.4\\
1200	6977.8\\
1800	7167.5\\
3000	6704.4\\
3600	6927.6\\
4200	6653\\
4800	6907.6\\
5400	7347.7\\
6000	6821.9\\
6600	6797\\
7200	7063.8\\
7800	7007.5\\
8400	6878.9\\
9000	6908.3\\
10200	7036.2\\
10800	7356.8\\
11400	7129.1\\
12000	6883.2\\
12600	7063.1\\
13200	7490.7\\
13800	7191.4\\
14400	7065.9\\
15000	7094.9\\
15600	7506.1\\
16200	6999.9\\
16800	7066\\
17400	7202.3\\
18000	6648.2\\
18600	6886.5\\
19200	6898.4\\
19800	6827\\
20400	6943\\
21000	6829.5\\
21600	6979.9\\
22200	6883.7\\
22800	7027.5\\
23400	7143.3\\
24000	6741.1\\
24600	7246.3\\
25200	7136.8\\
25800	6944.7\\
26400	6965.3\\
27000	6948.5\\
27600	6894.7\\
28200	7287.2\\
28800	6887.9\\
29400	7143\\
30000	6836.2\\
};
\addplot [color=mycolor17, forget plot]
  table[row sep=crcr]{%
0	4395.3\\
600	11282\\
1200	15073\\
1800	15905\\
2400	15705\\
3000	15753\\
3600	15423\\
4200	15324\\
4800	15642\\
6000	15806\\
6600	15983\\
7200	15435\\
7800	16029\\
8400	15604\\
9000	16063\\
9600	15820\\
10200	16187\\
10800	16069\\
11400	15819\\
12000	16071\\
12600	15991\\
13200	15853\\
13800	15831\\
14400	16129\\
15000	15845\\
15600	16243\\
16200	16204\\
16800	16313\\
17400	16217\\
18000	16038\\
18600	15997\\
19200	16712\\
19800	16162\\
20400	16026\\
21000	16359\\
21600	16546\\
22200	15142\\
22800	14834\\
23400	15146\\
24000	15781\\
24600	15505\\
25200	16043\\
25800	15514\\
26400	15291\\
27000	15891\\
27600	15926\\
28200	15774\\
28800	16069\\
29400	16156\\
30000	15975\\
};
\addplot [color=mycolor18, forget plot]
  table[row sep=crcr]{%
0	4395.3\\
600	13942\\
1200	21121\\
1800	23260\\
2400	24480\\
3000	23382\\
3600	23609\\
4200	23630\\
4800	24825\\
5400	24609\\
6000	24133\\
6600	23335\\
7200	24705\\
7800	23581\\
9000	26158\\
9600	25362\\
10200	25493\\
10800	25312\\
11400	24573\\
12000	24703\\
12600	25165\\
13800	25723\\
14400	25108\\
15000	25187\\
15600	24632\\
16200	24718\\
16800	25390\\
17400	24826\\
18600	25144\\
19200	23980\\
19800	23906\\
20400	24902\\
21000	24584\\
21600	23125\\
22200	24647\\
22800	25259\\
23400	23921\\
24600	24780\\
25200	24245\\
25800	24695\\
26400	23797\\
27000	25015\\
27600	24344\\
28200	24192\\
28800	25342\\
29400	25416\\
30000	24599\\
};
\addplot [color=mycolor19, forget plot]
  table[row sep=crcr]{%
0	4395.3\\
600	10846\\
1200	14773\\
1800	15603\\
2400	14879\\
3000	14850\\
3600	15096\\
4200	14831\\
4800	14885\\
5400	14981\\
6600	14896\\
7200	14263\\
7800	14482\\
8400	15381\\
9000	14225\\
9600	14797\\
10200	15077\\
10800	15471\\
11400	14475\\
12000	14438\\
12600	14048\\
14400	14717\\
15000	14491\\
15600	14475\\
16200	14267\\
16800	14304\\
17400	13961\\
18000	15689\\
18600	15297\\
19200	14561\\
19800	14478\\
20400	14839\\
21000	15099\\
21600	15018\\
22200	14376\\
22800	14873\\
23400	14623\\
24000	14670\\
24600	14295\\
25200	14954\\
25800	16177\\
26400	15053\\
27000	14655\\
27600	14382\\
28200	14554\\
28800	13900\\
29400	14350\\
30000	13889\\
};
\addplot [color=mycolor20, forget plot]
  table[row sep=crcr]{%
0	4395.3\\
600	11608\\
1200	17352\\
1800	19164\\
2400	20298\\
3000	20039\\
3600	19033\\
4200	18497\\
4800	19915\\
5400	20078\\
6000	19625\\
6600	20186\\
7200	18893\\
7800	18023\\
8400	18127\\
9000	18624\\
9600	18717\\
10200	17828\\
10800	18661\\
11400	19249\\
12000	19007\\
12600	19411\\
13200	18873\\
13800	19290\\
14400	19475\\
15000	20343\\
15600	19824\\
16200	18715\\
16800	18583\\
17400	19032\\
18000	18401\\
18600	19090\\
19200	19654\\
19800	18891\\
20400	18962\\
21000	19493\\
21600	18876\\
22200	18563\\
22800	18961\\
23400	19425\\
24000	19236\\
24600	18980\\
25200	19118\\
25800	19529\\
26400	19323\\
27000	18862\\
27600	18591\\
28200	19447\\
28800	20587\\
29400	19783\\
30000	19133\\
};
\addplot [color=mycolor21, forget plot]
  table[row sep=crcr]{%
0	4395.3\\
600	7518.4\\
1200	9420.2\\
1800	9569.3\\
2400	10047\\
3000	9874.8\\
3600	9967.7\\
4200	9938.5\\
4800	9877.9\\
5400	9919\\
6000	9641.7\\
6600	9677.1\\
7200	9848\\
7800	9650.8\\
8400	9603.6\\
9000	9597.1\\
9600	9171.2\\
10200	9471\\
10800	9620.2\\
11400	9398.6\\
12000	9510.6\\
12600	9510.9\\
13200	9243.6\\
13800	9483\\
14400	9480.7\\
15000	9109.7\\
15600	9514.3\\
16200	9287.8\\
16800	9960.4\\
17400	9772.3\\
18000	9745.9\\
18600	9452.8\\
19200	9318.8\\
19800	9749.2\\
20400	9989.3\\
21000	10025\\
21600	9449.4\\
22200	10019\\
23400	9545.1\\
24000	9705.5\\
24600	9531.3\\
25200	8981.1\\
25800	9576.9\\
26400	9451.4\\
27000	9654.1\\
27600	9786.7\\
28200	9877.8\\
28800	10030\\
29400	9869.9\\
30000	9808.8\\
};
\addplot [color=mycolor22, forget plot]
  table[row sep=crcr]{%
0	4395.3\\
600	13713\\
1200	21611\\
1800	24666\\
2400	25366\\
3000	25750\\
3600	26339\\
4200	25508\\
4800	25738\\
5400	26366\\
6000	26245\\
6600	25697\\
7200	24963\\
7800	24657\\
8400	25241\\
9000	25359\\
9600	25790\\
10200	24621\\
10800	25230\\
11400	24459\\
12000	23618\\
12600	24618\\
13200	25336\\
13800	24849\\
14400	25067\\
15000	24974\\
15600	25234\\
16200	24933\\
16800	25082\\
18000	23722\\
18600	24993\\
19200	24457\\
19800	24691\\
20400	24789\\
21000	25227\\
21600	25561\\
22200	24089\\
22800	24303\\
23400	24168\\
24000	25021\\
24600	26762\\
25200	25923\\
25800	26428\\
26400	25866\\
27000	25052\\
27600	24783\\
28200	24943\\
28800	25926\\
29400	25356\\
30000	25444\\
};
\addplot [color=mycolor23, forget plot]
  table[row sep=crcr]{%
0	4395.3\\
600	8808.4\\
1200	11550\\
1800	11991\\
2400	11830\\
3600	12257\\
4200	12204\\
4800	11930\\
5400	12444\\
6000	12543\\
6600	12150\\
7200	12111\\
7800	12231\\
8400	12522\\
9600	12139\\
10200	12203\\
10800	12311\\
11400	12183\\
12000	12288\\
12600	12210\\
13200	12533\\
15000	12367\\
15600	12244\\
16200	12436\\
17400	12271\\
18000	12064\\
18600	12562\\
19200	11968\\
19800	12392\\
20400	12077\\
21000	12236\\
21600	12462\\
22200	12178\\
22800	12257\\
23400	12830\\
24600	12224\\
25200	12285\\
25800	12393\\
26400	12424\\
27000	12382\\
27600	12439\\
28200	12301\\
28800	12222\\
29400	12459\\
30000	12221\\
};
\addplot [color=mycolor24, forget plot]
  table[row sep=crcr]{%
0	4395.3\\
600	4862.9\\
1200	4958.6\\
1800	4697.2\\
2400	4748.3\\
3000	4382.8\\
3600	4860.7\\
4200	4900.9\\
4800	4725.3\\
5400	4633.8\\
6000	4917.3\\
6600	4823\\
7200	4673.1\\
7800	4764.9\\
8400	4811.8\\
9000	4980.8\\
9600	4907.4\\
10200	4978.1\\
10800	4992.4\\
11400	4756.6\\
12000	4776.7\\
12600	4815.9\\
13800	4820.6\\
14400	4891.8\\
15000	4789.4\\
15600	5075.7\\
16200	4780.3\\
16800	4817.4\\
17400	4777.5\\
18600	4633.3\\
19200	4669.5\\
19800	4599.4\\
20400	4925.6\\
21000	4952.3\\
21600	4880.1\\
22200	4851.6\\
22800	5003.3\\
23400	4667.8\\
24000	4982.2\\
24600	4659.6\\
25200	4584\\
25800	4651.9\\
26400	4491.9\\
27000	4688.8\\
27600	4605.5\\
28800	4810.2\\
29400	4833.1\\
30000	4717.1\\
};
\addplot [color=mycolor25, forget plot]
  table[row sep=crcr]{%
0	4395.3\\
600	9163\\
1200	12134\\
1800	12546\\
2400	12517\\
3000	12401\\
3600	12687\\
4200	11733\\
4800	12033\\
5400	12589\\
6000	12538\\
6600	12281\\
7200	11873\\
7800	12317\\
8400	11986\\
9000	12051\\
9600	11512\\
10200	12023\\
10800	12418\\
11400	12760\\
12000	12285\\
12600	12496\\
13200	12756\\
13800	12855\\
14400	13345\\
15000	12953\\
15600	11983\\
16200	12245\\
16800	12140\\
17400	12504\\
18000	12192\\
18600	11672\\
19800	12732\\
20400	12143\\
21000	12035\\
21600	12814\\
22200	12576\\
22800	12417\\
23400	11895\\
24000	11286\\
24600	12147\\
25200	12466\\
25800	12541\\
26400	12015\\
27000	11607\\
27600	12087\\
28200	12673\\
28800	12882\\
29400	12396\\
30000	12217\\
};
\addplot [color=mycolor26, forget plot]
  table[row sep=crcr]{%
0	4395.3\\
600	6876.9\\
1200	8070.5\\
1800	7562.4\\
2400	7934.7\\
3000	8413.6\\
3600	8251.5\\
4800	8474.2\\
5400	8331.5\\
6000	8129.5\\
6600	8640.5\\
7200	8336.6\\
7800	8759.1\\
8400	8498.1\\
9000	8391\\
9600	8245.5\\
10200	8781.3\\
10800	8324.7\\
11400	8128.5\\
12000	8592.3\\
12600	8489.1\\
13200	8627.3\\
13800	8270.8\\
14400	8576.4\\
15000	8403.6\\
15600	8318.4\\
16200	8449.8\\
16800	8047.7\\
17400	7860.1\\
18000	8196.5\\
18600	8187.4\\
19200	7939.3\\
19800	8262.8\\
20400	8230\\
21600	7981.3\\
22200	7977.8\\
22800	8097.4\\
23400	8159.5\\
24000	8089\\
24600	8312.2\\
25200	8461.8\\
26400	8714.8\\
27000	8518.7\\
27600	8558\\
28200	8872.8\\
28800	8420.9\\
29400	8435.7\\
30000	8170.7\\
};
\addplot [color=mycolor27, forget plot]
  table[row sep=crcr]{%
0	4395.3\\
600	7916.7\\
1200	9633.9\\
1800	9990\\
2400	10190\\
3000	9718.4\\
3600	9737.3\\
4200	9488.9\\
4800	9700.7\\
5400	9841.4\\
6000	9703.9\\
6600	9298.4\\
7200	9379.1\\
7800	9965.1\\
8400	10093\\
9000	9876.4\\
9600	10073\\
10200	9381.2\\
10800	9783.8\\
11400	9864\\
12000	9562.2\\
12600	9707.2\\
13800	9788.1\\
14400	9375.4\\
15000	10058\\
15600	10130\\
16200	9745\\
16800	10142\\
17400	9187.1\\
18000	9329\\
18600	9228.5\\
19200	9056.6\\
19800	9787.3\\
20400	10093\\
21000	9337.4\\
21600	9641.6\\
22200	9613.1\\
22800	9744.5\\
23400	10154\\
24000	9445.8\\
24600	9457.3\\
25200	9313.3\\
25800	9029.8\\
26400	9802.8\\
27000	9739\\
27600	9626.1\\
28200	8877.6\\
29400	9449.9\\
30000	10257\\
};
\addplot [color=mycolor28, forget plot]
  table[row sep=crcr]{%
0	4395.3\\
600	5459.4\\
1200	5721.5\\
1800	5837.1\\
2400	5201.4\\
3000	5432.8\\
3600	5592.3\\
4200	5500.8\\
4800	5242.6\\
5400	5448.7\\
6000	5615.9\\
6600	5482.1\\
7200	5267.9\\
7800	5665.4\\
8400	5436.7\\
9000	5530.8\\
9600	5466.2\\
10200	5330.5\\
10800	5326.2\\
11400	5642\\
12000	5486.3\\
12600	5609.5\\
13200	5584.2\\
13800	5316.6\\
14400	5416.8\\
15000	5459.2\\
15600	5607.7\\
16200	5496.1\\
17400	5682.6\\
18000	5471.5\\
18600	5611.8\\
19200	5786.4\\
19800	5512.5\\
20400	5659.2\\
21000	5561.8\\
21600	5425.6\\
22200	5456.6\\
22800	5566.6\\
23400	5200.9\\
24000	5480.9\\
24600	5444.8\\
25200	5521\\
25800	5394.6\\
26400	5618.4\\
27000	5597.8\\
27600	5517.1\\
28200	5689.9\\
28800	5574.7\\
29400	5330.9\\
30000	5290.7\\
};
\addplot [color=mycolor29, forget plot]
  table[row sep=crcr]{%
0	4395.3\\
600	8658.9\\
1200	10969\\
1800	11344\\
2400	11689\\
3000	11939\\
3600	11880\\
4200	11443\\
4800	11452\\
5400	11653\\
6000	11529\\
6600	11172\\
7200	10949\\
7800	12079\\
8400	11432\\
9000	11670\\
9600	11764\\
10200	11486\\
10800	11161\\
11400	11881\\
12000	11658\\
12600	11622\\
13200	11678\\
13800	11034\\
14400	11772\\
15000	11514\\
15600	11368\\
16200	11638\\
16800	11257\\
17400	11219\\
18000	11422\\
18600	11834\\
19200	12503\\
19800	12200\\
20400	11415\\
21000	11308\\
21600	12059\\
22200	11839\\
22800	11513\\
23400	11299\\
24000	12162\\
24600	12001\\
25200	12124\\
25800	12388\\
26400	11730\\
27000	11366\\
27600	10951\\
28200	11538\\
28800	11698\\
29400	11212\\
30000	11467\\
};
\addplot [color=mycolor30, forget plot]
  table[row sep=crcr]{%
0	4395.3\\
600	6623.6\\
1200	7943\\
1800	8167.5\\
2400	8040.2\\
3600	7947.5\\
4800	7872.3\\
5400	8084.9\\
6000	7944.4\\
6600	8073.6\\
7200	8063\\
7800	7779.7\\
8400	7839.5\\
9000	7932.7\\
9600	8196.3\\
10200	8027.5\\
10800	8056.4\\
11400	7842.5\\
12000	7862.8\\
12600	7685.1\\
13200	7808.4\\
13800	7900.5\\
14400	8130.6\\
15000	7764.6\\
15600	7961.5\\
16200	7847.4\\
16800	8015\\
17400	8017.2\\
18000	7857.6\\
18600	8265.6\\
19200	7886.2\\
19800	7868\\
20400	7816.6\\
21000	7986.9\\
21600	7903.5\\
22200	7503.7\\
22800	7853.1\\
23400	7956.9\\
24000	8167.6\\
24600	7868.3\\
25200	8088.6\\
25800	7861.1\\
26400	7845.7\\
27000	7991.6\\
27600	8063.9\\
28200	7847.2\\
28800	7546.6\\
29400	7720.6\\
30000	8362.7\\
};
\addplot [color=mycolor31, forget plot]
  table[row sep=crcr]{%
0	4395.3\\
600	5964.5\\
1200	6602.8\\
1800	6331.9\\
2400	6348.1\\
3000	6471.3\\
3600	6621.5\\
4200	6735.9\\
4800	6608.1\\
5400	6363.8\\
6000	6382.6\\
7200	6243.5\\
7800	6510.6\\
9000	6395.3\\
9600	6571\\
10200	6596.4\\
10800	6587.3\\
11400	6764.6\\
12000	6506.9\\
12600	6391.6\\
13200	6192.8\\
13800	6210.4\\
14400	6386\\
15000	6523.3\\
15600	6237.4\\
16200	6353.8\\
16800	6608.8\\
17400	6331.9\\
18000	6303.5\\
18600	6074.2\\
19200	6200.4\\
19800	6415.4\\
20400	6678.8\\
21000	6470.2\\
21600	6232.9\\
22200	6260.7\\
23400	6416.4\\
24000	6710.4\\
24600	6511.1\\
25200	6530.4\\
25800	6407.2\\
26400	6030.2\\
27000	6597\\
27600	6727.1\\
28200	6656.6\\
28800	6437.5\\
29400	6119.8\\
30000	6506.4\\
};
\addplot [color=mycolor32, forget plot]
  table[row sep=crcr]{%
0	4395.3\\
600	8220\\
1200	10545\\
1800	10531\\
2400	10280\\
3600	10192\\
4200	10478\\
4800	10443\\
5400	10118\\
6000	10577\\
7200	10635\\
7800	10346\\
8400	10283\\
9000	9984.8\\
9600	10634\\
10200	10132\\
10800	10500\\
11400	10419\\
12000	10464\\
12600	10606\\
13200	10518\\
13800	10846\\
14400	10790\\
15000	10375\\
15600	10899\\
16200	10882\\
16800	10199\\
17400	9884.7\\
18600	10591\\
19200	10110\\
19800	10554\\
21000	10683\\
21600	9991.5\\
22200	9980.5\\
22800	10150\\
23400	10898\\
24000	10579\\
24600	10132\\
25200	10623\\
25800	10465\\
26400	10270\\
27000	10278\\
27600	10350\\
28200	10771\\
28800	10359\\
29400	10596\\
30000	10299\\
};
\addplot [color=mycolor33, forget plot]
  table[row sep=crcr]{%
0	4395.3\\
600	6944\\
1200	7969.6\\
1800	8211.9\\
2400	8371.2\\
3000	8124.6\\
3600	8131\\
4200	7906.9\\
4800	8238.8\\
5400	7911.9\\
6000	8059.9\\
6600	8379\\
7200	8328.7\\
7800	8704.8\\
8400	8177.4\\
9000	8167.1\\
9600	8304\\
10200	8256.1\\
10800	8247.7\\
11400	8340.2\\
12000	7923\\
12600	8379.6\\
13200	8189.2\\
13800	8280\\
14400	8293.7\\
15000	7935.8\\
15600	8228.8\\
16200	8304.8\\
16800	8106.7\\
17400	8528.5\\
18000	8442.2\\
18600	8413.3\\
19200	8196.6\\
19800	8242.4\\
20400	8060.4\\
21000	8496.1\\
21600	8241.1\\
22200	8279.6\\
22800	8194.3\\
23400	8195.9\\
24000	8384.2\\
24600	7900.4\\
25200	8023.6\\
25800	8099\\
26400	8043.5\\
27000	8030.9\\
27600	7982\\
28200	8839.9\\
28800	8178.2\\
29400	7913.5\\
30000	8017.6\\
};
\addplot [color=mycolor34, forget plot]
  table[row sep=crcr]{%
0	4395.3\\
1200	3957.2\\
1800	4005.7\\
2400	4125.9\\
3000	3819.1\\
3600	4032.3\\
4200	4174.3\\
4800	4056.9\\
5400	4097.4\\
6000	4028.4\\
6600	3853.5\\
7200	3950.3\\
7800	4037.7\\
8400	3991.3\\
9000	4172.8\\
9600	4127.2\\
10200	4006.4\\
10800	4020.9\\
11400	3972.5\\
12000	4057.8\\
12600	3970.2\\
13200	3922.5\\
13800	4119.8\\
14400	3909.7\\
15000	3960.3\\
15600	4106.5\\
16800	4091.5\\
17400	4140.7\\
18000	4141.1\\
18600	4313.3\\
19200	3999.2\\
19800	3933.4\\
20400	3798.6\\
21000	4052.7\\
21600	4059.1\\
22200	4120\\
22800	4101.4\\
23400	4151.9\\
24000	4040.6\\
24600	4058\\
25200	4139.3\\
25800	4178.3\\
26400	4088\\
27000	4143.6\\
27600	3978.4\\
28200	4152.6\\
28800	4096.7\\
29400	4005.1\\
30000	3951.4\\
};
\addplot [color=mycolor35]
  table[row sep=crcr]{%
0	4395.3\\
600	7432.9\\
1200	9483.4\\
3000	10357\\
3600	10353\\
4800	10241\\
5400	10088\\
6000	10160\\
7200	9994.6\\
7800	10394\\
9000	9960.5\\
9600	10157\\
10200	10109\\
10800	10233\\
11400	10083\\
12000	10208\\
12600	10013\\
13200	10121\\
13800	10421\\
14400	10273\\
15000	10438\\
15600	10067\\
17400	10220\\
18000	10191\\
18600	10292\\
19200	10483\\
19800	10177\\
20400	10210\\
21000	10144\\
21600	10376\\
22200	10158\\
22800	10265\\
23400	10084\\
24000	10018\\
24600	10189\\
25200	9933.8\\
25800	10098\\
26400	10366\\
27000	10143\\
27600	9825.7\\
28800	10031\\
29400	10304\\
30000	10110\\
};
\addlegendentry{Classe k=83}

\addplot [color=mycolor36, forget plot]
  table[row sep=crcr]{%
0	4395.3\\
600	4675.6\\
1200	4381.3\\
1800	4245.9\\
2400	4094.2\\
3000	4154.2\\
3600	4472.1\\
4200	4204.9\\
4800	4130.3\\
5400	4446.6\\
6000	4501.1\\
6600	4504.7\\
7200	4287.2\\
7800	4274.6\\
8400	4130.3\\
9000	4295.6\\
9600	4125.6\\
10200	4371.4\\
10800	4336\\
11400	4116.5\\
12000	4100.3\\
12600	4266\\
13200	4311.6\\
13800	4380.6\\
14400	4465.1\\
15000	4209.4\\
15600	4362.4\\
16800	4553.2\\
17400	4157.2\\
18000	4180\\
18600	4169.3\\
19200	4360.4\\
19800	4143.6\\
20400	4153.5\\
21000	4123.7\\
21600	4465.3\\
22200	4266.2\\
22800	4256.7\\
23400	4374\\
24000	4210.9\\
24600	4339.1\\
25200	4229.2\\
26400	4225.2\\
27000	4099.2\\
27600	4399\\
28200	4340.6\\
28800	4437.1\\
29400	4309.4\\
30000	4356.7\\
};
\addplot [color=mycolor37, forget plot]
  table[row sep=crcr]{%
0	4395.3\\
600	3915.3\\
1200	3804.8\\
1800	3662.7\\
2400	3630.3\\
4800	3705.5\\
5400	3594.7\\
6000	3643.2\\
6600	3706.6\\
7200	3617.1\\
7800	3511.6\\
8400	3542.2\\
9000	3778.2\\
9600	3590.8\\
10200	3475.3\\
10800	3576.8\\
11400	3604.7\\
12000	3609.7\\
12600	3578.4\\
13200	3615.2\\
13800	3575.7\\
15600	3712.4\\
16200	3574.3\\
16800	3548.8\\
17400	3660.6\\
18000	3621.7\\
18600	3628.3\\
19200	3858.9\\
19800	3585.3\\
20400	3598.3\\
21000	3722.2\\
21600	3631.5\\
22200	3616.2\\
22800	3574.2\\
23400	3594.9\\
24000	3678.3\\
24600	3559.7\\
25200	3584.2\\
25800	3533.2\\
26400	3656.1\\
27000	3676.3\\
27600	3645.7\\
28800	3667.8\\
29400	3635\\
30000	3669.1\\
};
\addplot [color=mycolor38, forget plot]
  table[row sep=crcr]{%
0	4395.3\\
600	3571.6\\
1200	3146.6\\
1800	3017.2\\
2400	2903.2\\
3000	3077\\
3600	2927.1\\
4200	2955.8\\
4800	2861.6\\
5400	3095.9\\
6000	2914.7\\
6600	2881.5\\
7200	3085.5\\
7800	3086.9\\
8400	2892.1\\
9000	3114\\
10200	2994.7\\
10800	2994.6\\
11400	2935.5\\
12000	2869\\
12600	3013.1\\
13200	2928.8\\
13800	2989\\
14400	2978.3\\
15000	3109.7\\
15600	3133\\
16200	3005.4\\
16800	3020\\
17400	3085.8\\
18000	3087.2\\
18600	3058.3\\
19200	2795.9\\
19800	2926.2\\
20400	2906.6\\
21000	2897.4\\
21600	3022.7\\
22200	2774\\
22800	3057.8\\
24000	2979.3\\
24600	2963.2\\
25200	3029.4\\
25800	3142.2\\
26400	3158\\
27000	2970\\
27600	3026.5\\
28200	2856\\
28800	3070.6\\
29400	2985.4\\
30000	3062.3\\
};
\addplot [color=mycolor39, forget plot]
  table[row sep=crcr]{%
0	4395.3\\
600	8779.2\\
1200	10999\\
1800	11580\\
2400	11296\\
3000	10726\\
3600	11021\\
4200	11469\\
4800	11756\\
5400	11332\\
6000	11717\\
6600	11767\\
7800	11158\\
8400	11170\\
9000	11880\\
9600	11582\\
10200	11435\\
10800	11676\\
11400	11571\\
12600	11574\\
13200	12072\\
13800	11854\\
14400	11502\\
15000	11392\\
15600	11442\\
16200	11257\\
16800	12052\\
17400	12266\\
18000	11614\\
18600	11808\\
19200	11757\\
19800	11929\\
20400	11308\\
21000	11344\\
21600	11440\\
22200	12303\\
22800	12051\\
23400	12624\\
24000	11229\\
25200	11896\\
25800	11623\\
26400	11240\\
27000	11386\\
27600	11752\\
28200	12571\\
28800	11960\\
29400	12360\\
30000	12250\\
};
\addplot [color=mycolor40, forget plot]
  table[row sep=crcr]{%
0	4395.3\\
600	5529.4\\
1200	5892.3\\
1800	5711.1\\
2400	5857.6\\
3000	5979.4\\
3600	5856\\
4200	5685.5\\
4800	5854.7\\
5400	5639.1\\
6000	5488.9\\
6600	5595\\
7200	5594.2\\
7800	5823.3\\
8400	5876.8\\
9000	5897.1\\
10200	5692\\
11400	5803.6\\
12000	5705.3\\
12600	5590\\
13200	5701.7\\
13800	6007.5\\
14400	5869.4\\
15000	5670.2\\
15600	5636.2\\
16200	5682.1\\
16800	5679.7\\
17400	5642.1\\
18000	5771.4\\
18600	5659.1\\
19200	5756.7\\
19800	6051.2\\
20400	5779.6\\
21000	5662.4\\
21600	5627.7\\
22200	5677.4\\
22800	5774.8\\
23400	5855.9\\
24000	5789.2\\
24600	5790\\
25200	5677.2\\
25800	5702.6\\
26400	5516\\
27000	5702.9\\
27600	5668.1\\
28200	5775.2\\
28800	5666\\
29400	5891.6\\
30000	5776.4\\
};
\addplot [color=mycolor41, forget plot]
  table[row sep=crcr]{%
0	4395.3\\
600	5067.8\\
1200	5385.2\\
1800	5377\\
2400	5601.7\\
3000	5491.8\\
3600	5506.8\\
4200	5331.1\\
4800	5558.6\\
5400	5490.2\\
6000	5491.3\\
6600	5377.8\\
7200	5247.6\\
8400	5203.7\\
9000	5182.2\\
10200	5308\\
10800	5194.6\\
12000	5309.9\\
12600	5303.6\\
13800	5401.8\\
14400	5371.3\\
15000	5466.2\\
15600	5204.2\\
16200	5348.2\\
16800	5570.7\\
17400	5597.5\\
18600	5714.1\\
19200	5208\\
20400	5450.9\\
21000	5489.3\\
21600	5134.5\\
22200	5335.2\\
22800	5436.7\\
23400	5341.4\\
24000	5292.6\\
24600	5148.4\\
25200	5645.2\\
25800	5392.6\\
26400	5664\\
27000	5245\\
27600	5111.8\\
28200	5487.5\\
28800	5299.3\\
29400	5343.8\\
30000	5319.1\\
};
\addplot [color=mycolor42, forget plot]
  table[row sep=crcr]{%
0	4395.3\\
600	10834\\
1200	17316\\
1800	19418\\
2400	19882\\
3000	20555\\
3600	20198\\
4200	20356\\
4800	20387\\
5400	20065\\
6000	20660\\
6600	21372\\
7200	20790\\
7800	21184\\
8400	20730\\
9000	20893\\
9600	20902\\
10200	20553\\
10800	20886\\
11400	21647\\
12000	22173\\
12600	21211\\
13200	20219\\
13800	19531\\
14400	19706\\
15000	20073\\
15600	21138\\
16200	21828\\
16800	21344\\
17400	21713\\
18000	21624\\
18600	20564\\
19200	20829\\
20400	21152\\
21000	21251\\
21600	21158\\
22200	21319\\
22800	21898\\
23400	20537\\
24000	20661\\
24600	20600\\
25200	21713\\
25800	21140\\
26400	20780\\
27000	21804\\
28200	20838\\
28800	21746\\
29400	21459\\
30000	20645\\
};
\addplot [color=mycolor43, forget plot]
  table[row sep=crcr]{%
0	4395.3\\
600	5012.1\\
1200	5118.2\\
3000	5107.3\\
3600	5011.5\\
4200	4867.6\\
5400	4873.6\\
6000	5060.3\\
6600	4992.2\\
7200	4887.3\\
7800	4890.1\\
8400	5030.4\\
9000	5082.5\\
9600	5009.2\\
10200	4807.7\\
10800	4907.7\\
11400	4982.2\\
12000	4925.8\\
12600	4994\\
13200	4851.7\\
13800	4941.5\\
14400	5108.7\\
15000	4912.1\\
15600	4834.3\\
16200	4829\\
16800	4943.2\\
17400	4866.3\\
18000	4881.4\\
18600	5043.3\\
19200	5011.4\\
19800	4955.1\\
20400	5055.9\\
21600	5119.5\\
22800	4940.3\\
23400	4919.9\\
24000	4831.4\\
24600	4983.6\\
25200	4945.4\\
25800	5045\\
26400	4969.4\\
27000	4812.8\\
27600	5009.1\\
28200	4955.1\\
28800	5063.1\\
29400	4902.5\\
30000	5095\\
};
\addplot [color=mycolor44, forget plot]
  table[row sep=crcr]{%
0	4395.3\\
1200	3534.7\\
1800	3384.2\\
2400	3279.7\\
3000	3334.6\\
3600	3160.2\\
4200	3339.3\\
4800	3512.4\\
5400	3186.8\\
6000	3271.8\\
6600	3271.1\\
7200	3280.6\\
7800	3375.9\\
8400	3163.5\\
9000	3180.5\\
9600	3283.1\\
10200	3065.5\\
10800	3253.8\\
11400	3416.9\\
12000	3235\\
12600	3389.7\\
13200	3189\\
13800	3299\\
14400	3258.2\\
15000	3151.7\\
15600	3306.2\\
16200	3416.3\\
16800	3511.1\\
17400	3189.5\\
18000	3472.5\\
18600	3262.5\\
19200	3585\\
19800	3418.3\\
20400	3246.1\\
21000	3335\\
21600	3308.9\\
22800	3272.1\\
23400	3303.2\\
24000	3392.8\\
24600	3203.6\\
25200	3209.5\\
26400	3253.3\\
27000	3313.6\\
27600	3327.2\\
28200	3392.9\\
28800	3356.5\\
29400	3548.7\\
30000	3638.9\\
};
\addplot [color=mycolor45, forget plot]
  table[row sep=crcr]{%
0	4395.3\\
600	6160.6\\
1200	7166.5\\
1800	6978.1\\
2400	6868\\
3000	7198.3\\
3600	6918\\
4200	6745.1\\
4800	6967\\
5400	6880.7\\
6000	6840.2\\
6600	7002.2\\
7200	6982.6\\
7800	6746.4\\
8400	6799\\
9000	7146.6\\
9600	7287.3\\
10200	7253.8\\
10800	6851.6\\
11400	7066.1\\
12000	7043\\
12600	6819.7\\
13200	6962.5\\
14400	7138.7\\
15000	7175.7\\
15600	6802.9\\
16200	6954.2\\
16800	6803.5\\
17400	6755.7\\
18000	7129.8\\
18600	6843.5\\
19200	6637.1\\
20400	7031.1\\
21000	6962\\
21600	7245.3\\
22200	7221.6\\
22800	6990.2\\
23400	6865.4\\
25200	6979.4\\
25800	7053.5\\
26400	7035.8\\
27000	6933.6\\
27600	6894.3\\
28200	7068.4\\
28800	6984.7\\
29400	6766.5\\
30000	6843\\
};
\addplot [color=mycolor46, forget plot]
  table[row sep=crcr]{%
0	4395.3\\
600	4365\\
1200	4218.2\\
1800	4299.7\\
2400	4191.8\\
3000	4127.4\\
3600	4005\\
4200	4026.6\\
4800	3910.6\\
5400	4031.7\\
6000	4137\\
6600	4063.4\\
7200	4326.1\\
7800	4253.2\\
8400	4244.9\\
9000	4065.2\\
9600	4114.3\\
10200	4066.6\\
11400	4129.9\\
12000	4046.8\\
12600	4132.2\\
13200	4037.9\\
14400	3978.3\\
15000	3880.2\\
15600	3937.2\\
16200	4141.4\\
16800	4197.2\\
17400	4014.6\\
18000	4222.4\\
18600	3995.4\\
19200	4033.7\\
19800	4051.4\\
20400	4204\\
21000	4161.9\\
21600	4184.2\\
22200	4127.1\\
22800	4164.4\\
23400	4140.2\\
24000	4189.5\\
24600	4194.2\\
25800	4070.4\\
26400	4181.9\\
27000	4208.8\\
27600	4111.6\\
28200	4092.9\\
28800	4099.6\\
29400	4061.2\\
30000	3998.9\\
};
\addplot [color=mycolor47, forget plot]
  table[row sep=crcr]{%
0	4395.3\\
600	4948.8\\
1200	5177\\
1800	5325\\
2400	5175.8\\
3000	5398.6\\
3600	5269\\
4200	5293.6\\
4800	5297.7\\
5400	5093.2\\
6000	5090.3\\
6600	5298.3\\
7200	5213.1\\
7800	5148.1\\
8400	5217.2\\
9000	5151.7\\
9600	5354.8\\
10200	5103.9\\
10800	5193.5\\
11400	5391.7\\
12000	5388.2\\
12600	5340.3\\
13200	5386.8\\
13800	5368.5\\
14400	5047\\
15000	5373\\
15600	5372.7\\
16200	5297.6\\
16800	5353.5\\
17400	5371.6\\
18000	5330.4\\
18600	5268.8\\
19200	5173.6\\
19800	5020.2\\
20400	5182.6\\
21000	5174.6\\
21600	5596.5\\
22200	5489.7\\
22800	5194.1\\
23400	5319.8\\
24000	5281.7\\
24600	5227.4\\
25200	5121.7\\
25800	5333.9\\
26400	5411.4\\
27000	5230.2\\
27600	5112.3\\
28200	5100.9\\
28800	5343.3\\
30000	5413.5\\
};
\addplot [color=mycolor48, forget plot]
  table[row sep=crcr]{%
0	4395.3\\
600	3502.3\\
1200	3211.2\\
1800	3028.3\\
3000	2865.7\\
3600	2932.7\\
4200	2938.4\\
4800	2893.6\\
5400	2905.3\\
6000	2938.4\\
7200	2958.4\\
7800	2874.8\\
8400	2832.2\\
9000	2869.4\\
9600	2814.8\\
10200	2872.2\\
10800	2868.3\\
11400	2910.2\\
12600	2865\\
13200	2896.2\\
14400	2875.5\\
15600	2868.9\\
16200	2901.2\\
16800	2987.5\\
17400	2982.3\\
18000	2881.6\\
18600	2902.7\\
19200	2990.8\\
19800	2967.2\\
20400	2905.1\\
21000	2929.1\\
21600	2925\\
22200	2902.7\\
23400	2972.7\\
24000	2951.6\\
24600	2881.6\\
26400	2850.6\\
27000	2924.7\\
27600	3014.7\\
28200	3020.7\\
28800	2909.5\\
29400	2933.5\\
30000	2860.4\\
};
\addplot [color=mycolor49, forget plot]
  table[row sep=crcr]{%
0	4395.3\\
600	4562.3\\
1200	4653.9\\
1800	4490.2\\
2400	4473.4\\
3000	4438.1\\
3600	4421.7\\
4200	4328.8\\
4800	4453.1\\
5400	4653.3\\
6000	4303.9\\
7200	4380.8\\
7800	4427.1\\
8400	4290.3\\
9000	4247.7\\
9600	4430.1\\
10200	4270.7\\
10800	4318.8\\
11400	4665.6\\
12000	4586.6\\
12600	4240.4\\
13200	4036\\
13800	4261.5\\
14400	4264.2\\
15000	4375\\
15600	4503.7\\
16200	4235.5\\
16800	4287.6\\
17400	4283.1\\
18000	4298.4\\
18600	4152.2\\
19200	4411\\
19800	4286.2\\
20400	4418.8\\
21000	4257.1\\
21600	4300.6\\
22200	4317\\
22800	4460.9\\
23400	4470.4\\
24000	4341.6\\
24600	4589.5\\
25200	4500.3\\
25800	4351.8\\
26400	4337.4\\
27000	4593\\
27600	4412.7\\
28200	4462.8\\
28800	4484.8\\
29400	4361.5\\
30000	4334.3\\
};
\addplot [color=mycolor50, forget plot]
  table[row sep=crcr]{%
0	4395.3\\
600	3249.4\\
1200	2544.5\\
1800	2473.7\\
2400	2464.3\\
3000	2434.4\\
3600	2472.2\\
4200	2429.4\\
4800	2511.2\\
5400	2410.5\\
6000	2380\\
6600	2466.6\\
7200	2459.2\\
7800	2517.8\\
8400	2452\\
9000	2446.2\\
9600	2598.8\\
10200	2512.5\\
10800	2401.3\\
11400	2553.1\\
12000	2505.3\\
12600	2340.1\\
13200	2393\\
13800	2355.5\\
14400	2423.4\\
15000	2455.2\\
15600	2513.9\\
16200	2485\\
16800	2466.3\\
17400	2495.4\\
18000	2456.7\\
18600	2497.7\\
19200	2457\\
19800	2565.5\\
20400	2417.7\\
21600	2490.7\\
22800	2444.8\\
23400	2399.5\\
24000	2573.6\\
24600	2500.4\\
25200	2471.4\\
25800	2498.4\\
26400	2435.8\\
27000	2528.6\\
27600	2496.7\\
28200	2416.1\\
28800	2649.7\\
29400	2543.3\\
30000	2526.4\\
};
\addplot [color=mycolor51, forget plot]
  table[row sep=crcr]{%
0	4395.3\\
600	5643.9\\
1200	5972\\
1800	5832.3\\
2400	5867.7\\
3000	5920.6\\
3600	6301.4\\
4200	6165.9\\
4800	5372.1\\
5400	5746.3\\
6000	5755.8\\
6600	5843\\
7200	6106\\
7800	6318.8\\
8400	6562.5\\
9000	6290.7\\
9600	6182.7\\
10200	5937.2\\
11400	6150\\
12000	5916.4\\
12600	5893\\
13200	5760.6\\
13800	5504.1\\
14400	5921.2\\
15000	6064.8\\
15600	6098.6\\
16200	5974.2\\
16800	5951.2\\
17400	5768.4\\
18000	5988.1\\
18600	6167.3\\
19200	5787.6\\
19800	5695.8\\
20400	5702.2\\
21000	6478.4\\
21600	6039.9\\
22200	5848.7\\
22800	5730.1\\
23400	5825.5\\
24000	6362.3\\
24600	6247.4\\
25200	6095.3\\
26400	6044.4\\
27000	5518.2\\
27600	5576.9\\
28200	6016.8\\
28800	6206.1\\
29400	6009.4\\
30000	6010.6\\
};
\addplot [color=mycolor52, forget plot]
  table[row sep=crcr]{%
0	4395.3\\
600	3974\\
1200	3487.2\\
1800	3391.9\\
2400	3473.1\\
3000	3352.5\\
3600	3465\\
4200	3463.1\\
4800	3412.4\\
5400	3675.5\\
6000	3564.7\\
6600	3496.9\\
7200	3578.7\\
7800	3548.4\\
8400	3644.5\\
9000	3712.2\\
9600	3601.2\\
10200	3658.1\\
10800	3696.1\\
11400	3593\\
12000	3319.1\\
12600	3323\\
13200	3360.8\\
13800	3449.8\\
14400	3565\\
15000	3449.1\\
15600	3688.8\\
16200	3537\\
16800	3489.4\\
17400	3534.1\\
18000	3514.7\\
18600	3608.1\\
19200	3476.6\\
19800	3548.4\\
20400	3533.5\\
21000	3529.1\\
21600	3479.5\\
22200	3270.7\\
22800	3538.3\\
23400	3451.5\\
25200	3498.1\\
25800	3544.7\\
26400	3686.6\\
27000	3370.6\\
27600	3205.7\\
28200	3444.1\\
28800	3459.4\\
29400	3277.7\\
30000	3406.8\\
};
\addplot [color=mycolor53, forget plot]
  table[row sep=crcr]{%
0	4395.3\\
600	5409.7\\
1200	5939.8\\
3000	5591.6\\
3600	5528.2\\
4200	5289.5\\
4800	5436\\
5400	5515.2\\
6000	5757.2\\
6600	5773.2\\
7200	5891\\
7800	5834.3\\
8400	5850.2\\
9000	5546.4\\
9600	5479.2\\
10200	5884.3\\
10800	5749.2\\
11400	5799.3\\
12000	5670.7\\
12600	5776.8\\
13200	5606.3\\
13800	5561.3\\
14400	5437\\
15000	5395.6\\
15600	5542.8\\
16200	5817\\
16800	5761.8\\
17400	5669.3\\
18000	5602.4\\
18600	5937.6\\
19200	5792.1\\
19800	5584.5\\
20400	5567.8\\
21000	5972.8\\
21600	5971.8\\
22200	5748.3\\
22800	5401.8\\
23400	5465.9\\
24000	5554.8\\
24600	5809.6\\
25200	5762.6\\
25800	6062\\
26400	5641.2\\
27000	5448\\
27600	5853.3\\
28200	5703.8\\
28800	5775.3\\
29400	5648.4\\
30000	5675.6\\
};
\addplot [color=mycolor54, forget plot]
  table[row sep=crcr]{%
0	4395.3\\
600	4421.1\\
1200	4664.1\\
1800	4515\\
2400	4620.3\\
3000	4645.7\\
3600	4635\\
4200	4590.7\\
4800	4702.5\\
5400	4571.6\\
6000	4638.9\\
6600	4662.1\\
7200	4769.8\\
7800	4812.7\\
8400	4771.7\\
9000	4536.2\\
9600	4562.9\\
10200	4546.6\\
10800	4604.3\\
11400	4484.4\\
12000	4533.6\\
12600	4495.4\\
13200	4571.4\\
13800	4476.5\\
14400	4569.5\\
15000	4481\\
15600	4548.6\\
16800	4610.5\\
17400	4524.9\\
18000	4512.2\\
18600	4637.2\\
19200	4691.5\\
19800	4762.1\\
20400	4692.6\\
21000	4548.7\\
21600	4551.5\\
22200	4369.5\\
22800	4413\\
23400	4673.1\\
24000	4491.5\\
24600	4475.3\\
25200	4430.9\\
25800	4505.1\\
26400	4782.3\\
27000	4788.7\\
27600	4590.8\\
28200	4625.8\\
28800	4430.7\\
29400	4338.1\\
30000	4511.8\\
};
\addplot [color=mycolor55, forget plot]
  table[row sep=crcr]{%
0	4395.3\\
600	5316.5\\
1200	5716.7\\
1800	5703.3\\
2400	5745.2\\
3000	5681.3\\
3600	5660.3\\
4200	5676.5\\
4800	5795.9\\
5400	5794.5\\
6000	5893.3\\
7200	5886.5\\
7800	5642.6\\
8400	5679\\
9000	5692.1\\
9600	5722.4\\
10200	5810\\
10800	5849.5\\
11400	5643.8\\
12000	5655\\
12600	5867.2\\
13200	5711.5\\
13800	5922.4\\
14400	5960\\
15000	6135.3\\
15600	5875.2\\
16200	5697.5\\
16800	5668.8\\
17400	5710\\
18000	5799.9\\
18600	5736.4\\
19200	5858.3\\
20400	5926.2\\
21000	5827.1\\
21600	5784.1\\
22200	5580.5\\
22800	5690.2\\
23400	5906.3\\
24000	5690.3\\
24600	5647.2\\
25200	5624.4\\
25800	5717.2\\
26400	5738.1\\
27000	5914.3\\
27600	5952\\
28200	5824.7\\
28800	5577.7\\
29400	5771\\
30000	5766.3\\
};
\addplot [color=mycolor56, forget plot]
  table[row sep=crcr]{%
0	4395.3\\
600	4279.4\\
1200	4094.4\\
1800	4130.1\\
2400	4029.4\\
3000	4097\\
3600	4044.5\\
4200	4091\\
4800	3979.2\\
5400	4034.5\\
6000	3943.7\\
6600	3874.6\\
7200	3988.4\\
7800	4014.4\\
8400	4106.5\\
9000	3977.6\\
9600	3971.7\\
10200	4022.7\\
10800	3942.6\\
11400	4044.5\\
12000	4037.3\\
12600	4088.3\\
13200	3879.2\\
14400	4025.1\\
15000	4144.7\\
16200	4000.8\\
16800	3954.5\\
17400	4087.5\\
18000	3959.7\\
18600	3995.9\\
19200	3831.6\\
19800	3786.2\\
20400	3896.3\\
21000	4087.9\\
21600	4156.2\\
22200	4017.1\\
22800	4043.4\\
23400	4153.8\\
24000	3993.8\\
24600	3872\\
25200	4016.8\\
25800	3935\\
26400	4091.4\\
27000	3999.5\\
27600	4117\\
28200	3977.1\\
28800	4051.2\\
29400	4110.3\\
30000	4120.1\\
};
\addplot [color=mycolor57, forget plot]
  table[row sep=crcr]{%
0	4395.3\\
600	5264.6\\
1200	6380.5\\
1800	6655.9\\
2400	6815\\
3000	6819.8\\
3600	6707.5\\
4200	6794\\
4800	6752\\
5400	6662\\
6000	6592.2\\
6600	6595.3\\
7800	6830.4\\
8400	6854.8\\
9000	6859.4\\
9600	6889.8\\
10200	6826.2\\
10800	6718.5\\
11400	6536.3\\
12000	6506.2\\
13200	6685.6\\
13800	6801.3\\
14400	6786.8\\
15000	6865.2\\
16200	6465\\
16800	6527\\
17400	6704.8\\
18000	6833.8\\
18600	6878.9\\
19200	6814\\
19800	6963\\
20400	6839.3\\
21000	6691.8\\
21600	6824.8\\
22200	6830.4\\
22800	6735\\
23400	6582.8\\
24000	6664.2\\
24600	6547.9\\
25200	6884.8\\
25800	6925.5\\
26400	6758.2\\
27000	6645.8\\
28200	6743.1\\
28800	6841.6\\
29400	6744.3\\
30000	6979.2\\
};
\addplot [color=mycolor58, forget plot]
  table[row sep=crcr]{%
0	4395.3\\
600	4085.9\\
1200	3753.6\\
1800	3649.8\\
2400	3707.2\\
3000	3640.9\\
3600	3659.8\\
4200	3515\\
4800	3461.2\\
5400	3659.1\\
6000	3686.4\\
6600	3394\\
7200	3640.8\\
7800	3648.7\\
8400	3611.3\\
9000	3654.8\\
9600	3604.7\\
10200	3570.2\\
10800	3609.3\\
11400	3604.5\\
12000	3577.8\\
12600	3471.7\\
13200	3630.1\\
14400	3659.1\\
15000	3617.8\\
15600	3442.5\\
16200	3504.8\\
16800	3649.6\\
17400	3617\\
18000	3478.3\\
18600	3672.8\\
19200	3610.9\\
19800	3570.1\\
20400	3585.1\\
21000	3649.4\\
21600	3557.1\\
22200	3650.5\\
22800	3628.7\\
23400	3576\\
24000	3562.6\\
24600	3496.6\\
25200	3598.8\\
25800	3497.4\\
26400	3409.7\\
27000	3460.8\\
27600	3576.3\\
28200	3615\\
28800	3548.9\\
29400	3608.4\\
30000	3484.4\\
};
\addplot [color=mycolor59, forget plot]
  table[row sep=crcr]{%
0	4395.3\\
600	4810\\
1200	5187\\
1800	5189.8\\
2400	5050.3\\
3600	4829.8\\
4200	5032.9\\
4800	5123.2\\
5400	5020.6\\
6600	5048.4\\
7200	4881.5\\
8400	4873.5\\
9000	4997.2\\
9600	5162.2\\
10200	5116.1\\
10800	5129.7\\
11400	5187\\
12000	4985.8\\
12600	4952.9\\
13200	4999.6\\
13800	4917.5\\
14400	5046.8\\
15000	5042.5\\
15600	5192.1\\
16200	5197.6\\
16800	4974.3\\
17400	5162.1\\
18000	5051.7\\
18600	4965.1\\
19200	5053.5\\
19800	4911.5\\
20400	4981.9\\
21000	5074.2\\
21600	5099.8\\
22200	5261.2\\
22800	5149.2\\
23400	5202\\
24000	4984.9\\
24600	4986\\
25200	5024\\
25800	5012.3\\
26400	5122.7\\
27000	5107.7\\
27600	4998.2\\
28200	5101.3\\
28800	5053.5\\
29400	4957.5\\
30000	5035.4\\
};
\addplot [color=mycolor60, forget plot]
  table[row sep=crcr]{%
0	4395.3\\
600	3981.7\\
1200	3643.4\\
1800	3593.9\\
2400	3764.7\\
3000	3645.7\\
3600	3577.4\\
4200	3523\\
4800	3498.2\\
5400	3448.9\\
6000	3539\\
6600	3617.8\\
7200	3647.1\\
7800	3591.7\\
8400	3551.3\\
9000	3556.7\\
9600	3612.2\\
10200	3560\\
10800	3590.5\\
11400	3505.4\\
12000	3585.7\\
12600	3464.7\\
13200	3426.6\\
13800	3588.6\\
14400	3642.7\\
15600	3635.3\\
16200	3601.1\\
16800	3508.2\\
17400	3508.9\\
18000	3696.1\\
18600	3576.5\\
19200	3581.5\\
20400	3652.8\\
21600	3530.4\\
22200	3560.1\\
22800	3574.7\\
23400	3480.7\\
24000	3562.8\\
24600	3550\\
25800	3451.1\\
26400	3489.8\\
27000	3635.4\\
27600	3702.3\\
28800	3381\\
29400	3408.8\\
30000	3550.1\\
};
\addplot [color=mycolor61, forget plot]
  table[row sep=crcr]{%
0	4395.3\\
600	3735\\
1200	3320.1\\
1800	3203.8\\
2400	3139.6\\
3000	3153.9\\
4200	3094.3\\
4800	3108.2\\
5400	3082\\
6000	3076.8\\
7200	3204.6\\
7800	3102.7\\
8400	3063.3\\
9000	3073.3\\
9600	3120.3\\
10200	3138.6\\
10800	3097.5\\
12000	3060.9\\
12600	3106.3\\
13800	3218.9\\
14400	3170.4\\
15000	3034.8\\
15600	3060.8\\
16200	3152.7\\
16800	3165.7\\
17400	3131\\
18000	3051.2\\
18600	3056.6\\
19200	3103.7\\
19800	3064.9\\
20400	3087.2\\
21000	3097.4\\
21600	3060.9\\
22200	3104.9\\
22800	3169.6\\
23400	3143.5\\
24000	3088.4\\
24600	3104.5\\
25200	3039.1\\
25800	3138.5\\
26400	3159.1\\
27000	3119\\
27600	3212.1\\
28200	3217.9\\
28800	3157.1\\
29400	3087.7\\
30000	3103.5\\
};
\addplot [color=mycolor62, forget plot]
  table[row sep=crcr]{%
0	4395.3\\
600	4617.4\\
1200	4990.7\\
1800	5059.7\\
2400	4875.6\\
3000	5037.5\\
3600	4953.7\\
4200	4885.9\\
4800	4957.4\\
6000	4845.6\\
6600	5030.2\\
7200	5034.3\\
7800	5016.7\\
8400	4935.9\\
9000	5067\\
9600	5037.9\\
10200	5083.9\\
11400	4951.6\\
12000	4947\\
12600	4896.5\\
13200	4865.5\\
13800	4815.6\\
14400	4840.2\\
15000	4933.9\\
15600	4928.9\\
16200	5026.8\\
16800	4904.5\\
17400	4893.6\\
18000	5044.9\\
19200	5093.2\\
19800	5023.3\\
20400	4895.6\\
21000	5049.9\\
22200	4964.5\\
23400	4837.6\\
24600	4958.9\\
25200	4952.7\\
25800	4793\\
26400	4979.4\\
27000	5153.2\\
27600	5200.7\\
28200	5043.3\\
28800	5092.5\\
29400	5087.3\\
30000	4905.4\\
};
\addplot [color=mycolor63, forget plot]
  table[row sep=crcr]{%
0	4395.3\\
600	3265.8\\
1200	2614.5\\
1800	2476.2\\
2400	2400.4\\
3000	2358.3\\
3600	2418\\
4200	2363.3\\
4800	2382.7\\
5400	2414.8\\
6000	2439.3\\
7200	2406\\
7800	2403.6\\
8400	2360.7\\
9000	2407.2\\
9600	2385.4\\
10200	2439.5\\
10800	2369.8\\
11400	2386.7\\
12000	2327.7\\
12600	2373.7\\
13200	2394.4\\
13800	2395\\
14400	2375\\
15000	2344.2\\
15600	2356.4\\
16200	2428.8\\
16800	2364.5\\
17400	2388.4\\
18000	2449.5\\
18600	2345.3\\
19200	2336.8\\
19800	2371.9\\
20400	2453.6\\
21000	2439.5\\
21600	2405.8\\
22200	2407.1\\
23400	2358.4\\
24000	2478.9\\
24600	2412.2\\
25200	2322\\
25800	2366.3\\
26400	2386.2\\
27000	2383.5\\
27600	2355.6\\
28200	2403.6\\
29400	2391.9\\
30000	2403.8\\
};
\addplot [color=mycolor64, forget plot]
  table[row sep=crcr]{%
0	4395.3\\
600	3570\\
1200	3151.9\\
2400	3096.9\\
3000	3102.3\\
3600	3073.8\\
4200	3077.9\\
4800	3020.4\\
6000	3022\\
6600	3102.6\\
7200	3004.8\\
7800	3015.8\\
8400	3015.4\\
9000	3044.1\\
9600	3051.5\\
10200	3114.7\\
11400	3125\\
12000	3089.9\\
12600	3004.1\\
13200	2941.7\\
13800	2955.8\\
14400	3069.2\\
15000	3080.8\\
16200	2983.9\\
16800	3106.2\\
18000	2980.8\\
18600	3009.3\\
19200	3020.4\\
20400	3152.8\\
21000	3060.4\\
21600	3005.7\\
22200	3096.9\\
22800	3084\\
23400	3024.2\\
24600	3000.2\\
25200	3040.9\\
26400	3163.4\\
27000	3097\\
27600	3095.7\\
28200	3027\\
28800	3003.4\\
29400	3052.4\\
30000	3034.5\\
};
\addplot [color=mycolor65, forget plot]
  table[row sep=crcr]{%
0	4395.3\\
600	3346.3\\
1200	2648.8\\
1800	2575.2\\
2400	2487.5\\
3000	2584\\
3600	2441.8\\
4200	2554.1\\
4800	2558.8\\
5400	2509.3\\
6000	2568.5\\
6600	2590.1\\
7200	2498.9\\
7800	2540\\
8400	2490.3\\
9600	2584.1\\
10200	2537.9\\
10800	2524.9\\
11400	2541.6\\
12000	2596.8\\
12600	2547.9\\
13200	2526\\
13800	2567.7\\
14400	2503.5\\
15000	2477.3\\
15600	2497.2\\
16200	2597.6\\
16800	2524.2\\
17400	2477\\
18000	2511.2\\
18600	2507.9\\
19200	2541.2\\
19800	2560.1\\
20400	2509.7\\
21000	2515.3\\
21600	2572.8\\
22200	2602.1\\
22800	2499.5\\
23400	2456.7\\
24000	2486\\
24600	2550.2\\
25200	2574.4\\
25800	2575.2\\
26400	2536.3\\
27000	2575\\
27600	2530.9\\
28200	2615.2\\
28800	2498.7\\
29400	2481\\
30000	2559.5\\
};
\addplot [color=mycolor66, forget plot]
  table[row sep=crcr]{%
0	4395.3\\
600	3839.5\\
1200	3577.2\\
1800	3479.7\\
2400	3473\\
3000	3426.6\\
3600	3357.1\\
4200	3527.8\\
4800	3395.5\\
5400	3353.5\\
6000	3418.4\\
6600	3450.3\\
7200	3368\\
7800	3491\\
9000	3472.5\\
10200	3329.5\\
10800	3394.4\\
11400	3395.6\\
12000	3384.8\\
12600	3473.5\\
13200	3489.7\\
13800	3419\\
14400	3364.2\\
16200	3457.2\\
16800	3416\\
17400	3456\\
18000	3377.1\\
18600	3338.3\\
19200	3424.5\\
19800	3365.6\\
20400	3369.1\\
22200	3458.1\\
22800	3365.6\\
23400	3413.1\\
24000	3314.1\\
24600	3417.1\\
25200	3455.5\\
25800	3460.4\\
26400	3517.6\\
27000	3490.2\\
27600	3413.4\\
28800	3464.4\\
29400	3419\\
30000	3413.5\\
};
\addplot [color=mycolor67, forget plot]
  table[row sep=crcr]{%
0	4395.3\\
600	3774.4\\
1200	3400.2\\
1800	3234.4\\
2400	3171.1\\
3000	3118.5\\
3600	3173\\
4200	3146\\
4800	3147.5\\
5400	3178.8\\
6000	3116.6\\
6600	3224\\
7200	3238.2\\
7800	3183.2\\
8400	3211.3\\
9000	3148.3\\
9600	3148.8\\
10200	3207.9\\
10800	3248.6\\
11400	3181.2\\
12000	3128\\
12600	3185.3\\
13200	3196.5\\
13800	3290.5\\
14400	3188.7\\
15000	3195.6\\
15600	3258.9\\
16200	3245.4\\
16800	3247\\
17400	3207.4\\
18000	3207.1\\
18600	3146.1\\
19200	3286.7\\
19800	3249.1\\
21000	3240.5\\
21600	3252.4\\
22200	3183.4\\
22800	3198.5\\
23400	3130.7\\
24000	3127.9\\
24600	3105.1\\
25200	3176.2\\
25800	3139.9\\
26400	3215.2\\
27000	3205.5\\
27600	3166.9\\
28200	3198.1\\
28800	3204.8\\
29400	3179\\
30000	3194\\
};
\addplot [color=mycolor67]
  table[row sep=crcr]{%
0	4395.3\\
600	3425.7\\
1200	2813.5\\
1800	2693.5\\
2400	2659.6\\
3000	2654.7\\
3600	2623.6\\
4200	2633.6\\
4800	2665.4\\
5400	2600.3\\
6000	2617.9\\
6600	2690.1\\
8400	2644.9\\
9000	2670.8\\
9600	2657.7\\
10200	2682.9\\
10800	2663\\
11400	2653.5\\
12000	2682.2\\
12600	2668.8\\
13200	2633.2\\
13800	2660.5\\
14400	2674.7\\
15000	2670.1\\
15600	2689.5\\
16200	2698.8\\
16800	2657.9\\
18000	2636.4\\
18600	2653.4\\
19200	2655.6\\
19800	2694.9\\
21000	2628.3\\
21600	2607.1\\
22200	2639.3\\
22800	2642.6\\
23400	2606.1\\
24000	2603.6\\
24600	2644.8\\
25200	2649\\
25800	2583.2\\
26400	2579.2\\
27000	2560.1\\
27600	2622.1\\
28200	2648.6\\
28800	2691.8\\
29400	2636.7\\
30000	2600.2\\
};
\addlegendentry{Classe k=43}

\addplot [color=mycolor68, forget plot]
  table[row sep=crcr]{%
0	4395.3\\
600	3433.4\\
1200	2712.2\\
1800	2503.1\\
2400	2454.7\\
3000	2523.5\\
3600	2488.1\\
4200	2469.7\\
4800	2423.4\\
5400	2461\\
6000	2479.8\\
6600	2485.5\\
7200	2504\\
7800	2407.5\\
8400	2434.5\\
9000	2444.2\\
9600	2467.1\\
10800	2453.1\\
11400	2420.4\\
12000	2411.7\\
12600	2453.2\\
13200	2484.7\\
13800	2426.7\\
14400	2473.1\\
15000	2428.8\\
15600	2444.5\\
16200	2447.2\\
16800	2501.4\\
17400	2455.1\\
18000	2468.3\\
18600	2473.9\\
19200	2440.3\\
19800	2482.2\\
21000	2454.2\\
21600	2391.6\\
22200	2421.9\\
22800	2500\\
23400	2487.3\\
24000	2498.9\\
24600	2463.4\\
25200	2498.8\\
25800	2448\\
28800	2467.9\\
29400	2461.7\\
30000	2433.7\\
};
\addplot [color=mycolor69, forget plot]
  table[row sep=crcr]{%
0	4395.3\\
600	3639\\
1200	3116.4\\
1800	2974.1\\
2400	2898.6\\
3000	2839.4\\
3600	2886.6\\
4200	2970.2\\
4800	2993\\
5400	2917.9\\
6000	2914\\
6600	2873.6\\
7200	2915.5\\
7800	2903.7\\
8400	2920.8\\
9000	2867.5\\
9600	2903.8\\
10200	2833.5\\
10800	2882.3\\
11400	2891.5\\
12000	2853.4\\
12600	2900.6\\
14400	2878.3\\
15000	2909.9\\
15600	2955.7\\
17400	2946.8\\
18000	3013.5\\
18600	2994\\
19800	2858.4\\
20400	2873.8\\
22200	2869.6\\
22800	2937.2\\
23400	2934.3\\
24000	2945.1\\
24600	2916.7\\
25200	2923.2\\
25800	2888.8\\
26400	2892.8\\
27000	2870.6\\
28200	2874.8\\
28800	2848.9\\
29400	2848.1\\
30000	2903.7\\
};
\addplot [color=mycolor70, forget plot]
  table[row sep=crcr]{%
0	4395.3\\
600	3125.2\\
1200	2384.5\\
1800	2238.5\\
2400	2196.6\\
3600	2084\\
4200	2098.3\\
4800	2092.8\\
5400	2183\\
6000	2148.9\\
6600	2124.4\\
7200	2147.9\\
7800	2179.4\\
8400	2163.9\\
9600	2104.6\\
10200	2098.8\\
10800	2161.1\\
11400	2176\\
12000	2163.2\\
12600	2137.5\\
13200	2087.4\\
13800	2075.7\\
14400	2078.9\\
15000	2070.7\\
16200	2171.8\\
16800	2116.1\\
17400	2165.4\\
18000	2145\\
18600	2103.4\\
19200	2074.3\\
19800	2100.3\\
20400	2085.5\\
21000	2086.7\\
21600	2123.2\\
22200	2106.9\\
22800	2121.1\\
23400	2049\\
24000	2039.8\\
24600	2088.4\\
25200	2061.2\\
25800	2079.7\\
26400	2135.3\\
27000	2129.5\\
27600	2168.1\\
28200	2111.2\\
28800	2093.4\\
29400	2116.5\\
30000	2063.1\\
};
\addplot [color=mycolor71, forget plot]
  table[row sep=crcr]{%
0	4395.3\\
600	3335.6\\
1200	2666.2\\
1800	2603.6\\
2400	2574.9\\
3000	2537.3\\
3600	2520.3\\
4200	2483.3\\
4800	2488.1\\
5400	2427.4\\
6000	2454.7\\
6600	2445.3\\
7200	2492.8\\
7800	2489.9\\
8400	2468.4\\
9000	2464.8\\
9600	2442.3\\
10200	2471.7\\
10800	2444.6\\
11400	2496.7\\
12000	2507.3\\
12600	2495.2\\
13200	2453.2\\
13800	2508\\
14400	2473.6\\
15000	2466.7\\
16200	2558\\
16800	2526.5\\
17400	2543.1\\
18000	2496.2\\
18600	2468.1\\
19200	2467.2\\
19800	2530.1\\
20400	2482.1\\
21000	2467.4\\
21600	2526.6\\
22200	2537.2\\
22800	2536.3\\
23400	2527.8\\
24000	2494.5\\
24600	2422.2\\
25200	2443.1\\
25800	2493.6\\
26400	2533.7\\
27000	2536.4\\
27600	2499.5\\
28200	2515.6\\
28800	2476.5\\
29400	2498.7\\
30000	2502.4\\
};
\addplot [color=mycolor72, forget plot]
  table[row sep=crcr]{%
0	4395.3\\
600	3265.1\\
1200	2406.4\\
1800	2306.3\\
2400	2183.6\\
3000	2197.1\\
3600	2202.4\\
4200	2225.3\\
4800	2162.9\\
6000	2202.8\\
6600	2184\\
7200	2208.3\\
7800	2224.8\\
8400	2150.3\\
9000	2200.5\\
9600	2228.7\\
10200	2143.9\\
11400	2199.8\\
12600	2204\\
13200	2179.1\\
13800	2195.4\\
14400	2222.8\\
15000	2211.5\\
15600	2222.4\\
16200	2183.6\\
16800	2228.2\\
17400	2181.7\\
18000	2164.8\\
18600	2212.3\\
19200	2234.1\\
20400	2203.1\\
21000	2214.6\\
21600	2183.1\\
22200	2211.6\\
22800	2210.1\\
24000	2173.5\\
24600	2194.2\\
25200	2239.1\\
25800	2204.8\\
26400	2188.3\\
27000	2238.6\\
27600	2202.4\\
28200	2199\\
28800	2187.6\\
29400	2206.8\\
30000	2183.6\\
};
\addplot [color=mycolor73, forget plot]
  table[row sep=crcr]{%
0	4395.3\\
600	3424.5\\
1200	2791.8\\
1800	2645.3\\
3000	2592.9\\
4200	2587.9\\
4800	2580.3\\
5400	2618.8\\
6000	2606.7\\
6600	2621.1\\
7200	2582.6\\
7800	2632\\
8400	2630.6\\
9000	2654.6\\
9600	2628.3\\
10200	2552\\
10800	2568.7\\
11400	2524.3\\
12000	2599.5\\
12600	2649.6\\
15000	2632.5\\
15600	2604.2\\
16200	2595.4\\
16800	2563.7\\
18000	2541.9\\
18600	2590.8\\
19200	2584.7\\
19800	2604.3\\
20400	2557.6\\
21000	2587.4\\
21600	2582.9\\
22200	2565.5\\
22800	2620.1\\
23400	2573.5\\
24000	2594\\
24600	2595.7\\
25200	2566.7\\
25800	2604.2\\
27000	2642.1\\
27600	2593.8\\
28200	2563.2\\
28800	2580.5\\
29400	2615.3\\
30000	2607.8\\
};
\addplot [color=mycolor74, forget plot]
  table[row sep=crcr]{%
0	4395.3\\
600	2836.8\\
1200	1796\\
1800	1612.7\\
2400	1550.7\\
3000	1525.3\\
3600	1566.3\\
4200	1563.8\\
4800	1538.9\\
5400	1533.5\\
6000	1558.9\\
6600	1513.2\\
7200	1514.6\\
7800	1545.2\\
8400	1509\\
9000	1527.9\\
10200	1529.9\\
11400	1503.1\\
12000	1546.7\\
12600	1491.4\\
13200	1499.6\\
13800	1533.9\\
14400	1539.2\\
15000	1522.4\\
15600	1526.4\\
16200	1562.6\\
16800	1575.1\\
17400	1563\\
18000	1559.8\\
18600	1530.8\\
19200	1553.3\\
19800	1499.6\\
20400	1541.3\\
21000	1541.2\\
21600	1546.4\\
22200	1539.8\\
22800	1496.2\\
23400	1487.3\\
24000	1580.6\\
24600	1551.5\\
25200	1555.6\\
25800	1571.9\\
26400	1571.3\\
27000	1547.8\\
27600	1556.1\\
28200	1543.4\\
29400	1542\\
30000	1524.8\\
};
\addplot [color=mycolor75, forget plot]
  table[row sep=crcr]{%
0	4395.3\\
600	3267.2\\
1200	2536\\
1800	2291.3\\
2400	2179.4\\
3000	2225.6\\
3600	2196.4\\
4200	2174.7\\
4800	2211.6\\
5400	2223.7\\
6000	2165.2\\
6600	2239.7\\
7200	2195\\
8400	2200.4\\
9000	2247.2\\
9600	2234.5\\
10200	2192.7\\
10800	2185.2\\
11400	2222.6\\
12600	2196.5\\
13200	2242.7\\
13800	2230\\
14400	2250.7\\
15000	2228\\
15600	2235.2\\
16200	2209.2\\
16800	2281.8\\
17400	2248.3\\
18000	2269.2\\
18600	2227.3\\
19200	2205.7\\
19800	2176.6\\
20400	2189.9\\
21000	2210.3\\
21600	2183.3\\
22200	2208.7\\
22800	2178.5\\
23400	2192.7\\
24600	2231.6\\
25200	2208.5\\
25800	2238.3\\
26400	2244.9\\
27000	2238\\
27600	2214.6\\
28800	2218.7\\
30000	2165\\
};
\addplot [color=mycolor76, forget plot]
  table[row sep=crcr]{%
0	4395.3\\
600	3157.5\\
1200	2311.8\\
1800	2096.4\\
2400	2027.4\\
3000	2013.9\\
4200	2011.3\\
4800	2026.5\\
5400	2058.4\\
6000	2058.4\\
6600	2028.1\\
7200	2046.6\\
7800	2048.9\\
8400	2042.5\\
9600	2001.1\\
10200	2010.8\\
11400	1981.5\\
12000	1992.5\\
12600	2038.9\\
13200	2055.4\\
13800	2000.4\\
14400	1928.4\\
15000	1981.2\\
15600	1979.3\\
16800	2011.4\\
17400	2007.9\\
18000	2019.9\\
19800	1999.8\\
20400	2004.5\\
21000	1995.6\\
21600	2001.3\\
22200	2019.8\\
22800	2004.4\\
23400	2012.2\\
24000	2027.8\\
25200	1989.3\\
25800	2009.8\\
26400	2042.3\\
27000	2020.1\\
27600	2013.3\\
28200	1967.4\\
28800	1993.6\\
29400	2032.8\\
30000	2031.1\\
};
\addplot [color=mycolor77, forget plot]
  table[row sep=crcr]{%
0	4395.3\\
600	2900.6\\
1200	1842.4\\
1800	1599.9\\
2400	1602.1\\
3000	1577.9\\
3600	1615.5\\
4200	1589.4\\
4800	1585.2\\
5400	1598.4\\
6600	1537.3\\
7200	1537.1\\
7800	1564\\
8400	1545.9\\
9000	1557.6\\
9600	1496.4\\
10200	1548.8\\
10800	1540.8\\
11400	1543.7\\
12000	1568.8\\
12600	1607.9\\
13200	1584.3\\
13800	1552.2\\
14400	1561.9\\
15000	1545.8\\
15600	1564.2\\
16200	1555.8\\
16800	1608.4\\
17400	1564.8\\
18000	1562.3\\
18600	1553\\
19200	1531.1\\
19800	1526.4\\
20400	1573.6\\
21000	1542.8\\
21600	1559.2\\
22200	1563.7\\
22800	1550.7\\
23400	1547.9\\
24000	1567.3\\
24600	1559\\
25200	1561.7\\
25800	1571.3\\
26400	1541.3\\
27000	1537.9\\
27600	1525.8\\
28200	1547.8\\
28800	1537.7\\
29400	1569.6\\
30000	1564.5\\
};
\addplot [color=mycolor78, forget plot]
  table[row sep=crcr]{%
0	4395.3\\
1200	2649.1\\
1800	2352.5\\
2400	2307.5\\
3000	2308.8\\
3600	2272\\
4200	2336.7\\
4800	2276.9\\
5400	2324.4\\
6000	2299.5\\
6600	2241.1\\
7200	2293.4\\
7800	2283.8\\
8400	2308.1\\
9000	2285.9\\
9600	2234.2\\
10200	2270\\
10800	2248\\
11400	2363.6\\
12000	2362.2\\
12600	2314.3\\
13800	2284.1\\
14400	2345.8\\
15000	2282.1\\
15600	2299\\
16200	2265.6\\
16800	2258.3\\
17400	2288.5\\
18000	2330.8\\
18600	2309.8\\
19200	2234\\
19800	2277.3\\
20400	2268\\
21000	2240.6\\
21600	2224.1\\
22200	2259.4\\
22800	2325.3\\
23400	2285.7\\
24000	2276.6\\
24600	2249.3\\
25200	2277.3\\
26400	2282.6\\
27000	2313.7\\
27600	2311.9\\
28200	2266\\
28800	2250.3\\
29400	2336.2\\
30000	2327\\
};
\addplot [color=mycolor79, forget plot]
  table[row sep=crcr]{%
0	4395.3\\
600	3321.8\\
1200	2543.1\\
1800	2273.9\\
2400	2212\\
3000	2167.6\\
3600	2199.2\\
4200	2174.8\\
4800	2132.9\\
5400	2169.6\\
6000	2172.9\\
6600	2213.4\\
7200	2215.7\\
7800	2132.9\\
9000	2108.4\\
9600	2053.1\\
10200	2135.1\\
11400	2111.6\\
12000	2147.3\\
12600	2157\\
13200	2183.8\\
14400	2158.6\\
15000	2185.8\\
15600	2233.2\\
16200	2226.2\\
16800	2261.6\\
17400	2175.3\\
18000	2186.3\\
18600	2163.4\\
19200	2180.7\\
19800	2166.1\\
20400	2175.9\\
21000	2107.7\\
21600	2093.4\\
22200	2152.2\\
22800	2156.1\\
23400	2169.8\\
24000	2165\\
24600	2152.8\\
25200	2156.9\\
25800	2122\\
26400	2150.3\\
27000	2220.8\\
27600	2216.6\\
28200	2220.7\\
28800	2214.9\\
29400	2174\\
30000	2126.2\\
};
\addplot [color=mycolor79, forget plot]
  table[row sep=crcr]{%
0	4395.3\\
1200	2377.3\\
1800	2131.9\\
2400	2065.7\\
3000	2068.6\\
3600	2002\\
4200	1957.4\\
4800	2021.4\\
5400	2029\\
6000	2005.9\\
6600	2013.1\\
7200	1986.7\\
7800	2005.4\\
8400	2064.9\\
9000	2069.3\\
9600	2062.1\\
10200	2006.1\\
10800	2038.2\\
11400	2079.2\\
12000	2091\\
12600	2083.7\\
13800	1967.6\\
14400	2031\\
15000	2004.8\\
15600	1995.7\\
16200	1961.6\\
17400	2015.7\\
18000	2068\\
18600	2052\\
19200	2048.2\\
19800	2019.5\\
20400	2020.9\\
21000	1990.8\\
21600	2082.2\\
22200	2066.9\\
22800	2031.5\\
23400	2099\\
24000	2047\\
24600	2018.6\\
25200	2036.2\\
25800	1994.9\\
26400	1990\\
27000	2036.5\\
27600	2044.3\\
28200	2067.4\\
29400	2064.4\\
30000	2037.7\\
};
\addplot [color=mycolor80, forget plot]
  table[row sep=crcr]{%
0	4395.3\\
600	3028.3\\
1200	1977.1\\
1800	1674.2\\
2400	1588.9\\
3000	1576.8\\
3600	1571.6\\
4200	1545.6\\
4800	1510.9\\
5400	1534\\
6000	1550.3\\
6600	1581.2\\
7200	1569.7\\
7800	1589.6\\
8400	1552.2\\
9000	1533.7\\
9600	1556.5\\
10200	1541.8\\
10800	1571.6\\
11400	1576.1\\
12000	1575.4\\
12600	1579.7\\
13200	1594.5\\
13800	1555.7\\
14400	1551.3\\
16200	1572.9\\
16800	1536.7\\
17400	1545.8\\
18000	1537.9\\
18600	1549.9\\
19200	1567.4\\
19800	1559.8\\
20400	1540.9\\
21000	1550.2\\
21600	1570.2\\
22200	1551.2\\
22800	1553.3\\
23400	1576.4\\
24000	1529.5\\
24600	1542.3\\
25200	1538\\
25800	1517\\
26400	1536.9\\
27000	1571.9\\
27600	1574.7\\
28200	1569.2\\
28800	1533\\
29400	1534.4\\
30000	1550.1\\
};
\addplot [color=mycolor81, forget plot]
  table[row sep=crcr]{%
0	4395.3\\
600	2953\\
1200	1860.9\\
1800	1582.2\\
2400	1473.4\\
3000	1447\\
3600	1425.5\\
4200	1446.3\\
4800	1472\\
5400	1463\\
6000	1422.3\\
6600	1450.5\\
7200	1470.8\\
7800	1452.8\\
9000	1438.6\\
9600	1467.7\\
10200	1457.5\\
10800	1437.3\\
11400	1435\\
12000	1424.5\\
12600	1425.7\\
13200	1448.3\\
13800	1428.2\\
14400	1443.8\\
15600	1465.8\\
16200	1460.4\\
16800	1468.9\\
18000	1452.1\\
18600	1467.4\\
19200	1437.4\\
19800	1425.3\\
20400	1453\\
21000	1432.8\\
21600	1449.7\\
22200	1477\\
22800	1488.2\\
23400	1465.1\\
24000	1434.4\\
24600	1435.8\\
25200	1452.8\\
25800	1464.2\\
26400	1461\\
27000	1429.5\\
27600	1472.4\\
28200	1459.6\\
28800	1473\\
29400	1450\\
30000	1461.8\\
};
\addplot [color=mycolor82, forget plot]
  table[row sep=crcr]{%
0	4395.3\\
600	3068.3\\
1200	2093.3\\
1800	1792.6\\
2400	1743.6\\
3000	1735.6\\
3600	1702.9\\
4200	1684.4\\
4800	1711.2\\
5400	1730.9\\
6000	1660.8\\
7200	1725.9\\
7800	1691.4\\
8400	1701.6\\
9000	1685.8\\
9600	1693.1\\
10800	1676\\
11400	1708\\
12600	1743\\
13200	1697\\
13800	1690.8\\
14400	1695.4\\
15000	1671\\
15600	1683.7\\
16200	1713.5\\
16800	1700\\
17400	1694.1\\
18000	1709.8\\
18600	1730.5\\
19200	1711.6\\
20400	1730.2\\
21000	1719.3\\
21600	1687.5\\
22200	1678.8\\
22800	1661.4\\
23400	1651.3\\
24000	1653\\
25200	1729.9\\
25800	1751.4\\
26400	1756.3\\
27000	1712.9\\
27600	1703.9\\
28200	1686.6\\
29400	1684.6\\
30000	1688.7\\
};
\addplot [color=mycolor83, forget plot]
  table[row sep=crcr]{%
0	4395.3\\
600	3199\\
1200	2157.5\\
1800	1897.8\\
2400	1782.9\\
3000	1752.2\\
3600	1748.3\\
4200	1714.9\\
4800	1738.3\\
5400	1726\\
6000	1686.4\\
6600	1685\\
7200	1704.2\\
7800	1760.3\\
8400	1749.4\\
9000	1714.6\\
9600	1722.8\\
10200	1697.5\\
10800	1681.8\\
11400	1720.9\\
12600	1760.7\\
13200	1731.7\\
13800	1715.3\\
14400	1740.2\\
15600	1686.5\\
16200	1713.5\\
16800	1688.9\\
17400	1699.5\\
18000	1703.7\\
18600	1732.2\\
19200	1711\\
19800	1735\\
20400	1738.1\\
21000	1720.3\\
21600	1733.5\\
22200	1725\\
22800	1745.7\\
23400	1700.4\\
24000	1729.3\\
24600	1741.2\\
25200	1735.5\\
25800	1704.7\\
26400	1702.3\\
27600	1718.9\\
28200	1743.7\\
28800	1723.3\\
29400	1677.9\\
30000	1729.2\\
};
\addplot [color=mycolor83, forget plot]
  table[row sep=crcr]{%
0	4395.3\\
600	2858.3\\
1200	1680.3\\
1800	1401.4\\
2400	1335.8\\
3000	1306.8\\
3600	1256.1\\
4200	1244.8\\
4800	1270.6\\
5400	1284.4\\
6000	1322.6\\
6600	1282.7\\
7200	1262.3\\
7800	1257\\
8400	1235.6\\
9000	1228.8\\
9600	1258.6\\
10800	1235.5\\
11400	1240.9\\
12000	1269.3\\
13200	1286\\
13800	1244.8\\
14400	1258\\
15000	1261.1\\
15600	1257.7\\
16200	1259.7\\
16800	1254.2\\
17400	1253.1\\
18000	1295.2\\
18600	1287.4\\
19200	1265.3\\
19800	1258.7\\
20400	1242.3\\
21000	1254.8\\
21600	1289.5\\
22800	1310.7\\
24000	1269.5\\
24600	1272.5\\
25200	1249.1\\
25800	1252.8\\
26400	1232.1\\
27000	1269.2\\
27600	1255.9\\
28200	1250\\
28800	1257.7\\
30000	1254.5\\
};
\addplot [color=mycolor84, forget plot]
  table[row sep=crcr]{%
0	4395.3\\
600	2936.3\\
1200	1733.5\\
1800	1401.4\\
2400	1327.7\\
3000	1307.4\\
4200	1299.9\\
4800	1283.8\\
5400	1271.4\\
6000	1289.8\\
6600	1285.7\\
7200	1308.3\\
7800	1276.5\\
8400	1260.2\\
9000	1275\\
9600	1256.1\\
10200	1271\\
10800	1299.7\\
11400	1285.2\\
12000	1310.3\\
12600	1301.5\\
13800	1267\\
14400	1271.6\\
15000	1318.5\\
15600	1265.9\\
16200	1261.4\\
16800	1265.5\\
17400	1306.7\\
18000	1280.9\\
18600	1264.6\\
19200	1287.8\\
20400	1235.1\\
21000	1233\\
22200	1270.7\\
22800	1281.7\\
23400	1274.7\\
24000	1259.5\\
24600	1259.5\\
25200	1249.8\\
25800	1287.5\\
26400	1282.2\\
27000	1283.1\\
27600	1272.9\\
28200	1355.1\\
29400	1219.5\\
30000	1266.5\\
};
\addplot [color=mycolor85, forget plot]
  table[row sep=crcr]{%
0	4395.3\\
600	3003.4\\
1200	1761.6\\
1800	1543.1\\
2400	1448.4\\
3000	1432.4\\
3600	1412.1\\
4200	1437.2\\
4800	1450.2\\
6000	1389\\
6600	1425.5\\
7200	1425.1\\
7800	1392.7\\
8400	1402.8\\
9000	1392.3\\
9600	1397.4\\
10200	1355.9\\
10800	1330.6\\
11400	1297.5\\
12000	1372.7\\
12600	1401.3\\
13200	1373.8\\
13800	1365.9\\
14400	1413.2\\
15000	1477.5\\
15600	1412.2\\
16200	1393.3\\
17400	1379.6\\
18000	1385.2\\
19200	1352.6\\
20400	1373.7\\
21600	1398.8\\
22200	1391\\
22800	1389.8\\
23400	1407.2\\
24000	1417.9\\
24600	1395\\
25200	1387.7\\
25800	1362.2\\
26400	1392.6\\
27000	1370.8\\
27600	1392.7\\
28200	1341.9\\
28800	1389.5\\
29400	1391.1\\
30000	1377.8\\
};
\addplot [color=mycolor86, forget plot]
  table[row sep=crcr]{%
0	4395.3\\
600	2930.4\\
1200	1714\\
1800	1359.7\\
2400	1246.7\\
3000	1223.7\\
3600	1174.5\\
4200	1177.2\\
4800	1187.5\\
5400	1145.6\\
6000	1151.2\\
6600	1176.3\\
7200	1181.2\\
7800	1170.7\\
9000	1181.8\\
10200	1153.7\\
10800	1192.9\\
11400	1174.3\\
12000	1168.8\\
12600	1146.9\\
13800	1197.4\\
14400	1207.9\\
16200	1199.2\\
16800	1155.4\\
17400	1169.8\\
18000	1190.6\\
18600	1158\\
19200	1164\\
19800	1177.9\\
20400	1158.4\\
21000	1195.6\\
21600	1218\\
22200	1230.2\\
22800	1198.3\\
24000	1148.6\\
24600	1179.3\\
25200	1171.9\\
25800	1175.8\\
26400	1183\\
27000	1172.2\\
27600	1175.3\\
28200	1186.2\\
28800	1170.1\\
29400	1187.8\\
30000	1171.5\\
};
\addplot [color=mycolor87, forget plot]
  table[row sep=crcr]{%
0	4395.3\\
600	2789.2\\
1200	1590.3\\
1800	1182.1\\
2400	1072.8\\
3000	987.94\\
3600	1001.6\\
4200	973.02\\
4800	951.25\\
5400	1048.5\\
6000	1007.1\\
6600	970.15\\
7200	1007.3\\
7800	1028.1\\
8400	1002.7\\
9000	986.51\\
9600	987.97\\
10200	1004\\
10800	977.94\\
11400	973.09\\
12000	1039.1\\
12600	976.01\\
13200	968.71\\
13800	911.12\\
14400	881.96\\
15000	928.42\\
15600	953.19\\
16800	997.53\\
17400	973.47\\
18000	953.52\\
18600	953.34\\
19200	963.68\\
19800	989.51\\
20400	996.23\\
21000	1024.6\\
21600	1004.2\\
22200	949.25\\
22800	974.52\\
23400	935.31\\
24000	971.93\\
24600	992.22\\
25200	953.48\\
25800	976.83\\
26400	951.93\\
27000	939.61\\
27600	977.54\\
28200	1009.1\\
28800	1001\\
29400	1001.1\\
30000	1024.8\\
};
\addplot [color=mycolor88, forget plot]
  table[row sep=crcr]{%
0	4395.3\\
600	3055.8\\
1200	1773.6\\
1800	1388.8\\
2400	1257\\
3000	1207.1\\
3600	1197.5\\
4200	1160.5\\
4800	1111.1\\
5400	1162.6\\
6000	1165.7\\
6600	1163.6\\
7200	1187.7\\
7800	1160.7\\
8400	1143.6\\
9000	1163.3\\
9600	1168.5\\
10200	1180.9\\
10800	1154.7\\
11400	1194\\
12000	1204.2\\
12600	1193.8\\
13200	1242.2\\
13800	1190.7\\
14400	1182.6\\
15000	1168\\
15600	1184.5\\
16200	1182.5\\
17400	1147.7\\
18000	1151.1\\
18600	1162.9\\
19200	1153.1\\
19800	1174.8\\
20400	1168.8\\
21000	1144.6\\
21600	1166.8\\
22200	1204.8\\
22800	1190.6\\
23400	1201.5\\
24000	1173.9\\
24600	1180.1\\
25200	1154.1\\
25800	1121.9\\
26400	1142.3\\
27000	1154.1\\
27600	1146\\
28200	1178.2\\
29400	1173.9\\
30000	1138.4\\
};
\addplot [color=mycolor89, forget plot]
  table[row sep=crcr]{%
0	4395.3\\
600	2823.8\\
1200	1476.4\\
1800	1069.9\\
2400	913.2\\
3000	839.3\\
3600	827.52\\
4200	857.07\\
4800	849.77\\
5400	845.36\\
6000	848.84\\
6600	830.68\\
7200	819.48\\
7800	789.6\\
8400	828.75\\
9000	829.31\\
9600	822.16\\
10200	832.2\\
10800	834.24\\
11400	858.35\\
12000	835.85\\
12600	827.36\\
13200	842.99\\
13800	847.24\\
14400	828.94\\
15000	841.02\\
15600	863.15\\
16200	848.28\\
16800	856.95\\
17400	855.51\\
18000	871.62\\
18600	848.23\\
19200	823\\
19800	821.67\\
20400	847.71\\
21000	865.11\\
22200	828.54\\
22800	817\\
23400	824.2\\
24000	865.39\\
24600	826.98\\
25200	843.55\\
25800	871.35\\
26400	858.37\\
27000	855.35\\
27600	843.1\\
28200	837.55\\
28800	841.92\\
29400	868.51\\
30000	867.95\\
};
\addplot [color=mycolor90, forget plot]
  table[row sep=crcr]{%
0	4395.3\\
600	2808.2\\
1200	1440.9\\
1800	1048.4\\
2400	913.39\\
3000	841.34\\
3600	829.3\\
4200	814.95\\
5400	838.45\\
6000	834.25\\
6600	811.45\\
7200	834.91\\
7800	829.31\\
8400	816.11\\
9000	827.93\\
9600	805.58\\
10200	789.38\\
10800	789.41\\
11400	795.1\\
12000	814.31\\
12600	829.21\\
13200	831.55\\
13800	838.38\\
14400	821.91\\
15000	809.8\\
15600	823.63\\
16200	811.37\\
16800	815.54\\
17400	832.16\\
18000	822.07\\
18600	842.71\\
19200	836.54\\
20400	818.83\\
21000	828.32\\
22800	813.83\\
23400	847.32\\
24000	808.74\\
24600	800.45\\
25800	804.88\\
26400	815.95\\
27000	835.46\\
27600	818.66\\
28800	811.07\\
29400	813.84\\
30000	821.97\\
};
\addplot [color=mycolor91, forget plot]
  table[row sep=crcr]{%
0	4395.3\\
600	2750\\
1200	1404.4\\
1800	932.42\\
2400	789.77\\
3000	716.89\\
3600	713.78\\
4200	680.55\\
4800	668.58\\
5400	676.88\\
6000	705.84\\
6600	698.69\\
7200	683.49\\
7800	656.27\\
8400	683.34\\
9000	700.42\\
9600	700.8\\
10200	728.47\\
10800	718.71\\
11400	721.94\\
12000	700.75\\
12600	709.29\\
13200	728.73\\
13800	725.21\\
14400	724.09\\
15000	709.11\\
15600	690.23\\
16200	719.14\\
16800	740.08\\
17400	725.54\\
18000	717.95\\
18600	701.83\\
19200	682.73\\
19800	699.37\\
20400	714.33\\
21000	691.14\\
21600	672\\
22200	676.3\\
22800	699.64\\
23400	710.81\\
24000	708.33\\
25200	718.56\\
25800	710.77\\
26400	730.3\\
27000	723.57\\
27600	730.19\\
28200	716.67\\
28800	692.08\\
29400	695.41\\
30000	671.95\\
};
\addplot [color=mycolor92, forget plot]
  table[row sep=crcr]{%
0	4395.3\\
600	2962.5\\
1200	1669.3\\
1800	1204.5\\
2400	1009.1\\
3000	923.43\\
3600	904.29\\
4200	910.69\\
4800	862.64\\
5400	881.64\\
6000	897.86\\
6600	877.1\\
7200	885.27\\
7800	881.33\\
8400	881.5\\
9000	891.52\\
9600	877.15\\
10200	903.61\\
10800	885.42\\
11400	908.91\\
12600	894.51\\
13200	875.88\\
13800	853.96\\
14400	859.05\\
15600	881.2\\
16200	877.51\\
16800	879.42\\
17400	863.41\\
18000	873.65\\
18600	906.09\\
19200	881.9\\
19800	854.71\\
20400	879.59\\
21000	883.13\\
21600	876.59\\
22200	891.46\\
23400	853.78\\
24000	895.89\\
24600	885.96\\
25200	908.02\\
25800	881.38\\
26400	883.08\\
27000	891.68\\
28200	881.11\\
28800	873.4\\
29400	890.65\\
30000	876.04\\
};
\addplot [color=mycolor93, forget plot]
  table[row sep=crcr]{%
0	4395.3\\
600	2999.4\\
1200	1535.7\\
1800	1066.2\\
2400	893.76\\
3000	820.13\\
3600	757.98\\
4200	720.23\\
5400	738.99\\
6000	739.12\\
6600	772.62\\
7200	757.66\\
7800	752.45\\
8400	769.15\\
9000	750.59\\
9600	725.69\\
10200	744.04\\
12000	752.85\\
12600	743.97\\
13200	721.77\\
13800	730.91\\
14400	723.91\\
15000	742.73\\
15600	741\\
16200	752.45\\
16800	751.71\\
17400	735.84\\
18000	776.01\\
18600	808.79\\
19200	749.28\\
19800	783.79\\
20400	752.64\\
21000	769.39\\
21600	767.26\\
22200	743.76\\
22800	717.58\\
23400	703.03\\
24000	752.51\\
24600	759.73\\
25200	784\\
25800	779\\
26400	769.4\\
27000	741.64\\
27600	744.73\\
28200	770.24\\
28800	766.02\\
29400	727.09\\
30000	708.29\\
};
\addplot [color=mycolor94, forget plot]
  table[row sep=crcr]{%
0	4395.3\\
600	3041.7\\
1200	1606\\
1800	1084\\
2400	853.6\\
3000	745.42\\
3600	691.94\\
4800	673.66\\
5400	680.82\\
6000	708.87\\
7200	702.23\\
7800	715.93\\
8400	715.44\\
9000	718.78\\
9600	707.51\\
10200	717.37\\
10800	723.79\\
12000	706.53\\
12600	711.05\\
13800	707.58\\
14400	712.18\\
15000	731.84\\
15600	705.43\\
16200	712.54\\
16800	686.28\\
17400	697.41\\
18000	702.53\\
18600	701.2\\
19200	681.72\\
19800	677.27\\
20400	695.84\\
21000	700.77\\
21600	709.33\\
22800	697.97\\
23400	714.26\\
24000	739.96\\
24600	731.44\\
25200	714.49\\
25800	714.35\\
26400	718.36\\
27000	707.07\\
27600	693.91\\
28200	695.88\\
29400	705.64\\
30000	720.22\\
};
\addplot [color=mycolor95, forget plot]
  table[row sep=crcr]{%
0	4395.3\\
600	2749.8\\
1200	1282\\
1800	761.81\\
2400	556.66\\
3000	480.74\\
3600	441.78\\
4200	433.86\\
4800	423.79\\
5400	407.48\\
6000	406.98\\
6600	411.09\\
7200	417.12\\
7800	418.69\\
8400	416.3\\
9600	437.35\\
10200	425.84\\
10800	428.61\\
11400	425.41\\
12000	429.46\\
12600	424.34\\
13200	429.71\\
13800	396.1\\
14400	396.9\\
15000	409.08\\
15600	424.03\\
16200	419.44\\
16800	410.23\\
17400	390.05\\
18000	403.46\\
18600	412.55\\
19200	427.44\\
19800	440.56\\
20400	429.49\\
21000	416.8\\
21600	407.44\\
22200	424.81\\
22800	422.51\\
23400	432.06\\
24000	429.61\\
24600	416.47\\
25200	421.27\\
25800	417.34\\
26400	418.67\\
27000	426.4\\
27600	412.45\\
28200	415.72\\
28800	410.6\\
29400	427.46\\
30000	422.35\\
};
\addplot [color=mycolor96, forget plot]
  table[row sep=crcr]{%
0	4395.3\\
600	2931.5\\
1200	1427.9\\
1800	870.22\\
2400	635.11\\
3000	539.24\\
3600	477.02\\
4200	431.72\\
4800	422.71\\
5400	446.44\\
6000	439.18\\
6600	454.33\\
7200	442.58\\
7800	449.83\\
8400	414.19\\
9000	415.36\\
9600	425.8\\
10200	458.46\\
10800	448.29\\
12000	462.67\\
12600	454.65\\
13200	468.29\\
13800	446.36\\
14400	475.33\\
15000	473.83\\
15600	479.27\\
16200	448.45\\
16800	449.26\\
17400	437.99\\
18000	428.93\\
18600	438.05\\
19200	439.32\\
19800	432.93\\
20400	441.43\\
21000	427.47\\
21600	423.43\\
22200	456\\
22800	478.55\\
23400	470.59\\
24000	489.11\\
24600	453.51\\
25200	454.08\\
25800	430.11\\
26400	409.15\\
27000	436.17\\
28200	455.49\\
28800	492.53\\
29400	485\\
30000	457.44\\
};
\addplot [color=mycolor97, forget plot]
  table[row sep=crcr]{%
0	4395.3\\
600	2858\\
1200	1379\\
1800	759.76\\
2400	476.28\\
3000	363.08\\
3600	319.07\\
4200	323.54\\
4800	334.86\\
5400	332.33\\
6000	332.67\\
6600	347.74\\
7200	346.5\\
7800	297.23\\
8400	286.34\\
9000	313.47\\
9600	298.48\\
10200	306.15\\
10800	291.16\\
11400	285.89\\
12000	289.98\\
12600	288.58\\
13200	318.86\\
13800	312.44\\
14400	300.32\\
15000	297.06\\
15600	291.36\\
16200	299.14\\
16800	305.48\\
17400	299.86\\
18000	301.83\\
18600	311.16\\
19200	312.54\\
19800	319.92\\
20400	323.24\\
21000	323.52\\
21600	313.46\\
22200	291.04\\
22800	304.65\\
23400	330.14\\
24000	328.82\\
24600	308.06\\
25200	289.92\\
25800	293.78\\
26400	316.33\\
27000	316.44\\
27600	292.52\\
28200	304.78\\
28800	315.99\\
29400	303.24\\
30000	311.6\\
};
\addplot [color=mycolor98]
  table[row sep=crcr]{%
0	4395.3\\
600	3121.1\\
1200	1694.7\\
1800	1020.9\\
2400	719.61\\
3000	576.4\\
3600	506.43\\
4200	472.97\\
4800	465.36\\
5400	438.71\\
6000	411.05\\
6600	401.09\\
7200	405.07\\
7800	400.23\\
8400	401.19\\
9000	397.68\\
9600	416.89\\
10200	423.01\\
10800	425.79\\
11400	437.5\\
12000	447.08\\
12600	428.26\\
13200	417.98\\
13800	420.35\\
14400	414.45\\
15000	419.58\\
15600	429.73\\
16800	423.14\\
17400	412.51\\
18000	423.26\\
18600	432\\
19200	438.88\\
19800	440.05\\
20400	432.52\\
21000	409.73\\
21600	411.54\\
22200	398.58\\
22800	398.29\\
23400	404.73\\
24000	395.25\\
24600	397.8\\
25200	396.78\\
25800	397.36\\
26400	388.64\\
27000	403.72\\
27600	389.45\\
28200	405.08\\
30000	431.57\\
};
\addlegendentry{Classe k=10}

            % \input{../../../.tex/20/KS/SizeEvolutionsfig2.tex}
        \end{groupplot}
    %\endgroup

    %\begingroup Parametri
        \newcommand\betaExact{\num{1.25+-0.01}}
        \newcommand\betaAprx{\num{1.27+-0.01}}
        
        \newcommand\xiExact{\num{.70+-.04}}
        \newcommand\xiAprx{\num{.67+-.04}}

        \newcommand\betaReal{\num{1.27}}
        \newcommand\xiReal{.93}

        \definecolor{blueish}{rgb}{0.00000,0.44700,0.74100}%
        \definecolor{reddish}{rgb}{0.83529,0.36863,0.00000}%

        \tikzset{
            Parametri/.style={
                align=center,
                font=\dimTesto{\ParameterSize},
                anchor=north east
            }
        }
        \newcommand\nodeAnchor{north east}

        \newcommand\xShift{-1em}
        \newcommand\yShift{-1em}
        \newcommand\yLineShift{-0.5em}


        \coordinate (EA) at (Row2 c1r1.\nodeAnchor);
        \coordinate (RE) at (Row2 c2r1.\nodeAnchor);
        \coordinate (RA) at (Row2 c3r1.\nodeAnchor);

        \node[Parametri] at ([shift={(\xShift,\yShift)}]EA) {
            \(\begin{aligned}
                \textcolor{blueish}{\beta}&\textcolor{blueish}{=\betaExact}\\[\yLineShift]
                \textcolor{reddish}{\beta}&\textcolor{reddish}{=\betaAprx}
            \end{aligned}\)
        };
        \node[Parametri] at ([shift={(\xShift,\yShift)}]RE) {
            \(\begin{aligned}
                \textcolor{blueish}{\beta}&\textcolor{blueish}{=\betaExact}\\[\yLineShift]
                \textcolor{black}{\beta}&\textcolor{black}{=\betaReal}
            \end{aligned}\)
        };
        \node[Parametri] at ([shift={(\xShift,\yShift)}]RA) {
            \(\begin{aligned}
                \textcolor{reddish}{\beta}&\textcolor{reddish}{=\betaAprx}\\[\yLineShift]
                \textcolor{black}{\beta}&\textcolor{black}{=\betaReal}
            \end{aligned}\)
        };


        \coordinate (R) at (Row3 c1r1.\nodeAnchor);
        \coordinate (E) at (Row3 c2r1.\nodeAnchor);
        \coordinate (A) at (Row3 c3r1.\nodeAnchor);

        \node[Parametri] at ([shift={(\xShift,\yShift)}]R) {
            \textcolor{black}{
                \(\begin{aligned}
                    \beta&=\betaReal\\[\yLineShift]
                    \xi&=\xiReal
                \end{aligned}\)
            }
        };

        \node[Parametri] at ([shift={(\xShift,\yShift)}]E) {
            \textcolor{blueish}{
                \(\begin{aligned}
                    \beta&=\betaExact\\[\yLineShift]
                    \xi&=\xiExact
                \end{aligned}\)
            }
        };

        \node[Parametri] at ([shift={(\xShift,\yShift)}]A) {
            \textcolor{reddish}{
                \(\begin{aligned}
                    \beta&=\betaAprx\\[\yLineShift]
                    \xi&=\xiAprx
                \end{aligned}\)
            }
        };
    %\endgroup

    %\begingroup Didascalie
        \tikzset{
            SubCaption/.style={
                % text width=\plotWidth cm,
                align=center,
                font=\dimTesto{\subCaptionSize},
            },
            SubCaptionNW/.style={SubCaption,anchor=north west},
            SubCaptionNE/.style={SubCaption,anchor=north east},
            SubCaptionSW/.style={SubCaption,anchor=south west},
            SubCaptionSE/.style={SubCaption,anchor=south east},
        }

        \node[SubCaptionSE] at (Row1 c1r1.south east) {\sottoDidascalia[]{}};
        \node[SubCaptionSE] at (Row1 c2r1.south east) {\sottoDidascalia[]{}};
        \node[SubCaptionSE] at (Row1 c3r1.south east) {\sottoDidascalia[]{}};
    
        \node[SubCaptionNE] at (Row2 c1r1.north east) {\sottoDidascalia[]{}};
        \node[SubCaptionNE] at (Row2 c2r1.north east) {\sottoDidascalia[]{}};
        \node[SubCaptionNE] at (Row2 c3r1.north east) {\sottoDidascalia[]{}};

        \node[SubCaptionNE] at (Row3 c1r1.north east) {\sottoDidascalia[]{}};
        \node[SubCaptionNE] at (Row3 c2r1.north east) {\sottoDidascalia[]{}};
        \node[SubCaptionNE] at (Row3 c3r1.north east) {\sottoDidascalia[]{}};

        \node[SubCaptionSE] at (Row4 c1r1.south east) {\sottoDidascalia[]{}};
        \node[SubCaptionSE] at (Row4 c2r1.south east) {\sottoDidascalia[]{}};
        \node[SubCaptionSE] at (Row4 c3r1.south east) {\sottoDidascalia[]{}};
    
        \node[SubCaptionSW] at (Row5 c1r1.south west) {\sottoDidascalia[]{}};
        \node[SubCaptionSW] at (Row5 c2r1.south west) {\sottoDidascalia[]{}};
    
        % https://tex.stackexchange.com/questions/632045/get-subcaption-for-every-plot-using-groupplot-pgf-tikz-after-2020-update
    %\endgroup

    \redefineTikZbounds{Row1 c1r1}{Row4 c3r1}
\end{tikzpicture}%

%\begingroup Vecchia implementazione a due colonne
    %\begingroup Comandi
        % % \renewcommand\titleSize{12}
        % \renewcommand\subCaptionSize{9}
        % \renewcommand\labelSize{8}
        % \renewcommand\tickSize{7}
        % \renewcommand\legendSize{6}
    
        % \pgfmathsetmacro\plotWidth{7}
        % \pgfmathsetmacro\plotHeightI{.618*\plotWidth}
        % \pgfmathsetmacro\plotHeightII{.5*\plotWidth}
        % % \pgfmathsetmacro\halfHeightPlot{\plotHeightI/2}
        % % \pgfmathsetmacro\halfWidthPlot{\plotWidth/2}
    
        % \pgfmathsetmacro\xPlotSep{2}
        % \pgfmathsetmacro\yPlotSep{0.6}
        % \pgfmathsetmacro\yGroupSep{2.5}
        % % \pgfmathsetmacro\halfxPlotSep{\xPlotSep/2}
    %\endgroup
    
    %\begingroup Impostazioni pgfplots
        % %\begingroup Gruppi
        % \pgfplotsset{
        %     /pgfplots/group/plotGroup/.style={
        %         horizontal sep=\xPlotSep em,
        %         vertical sep=\yPlotSep em,
        %         x descriptions at=edge bottom,
        %     },
        %     /pgfplots/group/plotLeftGroup/.style={
        %         plotGroup,
        %         group size=1 by 3,
        %         y descriptions at=edge left,
        %     },
        %     /pgfplots/group/plotRightGroup/.style={
        %         plotGroup,
        %         group size=1 by 3,
        %         y descriptions at=edge right,
        %     },
        % }
        % %\endgroup
    
        % %\begingroup Stili
        % \pgfplotsset{
        %     plotShared/.style={
        %         width=\plotWidth cm,
        %         scale only axis,
        %         xlabel style={font=\color{white!15!black}\dimTesto{\labelSize}},
        %         ylabel style={font=\color{white!15!black}\dimTesto{\labelSize}},
        %         xticklabel style={font=\dimTesto{\tickSize}},
        %         yticklabel style={font=\dimTesto{\tickSize}},
        %         xlabel={\(s\)},ylabel={\(\underhat{\bar f}_N(s)\)},
        %         axis x line*=bottom,
        %         xmajorgrids,ymajorgrids,
        %         xminorticks=true,yminorticks=true,
        %         axis background/.style={fill=white},
        %         grid style={dashed}
        %     },
        %     plotTop/.style={
        %         height=\plotHeightI cm,
        %     },
        %     plotBottom/.style={
        %         height=\plotHeightII cm,
        %     },
        %     %
        %     plotLeftCol/.style={
        %         plotShared,
        %         axis y line*=left,
        %         y tick scale label style={
        %             xshift=-.4cm,
        %             at={(0,1)},
        %             anchor=south east,
        %         },
        %         xlabel={\(s\)},
        %         ylabel={\(\underhat{\bar f}_N(s)\)},
        %     },
        %     %
        %     plotRightCol/.style={
        %         plotShared,
        %         axis y line*=right,
        %         y tick scale label style={
        %             xshift=.4cm,
        %             at={(1,1)},
        %             anchor=south west,
        %         },
        %         enlarge x limits=0.1,
        %         enlarge y limits=0.1,
        %     },
        %     %
        %     plotRow/.style={
        %         plotShared,
        %         ylabel={\(\underhat{\bar f}_N(s)\)},
        %         y tick scale label style={
        %             xshift=.4cm,
        %             at={(1,1)},
        %             anchor=south west,
        %         },
        %     },
        % }
        % %\endgroup
    
        % %\begingroup Legende
        % \pgfplotsset{
        %     plotLegend/.style={
        %         font=\dimTesto{\legendSize},
        %         legend cell align=left,
        %         align=left,
        %         % draw=white!15!black,
        %         draw=none, % Bordo invisibile
        %         fill=white,
        %         fill opacity=0.5,
        %         text opacity=1,
        %     },
        %     plotLegendNE/.style={legend style={plotLegend},legend pos=north east},
        %     plotLegendSW/.style={legend style={plotLegend},legend pos=south west},
        %     plotLegendNW/.style={legend style={plotLegend},legend pos=north west},
        % }
        % %\endgroup
    %\endgroup
    
    %\begingroup 2 Colonne
        % % Left column
        % \begin{groupplot}[
        %     group style={
        %         group name=ColL1,
        %         plotLeftGroup
        %     },
        %     plotLeftCol,
        %     plotTop,
        %     plotLegendNE,
        %     %
        %     enlarge x limits=0.01,
        %     xmode=log,
        %     xmin=78.6017269745556,xmax=1000000,
        % ]
        %     \nextgroupplot[%
        %         % xmin=78.6017269745556,xmax=1000000,
        %         ymin=0,ymax=0.0004,
        %     ] % This file was created by matlab2tikz.
%
\definecolor{mycolor1}{rgb}{0.00000,0.44706,0.69804}%
\definecolor{mycolor2}{rgb}{0.83529,0.36863,0.00000}%
%
\addplot[ybar interval, fill=mycolor1, fill opacity=0.35, area legend, draw=none] table[row sep=crcr, x=Lower, y=Count] {%
Lower	Upper	Count\\
27.667	39.83	2.1925e-06\\
39.83	57.339	7.6151e-06\\
57.339	82.545	2.4333e-05\\
82.545	118.83	3.8949e-05\\
118.83	171.07	8.3719e-05\\
171.07	246.27	0.000139\\
246.27	354.53	0.00020937\\
354.53	510.38	0.00027496\\
510.38	734.75	0.0003303\\
734.75	1057.7	0.00032909\\
1057.7	1522.7	0.00028549\\
1522.7	2192.1	0.00021791\\
2192.1	3155.8	0.00013787\\
3155.8	4543	7.8543e-05\\
4543	6540.1	3.7174e-05\\
6540.1	9415.1	1.7271e-05\\
9415.1	13554	8.5112e-06\\
13554	19512	3.934e-06\\
19512	28090	1.7783e-06\\
28090	40438	6.8458e-07\\
40438	58215	2.6702e-07\\
58215	83806	1.0212e-07\\
83806	1.2065e+05	1.5201e-08\\
1.2065e+05	1.7368e+05	3.9722e-08\\
1.7368e+05	2.5003e+05	4.5405e-09\\
2.5003e+05	2.5003e+05	4.5405e-09\\
};
\addlegendentry{Istogramma esatto}

\addplot [color=mycolor1, only marks, every error bar/.append style={opacity=0.45}, mark=*, mark size=0pt, draw=none, forget plot]
 plot [error bars/.cd, y dir=both, y explicit, error bar style={line width=1pt, color=mycolor1}, error mark options={mark=none,mark size=0pt}]
 table[row sep=crcr, y error plus index=2, y error minus index=3]{%
33.196	2.1925e-06	4.3504e-06	2.1925e-06\\
47.789	7.6151e-06	6.6194e-06	6.6194e-06\\
68.797	2.4333e-05	9.8302e-06	9.8302e-06\\
99.041	3.8949e-05	1.1624e-05	1.1624e-05\\
142.58	8.3719e-05	1.1107e-05	1.1107e-05\\
205.26	0.000139	1.2478e-05	1.2478e-05\\
295.49	0.00020937	1.4397e-05	1.4397e-05\\
425.38	0.00027496	1.2896e-05	1.2896e-05\\
612.38	0.0003303	1.3472e-05	1.3472e-05\\
881.58	0.00032909	1.0604e-05	1.0604e-05\\
1269.1	0.00028549	7.4112e-06	7.4112e-06\\
1827	0.00021791	4.8252e-06	4.8252e-06\\
2630.2	0.00013787	3.6233e-06	3.6233e-06\\
3786.4	7.8543e-05	2.1968e-06	2.1968e-06\\
5450.9	3.7174e-05	1.306e-06	1.306e-06\\
7847	1.7271e-05	6.401e-07	6.401e-07\\
11297	8.5112e-06	4.1607e-07	4.1607e-07\\
16263	3.934e-06	2.0325e-07	2.0325e-07\\
23411	1.7783e-06	1.1733e-07	1.1733e-07\\
33703	6.8458e-07	6.3602e-08	6.3602e-08\\
48519	2.6702e-07	3.0968e-08	3.0968e-08\\
69848	1.0212e-07	1.5545e-08	1.5545e-08\\
1.0055e+05	1.5201e-08	6.8607e-09	6.8607e-09\\
1.4476e+05	3.9722e-08	4.0841e-09	4.0841e-09\\
2.0839e+05	4.5405e-09	2.3424e-09	2.3424e-09\\
};
\addplot [color=mycolor1]
  table[row sep=crcr]{%
27.667	5.7757e-06\\
31.435	7.1447e-06\\
35.391	8.6843e-06\\
39.483	1.0382e-05\\
43.648	1.2217e-05\\
47.815	1.4156e-05\\
51.905	1.6158e-05\\
56.344	1.8439e-05\\
60.608	2.0732e-05\\
65.194	2.3307e-05\\
69.491	2.5817e-05\\
74.071	2.8594e-05\\
78.953	3.1662e-05\\
83.393	3.4544e-05\\
88.082	3.7679e-05\\
93.035	4.1085e-05\\
98.267	4.4781e-05\\
103.79	4.8786e-05\\
108.63	5.2373e-05\\
113.7	5.6199e-05\\
119.01	6.0274e-05\\
124.56	6.4608e-05\\
130.37	6.9211e-05\\
136.45	7.4092e-05\\
142.81	7.9257e-05\\
149.47	8.4714e-05\\
156.45	9.0467e-05\\
163.74	9.6521e-05\\
171.38	0.00010288\\
179.38	0.00010954\\
187.74	0.00011649\\
196.5	0.00012375\\
205.67	0.00013129\\
215.26	0.00013911\\
225.3	0.00014719\\
235.81	0.00015552\\
249.07	0.00016582\\
263.08	0.00017641\\
280.42	0.00018907\\
301.64	0.00020383\\
382.34	0.00025209\\
403.84	0.00026281\\
422.67	0.0002715\\
442.39	0.00027989\\
458.82	0.00028636\\
475.87	0.00029258\\
493.54	0.00029852\\
511.88	0.00030416\\
530.89	0.00030946\\
545.61	0.0003132\\
560.74	0.00031673\\
576.29	0.00032004\\
592.28	0.00032311\\
608.7	0.00032594\\
625.58	0.00032852\\
642.93	0.00033084\\
660.76	0.00033288\\
679.09	0.00033465\\
697.92	0.00033614\\
717.27	0.00033734\\
737.16	0.00033824\\
757.61	0.00033884\\
778.62	0.00033915\\
800.21	0.00033914\\
822.4	0.00033883\\
845.21	0.00033821\\
868.65	0.00033729\\
892.74	0.00033606\\
917.5	0.00033452\\
942.94	0.00033268\\
969.09	0.00033055\\
995.96	0.00032812\\
1023.6	0.00032541\\
1052	0.00032242\\
1081.1	0.00031915\\
1111.1	0.00031563\\
1141.9	0.00031184\\
1173.6	0.00030781\\
1206.2	0.00030355\\
1239.6	0.00029906\\
1274	0.00029436\\
1321.3	0.00028778\\
1370.4	0.00028088\\
1421.3	0.00027369\\
1474.1	0.00026624\\
1542.8	0.0002566\\
1614.8	0.00024668\\
1705.6	0.00023448\\
1834.7	0.00021789\\
2142.3	0.00018255\\
2262.8	0.00017036\\
2390	0.00015845\\
2501.5	0.00014879\\
2618.2	0.00013941\\
2740.3	0.00013034\\
2842.1	0.00012333\\
2947.7	0.00011654\\
3057.2	0.00010998\\
3170.8	0.00010367\\
3288.5	9.7602e-05\\
3410.7	9.1784e-05\\
3537.4	8.6216e-05\\
3668.8	8.0899e-05\\
3805.1	7.5832e-05\\
3946.4	7.1011e-05\\
4093	6.6434e-05\\
4245	6.2096e-05\\
4402.7	5.7991e-05\\
4566.3	5.4113e-05\\
4735.9	5.0455e-05\\
4956.8	4.618e-05\\
5188	4.2222e-05\\
5430	3.8566e-05\\
5683.3	3.5194e-05\\
5948.4	3.209e-05\\
6225.9	2.9238e-05\\
6516.3	2.6622e-05\\
6882.7	2.377e-05\\
7269.8	2.1207e-05\\
7678.6	1.8909e-05\\
8110.4	1.685e-05\\
8644.9	1.4721e-05\\
9214.7	1.2854e-05\\
9912	1.1004e-05\\
10662	9.4154e-06\\
11574	7.8976e-06\\
12679	6.4932e-06\\
14017	5.2318e-06\\
15637	4.1294e-06\\
17766	3.1276e-06\\
20557	2.2689e-06\\
24446	1.5395e-06\\
29876	9.7051e-07\\
38565	5.2623e-07\\
54040	2.224e-07\\
90874	5.1572e-08\\
2.5003e+05	1.842e-09\\
};
\addlegendentry{Adatt. BLN esatto ($\mathit{ML}$)}


\addplot[area legend, draw=none, fill=mycolor1, fill opacity=0.15, forget plot]
table[row sep=crcr] {%
x	y\\
27.667	7.3893e-06\\
27.921	7.4978e-06\\
28.177	7.6077e-06\\
28.435	7.7191e-06\\
28.695	7.832e-06\\
28.958	7.9463e-06\\
29.223	8.0622e-06\\
29.491	8.1796e-06\\
29.761	8.2986e-06\\
30.034	8.4191e-06\\
30.309	8.5412e-06\\
30.586	8.6649e-06\\
30.867	8.7902e-06\\
31.149	8.9172e-06\\
31.435	9.0459e-06\\
31.723	9.1762e-06\\
32.013	9.3083e-06\\
32.306	9.4421e-06\\
32.602	9.5776e-06\\
32.901	9.715e-06\\
33.202	9.8541e-06\\
33.506	9.995e-06\\
33.813	1.0138e-05\\
34.123	1.0283e-05\\
34.436	1.0429e-05\\
34.751	1.0578e-05\\
35.069	1.0728e-05\\
35.391	1.088e-05\\
35.715	1.1035e-05\\
36.042	1.1191e-05\\
36.372	1.135e-05\\
36.705	1.151e-05\\
37.042	1.1673e-05\\
37.381	1.1838e-05\\
37.723	1.2005e-05\\
38.069	1.2174e-05\\
38.417	1.2345e-05\\
38.769	1.2518e-05\\
39.124	1.2694e-05\\
39.483	1.2872e-05\\
39.845	1.3053e-05\\
40.209	1.3236e-05\\
40.578	1.3421e-05\\
40.949	1.3608e-05\\
41.325	1.3799e-05\\
41.703	1.3991e-05\\
42.085	1.4186e-05\\
42.471	1.4384e-05\\
42.86	1.4584e-05\\
43.252	1.4787e-05\\
43.648	1.4992e-05\\
44.048	1.5201e-05\\
44.452	1.5412e-05\\
44.859	1.5625e-05\\
45.27	1.5842e-05\\
45.684	1.6061e-05\\
46.103	1.6283e-05\\
46.525	1.6508e-05\\
46.951	1.6737e-05\\
47.381	1.6968e-05\\
47.815	1.7202e-05\\
48.253	1.7439e-05\\
48.695	1.7679e-05\\
49.141	1.7923e-05\\
49.592	1.8169e-05\\
50.046	1.8419e-05\\
50.504	1.8673e-05\\
50.967	1.8929e-05\\
51.434	1.9189e-05\\
51.905	1.9452e-05\\
52.38	1.9719e-05\\
52.86	1.9989e-05\\
53.344	2.0263e-05\\
53.833	2.0541e-05\\
54.326	2.0822e-05\\
54.824	2.1107e-05\\
55.326	2.1395e-05\\
55.833	2.1688e-05\\
56.344	2.1984e-05\\
56.86	2.2284e-05\\
57.381	2.2588e-05\\
57.907	2.2896e-05\\
58.437	2.3209e-05\\
58.972	2.3525e-05\\
59.512	2.3846e-05\\
60.058	2.417e-05\\
60.608	2.4499e-05\\
61.163	2.4833e-05\\
61.723	2.517e-05\\
62.289	2.5513e-05\\
62.859	2.5859e-05\\
63.435	2.6211e-05\\
64.016	2.6567e-05\\
64.602	2.6928e-05\\
65.194	2.7293e-05\\
65.791	2.7663e-05\\
66.394	2.8038e-05\\
67.002	2.8419e-05\\
67.616	2.8804e-05\\
68.235	2.9194e-05\\
68.86	2.9589e-05\\
69.491	2.999e-05\\
70.127	3.0396e-05\\
70.77	3.0807e-05\\
71.418	3.1223e-05\\
72.072	3.1646e-05\\
72.732	3.2073e-05\\
73.399	3.2506e-05\\
74.071	3.2945e-05\\
74.749	3.339e-05\\
75.434	3.3841e-05\\
76.125	3.4297e-05\\
76.822	3.476e-05\\
77.526	3.5228e-05\\
78.236	3.5703e-05\\
78.953	3.6184e-05\\
79.676	3.6671e-05\\
80.406	3.7164e-05\\
81.142	3.7664e-05\\
81.886	3.8171e-05\\
82.636	3.8684e-05\\
83.393	3.9203e-05\\
84.156	3.973e-05\\
84.927	4.0263e-05\\
85.705	4.0803e-05\\
86.49	4.135e-05\\
87.283	4.1904e-05\\
88.082	4.2466e-05\\
88.889	4.3034e-05\\
89.703	4.361e-05\\
90.525	4.4193e-05\\
91.354	4.4784e-05\\
92.191	4.5382e-05\\
93.035	4.5988e-05\\
93.887	4.6601e-05\\
94.747	4.7223e-05\\
95.615	4.7852e-05\\
96.491	4.8489e-05\\
97.375	4.9134e-05\\
98.267	4.9788e-05\\
99.167	5.0449e-05\\
100.08	5.1119e-05\\
100.99	5.1798e-05\\
101.92	5.2484e-05\\
102.85	5.318e-05\\
103.79	5.3883e-05\\
104.74	5.4596e-05\\
105.7	5.5318e-05\\
106.67	5.6048e-05\\
107.65	5.6787e-05\\
108.63	5.7535e-05\\
109.63	5.8293e-05\\
110.63	5.906e-05\\
111.65	5.9836e-05\\
112.67	6.0621e-05\\
113.7	6.1416e-05\\
114.74	6.222e-05\\
115.79	6.3034e-05\\
116.85	6.3858e-05\\
117.93	6.4691e-05\\
119.01	6.5535e-05\\
120.1	6.6388e-05\\
121.2	6.7251e-05\\
122.31	6.8124e-05\\
123.43	6.9008e-05\\
124.56	6.9902e-05\\
125.7	7.0806e-05\\
126.85	7.172e-05\\
128.01	7.2645e-05\\
129.18	7.3581e-05\\
130.37	7.4527e-05\\
131.56	7.5484e-05\\
132.77	7.6451e-05\\
133.98	7.743e-05\\
135.21	7.8419e-05\\
136.45	7.9419e-05\\
137.7	8.043e-05\\
138.96	8.1452e-05\\
140.23	8.2486e-05\\
141.52	8.353e-05\\
142.81	8.4586e-05\\
144.12	8.5653e-05\\
145.44	8.6731e-05\\
146.77	8.7821e-05\\
148.12	8.8922e-05\\
149.47	9.0035e-05\\
150.84	9.1159e-05\\
152.23	9.2295e-05\\
153.62	9.3442e-05\\
155.03	9.4601e-05\\
156.45	9.5772e-05\\
157.88	9.6955e-05\\
159.33	9.8149e-05\\
160.79	9.9355e-05\\
162.26	0.00010057\\
163.74	0.0001018\\
165.24	0.00010304\\
166.76	0.0001043\\
168.29	0.00010556\\
169.83	0.00010684\\
171.38	0.00010813\\
172.95	0.00010943\\
174.54	0.00011074\\
176.14	0.00011207\\
177.75	0.0001134\\
179.38	0.00011475\\
181.02	0.00011611\\
182.68	0.00011749\\
184.35	0.00011887\\
186.04	0.00012027\\
187.74	0.00012167\\
189.46	0.00012309\\
191.2	0.00012453\\
192.95	0.00012597\\
194.72	0.00012742\\
196.5	0.00012889\\
198.3	0.00013037\\
200.12	0.00013186\\
201.95	0.00013336\\
203.8	0.00013487\\
205.67	0.00013639\\
207.55	0.00013793\\
209.45	0.00013947\\
211.37	0.00014103\\
213.31	0.0001426\\
215.26	0.00014418\\
217.23	0.00014577\\
219.22	0.00014737\\
221.23	0.00014898\\
223.26	0.0001506\\
225.3	0.00015223\\
227.37	0.00015388\\
229.45	0.00015553\\
231.55	0.00015719\\
233.67	0.00015886\\
235.81	0.00016054\\
237.97	0.00016223\\
240.15	0.00016393\\
242.35	0.00016564\\
244.57	0.00016736\\
246.81	0.00016909\\
249.07	0.00017083\\
251.35	0.00017257\\
253.66	0.00017432\\
255.98	0.00017608\\
258.32	0.00017785\\
260.69	0.00017963\\
263.08	0.00018141\\
265.49	0.0001832\\
267.92	0.000185\\
270.37	0.0001868\\
272.85	0.00018861\\
275.35	0.00019043\\
277.87	0.00019225\\
280.42	0.00019408\\
282.99	0.00019591\\
285.58	0.00019775\\
288.19	0.00019959\\
290.83	0.00020144\\
293.5	0.00020329\\
296.19	0.00020515\\
298.9	0.00020701\\
301.64	0.00020887\\
304.4	0.00021074\\
307.19	0.0002126\\
310	0.00021448\\
312.84	0.00021635\\
315.71	0.00021822\\
318.6	0.0002201\\
321.52	0.00022197\\
324.46	0.00022385\\
327.44	0.00022572\\
330.43	0.0002276\\
333.46	0.00022948\\
336.52	0.00023135\\
339.6	0.00023322\\
342.71	0.00023509\\
345.85	0.00023696\\
349.02	0.00023883\\
352.21	0.00024069\\
355.44	0.00024255\\
358.7	0.00024441\\
361.98	0.00024626\\
365.3	0.00024811\\
368.64	0.00024995\\
372.02	0.00025179\\
375.43	0.00025362\\
378.87	0.00025544\\
382.34	0.00025726\\
385.84	0.00025907\\
389.37	0.00026087\\
392.94	0.00026267\\
396.54	0.00026445\\
400.17	0.00026623\\
403.84	0.000268\\
407.54	0.00026976\\
411.27	0.0002715\\
415.04	0.00027324\\
418.84	0.00027497\\
422.67	0.00027668\\
426.55	0.00027838\\
430.45	0.00028007\\
434.4	0.00028175\\
438.37	0.00028341\\
442.39	0.00028506\\
446.44	0.00028669\\
450.53	0.00028831\\
454.66	0.00028992\\
458.82	0.00029151\\
463.03	0.00029308\\
467.27	0.00029464\\
471.55	0.00029617\\
475.87	0.0002977\\
480.23	0.0002992\\
484.62	0.00030068\\
489.06	0.00030215\\
493.54	0.0003036\\
498.06	0.00030503\\
502.63	0.00030643\\
507.23	0.00030782\\
511.88	0.00030919\\
516.56	0.00031053\\
521.3	0.00031185\\
526.07	0.00031316\\
530.89	0.00031443\\
535.75	0.00031569\\
540.66	0.00031692\\
545.61	0.00031813\\
550.61	0.00031931\\
555.65	0.00032047\\
560.74	0.00032161\\
565.88	0.00032272\\
571.06	0.0003238\\
576.29	0.00032486\\
581.57	0.00032589\\
586.9	0.0003269\\
592.28	0.00032788\\
597.7	0.00032883\\
603.18	0.00032975\\
608.7	0.00033065\\
614.28	0.00033151\\
619.9	0.00033235\\
625.58	0.00033316\\
631.31	0.00033394\\
637.1	0.00033469\\
642.93	0.00033542\\
648.82	0.00033611\\
654.76	0.00033677\\
660.76	0.0003374\\
666.81	0.000338\\
672.92	0.00033857\\
679.09	0.0003391\\
685.31	0.00033961\\
691.58	0.00034008\\
697.92	0.00034052\\
704.31	0.00034093\\
710.76	0.00034131\\
717.27	0.00034165\\
723.84	0.00034197\\
730.47	0.00034224\\
737.16	0.00034249\\
743.92	0.0003427\\
750.73	0.00034288\\
757.61	0.00034303\\
764.55	0.00034314\\
771.55	0.00034322\\
778.62	0.00034326\\
785.75	0.00034327\\
792.95	0.00034325\\
800.21	0.00034319\\
807.54	0.0003431\\
814.94	0.00034297\\
822.4	0.00034281\\
829.94	0.00034261\\
837.54	0.00034239\\
845.21	0.00034212\\
852.95	0.00034182\\
860.76	0.00034149\\
868.65	0.00034112\\
876.61	0.00034072\\
884.64	0.00034029\\
892.74	0.00033982\\
900.92	0.00033932\\
909.17	0.00033878\\
917.5	0.00033821\\
925.9	0.0003376\\
934.38	0.00033697\\
942.94	0.00033629\\
951.58	0.00033559\\
960.29	0.00033485\\
969.09	0.00033408\\
977.97	0.00033328\\
986.92	0.00033244\\
995.96	0.00033157\\
1005.1	0.00033067\\
1014.3	0.00032974\\
1023.6	0.00032877\\
1033	0.00032778\\
1042.4	0.00032675\\
1052	0.00032569\\
1061.6	0.00032461\\
1071.3	0.00032349\\
1081.1	0.00032234\\
1091	0.00032116\\
1101	0.00031996\\
1111.1	0.00031872\\
1121.3	0.00031746\\
1131.6	0.00031617\\
1141.9	0.00031485\\
1152.4	0.0003135\\
1163	0.00031213\\
1173.6	0.00031073\\
1184.4	0.00030931\\
1195.2	0.00030786\\
1206.2	0.00030638\\
1217.2	0.00030488\\
1228.4	0.00030336\\
1239.6	0.00030181\\
1251	0.00030024\\
1262.4	0.00029864\\
1274	0.00029702\\
1285.7	0.00029539\\
1297.4	0.00029373\\
1309.3	0.00029205\\
1321.3	0.00029035\\
1333.4	0.00028862\\
1345.6	0.00028688\\
1357.9	0.00028513\\
1370.4	0.00028335\\
1382.9	0.00028155\\
1395.6	0.00027974\\
1408.4	0.00027791\\
1421.3	0.00027607\\
1434.3	0.00027421\\
1447.4	0.00027233\\
1460.7	0.00027044\\
1474.1	0.00026853\\
1487.6	0.00026661\\
1501.2	0.00026468\\
1515	0.00026274\\
1528.8	0.00026078\\
1542.8	0.00025881\\
1557	0.00025683\\
1571.2	0.00025484\\
1585.6	0.00025284\\
1600.2	0.00025083\\
1614.8	0.00024881\\
1629.6	0.00024678\\
1644.5	0.00024474\\
1659.6	0.0002427\\
1674.8	0.00024065\\
1690.1	0.00023859\\
1705.6	0.00023653\\
1721.2	0.00023446\\
1737	0.00023238\\
1752.9	0.0002303\\
1769	0.00022821\\
1785.2	0.00022613\\
1801.5	0.00022403\\
1818	0.00022194\\
1834.7	0.00021984\\
1851.5	0.00021774\\
1868.5	0.00021564\\
1885.6	0.00021354\\
1902.8	0.00021144\\
1920.3	0.00020934\\
1937.9	0.00020723\\
1955.6	0.00020513\\
1973.5	0.00020303\\
1991.6	0.00020093\\
2009.8	0.00019884\\
2028.3	0.00019675\\
2046.8	0.00019465\\
2065.6	0.00019257\\
2084.5	0.00019048\\
2103.6	0.0001884\\
2122.9	0.00018633\\
2142.3	0.00018426\\
2161.9	0.0001822\\
2181.7	0.00018014\\
2201.7	0.00017809\\
2221.9	0.00017604\\
2242.2	0.000174\\
2262.8	0.00017197\\
2283.5	0.00016995\\
2304.4	0.00016793\\
2325.5	0.00016592\\
2346.8	0.00016393\\
2368.3	0.00016194\\
2390	0.00015996\\
2411.9	0.00015799\\
2434	0.00015603\\
2456.3	0.00015408\\
2478.8	0.00015214\\
2501.5	0.00015021\\
2524.4	0.0001483\\
2547.6	0.00014639\\
2570.9	0.0001445\\
2594.4	0.00014262\\
2618.2	0.00014075\\
2642.2	0.00013889\\
2666.4	0.00013705\\
2690.8	0.00013522\\
2715.5	0.0001334\\
2740.3	0.0001316\\
2765.4	0.00012981\\
2790.8	0.00012803\\
2816.3	0.00012627\\
2842.1	0.00012452\\
2868.2	0.00012278\\
2894.4	0.00012106\\
2920.9	0.00011936\\
2947.7	0.00011767\\
2974.7	0.00011599\\
3001.9	0.00011433\\
3029.4	0.00011269\\
3057.2	0.00011106\\
3085.2	0.00010944\\
3113.5	0.00010784\\
3142	0.00010626\\
3170.8	0.00010469\\
3199.8	0.00010314\\
3229.1	0.0001016\\
3258.7	0.00010008\\
3288.5	9.8574e-05\\
3318.7	9.7084e-05\\
3349.1	9.561e-05\\
3379.7	9.4152e-05\\
3410.7	9.271e-05\\
3441.9	9.1283e-05\\
3473.5	8.9873e-05\\
3505.3	8.8478e-05\\
3537.4	8.7099e-05\\
3569.8	8.5736e-05\\
3602.5	8.4389e-05\\
3635.5	8.3058e-05\\
3668.8	8.1742e-05\\
3702.4	8.0442e-05\\
3736.3	7.9158e-05\\
3770.5	7.789e-05\\
3805.1	7.6637e-05\\
3839.9	7.54e-05\\
3875.1	7.4178e-05\\
3910.6	7.2972e-05\\
3946.4	7.1781e-05\\
3982.6	7.0605e-05\\
4019	6.9444e-05\\
4055.9	6.8299e-05\\
4093	6.7169e-05\\
4130.5	6.6054e-05\\
4168.3	6.4953e-05\\
4206.5	6.3868e-05\\
4245	6.2797e-05\\
4283.9	6.1741e-05\\
4323.2	6.0699e-05\\
4362.8	5.9672e-05\\
4402.7	5.8659e-05\\
4443.1	5.766e-05\\
4483.8	5.6675e-05\\
4524.8	5.5705e-05\\
4566.3	5.4748e-05\\
4608.1	5.3805e-05\\
4650.3	5.2876e-05\\
4692.9	5.196e-05\\
4735.9	5.1058e-05\\
4779.3	5.0169e-05\\
4823.1	4.9293e-05\\
4867.2	4.8431e-05\\
4911.8	4.7581e-05\\
4956.8	4.6744e-05\\
5002.2	4.592e-05\\
5048	4.5108e-05\\
5094.3	4.4309e-05\\
5140.9	4.3523e-05\\
5188	4.2748e-05\\
5235.5	4.1986e-05\\
5283.5	4.1235e-05\\
5331.9	4.0496e-05\\
5380.7	3.9769e-05\\
5430	3.9054e-05\\
5479.8	3.835e-05\\
5530	3.7657e-05\\
5580.6	3.6976e-05\\
5631.7	3.6306e-05\\
5683.3	3.5646e-05\\
5735.4	3.4997e-05\\
5787.9	3.4359e-05\\
5840.9	3.3732e-05\\
5894.4	3.3114e-05\\
5948.4	3.2507e-05\\
6002.9	3.1911e-05\\
6057.9	3.1324e-05\\
6113.4	3.0747e-05\\
6169.4	3.018e-05\\
6225.9	2.9622e-05\\
6282.9	2.9074e-05\\
6340.5	2.8536e-05\\
6398.5	2.8006e-05\\
6457.2	2.7486e-05\\
6516.3	2.6974e-05\\
6576	2.6472e-05\\
6636.2	2.5978e-05\\
6697	2.5493e-05\\
6758.4	2.5016e-05\\
6820.3	2.4548e-05\\
6882.7	2.4087e-05\\
6945.8	2.3635e-05\\
7009.4	2.3191e-05\\
7073.6	2.2755e-05\\
7138.4	2.2327e-05\\
7203.8	2.1906e-05\\
7269.8	2.1493e-05\\
7336.4	2.1087e-05\\
7403.6	2.0689e-05\\
7471.4	2.0297e-05\\
7539.8	1.9913e-05\\
7608.9	1.9536e-05\\
7678.6	1.9165e-05\\
7748.9	1.8801e-05\\
7819.9	1.8444e-05\\
7891.5	1.8094e-05\\
7963.8	1.775e-05\\
8036.8	1.7412e-05\\
8110.4	1.708e-05\\
8184.7	1.6755e-05\\
8259.6	1.6435e-05\\
8335.3	1.6122e-05\\
8411.6	1.5814e-05\\
8488.7	1.5512e-05\\
8566.5	1.5215e-05\\
8644.9	1.4924e-05\\
8724.1	1.4639e-05\\
8804	1.4359e-05\\
8884.7	1.4084e-05\\
8966	1.3814e-05\\
9048.2	1.3549e-05\\
9131.1	1.3289e-05\\
9214.7	1.3035e-05\\
9299.1	1.2784e-05\\
9384.3	1.2539e-05\\
9470.2	1.2298e-05\\
9557	1.2062e-05\\
9644.5	1.183e-05\\
9732.9	1.1603e-05\\
9822	1.138e-05\\
9912	1.1161e-05\\
10003	1.0946e-05\\
10094	1.0736e-05\\
10187	1.0529e-05\\
10280	1.0326e-05\\
10374	1.0128e-05\\
10469	9.9326e-06\\
10565	9.7412e-06\\
10662	9.5535e-06\\
10760	9.3694e-06\\
10858	9.1888e-06\\
10958	9.0116e-06\\
11058	8.8378e-06\\
11159	8.6674e-06\\
11262	8.5002e-06\\
11365	8.3361e-06\\
11469	8.1752e-06\\
11574	8.0174e-06\\
11680	7.8626e-06\\
11787	7.7108e-06\\
11895	7.5618e-06\\
12004	7.4157e-06\\
12114	7.2724e-06\\
12225	7.1318e-06\\
12337	6.994e-06\\
12450	6.8587e-06\\
12564	6.7261e-06\\
12679	6.5959e-06\\
12795	6.4683e-06\\
12912	6.3431e-06\\
13030	6.2203e-06\\
13150	6.0998e-06\\
13270	5.9817e-06\\
13392	5.8657e-06\\
13515	5.752e-06\\
13638	5.6405e-06\\
13763	5.5311e-06\\
13889	5.4238e-06\\
14017	5.3186e-06\\
14145	5.2153e-06\\
14274	5.114e-06\\
14405	5.0147e-06\\
14537	4.9172e-06\\
14670	4.8216e-06\\
14805	4.7278e-06\\
14940	4.6358e-06\\
15077	4.5456e-06\\
15215	4.457e-06\\
15355	4.3702e-06\\
15495	4.285e-06\\
15637	4.2014e-06\\
15780	4.1194e-06\\
15925	4.039e-06\\
16071	3.9601e-06\\
16218	3.8827e-06\\
16367	3.8068e-06\\
16517	3.7323e-06\\
16668	3.6592e-06\\
16821	3.5875e-06\\
16975	3.5172e-06\\
17130	3.4482e-06\\
17287	3.3805e-06\\
17445	3.3141e-06\\
17605	3.2489e-06\\
17766	3.185e-06\\
17929	3.1223e-06\\
18093	3.0608e-06\\
18259	3.0005e-06\\
18426	2.9412e-06\\
18595	2.8832e-06\\
18765	2.8262e-06\\
18937	2.7703e-06\\
19111	2.7154e-06\\
19286	2.6616e-06\\
19463	2.6088e-06\\
19641	2.557e-06\\
19821	2.5062e-06\\
20002	2.4564e-06\\
20186	2.4075e-06\\
20370	2.3595e-06\\
20557	2.3124e-06\\
20745	2.2662e-06\\
20935	2.2209e-06\\
21127	2.1765e-06\\
21321	2.1329e-06\\
21516	2.0901e-06\\
21713	2.0481e-06\\
21912	2.0069e-06\\
22113	1.9666e-06\\
22315	1.9269e-06\\
22520	1.8881e-06\\
22726	1.8499e-06\\
22934	1.8125e-06\\
23144	1.7758e-06\\
23356	1.7398e-06\\
23570	1.7045e-06\\
23786	1.6698e-06\\
24004	1.6358e-06\\
24224	1.6025e-06\\
24446	1.5698e-06\\
24669	1.5377e-06\\
24895	1.5063e-06\\
25123	1.4754e-06\\
25354	1.4451e-06\\
25586	1.4154e-06\\
25820	1.3863e-06\\
26057	1.3577e-06\\
26295	1.3297e-06\\
26536	1.3022e-06\\
26779	1.2752e-06\\
27025	1.2488e-06\\
27272	1.2229e-06\\
27522	1.1974e-06\\
27774	1.1725e-06\\
28028	1.148e-06\\
28285	1.124e-06\\
28544	1.1005e-06\\
28806	1.0774e-06\\
29070	1.0548e-06\\
29336	1.0326e-06\\
29605	1.0109e-06\\
29876	9.8953e-07\\
30149	9.6861e-07\\
30426	9.481e-07\\
30704	9.28e-07\\
30986	9.0828e-07\\
31269	8.8895e-07\\
31556	8.7e-07\\
31845	8.5143e-07\\
32137	8.3321e-07\\
32431	8.1536e-07\\
32728	7.9786e-07\\
33028	7.807e-07\\
33330	7.6388e-07\\
33636	7.474e-07\\
33944	7.3124e-07\\
34255	7.154e-07\\
34568	6.9988e-07\\
34885	6.8467e-07\\
35205	6.6976e-07\\
35527	6.5515e-07\\
35852	6.4084e-07\\
36181	6.2681e-07\\
36512	6.1306e-07\\
36847	5.9959e-07\\
37184	5.8639e-07\\
37525	5.7346e-07\\
37869	5.6079e-07\\
38215	5.4837e-07\\
38565	5.3621e-07\\
38919	5.243e-07\\
39275	5.1263e-07\\
39635	5.012e-07\\
39998	4.9e-07\\
40364	4.7903e-07\\
40734	4.6829e-07\\
41107	4.5777e-07\\
41484	4.4747e-07\\
41864	4.3738e-07\\
42247	4.275e-07\\
42634	4.1782e-07\\
43025	4.0835e-07\\
43419	3.9907e-07\\
43817	3.8999e-07\\
44218	3.8109e-07\\
44623	3.7239e-07\\
45032	3.6386e-07\\
45444	3.5552e-07\\
45860	3.4735e-07\\
46281	3.3936e-07\\
46704	3.3153e-07\\
47132	3.2387e-07\\
47564	3.1638e-07\\
48000	3.0904e-07\\
48439	3.0186e-07\\
48883	2.9484e-07\\
49331	2.8796e-07\\
49783	2.8123e-07\\
50239	2.7465e-07\\
50699	2.6821e-07\\
51163	2.6191e-07\\
51632	2.5575e-07\\
52105	2.4972e-07\\
52582	2.4382e-07\\
53064	2.3805e-07\\
53550	2.324e-07\\
54040	2.2688e-07\\
54535	2.2148e-07\\
55035	2.162e-07\\
55539	2.1104e-07\\
56048	2.0599e-07\\
56561	2.0105e-07\\
57079	1.9622e-07\\
57602	1.915e-07\\
58130	1.8688e-07\\
58662	1.8237e-07\\
59200	1.7795e-07\\
59742	1.7364e-07\\
60289	1.6942e-07\\
60841	1.653e-07\\
61399	1.6127e-07\\
61961	1.5733e-07\\
62529	1.5348e-07\\
63101	1.4972e-07\\
63679	1.4604e-07\\
64263	1.4244e-07\\
64851	1.3893e-07\\
65445	1.355e-07\\
66045	1.3215e-07\\
66650	1.2887e-07\\
67260	1.2567e-07\\
67876	1.2254e-07\\
68498	1.1949e-07\\
69125	1.165e-07\\
69759	1.1359e-07\\
70398	1.1074e-07\\
71042	1.0796e-07\\
71693	1.0524e-07\\
72350	1.0259e-07\\
73013	9.9998e-08\\
73681	9.7467e-08\\
74356	9.4996e-08\\
75037	9.2583e-08\\
75725	9.0227e-08\\
76418	8.7926e-08\\
77118	8.568e-08\\
77825	8.3487e-08\\
78538	8.1347e-08\\
79257	7.9257e-08\\
79983	7.7217e-08\\
80716	7.5226e-08\\
81455	7.3283e-08\\
82201	7.1386e-08\\
82954	6.9535e-08\\
83714	6.7729e-08\\
84481	6.5966e-08\\
85255	6.4246e-08\\
86036	6.2568e-08\\
86824	6.093e-08\\
87619	5.9333e-08\\
88421	5.7774e-08\\
89231	5.6254e-08\\
90049	5.477e-08\\
90874	5.3324e-08\\
91706	5.1912e-08\\
92546	5.0536e-08\\
93394	4.9194e-08\\
94249	4.7885e-08\\
95113	4.6608e-08\\
95984	4.5363e-08\\
96863	4.4149e-08\\
97750	4.2966e-08\\
98646	4.1812e-08\\
99549	4.0687e-08\\
1.0046e+05	3.959e-08\\
1.0138e+05	3.8521e-08\\
1.0231e+05	3.7479e-08\\
1.0325e+05	3.6463e-08\\
1.0419e+05	3.5473e-08\\
1.0515e+05	3.4508e-08\\
1.0611e+05	3.3568e-08\\
1.0708e+05	3.2652e-08\\
1.0806e+05	3.1759e-08\\
1.0905e+05	3.0889e-08\\
1.1005e+05	3.0041e-08\\
1.1106e+05	2.9215e-08\\
1.1208e+05	2.841e-08\\
1.131e+05	2.7626e-08\\
1.1414e+05	2.6863e-08\\
1.1519e+05	2.6119e-08\\
1.1624e+05	2.5394e-08\\
1.1731e+05	2.4689e-08\\
1.1838e+05	2.4001e-08\\
1.1946e+05	2.3332e-08\\
1.2056e+05	2.268e-08\\
1.2166e+05	2.2045e-08\\
1.2278e+05	2.1427e-08\\
1.239e+05	2.0826e-08\\
1.2504e+05	2.024e-08\\
1.2618e+05	1.9669e-08\\
1.2734e+05	1.9114e-08\\
1.285e+05	1.8574e-08\\
1.2968e+05	1.8048e-08\\
1.3087e+05	1.7535e-08\\
1.3207e+05	1.7037e-08\\
1.3328e+05	1.6552e-08\\
1.345e+05	1.608e-08\\
1.3573e+05	1.562e-08\\
1.3697e+05	1.5173e-08\\
1.3823e+05	1.4738e-08\\
1.3949e+05	1.4315e-08\\
1.4077e+05	1.3903e-08\\
1.4206e+05	1.3503e-08\\
1.4336e+05	1.3113e-08\\
1.4468e+05	1.2734e-08\\
1.46e+05	1.2365e-08\\
1.4734e+05	1.2007e-08\\
1.4869e+05	1.1658e-08\\
1.5005e+05	1.1319e-08\\
1.5142e+05	1.0989e-08\\
1.5281e+05	1.0668e-08\\
1.5421e+05	1.0356e-08\\
1.5562e+05	1.0053e-08\\
1.5705e+05	9.7577e-09\\
1.5849e+05	9.4709e-09\\
1.5994e+05	9.1921e-09\\
1.614e+05	8.9211e-09\\
1.6288e+05	8.6577e-09\\
1.6438e+05	8.4016e-09\\
1.6588e+05	8.1527e-09\\
1.674e+05	7.9108e-09\\
1.6893e+05	7.6757e-09\\
1.7048e+05	7.4473e-09\\
1.7204e+05	7.2252e-09\\
1.7362e+05	7.0095e-09\\
1.7521e+05	6.7999e-09\\
1.7681e+05	6.5962e-09\\
1.7843e+05	6.3983e-09\\
1.8007e+05	6.206e-09\\
1.8172e+05	6.0193e-09\\
1.8338e+05	5.8379e-09\\
1.8506e+05	5.6616e-09\\
1.8676e+05	5.4905e-09\\
1.8847e+05	5.3242e-09\\
1.9019e+05	5.1627e-09\\
1.9194e+05	5.0059e-09\\
1.9369e+05	4.8537e-09\\
1.9547e+05	4.7058e-09\\
1.9726e+05	4.5622e-09\\
1.9907e+05	4.4228e-09\\
2.0089e+05	4.2875e-09\\
2.0273e+05	4.1561e-09\\
2.0459e+05	4.0285e-09\\
2.0646e+05	3.9047e-09\\
2.0835e+05	3.7845e-09\\
2.1026e+05	3.6678e-09\\
2.1219e+05	3.5545e-09\\
2.1413e+05	3.4446e-09\\
2.1609e+05	3.338e-09\\
2.1807e+05	3.2344e-09\\
2.2007e+05	3.134e-09\\
2.2208e+05	3.0365e-09\\
2.2412e+05	2.9419e-09\\
2.2617e+05	2.8502e-09\\
2.2824e+05	2.7611e-09\\
2.3033e+05	2.6748e-09\\
2.3244e+05	2.591e-09\\
2.3457e+05	2.5097e-09\\
2.3672e+05	2.4309e-09\\
2.3889e+05	2.3544e-09\\
2.4108e+05	2.2802e-09\\
2.4329e+05	2.2083e-09\\
2.4551e+05	2.1385e-09\\
2.4776e+05	2.0709e-09\\
2.5003e+05	2.0052e-09\\
2.5003e+05	1.6788e-09\\
2.4776e+05	1.7359e-09\\
2.4551e+05	1.7947e-09\\
2.4329e+05	1.8555e-09\\
2.4108e+05	1.9183e-09\\
2.3889e+05	1.983e-09\\
2.3672e+05	2.0499e-09\\
2.3457e+05	2.1189e-09\\
2.3244e+05	2.1901e-09\\
2.3033e+05	2.2636e-09\\
2.2824e+05	2.3395e-09\\
2.2617e+05	2.4178e-09\\
2.2412e+05	2.4985e-09\\
2.2208e+05	2.5819e-09\\
2.2007e+05	2.6678e-09\\
2.1807e+05	2.7566e-09\\
2.1609e+05	2.8481e-09\\
2.1413e+05	2.9425e-09\\
2.1219e+05	3.0399e-09\\
2.1026e+05	3.1403e-09\\
2.0835e+05	3.2439e-09\\
2.0646e+05	3.3507e-09\\
2.0459e+05	3.4609e-09\\
2.0273e+05	3.5746e-09\\
2.0089e+05	3.6917e-09\\
1.9907e+05	3.8125e-09\\
1.9726e+05	3.9371e-09\\
1.9547e+05	4.0655e-09\\
1.9369e+05	4.198e-09\\
1.9194e+05	4.3344e-09\\
1.9019e+05	4.4752e-09\\
1.8847e+05	4.6202e-09\\
1.8676e+05	4.7697e-09\\
1.8506e+05	4.9237e-09\\
1.8338e+05	5.0825e-09\\
1.8172e+05	5.2461e-09\\
1.8007e+05	5.4148e-09\\
1.7843e+05	5.5885e-09\\
1.7681e+05	5.7675e-09\\
1.7521e+05	5.952e-09\\
1.7362e+05	6.142e-09\\
1.7204e+05	6.3377e-09\\
1.7048e+05	6.5394e-09\\
1.6893e+05	6.7471e-09\\
1.674e+05	6.961e-09\\
1.6588e+05	7.1814e-09\\
1.6438e+05	7.4083e-09\\
1.6288e+05	7.642e-09\\
1.614e+05	7.8826e-09\\
1.5994e+05	8.1304e-09\\
1.5849e+05	8.3856e-09\\
1.5705e+05	8.6482e-09\\
1.5562e+05	8.9187e-09\\
1.5421e+05	9.1971e-09\\
1.5281e+05	9.4837e-09\\
1.5142e+05	9.7787e-09\\
1.5005e+05	1.0082e-08\\
1.4869e+05	1.0395e-08\\
1.4734e+05	1.0716e-08\\
1.46e+05	1.1047e-08\\
1.4468e+05	1.1388e-08\\
1.4336e+05	1.1738e-08\\
1.4206e+05	1.2099e-08\\
1.4077e+05	1.247e-08\\
1.3949e+05	1.2852e-08\\
1.3823e+05	1.3244e-08\\
1.3697e+05	1.3648e-08\\
1.3573e+05	1.4064e-08\\
1.345e+05	1.4491e-08\\
1.3328e+05	1.4931e-08\\
1.3207e+05	1.5383e-08\\
1.3087e+05	1.5847e-08\\
1.2968e+05	1.6325e-08\\
1.285e+05	1.6817e-08\\
1.2734e+05	1.7322e-08\\
1.2618e+05	1.7842e-08\\
1.2504e+05	1.8376e-08\\
1.239e+05	1.8925e-08\\
1.2278e+05	1.9489e-08\\
1.2166e+05	2.0069e-08\\
1.2056e+05	2.0665e-08\\
1.1946e+05	2.1278e-08\\
1.1838e+05	2.1908e-08\\
1.1731e+05	2.2555e-08\\
1.1624e+05	2.322e-08\\
1.1519e+05	2.3903e-08\\
1.1414e+05	2.4605e-08\\
1.131e+05	2.5326e-08\\
1.1208e+05	2.6067e-08\\
1.1106e+05	2.6828e-08\\
1.1005e+05	2.761e-08\\
1.0905e+05	2.8413e-08\\
1.0806e+05	2.9237e-08\\
1.0708e+05	3.0084e-08\\
1.0611e+05	3.0954e-08\\
1.0515e+05	3.1847e-08\\
1.0419e+05	3.2764e-08\\
1.0325e+05	3.3706e-08\\
1.0231e+05	3.4672e-08\\
1.0138e+05	3.5665e-08\\
1.0046e+05	3.6683e-08\\
99549	3.7729e-08\\
98646	3.8802e-08\\
97750	3.9904e-08\\
96863	4.1034e-08\\
95984	4.2195e-08\\
95113	4.3385e-08\\
94249	4.4607e-08\\
93394	4.5861e-08\\
92546	4.7147e-08\\
91706	4.8466e-08\\
90874	4.9819e-08\\
90049	5.1208e-08\\
89231	5.2632e-08\\
88421	5.4093e-08\\
87619	5.5591e-08\\
86824	5.7127e-08\\
86036	5.8702e-08\\
85255	6.0318e-08\\
84481	6.1974e-08\\
83714	6.3672e-08\\
82954	6.5413e-08\\
82201	6.7198e-08\\
81455	6.9027e-08\\
80716	7.0902e-08\\
79983	7.2825e-08\\
79257	7.4794e-08\\
78538	7.6813e-08\\
77825	7.8882e-08\\
77118	8.1002e-08\\
76418	8.3174e-08\\
75725	8.5399e-08\\
75037	8.7679e-08\\
74356	9.0015e-08\\
73681	9.2407e-08\\
73013	9.4858e-08\\
72350	9.7368e-08\\
71693	9.9939e-08\\
71042	1.0257e-07\\
70398	1.0527e-07\\
69759	1.0803e-07\\
69125	1.1086e-07\\
68498	1.1375e-07\\
67876	1.1671e-07\\
67260	1.1975e-07\\
66650	1.2285e-07\\
66045	1.2603e-07\\
65445	1.2928e-07\\
64851	1.3261e-07\\
64263	1.3602e-07\\
63679	1.3951e-07\\
63101	1.4308e-07\\
62529	1.4673e-07\\
61961	1.5047e-07\\
61399	1.5429e-07\\
60841	1.582e-07\\
60289	1.622e-07\\
59742	1.6629e-07\\
59200	1.7048e-07\\
58662	1.7476e-07\\
58130	1.7914e-07\\
57602	1.8362e-07\\
57079	1.882e-07\\
56561	1.9288e-07\\
56048	1.9767e-07\\
55539	2.0257e-07\\
55035	2.0757e-07\\
54535	2.1269e-07\\
54040	2.1792e-07\\
53550	2.2327e-07\\
53064	2.2874e-07\\
52582	2.3433e-07\\
52105	2.4004e-07\\
51632	2.4588e-07\\
51163	2.5184e-07\\
50699	2.5794e-07\\
50239	2.6417e-07\\
49783	2.7054e-07\\
49331	2.7704e-07\\
48883	2.8369e-07\\
48439	2.9048e-07\\
48000	2.9742e-07\\
47564	3.045e-07\\
47132	3.1174e-07\\
46704	3.1914e-07\\
46281	3.2669e-07\\
45860	3.3441e-07\\
45444	3.4229e-07\\
45032	3.5034e-07\\
44623	3.5856e-07\\
44218	3.6695e-07\\
43817	3.7553e-07\\
43419	3.8428e-07\\
43025	3.9322e-07\\
42634	4.0235e-07\\
42247	4.1166e-07\\
41864	4.2118e-07\\
41484	4.3089e-07\\
41107	4.4081e-07\\
40734	4.5093e-07\\
40364	4.6127e-07\\
39998	4.7182e-07\\
39635	4.8259e-07\\
39275	4.9358e-07\\
38919	5.048e-07\\
38565	5.1625e-07\\
38215	5.2794e-07\\
37869	5.3987e-07\\
37525	5.5205e-07\\
37184	5.6447e-07\\
36847	5.7715e-07\\
36512	5.9009e-07\\
36181	6.033e-07\\
35852	6.1677e-07\\
35527	6.3052e-07\\
35205	6.4455e-07\\
34885	6.5886e-07\\
34568	6.7346e-07\\
34255	6.8836e-07\\
33944	7.0357e-07\\
33636	7.1908e-07\\
33330	7.349e-07\\
33028	7.5104e-07\\
32728	7.6751e-07\\
32431	7.8431e-07\\
32137	8.0144e-07\\
31845	8.1893e-07\\
31556	8.3676e-07\\
31269	8.5495e-07\\
30986	8.735e-07\\
30704	8.9242e-07\\
30426	9.1172e-07\\
30149	9.3141e-07\\
29876	9.5149e-07\\
29605	9.7197e-07\\
29336	9.9285e-07\\
29070	1.0142e-06\\
28806	1.0359e-06\\
28544	1.058e-06\\
28285	1.0806e-06\\
28028	1.1037e-06\\
27774	1.1272e-06\\
27522	1.1511e-06\\
27272	1.1756e-06\\
27025	1.2005e-06\\
26779	1.2259e-06\\
26536	1.2518e-06\\
26295	1.2782e-06\\
26057	1.3052e-06\\
25820	1.3327e-06\\
25586	1.3607e-06\\
25354	1.3892e-06\\
25123	1.4184e-06\\
24895	1.4481e-06\\
24669	1.4783e-06\\
24446	1.5092e-06\\
24224	1.5407e-06\\
24004	1.5728e-06\\
23786	1.6055e-06\\
23570	1.6389e-06\\
23356	1.6729e-06\\
23144	1.7076e-06\\
22934	1.743e-06\\
22726	1.779e-06\\
22520	1.8158e-06\\
22315	1.8533e-06\\
22113	1.8915e-06\\
21912	1.9305e-06\\
21713	1.9702e-06\\
21516	2.0107e-06\\
21321	2.052e-06\\
21127	2.0941e-06\\
20935	2.137e-06\\
20745	2.1808e-06\\
20557	2.2254e-06\\
20370	2.2709e-06\\
20186	2.3173e-06\\
20002	2.3646e-06\\
19821	2.4128e-06\\
19641	2.4619e-06\\
19463	2.512e-06\\
19286	2.5631e-06\\
19111	2.6152e-06\\
18937	2.6683e-06\\
18765	2.7224e-06\\
18595	2.7776e-06\\
18426	2.8339e-06\\
18259	2.8913e-06\\
18093	2.9497e-06\\
17929	3.0094e-06\\
17766	3.0702e-06\\
17605	3.1322e-06\\
17445	3.1954e-06\\
17287	3.2598e-06\\
17130	3.3255e-06\\
16975	3.3925e-06\\
16821	3.4607e-06\\
16668	3.5304e-06\\
16517	3.6013e-06\\
16367	3.6737e-06\\
16218	3.7475e-06\\
16071	3.8227e-06\\
15925	3.8994e-06\\
15780	3.9776e-06\\
15637	4.0573e-06\\
15495	4.1386e-06\\
15355	4.2215e-06\\
15215	4.306e-06\\
15077	4.3922e-06\\
14940	4.48e-06\\
14805	4.5696e-06\\
14670	4.6609e-06\\
14537	4.754e-06\\
14405	4.8489e-06\\
14274	4.9457e-06\\
14145	5.0444e-06\\
14017	5.145e-06\\
13889	5.2476e-06\\
13763	5.3521e-06\\
13638	5.4588e-06\\
13515	5.5675e-06\\
13392	5.6784e-06\\
13270	5.7914e-06\\
13150	5.9066e-06\\
13030	6.0241e-06\\
12912	6.1439e-06\\
12795	6.266e-06\\
12679	6.3905e-06\\
12564	6.5175e-06\\
12450	6.6469e-06\\
12337	6.7789e-06\\
12225	6.9135e-06\\
12114	7.0506e-06\\
12004	7.1905e-06\\
11895	7.3331e-06\\
11787	7.4785e-06\\
11680	7.6267e-06\\
11574	7.7778e-06\\
11469	7.9318e-06\\
11365	8.0889e-06\\
11262	8.249e-06\\
11159	8.4122e-06\\
11058	8.5786e-06\\
10958	8.7482e-06\\
10858	8.9212e-06\\
10760	9.0975e-06\\
10662	9.2772e-06\\
10565	9.4604e-06\\
10469	9.6471e-06\\
10374	9.8375e-06\\
10280	1.0032e-05\\
10187	1.0229e-05\\
10094	1.0431e-05\\
10003	1.0637e-05\\
9912	1.0846e-05\\
9822	1.106e-05\\
9732.9	1.1277e-05\\
9644.5	1.1499e-05\\
9557	1.1725e-05\\
9470.2	1.1956e-05\\
9384.3	1.2191e-05\\
9299.1	1.243e-05\\
9214.7	1.2674e-05\\
9131.1	1.2923e-05\\
9048.2	1.3176e-05\\
8966	1.3434e-05\\
8884.7	1.3697e-05\\
8804	1.3966e-05\\
8724.1	1.4239e-05\\
8644.9	1.4517e-05\\
8566.5	1.4801e-05\\
8488.7	1.509e-05\\
8411.6	1.5385e-05\\
8335.3	1.5685e-05\\
8259.6	1.599e-05\\
8184.7	1.6302e-05\\
8110.4	1.6619e-05\\
8036.8	1.6943e-05\\
7963.8	1.7272e-05\\
7891.5	1.7608e-05\\
7819.9	1.7949e-05\\
7748.9	1.8297e-05\\
7678.6	1.8652e-05\\
7608.9	1.9013e-05\\
7539.8	1.9381e-05\\
7471.4	1.9756e-05\\
7403.6	2.0137e-05\\
7336.4	2.0526e-05\\
7269.8	2.0922e-05\\
7203.8	2.1325e-05\\
7138.4	2.1735e-05\\
7073.6	2.2153e-05\\
7009.4	2.2578e-05\\
6945.8	2.3011e-05\\
6882.7	2.3452e-05\\
6820.3	2.3901e-05\\
6758.4	2.4358e-05\\
6697	2.4823e-05\\
6636.2	2.5296e-05\\
6576	2.5778e-05\\
6516.3	2.6269e-05\\
6457.2	2.6768e-05\\
6398.5	2.7276e-05\\
6340.5	2.7793e-05\\
6282.9	2.8319e-05\\
6225.9	2.8854e-05\\
6169.4	2.9399e-05\\
6113.4	2.9953e-05\\
6057.9	3.0516e-05\\
6002.9	3.109e-05\\
5948.4	3.1673e-05\\
5894.4	3.2266e-05\\
5840.9	3.287e-05\\
5787.9	3.3483e-05\\
5735.4	3.4107e-05\\
5683.3	3.4742e-05\\
5631.7	3.5387e-05\\
5580.6	3.6043e-05\\
5530	3.671e-05\\
5479.8	3.7388e-05\\
5430	3.8078e-05\\
5380.7	3.8778e-05\\
5331.9	3.949e-05\\
5283.5	4.0214e-05\\
5235.5	4.095e-05\\
5188	4.1697e-05\\
5140.9	4.2456e-05\\
5094.3	4.3228e-05\\
5048	4.4011e-05\\
5002.2	4.4808e-05\\
4956.8	4.5616e-05\\
4911.8	4.6437e-05\\
4867.2	4.7271e-05\\
4823.1	4.8118e-05\\
4779.3	4.8978e-05\\
4735.9	4.9851e-05\\
4692.9	5.0738e-05\\
4650.3	5.1638e-05\\
4608.1	5.2551e-05\\
4566.3	5.3478e-05\\
4524.8	5.4418e-05\\
4483.8	5.5373e-05\\
4443.1	5.6341e-05\\
4402.7	5.7323e-05\\
4362.8	5.832e-05\\
4323.2	5.9331e-05\\
4283.9	6.0356e-05\\
4245	6.1396e-05\\
4206.5	6.245e-05\\
4168.3	6.3518e-05\\
4130.5	6.4602e-05\\
4093	6.57e-05\\
4055.9	6.6813e-05\\
4019	6.7941e-05\\
3982.6	6.9084e-05\\
3946.4	7.0242e-05\\
3910.6	7.1415e-05\\
3875.1	7.2604e-05\\
3839.9	7.3807e-05\\
3805.1	7.5026e-05\\
3770.5	7.6261e-05\\
3736.3	7.751e-05\\
3702.4	7.8775e-05\\
3668.8	8.0056e-05\\
3635.5	8.1352e-05\\
3602.5	8.2663e-05\\
3569.8	8.399e-05\\
3537.4	8.5333e-05\\
3505.3	8.6691e-05\\
3473.5	8.8064e-05\\
3441.9	8.9453e-05\\
3410.7	9.0858e-05\\
3379.7	9.2278e-05\\
3349.1	9.3713e-05\\
3318.7	9.5164e-05\\
3288.5	9.663e-05\\
3258.7	9.8112e-05\\
3229.1	9.9609e-05\\
3199.8	0.00010112\\
3170.8	0.00010265\\
3142	0.00010419\\
3113.5	0.00010575\\
3085.2	0.00010732\\
3057.2	0.00010891\\
3029.4	0.00011051\\
3001.9	0.00011213\\
2974.7	0.00011376\\
2947.7	0.00011541\\
2920.9	0.00011707\\
2894.4	0.00011874\\
2868.2	0.00012043\\
2842.1	0.00012214\\
2816.3	0.00012385\\
2790.8	0.00012558\\
2765.4	0.00012733\\
2740.3	0.00012909\\
2715.5	0.00013086\\
2690.8	0.00013264\\
2666.4	0.00013444\\
2642.2	0.00013625\\
2618.2	0.00013807\\
2594.4	0.00013991\\
2570.9	0.00014175\\
2547.6	0.00014361\\
2524.4	0.00014548\\
2501.5	0.00014736\\
2478.8	0.00014926\\
2456.3	0.00015116\\
2434	0.00015308\\
2411.9	0.000155\\
2390	0.00015694\\
2368.3	0.00015888\\
2346.8	0.00016084\\
2325.5	0.0001628\\
2304.4	0.00016477\\
2283.5	0.00016676\\
2262.8	0.00016875\\
2242.2	0.00017074\\
2221.9	0.00017275\\
2201.7	0.00017476\\
2181.7	0.00017678\\
2161.9	0.00017881\\
2142.3	0.00018084\\
2122.9	0.00018288\\
2103.6	0.00018492\\
2084.5	0.00018697\\
2065.6	0.00018903\\
2046.8	0.00019108\\
2028.3	0.00019315\\
2009.8	0.00019521\\
1991.6	0.00019728\\
1973.5	0.00019935\\
1955.6	0.00020142\\
1937.9	0.0002035\\
1920.3	0.00020557\\
1902.8	0.00020765\\
1885.6	0.00020972\\
1868.5	0.0002118\\
1851.5	0.00021387\\
1834.7	0.00021595\\
1818	0.00021802\\
1801.5	0.00022009\\
1785.2	0.00022215\\
1769	0.00022422\\
1752.9	0.00022628\\
1737	0.00022833\\
1721.2	0.00023038\\
1705.6	0.00023243\\
1690.1	0.00023446\\
1674.8	0.0002365\\
1659.6	0.00023852\\
1644.5	0.00024054\\
1629.6	0.00024255\\
1614.8	0.00024455\\
1600.2	0.00024654\\
1585.6	0.00024852\\
1571.2	0.00025049\\
1557	0.00025245\\
1542.8	0.00025439\\
1528.8	0.00025633\\
1515	0.00025825\\
1501.2	0.00026016\\
1487.6	0.00026206\\
1474.1	0.00026394\\
1460.7	0.00026581\\
1447.4	0.00026766\\
1434.3	0.00026949\\
1421.3	0.00027131\\
1408.4	0.00027312\\
1395.6	0.0002749\\
1382.9	0.00027667\\
1370.4	0.00027842\\
1357.9	0.00028015\\
1345.6	0.00028186\\
1333.4	0.00028355\\
1321.3	0.00028522\\
1309.3	0.00028687\\
1297.4	0.0002885\\
1285.7	0.00029011\\
1274	0.00029169\\
1262.4	0.00029326\\
1251	0.00029479\\
1239.6	0.00029631\\
1228.4	0.0002978\\
1217.2	0.00029927\\
1206.2	0.00030071\\
1195.2	0.00030213\\
1184.4	0.00030352\\
1173.6	0.00030489\\
1163	0.00030623\\
1152.4	0.00030754\\
1141.9	0.00030883\\
1131.6	0.00031009\\
1121.3	0.00031132\\
1111.1	0.00031253\\
1101	0.0003137\\
1091	0.00031485\\
1081.1	0.00031597\\
1071.3	0.00031706\\
1061.6	0.00031812\\
1052	0.00031915\\
1042.4	0.00032014\\
1033	0.00032111\\
1023.6	0.00032205\\
1014.3	0.00032296\\
1005.1	0.00032383\\
995.96	0.00032468\\
986.92	0.00032549\\
977.97	0.00032627\\
969.09	0.00032702\\
960.29	0.00032774\\
951.58	0.00032842\\
942.94	0.00032908\\
934.38	0.0003297\\
925.9	0.00033028\\
917.5	0.00033083\\
909.17	0.00033135\\
900.92	0.00033184\\
892.74	0.00033229\\
884.64	0.00033271\\
876.61	0.0003331\\
868.65	0.00033345\\
860.76	0.00033377\\
852.95	0.00033405\\
845.21	0.0003343\\
837.54	0.00033452\\
829.94	0.0003347\\
822.4	0.00033485\\
814.94	0.00033497\\
807.54	0.00033505\\
800.21	0.00033509\\
792.95	0.00033511\\
785.75	0.00033508\\
778.62	0.00033503\\
771.55	0.00033494\\
764.55	0.00033482\\
757.61	0.00033466\\
750.73	0.00033447\\
743.92	0.00033425\\
737.16	0.00033399\\
730.47	0.0003337\\
723.84	0.00033338\\
717.27	0.00033302\\
710.76	0.00033263\\
704.31	0.00033221\\
697.92	0.00033176\\
691.58	0.00033127\\
685.31	0.00033075\\
679.09	0.00033021\\
672.92	0.00032962\\
666.81	0.00032901\\
660.76	0.00032837\\
654.76	0.0003277\\
648.82	0.00032699\\
642.93	0.00032626\\
637.1	0.0003255\\
631.31	0.0003247\\
625.58	0.00032388\\
619.9	0.00032303\\
614.28	0.00032215\\
608.7	0.00032124\\
603.18	0.00032031\\
597.7	0.00031934\\
592.28	0.00031835\\
586.9	0.00031733\\
581.57	0.00031629\\
576.29	0.00031522\\
571.06	0.00031413\\
565.88	0.00031301\\
560.74	0.00031186\\
555.65	0.00031069\\
550.61	0.0003095\\
545.61	0.00030828\\
540.66	0.00030704\\
535.75	0.00030577\\
530.89	0.00030449\\
526.07	0.00030318\\
521.3	0.00030185\\
516.56	0.0003005\\
511.88	0.00029913\\
507.23	0.00029773\\
502.63	0.00029632\\
498.06	0.00029489\\
493.54	0.00029344\\
489.06	0.00029197\\
484.62	0.00029049\\
480.23	0.00028898\\
475.87	0.00028746\\
471.55	0.00028592\\
467.27	0.00028437\\
463.03	0.0002828\\
458.82	0.00028121\\
454.66	0.00027961\\
450.53	0.00027799\\
446.44	0.00027636\\
442.39	0.00027472\\
438.37	0.00027306\\
434.4	0.00027139\\
430.45	0.00026971\\
426.55	0.00026802\\
422.67	0.00026631\\
418.84	0.0002646\\
415.04	0.00026287\\
411.27	0.00026113\\
407.54	0.00025939\\
403.84	0.00025763\\
400.17	0.00025586\\
396.54	0.00025409\\
392.94	0.00025231\\
389.37	0.00025052\\
385.84	0.00024872\\
382.34	0.00024691\\
378.87	0.0002451\\
375.43	0.00024329\\
372.02	0.00024146\\
368.64	0.00023964\\
365.3	0.0002378\\
361.98	0.00023597\\
358.7	0.00023412\\
355.44	0.00023228\\
352.21	0.00023043\\
349.02	0.00022858\\
345.85	0.00022672\\
342.71	0.00022486\\
339.6	0.000223\\
336.52	0.00022114\\
333.46	0.00021928\\
330.43	0.00021742\\
327.44	0.00021555\\
324.46	0.00021369\\
321.52	0.00021182\\
318.6	0.00020996\\
315.71	0.00020809\\
312.84	0.00020623\\
310	0.00020437\\
307.19	0.00020251\\
304.4	0.00020065\\
301.64	0.0001988\\
298.9	0.00019694\\
296.19	0.00019509\\
293.5	0.00019325\\
290.83	0.0001914\\
288.19	0.00018956\\
285.58	0.00018772\\
282.99	0.00018589\\
280.42	0.00018406\\
277.87	0.00018224\\
275.35	0.00018042\\
272.85	0.00017861\\
270.37	0.0001768\\
267.92	0.000175\\
265.49	0.00017321\\
263.08	0.00017142\\
260.69	0.00016963\\
258.32	0.00016786\\
255.98	0.00016609\\
253.66	0.00016432\\
251.35	0.00016257\\
249.07	0.00016082\\
246.81	0.00015908\\
244.57	0.00015735\\
242.35	0.00015563\\
240.15	0.00015391\\
237.97	0.0001522\\
235.81	0.00015051\\
233.67	0.00014882\\
231.55	0.00014714\\
229.45	0.00014547\\
227.37	0.0001438\\
225.3	0.00014215\\
223.26	0.00014051\\
221.23	0.00013888\\
219.22	0.00013725\\
217.23	0.00013564\\
215.26	0.00013404\\
213.31	0.00013245\\
211.37	0.00013087\\
209.45	0.00012929\\
207.55	0.00012773\\
205.67	0.00012619\\
203.8	0.00012465\\
201.95	0.00012312\\
200.12	0.0001216\\
198.3	0.0001201\\
196.5	0.00011861\\
194.72	0.00011713\\
192.95	0.00011566\\
191.2	0.0001142\\
189.46	0.00011275\\
187.74	0.00011132\\
186.04	0.00010989\\
184.35	0.00010848\\
182.68	0.00010708\\
181.02	0.00010569\\
179.38	0.00010432\\
177.75	0.00010296\\
176.14	0.00010161\\
174.54	0.00010027\\
172.95	9.8941e-05\\
171.38	9.7627e-05\\
169.83	9.6325e-05\\
168.29	9.5035e-05\\
166.76	9.3758e-05\\
165.24	9.2493e-05\\
163.74	9.1241e-05\\
162.26	9e-05\\
160.79	8.8772e-05\\
159.33	8.7557e-05\\
157.88	8.6354e-05\\
156.45	8.5163e-05\\
155.03	8.3984e-05\\
153.62	8.2818e-05\\
152.23	8.1664e-05\\
150.84	8.0522e-05\\
149.47	7.9393e-05\\
148.12	7.8275e-05\\
146.77	7.717e-05\\
145.44	7.6077e-05\\
144.12	7.4997e-05\\
142.81	7.3928e-05\\
141.52	7.2871e-05\\
140.23	7.1827e-05\\
138.96	7.0794e-05\\
137.7	6.9773e-05\\
136.45	6.8764e-05\\
135.21	6.7767e-05\\
133.98	6.6782e-05\\
132.77	6.5808e-05\\
131.56	6.4846e-05\\
130.37	6.3896e-05\\
129.18	6.2957e-05\\
128.01	6.2029e-05\\
126.85	6.1113e-05\\
125.7	6.0208e-05\\
124.56	5.9315e-05\\
123.43	5.8433e-05\\
122.31	5.7561e-05\\
121.2	5.6701e-05\\
120.1	5.5852e-05\\
119.01	5.5014e-05\\
117.93	5.4186e-05\\
116.85	5.3369e-05\\
115.79	5.2563e-05\\
114.74	5.1768e-05\\
113.7	5.0982e-05\\
112.67	5.0208e-05\\
111.65	4.9443e-05\\
110.63	4.8689e-05\\
109.63	4.7945e-05\\
108.63	4.7211e-05\\
107.65	4.6487e-05\\
106.67	4.5773e-05\\
105.7	4.5068e-05\\
104.74	4.4374e-05\\
103.79	4.3688e-05\\
102.85	4.3013e-05\\
101.92	4.2347e-05\\
100.99	4.169e-05\\
100.08	4.1042e-05\\
99.167	4.0404e-05\\
98.267	3.9774e-05\\
97.375	3.9154e-05\\
96.491	3.8542e-05\\
95.615	3.7939e-05\\
94.747	3.7345e-05\\
93.887	3.6759e-05\\
93.035	3.6182e-05\\
92.191	3.5614e-05\\
91.354	3.5053e-05\\
90.525	3.4501e-05\\
89.703	3.3957e-05\\
88.889	3.3421e-05\\
88.082	3.2893e-05\\
87.283	3.2372e-05\\
86.49	3.186e-05\\
85.705	3.1355e-05\\
84.927	3.0857e-05\\
84.156	3.0368e-05\\
83.393	2.9885e-05\\
82.636	2.941e-05\\
81.886	2.8942e-05\\
81.142	2.8481e-05\\
80.406	2.8027e-05\\
79.676	2.758e-05\\
78.953	2.714e-05\\
78.236	2.6706e-05\\
77.526	2.6279e-05\\
76.822	2.5859e-05\\
76.125	2.5445e-05\\
75.434	2.5038e-05\\
74.749	2.4637e-05\\
74.071	2.4242e-05\\
73.399	2.3853e-05\\
72.732	2.3471e-05\\
72.072	2.3094e-05\\
71.418	2.2723e-05\\
70.77	2.2358e-05\\
70.127	2.1999e-05\\
69.491	2.1645e-05\\
68.86	2.1297e-05\\
68.235	2.0954e-05\\
67.616	2.0617e-05\\
67.002	2.0285e-05\\
66.394	1.9959e-05\\
65.791	1.9637e-05\\
65.194	1.9321e-05\\
64.602	1.901e-05\\
64.016	1.8703e-05\\
63.435	1.8402e-05\\
62.859	1.8105e-05\\
62.289	1.7813e-05\\
61.723	1.7526e-05\\
61.163	1.7243e-05\\
60.608	1.6965e-05\\
60.058	1.6691e-05\\
59.512	1.6422e-05\\
58.972	1.6157e-05\\
58.437	1.5896e-05\\
57.907	1.564e-05\\
57.381	1.5387e-05\\
56.86	1.5139e-05\\
56.344	1.4894e-05\\
55.833	1.4654e-05\\
55.326	1.4417e-05\\
54.824	1.4185e-05\\
54.326	1.3956e-05\\
53.833	1.373e-05\\
53.344	1.3509e-05\\
52.86	1.329e-05\\
52.38	1.3076e-05\\
51.905	1.2865e-05\\
51.434	1.2657e-05\\
50.967	1.2452e-05\\
50.504	1.2251e-05\\
50.046	1.2053e-05\\
49.592	1.1859e-05\\
49.141	1.1667e-05\\
48.695	1.1479e-05\\
48.253	1.1294e-05\\
47.815	1.1111e-05\\
47.381	1.0932e-05\\
46.951	1.0755e-05\\
46.525	1.0581e-05\\
46.103	1.0411e-05\\
45.684	1.0242e-05\\
45.27	1.0077e-05\\
44.859	9.9142e-06\\
44.452	9.754e-06\\
44.048	9.5965e-06\\
43.648	9.4414e-06\\
43.252	9.2889e-06\\
42.86	9.1388e-06\\
42.471	8.9912e-06\\
42.085	8.8459e-06\\
41.703	8.703e-06\\
41.325	8.5623e-06\\
40.949	8.4239e-06\\
40.578	8.2877e-06\\
40.209	8.1538e-06\\
39.845	8.0219e-06\\
39.483	7.8922e-06\\
39.124	7.7646e-06\\
38.769	7.639e-06\\
38.417	7.5154e-06\\
38.069	7.3938e-06\\
37.723	7.2741e-06\\
37.381	7.1563e-06\\
37.042	7.0405e-06\\
36.705	6.9265e-06\\
36.372	6.8142e-06\\
36.042	6.7038e-06\\
35.715	6.5952e-06\\
35.391	6.4883e-06\\
35.069	6.383e-06\\
34.751	6.2795e-06\\
34.436	6.1776e-06\\
34.123	6.0773e-06\\
33.813	5.9786e-06\\
33.506	5.8815e-06\\
33.202	5.7859e-06\\
32.901	5.6919e-06\\
32.602	5.5993e-06\\
32.306	5.5082e-06\\
32.013	5.4186e-06\\
31.723	5.3303e-06\\
31.435	5.2435e-06\\
31.149	5.158e-06\\
30.867	5.0739e-06\\
30.586	4.9911e-06\\
30.309	4.9096e-06\\
30.034	4.8294e-06\\
29.761	4.7505e-06\\
29.491	4.6728e-06\\
29.223	4.5963e-06\\
28.958	4.5211e-06\\
28.695	4.447e-06\\
28.435	4.3741e-06\\
28.177	4.3023e-06\\
27.921	4.2317e-06\\
27.667	4.1622e-06\\
}--cycle;
\addplot[ybar interval, fill=mycolor2, fill opacity=0.35, area legend, draw=none] table[row sep=crcr, x=Lower, y=Count] {%
Lower	Upper	Count\\
27.667	39.83	6.5776e-06\\
39.83	57.339	6.092e-06\\
57.339	82.545	2.6449e-05\\
82.545	118.83	5.2179e-05\\
118.83	171.07	8.3727e-05\\
171.07	246.27	0.00014504\\
246.27	354.53	0.00022145\\
354.53	510.38	0.00027069\\
510.38	734.75	0.00031378\\
734.75	1057.7	0.00032455\\
1057.7	1522.7	0.00029226\\
1522.7	2192.1	0.0002176\\
2192.1	3155.8	0.00014089\\
3155.8	4543	7.693e-05\\
4543	6540.1	3.7509e-05\\
6540.1	9415.1	1.7828e-05\\
9415.1	13554	8.1764e-06\\
13554	19512	3.8043e-06\\
19512	28090	1.8001e-06\\
28090	40438	6.9973e-07\\
40438	58215	2.3402e-07\\
58215	83806	8.9618e-08\\
83806	1.2065e+05	2.461e-08\\
1.2065e+05	1.7368e+05	3.57e-08\\
1.7368e+05	2.5003e+05	6.9854e-09\\
2.5003e+05	2.5003e+05	6.9854e-09\\
};
\addlegendentry{Istogramma appr.}

\addplot [color=mycolor2, only marks, every error bar/.append style={opacity=0.45}, mark=*, mark size=0pt, draw=none, forget plot]
 plot [error bars/.cd, y dir=both, y explicit, error bar style={line width=1pt, color=mycolor2}, error mark options={mark=none,mark size=0pt}]
 table[row sep=crcr, y error plus index=2, y error minus index=3]{%
33.196	6.5776e-06	7.4587e-06	6.5776e-06\\
47.789	6.092e-06	5.9517e-06	5.9517e-06\\
68.797	2.6449e-05	9.1355e-06	9.1355e-06\\
99.041	5.2179e-05	1.1778e-05	1.1778e-05\\
142.58	8.3727e-05	1.3241e-05	1.3241e-05\\
205.26	0.00014504	1.2811e-05	1.2811e-05\\
295.49	0.00022145	1.2949e-05	1.2949e-05\\
425.38	0.00027069	1.1259e-05	1.1259e-05\\
612.38	0.00031378	1.1163e-05	1.1163e-05\\
881.58	0.00032455	8.5579e-06	8.5579e-06\\
1269.1	0.00029226	6.0378e-06	6.0378e-06\\
1827	0.0002176	5.8867e-06	5.8867e-06\\
2630.2	0.00014089	3.4833e-06	3.4833e-06\\
3786.4	7.693e-05	2.2862e-06	2.2862e-06\\
5450.9	3.7509e-05	1.2892e-06	1.2892e-06\\
7847	1.7828e-05	6.6986e-07	6.6986e-07\\
11297	8.1764e-06	3.7143e-07	3.7143e-07\\
16263	3.8043e-06	2.3187e-07	2.3187e-07\\
23411	1.8001e-06	1.3391e-07	1.3391e-07\\
33703	6.9973e-07	6.7317e-08	6.7317e-08\\
48519	2.3402e-07	3.2309e-08	3.2309e-08\\
69848	8.9618e-08	1.47e-08	1.47e-08\\
1.0055e+05	2.461e-08	7.9641e-09	7.9641e-09\\
1.4476e+05	3.57e-08	4.55e-09	4.55e-09\\
2.0839e+05	6.9854e-09	2.7861e-09	2.7861e-09\\
};
\addplot [color=mycolor2]
  table[row sep=crcr]{%
27.667	6.8918e-06\\
31.149	8.4524e-06\\
34.751	1.0169e-05\\
38.417	1.2013e-05\\
42.085	1.3949e-05\\
46.103	1.6166e-05\\
50.046	1.8432e-05\\
53.833	2.0688e-05\\
57.907	2.3195e-05\\
62.289	2.598e-05\\
66.394	2.8666e-05\\
70.77	3.1604e-05\\
75.434	3.4817e-05\\
80.406	3.8326e-05\\
84.927	4.1586e-05\\
89.703	4.5095e-05\\
94.747	4.8868e-05\\
100.08	5.2918e-05\\
105.7	5.7262e-05\\
111.65	6.1914e-05\\
116.85	6.6035e-05\\
122.31	7.0386e-05\\
128.01	7.4976e-05\\
133.98	7.9809e-05\\
140.23	8.4892e-05\\
146.77	9.0228e-05\\
153.62	9.5822e-05\\
160.79	0.00010168\\
168.29	0.00010779\\
176.14	0.00011416\\
184.35	0.00012079\\
192.95	0.00012767\\
201.95	0.00013479\\
211.37	0.00014215\\
223.26	0.00015128\\
235.81	0.0001607\\
249.07	0.00017039\\
265.49	0.00018198\\
285.58	0.00019553\\
315.71	0.00021448\\
361.98	0.00024029\\
385.84	0.00025206\\
407.54	0.0002619\\
426.55	0.00026985\\
446.44	0.00027753\\
463.03	0.00028345\\
480.23	0.00028915\\
498.06	0.00029459\\
516.56	0.00029976\\
535.75	0.00030463\\
555.65	0.00030918\\
576.29	0.00031339\\
592.28	0.0003163\\
608.7	0.00031901\\
625.58	0.00032149\\
642.93	0.00032374\\
660.76	0.00032575\\
679.09	0.00032751\\
697.92	0.00032902\\
717.27	0.00033028\\
737.16	0.00033127\\
757.61	0.00033199\\
778.62	0.00033244\\
800.21	0.00033261\\
822.4	0.00033251\\
845.21	0.00033213\\
868.65	0.00033146\\
892.74	0.00033052\\
917.5	0.0003293\\
942.94	0.0003278\\
969.09	0.00032603\\
995.96	0.00032399\\
1023.6	0.00032169\\
1052	0.00031912\\
1081.1	0.0003163\\
1111.1	0.00031322\\
1141.9	0.0003099\\
1173.6	0.00030634\\
1206.2	0.00030255\\
1239.6	0.00029853\\
1274	0.00029428\\
1309.3	0.00028982\\
1345.6	0.00028515\\
1395.6	0.00027862\\
1447.4	0.00027174\\
1501.2	0.00026455\\
1557	0.00025706\\
1614.8	0.00024932\\
1690.1	0.00023931\\
1769	0.00022903\\
1868.5	0.00021642\\
2262.8	0.00017198\\
2368.3	0.00016175\\
2478.8	0.00015177\\
2594.4	0.0001421\\
2690.8	0.00013461\\
2790.8	0.00012735\\
2894.4	0.00012034\\
3001.9	0.00011358\\
3113.5	0.00010707\\
3229.1	0.00010083\\
3349.1	9.4848e-05\\
3473.5	8.9131e-05\\
3602.5	8.3676e-05\\
3736.3	7.8479e-05\\
3875.1	7.3536e-05\\
4019	6.8843e-05\\
4168.3	6.4394e-05\\
4323.2	6.0181e-05\\
4483.8	5.6199e-05\\
4650.3	5.2439e-05\\
4867.2	4.8041e-05\\
5094.3	4.3964e-05\\
5331.9	4.019e-05\\
5580.6	3.6704e-05\\
5840.9	3.349e-05\\
6113.4	3.0531e-05\\
6398.5	2.7811e-05\\
6758.4	2.4841e-05\\
7138.4	2.2167e-05\\
7539.8	1.9764e-05\\
7963.8	1.7608e-05\\
8488.7	1.5374e-05\\
9048.2	1.3414e-05\\
9732.9	1.1468e-05\\
10469	9.7966e-06\\
11365	8.1998e-06\\
12450	6.7234e-06\\
13763	5.3996e-06\\
15355	4.2462e-06\\
17287	3.268e-06\\
19821	2.4098e-06\\
23356	1.6641e-06\\
28285	1.0709e-06\\
35852	6.0983e-07\\
49331	2.7489e-07\\
78538	7.825e-08\\
2.0089e+05	4.2816e-09\\
2.5003e+05	2.0329e-09\\
};
\addlegendentry{Adatt. BLN appr. ($\mathit{ML}$)}


\addplot[area legend, draw=none, fill=mycolor2, fill opacity=0.15, forget plot]
table[row sep=crcr] {%
x	y\\
27.667	8.2361e-06\\
27.921	8.3634e-06\\
28.177	8.4924e-06\\
28.435	8.6231e-06\\
28.695	8.7557e-06\\
28.958	8.89e-06\\
29.223	9.0261e-06\\
29.491	9.1641e-06\\
29.761	9.304e-06\\
30.034	9.4457e-06\\
30.309	9.5893e-06\\
30.586	9.7349e-06\\
30.867	9.8824e-06\\
31.149	1.0032e-05\\
31.435	1.0183e-05\\
31.723	1.0337e-05\\
32.013	1.0492e-05\\
32.306	1.065e-05\\
32.602	1.081e-05\\
32.901	1.0972e-05\\
33.202	1.1136e-05\\
33.506	1.1302e-05\\
33.813	1.147e-05\\
34.123	1.1641e-05\\
34.436	1.1813e-05\\
34.751	1.1988e-05\\
35.069	1.2166e-05\\
35.391	1.2345e-05\\
35.715	1.2527e-05\\
36.042	1.2712e-05\\
36.372	1.2899e-05\\
36.705	1.3088e-05\\
37.042	1.328e-05\\
37.381	1.3474e-05\\
37.723	1.3671e-05\\
38.069	1.387e-05\\
38.417	1.4072e-05\\
38.769	1.4276e-05\\
39.124	1.4483e-05\\
39.483	1.4693e-05\\
39.845	1.4905e-05\\
40.209	1.5121e-05\\
40.578	1.5338e-05\\
40.949	1.5559e-05\\
41.325	1.5783e-05\\
41.703	1.6009e-05\\
42.085	1.6238e-05\\
42.471	1.6471e-05\\
42.86	1.6706e-05\\
43.252	1.6944e-05\\
43.648	1.7185e-05\\
44.048	1.7429e-05\\
44.452	1.7677e-05\\
44.859	1.7927e-05\\
45.27	1.8181e-05\\
45.684	1.8438e-05\\
46.103	1.8698e-05\\
46.525	1.8961e-05\\
46.951	1.9228e-05\\
47.381	1.9498e-05\\
47.815	1.9771e-05\\
48.253	2.0048e-05\\
48.695	2.0329e-05\\
49.141	2.0612e-05\\
49.592	2.09e-05\\
50.046	2.1191e-05\\
50.504	2.1485e-05\\
50.967	2.1784e-05\\
51.434	2.2086e-05\\
51.905	2.2391e-05\\
52.38	2.2701e-05\\
52.86	2.3014e-05\\
53.344	2.3331e-05\\
53.833	2.3653e-05\\
54.326	2.3978e-05\\
54.824	2.4307e-05\\
55.326	2.464e-05\\
55.833	2.4977e-05\\
56.344	2.5319e-05\\
56.86	2.5665e-05\\
57.381	2.6014e-05\\
57.907	2.6369e-05\\
58.437	2.6727e-05\\
58.972	2.709e-05\\
59.512	2.7458e-05\\
60.058	2.7829e-05\\
60.608	2.8206e-05\\
61.163	2.8587e-05\\
61.723	2.8972e-05\\
62.289	2.9363e-05\\
62.859	2.9758e-05\\
63.435	3.0158e-05\\
64.016	3.0562e-05\\
64.602	3.0972e-05\\
65.194	3.1386e-05\\
65.791	3.1806e-05\\
66.394	3.2231e-05\\
67.002	3.266e-05\\
67.616	3.3095e-05\\
68.235	3.3535e-05\\
68.86	3.3981e-05\\
69.491	3.4431e-05\\
70.127	3.4888e-05\\
70.77	3.5349e-05\\
71.418	3.5816e-05\\
72.072	3.6289e-05\\
72.732	3.6767e-05\\
73.399	3.7251e-05\\
74.071	3.7741e-05\\
74.749	3.8236e-05\\
75.434	3.8737e-05\\
76.125	3.9245e-05\\
76.822	3.9758e-05\\
77.526	4.0277e-05\\
78.236	4.0802e-05\\
78.953	4.1334e-05\\
79.676	4.1872e-05\\
80.406	4.2416e-05\\
81.142	4.2966e-05\\
81.886	4.3523e-05\\
82.636	4.4086e-05\\
83.393	4.4656e-05\\
84.156	4.5232e-05\\
84.927	4.5816e-05\\
85.705	4.6405e-05\\
86.49	4.7002e-05\\
87.283	4.7605e-05\\
88.082	4.8216e-05\\
88.889	4.8833e-05\\
89.703	4.9458e-05\\
90.525	5.0089e-05\\
91.354	5.0728e-05\\
92.191	5.1374e-05\\
93.035	5.2027e-05\\
93.887	5.2688e-05\\
94.747	5.3356e-05\\
95.615	5.4032e-05\\
96.491	5.4715e-05\\
97.375	5.5406e-05\\
98.267	5.6104e-05\\
99.167	5.6811e-05\\
100.08	5.7525e-05\\
100.99	5.8247e-05\\
101.92	5.8977e-05\\
102.85	5.9715e-05\\
103.79	6.0461e-05\\
104.74	6.1216e-05\\
105.7	6.1978e-05\\
106.67	6.2749e-05\\
107.65	6.3528e-05\\
108.63	6.4316e-05\\
109.63	6.5112e-05\\
110.63	6.5917e-05\\
111.65	6.673e-05\\
112.67	6.7552e-05\\
113.7	6.8383e-05\\
114.74	6.9222e-05\\
115.79	7.007e-05\\
116.85	7.0928e-05\\
117.93	7.1794e-05\\
119.01	7.2669e-05\\
120.1	7.3553e-05\\
121.2	7.4447e-05\\
122.31	7.5349e-05\\
123.43	7.6261e-05\\
124.56	7.7182e-05\\
125.7	7.8113e-05\\
126.85	7.9053e-05\\
128.01	8.0002e-05\\
129.18	8.0961e-05\\
130.37	8.1929e-05\\
131.56	8.2907e-05\\
132.77	8.3895e-05\\
133.98	8.4892e-05\\
135.21	8.5899e-05\\
136.45	8.6916e-05\\
137.7	8.7943e-05\\
138.96	8.898e-05\\
140.23	9.0026e-05\\
141.52	9.1082e-05\\
142.81	9.2149e-05\\
144.12	9.3225e-05\\
145.44	9.4312e-05\\
146.77	9.5408e-05\\
148.12	9.6515e-05\\
149.47	9.7631e-05\\
150.84	9.8758e-05\\
152.23	9.9895e-05\\
153.62	0.00010104\\
155.03	0.0001022\\
156.45	0.00010337\\
157.88	0.00010455\\
159.33	0.00010573\\
160.79	0.00010693\\
162.26	0.00010814\\
163.74	0.00010936\\
165.24	0.00011059\\
166.76	0.00011183\\
168.29	0.00011308\\
169.83	0.00011434\\
171.38	0.00011561\\
172.95	0.00011689\\
174.54	0.00011818\\
176.14	0.00011948\\
177.75	0.00012079\\
179.38	0.00012211\\
181.02	0.00012344\\
182.68	0.00012478\\
184.35	0.00012614\\
186.04	0.0001275\\
187.74	0.00012887\\
189.46	0.00013025\\
191.2	0.00013164\\
192.95	0.00013304\\
194.72	0.00013445\\
196.5	0.00013588\\
198.3	0.00013731\\
200.12	0.00013875\\
201.95	0.0001402\\
203.8	0.00014165\\
205.67	0.00014312\\
207.55	0.0001446\\
209.45	0.00014609\\
211.37	0.00014758\\
213.31	0.00014909\\
215.26	0.0001506\\
217.23	0.00015213\\
219.22	0.00015366\\
221.23	0.0001552\\
223.26	0.00015675\\
225.3	0.00015831\\
227.37	0.00015987\\
229.45	0.00016144\\
231.55	0.00016303\\
233.67	0.00016462\\
235.81	0.00016621\\
237.97	0.00016782\\
240.15	0.00016943\\
242.35	0.00017105\\
244.57	0.00017268\\
246.81	0.00017431\\
249.07	0.00017595\\
251.35	0.0001776\\
253.66	0.00017925\\
255.98	0.00018091\\
258.32	0.00018258\\
260.69	0.00018425\\
263.08	0.00018593\\
265.49	0.00018761\\
267.92	0.0001893\\
270.37	0.00019099\\
272.85	0.00019269\\
275.35	0.00019439\\
277.87	0.0001961\\
280.42	0.00019781\\
282.99	0.00019952\\
285.58	0.00020124\\
288.19	0.00020296\\
290.83	0.00020469\\
293.5	0.00020641\\
296.19	0.00020814\\
298.9	0.00020988\\
301.64	0.00021161\\
304.4	0.00021335\\
307.19	0.00021509\\
310	0.00021683\\
312.84	0.00021857\\
315.71	0.00022031\\
318.6	0.00022205\\
321.52	0.00022379\\
324.46	0.00022553\\
327.44	0.00022727\\
330.43	0.00022901\\
333.46	0.00023075\\
336.52	0.00023248\\
339.6	0.00023422\\
342.71	0.00023595\\
345.85	0.00023768\\
349.02	0.00023941\\
352.21	0.00024113\\
355.44	0.00024285\\
358.7	0.00024457\\
361.98	0.00024628\\
365.3	0.00024798\\
368.64	0.00024969\\
372.02	0.00025138\\
375.43	0.00025307\\
378.87	0.00025476\\
382.34	0.00025644\\
385.84	0.00025811\\
389.37	0.00025977\\
392.94	0.00026143\\
396.54	0.00026308\\
400.17	0.00026472\\
403.84	0.00026635\\
407.54	0.00026797\\
411.27	0.00026959\\
415.04	0.00027119\\
418.84	0.00027278\\
422.67	0.00027437\\
426.55	0.00027594\\
430.45	0.0002775\\
434.4	0.00027905\\
438.37	0.00028058\\
442.39	0.0002821\\
446.44	0.00028361\\
450.53	0.00028511\\
454.66	0.00028659\\
458.82	0.00028806\\
463.03	0.00028952\\
467.27	0.00029096\\
471.55	0.00029238\\
475.87	0.00029379\\
480.23	0.00029518\\
484.62	0.00029656\\
489.06	0.00029791\\
493.54	0.00029926\\
498.06	0.00030058\\
502.63	0.00030189\\
507.23	0.00030317\\
511.88	0.00030444\\
516.56	0.00030569\\
521.3	0.00030692\\
526.07	0.00030813\\
530.89	0.00030932\\
535.75	0.00031049\\
540.66	0.00031164\\
545.61	0.00031277\\
550.61	0.00031387\\
555.65	0.00031496\\
560.74	0.00031602\\
565.88	0.00031706\\
571.06	0.00031807\\
576.29	0.00031906\\
581.57	0.00032003\\
586.9	0.00032098\\
592.28	0.0003219\\
597.7	0.0003228\\
603.18	0.00032367\\
608.7	0.00032451\\
614.28	0.00032534\\
619.9	0.00032613\\
625.58	0.0003269\\
631.31	0.00032764\\
637.1	0.00032836\\
642.93	0.00032905\\
648.82	0.00032972\\
654.76	0.00033035\\
660.76	0.00033096\\
666.81	0.00033154\\
672.92	0.0003321\\
679.09	0.00033262\\
685.31	0.00033312\\
691.58	0.00033359\\
697.92	0.00033403\\
704.31	0.00033444\\
710.76	0.00033482\\
717.27	0.00033518\\
723.84	0.0003355\\
730.47	0.0003358\\
737.16	0.00033606\\
743.92	0.0003363\\
750.73	0.0003365\\
757.61	0.00033668\\
764.55	0.00033683\\
771.55	0.00033694\\
778.62	0.00033703\\
785.75	0.00033708\\
792.95	0.00033711\\
800.21	0.0003371\\
807.54	0.00033707\\
814.94	0.000337\\
822.4	0.0003369\\
829.94	0.00033678\\
837.54	0.00033662\\
845.21	0.00033643\\
852.95	0.00033621\\
860.76	0.00033596\\
868.65	0.00033567\\
876.61	0.00033536\\
884.64	0.00033502\\
892.74	0.00033464\\
900.92	0.00033423\\
909.17	0.0003338\\
917.5	0.00033333\\
925.9	0.00033283\\
934.38	0.0003323\\
942.94	0.00033174\\
951.58	0.00033115\\
960.29	0.00033053\\
969.09	0.00032988\\
977.97	0.0003292\\
986.92	0.00032849\\
995.96	0.00032774\\
1005.1	0.00032697\\
1014.3	0.00032617\\
1023.6	0.00032534\\
1033	0.00032448\\
1042.4	0.00032359\\
1052	0.00032267\\
1061.6	0.00032173\\
1071.3	0.00032075\\
1081.1	0.00031975\\
1091	0.00031872\\
1101	0.00031766\\
1111.1	0.00031658\\
1121.3	0.00031546\\
1131.6	0.00031433\\
1141.9	0.00031316\\
1152.4	0.00031197\\
1163	0.00031076\\
1173.6	0.00030952\\
1184.4	0.00030826\\
1195.2	0.00030697\\
1206.2	0.00030565\\
1217.2	0.00030432\\
1228.4	0.00030296\\
1239.6	0.00030158\\
1251	0.00030017\\
1262.4	0.00029875\\
1274	0.0002973\\
1285.7	0.00029583\\
1297.4	0.00029433\\
1309.3	0.00029282\\
1321.3	0.00029128\\
1333.4	0.00028973\\
1345.6	0.00028815\\
1357.9	0.00028655\\
1370.4	0.00028493\\
1382.9	0.00028329\\
1395.6	0.00028163\\
1408.4	0.00027995\\
1421.3	0.00027825\\
1434.3	0.00027653\\
1447.4	0.00027478\\
1460.7	0.00027302\\
1474.1	0.00027124\\
1487.6	0.00026943\\
1501.2	0.00026761\\
1515	0.00026577\\
1528.8	0.00026391\\
1542.8	0.00026203\\
1557	0.00026013\\
1571.2	0.00025821\\
1585.6	0.00025628\\
1600.2	0.00025433\\
1614.8	0.00025236\\
1629.6	0.00025037\\
1644.5	0.00024837\\
1659.6	0.00024636\\
1674.8	0.00024432\\
1690.1	0.00024228\\
1705.6	0.00024022\\
1721.2	0.00023815\\
1737	0.00023606\\
1752.9	0.00023397\\
1769	0.00023186\\
1785.2	0.00022974\\
1801.5	0.00022762\\
1818	0.00022548\\
1834.7	0.00022334\\
1851.5	0.00022119\\
1868.5	0.00021904\\
1885.6	0.00021688\\
1902.8	0.00021471\\
1920.3	0.00021254\\
1937.9	0.00021037\\
1955.6	0.0002082\\
1973.5	0.00020602\\
1991.6	0.00020384\\
2009.8	0.00020167\\
2028.3	0.00019949\\
2046.8	0.00019732\\
2065.6	0.00019514\\
2084.5	0.00019297\\
2103.6	0.00019081\\
2122.9	0.00018865\\
2142.3	0.00018649\\
2161.9	0.00018434\\
2181.7	0.00018219\\
2201.7	0.00018006\\
2221.9	0.00017793\\
2242.2	0.0001758\\
2262.8	0.00017369\\
2283.5	0.00017159\\
2304.4	0.00016949\\
2325.5	0.00016741\\
2346.8	0.00016533\\
2368.3	0.00016327\\
2390	0.00016122\\
2411.9	0.00015918\\
2434	0.00015715\\
2456.3	0.00015514\\
2478.8	0.00015314\\
2501.5	0.00015115\\
2524.4	0.00014918\\
2547.6	0.00014722\\
2570.9	0.00014527\\
2594.4	0.00014334\\
2618.2	0.00014143\\
2642.2	0.00013953\\
2666.4	0.00013764\\
2690.8	0.00013577\\
2715.5	0.00013392\\
2740.3	0.00013208\\
2765.4	0.00013026\\
2790.8	0.00012846\\
2816.3	0.00012667\\
2842.1	0.00012489\\
2868.2	0.00012314\\
2894.4	0.0001214\\
2920.9	0.00011967\\
2947.7	0.00011797\\
2974.7	0.00011628\\
3001.9	0.0001146\\
3029.4	0.00011294\\
3057.2	0.0001113\\
3085.2	0.00010968\\
3113.5	0.00010807\\
3142	0.00010648\\
3170.8	0.0001049\\
3199.8	0.00010335\\
3229.1	0.0001018\\
3258.7	0.00010028\\
3288.5	9.8769e-05\\
3318.7	9.7276e-05\\
3349.1	9.58e-05\\
3379.7	9.434e-05\\
3410.7	9.2896e-05\\
3441.9	9.1469e-05\\
3473.5	9.0058e-05\\
3505.3	8.8663e-05\\
3537.4	8.7284e-05\\
3569.8	8.5921e-05\\
3602.5	8.4574e-05\\
3635.5	8.3243e-05\\
3668.8	8.1928e-05\\
3702.4	8.0629e-05\\
3736.3	7.9346e-05\\
3770.5	7.8078e-05\\
3805.1	7.6826e-05\\
3839.9	7.559e-05\\
3875.1	7.4369e-05\\
3910.6	7.3163e-05\\
3946.4	7.1973e-05\\
3982.6	7.0798e-05\\
4019	6.9639e-05\\
4055.9	6.8494e-05\\
4093	6.7364e-05\\
4130.5	6.625e-05\\
4168.3	6.515e-05\\
4206.5	6.4065e-05\\
4245	6.2994e-05\\
4283.9	6.1939e-05\\
4323.2	6.0897e-05\\
4362.8	5.987e-05\\
4402.7	5.8857e-05\\
4443.1	5.7858e-05\\
4483.8	5.6874e-05\\
4524.8	5.5903e-05\\
4566.3	5.4946e-05\\
4608.1	5.4002e-05\\
4650.3	5.3073e-05\\
4692.9	5.2156e-05\\
4735.9	5.1253e-05\\
4779.3	5.0363e-05\\
4823.1	4.9487e-05\\
4867.2	4.8623e-05\\
4911.8	4.7772e-05\\
4956.8	4.6934e-05\\
5002.2	4.6109e-05\\
5048	4.5296e-05\\
5094.3	4.4496e-05\\
5140.9	4.3707e-05\\
5188	4.2931e-05\\
5235.5	4.2167e-05\\
5283.5	4.1415e-05\\
5331.9	4.0674e-05\\
5380.7	3.9945e-05\\
5430	3.9228e-05\\
5479.8	3.8522e-05\\
5530	3.7827e-05\\
5580.6	3.7144e-05\\
5631.7	3.6471e-05\\
5683.3	3.5809e-05\\
5735.4	3.5158e-05\\
5787.9	3.4517e-05\\
5840.9	3.3887e-05\\
5894.4	3.3268e-05\\
5948.4	3.2658e-05\\
6002.9	3.2059e-05\\
6057.9	3.147e-05\\
6113.4	3.089e-05\\
6169.4	3.032e-05\\
6225.9	2.976e-05\\
6282.9	2.9209e-05\\
6340.5	2.8668e-05\\
6398.5	2.8135e-05\\
6457.2	2.7612e-05\\
6516.3	2.7098e-05\\
6576	2.6593e-05\\
6636.2	2.6096e-05\\
6697	2.5608e-05\\
6758.4	2.5128e-05\\
6820.3	2.4657e-05\\
6882.7	2.4194e-05\\
6945.8	2.3739e-05\\
7009.4	2.3292e-05\\
7073.6	2.2853e-05\\
7138.4	2.2422e-05\\
7203.8	2.1998e-05\\
7269.8	2.1582e-05\\
7336.4	2.1174e-05\\
7403.6	2.0772e-05\\
7471.4	2.0378e-05\\
7539.8	1.9991e-05\\
7608.9	1.9611e-05\\
7678.6	1.9238e-05\\
7748.9	1.8871e-05\\
7819.9	1.8511e-05\\
7891.5	1.8158e-05\\
7963.8	1.7811e-05\\
8036.8	1.7471e-05\\
8110.4	1.7137e-05\\
8184.7	1.6808e-05\\
8259.6	1.6486e-05\\
8335.3	1.617e-05\\
8411.6	1.586e-05\\
8488.7	1.5555e-05\\
8566.5	1.5256e-05\\
8644.9	1.4963e-05\\
8724.1	1.4675e-05\\
8804	1.4392e-05\\
8884.7	1.4115e-05\\
8966	1.3842e-05\\
9048.2	1.3575e-05\\
9131.1	1.3313e-05\\
9214.7	1.3056e-05\\
9299.1	1.2804e-05\\
9384.3	1.2556e-05\\
9470.2	1.2313e-05\\
9557	1.2075e-05\\
9644.5	1.1841e-05\\
9732.9	1.1611e-05\\
9822	1.1386e-05\\
9912	1.1165e-05\\
10003	1.0949e-05\\
10094	1.0736e-05\\
10187	1.0527e-05\\
10280	1.0323e-05\\
10374	1.0122e-05\\
10469	9.9251e-06\\
10565	9.732e-06\\
10662	9.5425e-06\\
10760	9.3566e-06\\
10858	9.1743e-06\\
10958	8.9955e-06\\
11058	8.8201e-06\\
11159	8.648e-06\\
11262	8.4792e-06\\
11365	8.3137e-06\\
11469	8.1513e-06\\
11574	7.992e-06\\
11680	7.8358e-06\\
11787	7.6826e-06\\
11895	7.5323e-06\\
12004	7.3849e-06\\
12114	7.2403e-06\\
12225	7.0986e-06\\
12337	6.9595e-06\\
12450	6.8231e-06\\
12564	6.6893e-06\\
12679	6.5581e-06\\
12795	6.4295e-06\\
12912	6.3033e-06\\
13030	6.1795e-06\\
13150	6.0581e-06\\
13270	5.9391e-06\\
13392	5.8223e-06\\
13515	5.7078e-06\\
13638	5.5955e-06\\
13763	5.4854e-06\\
13889	5.3774e-06\\
14017	5.2714e-06\\
14145	5.1676e-06\\
14274	5.0657e-06\\
14405	4.9658e-06\\
14537	4.8678e-06\\
14670	4.7717e-06\\
14805	4.6775e-06\\
14940	4.585e-06\\
15077	4.4944e-06\\
15215	4.4055e-06\\
15355	4.3183e-06\\
15495	4.2329e-06\\
15637	4.149e-06\\
15780	4.0668e-06\\
15925	3.9862e-06\\
16071	3.9071e-06\\
16218	3.8296e-06\\
16367	3.7535e-06\\
16517	3.6789e-06\\
16668	3.6058e-06\\
16821	3.5341e-06\\
16975	3.4637e-06\\
17130	3.3948e-06\\
17287	3.3271e-06\\
17445	3.2608e-06\\
17605	3.1957e-06\\
17766	3.1319e-06\\
17929	3.0694e-06\\
18093	3.008e-06\\
18259	2.9478e-06\\
18426	2.8888e-06\\
18595	2.831e-06\\
18765	2.7742e-06\\
18937	2.7186e-06\\
19111	2.664e-06\\
19286	2.6105e-06\\
19463	2.558e-06\\
19641	2.5065e-06\\
19821	2.4561e-06\\
20002	2.4066e-06\\
20186	2.358e-06\\
20370	2.3105e-06\\
20557	2.2638e-06\\
20745	2.218e-06\\
20935	2.1731e-06\\
21127	2.1291e-06\\
21321	2.086e-06\\
21516	2.0436e-06\\
21713	2.0021e-06\\
21912	1.9614e-06\\
22113	1.9215e-06\\
22315	1.8824e-06\\
22520	1.844e-06\\
22726	1.8064e-06\\
22934	1.7695e-06\\
23144	1.7333e-06\\
23356	1.6979e-06\\
23570	1.6631e-06\\
23786	1.629e-06\\
24004	1.5955e-06\\
24224	1.5627e-06\\
24446	1.5306e-06\\
24669	1.499e-06\\
24895	1.4681e-06\\
25123	1.4378e-06\\
25354	1.4081e-06\\
25586	1.379e-06\\
25820	1.3504e-06\\
26057	1.3224e-06\\
26295	1.2949e-06\\
26536	1.268e-06\\
26779	1.2416e-06\\
27025	1.2157e-06\\
27272	1.1903e-06\\
27522	1.1655e-06\\
27774	1.1411e-06\\
28028	1.1172e-06\\
28285	1.0937e-06\\
28544	1.0707e-06\\
28806	1.0482e-06\\
29070	1.0261e-06\\
29336	1.0045e-06\\
29605	9.8328e-07\\
29876	9.6248e-07\\
30149	9.421e-07\\
30426	9.2211e-07\\
30704	9.0253e-07\\
30986	8.8333e-07\\
31269	8.6451e-07\\
31556	8.4607e-07\\
31845	8.2799e-07\\
32137	8.1027e-07\\
32431	7.9291e-07\\
32728	7.7589e-07\\
33028	7.5922e-07\\
33330	7.4287e-07\\
33636	7.2686e-07\\
33944	7.1116e-07\\
34255	6.9578e-07\\
34568	6.8071e-07\\
34885	6.6594e-07\\
35205	6.5147e-07\\
35527	6.3729e-07\\
35852	6.234e-07\\
36181	6.0979e-07\\
36512	5.9646e-07\\
36847	5.8339e-07\\
37184	5.7059e-07\\
37525	5.5805e-07\\
37869	5.4577e-07\\
38215	5.3374e-07\\
38565	5.2195e-07\\
38919	5.104e-07\\
39275	4.9909e-07\\
39635	4.8802e-07\\
39998	4.7717e-07\\
40364	4.6654e-07\\
40734	4.5613e-07\\
41107	4.4594e-07\\
41484	4.3596e-07\\
41864	4.2619e-07\\
42247	4.1662e-07\\
42634	4.0724e-07\\
43025	3.9807e-07\\
43419	3.8908e-07\\
43817	3.8028e-07\\
44218	3.7167e-07\\
44623	3.6323e-07\\
45032	3.5498e-07\\
45444	3.469e-07\\
45860	3.3898e-07\\
46281	3.3124e-07\\
46704	3.2366e-07\\
47132	3.1624e-07\\
47564	3.0897e-07\\
48000	3.0187e-07\\
48439	2.9491e-07\\
48883	2.881e-07\\
49331	2.8144e-07\\
49783	2.7492e-07\\
50239	2.6854e-07\\
50699	2.623e-07\\
51163	2.5619e-07\\
51632	2.5021e-07\\
52105	2.4437e-07\\
52582	2.3865e-07\\
53064	2.3305e-07\\
53550	2.2758e-07\\
54040	2.2222e-07\\
54535	2.1699e-07\\
55035	2.1187e-07\\
55539	2.0685e-07\\
56048	2.0195e-07\\
56561	1.9716e-07\\
57079	1.9248e-07\\
57602	1.8789e-07\\
58130	1.8341e-07\\
58662	1.7903e-07\\
59200	1.7475e-07\\
59742	1.7056e-07\\
60289	1.6646e-07\\
60841	1.6246e-07\\
61399	1.5854e-07\\
61961	1.5472e-07\\
62529	1.5098e-07\\
63101	1.4732e-07\\
63679	1.4375e-07\\
64263	1.4025e-07\\
64851	1.3684e-07\\
65445	1.335e-07\\
66045	1.3024e-07\\
66650	1.2706e-07\\
67260	1.2395e-07\\
67876	1.209e-07\\
68498	1.1793e-07\\
69125	1.1503e-07\\
69759	1.1219e-07\\
70398	1.0942e-07\\
71042	1.0671e-07\\
71693	1.0407e-07\\
72350	1.0148e-07\\
73013	9.8958e-08\\
73681	9.6493e-08\\
74356	9.4086e-08\\
75037	9.1734e-08\\
75725	8.9438e-08\\
76418	8.7195e-08\\
77118	8.5005e-08\\
77825	8.2867e-08\\
78538	8.0779e-08\\
79257	7.874e-08\\
79983	7.6749e-08\\
80716	7.4805e-08\\
81455	7.2908e-08\\
82201	7.1056e-08\\
82954	6.9248e-08\\
83714	6.7482e-08\\
84481	6.576e-08\\
85255	6.4078e-08\\
86036	6.2437e-08\\
86824	6.0835e-08\\
87619	5.9271e-08\\
88421	5.7746e-08\\
89231	5.6257e-08\\
90049	5.4804e-08\\
90874	5.3387e-08\\
91706	5.2003e-08\\
92546	5.0654e-08\\
93394	4.9337e-08\\
94249	4.8053e-08\\
95113	4.68e-08\\
95984	4.5577e-08\\
96863	4.4385e-08\\
97750	4.3222e-08\\
98646	4.2088e-08\\
99549	4.0981e-08\\
1.0046e+05	3.9902e-08\\
1.0138e+05	3.885e-08\\
1.0231e+05	3.7824e-08\\
1.0325e+05	3.6823e-08\\
1.0419e+05	3.5847e-08\\
1.0515e+05	3.4895e-08\\
1.0611e+05	3.3968e-08\\
1.0708e+05	3.3063e-08\\
1.0806e+05	3.2181e-08\\
1.0905e+05	3.1322e-08\\
1.1005e+05	3.0483e-08\\
1.1106e+05	2.9667e-08\\
1.1208e+05	2.887e-08\\
1.131e+05	2.8094e-08\\
1.1414e+05	2.7337e-08\\
1.1519e+05	2.66e-08\\
1.1624e+05	2.5882e-08\\
1.1731e+05	2.5181e-08\\
1.1838e+05	2.4499e-08\\
1.1946e+05	2.3834e-08\\
1.2056e+05	2.3186e-08\\
1.2166e+05	2.2555e-08\\
1.2278e+05	2.1939e-08\\
1.239e+05	2.134e-08\\
1.2504e+05	2.0756e-08\\
1.2618e+05	2.0188e-08\\
1.2734e+05	1.9634e-08\\
1.285e+05	1.9094e-08\\
1.2968e+05	1.8568e-08\\
1.3087e+05	1.8056e-08\\
1.3207e+05	1.7558e-08\\
1.3328e+05	1.7072e-08\\
1.345e+05	1.6599e-08\\
1.3573e+05	1.6139e-08\\
1.3697e+05	1.569e-08\\
1.3823e+05	1.5253e-08\\
1.3949e+05	1.4828e-08\\
1.4077e+05	1.4414e-08\\
1.4206e+05	1.4011e-08\\
1.4336e+05	1.3619e-08\\
1.4468e+05	1.3237e-08\\
1.46e+05	1.2865e-08\\
1.4734e+05	1.2503e-08\\
1.4869e+05	1.2151e-08\\
1.5005e+05	1.1808e-08\\
1.5142e+05	1.1475e-08\\
1.5281e+05	1.115e-08\\
1.5421e+05	1.0834e-08\\
1.5562e+05	1.0526e-08\\
1.5705e+05	1.0227e-08\\
1.5849e+05	9.9358e-09\\
1.5994e+05	9.6525e-09\\
1.614e+05	9.3768e-09\\
1.6288e+05	9.1086e-09\\
1.6438e+05	8.8477e-09\\
1.6588e+05	8.5939e-09\\
1.674e+05	8.3471e-09\\
1.6893e+05	8.1069e-09\\
1.7048e+05	7.8734e-09\\
1.7204e+05	7.6462e-09\\
1.7362e+05	7.4252e-09\\
1.7521e+05	7.2104e-09\\
1.7681e+05	7.0014e-09\\
1.7843e+05	6.7982e-09\\
1.8007e+05	6.6006e-09\\
1.8172e+05	6.4085e-09\\
1.8338e+05	6.2217e-09\\
1.8506e+05	6.0401e-09\\
1.8676e+05	5.8635e-09\\
1.8847e+05	5.6919e-09\\
1.9019e+05	5.525e-09\\
1.9194e+05	5.3628e-09\\
1.9369e+05	5.2052e-09\\
1.9547e+05	5.0519e-09\\
1.9726e+05	4.903e-09\\
1.9907e+05	4.7583e-09\\
2.0089e+05	4.6176e-09\\
2.0273e+05	4.4809e-09\\
2.0459e+05	4.348e-09\\
2.0646e+05	4.2189e-09\\
2.0835e+05	4.0935e-09\\
2.1026e+05	3.9717e-09\\
2.1219e+05	3.8533e-09\\
2.1413e+05	3.7382e-09\\
2.1609e+05	3.6265e-09\\
2.1807e+05	3.5179e-09\\
2.2007e+05	3.4125e-09\\
2.2208e+05	3.3101e-09\\
2.2412e+05	3.2106e-09\\
2.2617e+05	3.114e-09\\
2.2824e+05	3.0201e-09\\
2.3033e+05	2.929e-09\\
2.3244e+05	2.8405e-09\\
2.3457e+05	2.7545e-09\\
2.3672e+05	2.6711e-09\\
2.3889e+05	2.5901e-09\\
2.4108e+05	2.5114e-09\\
2.4329e+05	2.435e-09\\
2.4551e+05	2.3608e-09\\
2.4776e+05	2.2888e-09\\
2.5003e+05	2.2189e-09\\
2.5003e+05	1.8469e-09\\
2.4776e+05	1.9072e-09\\
2.4551e+05	1.9694e-09\\
2.4329e+05	2.0335e-09\\
2.4108e+05	2.0996e-09\\
2.3889e+05	2.1678e-09\\
2.3672e+05	2.2381e-09\\
2.3457e+05	2.3106e-09\\
2.3244e+05	2.3853e-09\\
2.3033e+05	2.4624e-09\\
2.2824e+05	2.5418e-09\\
2.2617e+05	2.6236e-09\\
2.2412e+05	2.708e-09\\
2.2208e+05	2.7949e-09\\
2.2007e+05	2.8846e-09\\
2.1807e+05	2.9769e-09\\
2.1609e+05	3.0721e-09\\
2.1413e+05	3.1702e-09\\
2.1219e+05	3.2713e-09\\
2.1026e+05	3.3754e-09\\
2.0835e+05	3.4828e-09\\
2.0646e+05	3.5933e-09\\
2.0459e+05	3.7073e-09\\
2.0273e+05	3.8246e-09\\
2.0089e+05	3.9455e-09\\
1.9907e+05	4.0701e-09\\
1.9726e+05	4.1984e-09\\
1.9547e+05	4.3305e-09\\
1.9369e+05	4.4666e-09\\
1.9194e+05	4.6068e-09\\
1.9019e+05	4.7511e-09\\
1.8847e+05	4.8998e-09\\
1.8676e+05	5.0529e-09\\
1.8506e+05	5.2105e-09\\
1.8338e+05	5.3728e-09\\
1.8172e+05	5.5399e-09\\
1.8007e+05	5.712e-09\\
1.7843e+05	5.8891e-09\\
1.7681e+05	6.0715e-09\\
1.7521e+05	6.2592e-09\\
1.7362e+05	6.4525e-09\\
1.7204e+05	6.6514e-09\\
1.7048e+05	6.8561e-09\\
1.6893e+05	7.0668e-09\\
1.674e+05	7.2837e-09\\
1.6588e+05	7.5068e-09\\
1.6438e+05	7.7365e-09\\
1.6288e+05	7.9728e-09\\
1.614e+05	8.2159e-09\\
1.5994e+05	8.4661e-09\\
1.5849e+05	8.7235e-09\\
1.5705e+05	8.9883e-09\\
1.5562e+05	9.2607e-09\\
1.5421e+05	9.5409e-09\\
1.5281e+05	9.8292e-09\\
1.5142e+05	1.0126e-08\\
1.5005e+05	1.0431e-08\\
1.4869e+05	1.0744e-08\\
1.4734e+05	1.1067e-08\\
1.46e+05	1.1399e-08\\
1.4468e+05	1.174e-08\\
1.4336e+05	1.209e-08\\
1.4206e+05	1.2451e-08\\
1.4077e+05	1.2822e-08\\
1.3949e+05	1.3203e-08\\
1.3823e+05	1.3595e-08\\
1.3697e+05	1.3998e-08\\
1.3573e+05	1.4412e-08\\
1.345e+05	1.4838e-08\\
1.3328e+05	1.5275e-08\\
1.3207e+05	1.5725e-08\\
1.3087e+05	1.6187e-08\\
1.2968e+05	1.6662e-08\\
1.285e+05	1.715e-08\\
1.2734e+05	1.7651e-08\\
1.2618e+05	1.8167e-08\\
1.2504e+05	1.8696e-08\\
1.239e+05	1.924e-08\\
1.2278e+05	1.9799e-08\\
1.2166e+05	2.0373e-08\\
1.2056e+05	2.0962e-08\\
1.1946e+05	2.1568e-08\\
1.1838e+05	2.219e-08\\
1.1731e+05	2.2829e-08\\
1.1624e+05	2.3485e-08\\
1.1519e+05	2.4159e-08\\
1.1414e+05	2.4851e-08\\
1.131e+05	2.5561e-08\\
1.1208e+05	2.6291e-08\\
1.1106e+05	2.704e-08\\
1.1005e+05	2.7809e-08\\
1.0905e+05	2.8598e-08\\
1.0806e+05	2.9408e-08\\
1.0708e+05	3.024e-08\\
1.0611e+05	3.1094e-08\\
1.0515e+05	3.197e-08\\
1.0419e+05	3.2869e-08\\
1.0325e+05	3.3792e-08\\
1.0231e+05	3.4739e-08\\
1.0138e+05	3.5711e-08\\
1.0046e+05	3.6708e-08\\
99549	3.7731e-08\\
98646	3.878e-08\\
97750	3.9857e-08\\
96863	4.0961e-08\\
95984	4.2094e-08\\
95113	4.3256e-08\\
94249	4.4448e-08\\
93394	4.567e-08\\
92546	4.6924e-08\\
91706	4.8209e-08\\
90874	4.9527e-08\\
90049	5.0879e-08\\
89231	5.2265e-08\\
88421	5.3685e-08\\
87619	5.5142e-08\\
86824	5.6635e-08\\
86036	5.8165e-08\\
85255	5.9734e-08\\
84481	6.1342e-08\\
83714	6.299e-08\\
82954	6.4678e-08\\
82201	6.6409e-08\\
81455	6.8182e-08\\
80716	6.9999e-08\\
79983	7.186e-08\\
79257	7.3768e-08\\
78538	7.5721e-08\\
77825	7.7723e-08\\
77118	7.9773e-08\\
76418	8.1873e-08\\
75725	8.4024e-08\\
75037	8.6227e-08\\
74356	8.8483e-08\\
73681	9.0794e-08\\
73013	9.3159e-08\\
72350	9.5582e-08\\
71693	9.8062e-08\\
71042	1.006e-07\\
70398	1.032e-07\\
69759	1.0586e-07\\
69125	1.0859e-07\\
68498	1.1138e-07\\
67876	1.1423e-07\\
67260	1.1715e-07\\
66650	1.2014e-07\\
66045	1.232e-07\\
65445	1.2633e-07\\
64851	1.2954e-07\\
64263	1.3282e-07\\
63679	1.3617e-07\\
63101	1.396e-07\\
62529	1.4312e-07\\
61961	1.4671e-07\\
61399	1.5038e-07\\
60841	1.5414e-07\\
60289	1.5799e-07\\
59742	1.6192e-07\\
59200	1.6594e-07\\
58662	1.7006e-07\\
58130	1.7427e-07\\
57602	1.7857e-07\\
57079	1.8297e-07\\
56561	1.8747e-07\\
56048	1.9207e-07\\
55539	1.9677e-07\\
55035	2.0158e-07\\
54535	2.065e-07\\
54040	2.1152e-07\\
53550	2.1666e-07\\
53064	2.2191e-07\\
52582	2.2728e-07\\
52105	2.3277e-07\\
51632	2.3838e-07\\
51163	2.4411e-07\\
50699	2.4997e-07\\
50239	2.5596e-07\\
49783	2.6208e-07\\
49331	2.6833e-07\\
48883	2.7472e-07\\
48439	2.8125e-07\\
48000	2.8792e-07\\
47564	2.9474e-07\\
47132	3.017e-07\\
46704	3.0882e-07\\
46281	3.1609e-07\\
45860	3.2351e-07\\
45444	3.311e-07\\
45032	3.3885e-07\\
44623	3.4676e-07\\
44218	3.5485e-07\\
43817	3.6311e-07\\
43419	3.7154e-07\\
43025	3.8016e-07\\
42634	3.8896e-07\\
42247	3.9794e-07\\
41864	4.0712e-07\\
41484	4.1649e-07\\
41107	4.2606e-07\\
40734	4.3583e-07\\
40364	4.4581e-07\\
39998	4.56e-07\\
39635	4.664e-07\\
39275	4.7703e-07\\
38919	4.8787e-07\\
38565	4.9894e-07\\
38215	5.1025e-07\\
37869	5.2179e-07\\
37525	5.3357e-07\\
37184	5.456e-07\\
36847	5.5788e-07\\
36512	5.7041e-07\\
36181	5.832e-07\\
35852	5.9626e-07\\
35527	6.0959e-07\\
35205	6.232e-07\\
34885	6.3709e-07\\
34568	6.5126e-07\\
34255	6.6572e-07\\
33944	6.8049e-07\\
33636	6.9556e-07\\
33330	7.1093e-07\\
33028	7.2663e-07\\
32728	7.4264e-07\\
32431	7.5898e-07\\
32137	7.7566e-07\\
31845	7.9268e-07\\
31556	8.1005e-07\\
31269	8.2777e-07\\
30986	8.4585e-07\\
30704	8.643e-07\\
30426	8.8312e-07\\
30149	9.0233e-07\\
29876	9.2193e-07\\
29605	9.4193e-07\\
29336	9.6233e-07\\
29070	9.8315e-07\\
28806	1.0044e-06\\
28544	1.0261e-06\\
28285	1.0482e-06\\
28028	1.0707e-06\\
27774	1.0937e-06\\
27522	1.1172e-06\\
27272	1.1411e-06\\
27025	1.1656e-06\\
26779	1.1905e-06\\
26536	1.2159e-06\\
26295	1.2418e-06\\
26057	1.2683e-06\\
25820	1.2953e-06\\
25586	1.3228e-06\\
25354	1.3509e-06\\
25123	1.3795e-06\\
24895	1.4088e-06\\
24669	1.4386e-06\\
24446	1.469e-06\\
24224	1.5e-06\\
24004	1.5316e-06\\
23786	1.5639e-06\\
23570	1.5968e-06\\
23356	1.6304e-06\\
23144	1.6646e-06\\
22934	1.6995e-06\\
22726	1.7352e-06\\
22520	1.7715e-06\\
22315	1.8086e-06\\
22113	1.8464e-06\\
21912	1.885e-06\\
21713	1.9243e-06\\
21516	1.9644e-06\\
21321	2.0053e-06\\
21127	2.0471e-06\\
20935	2.0897e-06\\
20745	2.1331e-06\\
20557	2.1774e-06\\
20370	2.2225e-06\\
20186	2.2686e-06\\
20002	2.3156e-06\\
19821	2.3636e-06\\
19641	2.4124e-06\\
19463	2.4623e-06\\
19286	2.5132e-06\\
19111	2.5651e-06\\
18937	2.618e-06\\
18765	2.6719e-06\\
18595	2.727e-06\\
18426	2.7831e-06\\
18259	2.8404e-06\\
18093	2.8988e-06\\
17929	2.9584e-06\\
17766	3.0192e-06\\
17605	3.0811e-06\\
17445	3.1443e-06\\
17287	3.2088e-06\\
17130	3.2746e-06\\
16975	3.3417e-06\\
16821	3.4101e-06\\
16668	3.4799e-06\\
16517	3.551e-06\\
16367	3.6236e-06\\
16218	3.6977e-06\\
16071	3.7732e-06\\
15925	3.8502e-06\\
15780	3.9288e-06\\
15637	4.0089e-06\\
15495	4.0906e-06\\
15355	4.174e-06\\
15215	4.259e-06\\
15077	4.3457e-06\\
14940	4.4342e-06\\
14805	4.5244e-06\\
14670	4.6164e-06\\
14537	4.7102e-06\\
14405	4.8059e-06\\
14274	4.9036e-06\\
14145	5.0031e-06\\
14017	5.1047e-06\\
13889	5.2082e-06\\
13763	5.3138e-06\\
13638	5.4216e-06\\
13515	5.5314e-06\\
13392	5.6435e-06\\
13270	5.7577e-06\\
13150	5.8743e-06\\
13030	5.9931e-06\\
12912	6.1144e-06\\
12795	6.238e-06\\
12679	6.364e-06\\
12564	6.4926e-06\\
12450	6.6237e-06\\
12337	6.7574e-06\\
12225	6.8938e-06\\
12114	7.0328e-06\\
12004	7.1746e-06\\
11895	7.3191e-06\\
11787	7.4666e-06\\
11680	7.6169e-06\\
11574	7.7702e-06\\
11469	7.9265e-06\\
11365	8.0859e-06\\
11262	8.2484e-06\\
11159	8.414e-06\\
11058	8.583e-06\\
10958	8.7552e-06\\
10858	8.9308e-06\\
10760	9.1098e-06\\
10662	9.2923e-06\\
10565	9.4784e-06\\
10469	9.6681e-06\\
10374	9.8615e-06\\
10280	1.0059e-05\\
10187	1.026e-05\\
10094	1.0464e-05\\
10003	1.0673e-05\\
9912	1.0886e-05\\
9822	1.1103e-05\\
9732.9	1.1324e-05\\
9644.5	1.1549e-05\\
9557	1.1779e-05\\
9470.2	1.2013e-05\\
9384.3	1.2252e-05\\
9299.1	1.2495e-05\\
9214.7	1.2742e-05\\
9131.1	1.2995e-05\\
9048.2	1.3252e-05\\
8966	1.3514e-05\\
8884.7	1.3781e-05\\
8804	1.4053e-05\\
8724.1	1.433e-05\\
8644.9	1.4613e-05\\
8566.5	1.49e-05\\
8488.7	1.5194e-05\\
8411.6	1.5492e-05\\
8335.3	1.5796e-05\\
8259.6	1.6106e-05\\
8184.7	1.6422e-05\\
8110.4	1.6743e-05\\
8036.8	1.707e-05\\
7963.8	1.7404e-05\\
7891.5	1.7743e-05\\
7819.9	1.8089e-05\\
7748.9	1.8441e-05\\
7678.6	1.88e-05\\
7608.9	1.9165e-05\\
7539.8	1.9536e-05\\
7471.4	1.9915e-05\\
7403.6	2.03e-05\\
7336.4	2.0692e-05\\
7269.8	2.1092e-05\\
7203.8	2.1498e-05\\
7138.4	2.1912e-05\\
7073.6	2.2333e-05\\
7009.4	2.2762e-05\\
6945.8	2.3198e-05\\
6882.7	2.3642e-05\\
6820.3	2.4094e-05\\
6758.4	2.4554e-05\\
6697	2.5022e-05\\
6636.2	2.5498e-05\\
6576	2.5982e-05\\
6516.3	2.6475e-05\\
6457.2	2.6977e-05\\
6398.5	2.7487e-05\\
6340.5	2.8006e-05\\
6282.9	2.8533e-05\\
6225.9	2.907e-05\\
6169.4	2.9616e-05\\
6113.4	3.0172e-05\\
6057.9	3.0736e-05\\
6002.9	3.1311e-05\\
5948.4	3.1895e-05\\
5894.4	3.2489e-05\\
5840.9	3.3092e-05\\
5787.9	3.3706e-05\\
5735.4	3.433e-05\\
5683.3	3.4965e-05\\
5631.7	3.561e-05\\
5580.6	3.6265e-05\\
5530	3.6931e-05\\
5479.8	3.7608e-05\\
5430	3.8296e-05\\
5380.7	3.8995e-05\\
5331.9	3.9706e-05\\
5283.5	4.0428e-05\\
5235.5	4.1161e-05\\
5188	4.1906e-05\\
5140.9	4.2663e-05\\
5094.3	4.3432e-05\\
5048	4.4213e-05\\
5002.2	4.5006e-05\\
4956.8	4.5811e-05\\
4911.8	4.6629e-05\\
4867.2	4.746e-05\\
4823.1	4.8303e-05\\
4779.3	4.9159e-05\\
4735.9	5.0028e-05\\
4692.9	5.0911e-05\\
4650.3	5.1806e-05\\
4608.1	5.2715e-05\\
4566.3	5.3638e-05\\
4524.8	5.4574e-05\\
4483.8	5.5524e-05\\
4443.1	5.6488e-05\\
4402.7	5.7466e-05\\
4362.8	5.8459e-05\\
4323.2	5.9465e-05\\
4283.9	6.0486e-05\\
4245	6.1522e-05\\
4206.5	6.2572e-05\\
4168.3	6.3637e-05\\
4130.5	6.4717e-05\\
4093	6.5812e-05\\
4055.9	6.6922e-05\\
4019	6.8048e-05\\
3982.6	6.9188e-05\\
3946.4	7.0344e-05\\
3910.6	7.1516e-05\\
3875.1	7.2704e-05\\
3839.9	7.3907e-05\\
3805.1	7.5126e-05\\
3770.5	7.6361e-05\\
3736.3	7.7612e-05\\
3702.4	7.8879e-05\\
3668.8	8.0162e-05\\
3635.5	8.1461e-05\\
3602.5	8.2777e-05\\
3569.8	8.4109e-05\\
3537.4	8.5458e-05\\
3505.3	8.6823e-05\\
3473.5	8.8204e-05\\
3441.9	8.9602e-05\\
3410.7	9.1017e-05\\
3379.7	9.2448e-05\\
3349.1	9.3896e-05\\
3318.7	9.536e-05\\
3288.5	9.6841e-05\\
3258.7	9.8338e-05\\
3229.1	9.9852e-05\\
3199.8	0.00010138\\
3170.8	0.00010293\\
3142	0.00010449\\
3113.5	0.00010607\\
3085.2	0.00010767\\
3057.2	0.00010928\\
3029.4	0.00011091\\
3001.9	0.00011255\\
2974.7	0.00011421\\
2947.7	0.00011588\\
2920.9	0.00011757\\
2894.4	0.00011928\\
2868.2	0.000121\\
2842.1	0.00012274\\
2816.3	0.00012449\\
2790.8	0.00012625\\
2765.4	0.00012803\\
2740.3	0.00012982\\
2715.5	0.00013163\\
2690.8	0.00013345\\
2666.4	0.00013529\\
2642.2	0.00013713\\
2618.2	0.00013899\\
2594.4	0.00014087\\
2570.9	0.00014275\\
2547.6	0.00014465\\
2524.4	0.00014656\\
2501.5	0.00014848\\
2478.8	0.00015041\\
2456.3	0.00015235\\
2434	0.00015431\\
2411.9	0.00015627\\
2390	0.00015824\\
2368.3	0.00016023\\
2346.8	0.00016222\\
2325.5	0.00016422\\
2304.4	0.00016623\\
2283.5	0.00016825\\
2262.8	0.00017027\\
2242.2	0.00017231\\
2221.9	0.00017434\\
2201.7	0.00017639\\
2181.7	0.00017844\\
2161.9	0.0001805\\
2142.3	0.00018256\\
2122.9	0.00018463\\
2103.6	0.0001867\\
2084.5	0.00018878\\
2065.6	0.00019086\\
2046.8	0.00019294\\
2028.3	0.00019503\\
2009.8	0.00019711\\
1991.6	0.0001992\\
1973.5	0.00020129\\
1955.6	0.00020337\\
1937.9	0.00020546\\
1920.3	0.00020755\\
1902.8	0.00020963\\
1885.6	0.00021172\\
1868.5	0.0002138\\
1851.5	0.00021587\\
1834.7	0.00021795\\
1818	0.00022002\\
1801.5	0.00022208\\
1785.2	0.00022414\\
1769	0.00022619\\
1752.9	0.00022824\\
1737	0.00023028\\
1721.2	0.00023231\\
1705.6	0.00023433\\
1690.1	0.00023635\\
1674.8	0.00023836\\
1659.6	0.00024035\\
1644.5	0.00024234\\
1629.6	0.00024431\\
1614.8	0.00024627\\
1600.2	0.00024822\\
1585.6	0.00025016\\
1571.2	0.00025209\\
1557	0.000254\\
1542.8	0.00025589\\
1528.8	0.00025777\\
1515	0.00025964\\
1501.2	0.00026148\\
1487.6	0.00026332\\
1474.1	0.00026513\\
1460.7	0.00026692\\
1447.4	0.0002687\\
1434.3	0.00027046\\
1421.3	0.00027219\\
1408.4	0.00027391\\
1395.6	0.0002756\\
1382.9	0.00027728\\
1370.4	0.00027893\\
1357.9	0.00028055\\
1345.6	0.00028216\\
1333.4	0.00028374\\
1321.3	0.00028529\\
1309.3	0.00028683\\
1297.4	0.00028833\\
1285.7	0.00028981\\
1274	0.00029127\\
1262.4	0.0002927\\
1251	0.0002941\\
1239.6	0.00029547\\
1228.4	0.00029682\\
1217.2	0.00029815\\
1206.2	0.00029944\\
1195.2	0.00030071\\
1184.4	0.00030195\\
1173.6	0.00030316\\
1163	0.00030435\\
1152.4	0.00030551\\
1141.9	0.00030664\\
1131.6	0.00030774\\
1121.3	0.00030882\\
1111.1	0.00030987\\
1101	0.00031089\\
1091	0.00031188\\
1081.1	0.00031284\\
1071.3	0.00031378\\
1061.6	0.00031469\\
1052	0.00031557\\
1042.4	0.00031642\\
1033	0.00031724\\
1023.6	0.00031804\\
1014.3	0.0003188\\
1005.1	0.00031954\\
995.96	0.00032024\\
986.92	0.00032092\\
977.97	0.00032157\\
969.09	0.00032219\\
960.29	0.00032278\\
951.58	0.00032334\\
942.94	0.00032387\\
934.38	0.00032437\\
925.9	0.00032484\\
917.5	0.00032527\\
909.17	0.00032568\\
900.92	0.00032606\\
892.74	0.0003264\\
884.64	0.00032672\\
876.61	0.000327\\
868.65	0.00032726\\
860.76	0.00032748\\
852.95	0.00032767\\
845.21	0.00032783\\
837.54	0.00032795\\
829.94	0.00032805\\
822.4	0.00032812\\
814.94	0.00032815\\
807.54	0.00032815\\
800.21	0.00032812\\
792.95	0.00032806\\
785.75	0.00032797\\
778.62	0.00032785\\
771.55	0.0003277\\
764.55	0.00032751\\
757.61	0.0003273\\
750.73	0.00032706\\
743.92	0.00032678\\
737.16	0.00032647\\
730.47	0.00032614\\
723.84	0.00032577\\
717.27	0.00032538\\
710.76	0.00032496\\
704.31	0.0003245\\
697.92	0.00032402\\
691.58	0.00032351\\
685.31	0.00032297\\
679.09	0.0003224\\
672.92	0.00032181\\
666.81	0.00032118\\
660.76	0.00032053\\
654.76	0.00031986\\
648.82	0.00031915\\
642.93	0.00031842\\
637.1	0.00031766\\
631.31	0.00031688\\
625.58	0.00031607\\
619.9	0.00031524\\
614.28	0.00031438\\
608.7	0.0003135\\
603.18	0.00031259\\
597.7	0.00031166\\
592.28	0.00031071\\
586.9	0.00030973\\
581.57	0.00030873\\
576.29	0.00030771\\
571.06	0.00030667\\
565.88	0.0003056\\
560.74	0.00030451\\
555.65	0.00030341\\
550.61	0.00030228\\
545.61	0.00030113\\
540.66	0.00029996\\
535.75	0.00029877\\
530.89	0.00029756\\
526.07	0.00029634\\
521.3	0.00029509\\
516.56	0.00029383\\
511.88	0.00029255\\
507.23	0.00029125\\
502.63	0.00028993\\
498.06	0.0002886\\
493.54	0.00028725\\
489.06	0.00028589\\
484.62	0.00028451\\
480.23	0.00028311\\
475.87	0.0002817\\
471.55	0.00028028\\
467.27	0.00027884\\
463.03	0.00027739\\
458.82	0.00027592\\
454.66	0.00027444\\
450.53	0.00027295\\
446.44	0.00027145\\
442.39	0.00026994\\
438.37	0.00026841\\
434.4	0.00026687\\
430.45	0.00026532\\
426.55	0.00026376\\
422.67	0.00026219\\
418.84	0.00026061\\
415.04	0.00025903\\
411.27	0.00025743\\
407.54	0.00025582\\
403.84	0.00025421\\
400.17	0.00025258\\
396.54	0.00025095\\
392.94	0.00024932\\
389.37	0.00024767\\
385.84	0.00024602\\
382.34	0.00024436\\
378.87	0.0002427\\
375.43	0.00024103\\
372.02	0.00023935\\
368.64	0.00023767\\
365.3	0.00023599\\
361.98	0.0002343\\
358.7	0.00023261\\
355.44	0.00023091\\
352.21	0.00022921\\
349.02	0.00022751\\
345.85	0.0002258\\
342.71	0.00022409\\
339.6	0.00022238\\
336.52	0.00022067\\
333.46	0.00021895\\
330.43	0.00021724\\
327.44	0.00021552\\
324.46	0.0002138\\
321.52	0.00021208\\
318.6	0.00021037\\
315.71	0.00020865\\
312.84	0.00020693\\
310	0.00020521\\
307.19	0.00020349\\
304.4	0.00020178\\
301.64	0.00020007\\
298.9	0.00019835\\
296.19	0.00019664\\
293.5	0.00019493\\
290.83	0.00019323\\
288.19	0.00019153\\
285.58	0.00018983\\
282.99	0.00018813\\
280.42	0.00018644\\
277.87	0.00018475\\
275.35	0.00018306\\
272.85	0.00018138\\
270.37	0.0001797\\
267.92	0.00017803\\
265.49	0.00017636\\
263.08	0.00017469\\
260.69	0.00017304\\
258.32	0.00017138\\
255.98	0.00016974\\
253.66	0.00016809\\
251.35	0.00016646\\
249.07	0.00016483\\
246.81	0.0001632\\
244.57	0.00016159\\
242.35	0.00015998\\
240.15	0.00015837\\
237.97	0.00015678\\
235.81	0.00015519\\
233.67	0.0001536\\
231.55	0.00015203\\
229.45	0.00015046\\
227.37	0.0001489\\
225.3	0.00014735\\
223.26	0.00014581\\
221.23	0.00014427\\
219.22	0.00014274\\
217.23	0.00014123\\
215.26	0.00013972\\
213.31	0.00013821\\
211.37	0.00013672\\
209.45	0.00013524\\
207.55	0.00013376\\
205.67	0.00013229\\
203.8	0.00013084\\
201.95	0.00012939\\
200.12	0.00012795\\
198.3	0.00012652\\
196.5	0.0001251\\
194.72	0.00012369\\
192.95	0.00012229\\
191.2	0.0001209\\
189.46	0.00011952\\
187.74	0.00011815\\
186.04	0.00011679\\
184.35	0.00011544\\
182.68	0.0001141\\
181.02	0.00011277\\
179.38	0.00011145\\
177.75	0.00011014\\
176.14	0.00010884\\
174.54	0.00010755\\
172.95	0.00010627\\
171.38	0.00010501\\
169.83	0.00010375\\
168.29	0.0001025\\
166.76	0.00010126\\
165.24	0.00010004\\
163.74	9.882e-05\\
162.26	9.7615e-05\\
160.79	9.6419e-05\\
159.33	9.5235e-05\\
157.88	9.4061e-05\\
156.45	9.2897e-05\\
155.03	9.1745e-05\\
153.62	9.0602e-05\\
152.23	8.9471e-05\\
150.84	8.8349e-05\\
149.47	8.7239e-05\\
148.12	8.6138e-05\\
146.77	8.5049e-05\\
145.44	8.397e-05\\
144.12	8.2901e-05\\
142.81	8.1843e-05\\
141.52	8.0795e-05\\
140.23	7.9758e-05\\
138.96	7.8731e-05\\
137.7	7.7714e-05\\
136.45	7.6708e-05\\
135.21	7.5712e-05\\
133.98	7.4726e-05\\
132.77	7.375e-05\\
131.56	7.2785e-05\\
130.37	7.183e-05\\
129.18	7.0885e-05\\
128.01	6.995e-05\\
126.85	6.9025e-05\\
125.7	6.811e-05\\
124.56	6.7204e-05\\
123.43	6.6309e-05\\
122.31	6.5424e-05\\
121.2	6.4548e-05\\
120.1	6.3682e-05\\
119.01	6.2826e-05\\
117.93	6.1979e-05\\
116.85	6.1142e-05\\
115.79	6.0314e-05\\
114.74	5.9496e-05\\
113.7	5.8687e-05\\
112.67	5.7888e-05\\
111.65	5.7097e-05\\
110.63	5.6316e-05\\
109.63	5.5544e-05\\
108.63	5.4781e-05\\
107.65	5.4027e-05\\
106.67	5.3282e-05\\
105.7	5.2546e-05\\
104.74	5.1819e-05\\
103.79	5.11e-05\\
102.85	5.039e-05\\
101.92	4.9689e-05\\
100.99	4.8996e-05\\
100.08	4.8312e-05\\
99.167	4.7636e-05\\
98.267	4.6968e-05\\
97.375	4.6309e-05\\
96.491	4.5657e-05\\
95.615	4.5014e-05\\
94.747	4.4379e-05\\
93.887	4.3752e-05\\
93.035	4.3133e-05\\
92.191	4.2521e-05\\
91.354	4.1917e-05\\
90.525	4.1321e-05\\
89.703	4.0733e-05\\
88.889	4.0152e-05\\
88.082	3.9578e-05\\
87.283	3.9012e-05\\
86.49	3.8453e-05\\
85.705	3.7902e-05\\
84.927	3.7357e-05\\
84.156	3.682e-05\\
83.393	3.6289e-05\\
82.636	3.5766e-05\\
81.886	3.5249e-05\\
81.142	3.474e-05\\
80.406	3.4237e-05\\
79.676	3.374e-05\\
78.953	3.325e-05\\
78.236	3.2767e-05\\
77.526	3.229e-05\\
76.822	3.182e-05\\
76.125	3.1355e-05\\
75.434	3.0897e-05\\
74.749	3.0445e-05\\
74.071	3e-05\\
73.399	2.956e-05\\
72.732	2.9126e-05\\
72.072	2.8698e-05\\
71.418	2.8276e-05\\
70.77	2.786e-05\\
70.127	2.7449e-05\\
69.491	2.7044e-05\\
68.86	2.6644e-05\\
68.235	2.625e-05\\
67.616	2.5862e-05\\
67.002	2.5478e-05\\
66.394	2.51e-05\\
65.791	2.4728e-05\\
65.194	2.436e-05\\
64.602	2.3997e-05\\
64.016	2.364e-05\\
63.435	2.3287e-05\\
62.859	2.294e-05\\
62.289	2.2597e-05\\
61.723	2.2259e-05\\
61.163	2.1926e-05\\
60.608	2.1597e-05\\
60.058	2.1273e-05\\
59.512	2.0953e-05\\
58.972	2.0638e-05\\
58.437	2.0328e-05\\
57.907	2.0022e-05\\
57.381	1.972e-05\\
56.86	1.9422e-05\\
56.344	1.9129e-05\\
55.833	1.884e-05\\
55.326	1.8554e-05\\
54.824	1.8273e-05\\
54.326	1.7996e-05\\
53.833	1.7723e-05\\
53.344	1.7454e-05\\
52.86	1.7188e-05\\
52.38	1.6927e-05\\
51.905	1.6669e-05\\
51.434	1.6414e-05\\
50.967	1.6164e-05\\
50.504	1.5917e-05\\
50.046	1.5673e-05\\
49.592	1.5433e-05\\
49.141	1.5197e-05\\
48.695	1.4964e-05\\
48.253	1.4734e-05\\
47.815	1.4507e-05\\
47.381	1.4284e-05\\
46.951	1.4064e-05\\
46.525	1.3847e-05\\
46.103	1.3634e-05\\
45.684	1.3423e-05\\
45.27	1.3216e-05\\
44.859	1.3011e-05\\
44.452	1.2809e-05\\
44.048	1.2611e-05\\
43.648	1.2415e-05\\
43.252	1.2222e-05\\
42.86	1.2032e-05\\
42.471	1.1844e-05\\
42.085	1.166e-05\\
41.703	1.1478e-05\\
41.325	1.1298e-05\\
40.949	1.1122e-05\\
40.578	1.0948e-05\\
40.209	1.0776e-05\\
39.845	1.0607e-05\\
39.483	1.044e-05\\
39.124	1.0276e-05\\
38.769	1.0114e-05\\
38.417	9.9551e-06\\
38.069	9.7981e-06\\
37.723	9.6433e-06\\
37.381	9.4909e-06\\
37.042	9.3406e-06\\
36.705	9.1926e-06\\
36.372	9.0468e-06\\
36.042	8.9031e-06\\
35.715	8.7615e-06\\
35.391	8.622e-06\\
35.069	8.4846e-06\\
34.751	8.3492e-06\\
34.436	8.2158e-06\\
34.123	8.0844e-06\\
33.813	7.9549e-06\\
33.506	7.8274e-06\\
33.202	7.7017e-06\\
32.901	7.5779e-06\\
32.602	7.456e-06\\
32.306	7.3358e-06\\
32.013	7.2175e-06\\
31.723	7.1009e-06\\
31.435	6.986e-06\\
31.149	6.8729e-06\\
30.867	6.7614e-06\\
30.586	6.6516e-06\\
30.309	6.5435e-06\\
30.034	6.437e-06\\
29.761	6.332e-06\\
29.491	6.2287e-06\\
29.223	6.1269e-06\\
28.958	6.0266e-06\\
28.695	5.9278e-06\\
28.435	5.8305e-06\\
28.177	5.7347e-06\\
27.921	5.6403e-06\\
27.667	5.5474e-06\\
}--cycle;
    
        %     \nextgroupplot[%
        %         % xmin=100,xmax=1000000,
        %         ymin=0,ymax=0.0004,
        %     ] % This file was created by matlab2tikz.
%
\definecolor{mycolor1}{rgb}{0.00000,0.44706,0.69804}%
%
\addplot[ybar interval, fill=mycolor1, fill opacity=0.35, area legend, draw=none] table[row sep=crcr, x=Lower, y=Count] {%
Lower	Upper	Count\\
29.871	42.524	0\\
42.524	60.539	1.0362e-05\\
60.539	86.185	2.4955e-05\\
86.185	122.69	4.3095e-05\\
122.69	174.67	8.7222e-05\\
174.67	248.67	0.00013947\\
248.67	354.01	0.00021139\\
354.01	503.97	0.00027402\\
503.97	717.47	0.00032876\\
717.47	1021.4	0.00033069\\
1021.4	1454.1	0.00029176\\
1454.1	2070.1	0.00022676\\
2070.1	2947	0.00015299\\
2947	4195.5	9.0099e-05\\
4195.5	5972.8	4.3348e-05\\
5972.8	8503	2.1469e-05\\
8503	12105	1.0661e-05\\
12105	17233	4.9246e-06\\
17233	24534	2.4584e-06\\
24534	34927	9.9816e-07\\
34927	49722	3.6228e-07\\
49722	70786	1.6712e-07\\
70786	1.0077e+05	4.0018e-08\\
1.0077e+05	1.4346e+05	2.4988e-08\\
1.4346e+05	2.0424e+05	2.6766e-08\\
2.0424e+05	2.0424e+05	2.6766e-08\\
};
\addlegendentry{Istogramma esatto}

\addplot [color=mycolor1, only marks, every error bar/.append style={opacity=0.45}, mark=*, mark size=0pt, draw=none, forget plot]
 plot [error bars/.cd, y dir=both, y explicit, error bar style={line width=1pt, color=mycolor1}, error mark options={mark=none,mark size=0pt}]
 table[row sep=crcr, y error plus index=2, y error minus index=3]{%
35.64	0	0	0\\
50.738	1.0362e-05	7.532e-06	7.532e-06\\
72.232	2.4955e-05	9.7789e-06	9.7789e-06\\
102.83	4.3095e-05	1.1487e-05	1.1487e-05\\
146.39	8.7222e-05	1.1439e-05	1.1439e-05\\
208.41	0.00013947	1.2922e-05	1.2922e-05\\
296.7	0.00021139	1.4379e-05	1.4379e-05\\
422.39	0.00027402	1.3278e-05	1.3278e-05\\
601.32	0.00032876	1.3347e-05	1.3347e-05\\
856.06	0.00033069	1.058e-05	1.058e-05\\
1218.7	0.00029176	7.3766e-06	7.3766e-06\\
1735	0.00022676	5.0392e-06	5.0392e-06\\
2470	0.00015299	4.1912e-06	4.1912e-06\\
3516.3	9.0099e-05	2.3474e-06	2.3474e-06\\
5005.9	4.3348e-05	1.5021e-06	1.5021e-06\\
7126.5	2.1469e-05	7.8454e-07	7.8454e-07\\
10145	1.0661e-05	5.1317e-07	5.1317e-07\\
14443	4.9246e-06	2.8187e-07	2.8187e-07\\
20562	2.4584e-06	1.5887e-07	1.5887e-07\\
29272	9.9816e-07	8.5309e-08	8.5309e-08\\
41673	3.6228e-07	4.1762e-08	4.1762e-08\\
59327	1.6712e-07	2.3919e-08	2.3919e-08\\
84459	4.0018e-08	1.0751e-08	1.0751e-08\\
1.2024e+05	2.4988e-08	6.3523e-09	6.3523e-09\\
1.7117e+05	2.6766e-08	4.2679e-09	4.2679e-09\\
};
\addplot [color=mycolor1]
  table[row sep=crcr]{%
29.871	6.5645e-06\\
34.105	8.173e-06\\
38.258	9.8633e-06\\
42.539	1.1718e-05\\
46.883	1.3714e-05\\
51.216	1.5814e-05\\
55.949	1.8232e-05\\
60.582	2.0718e-05\\
65.021	2.3208e-05\\
69.785	2.5993e-05\\
74.898	2.9106e-05\\
79.679	3.2127e-05\\
84.764	3.5452e-05\\
90.175	3.9107e-05\\
95.086	4.2522e-05\\
100.26	4.6218e-05\\
105.73	5.021e-05\\
111.48	5.4516e-05\\
117.56	5.9154e-05\\
123.96	6.4139e-05\\
129.56	6.8569e-05\\
135.41	7.3258e-05\\
141.53	7.8215e-05\\
147.93	8.3444e-05\\
154.61	8.8951e-05\\
161.6	9.474e-05\\
168.9	0.00010081\\
176.53	0.00010717\\
184.51	0.00011381\\
192.85	0.00012073\\
201.56	0.00012792\\
210.67	0.00013538\\
222.14	0.00014466\\
234.24	0.00015429\\
247	0.00016423\\
260.45	0.00017445\\
277.08	0.00018667\\
300.02	0.00020274\\
384.27	0.00025309\\
405.2	0.00026346\\
427.27	0.00027352\\
446.58	0.00028158\\
466.75	0.00028931\\
483.55	0.00029522\\
500.95	0.00030086\\
518.98	0.0003062\\
537.66	0.00031122\\
557.01	0.00031589\\
577.06	0.0003202\\
597.82	0.00032411\\
613.89	0.00032677\\
630.38	0.0003292\\
647.32	0.00033137\\
664.72	0.00033329\\
682.58	0.00033495\\
700.92	0.00033635\\
719.76	0.00033747\\
739.1	0.00033831\\
758.96	0.00033887\\
779.35	0.00033915\\
800.3	0.00033914\\
821.8	0.00033884\\
843.88	0.00033826\\
866.56	0.00033738\\
889.85	0.00033622\\
913.76	0.00033477\\
938.31	0.00033304\\
963.53	0.00033102\\
989.42	0.00032874\\
1016	0.00032618\\
1043.3	0.00032335\\
1071.3	0.00032027\\
1100.1	0.00031694\\
1129.7	0.00031336\\
1160.1	0.00030955\\
1201.8	0.00030412\\
1245.1	0.00029832\\
1289.9	0.00029216\\
1336.3	0.00028569\\
1384.4	0.00027891\\
1434.2	0.00027187\\
1499	0.00026273\\
1566.7	0.00025329\\
1652.1	0.00024162\\
1757.5	0.0002277\\
1971.5	0.00020142\\
2153.7	0.00018136\\
2291.2	0.00016762\\
2416	0.00015614\\
2547.6	0.000145\\
2662.7	0.00013602\\
2783	0.00012734\\
2908.7	0.00011899\\
3040.2	0.00011097\\
3177.6	0.00010331\\
3321.1	9.6004e-05\\
3471.2	8.9068e-05\\
3628.1	8.25e-05\\
3792	7.6299e-05\\
3963.3	7.0462e-05\\
4142.4	6.4981e-05\\
4329.6	5.9848e-05\\
4525.2	5.5052e-05\\
4729.7	5.0582e-05\\
4943.4	4.6424e-05\\
5166.8	4.2566e-05\\
5400.3	3.8991e-05\\
5644.3	3.5686e-05\\
5951.7	3.2054e-05\\
6275.9	2.8763e-05\\
6617.7	2.5786e-05\\
6978.1	2.3099e-05\\
7423.5	2.0298e-05\\
7897.4	1.7823e-05\\
8401.4	1.5639e-05\\
9017	1.3461e-05\\
9677.7	1.158e-05\\
10479	9.7706e-06\\
11447	8.0859e-06\\
12616	6.5626e-06\\
14028	5.2225e-06\\
15736	4.0732e-06\\
17967	3.0516e-06\\
21066	2.1491e-06\\
25363	1.416e-06\\
31915	8.3081e-07\\
42725	4.079e-07\\
64729	1.3648e-07\\
1.3481e+05	1.5173e-08\\
2.0424e+05	3.767e-09\\
};
\addlegendentry{Adatt. BLN esatto ($\mathit{ML}$)}


\addplot[area legend, draw=none, fill=mycolor1, fill opacity=0.15, forget plot]
table[row sep=crcr] {%
x	y\\
29.871	8.3469e-06\\
30.136	8.4643e-06\\
30.403	8.5832e-06\\
30.673	8.7037e-06\\
30.946	8.8256e-06\\
31.22	8.9492e-06\\
31.497	9.0743e-06\\
31.777	9.2009e-06\\
32.059	9.3293e-06\\
32.344	9.4592e-06\\
32.631	9.5908e-06\\
32.921	9.7241e-06\\
33.213	9.859e-06\\
33.508	9.9957e-06\\
33.805	1.0134e-05\\
34.105	1.0274e-05\\
34.408	1.0416e-05\\
34.714	1.056e-05\\
35.022	1.0706e-05\\
35.333	1.0853e-05\\
35.647	1.1002e-05\\
35.963	1.1153e-05\\
36.282	1.1307e-05\\
36.604	1.1462e-05\\
36.929	1.1619e-05\\
37.257	1.1778e-05\\
37.588	1.1939e-05\\
37.922	1.2102e-05\\
38.258	1.2267e-05\\
38.598	1.2434e-05\\
38.941	1.2603e-05\\
39.287	1.2775e-05\\
39.635	1.2948e-05\\
39.987	1.3124e-05\\
40.342	1.3302e-05\\
40.7	1.3483e-05\\
41.062	1.3665e-05\\
41.426	1.385e-05\\
41.794	1.4038e-05\\
42.165	1.4227e-05\\
42.539	1.4419e-05\\
42.917	1.4614e-05\\
43.298	1.4811e-05\\
43.683	1.501e-05\\
44.07	1.5212e-05\\
44.462	1.5417e-05\\
44.856	1.5624e-05\\
45.255	1.5834e-05\\
45.656	1.6046e-05\\
46.062	1.6261e-05\\
46.471	1.6479e-05\\
46.883	1.67e-05\\
47.3	1.6924e-05\\
47.719	1.715e-05\\
48.143	1.7379e-05\\
48.571	1.7611e-05\\
49.002	1.7846e-05\\
49.437	1.8084e-05\\
49.876	1.8326e-05\\
50.319	1.857e-05\\
50.765	1.8817e-05\\
51.216	1.9068e-05\\
51.671	1.9321e-05\\
52.129	1.9578e-05\\
52.592	1.9838e-05\\
53.059	2.0102e-05\\
53.53	2.0369e-05\\
54.005	2.0639e-05\\
54.485	2.0913e-05\\
54.969	2.119e-05\\
55.457	2.1471e-05\\
55.949	2.1755e-05\\
56.446	2.2043e-05\\
56.947	2.2335e-05\\
57.453	2.263e-05\\
57.963	2.2929e-05\\
58.477	2.3232e-05\\
58.996	2.3539e-05\\
59.52	2.385e-05\\
60.049	2.4165e-05\\
60.582	2.4484e-05\\
61.12	2.4807e-05\\
61.662	2.5134e-05\\
62.21	2.5465e-05\\
62.762	2.58e-05\\
63.319	2.614e-05\\
63.881	2.6484e-05\\
64.448	2.6833e-05\\
65.021	2.7186e-05\\
65.598	2.7543e-05\\
66.18	2.7905e-05\\
66.768	2.8272e-05\\
67.361	2.8643e-05\\
67.959	2.902e-05\\
68.562	2.94e-05\\
69.171	2.9786e-05\\
69.785	3.0177e-05\\
70.404	3.0573e-05\\
71.029	3.0974e-05\\
71.66	3.1379e-05\\
72.296	3.1791e-05\\
72.938	3.2207e-05\\
73.586	3.2628e-05\\
74.239	3.3055e-05\\
74.898	3.3488e-05\\
75.563	3.3926e-05\\
76.234	3.4369e-05\\
76.911	3.4818e-05\\
77.594	3.5273e-05\\
78.282	3.5734e-05\\
78.977	3.62e-05\\
79.679	3.6673e-05\\
80.386	3.7151e-05\\
81.1	3.7635e-05\\
81.82	3.8126e-05\\
82.546	3.8622e-05\\
83.279	3.9125e-05\\
84.018	3.9634e-05\\
84.764	4.015e-05\\
85.517	4.0672e-05\\
86.276	4.1201e-05\\
87.042	4.1736e-05\\
87.815	4.2278e-05\\
88.595	4.2826e-05\\
89.381	4.3382e-05\\
90.175	4.3944e-05\\
90.975	4.4514e-05\\
91.783	4.509e-05\\
92.598	4.5674e-05\\
93.42	4.6264e-05\\
94.249	4.6863e-05\\
95.086	4.7468e-05\\
95.93	4.8081e-05\\
96.782	4.8701e-05\\
97.641	4.9329e-05\\
98.508	4.9965e-05\\
99.383	5.0608e-05\\
100.26	5.1259e-05\\
101.16	5.1918e-05\\
102.05	5.2585e-05\\
102.96	5.3261e-05\\
103.87	5.3944e-05\\
104.8	5.4635e-05\\
105.73	5.5335e-05\\
106.66	5.6043e-05\\
107.61	5.6759e-05\\
108.57	5.7484e-05\\
109.53	5.8218e-05\\
110.5	5.896e-05\\
111.48	5.9711e-05\\
112.47	6.0471e-05\\
113.47	6.1239e-05\\
114.48	6.2017e-05\\
115.5	6.2803e-05\\
116.52	6.3599e-05\\
117.56	6.4404e-05\\
118.6	6.5218e-05\\
119.65	6.6041e-05\\
120.72	6.6874e-05\\
121.79	6.7716e-05\\
122.87	6.8568e-05\\
123.96	6.9429e-05\\
125.06	7.03e-05\\
126.17	7.1181e-05\\
127.29	7.2071e-05\\
128.42	7.2972e-05\\
129.56	7.3882e-05\\
130.71	7.4802e-05\\
131.87	7.5733e-05\\
133.04	7.6673e-05\\
134.22	7.7624e-05\\
135.41	7.8584e-05\\
136.62	7.9556e-05\\
137.83	8.0537e-05\\
139.05	8.1529e-05\\
140.29	8.2531e-05\\
141.53	8.3544e-05\\
142.79	8.4567e-05\\
144.06	8.5601e-05\\
145.34	8.6646e-05\\
146.63	8.7701e-05\\
147.93	8.8768e-05\\
149.24	8.9844e-05\\
150.57	9.0932e-05\\
151.9	9.2031e-05\\
153.25	9.314e-05\\
154.61	9.4261e-05\\
155.99	9.5392e-05\\
157.37	9.6535e-05\\
158.77	9.7688e-05\\
160.18	9.8853e-05\\
161.6	0.00010003\\
163.03	0.00010121\\
164.48	0.00010241\\
165.94	0.00010362\\
167.42	0.00010484\\
168.9	0.00010607\\
170.4	0.00010732\\
171.91	0.00010857\\
173.44	0.00010983\\
174.98	0.00011111\\
176.53	0.0001124\\
178.1	0.0001137\\
179.68	0.00011501\\
181.28	0.00011633\\
182.89	0.00011766\\
184.51	0.000119\\
186.15	0.00012036\\
187.8	0.00012172\\
189.47	0.0001231\\
191.15	0.00012449\\
192.85	0.00012588\\
194.56	0.00012729\\
196.29	0.00012871\\
198.03	0.00013014\\
199.79	0.00013159\\
201.56	0.00013304\\
203.35	0.0001345\\
205.16	0.00013598\\
206.98	0.00013746\\
208.82	0.00013896\\
210.67	0.00014046\\
212.54	0.00014198\\
214.43	0.00014351\\
216.33	0.00014504\\
218.25	0.00014659\\
220.19	0.00014814\\
222.14	0.00014971\\
224.12	0.00015129\\
226.11	0.00015287\\
228.11	0.00015447\\
230.14	0.00015607\\
232.18	0.00015769\\
234.24	0.00015931\\
236.32	0.00016094\\
238.42	0.00016258\\
240.54	0.00016423\\
242.67	0.00016589\\
244.83	0.00016756\\
247	0.00016924\\
249.19	0.00017092\\
251.41	0.00017261\\
253.64	0.00017431\\
255.89	0.00017602\\
258.16	0.00017773\\
260.45	0.00017945\\
262.77	0.00018118\\
265.1	0.00018291\\
267.45	0.00018465\\
269.83	0.0001864\\
272.22	0.00018815\\
274.64	0.00018991\\
277.08	0.00019168\\
279.54	0.00019345\\
282.02	0.00019522\\
284.52	0.000197\\
287.05	0.00019879\\
289.6	0.00020058\\
292.17	0.00020237\\
294.76	0.00020417\\
297.38	0.00020597\\
300.02	0.00020777\\
302.68	0.00020958\\
305.37	0.00021139\\
308.08	0.0002132\\
310.82	0.00021501\\
313.58	0.00021683\\
316.36	0.00021865\\
319.17	0.00022046\\
322	0.00022228\\
324.86	0.0002241\\
327.75	0.00022592\\
330.66	0.00022774\\
333.59	0.00022956\\
336.55	0.00023137\\
339.54	0.00023319\\
342.56	0.000235\\
345.6	0.00023681\\
348.67	0.00023862\\
351.76	0.00024043\\
354.88	0.00024223\\
358.04	0.00024403\\
361.21	0.00024583\\
364.42	0.00024762\\
367.66	0.00024941\\
370.92	0.00025119\\
374.21	0.00025297\\
377.54	0.00025474\\
380.89	0.0002565\\
384.27	0.00025826\\
387.68	0.00026001\\
391.12	0.00026176\\
394.6	0.00026349\\
398.1	0.00026522\\
401.63	0.00026694\\
405.2	0.00026865\\
408.8	0.00027035\\
412.43	0.00027204\\
416.09	0.00027372\\
419.78	0.00027539\\
423.51	0.00027705\\
427.27	0.0002787\\
431.06	0.00028033\\
434.89	0.00028196\\
438.75	0.00028357\\
442.65	0.00028516\\
446.58	0.00028675\\
450.54	0.00028832\\
454.54	0.00028987\\
458.58	0.00029141\\
462.65	0.00029294\\
466.75	0.00029445\\
470.9	0.00029594\\
475.08	0.00029742\\
479.3	0.00029888\\
483.55	0.00030033\\
487.84	0.00030175\\
492.18	0.00030316\\
496.55	0.00030455\\
500.95	0.00030592\\
505.4	0.00030727\\
509.89	0.00030861\\
514.42	0.00030992\\
518.98	0.00031121\\
523.59	0.00031248\\
528.24	0.00031373\\
532.93	0.00031496\\
537.66	0.00031617\\
542.43	0.00031736\\
547.25	0.00031852\\
552.11	0.00031966\\
557.01	0.00032078\\
561.95	0.00032187\\
566.94	0.00032294\\
571.98	0.00032399\\
577.06	0.00032501\\
582.18	0.00032601\\
587.35	0.00032698\\
592.56	0.00032793\\
597.82	0.00032885\\
603.13	0.00032974\\
608.49	0.00033061\\
613.89	0.00033146\\
619.34	0.00033227\\
624.84	0.00033306\\
630.38	0.00033382\\
635.98	0.00033455\\
641.63	0.00033526\\
647.32	0.00033593\\
653.07	0.00033658\\
658.87	0.0003372\\
664.72	0.00033779\\
670.62	0.00033836\\
676.57	0.00033889\\
682.58	0.00033939\\
688.64	0.00033986\\
694.75	0.00034031\\
700.92	0.00034072\\
707.14	0.0003411\\
713.42	0.00034146\\
719.76	0.00034178\\
726.15	0.00034207\\
732.59	0.00034233\\
739.1	0.00034256\\
745.66	0.00034275\\
752.28	0.00034292\\
758.96	0.00034305\\
765.7	0.00034315\\
772.5	0.00034323\\
779.35	0.00034326\\
786.27	0.00034327\\
793.25	0.00034325\\
800.3	0.00034319\\
807.4	0.0003431\\
814.57	0.00034298\\
821.8	0.00034282\\
829.1	0.00034264\\
836.46	0.00034242\\
843.88	0.00034217\\
851.38	0.00034189\\
858.94	0.00034157\\
866.56	0.00034122\\
874.25	0.00034085\\
882.02	0.00034043\\
889.85	0.00033999\\
897.75	0.00033951\\
905.72	0.00033901\\
913.76	0.00033847\\
921.87	0.0003379\\
930.06	0.00033729\\
938.31	0.00033666\\
946.64	0.00033599\\
955.05	0.0003353\\
963.53	0.00033457\\
972.08	0.00033381\\
980.71	0.00033302\\
989.42	0.0003322\\
998.2	0.00033135\\
1007.1	0.00033047\\
1016	0.00032956\\
1025	0.00032862\\
1034.1	0.00032765\\
1043.3	0.00032665\\
1052.6	0.00032563\\
1061.9	0.00032457\\
1071.3	0.00032349\\
1080.9	0.00032238\\
1090.5	0.00032124\\
1100.1	0.00032007\\
1109.9	0.00031887\\
1119.8	0.00031765\\
1129.7	0.00031641\\
1139.7	0.00031513\\
1149.8	0.00031383\\
1160.1	0.00031251\\
1170.4	0.00031116\\
1180.7	0.00030979\\
1191.2	0.00030839\\
1201.8	0.00030697\\
1212.5	0.00030552\\
1223.2	0.00030406\\
1234.1	0.00030257\\
1245.1	0.00030105\\
1256.1	0.00029952\\
1267.3	0.00029797\\
1278.5	0.00029639\\
1289.9	0.00029479\\
1301.3	0.00029318\\
1312.9	0.00029154\\
1324.5	0.00028989\\
1336.3	0.00028822\\
1348.1	0.00028653\\
1360.1	0.00028482\\
1372.2	0.00028309\\
1384.4	0.00028135\\
1396.7	0.00027959\\
1409.1	0.00027782\\
1421.6	0.00027603\\
1434.2	0.00027422\\
1446.9	0.00027241\\
1459.8	0.00027057\\
1472.7	0.00026873\\
1485.8	0.00026687\\
1499	0.000265\\
1512.3	0.00026311\\
1525.7	0.00026122\\
1539.3	0.00025931\\
1552.9	0.0002574\\
1566.7	0.00025547\\
1580.6	0.00025353\\
1594.7	0.00025159\\
1608.8	0.00024963\\
1623.1	0.00024767\\
1637.5	0.0002457\\
1652.1	0.00024372\\
1666.7	0.00024174\\
1681.5	0.00023974\\
1696.5	0.00023775\\
1711.5	0.00023574\\
1726.7	0.00023373\\
1742	0.00023172\\
1757.5	0.0002297\\
1773.1	0.00022768\\
1788.9	0.00022566\\
1804.7	0.00022363\\
1820.8	0.0002216\\
1836.9	0.00021956\\
1853.2	0.00021753\\
1869.7	0.00021549\\
1886.3	0.00021345\\
1903	0.00021142\\
1919.9	0.00020938\\
1937	0.00020734\\
1954.2	0.0002053\\
1971.5	0.00020327\\
1989	0.00020123\\
2006.7	0.0001992\\
2024.5	0.00019717\\
2042.5	0.00019514\\
2060.6	0.00019312\\
2078.9	0.0001911\\
2097.4	0.00018908\\
2116	0.00018707\\
2134.8	0.00018506\\
2153.7	0.00018306\\
2172.8	0.00018106\\
2192.1	0.00017907\\
2211.6	0.00017708\\
2231.2	0.0001751\\
2251	0.00017313\\
2271	0.00017116\\
2291.2	0.0001692\\
2311.5	0.00016725\\
2332	0.00016531\\
2352.7	0.00016338\\
2373.6	0.00016145\\
2394.7	0.00015953\\
2416	0.00015763\\
2437.4	0.00015573\\
2459.1	0.00015384\\
2480.9	0.00015196\\
2502.9	0.00015009\\
2525.1	0.00014824\\
2547.6	0.00014639\\
2570.2	0.00014456\\
2593	0.00014273\\
2616	0.00014092\\
2639.2	0.00013912\\
2662.7	0.00013733\\
2686.3	0.00013555\\
2710.2	0.00013379\\
2734.2	0.00013204\\
2758.5	0.0001303\\
2783	0.00012857\\
2807.7	0.00012686\\
2832.6	0.00012516\\
2857.8	0.00012347\\
2883.1	0.0001218\\
2908.7	0.00012014\\
2934.6	0.00011849\\
2960.6	0.00011686\\
2986.9	0.00011525\\
3013.4	0.00011364\\
3040.2	0.00011205\\
3067.2	0.00011048\\
3094.4	0.00010892\\
3121.9	0.00010737\\
3149.6	0.00010584\\
3177.6	0.00010432\\
3205.8	0.00010282\\
3234.2	0.00010134\\
3262.9	9.9863e-05\\
3291.9	9.8406e-05\\
3321.1	9.6963e-05\\
3350.6	9.5535e-05\\
3380.4	9.4122e-05\\
3410.4	9.2724e-05\\
3440.7	9.1341e-05\\
3471.2	8.9973e-05\\
3502	8.8619e-05\\
3533.1	8.7281e-05\\
3564.5	8.5957e-05\\
3596.1	8.4649e-05\\
3628.1	8.3355e-05\\
3660.3	8.2076e-05\\
3692.8	8.0812e-05\\
3725.5	7.9563e-05\\
3758.6	7.8328e-05\\
3792	7.7108e-05\\
3825.7	7.5903e-05\\
3859.6	7.4712e-05\\
3893.9	7.3536e-05\\
3928.5	7.2374e-05\\
3963.3	7.1227e-05\\
3998.5	7.0094e-05\\
4034	6.8975e-05\\
4069.8	6.7871e-05\\
4106	6.678e-05\\
4142.4	6.5704e-05\\
4179.2	6.4642e-05\\
4216.3	6.3593e-05\\
4253.7	6.2558e-05\\
4291.5	6.1538e-05\\
4329.6	6.053e-05\\
4368	5.9537e-05\\
4406.8	5.8556e-05\\
4445.9	5.7589e-05\\
4485.4	5.6636e-05\\
4525.2	5.5695e-05\\
4565.4	5.4768e-05\\
4605.9	5.3853e-05\\
4646.8	5.2951e-05\\
4688.1	5.2062e-05\\
4729.7	5.1186e-05\\
4771.7	5.0322e-05\\
4814.1	4.9471e-05\\
4856.8	4.8632e-05\\
4899.9	4.7805e-05\\
4943.4	4.6991e-05\\
4987.3	4.6188e-05\\
5031.6	4.5397e-05\\
5076.3	4.4618e-05\\
5121.3	4.385e-05\\
5166.8	4.3094e-05\\
5212.7	4.235e-05\\
5259	4.1617e-05\\
5305.7	4.0895e-05\\
5352.8	4.0184e-05\\
5400.3	3.9484e-05\\
5448.2	3.8794e-05\\
5496.6	3.8116e-05\\
5545.4	3.7448e-05\\
5594.6	3.679e-05\\
5644.3	3.6143e-05\\
5694.4	3.5506e-05\\
5745	3.4879e-05\\
5796	3.4263e-05\\
5847.4	3.3656e-05\\
5899.3	3.3058e-05\\
5951.7	3.2471e-05\\
6004.6	3.1893e-05\\
6057.9	3.1324e-05\\
6111.7	3.0765e-05\\
6165.9	3.0215e-05\\
6220.7	2.9673e-05\\
6275.9	2.9141e-05\\
6331.6	2.8618e-05\\
6387.8	2.8103e-05\\
6444.5	2.7597e-05\\
6501.7	2.7099e-05\\
6559.5	2.661e-05\\
6617.7	2.6128e-05\\
6676.5	2.5655e-05\\
6735.7	2.519e-05\\
6795.5	2.4733e-05\\
6855.9	2.4284e-05\\
6916.7	2.3842e-05\\
6978.1	2.3408e-05\\
7040.1	2.2981e-05\\
7102.6	2.2562e-05\\
7165.7	2.215e-05\\
7229.3	2.1745e-05\\
7293.5	2.1347e-05\\
7358.2	2.0956e-05\\
7423.5	2.0572e-05\\
7489.4	2.0195e-05\\
7555.9	1.9824e-05\\
7623	1.946e-05\\
7690.7	1.9102e-05\\
7759	1.875e-05\\
7827.9	1.8405e-05\\
7897.4	1.8066e-05\\
7967.5	1.7733e-05\\
8038.2	1.7405e-05\\
8109.6	1.7084e-05\\
8181.6	1.6768e-05\\
8254.2	1.6458e-05\\
8327.5	1.6154e-05\\
8401.4	1.5855e-05\\
8476	1.5561e-05\\
8551.3	1.5273e-05\\
8627.2	1.499e-05\\
8703.8	1.4711e-05\\
8781.1	1.4438e-05\\
8859	1.417e-05\\
8937.7	1.3907e-05\\
9017	1.3649e-05\\
9097.1	1.3395e-05\\
9177.8	1.3146e-05\\
9259.3	1.2901e-05\\
9341.5	1.2661e-05\\
9424.5	1.2426e-05\\
9508.1	1.2194e-05\\
9592.5	1.1967e-05\\
9677.7	1.1744e-05\\
9763.6	1.1525e-05\\
9850.3	1.131e-05\\
9937.8	1.11e-05\\
10026	1.0892e-05\\
10115	1.0689e-05\\
10205	1.049e-05\\
10295	1.0294e-05\\
10387	1.0102e-05\\
10479	9.9131e-06\\
10572	9.7279e-06\\
10666	9.5461e-06\\
10761	9.3677e-06\\
10856	9.1926e-06\\
10953	9.0208e-06\\
11050	8.8521e-06\\
11148	8.6865e-06\\
11247	8.524e-06\\
11347	8.3645e-06\\
11447	8.208e-06\\
11549	8.0543e-06\\
11652	7.9036e-06\\
11755	7.7556e-06\\
11859	7.6103e-06\\
11965	7.4677e-06\\
12071	7.3278e-06\\
12178	7.1905e-06\\
12286	7.0557e-06\\
12395	6.9234e-06\\
12505	6.7936e-06\\
12616	6.6661e-06\\
12728	6.5411e-06\\
12841	6.4183e-06\\
12955	6.2978e-06\\
13070	6.1796e-06\\
13186	6.0635e-06\\
13304	5.9496e-06\\
13422	5.8378e-06\\
13541	5.7281e-06\\
13661	5.6204e-06\\
13782	5.5147e-06\\
13905	5.4109e-06\\
14028	5.3091e-06\\
14153	5.2092e-06\\
14278	5.1111e-06\\
14405	5.0148e-06\\
14533	4.9203e-06\\
14662	4.8275e-06\\
14792	4.7365e-06\\
14923	4.6471e-06\\
15056	4.5594e-06\\
15190	4.4733e-06\\
15325	4.3888e-06\\
15461	4.3058e-06\\
15598	4.2244e-06\\
15736	4.1445e-06\\
15876	4.066e-06\\
16017	3.989e-06\\
16159	3.9134e-06\\
16303	3.8392e-06\\
16447	3.7664e-06\\
16593	3.6949e-06\\
16741	3.6247e-06\\
16889	3.5559e-06\\
17039	3.4882e-06\\
17191	3.4219e-06\\
17343	3.3567e-06\\
17497	3.2928e-06\\
17652	3.23e-06\\
17809	3.1684e-06\\
17967	3.1079e-06\\
18127	3.0485e-06\\
18288	2.9902e-06\\
18450	2.933e-06\\
18614	2.8768e-06\\
18779	2.8217e-06\\
18946	2.7675e-06\\
19114	2.7144e-06\\
19284	2.6622e-06\\
19455	2.611e-06\\
19628	2.5608e-06\\
19802	2.5114e-06\\
19978	2.463e-06\\
20155	2.4155e-06\\
20334	2.3688e-06\\
20515	2.323e-06\\
20697	2.278e-06\\
20881	2.2338e-06\\
21066	2.1905e-06\\
21253	2.148e-06\\
21442	2.1062e-06\\
21632	2.0652e-06\\
21824	2.025e-06\\
22018	1.9855e-06\\
22213	1.9467e-06\\
22411	1.9087e-06\\
22609	1.8713e-06\\
22810	1.8346e-06\\
23013	1.7986e-06\\
23217	1.7633e-06\\
23423	1.7286e-06\\
23631	1.6946e-06\\
23841	1.6612e-06\\
24053	1.6284e-06\\
24266	1.5962e-06\\
24482	1.5646e-06\\
24699	1.5336e-06\\
24918	1.5031e-06\\
25139	1.4733e-06\\
25363	1.4439e-06\\
25588	1.4152e-06\\
25815	1.3869e-06\\
26044	1.3592e-06\\
26275	1.332e-06\\
26509	1.3053e-06\\
26744	1.2791e-06\\
26981	1.2534e-06\\
27221	1.2282e-06\\
27463	1.2034e-06\\
27706	1.1791e-06\\
27952	1.1552e-06\\
28201	1.1318e-06\\
28451	1.1089e-06\\
28704	1.0864e-06\\
28958	1.0643e-06\\
29216	1.0426e-06\\
29475	1.0213e-06\\
29737	1.0004e-06\\
30001	9.7991e-07\\
30267	9.5981e-07\\
30536	9.4009e-07\\
30807	9.2075e-07\\
31080	9.0177e-07\\
31356	8.8315e-07\\
31635	8.6489e-07\\
31915	8.4697e-07\\
32199	8.2939e-07\\
32485	8.1215e-07\\
32773	7.9524e-07\\
33064	7.7866e-07\\
33358	7.6239e-07\\
33654	7.4643e-07\\
33953	7.3078e-07\\
34254	7.1544e-07\\
34558	7.0038e-07\\
34865	6.8562e-07\\
35174	6.7115e-07\\
35487	6.5696e-07\\
35802	6.4304e-07\\
36120	6.2939e-07\\
36440	6.1601e-07\\
36764	6.0289e-07\\
37090	5.9003e-07\\
37420	5.7742e-07\\
37752	5.6505e-07\\
38087	5.5293e-07\\
38425	5.4105e-07\\
38766	5.294e-07\\
39110	5.1798e-07\\
39458	5.0679e-07\\
39808	4.9582e-07\\
40161	4.8507e-07\\
40518	4.7454e-07\\
40878	4.6421e-07\\
41241	4.5409e-07\\
41607	4.4417e-07\\
41976	4.3446e-07\\
42349	4.2493e-07\\
42725	4.156e-07\\
43104	4.0646e-07\\
43487	3.975e-07\\
43873	3.8873e-07\\
44262	3.8013e-07\\
44655	3.7171e-07\\
45052	3.6345e-07\\
45452	3.5537e-07\\
45855	3.4745e-07\\
46262	3.397e-07\\
46673	3.3211e-07\\
47087	3.2467e-07\\
47506	3.1738e-07\\
47927	3.1025e-07\\
48353	3.0326e-07\\
48782	2.9642e-07\\
49215	2.8972e-07\\
49652	2.8316e-07\\
50093	2.7673e-07\\
50538	2.7045e-07\\
50986	2.6429e-07\\
51439	2.5826e-07\\
51896	2.5236e-07\\
52356	2.4658e-07\\
52821	2.4093e-07\\
53290	2.354e-07\\
53763	2.2998e-07\\
54241	2.2468e-07\\
54722	2.1949e-07\\
55208	2.1441e-07\\
55698	2.0944e-07\\
56193	2.0458e-07\\
56692	1.9982e-07\\
57195	1.9516e-07\\
57703	1.9061e-07\\
58215	1.8615e-07\\
58732	1.8179e-07\\
59253	1.7752e-07\\
59779	1.7334e-07\\
60310	1.6926e-07\\
60845	1.6527e-07\\
61386	1.6136e-07\\
61931	1.5754e-07\\
62481	1.538e-07\\
63035	1.5014e-07\\
63595	1.4657e-07\\
64159	1.4307e-07\\
64729	1.3965e-07\\
65304	1.3631e-07\\
65884	1.3304e-07\\
66468	1.2984e-07\\
67059	1.2672e-07\\
67654	1.2366e-07\\
68255	1.2067e-07\\
68861	1.1775e-07\\
69472	1.149e-07\\
70089	1.1211e-07\\
70711	1.0938e-07\\
71339	1.0671e-07\\
71972	1.041e-07\\
72611	1.0156e-07\\
73256	9.9068e-08\\
73906	9.6635e-08\\
74562	9.4258e-08\\
75224	9.1935e-08\\
75892	8.9665e-08\\
76566	8.7447e-08\\
77246	8.528e-08\\
77931	8.3162e-08\\
78623	8.1094e-08\\
79321	7.9073e-08\\
80026	7.71e-08\\
80736	7.5172e-08\\
81453	7.3288e-08\\
82176	7.1449e-08\\
82906	6.9653e-08\\
83642	6.7898e-08\\
84384	6.6185e-08\\
85133	6.4512e-08\\
85889	6.2878e-08\\
86652	6.1283e-08\\
87421	5.9725e-08\\
88197	5.8204e-08\\
88980	5.672e-08\\
89770	5.527e-08\\
90567	5.3855e-08\\
91371	5.2474e-08\\
92183	5.1126e-08\\
93001	4.981e-08\\
93827	4.8526e-08\\
94660	4.7272e-08\\
95500	4.6049e-08\\
96348	4.4855e-08\\
97203	4.369e-08\\
98066	4.2554e-08\\
98937	4.1445e-08\\
99815	4.0363e-08\\
1.007e+05	3.9307e-08\\
1.016e+05	3.8277e-08\\
1.025e+05	3.7273e-08\\
1.0341e+05	3.6293e-08\\
1.0433e+05	3.5337e-08\\
1.0525e+05	3.4405e-08\\
1.0619e+05	3.3495e-08\\
1.0713e+05	3.2608e-08\\
1.0808e+05	3.1744e-08\\
1.0904e+05	3.09e-08\\
1.1001e+05	3.0078e-08\\
1.1098e+05	2.9276e-08\\
1.1197e+05	2.8494e-08\\
1.1296e+05	2.7732e-08\\
1.1397e+05	2.6988e-08\\
1.1498e+05	2.6264e-08\\
1.16e+05	2.5558e-08\\
1.1703e+05	2.4869e-08\\
1.1807e+05	2.4198e-08\\
1.1912e+05	2.3544e-08\\
1.2017e+05	2.2906e-08\\
1.2124e+05	2.2285e-08\\
1.2232e+05	2.168e-08\\
1.234e+05	2.109e-08\\
1.245e+05	2.0515e-08\\
1.256e+05	1.9955e-08\\
1.2672e+05	1.9409e-08\\
1.2784e+05	1.8877e-08\\
1.2898e+05	1.8359e-08\\
1.3012e+05	1.7854e-08\\
1.3128e+05	1.7363e-08\\
1.3245e+05	1.6884e-08\\
1.3362e+05	1.6417e-08\\
1.3481e+05	1.5963e-08\\
1.36e+05	1.5521e-08\\
1.3721e+05	1.509e-08\\
1.3843e+05	1.467e-08\\
1.3966e+05	1.4261e-08\\
1.409e+05	1.3863e-08\\
1.4215e+05	1.3476e-08\\
1.4341e+05	1.3099e-08\\
1.4469e+05	1.2732e-08\\
1.4597e+05	1.2374e-08\\
1.4727e+05	1.2026e-08\\
1.4857e+05	1.1687e-08\\
1.4989e+05	1.1357e-08\\
1.5122e+05	1.1036e-08\\
1.5257e+05	1.0724e-08\\
1.5392e+05	1.042e-08\\
1.5529e+05	1.0124e-08\\
1.5667e+05	9.8361e-09\\
1.5806e+05	9.5559e-09\\
1.5946e+05	9.2833e-09\\
1.6088e+05	9.018e-09\\
1.623e+05	8.7599e-09\\
1.6374e+05	8.5088e-09\\
1.652e+05	8.2645e-09\\
1.6666e+05	8.0268e-09\\
1.6814e+05	7.7957e-09\\
1.6964e+05	7.5708e-09\\
1.7114e+05	7.3521e-09\\
1.7266e+05	7.1394e-09\\
1.742e+05	6.9325e-09\\
1.7574e+05	6.7314e-09\\
1.773e+05	6.5357e-09\\
1.7888e+05	6.3455e-09\\
1.8046e+05	6.1605e-09\\
1.8207e+05	5.9806e-09\\
1.8368e+05	5.8057e-09\\
1.8531e+05	5.6357e-09\\
1.8696e+05	5.4705e-09\\
1.8862e+05	5.3098e-09\\
1.9029e+05	5.1536e-09\\
1.9198e+05	5.0018e-09\\
1.9369e+05	4.8542e-09\\
1.9541e+05	4.7108e-09\\
1.9714e+05	4.5714e-09\\
1.9889e+05	4.436e-09\\
2.0066e+05	4.3043e-09\\
2.0244e+05	4.1764e-09\\
2.0424e+05	4.0521e-09\\
2.0424e+05	3.482e-09\\
2.0244e+05	3.5927e-09\\
2.0066e+05	3.7068e-09\\
1.9889e+05	3.8243e-09\\
1.9714e+05	3.9453e-09\\
1.9541e+05	4.07e-09\\
1.9369e+05	4.1985e-09\\
1.9198e+05	4.3307e-09\\
1.9029e+05	4.4669e-09\\
1.8862e+05	4.6072e-09\\
1.8696e+05	4.7517e-09\\
1.8531e+05	4.9004e-09\\
1.8368e+05	5.0536e-09\\
1.8207e+05	5.2113e-09\\
1.8046e+05	5.3736e-09\\
1.7888e+05	5.5407e-09\\
1.773e+05	5.7128e-09\\
1.7574e+05	5.8899e-09\\
1.742e+05	6.0722e-09\\
1.7266e+05	6.2598e-09\\
1.7114e+05	6.4529e-09\\
1.6964e+05	6.6517e-09\\
1.6814e+05	6.8562e-09\\
1.6666e+05	7.0667e-09\\
1.652e+05	7.2832e-09\\
1.6374e+05	7.506e-09\\
1.623e+05	7.7353e-09\\
1.6088e+05	7.9712e-09\\
1.5946e+05	8.2138e-09\\
1.5806e+05	8.4634e-09\\
1.5667e+05	8.7202e-09\\
1.5529e+05	8.9843e-09\\
1.5392e+05	9.2559e-09\\
1.5257e+05	9.5353e-09\\
1.5122e+05	9.8226e-09\\
1.4989e+05	1.0118e-08\\
1.4857e+05	1.0422e-08\\
1.4727e+05	1.0734e-08\\
1.4597e+05	1.1055e-08\\
1.4469e+05	1.1386e-08\\
1.4341e+05	1.1725e-08\\
1.4215e+05	1.2074e-08\\
1.409e+05	1.2433e-08\\
1.3966e+05	1.2802e-08\\
1.3843e+05	1.3181e-08\\
1.3721e+05	1.357e-08\\
1.36e+05	1.3971e-08\\
1.3481e+05	1.4382e-08\\
1.3362e+05	1.4805e-08\\
1.3245e+05	1.524e-08\\
1.3128e+05	1.5686e-08\\
1.3012e+05	1.6145e-08\\
1.2898e+05	1.6616e-08\\
1.2784e+05	1.71e-08\\
1.2672e+05	1.7598e-08\\
1.256e+05	1.8108e-08\\
1.245e+05	1.8633e-08\\
1.234e+05	1.9172e-08\\
1.2232e+05	1.9726e-08\\
1.2124e+05	2.0294e-08\\
1.2017e+05	2.0878e-08\\
1.1912e+05	2.1477e-08\\
1.1807e+05	2.2093e-08\\
1.1703e+05	2.2725e-08\\
1.16e+05	2.3374e-08\\
1.1498e+05	2.404e-08\\
1.1397e+05	2.4724e-08\\
1.1296e+05	2.5426e-08\\
1.1197e+05	2.6146e-08\\
1.1098e+05	2.6886e-08\\
1.1001e+05	2.7645e-08\\
1.0904e+05	2.8424e-08\\
1.0808e+05	2.9223e-08\\
1.0713e+05	3.0043e-08\\
1.0619e+05	3.0885e-08\\
1.0525e+05	3.1748e-08\\
1.0433e+05	3.2634e-08\\
1.0341e+05	3.3543e-08\\
1.025e+05	3.4476e-08\\
1.016e+05	3.5432e-08\\
1.007e+05	3.6414e-08\\
99815	3.742e-08\\
98937	3.8452e-08\\
98066	3.9511e-08\\
97203	4.0596e-08\\
96348	4.1709e-08\\
95500	4.2851e-08\\
94660	4.4021e-08\\
93827	4.5221e-08\\
93001	4.6451e-08\\
92183	4.7712e-08\\
91371	4.9004e-08\\
90567	5.0329e-08\\
89770	5.1688e-08\\
88980	5.3079e-08\\
88197	5.4506e-08\\
87421	5.5968e-08\\
86652	5.7466e-08\\
85889	5.9001e-08\\
85133	6.0573e-08\\
84384	6.2184e-08\\
83642	6.3835e-08\\
82906	6.5526e-08\\
82176	6.7258e-08\\
81453	6.9033e-08\\
80736	7.085e-08\\
80026	7.2711e-08\\
79321	7.4617e-08\\
78623	7.6569e-08\\
77931	7.8568e-08\\
77246	8.0615e-08\\
76566	8.271e-08\\
75892	8.4856e-08\\
75224	8.7052e-08\\
74562	8.93e-08\\
73906	9.1602e-08\\
73256	9.3958e-08\\
72611	9.6369e-08\\
71972	9.8837e-08\\
71339	1.0136e-07\\
70711	1.0395e-07\\
70089	1.0659e-07\\
69472	1.093e-07\\
68861	1.1207e-07\\
68255	1.149e-07\\
67654	1.178e-07\\
67059	1.2076e-07\\
66468	1.2379e-07\\
65884	1.269e-07\\
65304	1.3007e-07\\
64729	1.3331e-07\\
64159	1.3663e-07\\
63595	1.4002e-07\\
63035	1.4349e-07\\
62481	1.4704e-07\\
61931	1.5067e-07\\
61386	1.5438e-07\\
60845	1.5817e-07\\
60310	1.6205e-07\\
59779	1.6601e-07\\
59253	1.7006e-07\\
58732	1.742e-07\\
58215	1.7843e-07\\
57703	1.8275e-07\\
57195	1.8717e-07\\
56692	1.9169e-07\\
56193	1.963e-07\\
55698	2.0102e-07\\
55208	2.0584e-07\\
54722	2.1076e-07\\
54241	2.1579e-07\\
53763	2.2093e-07\\
53290	2.2617e-07\\
52821	2.3153e-07\\
52356	2.3701e-07\\
51896	2.426e-07\\
51439	2.4831e-07\\
50986	2.5415e-07\\
50538	2.601e-07\\
50093	2.6619e-07\\
49652	2.724e-07\\
49215	2.7874e-07\\
48782	2.8522e-07\\
48353	2.9183e-07\\
47927	2.9858e-07\\
47506	3.0547e-07\\
47087	3.1251e-07\\
46673	3.1969e-07\\
46262	3.2702e-07\\
45855	3.3451e-07\\
45452	3.4215e-07\\
45052	3.4994e-07\\
44655	3.579e-07\\
44262	3.6602e-07\\
43873	3.7431e-07\\
43487	3.8277e-07\\
43104	3.914e-07\\
42725	4.0021e-07\\
42349	4.0919e-07\\
41976	4.1836e-07\\
41607	4.2772e-07\\
41241	4.3727e-07\\
40878	4.4701e-07\\
40518	4.5694e-07\\
40161	4.6708e-07\\
39808	4.7742e-07\\
39458	4.8797e-07\\
39110	4.9873e-07\\
38766	5.0971e-07\\
38425	5.209e-07\\
38087	5.3232e-07\\
37752	5.4397e-07\\
37420	5.5585e-07\\
37090	5.6797e-07\\
36764	5.8032e-07\\
36440	5.9293e-07\\
36120	6.0578e-07\\
35802	6.1888e-07\\
35487	6.3225e-07\\
35174	6.4588e-07\\
34865	6.5978e-07\\
34558	6.7395e-07\\
34254	6.884e-07\\
33953	7.0313e-07\\
33654	7.1815e-07\\
33358	7.3347e-07\\
33064	7.4908e-07\\
32773	7.65e-07\\
32485	7.8123e-07\\
32199	7.9778e-07\\
31915	8.1465e-07\\
31635	8.3184e-07\\
31356	8.4938e-07\\
31080	8.6725e-07\\
30807	8.8546e-07\\
30536	9.0403e-07\\
30267	9.2296e-07\\
30001	9.4226e-07\\
29737	9.6193e-07\\
29475	9.8197e-07\\
29216	1.0024e-06\\
28958	1.0232e-06\\
28704	1.0445e-06\\
28451	1.0661e-06\\
28201	1.0881e-06\\
27952	1.1106e-06\\
27706	1.1335e-06\\
27463	1.1569e-06\\
27221	1.1807e-06\\
26981	1.2049e-06\\
26744	1.2296e-06\\
26509	1.2548e-06\\
26275	1.2805e-06\\
26044	1.3066e-06\\
25815	1.3333e-06\\
25588	1.3604e-06\\
25363	1.3881e-06\\
25139	1.4163e-06\\
24918	1.4451e-06\\
24699	1.4744e-06\\
24482	1.5042e-06\\
24266	1.5346e-06\\
24053	1.5656e-06\\
23841	1.5972e-06\\
23631	1.6294e-06\\
23423	1.6622e-06\\
23217	1.6956e-06\\
23013	1.7296e-06\\
22810	1.7643e-06\\
22609	1.7997e-06\\
22411	1.8357e-06\\
22213	1.8724e-06\\
22018	1.9098e-06\\
21824	1.9479e-06\\
21632	1.9867e-06\\
21442	2.0263e-06\\
21253	2.0666e-06\\
21066	2.1077e-06\\
20881	2.1495e-06\\
20697	2.1922e-06\\
20515	2.2356e-06\\
20334	2.2799e-06\\
20155	2.325e-06\\
19978	2.371e-06\\
19802	2.4178e-06\\
19628	2.4655e-06\\
19455	2.5142e-06\\
19284	2.5637e-06\\
19114	2.6142e-06\\
18946	2.6656e-06\\
18779	2.718e-06\\
18614	2.7715e-06\\
18450	2.8259e-06\\
18288	2.8813e-06\\
18127	2.9378e-06\\
17967	2.9954e-06\\
17809	3.054e-06\\
17652	3.1138e-06\\
17497	3.1747e-06\\
17343	3.2367e-06\\
17191	3.3e-06\\
17039	3.3644e-06\\
16889	3.43e-06\\
16741	3.4969e-06\\
16593	3.5651e-06\\
16447	3.6345e-06\\
16303	3.7053e-06\\
16159	3.7774e-06\\
16017	3.8508e-06\\
15876	3.9257e-06\\
15736	4.002e-06\\
15598	4.0797e-06\\
15461	4.1589e-06\\
15325	4.2396e-06\\
15190	4.3218e-06\\
15056	4.4056e-06\\
14923	4.491e-06\\
14792	4.578e-06\\
14662	4.6666e-06\\
14533	4.757e-06\\
14405	4.849e-06\\
14278	4.9428e-06\\
14153	5.0384e-06\\
14028	5.1358e-06\\
13905	5.235e-06\\
13782	5.3361e-06\\
13661	5.4392e-06\\
13541	5.5442e-06\\
13422	5.6511e-06\\
13304	5.7602e-06\\
13186	5.8712e-06\\
13070	5.9844e-06\\
12955	6.0998e-06\\
12841	6.2173e-06\\
12728	6.337e-06\\
12616	6.459e-06\\
12505	6.5834e-06\\
12395	6.7101e-06\\
12286	6.8391e-06\\
12178	6.9707e-06\\
12071	7.1047e-06\\
11965	7.2413e-06\\
11859	7.3804e-06\\
11755	7.5222e-06\\
11652	7.6666e-06\\
11549	7.8138e-06\\
11447	7.9638e-06\\
11347	8.1166e-06\\
11247	8.2723e-06\\
11148	8.4309e-06\\
11050	8.5925e-06\\
10953	8.7571e-06\\
10856	8.9249e-06\\
10761	9.0958e-06\\
10666	9.27e-06\\
10572	9.4474e-06\\
10479	9.6281e-06\\
10387	9.8122e-06\\
10295	9.9998e-06\\
10205	1.0191e-05\\
10115	1.0386e-05\\
10026	1.0584e-05\\
9937.8	1.0786e-05\\
9850.3	1.0992e-05\\
9763.6	1.1202e-05\\
9677.7	1.1415e-05\\
9592.5	1.1633e-05\\
9508.1	1.1854e-05\\
9424.5	1.208e-05\\
9341.5	1.231e-05\\
9259.3	1.2544e-05\\
9177.8	1.2783e-05\\
9097.1	1.3026e-05\\
9017	1.3273e-05\\
8937.7	1.3525e-05\\
8859	1.3782e-05\\
8781.1	1.4043e-05\\
8703.8	1.431e-05\\
8627.2	1.4581e-05\\
8551.3	1.4857e-05\\
8476	1.5138e-05\\
8401.4	1.5424e-05\\
8327.5	1.5716e-05\\
8254.2	1.6013e-05\\
8181.6	1.6315e-05\\
8109.6	1.6623e-05\\
8038.2	1.6936e-05\\
7967.5	1.7255e-05\\
7897.4	1.758e-05\\
7827.9	1.7911e-05\\
7759	1.8248e-05\\
7690.7	1.859e-05\\
7623	1.8939e-05\\
7555.9	1.9294e-05\\
7489.4	1.9656e-05\\
7423.5	2.0024e-05\\
7358.2	2.0398e-05\\
7293.5	2.078e-05\\
7229.3	2.1168e-05\\
7165.7	2.1563e-05\\
7102.6	2.1964e-05\\
7040.1	2.2373e-05\\
6978.1	2.2789e-05\\
6916.7	2.3213e-05\\
6855.9	2.3644e-05\\
6795.5	2.4082e-05\\
6735.7	2.4528e-05\\
6676.5	2.4982e-05\\
6617.7	2.5443e-05\\
6559.5	2.5913e-05\\
6501.7	2.639e-05\\
6444.5	2.6876e-05\\
6387.8	2.737e-05\\
6331.6	2.7873e-05\\
6275.9	2.8384e-05\\
6220.7	2.8904e-05\\
6165.9	2.9433e-05\\
6111.7	2.997e-05\\
6057.9	3.0517e-05\\
6004.6	3.1072e-05\\
5951.7	3.1637e-05\\
5899.3	3.2212e-05\\
5847.4	3.2796e-05\\
5796	3.3389e-05\\
5745	3.3992e-05\\
5694.4	3.4605e-05\\
5644.3	3.5229e-05\\
5594.6	3.5862e-05\\
5545.4	3.6505e-05\\
5496.6	3.7159e-05\\
5448.2	3.7823e-05\\
5400.3	3.8498e-05\\
5352.8	3.9184e-05\\
5305.7	3.9881e-05\\
5259	4.0588e-05\\
5212.7	4.1307e-05\\
5166.8	4.2037e-05\\
5121.3	4.2778e-05\\
5076.3	4.353e-05\\
5031.6	4.4295e-05\\
4987.3	4.507e-05\\
4943.4	4.5858e-05\\
4899.9	4.6658e-05\\
4856.8	4.7469e-05\\
4814.1	4.8293e-05\\
4771.7	4.9129e-05\\
4729.7	4.9977e-05\\
4688.1	5.0838e-05\\
4646.8	5.1712e-05\\
4605.9	5.2598e-05\\
4565.4	5.3497e-05\\
4525.2	5.4409e-05\\
4485.4	5.5334e-05\\
4445.9	5.6272e-05\\
4406.8	5.7223e-05\\
4368	5.8187e-05\\
4329.6	5.9165e-05\\
4291.5	6.0156e-05\\
4253.7	6.1161e-05\\
4216.3	6.2179e-05\\
4179.2	6.3212e-05\\
4142.4	6.4258e-05\\
4106	6.5317e-05\\
4069.8	6.6391e-05\\
4034	6.7479e-05\\
3998.5	6.8581e-05\\
3963.3	6.9697e-05\\
3928.5	7.0827e-05\\
3893.9	7.1971e-05\\
3859.6	7.313e-05\\
3825.7	7.4303e-05\\
3792	7.5491e-05\\
3758.6	7.6692e-05\\
3725.5	7.7909e-05\\
3692.8	7.914e-05\\
3660.3	8.0385e-05\\
3628.1	8.1645e-05\\
3596.1	8.2919e-05\\
3564.5	8.4208e-05\\
3533.1	8.5512e-05\\
3502	8.683e-05\\
3471.2	8.8163e-05\\
3440.7	8.951e-05\\
3410.4	9.0872e-05\\
3380.4	9.2248e-05\\
3350.6	9.3639e-05\\
3321.1	9.5045e-05\\
3291.9	9.6465e-05\\
3262.9	9.7899e-05\\
3234.2	9.9348e-05\\
3205.8	0.00010081\\
3177.6	0.00010229\\
3149.6	0.00010378\\
3121.9	0.00010529\\
3094.4	0.00010681\\
3067.2	0.00010834\\
3040.2	0.00010989\\
3013.4	0.00011145\\
2986.9	0.00011303\\
2960.6	0.00011462\\
2934.6	0.00011622\\
2908.7	0.00011783\\
2883.1	0.00011946\\
2857.8	0.00012111\\
2832.6	0.00012276\\
2807.7	0.00012443\\
2783	0.00012612\\
2758.5	0.00012781\\
2734.2	0.00012952\\
2710.2	0.00013124\\
2686.3	0.00013297\\
2662.7	0.00013471\\
2639.2	0.00013647\\
2616	0.00013824\\
2593	0.00014002\\
2570.2	0.00014181\\
2547.6	0.00014361\\
2525.1	0.00014542\\
2502.9	0.00014725\\
2480.9	0.00014908\\
2459.1	0.00015093\\
2437.4	0.00015278\\
2416	0.00015465\\
2394.7	0.00015652\\
2373.6	0.0001584\\
2352.7	0.0001603\\
2332	0.0001622\\
2311.5	0.00016411\\
2291.2	0.00016603\\
2271	0.00016795\\
2251	0.00016989\\
2231.2	0.00017183\\
2211.6	0.00017377\\
2192.1	0.00017573\\
2172.8	0.00017769\\
2153.7	0.00017966\\
2134.8	0.00018163\\
2116	0.00018361\\
2097.4	0.00018559\\
2078.9	0.00018758\\
2060.6	0.00018957\\
2042.5	0.00019157\\
2024.5	0.00019357\\
2006.7	0.00019557\\
1989	0.00019757\\
1971.5	0.00019958\\
1954.2	0.00020159\\
1937	0.0002036\\
1919.9	0.00020561\\
1903	0.00020762\\
1886.3	0.00020964\\
1869.7	0.00021165\\
1853.2	0.00021366\\
1836.9	0.00021567\\
1820.8	0.00021768\\
1804.7	0.00021969\\
1788.9	0.00022169\\
1773.1	0.00022369\\
1757.5	0.00022569\\
1742	0.00022768\\
1726.7	0.00022967\\
1711.5	0.00023165\\
1696.5	0.00023363\\
1681.5	0.0002356\\
1666.7	0.00023757\\
1652.1	0.00023953\\
1637.5	0.00024148\\
1623.1	0.00024342\\
1608.8	0.00024536\\
1594.7	0.00024728\\
1580.6	0.0002492\\
1566.7	0.00025111\\
1552.9	0.000253\\
1539.3	0.00025489\\
1525.7	0.00025676\\
1512.3	0.00025862\\
1499	0.00026047\\
1485.8	0.00026231\\
1472.7	0.00026413\\
1459.8	0.00026594\\
1446.9	0.00026773\\
1434.2	0.00026951\\
1421.6	0.00027128\\
1409.1	0.00027302\\
1396.7	0.00027476\\
1384.4	0.00027647\\
1372.2	0.00027817\\
1360.1	0.00027985\\
1348.1	0.00028151\\
1336.3	0.00028315\\
1324.5	0.00028478\\
1312.9	0.00028638\\
1301.3	0.00028797\\
1289.9	0.00028953\\
1278.5	0.00029108\\
1267.3	0.0002926\\
1256.1	0.0002941\\
1245.1	0.00029558\\
1234.1	0.00029704\\
1223.2	0.00029848\\
1212.5	0.00029989\\
1201.8	0.00030128\\
1191.2	0.00030264\\
1180.7	0.00030399\\
1170.4	0.0003053\\
1160.1	0.00030659\\
1149.8	0.00030786\\
1139.7	0.0003091\\
1129.7	0.00031032\\
1119.8	0.00031151\\
1109.9	0.00031267\\
1100.1	0.00031381\\
1090.5	0.00031492\\
1080.9	0.000316\\
1071.3	0.00031706\\
1061.9	0.00031808\\
1052.6	0.00031908\\
1043.3	0.00032005\\
1034.1	0.00032099\\
1025	0.00032191\\
1016	0.00032279\\
1007.1	0.00032365\\
998.2	0.00032447\\
989.42	0.00032527\\
980.71	0.00032604\\
972.08	0.00032677\\
963.53	0.00032748\\
955.05	0.00032815\\
946.64	0.0003288\\
938.31	0.00032941\\
930.06	0.00033\\
921.87	0.00033055\\
913.76	0.00033107\\
905.72	0.00033156\\
897.75	0.00033202\\
889.85	0.00033245\\
882.02	0.00033284\\
874.25	0.00033321\\
866.56	0.00033354\\
858.94	0.00033384\\
851.38	0.00033411\\
843.88	0.00033434\\
836.46	0.00033455\\
829.1	0.00033472\\
821.8	0.00033486\\
814.57	0.00033497\\
807.4	0.00033505\\
800.3	0.00033509\\
793.25	0.00033511\\
786.27	0.00033509\\
779.35	0.00033504\\
772.5	0.00033495\\
765.7	0.00033484\\
758.96	0.00033469\\
752.28	0.00033452\\
745.66	0.00033431\\
739.1	0.00033407\\
732.59	0.0003338\\
726.15	0.00033349\\
719.76	0.00033316\\
713.42	0.0003328\\
707.14	0.0003324\\
700.92	0.00033198\\
694.75	0.00033152\\
688.64	0.00033103\\
682.58	0.00033052\\
676.57	0.00032997\\
670.62	0.0003294\\
664.72	0.00032879\\
658.87	0.00032816\\
653.07	0.0003275\\
647.32	0.00032681\\
641.63	0.00032609\\
635.98	0.00032535\\
630.38	0.00032457\\
624.84	0.00032377\\
619.34	0.00032294\\
613.89	0.00032209\\
608.49	0.00032121\\
603.13	0.0003203\\
597.82	0.00031936\\
592.56	0.0003184\\
587.35	0.00031742\\
582.18	0.00031641\\
577.06	0.00031538\\
571.98	0.00031432\\
566.94	0.00031324\\
561.95	0.00031213\\
557.01	0.000311\\
552.11	0.00030985\\
547.25	0.00030868\\
542.43	0.00030749\\
537.66	0.00030627\\
532.93	0.00030503\\
528.24	0.00030377\\
523.59	0.00030249\\
518.98	0.00030119\\
514.42	0.00029987\\
509.89	0.00029853\\
505.4	0.00029718\\
500.95	0.0002958\\
496.55	0.00029441\\
492.18	0.000293\\
487.84	0.00029157\\
483.55	0.00029012\\
479.3	0.00028866\\
475.08	0.00028718\\
470.9	0.00028569\\
466.75	0.00028418\\
462.65	0.00028265\\
458.58	0.00028112\\
454.54	0.00027956\\
450.54	0.000278\\
446.58	0.00027642\\
442.65	0.00027482\\
438.75	0.00027322\\
434.89	0.0002716\\
431.06	0.00026997\\
427.27	0.00026833\\
423.51	0.00026668\\
419.78	0.00026502\\
416.09	0.00026335\\
412.43	0.00026167\\
408.8	0.00025998\\
405.2	0.00025828\\
401.63	0.00025657\\
398.1	0.00025485\\
394.6	0.00025313\\
391.12	0.0002514\\
387.68	0.00024966\\
384.27	0.00024791\\
380.89	0.00024616\\
377.54	0.0002444\\
374.21	0.00024264\\
370.92	0.00024087\\
367.66	0.0002391\\
364.42	0.00023732\\
361.21	0.00023554\\
358.04	0.00023375\\
354.88	0.00023196\\
351.76	0.00023017\\
348.67	0.00022837\\
345.6	0.00022657\\
342.56	0.00022477\\
339.54	0.00022297\\
336.55	0.00022117\\
333.59	0.00021936\\
330.66	0.00021755\\
327.75	0.00021575\\
324.86	0.00021394\\
322	0.00021213\\
319.17	0.00021032\\
316.36	0.00020852\\
313.58	0.00020671\\
310.82	0.00020491\\
308.08	0.0002031\\
305.37	0.0002013\\
302.68	0.0001995\\
300.02	0.0001977\\
297.38	0.00019591\\
294.76	0.00019412\\
292.17	0.00019233\\
289.6	0.00019054\\
287.05	0.00018876\\
284.52	0.00018698\\
282.02	0.00018521\\
279.54	0.00018344\\
277.08	0.00018167\\
274.64	0.00017991\\
272.22	0.00017815\\
269.83	0.0001764\\
267.45	0.00017466\\
265.1	0.00017292\\
262.77	0.00017118\\
260.45	0.00016946\\
258.16	0.00016773\\
255.89	0.00016602\\
253.64	0.00016431\\
251.41	0.00016261\\
249.19	0.00016092\\
247	0.00015923\\
244.83	0.00015755\\
242.67	0.00015588\\
240.54	0.00015421\\
238.42	0.00015256\\
236.32	0.00015091\\
234.24	0.00014927\\
232.18	0.00014764\\
230.14	0.00014601\\
228.11	0.0001444\\
226.11	0.00014279\\
224.12	0.0001412\\
222.14	0.00013961\\
220.19	0.00013803\\
218.25	0.00013647\\
216.33	0.00013491\\
214.43	0.00013336\\
212.54	0.00013182\\
210.67	0.00013029\\
208.82	0.00012877\\
206.98	0.00012726\\
205.16	0.00012576\\
203.35	0.00012428\\
201.56	0.0001228\\
199.79	0.00012133\\
198.03	0.00011987\\
196.29	0.00011843\\
194.56	0.00011699\\
192.85	0.00011557\\
191.15	0.00011416\\
189.47	0.00011275\\
187.8	0.00011136\\
186.15	0.00010998\\
184.51	0.00010861\\
182.89	0.00010726\\
181.28	0.00010591\\
179.68	0.00010458\\
178.1	0.00010325\\
176.53	0.00010194\\
174.98	0.00010064\\
173.44	9.935e-05\\
171.91	9.8072e-05\\
170.4	9.6806e-05\\
168.9	9.5551e-05\\
167.42	9.4308e-05\\
165.94	9.3076e-05\\
164.48	9.1856e-05\\
163.03	9.0648e-05\\
161.6	8.9451e-05\\
160.18	8.8266e-05\\
158.77	8.7092e-05\\
157.37	8.593e-05\\
155.99	8.478e-05\\
154.61	8.3641e-05\\
153.25	8.2514e-05\\
151.9	8.1398e-05\\
150.57	8.0294e-05\\
149.24	7.9201e-05\\
147.93	7.812e-05\\
146.63	7.705e-05\\
145.34	7.5992e-05\\
144.06	7.4945e-05\\
142.79	7.3909e-05\\
141.53	7.2885e-05\\
140.29	7.1872e-05\\
139.05	7.087e-05\\
137.83	6.988e-05\\
136.62	6.89e-05\\
135.41	6.7932e-05\\
134.22	6.6975e-05\\
133.04	6.6029e-05\\
131.87	6.5093e-05\\
130.71	6.4169e-05\\
129.56	6.3255e-05\\
128.42	6.2353e-05\\
127.29	6.1461e-05\\
126.17	6.0579e-05\\
125.06	5.9709e-05\\
123.96	5.8848e-05\\
122.87	5.7998e-05\\
121.79	5.7159e-05\\
120.72	5.633e-05\\
119.65	5.5511e-05\\
118.6	5.4703e-05\\
117.56	5.3904e-05\\
116.52	5.3116e-05\\
115.5	5.2337e-05\\
114.48	5.1569e-05\\
113.47	5.081e-05\\
112.47	5.0061e-05\\
111.48	4.9322e-05\\
110.5	4.8592e-05\\
109.53	4.7872e-05\\
108.57	4.7161e-05\\
107.61	4.646e-05\\
106.66	4.5768e-05\\
105.73	4.5085e-05\\
104.8	4.4411e-05\\
103.87	4.3746e-05\\
102.96	4.309e-05\\
102.05	4.2443e-05\\
101.16	4.1805e-05\\
100.26	4.1176e-05\\
99.383	4.0555e-05\\
98.508	3.9942e-05\\
97.641	3.9339e-05\\
96.782	3.8743e-05\\
95.93	3.8156e-05\\
95.086	3.7577e-05\\
94.249	3.7005e-05\\
93.42	3.6442e-05\\
92.598	3.5887e-05\\
91.783	3.534e-05\\
90.975	3.4801e-05\\
90.175	3.4269e-05\\
89.381	3.3745e-05\\
88.595	3.3228e-05\\
87.815	3.2718e-05\\
87.042	3.2216e-05\\
86.276	3.1722e-05\\
85.517	3.1234e-05\\
84.764	3.0754e-05\\
84.018	3.028e-05\\
83.279	2.9814e-05\\
82.546	2.9354e-05\\
81.82	2.8901e-05\\
81.1	2.8454e-05\\
80.386	2.8015e-05\\
79.679	2.7581e-05\\
78.977	2.7154e-05\\
78.282	2.6734e-05\\
77.594	2.632e-05\\
76.911	2.5912e-05\\
76.234	2.551e-05\\
75.563	2.5114e-05\\
74.898	2.4724e-05\\
74.239	2.4339e-05\\
73.586	2.3961e-05\\
72.938	2.3588e-05\\
72.296	2.3221e-05\\
71.66	2.286e-05\\
71.029	2.2504e-05\\
70.404	2.2153e-05\\
69.785	2.1808e-05\\
69.171	2.1468e-05\\
68.562	2.1133e-05\\
67.959	2.0804e-05\\
67.361	2.0479e-05\\
66.768	2.0159e-05\\
66.18	1.9845e-05\\
65.598	1.9535e-05\\
65.021	1.923e-05\\
64.448	1.8929e-05\\
63.881	1.8633e-05\\
63.319	1.8342e-05\\
62.762	1.8055e-05\\
62.21	1.7773e-05\\
61.662	1.7495e-05\\
61.12	1.7222e-05\\
60.582	1.6952e-05\\
60.049	1.6687e-05\\
59.52	1.6426e-05\\
58.996	1.6169e-05\\
58.477	1.5916e-05\\
57.963	1.5667e-05\\
57.453	1.5422e-05\\
56.947	1.518e-05\\
56.446	1.4943e-05\\
55.949	1.4709e-05\\
55.457	1.4478e-05\\
54.969	1.4252e-05\\
54.485	1.4029e-05\\
54.005	1.3809e-05\\
53.53	1.3593e-05\\
53.059	1.338e-05\\
52.592	1.317e-05\\
52.129	1.2964e-05\\
51.671	1.2761e-05\\
51.216	1.2561e-05\\
50.765	1.2365e-05\\
50.319	1.2171e-05\\
49.876	1.198e-05\\
49.437	1.1793e-05\\
49.002	1.1608e-05\\
48.571	1.1426e-05\\
48.143	1.1247e-05\\
47.719	1.1071e-05\\
47.3	1.0898e-05\\
46.883	1.0727e-05\\
46.471	1.0559e-05\\
46.062	1.0394e-05\\
45.656	1.0231e-05\\
45.255	1.0071e-05\\
44.856	9.9132e-06\\
44.462	9.7579e-06\\
44.07	9.6051e-06\\
43.683	9.4547e-06\\
43.298	9.3066e-06\\
42.917	9.1608e-06\\
42.539	9.0173e-06\\
42.165	8.876e-06\\
41.794	8.7369e-06\\
41.426	8.6e-06\\
41.062	8.4652e-06\\
40.7	8.3325e-06\\
40.342	8.2019e-06\\
39.987	8.0734e-06\\
39.635	7.9468e-06\\
39.287	7.8222e-06\\
38.941	7.6995e-06\\
38.598	7.5787e-06\\
38.258	7.4598e-06\\
37.922	7.3428e-06\\
37.588	7.2275e-06\\
37.257	7.1141e-06\\
36.929	7.0024e-06\\
36.604	6.8924e-06\\
36.282	6.7841e-06\\
35.963	6.6775e-06\\
35.647	6.5726e-06\\
35.333	6.4693e-06\\
35.022	6.3676e-06\\
34.714	6.2674e-06\\
34.408	6.1688e-06\\
34.105	6.0717e-06\\
33.805	5.9761e-06\\
33.508	5.882e-06\\
33.213	5.7893e-06\\
32.921	5.698e-06\\
32.631	5.6082e-06\\
32.344	5.5197e-06\\
32.059	5.4326e-06\\
31.777	5.3468e-06\\
31.497	5.2624e-06\\
31.22	5.1792e-06\\
30.946	5.0973e-06\\
30.673	5.0167e-06\\
30.403	4.9373e-06\\
30.136	4.8591e-06\\
29.871	4.7821e-06\\
}--cycle;
\addplot[ybar interval, fill=black, fill opacity=0.35, area legend, draw=none] table[row sep=crcr, x=Lower, y=Count] {%
Lower	Upper	Count\\
29.871	42.524	0\\
42.524	60.539	0\\
60.539	86.185	0\\
86.185	122.69	7.3039e-05\\
122.69	174.67	5.1305e-05\\
174.67	248.67	0.00025227\\
248.67	354.01	0.00027846\\
354.01	503.97	0.00017782\\
503.97	717.47	0.00039969\\
717.47	1021.4	0.00032463\\
1021.4	1454.1	0.00033896\\
1454.1	2070.1	0.00024676\\
2070.1	2947	0.0001338\\
2947	4195.5	9.3984e-05\\
4195.5	5972.8	3.751e-05\\
5972.8	8503	1.8971e-05\\
8503	12105	9.624e-06\\
12105	17233	4.1601e-06\\
17233	24534	1.0958e-06\\
24534	34927	7.6975e-07\\
34927	49722	5.4069e-07\\
49722	70786	1.266e-07\\
70786	1.0077e+05	0\\
1.0077e+05	1.4346e+05	6.2466e-08\\
1.4346e+05	2.0424e+05	4.3878e-08\\
2.0424e+05	2.0424e+05	4.3878e-08\\
};
\addlegendentry{Istogramma reale}

\addplot [color=black]
  table[row sep=crcr]{%
29.871	1.2966e-06\\
34.105	2.0618e-06\\
38.258	3.0331e-06\\
42.165	4.1559e-06\\
46.062	5.4827e-06\\
49.876	6.9826e-06\\
53.53	8.6051e-06\\
57.453	1.0545e-05\\
61.12	1.2539e-05\\
65.021	1.4847e-05\\
68.562	1.7101e-05\\
72.296	1.9634e-05\\
76.234	2.2471e-05\\
80.386	2.5636e-05\\
84.764	2.9153e-05\\
88.595	3.2372e-05\\
92.598	3.5867e-05\\
96.782	3.965e-05\\
101.16	4.3737e-05\\
105.73	4.8137e-05\\
110.5	5.2864e-05\\
115.5	5.7926e-05\\
120.72	6.3332e-05\\
126.17	6.909e-05\\
131.87	7.5205e-05\\
137.83	8.168e-05\\
144.06	8.8516e-05\\
150.57	9.5713e-05\\
157.37	0.00010327\\
164.48	0.00011117\\
171.91	0.00011941\\
179.68	0.00012798\\
187.8	0.00013686\\
196.29	0.00014603\\
205.16	0.00015548\\
216.33	0.00016713\\
228.11	0.00017909\\
242.67	0.00019334\\
262.77	0.00021199\\
310.82	0.00025149\\
330.66	0.00026568\\
348.67	0.00027752\\
364.42	0.00028709\\
380.89	0.00029633\\
394.6	0.00030345\\
408.8	0.00031031\\
423.51	0.00031687\\
438.75	0.00032312\\
454.54	0.00032902\\
470.9	0.00033455\\
487.84	0.0003397\\
500.95	0.0003433\\
514.42	0.00034665\\
528.24	0.00034977\\
542.43	0.00035262\\
557.01	0.00035522\\
571.98	0.00035756\\
587.35	0.00035962\\
603.13	0.00036141\\
619.34	0.00036292\\
635.98	0.00036415\\
653.07	0.00036509\\
670.62	0.00036574\\
688.64	0.00036611\\
707.14	0.00036618\\
726.15	0.00036596\\
745.66	0.00036546\\
765.7	0.00036466\\
786.27	0.00036358\\
807.4	0.00036221\\
829.1	0.00036057\\
851.38	0.00035865\\
874.25	0.00035645\\
897.75	0.00035399\\
921.87	0.00035127\\
946.64	0.00034829\\
972.08	0.00034507\\
998.2	0.0003416\\
1034.1	0.00033663\\
1071.3	0.00033125\\
1109.9	0.00032551\\
1149.8	0.00031942\\
1191.2	0.00031301\\
1234.1	0.0003063\\
1278.5	0.00029931\\
1336.3	0.00029023\\
1396.7	0.00028081\\
1459.8	0.00027111\\
1539.3	0.00025917\\
1637.5	0.00024493\\
1820.8	0.00022015\\
2024.5	0.00019544\\
2153.7	0.00018129\\
2271	0.00016943\\
2394.7	0.00015786\\
2525.1	0.00014664\\
2639.2	0.00013758\\
2758.5	0.00012882\\
2883.1	0.00012036\\
3013.4	0.00011223\\
3149.6	0.00010444\\
3291.9	9.6998e-05\\
3440.7	8.9903e-05\\
3596.1	8.3164e-05\\
3758.6	7.678e-05\\
3928.5	7.0751e-05\\
4106	6.5071e-05\\
4291.5	5.9736e-05\\
4485.4	5.4738e-05\\
4688.1	5.0069e-05\\
4899.9	4.5718e-05\\
5121.3	4.1673e-05\\
5352.8	3.7923e-05\\
5594.6	3.4456e-05\\
5847.4	3.1256e-05\\
6165.9	2.7753e-05\\
6501.7	2.4592e-05\\
6855.9	2.1749e-05\\
7229.3	1.9202e-05\\
7690.7	1.6571e-05\\
8181.6	1.4273e-05\\
8703.8	1.2274e-05\\
9341.5	1.0313e-05\\
10026	8.6558e-06\\
10856	7.102e-06\\
11859	5.7006e-06\\
13070	4.4823e-06\\
14533	3.4596e-06\\
16447	2.5743e-06\\
19114	1.8202e-06\\
23013	1.2112e-06\\
29475	7.2692e-07\\
41976	3.63e-07\\
73906	1.1579e-07\\
2.0424e+05	1.0036e-08\\
};
\addlegendentry{Adatt. BLN reale ($\mathit{ML}$)}

    
        %     \nextgroupplot[%
        %         % xmin=78.6017269745556,xmax=1000000,
        %         ymin=0,ymax=0.0004,
        %     ] % This file was created by matlab2tikz.
%
\definecolor{mycolor1}{rgb}{0.83529,0.36863,0.00000}%
%
\addplot[ybar interval, fill=mycolor1, fill opacity=0.35, area legend, draw=none] table[row sep=crcr, x=Lower, y=Count] {%
Lower	Upper	Count\\
27.667	40.439	6.2638e-06\\
40.439	59.107	8.571e-06\\
59.107	86.392	2.7366e-05\\
86.392	126.27	5.7507e-05\\
126.27	184.56	9.5623e-05\\
184.56	269.76	0.00015525\\
269.76	394.29	0.00023942\\
394.29	576.29	0.00028922\\
576.29	842.32	0.00032027\\
842.32	1231.2	0.00031651\\
1231.2	1799.5	0.00026051\\
1799.5	2630.2	0.00017936\\
2630.2	3844.3	0.00010532\\
3844.3	5618.9	5.243e-05\\
5618.9	8212.7	2.3852e-05\\
8212.7	12004	1.0931e-05\\
12004	17545	4.7114e-06\\
17545	25644	2.262e-06\\
25644	37482	8.8534e-07\\
37482	54785	2.8205e-07\\
54785	80074	1.065e-07\\
80074	1.1704e+05	2.6693e-08\\
1.1704e+05	1.7107e+05	3.4058e-08\\
1.7107e+05	2.5003e+05	7.7669e-09\\
2.5003e+05	2.5003e+05	7.7669e-09\\
};
\addlegendentry{Istogramma appr.}

\addplot [color=mycolor1, only marks, every error bar/.append style={opacity=0.45}, mark=*, mark size=0pt, draw=none, forget plot]
 plot [error bars/.cd, y dir=both, y explicit, error bar style={line width=1pt, color=mycolor1}, error mark options={mark=none,mark size=0pt}]
 table[row sep=crcr, y error plus index=2, y error minus index=3]{%
33.449	6.2638e-06	7.1029e-06	6.2638e-06\\
48.89	8.571e-06	6.7654e-06	6.7654e-06\\
71.459	2.7366e-05	9.1749e-06	9.1749e-06\\
104.45	5.7507e-05	1.1318e-05	1.1318e-05\\
152.66	9.5623e-05	1.3543e-05	1.3543e-05\\
223.13	0.00015525	1.3268e-05	1.3268e-05\\
326.13	0.00023942	1.2325e-05	1.2325e-05\\
476.68	0.00028922	1.1596e-05	1.1596e-05\\
696.73	0.00032027	1.0521e-05	1.0521e-05\\
1018.3	0.00031651	7.8553e-06	7.8553e-06\\
1488.4	0.00026051	5.6374e-06	5.6374e-06\\
2175.5	0.00017936	4.9436e-06	4.9436e-06\\
3179.8	0.00010532	2.6398e-06	2.6398e-06\\
4647.7	5.243e-05	1.4394e-06	1.4394e-06\\
6793.1	2.3852e-05	7.583e-07	7.583e-07\\
9929	1.0931e-05	4.8087e-07	4.8087e-07\\
14512	4.7114e-06	2.5254e-07	2.5254e-07\\
21212	2.262e-06	1.4991e-07	1.4991e-07\\
31003	8.8534e-07	8.0187e-08	8.0187e-08\\
45315	2.8205e-07	3.7656e-08	3.7656e-08\\
66233	1.065e-07	1.4104e-08	1.4104e-08\\
96808	2.6693e-08	8.0502e-09	8.0502e-09\\
1.415e+05	3.4058e-08	4.5525e-09	4.5525e-09\\
2.0681e+05	7.7669e-09	2.834e-09	2.834e-09\\
};
\addplot [color=mycolor1]
  table[row sep=crcr]{%
27.667	6.8918e-06\\
31.435	8.5847e-06\\
35.069	1.0325e-05\\
39.124	1.238e-05\\
43.252	1.4583e-05\\
47.381	1.6891e-05\\
51.434	1.925e-05\\
55.833	2.1909e-05\\
60.058	2.4551e-05\\
64.602	2.7485e-05\\
69.491	3.0738e-05\\
74.071	3.387e-05\\
78.953	3.7292e-05\\
84.156	4.1026e-05\\
89.703	4.5095e-05\\
94.747	4.8868e-05\\
100.08	5.2918e-05\\
105.7	5.7262e-05\\
111.65	6.1914e-05\\
117.93	6.6886e-05\\
124.56	7.2193e-05\\
131.56	7.7846e-05\\
138.96	8.3855e-05\\
145.44	8.9141e-05\\
152.23	9.4683e-05\\
159.33	0.00010048\\
166.76	0.00010655\\
174.54	0.00011287\\
182.68	0.00011944\\
192.95	0.00012767\\
203.8	0.00013625\\
215.26	0.00014516\\
227.37	0.00015439\\
240.15	0.0001639\\
255.98	0.00017532\\
275.35	0.00018872\\
301.64	0.00020584\\
372.02	0.00024537\\
396.54	0.00025702\\
418.84	0.0002667\\
438.37	0.00027449\\
458.82	0.00028199\\
480.23	0.00028915\\
498.06	0.00029459\\
516.56	0.00029976\\
535.75	0.00030463\\
555.65	0.00030918\\
576.29	0.00031339\\
597.7	0.00031723\\
619.9	0.00032069\\
637.1	0.00032301\\
654.76	0.0003251\\
672.92	0.00032695\\
691.58	0.00032855\\
710.76	0.00032989\\
730.47	0.00033097\\
750.73	0.00033178\\
771.55	0.00033232\\
792.95	0.00033259\\
814.94	0.00033258\\
837.54	0.00033229\\
860.76	0.00033172\\
884.64	0.00033087\\
909.17	0.00032974\\
934.38	0.00032833\\
960.29	0.00032665\\
986.92	0.0003247\\
1014.3	0.00032249\\
1042.4	0.00032\\
1071.3	0.00031727\\
1101	0.00031427\\
1131.6	0.00031104\\
1163	0.00030755\\
1195.2	0.00030384\\
1239.6	0.00029853\\
1285.7	0.00029282\\
1333.4	0.00028673\\
1382.9	0.00028028\\
1434.3	0.00027349\\
1487.6	0.00026637\\
1542.8	0.00025896\\
1600.2	0.00025128\\
1674.8	0.00024134\\
1752.9	0.0002311\\
1868.5	0.00021642\\
2283.5	0.00016992\\
2411.9	0.00015772\\
2524.4	0.00014787\\
2642.2	0.00013833\\
2765.4	0.00012915\\
2868.2	0.00012207\\
2974.7	0.00011524\\
3085.2	0.00010867\\
3199.8	0.00010236\\
3318.7	9.6318e-05\\
3441.9	9.0536e-05\\
3569.8	8.5015e-05\\
3702.4	7.9754e-05\\
3839.9	7.4748e-05\\
3982.6	6.9993e-05\\
4168.3	6.4394e-05\\
4362.8	5.9164e-05\\
4566.3	5.4292e-05\\
4779.3	4.9761e-05\\
5002.2	4.5557e-05\\
5235.5	4.1664e-05\\
5479.8	3.8065e-05\\
5735.4	3.4744e-05\\
6002.9	3.1685e-05\\
6340.5	2.8337e-05\\
6697	2.5315e-05\\
7073.6	2.2593e-05\\
7471.4	2.0147e-05\\
7963.8	1.7608e-05\\
8488.7	1.5374e-05\\
9131.1	1.3154e-05\\
9822	1.1245e-05\\
10662	9.4174e-06\\
11680	7.7264e-06\\
12912	6.2088e-06\\
14405	4.8859e-06\\
16218	3.7636e-06\\
18595	2.779e-06\\
21912	1.9232e-06\\
26536	1.2419e-06\\
33636	7.1121e-07\\
46281	3.2366e-07\\
73681	9.3644e-08\\
1.8506e+05	5.6253e-09\\
2.5003e+05	2.0329e-09\\
};
\addlegendentry{Adatt. BLN appr. ($\mathit{ML}$)}


\addplot[area legend, draw=none, fill=mycolor1, fill opacity=0.15, forget plot]
table[row sep=crcr] {%
x	y\\
27.667	8.2361e-06\\
27.921	8.3634e-06\\
28.177	8.4924e-06\\
28.435	8.6231e-06\\
28.695	8.7557e-06\\
28.958	8.89e-06\\
29.223	9.0261e-06\\
29.491	9.1641e-06\\
29.761	9.304e-06\\
30.034	9.4457e-06\\
30.309	9.5893e-06\\
30.586	9.7349e-06\\
30.867	9.8824e-06\\
31.149	1.0032e-05\\
31.435	1.0183e-05\\
31.723	1.0337e-05\\
32.013	1.0492e-05\\
32.306	1.065e-05\\
32.602	1.081e-05\\
32.901	1.0972e-05\\
33.202	1.1136e-05\\
33.506	1.1302e-05\\
33.813	1.147e-05\\
34.123	1.1641e-05\\
34.436	1.1813e-05\\
34.751	1.1988e-05\\
35.069	1.2166e-05\\
35.391	1.2345e-05\\
35.715	1.2527e-05\\
36.042	1.2712e-05\\
36.372	1.2899e-05\\
36.705	1.3088e-05\\
37.042	1.328e-05\\
37.381	1.3474e-05\\
37.723	1.3671e-05\\
38.069	1.387e-05\\
38.417	1.4072e-05\\
38.769	1.4276e-05\\
39.124	1.4483e-05\\
39.483	1.4693e-05\\
39.845	1.4905e-05\\
40.209	1.5121e-05\\
40.578	1.5338e-05\\
40.949	1.5559e-05\\
41.325	1.5783e-05\\
41.703	1.6009e-05\\
42.085	1.6238e-05\\
42.471	1.6471e-05\\
42.86	1.6706e-05\\
43.252	1.6944e-05\\
43.648	1.7185e-05\\
44.048	1.7429e-05\\
44.452	1.7677e-05\\
44.859	1.7927e-05\\
45.27	1.8181e-05\\
45.684	1.8438e-05\\
46.103	1.8698e-05\\
46.525	1.8961e-05\\
46.951	1.9228e-05\\
47.381	1.9498e-05\\
47.815	1.9771e-05\\
48.253	2.0048e-05\\
48.695	2.0329e-05\\
49.141	2.0612e-05\\
49.592	2.09e-05\\
50.046	2.1191e-05\\
50.504	2.1485e-05\\
50.967	2.1784e-05\\
51.434	2.2086e-05\\
51.905	2.2391e-05\\
52.38	2.2701e-05\\
52.86	2.3014e-05\\
53.344	2.3331e-05\\
53.833	2.3653e-05\\
54.326	2.3978e-05\\
54.824	2.4307e-05\\
55.326	2.464e-05\\
55.833	2.4977e-05\\
56.344	2.5319e-05\\
56.86	2.5665e-05\\
57.381	2.6014e-05\\
57.907	2.6369e-05\\
58.437	2.6727e-05\\
58.972	2.709e-05\\
59.512	2.7458e-05\\
60.058	2.7829e-05\\
60.608	2.8206e-05\\
61.163	2.8587e-05\\
61.723	2.8972e-05\\
62.289	2.9363e-05\\
62.859	2.9758e-05\\
63.435	3.0158e-05\\
64.016	3.0562e-05\\
64.602	3.0972e-05\\
65.194	3.1386e-05\\
65.791	3.1806e-05\\
66.394	3.2231e-05\\
67.002	3.266e-05\\
67.616	3.3095e-05\\
68.235	3.3535e-05\\
68.86	3.3981e-05\\
69.491	3.4431e-05\\
70.127	3.4888e-05\\
70.77	3.5349e-05\\
71.418	3.5816e-05\\
72.072	3.6289e-05\\
72.732	3.6767e-05\\
73.399	3.7251e-05\\
74.071	3.7741e-05\\
74.749	3.8236e-05\\
75.434	3.8737e-05\\
76.125	3.9245e-05\\
76.822	3.9758e-05\\
77.526	4.0277e-05\\
78.236	4.0802e-05\\
78.953	4.1334e-05\\
79.676	4.1872e-05\\
80.406	4.2416e-05\\
81.142	4.2966e-05\\
81.886	4.3523e-05\\
82.636	4.4086e-05\\
83.393	4.4656e-05\\
84.156	4.5232e-05\\
84.927	4.5816e-05\\
85.705	4.6405e-05\\
86.49	4.7002e-05\\
87.283	4.7605e-05\\
88.082	4.8216e-05\\
88.889	4.8833e-05\\
89.703	4.9458e-05\\
90.525	5.0089e-05\\
91.354	5.0728e-05\\
92.191	5.1374e-05\\
93.035	5.2027e-05\\
93.887	5.2688e-05\\
94.747	5.3356e-05\\
95.615	5.4032e-05\\
96.491	5.4715e-05\\
97.375	5.5406e-05\\
98.267	5.6104e-05\\
99.167	5.6811e-05\\
100.08	5.7525e-05\\
100.99	5.8247e-05\\
101.92	5.8977e-05\\
102.85	5.9715e-05\\
103.79	6.0461e-05\\
104.74	6.1216e-05\\
105.7	6.1978e-05\\
106.67	6.2749e-05\\
107.65	6.3528e-05\\
108.63	6.4316e-05\\
109.63	6.5112e-05\\
110.63	6.5917e-05\\
111.65	6.673e-05\\
112.67	6.7552e-05\\
113.7	6.8383e-05\\
114.74	6.9222e-05\\
115.79	7.007e-05\\
116.85	7.0928e-05\\
117.93	7.1794e-05\\
119.01	7.2669e-05\\
120.1	7.3553e-05\\
121.2	7.4447e-05\\
122.31	7.5349e-05\\
123.43	7.6261e-05\\
124.56	7.7182e-05\\
125.7	7.8113e-05\\
126.85	7.9053e-05\\
128.01	8.0002e-05\\
129.18	8.0961e-05\\
130.37	8.1929e-05\\
131.56	8.2907e-05\\
132.77	8.3895e-05\\
133.98	8.4892e-05\\
135.21	8.5899e-05\\
136.45	8.6916e-05\\
137.7	8.7943e-05\\
138.96	8.898e-05\\
140.23	9.0026e-05\\
141.52	9.1082e-05\\
142.81	9.2149e-05\\
144.12	9.3225e-05\\
145.44	9.4312e-05\\
146.77	9.5408e-05\\
148.12	9.6515e-05\\
149.47	9.7631e-05\\
150.84	9.8758e-05\\
152.23	9.9895e-05\\
153.62	0.00010104\\
155.03	0.0001022\\
156.45	0.00010337\\
157.88	0.00010455\\
159.33	0.00010573\\
160.79	0.00010693\\
162.26	0.00010814\\
163.74	0.00010936\\
165.24	0.00011059\\
166.76	0.00011183\\
168.29	0.00011308\\
169.83	0.00011434\\
171.38	0.00011561\\
172.95	0.00011689\\
174.54	0.00011818\\
176.14	0.00011948\\
177.75	0.00012079\\
179.38	0.00012211\\
181.02	0.00012344\\
182.68	0.00012478\\
184.35	0.00012614\\
186.04	0.0001275\\
187.74	0.00012887\\
189.46	0.00013025\\
191.2	0.00013164\\
192.95	0.00013304\\
194.72	0.00013445\\
196.5	0.00013588\\
198.3	0.00013731\\
200.12	0.00013875\\
201.95	0.0001402\\
203.8	0.00014165\\
205.67	0.00014312\\
207.55	0.0001446\\
209.45	0.00014609\\
211.37	0.00014758\\
213.31	0.00014909\\
215.26	0.0001506\\
217.23	0.00015213\\
219.22	0.00015366\\
221.23	0.0001552\\
223.26	0.00015675\\
225.3	0.00015831\\
227.37	0.00015987\\
229.45	0.00016144\\
231.55	0.00016303\\
233.67	0.00016462\\
235.81	0.00016621\\
237.97	0.00016782\\
240.15	0.00016943\\
242.35	0.00017105\\
244.57	0.00017268\\
246.81	0.00017431\\
249.07	0.00017595\\
251.35	0.0001776\\
253.66	0.00017925\\
255.98	0.00018091\\
258.32	0.00018258\\
260.69	0.00018425\\
263.08	0.00018593\\
265.49	0.00018761\\
267.92	0.0001893\\
270.37	0.00019099\\
272.85	0.00019269\\
275.35	0.00019439\\
277.87	0.0001961\\
280.42	0.00019781\\
282.99	0.00019952\\
285.58	0.00020124\\
288.19	0.00020296\\
290.83	0.00020469\\
293.5	0.00020641\\
296.19	0.00020814\\
298.9	0.00020988\\
301.64	0.00021161\\
304.4	0.00021335\\
307.19	0.00021509\\
310	0.00021683\\
312.84	0.00021857\\
315.71	0.00022031\\
318.6	0.00022205\\
321.52	0.00022379\\
324.46	0.00022553\\
327.44	0.00022727\\
330.43	0.00022901\\
333.46	0.00023075\\
336.52	0.00023248\\
339.6	0.00023422\\
342.71	0.00023595\\
345.85	0.00023768\\
349.02	0.00023941\\
352.21	0.00024113\\
355.44	0.00024285\\
358.7	0.00024457\\
361.98	0.00024628\\
365.3	0.00024798\\
368.64	0.00024969\\
372.02	0.00025138\\
375.43	0.00025307\\
378.87	0.00025476\\
382.34	0.00025644\\
385.84	0.00025811\\
389.37	0.00025977\\
392.94	0.00026143\\
396.54	0.00026308\\
400.17	0.00026472\\
403.84	0.00026635\\
407.54	0.00026797\\
411.27	0.00026959\\
415.04	0.00027119\\
418.84	0.00027278\\
422.67	0.00027437\\
426.55	0.00027594\\
430.45	0.0002775\\
434.4	0.00027905\\
438.37	0.00028058\\
442.39	0.0002821\\
446.44	0.00028361\\
450.53	0.00028511\\
454.66	0.00028659\\
458.82	0.00028806\\
463.03	0.00028952\\
467.27	0.00029096\\
471.55	0.00029238\\
475.87	0.00029379\\
480.23	0.00029518\\
484.62	0.00029656\\
489.06	0.00029791\\
493.54	0.00029926\\
498.06	0.00030058\\
502.63	0.00030189\\
507.23	0.00030317\\
511.88	0.00030444\\
516.56	0.00030569\\
521.3	0.00030692\\
526.07	0.00030813\\
530.89	0.00030932\\
535.75	0.00031049\\
540.66	0.00031164\\
545.61	0.00031277\\
550.61	0.00031387\\
555.65	0.00031496\\
560.74	0.00031602\\
565.88	0.00031706\\
571.06	0.00031807\\
576.29	0.00031906\\
581.57	0.00032003\\
586.9	0.00032098\\
592.28	0.0003219\\
597.7	0.0003228\\
603.18	0.00032367\\
608.7	0.00032451\\
614.28	0.00032534\\
619.9	0.00032613\\
625.58	0.0003269\\
631.31	0.00032764\\
637.1	0.00032836\\
642.93	0.00032905\\
648.82	0.00032972\\
654.76	0.00033035\\
660.76	0.00033096\\
666.81	0.00033154\\
672.92	0.0003321\\
679.09	0.00033262\\
685.31	0.00033312\\
691.58	0.00033359\\
697.92	0.00033403\\
704.31	0.00033444\\
710.76	0.00033482\\
717.27	0.00033518\\
723.84	0.0003355\\
730.47	0.0003358\\
737.16	0.00033606\\
743.92	0.0003363\\
750.73	0.0003365\\
757.61	0.00033668\\
764.55	0.00033683\\
771.55	0.00033694\\
778.62	0.00033703\\
785.75	0.00033708\\
792.95	0.00033711\\
800.21	0.0003371\\
807.54	0.00033707\\
814.94	0.000337\\
822.4	0.0003369\\
829.94	0.00033678\\
837.54	0.00033662\\
845.21	0.00033643\\
852.95	0.00033621\\
860.76	0.00033596\\
868.65	0.00033567\\
876.61	0.00033536\\
884.64	0.00033502\\
892.74	0.00033464\\
900.92	0.00033423\\
909.17	0.0003338\\
917.5	0.00033333\\
925.9	0.00033283\\
934.38	0.0003323\\
942.94	0.00033174\\
951.58	0.00033115\\
960.29	0.00033053\\
969.09	0.00032988\\
977.97	0.0003292\\
986.92	0.00032849\\
995.96	0.00032774\\
1005.1	0.00032697\\
1014.3	0.00032617\\
1023.6	0.00032534\\
1033	0.00032448\\
1042.4	0.00032359\\
1052	0.00032267\\
1061.6	0.00032173\\
1071.3	0.00032075\\
1081.1	0.00031975\\
1091	0.00031872\\
1101	0.00031766\\
1111.1	0.00031658\\
1121.3	0.00031546\\
1131.6	0.00031433\\
1141.9	0.00031316\\
1152.4	0.00031197\\
1163	0.00031076\\
1173.6	0.00030952\\
1184.4	0.00030826\\
1195.2	0.00030697\\
1206.2	0.00030565\\
1217.2	0.00030432\\
1228.4	0.00030296\\
1239.6	0.00030158\\
1251	0.00030017\\
1262.4	0.00029875\\
1274	0.0002973\\
1285.7	0.00029583\\
1297.4	0.00029433\\
1309.3	0.00029282\\
1321.3	0.00029128\\
1333.4	0.00028973\\
1345.6	0.00028815\\
1357.9	0.00028655\\
1370.4	0.00028493\\
1382.9	0.00028329\\
1395.6	0.00028163\\
1408.4	0.00027995\\
1421.3	0.00027825\\
1434.3	0.00027653\\
1447.4	0.00027478\\
1460.7	0.00027302\\
1474.1	0.00027124\\
1487.6	0.00026943\\
1501.2	0.00026761\\
1515	0.00026577\\
1528.8	0.00026391\\
1542.8	0.00026203\\
1557	0.00026013\\
1571.2	0.00025821\\
1585.6	0.00025628\\
1600.2	0.00025433\\
1614.8	0.00025236\\
1629.6	0.00025037\\
1644.5	0.00024837\\
1659.6	0.00024636\\
1674.8	0.00024432\\
1690.1	0.00024228\\
1705.6	0.00024022\\
1721.2	0.00023815\\
1737	0.00023606\\
1752.9	0.00023397\\
1769	0.00023186\\
1785.2	0.00022974\\
1801.5	0.00022762\\
1818	0.00022548\\
1834.7	0.00022334\\
1851.5	0.00022119\\
1868.5	0.00021904\\
1885.6	0.00021688\\
1902.8	0.00021471\\
1920.3	0.00021254\\
1937.9	0.00021037\\
1955.6	0.0002082\\
1973.5	0.00020602\\
1991.6	0.00020384\\
2009.8	0.00020167\\
2028.3	0.00019949\\
2046.8	0.00019732\\
2065.6	0.00019514\\
2084.5	0.00019297\\
2103.6	0.00019081\\
2122.9	0.00018865\\
2142.3	0.00018649\\
2161.9	0.00018434\\
2181.7	0.00018219\\
2201.7	0.00018006\\
2221.9	0.00017793\\
2242.2	0.0001758\\
2262.8	0.00017369\\
2283.5	0.00017159\\
2304.4	0.00016949\\
2325.5	0.00016741\\
2346.8	0.00016533\\
2368.3	0.00016327\\
2390	0.00016122\\
2411.9	0.00015918\\
2434	0.00015715\\
2456.3	0.00015514\\
2478.8	0.00015314\\
2501.5	0.00015115\\
2524.4	0.00014918\\
2547.6	0.00014722\\
2570.9	0.00014527\\
2594.4	0.00014334\\
2618.2	0.00014143\\
2642.2	0.00013953\\
2666.4	0.00013764\\
2690.8	0.00013577\\
2715.5	0.00013392\\
2740.3	0.00013208\\
2765.4	0.00013026\\
2790.8	0.00012846\\
2816.3	0.00012667\\
2842.1	0.00012489\\
2868.2	0.00012314\\
2894.4	0.0001214\\
2920.9	0.00011967\\
2947.7	0.00011797\\
2974.7	0.00011628\\
3001.9	0.0001146\\
3029.4	0.00011294\\
3057.2	0.0001113\\
3085.2	0.00010968\\
3113.5	0.00010807\\
3142	0.00010648\\
3170.8	0.0001049\\
3199.8	0.00010335\\
3229.1	0.0001018\\
3258.7	0.00010028\\
3288.5	9.8769e-05\\
3318.7	9.7276e-05\\
3349.1	9.58e-05\\
3379.7	9.434e-05\\
3410.7	9.2896e-05\\
3441.9	9.1469e-05\\
3473.5	9.0058e-05\\
3505.3	8.8663e-05\\
3537.4	8.7284e-05\\
3569.8	8.5921e-05\\
3602.5	8.4574e-05\\
3635.5	8.3243e-05\\
3668.8	8.1928e-05\\
3702.4	8.0629e-05\\
3736.3	7.9346e-05\\
3770.5	7.8078e-05\\
3805.1	7.6826e-05\\
3839.9	7.559e-05\\
3875.1	7.4369e-05\\
3910.6	7.3163e-05\\
3946.4	7.1973e-05\\
3982.6	7.0798e-05\\
4019	6.9639e-05\\
4055.9	6.8494e-05\\
4093	6.7364e-05\\
4130.5	6.625e-05\\
4168.3	6.515e-05\\
4206.5	6.4065e-05\\
4245	6.2994e-05\\
4283.9	6.1939e-05\\
4323.2	6.0897e-05\\
4362.8	5.987e-05\\
4402.7	5.8857e-05\\
4443.1	5.7858e-05\\
4483.8	5.6874e-05\\
4524.8	5.5903e-05\\
4566.3	5.4946e-05\\
4608.1	5.4002e-05\\
4650.3	5.3073e-05\\
4692.9	5.2156e-05\\
4735.9	5.1253e-05\\
4779.3	5.0363e-05\\
4823.1	4.9487e-05\\
4867.2	4.8623e-05\\
4911.8	4.7772e-05\\
4956.8	4.6934e-05\\
5002.2	4.6109e-05\\
5048	4.5296e-05\\
5094.3	4.4496e-05\\
5140.9	4.3707e-05\\
5188	4.2931e-05\\
5235.5	4.2167e-05\\
5283.5	4.1415e-05\\
5331.9	4.0674e-05\\
5380.7	3.9945e-05\\
5430	3.9228e-05\\
5479.8	3.8522e-05\\
5530	3.7827e-05\\
5580.6	3.7144e-05\\
5631.7	3.6471e-05\\
5683.3	3.5809e-05\\
5735.4	3.5158e-05\\
5787.9	3.4517e-05\\
5840.9	3.3887e-05\\
5894.4	3.3268e-05\\
5948.4	3.2658e-05\\
6002.9	3.2059e-05\\
6057.9	3.147e-05\\
6113.4	3.089e-05\\
6169.4	3.032e-05\\
6225.9	2.976e-05\\
6282.9	2.9209e-05\\
6340.5	2.8668e-05\\
6398.5	2.8135e-05\\
6457.2	2.7612e-05\\
6516.3	2.7098e-05\\
6576	2.6593e-05\\
6636.2	2.6096e-05\\
6697	2.5608e-05\\
6758.4	2.5128e-05\\
6820.3	2.4657e-05\\
6882.7	2.4194e-05\\
6945.8	2.3739e-05\\
7009.4	2.3292e-05\\
7073.6	2.2853e-05\\
7138.4	2.2422e-05\\
7203.8	2.1998e-05\\
7269.8	2.1582e-05\\
7336.4	2.1174e-05\\
7403.6	2.0772e-05\\
7471.4	2.0378e-05\\
7539.8	1.9991e-05\\
7608.9	1.9611e-05\\
7678.6	1.9238e-05\\
7748.9	1.8871e-05\\
7819.9	1.8511e-05\\
7891.5	1.8158e-05\\
7963.8	1.7811e-05\\
8036.8	1.7471e-05\\
8110.4	1.7137e-05\\
8184.7	1.6808e-05\\
8259.6	1.6486e-05\\
8335.3	1.617e-05\\
8411.6	1.586e-05\\
8488.7	1.5555e-05\\
8566.5	1.5256e-05\\
8644.9	1.4963e-05\\
8724.1	1.4675e-05\\
8804	1.4392e-05\\
8884.7	1.4115e-05\\
8966	1.3842e-05\\
9048.2	1.3575e-05\\
9131.1	1.3313e-05\\
9214.7	1.3056e-05\\
9299.1	1.2804e-05\\
9384.3	1.2556e-05\\
9470.2	1.2313e-05\\
9557	1.2075e-05\\
9644.5	1.1841e-05\\
9732.9	1.1611e-05\\
9822	1.1386e-05\\
9912	1.1165e-05\\
10003	1.0949e-05\\
10094	1.0736e-05\\
10187	1.0527e-05\\
10280	1.0323e-05\\
10374	1.0122e-05\\
10469	9.9251e-06\\
10565	9.732e-06\\
10662	9.5425e-06\\
10760	9.3566e-06\\
10858	9.1743e-06\\
10958	8.9955e-06\\
11058	8.8201e-06\\
11159	8.648e-06\\
11262	8.4792e-06\\
11365	8.3137e-06\\
11469	8.1513e-06\\
11574	7.992e-06\\
11680	7.8358e-06\\
11787	7.6826e-06\\
11895	7.5323e-06\\
12004	7.3849e-06\\
12114	7.2403e-06\\
12225	7.0986e-06\\
12337	6.9595e-06\\
12450	6.8231e-06\\
12564	6.6893e-06\\
12679	6.5581e-06\\
12795	6.4295e-06\\
12912	6.3033e-06\\
13030	6.1795e-06\\
13150	6.0581e-06\\
13270	5.9391e-06\\
13392	5.8223e-06\\
13515	5.7078e-06\\
13638	5.5955e-06\\
13763	5.4854e-06\\
13889	5.3774e-06\\
14017	5.2714e-06\\
14145	5.1676e-06\\
14274	5.0657e-06\\
14405	4.9658e-06\\
14537	4.8678e-06\\
14670	4.7717e-06\\
14805	4.6775e-06\\
14940	4.585e-06\\
15077	4.4944e-06\\
15215	4.4055e-06\\
15355	4.3183e-06\\
15495	4.2329e-06\\
15637	4.149e-06\\
15780	4.0668e-06\\
15925	3.9862e-06\\
16071	3.9071e-06\\
16218	3.8296e-06\\
16367	3.7535e-06\\
16517	3.6789e-06\\
16668	3.6058e-06\\
16821	3.5341e-06\\
16975	3.4637e-06\\
17130	3.3948e-06\\
17287	3.3271e-06\\
17445	3.2608e-06\\
17605	3.1957e-06\\
17766	3.1319e-06\\
17929	3.0694e-06\\
18093	3.008e-06\\
18259	2.9478e-06\\
18426	2.8888e-06\\
18595	2.831e-06\\
18765	2.7742e-06\\
18937	2.7186e-06\\
19111	2.664e-06\\
19286	2.6105e-06\\
19463	2.558e-06\\
19641	2.5065e-06\\
19821	2.4561e-06\\
20002	2.4066e-06\\
20186	2.358e-06\\
20370	2.3105e-06\\
20557	2.2638e-06\\
20745	2.218e-06\\
20935	2.1731e-06\\
21127	2.1291e-06\\
21321	2.086e-06\\
21516	2.0436e-06\\
21713	2.0021e-06\\
21912	1.9614e-06\\
22113	1.9215e-06\\
22315	1.8824e-06\\
22520	1.844e-06\\
22726	1.8064e-06\\
22934	1.7695e-06\\
23144	1.7333e-06\\
23356	1.6979e-06\\
23570	1.6631e-06\\
23786	1.629e-06\\
24004	1.5955e-06\\
24224	1.5627e-06\\
24446	1.5306e-06\\
24669	1.499e-06\\
24895	1.4681e-06\\
25123	1.4378e-06\\
25354	1.4081e-06\\
25586	1.379e-06\\
25820	1.3504e-06\\
26057	1.3224e-06\\
26295	1.2949e-06\\
26536	1.268e-06\\
26779	1.2416e-06\\
27025	1.2157e-06\\
27272	1.1903e-06\\
27522	1.1655e-06\\
27774	1.1411e-06\\
28028	1.1172e-06\\
28285	1.0937e-06\\
28544	1.0707e-06\\
28806	1.0482e-06\\
29070	1.0261e-06\\
29336	1.0045e-06\\
29605	9.8328e-07\\
29876	9.6248e-07\\
30149	9.421e-07\\
30426	9.2211e-07\\
30704	9.0253e-07\\
30986	8.8333e-07\\
31269	8.6451e-07\\
31556	8.4607e-07\\
31845	8.2799e-07\\
32137	8.1027e-07\\
32431	7.9291e-07\\
32728	7.7589e-07\\
33028	7.5922e-07\\
33330	7.4287e-07\\
33636	7.2686e-07\\
33944	7.1116e-07\\
34255	6.9578e-07\\
34568	6.8071e-07\\
34885	6.6594e-07\\
35205	6.5147e-07\\
35527	6.3729e-07\\
35852	6.234e-07\\
36181	6.0979e-07\\
36512	5.9646e-07\\
36847	5.8339e-07\\
37184	5.7059e-07\\
37525	5.5805e-07\\
37869	5.4577e-07\\
38215	5.3374e-07\\
38565	5.2195e-07\\
38919	5.104e-07\\
39275	4.9909e-07\\
39635	4.8802e-07\\
39998	4.7717e-07\\
40364	4.6654e-07\\
40734	4.5613e-07\\
41107	4.4594e-07\\
41484	4.3596e-07\\
41864	4.2619e-07\\
42247	4.1662e-07\\
42634	4.0724e-07\\
43025	3.9807e-07\\
43419	3.8908e-07\\
43817	3.8028e-07\\
44218	3.7167e-07\\
44623	3.6323e-07\\
45032	3.5498e-07\\
45444	3.469e-07\\
45860	3.3898e-07\\
46281	3.3124e-07\\
46704	3.2366e-07\\
47132	3.1624e-07\\
47564	3.0897e-07\\
48000	3.0187e-07\\
48439	2.9491e-07\\
48883	2.881e-07\\
49331	2.8144e-07\\
49783	2.7492e-07\\
50239	2.6854e-07\\
50699	2.623e-07\\
51163	2.5619e-07\\
51632	2.5021e-07\\
52105	2.4437e-07\\
52582	2.3865e-07\\
53064	2.3305e-07\\
53550	2.2758e-07\\
54040	2.2222e-07\\
54535	2.1699e-07\\
55035	2.1187e-07\\
55539	2.0685e-07\\
56048	2.0195e-07\\
56561	1.9716e-07\\
57079	1.9248e-07\\
57602	1.8789e-07\\
58130	1.8341e-07\\
58662	1.7903e-07\\
59200	1.7475e-07\\
59742	1.7056e-07\\
60289	1.6646e-07\\
60841	1.6246e-07\\
61399	1.5854e-07\\
61961	1.5472e-07\\
62529	1.5098e-07\\
63101	1.4732e-07\\
63679	1.4375e-07\\
64263	1.4025e-07\\
64851	1.3684e-07\\
65445	1.335e-07\\
66045	1.3024e-07\\
66650	1.2706e-07\\
67260	1.2395e-07\\
67876	1.209e-07\\
68498	1.1793e-07\\
69125	1.1503e-07\\
69759	1.1219e-07\\
70398	1.0942e-07\\
71042	1.0671e-07\\
71693	1.0407e-07\\
72350	1.0148e-07\\
73013	9.8958e-08\\
73681	9.6493e-08\\
74356	9.4086e-08\\
75037	9.1734e-08\\
75725	8.9438e-08\\
76418	8.7195e-08\\
77118	8.5005e-08\\
77825	8.2867e-08\\
78538	8.0779e-08\\
79257	7.874e-08\\
79983	7.6749e-08\\
80716	7.4805e-08\\
81455	7.2908e-08\\
82201	7.1056e-08\\
82954	6.9248e-08\\
83714	6.7482e-08\\
84481	6.576e-08\\
85255	6.4078e-08\\
86036	6.2437e-08\\
86824	6.0835e-08\\
87619	5.9271e-08\\
88421	5.7746e-08\\
89231	5.6257e-08\\
90049	5.4804e-08\\
90874	5.3387e-08\\
91706	5.2003e-08\\
92546	5.0654e-08\\
93394	4.9337e-08\\
94249	4.8053e-08\\
95113	4.68e-08\\
95984	4.5577e-08\\
96863	4.4385e-08\\
97750	4.3222e-08\\
98646	4.2088e-08\\
99549	4.0981e-08\\
1.0046e+05	3.9902e-08\\
1.0138e+05	3.885e-08\\
1.0231e+05	3.7824e-08\\
1.0325e+05	3.6823e-08\\
1.0419e+05	3.5847e-08\\
1.0515e+05	3.4895e-08\\
1.0611e+05	3.3968e-08\\
1.0708e+05	3.3063e-08\\
1.0806e+05	3.2181e-08\\
1.0905e+05	3.1322e-08\\
1.1005e+05	3.0483e-08\\
1.1106e+05	2.9667e-08\\
1.1208e+05	2.887e-08\\
1.131e+05	2.8094e-08\\
1.1414e+05	2.7337e-08\\
1.1519e+05	2.66e-08\\
1.1624e+05	2.5882e-08\\
1.1731e+05	2.5181e-08\\
1.1838e+05	2.4499e-08\\
1.1946e+05	2.3834e-08\\
1.2056e+05	2.3186e-08\\
1.2166e+05	2.2555e-08\\
1.2278e+05	2.1939e-08\\
1.239e+05	2.134e-08\\
1.2504e+05	2.0756e-08\\
1.2618e+05	2.0188e-08\\
1.2734e+05	1.9634e-08\\
1.285e+05	1.9094e-08\\
1.2968e+05	1.8568e-08\\
1.3087e+05	1.8056e-08\\
1.3207e+05	1.7558e-08\\
1.3328e+05	1.7072e-08\\
1.345e+05	1.6599e-08\\
1.3573e+05	1.6139e-08\\
1.3697e+05	1.569e-08\\
1.3823e+05	1.5253e-08\\
1.3949e+05	1.4828e-08\\
1.4077e+05	1.4414e-08\\
1.4206e+05	1.4011e-08\\
1.4336e+05	1.3619e-08\\
1.4468e+05	1.3237e-08\\
1.46e+05	1.2865e-08\\
1.4734e+05	1.2503e-08\\
1.4869e+05	1.2151e-08\\
1.5005e+05	1.1808e-08\\
1.5142e+05	1.1475e-08\\
1.5281e+05	1.115e-08\\
1.5421e+05	1.0834e-08\\
1.5562e+05	1.0526e-08\\
1.5705e+05	1.0227e-08\\
1.5849e+05	9.9358e-09\\
1.5994e+05	9.6525e-09\\
1.614e+05	9.3768e-09\\
1.6288e+05	9.1086e-09\\
1.6438e+05	8.8477e-09\\
1.6588e+05	8.5939e-09\\
1.674e+05	8.3471e-09\\
1.6893e+05	8.1069e-09\\
1.7048e+05	7.8734e-09\\
1.7204e+05	7.6462e-09\\
1.7362e+05	7.4252e-09\\
1.7521e+05	7.2104e-09\\
1.7681e+05	7.0014e-09\\
1.7843e+05	6.7982e-09\\
1.8007e+05	6.6006e-09\\
1.8172e+05	6.4085e-09\\
1.8338e+05	6.2217e-09\\
1.8506e+05	6.0401e-09\\
1.8676e+05	5.8635e-09\\
1.8847e+05	5.6919e-09\\
1.9019e+05	5.525e-09\\
1.9194e+05	5.3628e-09\\
1.9369e+05	5.2052e-09\\
1.9547e+05	5.0519e-09\\
1.9726e+05	4.903e-09\\
1.9907e+05	4.7583e-09\\
2.0089e+05	4.6176e-09\\
2.0273e+05	4.4809e-09\\
2.0459e+05	4.348e-09\\
2.0646e+05	4.2189e-09\\
2.0835e+05	4.0935e-09\\
2.1026e+05	3.9717e-09\\
2.1219e+05	3.8533e-09\\
2.1413e+05	3.7382e-09\\
2.1609e+05	3.6265e-09\\
2.1807e+05	3.5179e-09\\
2.2007e+05	3.4125e-09\\
2.2208e+05	3.3101e-09\\
2.2412e+05	3.2106e-09\\
2.2617e+05	3.114e-09\\
2.2824e+05	3.0201e-09\\
2.3033e+05	2.929e-09\\
2.3244e+05	2.8405e-09\\
2.3457e+05	2.7545e-09\\
2.3672e+05	2.6711e-09\\
2.3889e+05	2.5901e-09\\
2.4108e+05	2.5114e-09\\
2.4329e+05	2.435e-09\\
2.4551e+05	2.3608e-09\\
2.4776e+05	2.2888e-09\\
2.5003e+05	2.2189e-09\\
2.5003e+05	1.8469e-09\\
2.4776e+05	1.9072e-09\\
2.4551e+05	1.9694e-09\\
2.4329e+05	2.0335e-09\\
2.4108e+05	2.0996e-09\\
2.3889e+05	2.1678e-09\\
2.3672e+05	2.2381e-09\\
2.3457e+05	2.3106e-09\\
2.3244e+05	2.3853e-09\\
2.3033e+05	2.4624e-09\\
2.2824e+05	2.5418e-09\\
2.2617e+05	2.6236e-09\\
2.2412e+05	2.708e-09\\
2.2208e+05	2.7949e-09\\
2.2007e+05	2.8846e-09\\
2.1807e+05	2.9769e-09\\
2.1609e+05	3.0721e-09\\
2.1413e+05	3.1702e-09\\
2.1219e+05	3.2713e-09\\
2.1026e+05	3.3754e-09\\
2.0835e+05	3.4828e-09\\
2.0646e+05	3.5933e-09\\
2.0459e+05	3.7073e-09\\
2.0273e+05	3.8246e-09\\
2.0089e+05	3.9455e-09\\
1.9907e+05	4.0701e-09\\
1.9726e+05	4.1984e-09\\
1.9547e+05	4.3305e-09\\
1.9369e+05	4.4666e-09\\
1.9194e+05	4.6068e-09\\
1.9019e+05	4.7511e-09\\
1.8847e+05	4.8998e-09\\
1.8676e+05	5.0529e-09\\
1.8506e+05	5.2105e-09\\
1.8338e+05	5.3728e-09\\
1.8172e+05	5.5399e-09\\
1.8007e+05	5.712e-09\\
1.7843e+05	5.8891e-09\\
1.7681e+05	6.0715e-09\\
1.7521e+05	6.2592e-09\\
1.7362e+05	6.4525e-09\\
1.7204e+05	6.6514e-09\\
1.7048e+05	6.8561e-09\\
1.6893e+05	7.0668e-09\\
1.674e+05	7.2837e-09\\
1.6588e+05	7.5068e-09\\
1.6438e+05	7.7365e-09\\
1.6288e+05	7.9728e-09\\
1.614e+05	8.2159e-09\\
1.5994e+05	8.4661e-09\\
1.5849e+05	8.7235e-09\\
1.5705e+05	8.9883e-09\\
1.5562e+05	9.2607e-09\\
1.5421e+05	9.5409e-09\\
1.5281e+05	9.8292e-09\\
1.5142e+05	1.0126e-08\\
1.5005e+05	1.0431e-08\\
1.4869e+05	1.0744e-08\\
1.4734e+05	1.1067e-08\\
1.46e+05	1.1399e-08\\
1.4468e+05	1.174e-08\\
1.4336e+05	1.209e-08\\
1.4206e+05	1.2451e-08\\
1.4077e+05	1.2822e-08\\
1.3949e+05	1.3203e-08\\
1.3823e+05	1.3595e-08\\
1.3697e+05	1.3998e-08\\
1.3573e+05	1.4412e-08\\
1.345e+05	1.4838e-08\\
1.3328e+05	1.5275e-08\\
1.3207e+05	1.5725e-08\\
1.3087e+05	1.6187e-08\\
1.2968e+05	1.6662e-08\\
1.285e+05	1.715e-08\\
1.2734e+05	1.7651e-08\\
1.2618e+05	1.8167e-08\\
1.2504e+05	1.8696e-08\\
1.239e+05	1.924e-08\\
1.2278e+05	1.9799e-08\\
1.2166e+05	2.0373e-08\\
1.2056e+05	2.0962e-08\\
1.1946e+05	2.1568e-08\\
1.1838e+05	2.219e-08\\
1.1731e+05	2.2829e-08\\
1.1624e+05	2.3485e-08\\
1.1519e+05	2.4159e-08\\
1.1414e+05	2.4851e-08\\
1.131e+05	2.5561e-08\\
1.1208e+05	2.6291e-08\\
1.1106e+05	2.704e-08\\
1.1005e+05	2.7809e-08\\
1.0905e+05	2.8598e-08\\
1.0806e+05	2.9408e-08\\
1.0708e+05	3.024e-08\\
1.0611e+05	3.1094e-08\\
1.0515e+05	3.197e-08\\
1.0419e+05	3.2869e-08\\
1.0325e+05	3.3792e-08\\
1.0231e+05	3.4739e-08\\
1.0138e+05	3.5711e-08\\
1.0046e+05	3.6708e-08\\
99549	3.7731e-08\\
98646	3.878e-08\\
97750	3.9857e-08\\
96863	4.0961e-08\\
95984	4.2094e-08\\
95113	4.3256e-08\\
94249	4.4448e-08\\
93394	4.567e-08\\
92546	4.6924e-08\\
91706	4.8209e-08\\
90874	4.9527e-08\\
90049	5.0879e-08\\
89231	5.2265e-08\\
88421	5.3685e-08\\
87619	5.5142e-08\\
86824	5.6635e-08\\
86036	5.8165e-08\\
85255	5.9734e-08\\
84481	6.1342e-08\\
83714	6.299e-08\\
82954	6.4678e-08\\
82201	6.6409e-08\\
81455	6.8182e-08\\
80716	6.9999e-08\\
79983	7.186e-08\\
79257	7.3768e-08\\
78538	7.5721e-08\\
77825	7.7723e-08\\
77118	7.9773e-08\\
76418	8.1873e-08\\
75725	8.4024e-08\\
75037	8.6227e-08\\
74356	8.8483e-08\\
73681	9.0794e-08\\
73013	9.3159e-08\\
72350	9.5582e-08\\
71693	9.8062e-08\\
71042	1.006e-07\\
70398	1.032e-07\\
69759	1.0586e-07\\
69125	1.0859e-07\\
68498	1.1138e-07\\
67876	1.1423e-07\\
67260	1.1715e-07\\
66650	1.2014e-07\\
66045	1.232e-07\\
65445	1.2633e-07\\
64851	1.2954e-07\\
64263	1.3282e-07\\
63679	1.3617e-07\\
63101	1.396e-07\\
62529	1.4312e-07\\
61961	1.4671e-07\\
61399	1.5038e-07\\
60841	1.5414e-07\\
60289	1.5799e-07\\
59742	1.6192e-07\\
59200	1.6594e-07\\
58662	1.7006e-07\\
58130	1.7427e-07\\
57602	1.7857e-07\\
57079	1.8297e-07\\
56561	1.8747e-07\\
56048	1.9207e-07\\
55539	1.9677e-07\\
55035	2.0158e-07\\
54535	2.065e-07\\
54040	2.1152e-07\\
53550	2.1666e-07\\
53064	2.2191e-07\\
52582	2.2728e-07\\
52105	2.3277e-07\\
51632	2.3838e-07\\
51163	2.4411e-07\\
50699	2.4997e-07\\
50239	2.5596e-07\\
49783	2.6208e-07\\
49331	2.6833e-07\\
48883	2.7472e-07\\
48439	2.8125e-07\\
48000	2.8792e-07\\
47564	2.9474e-07\\
47132	3.017e-07\\
46704	3.0882e-07\\
46281	3.1609e-07\\
45860	3.2351e-07\\
45444	3.311e-07\\
45032	3.3885e-07\\
44623	3.4676e-07\\
44218	3.5485e-07\\
43817	3.6311e-07\\
43419	3.7154e-07\\
43025	3.8016e-07\\
42634	3.8896e-07\\
42247	3.9794e-07\\
41864	4.0712e-07\\
41484	4.1649e-07\\
41107	4.2606e-07\\
40734	4.3583e-07\\
40364	4.4581e-07\\
39998	4.56e-07\\
39635	4.664e-07\\
39275	4.7703e-07\\
38919	4.8787e-07\\
38565	4.9894e-07\\
38215	5.1025e-07\\
37869	5.2179e-07\\
37525	5.3357e-07\\
37184	5.456e-07\\
36847	5.5788e-07\\
36512	5.7041e-07\\
36181	5.832e-07\\
35852	5.9626e-07\\
35527	6.0959e-07\\
35205	6.232e-07\\
34885	6.3709e-07\\
34568	6.5126e-07\\
34255	6.6572e-07\\
33944	6.8049e-07\\
33636	6.9556e-07\\
33330	7.1093e-07\\
33028	7.2663e-07\\
32728	7.4264e-07\\
32431	7.5898e-07\\
32137	7.7566e-07\\
31845	7.9268e-07\\
31556	8.1005e-07\\
31269	8.2777e-07\\
30986	8.4585e-07\\
30704	8.643e-07\\
30426	8.8312e-07\\
30149	9.0233e-07\\
29876	9.2193e-07\\
29605	9.4193e-07\\
29336	9.6233e-07\\
29070	9.8315e-07\\
28806	1.0044e-06\\
28544	1.0261e-06\\
28285	1.0482e-06\\
28028	1.0707e-06\\
27774	1.0937e-06\\
27522	1.1172e-06\\
27272	1.1411e-06\\
27025	1.1656e-06\\
26779	1.1905e-06\\
26536	1.2159e-06\\
26295	1.2418e-06\\
26057	1.2683e-06\\
25820	1.2953e-06\\
25586	1.3228e-06\\
25354	1.3509e-06\\
25123	1.3795e-06\\
24895	1.4088e-06\\
24669	1.4386e-06\\
24446	1.469e-06\\
24224	1.5e-06\\
24004	1.5316e-06\\
23786	1.5639e-06\\
23570	1.5968e-06\\
23356	1.6304e-06\\
23144	1.6646e-06\\
22934	1.6995e-06\\
22726	1.7352e-06\\
22520	1.7715e-06\\
22315	1.8086e-06\\
22113	1.8464e-06\\
21912	1.885e-06\\
21713	1.9243e-06\\
21516	1.9644e-06\\
21321	2.0053e-06\\
21127	2.0471e-06\\
20935	2.0897e-06\\
20745	2.1331e-06\\
20557	2.1774e-06\\
20370	2.2225e-06\\
20186	2.2686e-06\\
20002	2.3156e-06\\
19821	2.3636e-06\\
19641	2.4124e-06\\
19463	2.4623e-06\\
19286	2.5132e-06\\
19111	2.5651e-06\\
18937	2.618e-06\\
18765	2.6719e-06\\
18595	2.727e-06\\
18426	2.7831e-06\\
18259	2.8404e-06\\
18093	2.8988e-06\\
17929	2.9584e-06\\
17766	3.0192e-06\\
17605	3.0811e-06\\
17445	3.1443e-06\\
17287	3.2088e-06\\
17130	3.2746e-06\\
16975	3.3417e-06\\
16821	3.4101e-06\\
16668	3.4799e-06\\
16517	3.551e-06\\
16367	3.6236e-06\\
16218	3.6977e-06\\
16071	3.7732e-06\\
15925	3.8502e-06\\
15780	3.9288e-06\\
15637	4.0089e-06\\
15495	4.0906e-06\\
15355	4.174e-06\\
15215	4.259e-06\\
15077	4.3457e-06\\
14940	4.4342e-06\\
14805	4.5244e-06\\
14670	4.6164e-06\\
14537	4.7102e-06\\
14405	4.8059e-06\\
14274	4.9036e-06\\
14145	5.0031e-06\\
14017	5.1047e-06\\
13889	5.2082e-06\\
13763	5.3138e-06\\
13638	5.4216e-06\\
13515	5.5314e-06\\
13392	5.6435e-06\\
13270	5.7577e-06\\
13150	5.8743e-06\\
13030	5.9931e-06\\
12912	6.1144e-06\\
12795	6.238e-06\\
12679	6.364e-06\\
12564	6.4926e-06\\
12450	6.6237e-06\\
12337	6.7574e-06\\
12225	6.8938e-06\\
12114	7.0328e-06\\
12004	7.1746e-06\\
11895	7.3191e-06\\
11787	7.4666e-06\\
11680	7.6169e-06\\
11574	7.7702e-06\\
11469	7.9265e-06\\
11365	8.0859e-06\\
11262	8.2484e-06\\
11159	8.414e-06\\
11058	8.583e-06\\
10958	8.7552e-06\\
10858	8.9308e-06\\
10760	9.1098e-06\\
10662	9.2923e-06\\
10565	9.4784e-06\\
10469	9.6681e-06\\
10374	9.8615e-06\\
10280	1.0059e-05\\
10187	1.026e-05\\
10094	1.0464e-05\\
10003	1.0673e-05\\
9912	1.0886e-05\\
9822	1.1103e-05\\
9732.9	1.1324e-05\\
9644.5	1.1549e-05\\
9557	1.1779e-05\\
9470.2	1.2013e-05\\
9384.3	1.2252e-05\\
9299.1	1.2495e-05\\
9214.7	1.2742e-05\\
9131.1	1.2995e-05\\
9048.2	1.3252e-05\\
8966	1.3514e-05\\
8884.7	1.3781e-05\\
8804	1.4053e-05\\
8724.1	1.433e-05\\
8644.9	1.4613e-05\\
8566.5	1.49e-05\\
8488.7	1.5194e-05\\
8411.6	1.5492e-05\\
8335.3	1.5796e-05\\
8259.6	1.6106e-05\\
8184.7	1.6422e-05\\
8110.4	1.6743e-05\\
8036.8	1.707e-05\\
7963.8	1.7404e-05\\
7891.5	1.7743e-05\\
7819.9	1.8089e-05\\
7748.9	1.8441e-05\\
7678.6	1.88e-05\\
7608.9	1.9165e-05\\
7539.8	1.9536e-05\\
7471.4	1.9915e-05\\
7403.6	2.03e-05\\
7336.4	2.0692e-05\\
7269.8	2.1092e-05\\
7203.8	2.1498e-05\\
7138.4	2.1912e-05\\
7073.6	2.2333e-05\\
7009.4	2.2762e-05\\
6945.8	2.3198e-05\\
6882.7	2.3642e-05\\
6820.3	2.4094e-05\\
6758.4	2.4554e-05\\
6697	2.5022e-05\\
6636.2	2.5498e-05\\
6576	2.5982e-05\\
6516.3	2.6475e-05\\
6457.2	2.6977e-05\\
6398.5	2.7487e-05\\
6340.5	2.8006e-05\\
6282.9	2.8533e-05\\
6225.9	2.907e-05\\
6169.4	2.9616e-05\\
6113.4	3.0172e-05\\
6057.9	3.0736e-05\\
6002.9	3.1311e-05\\
5948.4	3.1895e-05\\
5894.4	3.2489e-05\\
5840.9	3.3092e-05\\
5787.9	3.3706e-05\\
5735.4	3.433e-05\\
5683.3	3.4965e-05\\
5631.7	3.561e-05\\
5580.6	3.6265e-05\\
5530	3.6931e-05\\
5479.8	3.7608e-05\\
5430	3.8296e-05\\
5380.7	3.8995e-05\\
5331.9	3.9706e-05\\
5283.5	4.0428e-05\\
5235.5	4.1161e-05\\
5188	4.1906e-05\\
5140.9	4.2663e-05\\
5094.3	4.3432e-05\\
5048	4.4213e-05\\
5002.2	4.5006e-05\\
4956.8	4.5811e-05\\
4911.8	4.6629e-05\\
4867.2	4.746e-05\\
4823.1	4.8303e-05\\
4779.3	4.9159e-05\\
4735.9	5.0028e-05\\
4692.9	5.0911e-05\\
4650.3	5.1806e-05\\
4608.1	5.2715e-05\\
4566.3	5.3638e-05\\
4524.8	5.4574e-05\\
4483.8	5.5524e-05\\
4443.1	5.6488e-05\\
4402.7	5.7466e-05\\
4362.8	5.8459e-05\\
4323.2	5.9465e-05\\
4283.9	6.0486e-05\\
4245	6.1522e-05\\
4206.5	6.2572e-05\\
4168.3	6.3637e-05\\
4130.5	6.4717e-05\\
4093	6.5812e-05\\
4055.9	6.6922e-05\\
4019	6.8048e-05\\
3982.6	6.9188e-05\\
3946.4	7.0344e-05\\
3910.6	7.1516e-05\\
3875.1	7.2704e-05\\
3839.9	7.3907e-05\\
3805.1	7.5126e-05\\
3770.5	7.6361e-05\\
3736.3	7.7612e-05\\
3702.4	7.8879e-05\\
3668.8	8.0162e-05\\
3635.5	8.1461e-05\\
3602.5	8.2777e-05\\
3569.8	8.4109e-05\\
3537.4	8.5458e-05\\
3505.3	8.6823e-05\\
3473.5	8.8204e-05\\
3441.9	8.9602e-05\\
3410.7	9.1017e-05\\
3379.7	9.2448e-05\\
3349.1	9.3896e-05\\
3318.7	9.536e-05\\
3288.5	9.6841e-05\\
3258.7	9.8338e-05\\
3229.1	9.9852e-05\\
3199.8	0.00010138\\
3170.8	0.00010293\\
3142	0.00010449\\
3113.5	0.00010607\\
3085.2	0.00010767\\
3057.2	0.00010928\\
3029.4	0.00011091\\
3001.9	0.00011255\\
2974.7	0.00011421\\
2947.7	0.00011588\\
2920.9	0.00011757\\
2894.4	0.00011928\\
2868.2	0.000121\\
2842.1	0.00012274\\
2816.3	0.00012449\\
2790.8	0.00012625\\
2765.4	0.00012803\\
2740.3	0.00012982\\
2715.5	0.00013163\\
2690.8	0.00013345\\
2666.4	0.00013529\\
2642.2	0.00013713\\
2618.2	0.00013899\\
2594.4	0.00014087\\
2570.9	0.00014275\\
2547.6	0.00014465\\
2524.4	0.00014656\\
2501.5	0.00014848\\
2478.8	0.00015041\\
2456.3	0.00015235\\
2434	0.00015431\\
2411.9	0.00015627\\
2390	0.00015824\\
2368.3	0.00016023\\
2346.8	0.00016222\\
2325.5	0.00016422\\
2304.4	0.00016623\\
2283.5	0.00016825\\
2262.8	0.00017027\\
2242.2	0.00017231\\
2221.9	0.00017434\\
2201.7	0.00017639\\
2181.7	0.00017844\\
2161.9	0.0001805\\
2142.3	0.00018256\\
2122.9	0.00018463\\
2103.6	0.0001867\\
2084.5	0.00018878\\
2065.6	0.00019086\\
2046.8	0.00019294\\
2028.3	0.00019503\\
2009.8	0.00019711\\
1991.6	0.0001992\\
1973.5	0.00020129\\
1955.6	0.00020337\\
1937.9	0.00020546\\
1920.3	0.00020755\\
1902.8	0.00020963\\
1885.6	0.00021172\\
1868.5	0.0002138\\
1851.5	0.00021587\\
1834.7	0.00021795\\
1818	0.00022002\\
1801.5	0.00022208\\
1785.2	0.00022414\\
1769	0.00022619\\
1752.9	0.00022824\\
1737	0.00023028\\
1721.2	0.00023231\\
1705.6	0.00023433\\
1690.1	0.00023635\\
1674.8	0.00023836\\
1659.6	0.00024035\\
1644.5	0.00024234\\
1629.6	0.00024431\\
1614.8	0.00024627\\
1600.2	0.00024822\\
1585.6	0.00025016\\
1571.2	0.00025209\\
1557	0.000254\\
1542.8	0.00025589\\
1528.8	0.00025777\\
1515	0.00025964\\
1501.2	0.00026148\\
1487.6	0.00026332\\
1474.1	0.00026513\\
1460.7	0.00026692\\
1447.4	0.0002687\\
1434.3	0.00027046\\
1421.3	0.00027219\\
1408.4	0.00027391\\
1395.6	0.0002756\\
1382.9	0.00027728\\
1370.4	0.00027893\\
1357.9	0.00028055\\
1345.6	0.00028216\\
1333.4	0.00028374\\
1321.3	0.00028529\\
1309.3	0.00028683\\
1297.4	0.00028833\\
1285.7	0.00028981\\
1274	0.00029127\\
1262.4	0.0002927\\
1251	0.0002941\\
1239.6	0.00029547\\
1228.4	0.00029682\\
1217.2	0.00029815\\
1206.2	0.00029944\\
1195.2	0.00030071\\
1184.4	0.00030195\\
1173.6	0.00030316\\
1163	0.00030435\\
1152.4	0.00030551\\
1141.9	0.00030664\\
1131.6	0.00030774\\
1121.3	0.00030882\\
1111.1	0.00030987\\
1101	0.00031089\\
1091	0.00031188\\
1081.1	0.00031284\\
1071.3	0.00031378\\
1061.6	0.00031469\\
1052	0.00031557\\
1042.4	0.00031642\\
1033	0.00031724\\
1023.6	0.00031804\\
1014.3	0.0003188\\
1005.1	0.00031954\\
995.96	0.00032024\\
986.92	0.00032092\\
977.97	0.00032157\\
969.09	0.00032219\\
960.29	0.00032278\\
951.58	0.00032334\\
942.94	0.00032387\\
934.38	0.00032437\\
925.9	0.00032484\\
917.5	0.00032527\\
909.17	0.00032568\\
900.92	0.00032606\\
892.74	0.0003264\\
884.64	0.00032672\\
876.61	0.000327\\
868.65	0.00032726\\
860.76	0.00032748\\
852.95	0.00032767\\
845.21	0.00032783\\
837.54	0.00032795\\
829.94	0.00032805\\
822.4	0.00032812\\
814.94	0.00032815\\
807.54	0.00032815\\
800.21	0.00032812\\
792.95	0.00032806\\
785.75	0.00032797\\
778.62	0.00032785\\
771.55	0.0003277\\
764.55	0.00032751\\
757.61	0.0003273\\
750.73	0.00032706\\
743.92	0.00032678\\
737.16	0.00032647\\
730.47	0.00032614\\
723.84	0.00032577\\
717.27	0.00032538\\
710.76	0.00032496\\
704.31	0.0003245\\
697.92	0.00032402\\
691.58	0.00032351\\
685.31	0.00032297\\
679.09	0.0003224\\
672.92	0.00032181\\
666.81	0.00032118\\
660.76	0.00032053\\
654.76	0.00031986\\
648.82	0.00031915\\
642.93	0.00031842\\
637.1	0.00031766\\
631.31	0.00031688\\
625.58	0.00031607\\
619.9	0.00031524\\
614.28	0.00031438\\
608.7	0.0003135\\
603.18	0.00031259\\
597.7	0.00031166\\
592.28	0.00031071\\
586.9	0.00030973\\
581.57	0.00030873\\
576.29	0.00030771\\
571.06	0.00030667\\
565.88	0.0003056\\
560.74	0.00030451\\
555.65	0.00030341\\
550.61	0.00030228\\
545.61	0.00030113\\
540.66	0.00029996\\
535.75	0.00029877\\
530.89	0.00029756\\
526.07	0.00029634\\
521.3	0.00029509\\
516.56	0.00029383\\
511.88	0.00029255\\
507.23	0.00029125\\
502.63	0.00028993\\
498.06	0.0002886\\
493.54	0.00028725\\
489.06	0.00028589\\
484.62	0.00028451\\
480.23	0.00028311\\
475.87	0.0002817\\
471.55	0.00028028\\
467.27	0.00027884\\
463.03	0.00027739\\
458.82	0.00027592\\
454.66	0.00027444\\
450.53	0.00027295\\
446.44	0.00027145\\
442.39	0.00026994\\
438.37	0.00026841\\
434.4	0.00026687\\
430.45	0.00026532\\
426.55	0.00026376\\
422.67	0.00026219\\
418.84	0.00026061\\
415.04	0.00025903\\
411.27	0.00025743\\
407.54	0.00025582\\
403.84	0.00025421\\
400.17	0.00025258\\
396.54	0.00025095\\
392.94	0.00024932\\
389.37	0.00024767\\
385.84	0.00024602\\
382.34	0.00024436\\
378.87	0.0002427\\
375.43	0.00024103\\
372.02	0.00023935\\
368.64	0.00023767\\
365.3	0.00023599\\
361.98	0.0002343\\
358.7	0.00023261\\
355.44	0.00023091\\
352.21	0.00022921\\
349.02	0.00022751\\
345.85	0.0002258\\
342.71	0.00022409\\
339.6	0.00022238\\
336.52	0.00022067\\
333.46	0.00021895\\
330.43	0.00021724\\
327.44	0.00021552\\
324.46	0.0002138\\
321.52	0.00021208\\
318.6	0.00021037\\
315.71	0.00020865\\
312.84	0.00020693\\
310	0.00020521\\
307.19	0.00020349\\
304.4	0.00020178\\
301.64	0.00020007\\
298.9	0.00019835\\
296.19	0.00019664\\
293.5	0.00019493\\
290.83	0.00019323\\
288.19	0.00019153\\
285.58	0.00018983\\
282.99	0.00018813\\
280.42	0.00018644\\
277.87	0.00018475\\
275.35	0.00018306\\
272.85	0.00018138\\
270.37	0.0001797\\
267.92	0.00017803\\
265.49	0.00017636\\
263.08	0.00017469\\
260.69	0.00017304\\
258.32	0.00017138\\
255.98	0.00016974\\
253.66	0.00016809\\
251.35	0.00016646\\
249.07	0.00016483\\
246.81	0.0001632\\
244.57	0.00016159\\
242.35	0.00015998\\
240.15	0.00015837\\
237.97	0.00015678\\
235.81	0.00015519\\
233.67	0.0001536\\
231.55	0.00015203\\
229.45	0.00015046\\
227.37	0.0001489\\
225.3	0.00014735\\
223.26	0.00014581\\
221.23	0.00014427\\
219.22	0.00014274\\
217.23	0.00014123\\
215.26	0.00013972\\
213.31	0.00013821\\
211.37	0.00013672\\
209.45	0.00013524\\
207.55	0.00013376\\
205.67	0.00013229\\
203.8	0.00013084\\
201.95	0.00012939\\
200.12	0.00012795\\
198.3	0.00012652\\
196.5	0.0001251\\
194.72	0.00012369\\
192.95	0.00012229\\
191.2	0.0001209\\
189.46	0.00011952\\
187.74	0.00011815\\
186.04	0.00011679\\
184.35	0.00011544\\
182.68	0.0001141\\
181.02	0.00011277\\
179.38	0.00011145\\
177.75	0.00011014\\
176.14	0.00010884\\
174.54	0.00010755\\
172.95	0.00010627\\
171.38	0.00010501\\
169.83	0.00010375\\
168.29	0.0001025\\
166.76	0.00010126\\
165.24	0.00010004\\
163.74	9.882e-05\\
162.26	9.7615e-05\\
160.79	9.6419e-05\\
159.33	9.5235e-05\\
157.88	9.4061e-05\\
156.45	9.2897e-05\\
155.03	9.1745e-05\\
153.62	9.0602e-05\\
152.23	8.9471e-05\\
150.84	8.8349e-05\\
149.47	8.7239e-05\\
148.12	8.6138e-05\\
146.77	8.5049e-05\\
145.44	8.397e-05\\
144.12	8.2901e-05\\
142.81	8.1843e-05\\
141.52	8.0795e-05\\
140.23	7.9758e-05\\
138.96	7.8731e-05\\
137.7	7.7714e-05\\
136.45	7.6708e-05\\
135.21	7.5712e-05\\
133.98	7.4726e-05\\
132.77	7.375e-05\\
131.56	7.2785e-05\\
130.37	7.183e-05\\
129.18	7.0885e-05\\
128.01	6.995e-05\\
126.85	6.9025e-05\\
125.7	6.811e-05\\
124.56	6.7204e-05\\
123.43	6.6309e-05\\
122.31	6.5424e-05\\
121.2	6.4548e-05\\
120.1	6.3682e-05\\
119.01	6.2826e-05\\
117.93	6.1979e-05\\
116.85	6.1142e-05\\
115.79	6.0314e-05\\
114.74	5.9496e-05\\
113.7	5.8687e-05\\
112.67	5.7888e-05\\
111.65	5.7097e-05\\
110.63	5.6316e-05\\
109.63	5.5544e-05\\
108.63	5.4781e-05\\
107.65	5.4027e-05\\
106.67	5.3282e-05\\
105.7	5.2546e-05\\
104.74	5.1819e-05\\
103.79	5.11e-05\\
102.85	5.039e-05\\
101.92	4.9689e-05\\
100.99	4.8996e-05\\
100.08	4.8312e-05\\
99.167	4.7636e-05\\
98.267	4.6968e-05\\
97.375	4.6309e-05\\
96.491	4.5657e-05\\
95.615	4.5014e-05\\
94.747	4.4379e-05\\
93.887	4.3752e-05\\
93.035	4.3133e-05\\
92.191	4.2521e-05\\
91.354	4.1917e-05\\
90.525	4.1321e-05\\
89.703	4.0733e-05\\
88.889	4.0152e-05\\
88.082	3.9578e-05\\
87.283	3.9012e-05\\
86.49	3.8453e-05\\
85.705	3.7902e-05\\
84.927	3.7357e-05\\
84.156	3.682e-05\\
83.393	3.6289e-05\\
82.636	3.5766e-05\\
81.886	3.5249e-05\\
81.142	3.474e-05\\
80.406	3.4237e-05\\
79.676	3.374e-05\\
78.953	3.325e-05\\
78.236	3.2767e-05\\
77.526	3.229e-05\\
76.822	3.182e-05\\
76.125	3.1355e-05\\
75.434	3.0897e-05\\
74.749	3.0445e-05\\
74.071	3e-05\\
73.399	2.956e-05\\
72.732	2.9126e-05\\
72.072	2.8698e-05\\
71.418	2.8276e-05\\
70.77	2.786e-05\\
70.127	2.7449e-05\\
69.491	2.7044e-05\\
68.86	2.6644e-05\\
68.235	2.625e-05\\
67.616	2.5862e-05\\
67.002	2.5478e-05\\
66.394	2.51e-05\\
65.791	2.4728e-05\\
65.194	2.436e-05\\
64.602	2.3997e-05\\
64.016	2.364e-05\\
63.435	2.3287e-05\\
62.859	2.294e-05\\
62.289	2.2597e-05\\
61.723	2.2259e-05\\
61.163	2.1926e-05\\
60.608	2.1597e-05\\
60.058	2.1273e-05\\
59.512	2.0953e-05\\
58.972	2.0638e-05\\
58.437	2.0328e-05\\
57.907	2.0022e-05\\
57.381	1.972e-05\\
56.86	1.9422e-05\\
56.344	1.9129e-05\\
55.833	1.884e-05\\
55.326	1.8554e-05\\
54.824	1.8273e-05\\
54.326	1.7996e-05\\
53.833	1.7723e-05\\
53.344	1.7454e-05\\
52.86	1.7188e-05\\
52.38	1.6927e-05\\
51.905	1.6669e-05\\
51.434	1.6414e-05\\
50.967	1.6164e-05\\
50.504	1.5917e-05\\
50.046	1.5673e-05\\
49.592	1.5433e-05\\
49.141	1.5197e-05\\
48.695	1.4964e-05\\
48.253	1.4734e-05\\
47.815	1.4507e-05\\
47.381	1.4284e-05\\
46.951	1.4064e-05\\
46.525	1.3847e-05\\
46.103	1.3634e-05\\
45.684	1.3423e-05\\
45.27	1.3216e-05\\
44.859	1.3011e-05\\
44.452	1.2809e-05\\
44.048	1.2611e-05\\
43.648	1.2415e-05\\
43.252	1.2222e-05\\
42.86	1.2032e-05\\
42.471	1.1844e-05\\
42.085	1.166e-05\\
41.703	1.1478e-05\\
41.325	1.1298e-05\\
40.949	1.1122e-05\\
40.578	1.0948e-05\\
40.209	1.0776e-05\\
39.845	1.0607e-05\\
39.483	1.044e-05\\
39.124	1.0276e-05\\
38.769	1.0114e-05\\
38.417	9.9551e-06\\
38.069	9.7981e-06\\
37.723	9.6433e-06\\
37.381	9.4909e-06\\
37.042	9.3406e-06\\
36.705	9.1926e-06\\
36.372	9.0468e-06\\
36.042	8.9031e-06\\
35.715	8.7615e-06\\
35.391	8.622e-06\\
35.069	8.4846e-06\\
34.751	8.3492e-06\\
34.436	8.2158e-06\\
34.123	8.0844e-06\\
33.813	7.9549e-06\\
33.506	7.8274e-06\\
33.202	7.7017e-06\\
32.901	7.5779e-06\\
32.602	7.456e-06\\
32.306	7.3358e-06\\
32.013	7.2175e-06\\
31.723	7.1009e-06\\
31.435	6.986e-06\\
31.149	6.8729e-06\\
30.867	6.7614e-06\\
30.586	6.6516e-06\\
30.309	6.5435e-06\\
30.034	6.437e-06\\
29.761	6.332e-06\\
29.491	6.2287e-06\\
29.223	6.1269e-06\\
28.958	6.0266e-06\\
28.695	5.9278e-06\\
28.435	5.8305e-06\\
28.177	5.7347e-06\\
27.921	5.6403e-06\\
27.667	5.5474e-06\\
}--cycle;
\addplot[ybar interval, fill=black, fill opacity=0.35, area legend, draw=none] table[row sep=crcr, x=Lower, y=Count] {%
Lower	Upper	Count\\
27.667	40.439	0\\
40.439	59.107	0\\
59.107	86.392	0\\
86.392	126.27	6.6867e-05\\
126.27	184.56	9.1497e-05\\
184.56	269.76	0.0002817\\
269.76	394.29	0.00019273\\
394.29	576.29	0.00033698\\
576.29	842.32	0.00035084\\
842.32	1231.2	0.00038405\\
1231.2	1799.5	0.00025807\\
1799.5	2630.2	0.00016051\\
2630.2	3844.3	0.00010542\\
3844.3	5618.9	5.1091e-05\\
5618.9	8212.7	1.9534e-05\\
8212.7	12004	9.8475e-06\\
12004	17545	3.8499e-06\\
17545	25644	9.8776e-07\\
25644	37482	6.758e-07\\
37482	54785	4.6236e-07\\
54785	80074	1.0544e-07\\
80074	1.1704e+05	0\\
1.1704e+05	1.7107e+05	4.9358e-08\\
1.7107e+05	2.5003e+05	3.3769e-08\\
2.5003e+05	2.5003e+05	3.3769e-08\\
};
\addlegendentry{Istogramma reale}

\addplot [color=black]
  table[row sep=crcr]{%
27.667	9.8285e-07\\
32.013	1.6563e-06\\
36.042	2.4867e-06\\
39.845	3.4641e-06\\
43.648	4.6365e-06\\
47.381	5.9792e-06\\
50.967	7.4482e-06\\
54.824	9.2223e-06\\
58.437	1.1063e-05\\
62.289	1.3211e-05\\
65.791	1.5325e-05\\
69.491	1.7716e-05\\
73.399	2.0411e-05\\
77.526	2.3437e-05\\
81.142	2.623e-05\\
84.927	2.9288e-05\\
88.889	3.2625e-05\\
93.035	3.6256e-05\\
97.375	4.0197e-05\\
101.92	4.4462e-05\\
106.67	4.9063e-05\\
111.65	5.4013e-05\\
116.85	5.9322e-05\\
122.31	6.5001e-05\\
128.01	7.1055e-05\\
132.77	7.6172e-05\\
137.7	8.1536e-05\\
142.81	8.7145e-05\\
148.12	9.3e-05\\
155.03	0.00010066\\
162.26	0.0001087\\
169.83	0.0001171\\
177.75	0.00012585\\
186.04	0.00013494\\
194.72	0.00014435\\
203.8	0.00015405\\
215.26	0.00016603\\
227.37	0.00017834\\
242.35	0.00019303\\
263.08	0.00021227\\
310	0.00025088\\
330.43	0.00026552\\
349.02	0.00027774\\
365.3	0.0002876\\
382.34	0.00029711\\
396.54	0.00030442\\
411.27	0.00031145\\
426.55	0.00031816\\
442.39	0.00032453\\
458.82	0.00033052\\
475.87	0.00033612\\
489.06	0.00034005\\
502.63	0.00034373\\
516.56	0.00034716\\
530.89	0.00035033\\
545.61	0.00035322\\
560.74	0.00035584\\
576.29	0.00035817\\
592.28	0.00036022\\
608.7	0.00036197\\
625.58	0.00036342\\
642.93	0.00036457\\
660.76	0.00036541\\
679.09	0.00036595\\
697.92	0.00036618\\
717.27	0.0003661\\
737.16	0.00036571\\
757.61	0.00036501\\
778.62	0.00036401\\
800.21	0.00036271\\
822.4	0.0003611\\
845.21	0.0003592\\
868.65	0.00035701\\
892.74	0.00035453\\
917.5	0.00035178\\
942.94	0.00034875\\
969.09	0.00034545\\
995.96	0.00034191\\
1023.6	0.00033811\\
1052	0.00033407\\
1091	0.00032834\\
1131.6	0.00032222\\
1173.6	0.00031575\\
1217.2	0.00030895\\
1262.4	0.00030184\\
1309.3	0.00029446\\
1370.4	0.0002849\\
1434.3	0.000275\\
1515	0.00026278\\
1614.8	0.00024817\\
1769	0.00022691\\
2028.3	0.00019501\\
2161.9	0.00018043\\
2283.5	0.00016822\\
2411.9	0.00015632\\
2524.4	0.0001467\\
2642.2	0.00013736\\
2765.4	0.00012833\\
2894.4	0.00011963\\
3029.4	0.00011128\\
3170.8	0.00010329\\
3318.7	9.5672e-05\\
3473.5	8.8426e-05\\
3635.5	8.1559e-05\\
3805.1	7.5069e-05\\
3982.6	6.8956e-05\\
4168.3	6.3213e-05\\
4362.8	5.7835e-05\\
4566.3	5.2812e-05\\
4779.3	4.8134e-05\\
5002.2	4.379e-05\\
5235.5	3.9766e-05\\
5479.8	3.6049e-05\\
5735.4	3.2625e-05\\
6002.9	2.9478e-05\\
6282.9	2.6594e-05\\
6636.2	2.3458e-05\\
7009.4	2.0651e-05\\
7403.6	1.8147e-05\\
7819.9	1.5921e-05\\
8335.3	1.3642e-05\\
8884.7	1.167e-05\\
9557	9.7479e-06\\
10280	8.1337e-06\\
11159	6.6314e-06\\
12225	5.2874e-06\\
13515	4.1293e-06\\
15077	3.1661e-06\\
17130	2.3404e-06\\
20002	1.644e-06\\
24446	1.0661e-06\\
32137	6.1208e-07\\
48000	2.7914e-07\\
92546	7.0864e-08\\
2.5003e+05	5.7352e-09\\
};
\addlegendentry{Adatt. BLN reale ($\mathit{ML}$)}

        % \end{groupplot}
    
        % \begin{groupplot}[
        %     group style={
        %         group name=ColL2,
        %         plotLeftGroup
        %     },
        %     plotLeftCol,
        %     plotBottom,
        %     plotLegendSW,
        %     %
        %     enlarge x limits=0.1,
        %     enlarge y limits=0.1,
        %     xmode=log,ymode=log,
        %     xmin=3637.5,xmax=204237,
        % ]
        %     \nextgroupplot[%
        %         at={($(ColL1 c1r3.south west)-(0,\yGroupSep em)-(0,\plotHeightII cm)$)},
        %         % xmin=3825.04007879312,xmax=181881.391659954,
        %         % ymin=0.00531914893617021,ymax=1,
        %     ] % This file was created by matlab2tikz.
%
\definecolor{mycolor1}{rgb}{0.00000,0.44706,0.69804}%
\definecolor{mycolor2}{rgb}{0.00000,0.44700,0.74100}%
\definecolor{mycolor3}{rgb}{0.83529,0.36863,0.00000}%
\definecolor{mycolor4}{rgb}{0.49400,0.18400,0.55600}%
%
\addplot[only marks, mark=*, mark size=1.1180pt, color=mycolor1, fill=mycolor1, opacity=0.60, draw=none, mark options={draw=none,line width=0pt}] table[row sep=crcr]{%
x	y\\
4448.8	0.99468\\
4495.8	0.98404\\
4542.6	0.9734\\
4584	0.96277\\
4638.1	0.95213\\
4680.6	0.94149\\
4729.4	0.93085\\
4791.3	0.92021\\
4843.1	0.90957\\
4888.5	0.89894\\
4949.8	0.8883\\
5011.5	0.87766\\
5078.7	0.86702\\
5140.6	0.85638\\
5199.1	0.84574\\
5259.8	0.83511\\
5324.2	0.82447\\
5392	0.81383\\
5451.4	0.80319\\
5530.6	0.79255\\
5604.7	0.78191\\
5676.5	0.77128\\
5735.5	0.76064\\
5813.4	0.75\\
5888	0.73936\\
5972.3	0.72872\\
6040.5	0.71809\\
6135.5	0.70745\\
6219	0.69681\\
6301.4	0.68617\\
6379.4	0.67553\\
6489.5	0.66489\\
6580	0.65426\\
6680.1	0.64362\\
6784.4	0.63298\\
6879.3	0.62234\\
6999.8	0.6117\\
7088.5	0.60106\\
7211.9	0.59043\\
7315.2	0.57979\\
7430.2	0.56915\\
7594.1	0.55851\\
7737	0.54787\\
7880	0.53723\\
8036.3	0.5266\\
8163.1	0.51596\\
8296.5	0.50532\\
8423.4	0.49468\\
8580.8	0.48404\\
8734.8	0.4734\\
8903.6	0.46277\\
9095.9	0.45213\\
9259.6	0.44149\\
9466.5	0.43085\\
9627.4	0.42021\\
9822	0.40957\\
10012	0.39894\\
10230	0.3883\\
10440	0.37766\\
10691	0.36702\\
10922	0.35638\\
11164	0.34574\\
11407	0.33511\\
11670	0.32447\\
11947	0.31383\\
12239	0.30319\\
12557	0.29255\\
12855	0.28191\\
13179	0.27128\\
13570	0.26064\\
13919	0.25\\
14359	0.23936\\
14747	0.22872\\
15113	0.21809\\
15556	0.20745\\
16118	0.19681\\
16666	0.18617\\
17343	0.17553\\
17930	0.16489\\
18664	0.15426\\
19408	0.14362\\
20220	0.13298\\
21201	0.12234\\
22218	0.1117\\
23579	0.10106\\
24805	0.090426\\
26355	0.079787\\
28038	0.069149\\
31068	0.058511\\
34651	0.047872\\
39510	0.037234\\
46003	0.026596\\
56575	0.015957\\
81621	0.0053191\\
};
\addlegendentry{FRC empirica esatta}

\addplot [color=mycolor1, only marks, every error bar/.append style={opacity=0.30}, mark=*, mark size=0pt, draw=none, forget plot]
 plot [error bars/.cd, x dir=both, x explicit, error bar style={line width=1pt, color=mycolor1}, error mark options={mark=none,mark size=0pt}]
 table[row sep=crcr, x error plus index=2, x error minus index=3]{%
4448.8	0.99468	36.556	36.556\\
4495.8	0.98404	37.517	37.517\\
4542.6	0.9734	36.289	36.289\\
4584	0.96277	39.036	39.036\\
4638.1	0.95213	39.975	39.975\\
4680.6	0.94149	38.656	38.656\\
4729.4	0.93085	37.188	37.188\\
4791.3	0.92021	39.039	39.039\\
4843.1	0.90957	39.99	39.99\\
4888.5	0.89894	39.87	39.87\\
4949.8	0.8883	40.836	40.836\\
5011.5	0.87766	42.304	42.304\\
5078.7	0.86702	44.043	44.043\\
5140.6	0.85638	45.555	45.555\\
5199.1	0.84574	46.546	46.546\\
5259.8	0.83511	48.558	48.558\\
5324.2	0.82447	51.85	51.85\\
5392	0.81383	54.309	54.309\\
5451.4	0.80319	54.052	54.052\\
5530.6	0.79255	57.171	57.171\\
5604.7	0.78191	56.471	56.471\\
5676.5	0.77128	59.211	59.211\\
5735.5	0.76064	59.446	59.446\\
5813.4	0.75	63.624	63.624\\
5888	0.73936	63.978	63.978\\
5972.3	0.72872	68.81	68.81\\
6040.5	0.71809	72.094	72.094\\
6135.5	0.70745	75.986	75.986\\
6219	0.69681	76.447	76.447\\
6301.4	0.68617	78.64	78.64\\
6379.4	0.67553	81.78	81.78\\
6489.5	0.66489	84.094	84.094\\
6580	0.65426	85.688	85.688\\
6680.1	0.64362	87.716	87.716\\
6784.4	0.63298	89.416	89.416\\
6879.3	0.62234	88.855	88.855\\
6999.8	0.6117	93.364	93.364\\
7088.5	0.60106	93.401	93.401\\
7211.9	0.59043	95.147	95.147\\
7315.2	0.57979	96.001	96.001\\
7430.2	0.56915	96.668	96.668\\
7594.1	0.55851	98.875	98.875\\
7737	0.54787	100.12	100.12\\
7880	0.53723	102.08	102.08\\
8036.3	0.5266	110.95	110.95\\
8163.1	0.51596	110.26	110.26\\
8296.5	0.50532	113.92	113.92\\
8423.4	0.49468	113.62	113.62\\
8580.8	0.48404	111.99	111.99\\
8734.8	0.4734	110.02	110.02\\
8903.6	0.46277	114.81	114.81\\
9095.9	0.45213	111.92	111.92\\
9259.6	0.44149	109.77	109.77\\
9466.5	0.43085	116.35	116.35\\
9627.4	0.42021	115.04	115.04\\
9822	0.40957	119.95	119.95\\
10012	0.39894	123.64	123.64\\
10230	0.3883	130.29	130.29\\
10440	0.37766	131.46	131.46\\
10691	0.36702	138.43	138.43\\
10922	0.35638	129.17	129.17\\
11164	0.34574	134.78	134.78\\
11407	0.33511	126.73	126.73\\
11670	0.32447	140.53	140.53\\
11947	0.31383	150.71	150.71\\
12239	0.30319	152.52	152.52\\
12557	0.29255	153.84	153.84\\
12855	0.28191	167.86	167.86\\
13179	0.27128	176.34	176.34\\
13570	0.26064	189.33	189.33\\
13919	0.25	194.07	194.07\\
14359	0.23936	203.53	203.53\\
14747	0.22872	210.77	210.77\\
15113	0.21809	209.39	209.39\\
15556	0.20745	211.76	211.76\\
16118	0.19681	229.46	229.46\\
16666	0.18617	238.49	238.49\\
17343	0.17553	252.97	252.97\\
17930	0.16489	267.54	267.54\\
18664	0.15426	315.04	315.04\\
19408	0.14362	336.88	336.88\\
20220	0.13298	345.81	345.81\\
21201	0.12234	344.35	344.35\\
22218	0.1117	390.31	390.31\\
23579	0.10106	426.29	426.29\\
24805	0.090426	466.46	466.46\\
26355	0.079787	513.58	513.58\\
28038	0.069149	566.7	566.7\\
31068	0.058511	684.02	684.02\\
34651	0.047872	836.31	836.31\\
39510	0.037234	1174.9	1174.9\\
46003	0.026596	1412.7	1412.7\\
56575	0.015957	1860.4	1860.4\\
81621	0.0053191	3216.4	3216.4\\
};
\addplot [color=mycolor1]
  table[row sep=crcr]{%
4432.9	0.97921\\
4591.6	0.95142\\
4755.9	0.91405\\
7001.9	0.5555\\
16283	0.18779\\
37864	0.063695\\
88052	0.021674\\
1.4406e+05	0.011574\\
};
\addlegendentry{Adatt. Pareto esatto}% ($\mathit{ML}$)
% \addlegendentry{Adatt. Pareto esatto ($\mathit{ML}$)}


\addplot[area legend, draw=none, fill=mycolor1, fill opacity=0.15, forget plot]
table[row sep=crcr] {%
x	y\\
4432.9	0.98491\\
4591.6	0.95938\\
4755.9	0.9227\\
4926.1	0.8817\\
5102.4	0.84232\\
5285	0.80472\\
5474.1	0.76881\\
5670.1	0.73451\\
5873	0.70176\\
6083.2	0.67049\\
6300.9	0.64063\\
6526.4	0.61211\\
6759.9	0.58488\\
7001.9	0.55889\\
7252.5	0.53407\\
7512	0.51037\\
7780.9	0.48774\\
8059.3	0.46614\\
8347.8	0.44552\\
8646.5	0.42582\\
8956	0.40702\\
9276.5	0.38906\\
9608.5	0.37192\\
9952.4	0.35554\\
10309	0.3399\\
10677	0.32495\\
11060	0.31067\\
11455	0.29703\\
11865	0.284\\
12290	0.27154\\
12730	0.25963\\
13185	0.24825\\
13657	0.23738\\
14146	0.22698\\
14652	0.21705\\
15177	0.20755\\
15720	0.19846\\
16283	0.18978\\
16865	0.18148\\
17469	0.17355\\
18094	0.16596\\
18742	0.15871\\
19412	0.15177\\
20107	0.14514\\
20827	0.1388\\
21572	0.13274\\
22344	0.12695\\
23144	0.1214\\
23972	0.1161\\
24830	0.11104\\
25719	0.10619\\
26639	0.10156\\
27592	0.097129\\
28580	0.092892\\
29603	0.088841\\
30662	0.084967\\
31760	0.081263\\
32896	0.07772\\
34074	0.074333\\
35293	0.071093\\
36556	0.067995\\
37864	0.065033\\
39220	0.0622\\
40623	0.05949\\
42077	0.0569\\
43583	0.054422\\
45143	0.052052\\
46758	0.049786\\
48432	0.047619\\
50165	0.045546\\
51960	0.043564\\
53820	0.041668\\
55746	0.039855\\
57741	0.038122\\
59808	0.036463\\
61948	0.034878\\
64165	0.033361\\
66461	0.03191\\
68840	0.030523\\
71304	0.029196\\
73856	0.027927\\
76499	0.026713\\
79237	0.025552\\
82072	0.024442\\
85010	0.02338\\
88052	0.022365\\
91203	0.021393\\
94467	0.020464\\
97848	0.019575\\
1.0135e+05	0.018725\\
1.0498e+05	0.017913\\
1.0873e+05	0.017135\\
1.1263e+05	0.016391\\
1.1666e+05	0.01568\\
1.2083e+05	0.014999\\
1.2516e+05	0.014349\\
1.2964e+05	0.013726\\
1.3427e+05	0.013131\\
1.3908e+05	0.012561\\
1.4406e+05	0.012016\\
1.4406e+05	0.011131\\
1.3908e+05	0.011646\\
1.3427e+05	0.012186\\
1.2964e+05	0.01275\\
1.2516e+05	0.01334\\
1.2083e+05	0.013958\\
1.1666e+05	0.014605\\
1.1263e+05	0.015281\\
1.0873e+05	0.015989\\
1.0498e+05	0.01673\\
1.0135e+05	0.017505\\
97848	0.018316\\
94467	0.019165\\
91203	0.020054\\
88052	0.020983\\
85010	0.021956\\
82072	0.022974\\
79237	0.02404\\
76499	0.025155\\
73856	0.026322\\
71304	0.027543\\
68840	0.028821\\
66461	0.030158\\
64165	0.031558\\
61948	0.033022\\
59808	0.034555\\
57741	0.036159\\
55746	0.037838\\
53820	0.039595\\
51960	0.041434\\
50165	0.043359\\
48432	0.045373\\
46758	0.047481\\
45143	0.049687\\
43583	0.051995\\
42077	0.054412\\
40623	0.056941\\
39220	0.059588\\
37864	0.062358\\
36556	0.065257\\
35293	0.068291\\
34074	0.071467\\
32896	0.074791\\
31760	0.078269\\
30662	0.08191\\
29603	0.085721\\
28580	0.089709\\
27592	0.093883\\
26639	0.098252\\
25719	0.10283\\
24830	0.10761\\
23972	0.11262\\
23144	0.11786\\
22344	0.12335\\
21572	0.12909\\
20827	0.1351\\
20107	0.1414\\
19412	0.14798\\
18742	0.15487\\
18094	0.16208\\
17469	0.16963\\
16865	0.17753\\
16283	0.1858\\
15720	0.19446\\
15177	0.20351\\
14652	0.21299\\
14146	0.22291\\
13657	0.23329\\
13185	0.24416\\
12730	0.25552\\
12290	0.26742\\
11865	0.27986\\
11455	0.29288\\
11060	0.3065\\
10677	0.32075\\
10309	0.33566\\
9952.4	0.35125\\
9608.5	0.36755\\
9276.5	0.3846\\
8956	0.40243\\
8646.5	0.42107\\
8347.8	0.44056\\
8059.3	0.46093\\
7780.9	0.48223\\
7512	0.5045\\
7252.5	0.52778\\
7001.9	0.55212\\
6759.9	0.57756\\
6526.4	0.60416\\
6300.9	0.63198\\
6083.2	0.66106\\
5873	0.69148\\
5670.1	0.72328\\
5474.1	0.75653\\
5285	0.79131\\
5102.4	0.82768\\
4926.1	0.86572\\
4755.9	0.9054\\
4591.6	0.94347\\
4432.9	0.97351\\
}--cycle;
\addplot[only marks, mark=*, mark size=1.1180pt, color=mycolor3, fill=mycolor3, opacity=0.60, draw=none, mark options={draw=none,line width=0pt}] table[row sep=crcr]{%
x	y\\
4422.1	0.99468\\
4464.8	0.98404\\
4510.7	0.9734\\
4562.7	0.96277\\
4609.4	0.95213\\
4660.3	0.94149\\
4703.8	0.93085\\
4764.5	0.92021\\
4823.4	0.90957\\
4878.8	0.89894\\
4931.1	0.8883\\
4981.5	0.87766\\
5032.7	0.86702\\
5090.6	0.85638\\
5147.3	0.84574\\
5204.1	0.83511\\
5268.9	0.82447\\
5327.5	0.81383\\
5396.1	0.80319\\
5465.4	0.79255\\
5528.8	0.78191\\
5611.7	0.77128\\
5684.8	0.76064\\
5757.3	0.75\\
5826.3	0.73936\\
5907.5	0.72872\\
5985.8	0.71809\\
6067.1	0.70745\\
6146.4	0.69681\\
6223.3	0.68617\\
6317	0.67553\\
6420.1	0.66489\\
6512	0.65426\\
6617.6	0.64362\\
6721.5	0.63298\\
6820	0.62234\\
6914.7	0.6117\\
7008.4	0.60106\\
7109.9	0.59043\\
7218	0.57979\\
7331.2	0.56915\\
7454.3	0.55851\\
7588.2	0.54787\\
7729.2	0.53723\\
7865.4	0.5266\\
7997.5	0.51596\\
8145.7	0.50532\\
8288	0.49468\\
8465.9	0.48404\\
8612.1	0.4734\\
8772.6	0.46277\\
8971.5	0.45213\\
9149.8	0.44149\\
9376.5	0.43085\\
9583	0.42021\\
9775.6	0.40957\\
9973.9	0.39894\\
10173	0.3883\\
10378	0.37766\\
10624	0.36702\\
10863	0.35638\\
11083	0.34574\\
11390	0.33511\\
11621	0.32447\\
11887	0.31383\\
12143	0.30319\\
12426	0.29255\\
12689	0.28191\\
13007	0.27128\\
13315	0.26064\\
13699	0.25\\
14073	0.23936\\
14500	0.22872\\
14961	0.21809\\
15478	0.20745\\
16061	0.19681\\
16604	0.18617\\
17140	0.17553\\
17638	0.16489\\
18376	0.15426\\
19009	0.14362\\
20015	0.13298\\
21053	0.12234\\
22307	0.1117\\
23541	0.10106\\
25138	0.090426\\
26542	0.079787\\
28800	0.069149\\
31358	0.058511\\
34949	0.047872\\
39886	0.037234\\
46309	0.026596\\
57859	0.015957\\
86479	0.0053191\\
};
\addlegendentry{FRC empirica appr.}

\addplot [color=mycolor3, only marks, every error bar/.append style={opacity=0.30}, mark=*, mark size=0pt, draw=none, forget plot]
 plot [error bars/.cd, x dir=both, x explicit, error bar style={line width=1pt, color=mycolor3}, error mark options={mark=none,mark size=0pt}]
 table[row sep=crcr, x error plus index=2, x error minus index=3]{%
4422.1	0.99468	48.059	48.059\\
4464.8	0.98404	49.114	49.114\\
4510.7	0.9734	49.637	49.637\\
4562.7	0.96277	51.66	51.66\\
4609.4	0.95213	51.566	51.566\\
4660.3	0.94149	50.283	50.283\\
4703.8	0.93085	51.069	51.069\\
4764.5	0.92021	52.715	52.715\\
4823.4	0.90957	53.485	53.485\\
4878.8	0.89894	54.86	54.86\\
4931.1	0.8883	55.763	55.763\\
4981.5	0.87766	54.275	54.275\\
5032.7	0.86702	54.846	54.846\\
5090.6	0.85638	56.572	56.572\\
5147.3	0.84574	56.516	56.516\\
5204.1	0.83511	56.688	56.688\\
5268.9	0.82447	56.043	56.043\\
5327.5	0.81383	55.478	55.478\\
5396.1	0.80319	59.489	59.489\\
5465.4	0.79255	60.152	60.152\\
5528.8	0.78191	59.725	59.725\\
5611.7	0.77128	62.487	62.487\\
5684.8	0.76064	65.931	65.931\\
5757.3	0.75	67.565	67.565\\
5826.3	0.73936	68.645	68.645\\
5907.5	0.72872	69.665	69.665\\
5985.8	0.71809	70.821	70.821\\
6067.1	0.70745	70.517	70.517\\
6146.4	0.69681	71.213	71.213\\
6223.3	0.68617	71.633	71.633\\
6317	0.67553	70.716	70.716\\
6420.1	0.66489	75.234	75.234\\
6512	0.65426	75.997	75.997\\
6617.6	0.64362	76.277	76.277\\
6721.5	0.63298	80.82	80.82\\
6820	0.62234	82.177	82.177\\
6914.7	0.6117	86.711	86.711\\
7008.4	0.60106	88.348	88.348\\
7109.9	0.59043	88.354	88.354\\
7218	0.57979	89.402	89.402\\
7331.2	0.56915	89.672	89.672\\
7454.3	0.55851	86.852	86.852\\
7588.2	0.54787	90.66	90.66\\
7729.2	0.53723	92.844	92.844\\
7865.4	0.5266	93.134	93.134\\
7997.5	0.51596	95.207	95.207\\
8145.7	0.50532	97.355	97.355\\
8288	0.49468	101.71	101.71\\
8465.9	0.48404	105.09	105.09\\
8612.1	0.4734	106.08	106.08\\
8772.6	0.46277	109.61	109.61\\
8971.5	0.45213	116.49	116.49\\
9149.8	0.44149	117.94	117.94\\
9376.5	0.43085	116.97	116.97\\
9583	0.42021	122.95	122.95\\
9775.6	0.40957	122.73	122.73\\
9973.9	0.39894	122.87	122.87\\
10173	0.3883	128.39	128.39\\
10378	0.37766	133.59	133.59\\
10624	0.36702	142.75	142.75\\
10863	0.35638	150.6	150.6\\
11083	0.34574	145.08	145.08\\
11390	0.33511	149.72	149.72\\
11621	0.32447	147.49	147.49\\
11887	0.31383	154.6	154.6\\
12143	0.30319	155.93	155.93\\
12426	0.29255	166.44	166.44\\
12689	0.28191	178.08	178.08\\
13007	0.27128	180.81	180.81\\
13315	0.26064	181.05	181.05\\
13699	0.25	191.07	191.07\\
14073	0.23936	183.77	183.77\\
14500	0.22872	188.27	188.27\\
14961	0.21809	198.08	198.08\\
15478	0.20745	236.2	236.2\\
16061	0.19681	222.1	222.1\\
16604	0.18617	235.41	235.41\\
17140	0.17553	260.77	260.77\\
17638	0.16489	294.49	294.49\\
18376	0.15426	304.14	304.14\\
19009	0.14362	331.39	331.39\\
20015	0.13298	344.51	344.51\\
21053	0.12234	391.85	391.85\\
22307	0.1117	433.47	433.47\\
23541	0.10106	465.71	465.71\\
25138	0.090426	519.35	519.35\\
26542	0.079787	530.6	530.6\\
28800	0.069149	660.25	660.25\\
31358	0.058511	794.32	794.32\\
34949	0.047872	1018.4	1018.4\\
39886	0.037234	1187.3	1187.3\\
46309	0.026596	1605	1605\\
57859	0.015957	2254	2254\\
86479	0.0053191	4199	4199\\
};
\addplot [color=mycolor3]
  table[row sep=crcr]{%
4415	0.97229\\
4573.2	0.94455\\
4737.1	0.90915\\
4906.8	0.87129\\
5648.8	0.727\\
11422	0.2931\\
23095	0.11856\\
48370	0.045986\\
1.0131e+05	0.017897\\
1.4406e+05	0.011431\\
};
\addlegendentry{Adatt. Pareto appr.}% ($\mathit{ML}$)
% \addlegendentry{Adatt. Pareto appr. ($\mathit{ML}$)}


\addplot[area legend, draw=none, fill=mycolor3, fill opacity=0.15, forget plot]
table[row sep=crcr] {%
x	y\\
4415	0.97971\\
4573.2	0.95407\\
4737.1	0.91967\\
4906.8	0.88194\\
5082.6	0.84311\\
5264.7	0.80569\\
5453.4	0.76947\\
5648.8	0.73488\\
5851.2	0.70187\\
6060.8	0.67035\\
6278	0.64026\\
6503	0.61154\\
6736	0.58412\\
6977.3	0.55795\\
7227.3	0.53297\\
7486.3	0.50912\\
7754.5	0.48636\\
8032.4	0.46464\\
8320.2	0.44391\\
8618.3	0.42413\\
8927.1	0.40525\\
9247	0.38723\\
9578.3	0.37004\\
9921.5	0.35363\\
10277	0.33797\\
10645	0.32302\\
11027	0.30875\\
11422	0.29512\\
11831	0.28211\\
12255	0.26968\\
12694	0.25781\\
13149	0.24647\\
13620	0.23564\\
14108	0.22529\\
14614	0.2154\\
15137	0.20595\\
15680	0.19692\\
16241	0.18829\\
16823	0.18004\\
17426	0.17215\\
18051	0.16461\\
18697	0.1574\\
19367	0.15052\\
20061	0.14393\\
20780	0.13763\\
21525	0.13162\\
22296	0.12586\\
23095	0.12036\\
23922	0.1151\\
24779	0.11007\\
25667	0.10526\\
26587	0.10067\\
27540	0.096272\\
28526	0.092069\\
29548	0.088051\\
30607	0.084208\\
31704	0.080534\\
32840	0.077021\\
34017	0.073662\\
35235	0.07045\\
36498	0.067378\\
37806	0.064441\\
39160	0.061632\\
40563	0.058947\\
42017	0.056378\\
43522	0.053922\\
45082	0.051574\\
46697	0.049328\\
48370	0.04718\\
50104	0.045126\\
51899	0.043161\\
53758	0.041283\\
55685	0.039487\\
57680	0.037769\\
59747	0.036126\\
61887	0.034554\\
64105	0.033052\\
66402	0.031615\\
68781	0.03024\\
71246	0.028926\\
73798	0.027669\\
76443	0.026466\\
79182	0.025316\\
82019	0.024216\\
84958	0.023165\\
88002	0.022159\\
91155	0.021196\\
94421	0.020276\\
97804	0.019396\\
1.0131e+05	0.018554\\
1.0494e+05	0.017749\\
1.087e+05	0.016979\\
1.1259e+05	0.016242\\
1.1663e+05	0.015538\\
1.2081e+05	0.014864\\
1.2514e+05	0.014219\\
1.2962e+05	0.013603\\
1.3426e+05	0.013013\\
1.3907e+05	0.012449\\
1.4406e+05	0.011909\\
1.4406e+05	0.010953\\
1.3907e+05	0.011461\\
1.3426e+05	0.011992\\
1.2962e+05	0.012548\\
1.2514e+05	0.013131\\
1.2081e+05	0.01374\\
1.1663e+05	0.014377\\
1.1259e+05	0.015045\\
1.087e+05	0.015743\\
1.0494e+05	0.016474\\
1.0131e+05	0.017239\\
97804	0.01804\\
94421	0.018878\\
91155	0.019755\\
88002	0.020672\\
84958	0.021633\\
82019	0.022639\\
79182	0.023691\\
76443	0.024793\\
73798	0.025946\\
71246	0.027153\\
68781	0.028416\\
66402	0.029738\\
64105	0.031122\\
61887	0.03257\\
59747	0.034086\\
57680	0.035673\\
55685	0.037334\\
53758	0.039073\\
51899	0.040893\\
50104	0.042798\\
48370	0.044792\\
46697	0.04688\\
45082	0.049065\\
43522	0.051352\\
42017	0.053747\\
40563	0.056253\\
39160	0.058877\\
37806	0.061623\\
36498	0.064499\\
35235	0.067508\\
34017	0.070659\\
32840	0.073957\\
31704	0.07741\\
30607	0.081024\\
29548	0.084808\\
28526	0.088769\\
27540	0.092915\\
26587	0.097256\\
25667	0.1018\\
24779	0.10656\\
23922	0.11154\\
23095	0.11675\\
22296	0.12221\\
21525	0.12792\\
20780	0.1339\\
20061	0.14016\\
19367	0.14672\\
18697	0.15358\\
18051	0.16076\\
17426	0.16828\\
16823	0.17615\\
16241	0.18439\\
15680	0.19301\\
15137	0.20204\\
14614	0.21148\\
14108	0.22137\\
13620	0.23171\\
13149	0.24254\\
12694	0.25387\\
12255	0.26572\\
11831	0.27812\\
11422	0.29109\\
11027	0.30466\\
10645	0.31884\\
10277	0.33367\\
9921.5	0.34918\\
9578.3	0.3654\\
9247	0.38235\\
8927.1	0.40007\\
8618.3	0.41859\\
8320.2	0.43796\\
8032.4	0.4582\\
7754.5	0.47936\\
7486.3	0.50149\\
7227.3	0.52463\\
6977.3	0.54882\\
6736	0.57412\\
6503	0.60058\\
6278	0.62825\\
6060.8	0.65719\\
5851.2	0.68745\\
5648.8	0.71911\\
5453.4	0.75222\\
5264.7	0.78686\\
5082.6	0.82299\\
4906.8	0.86065\\
4737.1	0.89863\\
4573.2	0.93502\\
4415	0.96486\\
}--cycle;
    
        %     \nextgroupplot[%
        %         % xmin=3637.5,xmax=204237,
        %         % ymin=0.00531914893617021,ymax=1,
        %     ] % This file was created by matlab2tikz.
%
\definecolor{mycolor1}{rgb}{0.00000,0.44706,0.69804}%
\definecolor{mycolor2}{rgb}{0.00000,0.44700,0.74100}%
\definecolor{mycolor3}{rgb}{0.49400,0.18400,0.55600}%
%
\addplot[only marks, mark=*, mark size=1.2500pt, color=mycolor1, fill=mycolor1, opacity=0.60, draw=none, mark options={draw=none,line width=0pt}] table[row sep=crcr]{%
x	y\\
4009.8	0.99468\\
4042.8	0.98404\\
4083.2	0.9734\\
4118.9	0.96277\\
4158.9	0.95213\\
4199.9	0.94149\\
4241.8	0.93085\\
4293.5	0.92021\\
4346.1	0.90957\\
4392.3	0.89894\\
4440.5	0.8883\\
4485	0.87766\\
4529	0.86702\\
4574.2	0.85638\\
4618.5	0.84574\\
4670.1	0.83511\\
4730.3	0.82447\\
4787.1	0.81383\\
4841	0.80319\\
4903.8	0.79255\\
4960.2	0.78191\\
5017.2	0.77128\\
5087.5	0.76064\\
5155.4	0.75\\
5233.1	0.73936\\
5291.8	0.72872\\
5360.5	0.71809\\
5431.2	0.70745\\
5501.1	0.69681\\
5574	0.68617\\
5661.2	0.67553\\
5733.2	0.66489\\
5815.2	0.65426\\
5904.3	0.64362\\
5987.8	0.63298\\
6075.5	0.62234\\
6160.2	0.6117\\
6238.2	0.60106\\
6339.4	0.59043\\
6443.4	0.57979\\
6550.4	0.56915\\
6655.6	0.55851\\
6768.6	0.54787\\
6863	0.53723\\
6984.1	0.5266\\
7105.1	0.51596\\
7243	0.50532\\
7363.9	0.49468\\
7529.2	0.48404\\
7674.5	0.4734\\
7798.1	0.46277\\
7951.1	0.45213\\
8117.5	0.44149\\
8265.8	0.43085\\
8464.1	0.42021\\
8676.5	0.40957\\
8871.5	0.39894\\
9049.7	0.3883\\
9256	0.37766\\
9441.2	0.36702\\
9666.5	0.35638\\
9881.2	0.34574\\
10150	0.33511\\
10431	0.32447\\
10718	0.31383\\
11062	0.30319\\
11293	0.29255\\
11688	0.28191\\
12013	0.27128\\
12316	0.26064\\
12705	0.25\\
13150	0.23936\\
13536	0.22872\\
13995	0.21809\\
14547	0.20745\\
15079	0.19681\\
15757	0.18617\\
16494	0.17553\\
17200	0.16489\\
17972	0.15426\\
18737	0.14362\\
19593	0.13298\\
20816	0.12234\\
22198	0.1117\\
23437	0.10106\\
25016	0.090426\\
27001	0.079787\\
29590	0.069149\\
32563	0.058511\\
36426	0.047872\\
43570	0.037234\\
54044	0.026596\\
68767	0.015957\\
1.4931e+05	0.0053191\\
};
\addlegendentry{FRC empirica esatta}

\addplot [color=mycolor1, only marks, every error bar/.append style={opacity=0.30}, mark=*, mark size=0pt, draw=none, forget plot]
 plot [error bars/.cd, x dir=both, x explicit, error bar style={line width=1pt, color=mycolor1}, error mark options={mark=none,mark size=0pt}]
 table[row sep=crcr, x error plus index=2, x error minus index=3]{%
4009.8	0.99468	41.683	41.683\\
4042.8	0.98404	42.729	42.729\\
4083.2	0.9734	42.64	42.64\\
4118.9	0.96277	42.259	42.259\\
4158.9	0.95213	42.401	42.401\\
4199.9	0.94149	43.247	43.247\\
4241.8	0.93085	44.025	44.025\\
4293.5	0.92021	45.653	45.653\\
4346.1	0.90957	47.232	47.232\\
4392.3	0.89894	47.443	47.443\\
4440.5	0.8883	47.604	47.604\\
4485	0.87766	47.648	47.648\\
4529	0.86702	48.056	48.056\\
4574.2	0.85638	48.712	48.712\\
4618.5	0.84574	48.794	48.794\\
4670.1	0.83511	49.967	49.967\\
4730.3	0.82447	48.825	48.825\\
4787.1	0.81383	49.432	49.432\\
4841	0.80319	50.68	50.68\\
4903.8	0.79255	49.872	49.872\\
4960.2	0.78191	50.875	50.875\\
5017.2	0.77128	51.172	51.172\\
5087.5	0.76064	52.783	52.783\\
5155.4	0.75	53.548	53.548\\
5233.1	0.73936	54.7	54.7\\
5291.8	0.72872	56.659	56.659\\
5360.5	0.71809	57.557	57.557\\
5431.2	0.70745	56.553	56.553\\
5501.1	0.69681	55.696	55.696\\
5574	0.68617	59.561	59.561\\
5661.2	0.67553	62.198	62.198\\
5733.2	0.66489	60.853	60.853\\
5815.2	0.65426	65.634	65.634\\
5904.3	0.64362	63.226	63.226\\
5987.8	0.63298	65.228	65.228\\
6075.5	0.62234	68.996	68.996\\
6160.2	0.6117	69.293	69.293\\
6238.2	0.60106	72.946	72.946\\
6339.4	0.59043	75.51	75.51\\
6443.4	0.57979	74.961	74.961\\
6550.4	0.56915	74.817	74.817\\
6655.6	0.55851	79.725	79.725\\
6768.6	0.54787	82.852	82.852\\
6863	0.53723	81.443	81.443\\
6984.1	0.5266	81.923	81.923\\
7105.1	0.51596	88.89	88.89\\
7243	0.50532	93.199	93.199\\
7363.9	0.49468	97.499	97.499\\
7529.2	0.48404	96.758	96.758\\
7674.5	0.4734	95.477	95.477\\
7798.1	0.46277	93.925	93.925\\
7951.1	0.45213	98.413	98.413\\
8117.5	0.44149	97.624	97.624\\
8265.8	0.43085	101.94	101.94\\
8464.1	0.42021	108.66	108.66\\
8676.5	0.40957	112.81	112.81\\
8871.5	0.39894	110.39	110.39\\
9049.7	0.3883	117.1	117.1\\
9256	0.37766	118.93	118.93\\
9441.2	0.36702	117.23	117.23\\
9666.5	0.35638	114.85	114.85\\
9881.2	0.34574	119.48	119.48\\
10150	0.33511	123.96	123.96\\
10431	0.32447	124.98	124.98\\
10718	0.31383	133.42	133.42\\
11062	0.30319	143.38	143.38\\
11293	0.29255	150.62	150.62\\
11688	0.28191	166.57	166.57\\
12013	0.27128	171.22	171.22\\
12316	0.26064	181.72	181.72\\
12705	0.25	192.57	192.57\\
13150	0.23936	208.87	208.87\\
13536	0.22872	200.05	200.05\\
13995	0.21809	213.59	213.59\\
14547	0.20745	224.95	224.95\\
15079	0.19681	224.48	224.48\\
15757	0.18617	254.54	254.54\\
16494	0.17553	273.77	273.77\\
17200	0.16489	282.21	282.21\\
17972	0.15426	309.96	309.96\\
18737	0.14362	340.73	340.73\\
19593	0.13298	351.38	351.38\\
20816	0.12234	385	385\\
22198	0.1117	402.42	402.42\\
23437	0.10106	424.8	424.8\\
25016	0.090426	456.58	456.58\\
27001	0.079787	560.02	560.02\\
29590	0.069149	636.25	636.25\\
32563	0.058511	797.84	797.84\\
36426	0.047872	902.21	902.21\\
43570	0.037234	1186.4	1186.4\\
54044	0.026596	1844.3	1844.3\\
68767	0.015957	2565.5	2565.5\\
1.4931e+05	0.0053191	4008.6	4008.6\\
};
\addplot [color=mycolor1]
  table[row sep=crcr]{%
3997.8	0.97494\\
4159.8	0.94331\\
4328.4	0.90397\\
4686.4	0.81967\\
10374	0.30231\\
22965	0.1119\\
52898	0.039557\\
1.2185e+05	0.014037\\
2.0424e+05	0.0074052\\
};
\addlegendentry{Adatt. Pareto esatto ($\mathit{ML}$)}


\addplot[area legend, draw=none, fill=mycolor1, fill opacity=0.15, forget plot]
table[row sep=crcr] {%
x	y\\
3997.8	0.98138\\
4159.8	0.95214\\
4328.4	0.91395\\
4503.8	0.87112\\
4686.4	0.82863\\
4876.4	0.78787\\
5074	0.74913\\
5279.7	0.7123\\
5493.7	0.67731\\
5716.3	0.64405\\
5948	0.61244\\
6189.1	0.58241\\
6440	0.55387\\
6701	0.52675\\
6972.6	0.50098\\
7255.3	0.4765\\
7549.3	0.45324\\
7855.3	0.43114\\
8173.7	0.41014\\
8505	0.39019\\
8849.8	0.37124\\
9208.5	0.35323\\
9581.7	0.33612\\
9970.1	0.31985\\
10374	0.30439\\
10795	0.28969\\
11232	0.27571\\
11687	0.26242\\
12161	0.24977\\
12654	0.23775\\
13167	0.22631\\
13701	0.21542\\
14256	0.20507\\
14834	0.19522\\
15435	0.18584\\
16061	0.17692\\
16712	0.16843\\
17389	0.16034\\
18094	0.15265\\
18827	0.14533\\
19591	0.13837\\
20385	0.13173\\
21211	0.12542\\
22071	0.11941\\
22965	0.1137\\
23896	0.10825\\
24865	0.10307\\
25872	0.098136\\
26921	0.09344\\
28012	0.088969\\
29148	0.084713\\
30329	0.080662\\
31558	0.076805\\
32838	0.073134\\
34169	0.069638\\
35553	0.06631\\
36995	0.063142\\
38494	0.060125\\
40054	0.057254\\
41678	0.05452\\
43367	0.051917\\
45125	0.049438\\
46954	0.047079\\
48857	0.044832\\
50837	0.042693\\
52898	0.040656\\
55042	0.038717\\
57273	0.036871\\
59595	0.035113\\
62010	0.033439\\
64524	0.031845\\
67139	0.030327\\
69860	0.028882\\
72692	0.027506\\
75638	0.026196\\
78704	0.024948\\
81894	0.02376\\
85213	0.022629\\
88667	0.021552\\
92261	0.020526\\
96001	0.019549\\
99892	0.018619\\
1.0394e+05	0.017734\\
1.0815e+05	0.01689\\
1.1254e+05	0.016087\\
1.171e+05	0.015322\\
1.2185e+05	0.014594\\
1.2678e+05	0.0139\\
1.3192e+05	0.013239\\
1.3727e+05	0.01261\\
1.4283e+05	0.012011\\
1.4862e+05	0.01144\\
1.5465e+05	0.010897\\
1.6092e+05	0.01038\\
1.6744e+05	0.0098868\\
1.7423e+05	0.0094174\\
1.8129e+05	0.0089704\\
1.8864e+05	0.0085447\\
1.9628e+05	0.0081392\\
2.0424e+05	0.0077531\\
2.0424e+05	0.0070573\\
1.9628e+05	0.0074172\\
1.8864e+05	0.0077955\\
1.8129e+05	0.0081931\\
1.7423e+05	0.0086111\\
1.6744e+05	0.0090505\\
1.6092e+05	0.0095124\\
1.5465e+05	0.0099979\\
1.4862e+05	0.010508\\
1.4283e+05	0.011045\\
1.3727e+05	0.011609\\
1.3192e+05	0.012202\\
1.2678e+05	0.012825\\
1.2185e+05	0.013481\\
1.171e+05	0.014169\\
1.1254e+05	0.014894\\
1.0815e+05	0.015655\\
1.0394e+05	0.016455\\
99892	0.017297\\
96001	0.018182\\
92261	0.019112\\
88667	0.020089\\
85213	0.021117\\
81894	0.022198\\
78704	0.023334\\
75638	0.024529\\
72692	0.025785\\
69860	0.027106\\
67139	0.028494\\
64524	0.029954\\
62010	0.031489\\
59595	0.033102\\
57273	0.034799\\
55042	0.036583\\
52898	0.038459\\
50837	0.040431\\
48857	0.042505\\
46954	0.044685\\
45125	0.046978\\
43367	0.049388\\
41678	0.051923\\
40054	0.054588\\
38494	0.057391\\
36995	0.060338\\
35553	0.063436\\
34169	0.066694\\
32838	0.070121\\
31558	0.073723\\
30329	0.077512\\
29148	0.081495\\
28012	0.085684\\
26921	0.090089\\
25872	0.094721\\
24865	0.099591\\
23896	0.10471\\
22965	0.1101\\
22071	0.11576\\
21211	0.12172\\
20385	0.12798\\
19591	0.13457\\
18827	0.14149\\
18094	0.14877\\
17389	0.15643\\
16712	0.16448\\
16061	0.17294\\
15435	0.18185\\
14834	0.1912\\
14256	0.20104\\
13701	0.21139\\
13167	0.22227\\
12654	0.2337\\
12161	0.24572\\
11687	0.25835\\
11232	0.27163\\
10795	0.28558\\
10374	0.30024\\
9970.1	0.31564\\
9581.7	0.33182\\
9208.5	0.34882\\
8849.8	0.36666\\
8505	0.38541\\
8173.7	0.40509\\
7855.3	0.42575\\
7549.3	0.44745\\
7255.3	0.47023\\
6972.6	0.49416\\
6701	0.51928\\
6440	0.54566\\
6189.1	0.57338\\
5948	0.60248\\
5716.3	0.63305\\
5493.7	0.66517\\
5279.7	0.69891\\
5074	0.73435\\
4876.4	0.77158\\
4686.4	0.8107\\
4503.8	0.8517\\
4328.4	0.89399\\
4159.8	0.93448\\
3997.8	0.96849\\
}--cycle;
\addplot[only marks, mark=*, mark size=1.2500pt, color=black, fill=black, opacity=0.60, draw=none, mark options={draw=none,line width=0pt}] table[row sep=crcr]{%
x	y\\
3650	0.99468\\
3735	0.98404\\
3741	0.9734\\
3761	0.96277\\
3772	0.95213\\
3785	0.94149\\
3825	0.93085\\
3847	0.92021\\
3900	0.90957\\
3928	0.89894\\
3945	0.8883\\
3996	0.87766\\
4009	0.86702\\
4024	0.85638\\
4061	0.84574\\
4061	0.83511\\
4071	0.82447\\
4113	0.81383\\
4228	0.80319\\
4348	0.79255\\
4430	0.78191\\
4503	0.77128\\
4539	0.76064\\
4632	0.75\\
4654	0.73936\\
4663	0.72872\\
4679	0.71809\\
4702	0.70745\\
4779	0.69681\\
4861	0.68617\\
4953	0.67553\\
5029	0.66489\\
5048	0.65426\\
5236	0.64362\\
5264	0.63298\\
5327	0.62234\\
5379	0.6117\\
5458	0.60106\\
5463	0.59043\\
5524	0.57979\\
5607	0.56915\\
5857	0.55851\\
5868	0.54787\\
5982	0.53723\\
6317	0.5266\\
6332	0.51596\\
6342	0.50532\\
6356	0.49468\\
6367	0.48404\\
6468	0.4734\\
6629	0.46277\\
6926	0.45213\\
7252	0.44149\\
7294	0.43085\\
7320	0.42021\\
7348	0.40957\\
7596	0.39894\\
7724	0.3883\\
7877	0.37766\\
8035	0.36702\\
8499	0.35638\\
8518	0.34574\\
8994	0.33511\\
9128	0.32447\\
9267	0.31383\\
9435	0.30319\\
9599	0.29255\\
9837	0.28191\\
10119	0.27128\\
10336	0.26064\\
10377	0.25\\
11048	0.23936\\
11424	0.22872\\
11830	0.21809\\
12182	0.20745\\
12313	0.19681\\
13086	0.18617\\
13380	0.17553\\
13398	0.16489\\
13899	0.15426\\
14984	0.14362\\
16428	0.13298\\
20491	0.12234\\
21264	0.1117\\
23237	0.10106\\
30134	0.090426\\
30990	0.079787\\
32887	0.069149\\
37527	0.058511\\
39026	0.047872\\
41095	0.037234\\
61636	0.026596\\
1.2234e+05	0.015957\\
2.0424e+05	0.0053191\\
};
\addlegendentry{FRC empirica esatta}

\addplot [color=black]
  table[row sep=crcr]{%
3637.5	1\\
2.0424e+05	0.005939\\
};
\addlegendentry{Adatt. Pareto esatto ($\mathit{ML}$)}

    
        %     \nextgroupplot[%
        %         % xmin=3637.5,xmax=204237,
        %         % ymin=0.00531914893617021,ymax=1
        %     ] % This file was created by matlab2tikz.
%
\definecolor{mycolor1}{rgb}{0.83529,0.36863,0.00000}%
\definecolor{mycolor2}{rgb}{0.00000,0.44700,0.74100}%
\definecolor{mycolor3}{rgb}{0.49400,0.18400,0.55600}%
%
\addplot[only marks, mark=*, mark size=1.2500pt, color=mycolor1, fill=mycolor1, opacity=0.60, draw=none, mark options={draw=none,line width=0pt}] table[row sep=crcr]{%
x	y\\
4026.3	0.99468\\
4058.6	0.98404\\
4099.8	0.9734\\
4144	0.96277\\
4185.7	0.95213\\
4222.6	0.94149\\
4258.8	0.93085\\
4294	0.92021\\
4338.8	0.90957\\
4387.5	0.89894\\
4434.4	0.8883\\
4477.8	0.87766\\
4531.1	0.86702\\
4583.2	0.85638\\
4627.9	0.84574\\
4689.2	0.83511\\
4733	0.82447\\
4787	0.81383\\
4843.5	0.80319\\
4896.6	0.79255\\
4951.8	0.78191\\
5010.6	0.77128\\
5072.2	0.76064\\
5133.1	0.75\\
5190.8	0.73936\\
5257.5	0.72872\\
5330.8	0.71809\\
5395.5	0.70745\\
5462.9	0.69681\\
5544.5	0.68617\\
5620.8	0.67553\\
5702.2	0.66489\\
5790.5	0.65426\\
5881.8	0.64362\\
5964	0.63298\\
6028.3	0.62234\\
6133.8	0.6117\\
6237	0.60106\\
6320	0.59043\\
6406.8	0.57979\\
6506.9	0.56915\\
6616.8	0.55851\\
6722.3	0.54787\\
6828.5	0.53723\\
6953.2	0.5266\\
7088.7	0.51596\\
7213.5	0.50532\\
7325.9	0.49468\\
7446.2	0.48404\\
7581.1	0.4734\\
7724.5	0.46277\\
7861.5	0.45213\\
8007	0.44149\\
8164.3	0.43085\\
8339.4	0.42021\\
8505.8	0.40957\\
8697.7	0.39894\\
8854.2	0.3883\\
9063	0.37766\\
9293.7	0.36702\\
9542.3	0.35638\\
9795.1	0.34574\\
10050	0.33511\\
10298	0.32447\\
10572	0.31383\\
10877	0.30319\\
11188	0.29255\\
11456	0.28191\\
11809	0.27128\\
12134	0.26064\\
12539	0.25\\
12911	0.23936\\
13427	0.22872\\
13974	0.21809\\
14460	0.20745\\
15013	0.19681\\
15718	0.18617\\
16449	0.17553\\
17281	0.16489\\
18072	0.15426\\
18837	0.14362\\
19847	0.13298\\
20821	0.12234\\
22193	0.1117\\
23455	0.10106\\
24972	0.090426\\
26751	0.079787\\
29026	0.069149\\
31699	0.058511\\
35825	0.047872\\
42313	0.037234\\
53000	0.026596\\
72389	0.015957\\
1.5403e+05	0.0053191\\
};
\addlegendentry{FRC empirica appr.}

\addplot [color=mycolor1, only marks, every error bar/.append style={opacity=0.30}, mark=*, mark size=0pt, draw=none, forget plot]
 plot [error bars/.cd, x dir=both, x explicit, error bar style={line width=1pt, color=mycolor1}, error mark options={mark=none,mark size=0pt}]
 table[row sep=crcr, x error plus index=2, x error minus index=3]{%
4026.3	0.99468	36.588	36.588\\
4058.6	0.98404	35.959	35.959\\
4099.8	0.9734	36.838	36.838\\
4144	0.96277	37.465	37.465\\
4185.7	0.95213	38.893	38.893\\
4222.6	0.94149	39.784	39.784\\
4258.8	0.93085	40.702	40.702\\
4294	0.92021	40.402	40.402\\
4338.8	0.90957	39.037	39.037\\
4387.5	0.89894	40.059	40.059\\
4434.4	0.8883	39.704	39.704\\
4477.8	0.87766	40.479	40.479\\
4531.1	0.86702	41.553	41.553\\
4583.2	0.85638	42.897	42.897\\
4627.9	0.84574	42.524	42.524\\
4689.2	0.83511	46.655	46.655\\
4733	0.82447	48.383	48.383\\
4787	0.81383	48.919	48.919\\
4843.5	0.80319	48.945	48.945\\
4896.6	0.79255	49.791	49.791\\
4951.8	0.78191	49.148	49.148\\
5010.6	0.77128	47.503	47.503\\
5072.2	0.76064	48.508	48.508\\
5133.1	0.75	50.07	50.07\\
5190.8	0.73936	49.206	49.206\\
5257.5	0.72872	50.721	50.721\\
5330.8	0.71809	50.264	50.264\\
5395.5	0.70745	50.129	50.129\\
5462.9	0.69681	50.129	50.129\\
5544.5	0.68617	52.551	52.551\\
5620.8	0.67553	53.404	53.404\\
5702.2	0.66489	54.765	54.765\\
5790.5	0.65426	55.651	55.651\\
5881.8	0.64362	58.065	58.065\\
5964	0.63298	61.673	61.673\\
6028.3	0.62234	61.598	61.598\\
6133.8	0.6117	62.325	62.325\\
6237	0.60106	65.704	65.704\\
6320	0.59043	68.439	68.439\\
6406.8	0.57979	67.477	67.477\\
6506.9	0.56915	74.772	74.772\\
6616.8	0.55851	75.741	75.741\\
6722.3	0.54787	78.334	78.334\\
6828.5	0.53723	80.276	80.276\\
6953.2	0.5266	85.433	85.433\\
7088.7	0.51596	85.646	85.646\\
7213.5	0.50532	87.296	87.296\\
7325.9	0.49468	90.363	90.363\\
7446.2	0.48404	89.834	89.834\\
7581.1	0.4734	87.181	87.181\\
7724.5	0.46277	85.579	85.579\\
7861.5	0.45213	91.022	91.022\\
8007	0.44149	89.981	89.981\\
8164.3	0.43085	95.299	95.299\\
8339.4	0.42021	90.558	90.558\\
8505.8	0.40957	97.967	97.967\\
8697.7	0.39894	97.569	97.569\\
8854.2	0.3883	97.36	97.36\\
9063	0.37766	101.82	101.82\\
9293.7	0.36702	108.12	108.12\\
9542.3	0.35638	116.55	116.55\\
9795.1	0.34574	116.34	116.34\\
10050	0.33511	122.98	122.98\\
10298	0.32447	127.97	127.97\\
10572	0.31383	139.11	139.11\\
10877	0.30319	147.36	147.36\\
11188	0.29255	147.95	147.95\\
11456	0.28191	158.48	158.48\\
11809	0.27128	170.86	170.86\\
12134	0.26064	180.39	180.39\\
12539	0.25	194.84	194.84\\
12911	0.23936	200.17	200.17\\
13427	0.22872	209.38	209.38\\
13974	0.21809	224.5	224.5\\
14460	0.20745	234.84	234.84\\
15013	0.19681	233.15	233.15\\
15718	0.18617	258.63	258.63\\
16449	0.17553	271.52	271.52\\
17281	0.16489	294.36	294.36\\
18072	0.15426	345.21	345.21\\
18837	0.14362	336.44	336.44\\
19847	0.13298	382.5	382.5\\
20821	0.12234	385.99	385.99\\
22193	0.1117	410.26	410.26\\
23455	0.10106	437.73	437.73\\
24972	0.090426	505.73	505.73\\
26751	0.079787	587.47	587.47\\
29026	0.069149	684.85	684.85\\
31699	0.058511	813.07	813.07\\
35825	0.047872	967.25	967.25\\
42313	0.037234	1232	1232\\
53000	0.026596	1956	1956\\
72389	0.015957	2969.7	2969.7\\
1.5403e+05	0.0053191	5124.6	5124.6\\
};
\addplot [color=mycolor1]
  table[row sep=crcr]{%
4007.8	0.97865\\
4178.6	0.94382\\
4356.8	0.90082\\
4938.1	0.76874\\
11867	0.25265\\
29734	0.079066\\
74501	0.024844\\
1.8667e+05	0.0078379\\
2.5003e+05	0.0054343\\
};
\addlegendentry{Adatt. Pareto appr.}% ($\mathit{ML}$)
% \addlegendentry{Adatt. Pareto appr. ($\mathit{ML}$)}


\addplot[area legend, draw=none, fill=mycolor1, fill opacity=0.15, forget plot]
table[row sep=crcr] {%
x	y\\
4007.8	0.98443\\
4178.6	0.95203\\
4356.8	0.91008\\
4542.5	0.86353\\
4736.2	0.8185\\
4938.1	0.77583\\
5148.7	0.73541\\
5368.2	0.69711\\
5597.1	0.66082\\
5835.7	0.62645\\
6084.5	0.59388\\
6343.9	0.56303\\
6614.4	0.5338\\
6896.4	0.50612\\
7190.4	0.4799\\
7497	0.45506\\
7816.6	0.43154\\
8149.9	0.40926\\
8497.3	0.38816\\
8859.6	0.36816\\
9237.4	0.34922\\
9631.2	0.33128\\
10042	0.31427\\
10470	0.29815\\
10916	0.28287\\
11382	0.26838\\
11867	0.25465\\
12373	0.24163\\
12900	0.22927\\
13450	0.21756\\
14024	0.20645\\
14622	0.19591\\
15245	0.18591\\
15895	0.17643\\
16573	0.16743\\
17280	0.15889\\
18016	0.1508\\
18784	0.14311\\
19585	0.13582\\
20420	0.1289\\
21291	0.12234\\
22199	0.11611\\
23145	0.1102\\
24132	0.10459\\
25161	0.099274\\
26233	0.094224\\
27352	0.089432\\
28518	0.084885\\
29734	0.080569\\
31002	0.076474\\
32323	0.072588\\
33701	0.0689\\
35138	0.0654\\
36636	0.062078\\
38198	0.058925\\
39827	0.055933\\
41525	0.053094\\
43295	0.050399\\
45141	0.047841\\
47066	0.045414\\
49072	0.04311\\
51165	0.040923\\
53346	0.038848\\
55621	0.036878\\
57992	0.035008\\
60464	0.033233\\
63042	0.031549\\
65730	0.02995\\
68532	0.028433\\
71454	0.026992\\
74501	0.025625\\
77677	0.024327\\
80989	0.023096\\
84442	0.021926\\
88042	0.020816\\
91796	0.019763\\
95709	0.018763\\
99790	0.017814\\
1.0404e+05	0.016912\\
1.0848e+05	0.016057\\
1.1311e+05	0.015245\\
1.1793e+05	0.014474\\
1.2296e+05	0.013742\\
1.282e+05	0.013048\\
1.3366e+05	0.012388\\
1.3936e+05	0.011762\\
1.453e+05	0.011168\\
1.515e+05	0.010604\\
1.5796e+05	0.010068\\
1.6469e+05	0.0095597\\
1.7171e+05	0.009077\\
1.7904e+05	0.0086187\\
1.8667e+05	0.0081837\\
1.9463e+05	0.0077706\\
2.0292e+05	0.0073785\\
2.1158e+05	0.0070062\\
2.206e+05	0.0066528\\
2.3e+05	0.0063172\\
2.3981e+05	0.0059986\\
2.5003e+05	0.0056961\\
2.5003e+05	0.0051726\\
2.3981e+05	0.0054536\\
2.3e+05	0.0057499\\
2.206e+05	0.0060624\\
2.1158e+05	0.0063919\\
2.0292e+05	0.0067393\\
1.9463e+05	0.0071057\\
1.8667e+05	0.0074921\\
1.7904e+05	0.0078996\\
1.7171e+05	0.0083292\\
1.6469e+05	0.0087824\\
1.5796e+05	0.0092602\\
1.515e+05	0.0097641\\
1.453e+05	0.010295\\
1.3936e+05	0.010856\\
1.3366e+05	0.011447\\
1.282e+05	0.01207\\
1.2296e+05	0.012727\\
1.1793e+05	0.013421\\
1.1311e+05	0.014152\\
1.0848e+05	0.014923\\
1.0404e+05	0.015736\\
99790	0.016593\\
95709	0.017498\\
91796	0.018452\\
88042	0.019458\\
84442	0.020519\\
80989	0.021638\\
77677	0.022818\\
74501	0.024063\\
71454	0.025376\\
68532	0.02676\\
65730	0.028221\\
63042	0.029761\\
60464	0.031386\\
57992	0.033099\\
55621	0.034907\\
53346	0.036813\\
51165	0.038824\\
49072	0.040945\\
47066	0.043183\\
45141	0.045542\\
43295	0.048031\\
41525	0.050657\\
39827	0.053426\\
38198	0.056347\\
36636	0.059429\\
35138	0.062679\\
33701	0.066107\\
32323	0.069724\\
31002	0.073539\\
29734	0.077563\\
28518	0.081807\\
27352	0.086285\\
26233	0.091008\\
25161	0.09599\\
24132	0.10125\\
23145	0.10679\\
22199	0.11264\\
21291	0.11881\\
20420	0.12531\\
19585	0.13218\\
18784	0.13942\\
18016	0.14706\\
17280	0.15511\\
16573	0.16361\\
15895	0.17257\\
15245	0.18203\\
14622	0.192\\
14024	0.20252\\
13450	0.21361\\
12900	0.22531\\
12373	0.23765\\
11867	0.25066\\
11382	0.26437\\
10916	0.27884\\
10470	0.29408\\
10042	0.31015\\
9631.2	0.32709\\
9237.4	0.34494\\
8859.6	0.36375\\
8497.3	0.38357\\
8149.9	0.40444\\
7816.6	0.42643\\
7497	0.4496\\
7190.4	0.474\\
6896.4	0.49971\\
6614.4	0.52678\\
6343.9	0.55531\\
6084.5	0.58537\\
5835.7	0.61703\\
5597.1	0.6504\\
5368.2	0.68556\\
5148.7	0.72261\\
4938.1	0.76166\\
4736.2	0.80281\\
4542.5	0.84618\\
4356.8	0.89157\\
4178.6	0.93561\\
4007.8	0.97287\\
}--cycle;
\addplot[only marks, mark=*, mark size=1.2500pt, color=black, fill=black, opacity=0.60, draw=none, mark options={draw=none,line width=0pt}] table[row sep=crcr]{%
x	y\\
3650	0.99468\\
3735	0.98404\\
3741	0.9734\\
3761	0.96277\\
3772	0.95213\\
3785	0.94149\\
3825	0.93085\\
3847	0.92021\\
3900	0.90957\\
3928	0.89894\\
3945	0.8883\\
3996	0.87766\\
4009	0.86702\\
4024	0.85638\\
4061	0.84574\\
4061	0.83511\\
4071	0.82447\\
4113	0.81383\\
4228	0.80319\\
4348	0.79255\\
4430	0.78191\\
4503	0.77128\\
4539	0.76064\\
4632	0.75\\
4654	0.73936\\
4663	0.72872\\
4679	0.71809\\
4702	0.70745\\
4779	0.69681\\
4861	0.68617\\
4953	0.67553\\
5029	0.66489\\
5048	0.65426\\
5236	0.64362\\
5264	0.63298\\
5327	0.62234\\
5379	0.6117\\
5458	0.60106\\
5463	0.59043\\
5524	0.57979\\
5607	0.56915\\
5857	0.55851\\
5868	0.54787\\
5982	0.53723\\
6317	0.5266\\
6332	0.51596\\
6342	0.50532\\
6356	0.49468\\
6367	0.48404\\
6468	0.4734\\
6629	0.46277\\
6926	0.45213\\
7252	0.44149\\
7294	0.43085\\
7320	0.42021\\
7348	0.40957\\
7596	0.39894\\
7724	0.3883\\
7877	0.37766\\
8035	0.36702\\
8499	0.35638\\
8518	0.34574\\
8994	0.33511\\
9128	0.32447\\
9267	0.31383\\
9435	0.30319\\
9599	0.29255\\
9837	0.28191\\
10119	0.27128\\
10336	0.26064\\
10377	0.25\\
11048	0.23936\\
11424	0.22872\\
11830	0.21809\\
12182	0.20745\\
12313	0.19681\\
13086	0.18617\\
13380	0.17553\\
13398	0.16489\\
13899	0.15426\\
14984	0.14362\\
16428	0.13298\\
20491	0.12234\\
21264	0.1117\\
23237	0.10106\\
30134	0.090426\\
30990	0.079787\\
32887	0.069149\\
37527	0.058511\\
39026	0.047872\\
41095	0.037234\\
61636	0.026596\\
1.2234e+05	0.015957\\
2.0424e+05	0.0053191\\
};
\addlegendentry{FRC empirica reale}

\addplot [color=black]
  table[row sep=crcr]{%
3637.5	1\\
2.5003e+05	0.0045909\\
};
\addlegendentry{Adatt. Pareto reale}% ($\mathit{ML}$)
% \addlegendentry{Adatt. Pareto reale ($\mathit{ML}$)}

        % \end{groupplot}
    
        % % Right column
        % \begin{groupplot}[
        %     group style={
        %         group name=ColR1,
        %         plotRightGroup
        %     },
        %     plotRightCol,
        %     plotTop,
        %     plotLegendNW,
        %     %
        %     xlabel={\(k\)},ylabel={\(s(k)\)},
        %     xmin=10,xmax=283,
        % ]
        %     \nextgroupplot[%
        %         at={($(ColL1 c1r1.south east)+(\xPlotSep em,0)$)},
        %         xmode=log,ymode=log,
        %         % xmin=10,xmax=283,
        %         % ymin=100,ymax=149878.932518743,
        %     ] % This file was created by matlab2tikz.
%
\definecolor{mycolor1}{rgb}{0.00000,0.44706,0.69804}%
\definecolor{mycolor2}{rgb}{0.00000,0.44700,0.74100}%
\definecolor{mycolor3}{rgb}{0.83529,0.36863,0.00000}%
\definecolor{mycolor4}{rgb}{0.85000,0.32500,0.09800}%
%
\addplot[only marks, mark=*, mark size=1.2748pt, color=mycolor1, fill=mycolor1, opacity=0.70, draw=none, mark options={draw=none,line width=0pt}] table[row sep=crcr]{%
x	y\\
34	2204\\
29	1419.2\\
96	13522\\
31	1700.6\\
27	1321.7\\
70	7005.4\\
24	1130.8\\
37	2255.1\\
42	3197\\
16	494.22\\
29	1550.6\\
59	5347.6\\
69	7665.8\\
28	1380.2\\
11	251.04\\
20	826.08\\
45	3483.9\\
27	1356.1\\
22	896.32\\
27	1343.6\\
44	3154.9\\
16	416.75\\
39	2322.3\\
19	624.81\\
33	1754.9\\
37	2247.3\\
33	1736.9\\
18	698.43\\
29	1501.8\\
22	938.97\\
33	1900.6\\
22	820.12\\
49	3829.8\\
25	1137.7\\
43	2759.8\\
26	1142.4\\
31	1697.6\\
26	1234.6\\
18	612.08\\
13	363.43\\
32	1739.9\\
38	2481.6\\
32	1621.8\\
29	1528.5\\
35	2216.9\\
35	2056.4\\
135	22776\\
27	1169.5\\
42	2986.3\\
39	2542.7\\
51	4544.7\\
86	10360\\
15	467.95\\
39	2461.5\\
42	2697.7\\
43	3066.5\\
58	4720.8\\
97	12588\\
50	3905.6\\
20	734.78\\
28	1468.7\\
22	900.81\\
47	3358.1\\
154	28590\\
30	1629.8\\
10	201.96\\
37	2315.8\\
31	1764.1\\
54	4527\\
76	8282\\
55	4833\\
26	1175\\
33	1932.5\\
34	1895.7\\
25	1011.6\\
31	1534.5\\
34	1813.7\\
26	1337.6\\
39	2574.3\\
27	1258.7\\
32	1732.5\\
26	1099.6\\
29	1286\\
23	883.83\\
26	1343.9\\
14	417.2\\
33	1913.8\\
24	1009.8\\
20	804.89\\
51	4413.8\\
35	2120.5\\
29	1456\\
46	3451.1\\
44	3127.2\\
45	3174.9\\
29	1499.8\\
28	1370.2\\
77	8511.8\\
70	7002.3\\
70	7370.1\\
40	2639.4\\
90	11114\\
53	4510.9\\
62	6255.4\\
47	3579.9\\
34	1798.9\\
14	414.32\\
47	4092.3\\
43	2858.7\\
20	739.45\\
28	1395.1\\
80	9350.5\\
25	1077\\
36	1979.8\\
27	1223\\
82	9235.5\\
31	1524.3\\
39	2356.7\\
23	973.44\\
27	1260.4\\
37	2235.3\\
129	22339\\
63	5688.1\\
64	6257.3\\
91	11253\\
27	1190.2\\
26	1180.4\\
17	513.08\\
33	1938.7\\
30	1439.1\\
33	1680.6\\
182	39382\\
26	1092.2\\
48	3546.8\\
46	3210.5\\
10	236.63\\
25	1130.8\\
26	1097.1\\
206	50199\\
37	2271\\
43	3111.3\\
51	3929.2\\
57	5458.4\\
36	1971.6\\
38	2460.5\\
28	1316.5\\
19	602.37\\
36	2189.7\\
45	3167.3\\
53	4265.1\\
48	3379.8\\
64	5260.2\\
65	5828.1\\
45	3153\\
54	4367.4\\
16	531.23\\
24	1004.1\\
127	20709\\
48	3775.8\\
64	6498.4\\
42	2714.7\\
42	2934.6\\
17	615.17\\
48	3830.5\\
42	2739.2\\
20	798.7\\
34	1906.7\\
47	3562.6\\
42	2781.9\\
50	4176.2\\
87	11700\\
64	6034.2\\
32	1867.8\\
39	2505.3\\
35	2189.5\\
18	589.6\\
28	1228.2\\
27	1220.2\\
37	2366.2\\
30	1620.1\\
89	10471\\
22	916.34\\
28	1356\\
25	1016.5\\
15	431.33\\
41	3018.9\\
32	1619\\
34	2049\\
16	483.33\\
88	10508\\
39	2483.4\\
140	23991\\
39	2390\\
49	3731.4\\
72	7646\\
37	2322.2\\
41	2913.5\\
283	80140\\
31	1640.8\\
99	13329\\
103	14584\\
39	2592.4\\
35	2037.3\\
98	12986\\
45	3170.6\\
73	7099.8\\
34	1911.3\\
56	4827.9\\
39	2280.5\\
37	2148.8\\
55	4621.3\\
25	1189.4\\
29	1267.2\\
42	3041.6\\
35	2123.9\\
26	1198\\
45	3467.1\\
56	5112.6\\
39	2671.2\\
62	6339.9\\
94	11820\\
112	18004\\
25	954.26\\
57	4707.1\\
50	4079.6\\
47	3385.9\\
96	13325\\
54	4694.2\\
28	1475.3\\
38	2312.8\\
47	3642.3\\
35	2133.5\\
45	3270.8\\
50	4296.1\\
43	2851.4\\
29	1359.2\\
42	2928.4\\
71	7221.5\\
68	6679.9\\
149	26699\\
67	6692.6\\
51	4006.8\\
34	2098.3\\
120	19892\\
43	2973.8\\
117	19956\\
39	2370.6\\
72	7598.6\\
40	2675.5\\
43	3050.8\\
32	1717.2\\
49	4366.9\\
41	2715.3\\
98	13157\\
96	12755\\
45	3325.8\\
129	21231\\
43	2743\\
93	11214\\
41	2752.9\\
88	10775\\
73	7742.3\\
108	15642\\
52	3938.6\\
18	638.71\\
36	2315.5\\
50	4083.8\\
37	2234.8\\
83	9856.5\\
40	2577.9\\
26	1366\\
40	2448.3\\
49	3565.3\\
27	1199.4\\
41	2650\\
36	2217.7\\
52	4125.4\\
37	1998.5\\
68	6450.2\\
43	2786.6\\
132	22326\\
73	7705.9\\
46	3505.5\\
47	3546.1\\
44	2703.3\\
41	3139.2\\
30	1619.1\\
25	1128.5\\
24	1001.4\\
85	9409.3\\
44	3072.3\\
29	1408.1\\
34	1983\\
101	13874\\
18	608.97\\
21	861.32\\
111	17246\\
95	12600\\
23	1060.4\\
31	1536.5\\
84	10151\\
23	1014.6\\
55	5079.2\\
32	1711.9\\
27	1208.7\\
48	3429\\
18	569.1\\
28	1299.4\\
40	2830.5\\
30	1621.1\\
13	303.77\\
33	1772.6\\
14	470.46\\
52	4267.8\\
65	6344.5\\
49	3912.1\\
35	2308.6\\
109	15767\\
26	993.47\\
41	2571.4\\
39	2735.8\\
61	6253.4\\
40	2329.2\\
36	2197.9\\
21	780.94\\
80	9303.2\\
36	2201.7\\
36	2231.3\\
38	2683.5\\
42	2915.2\\
27	1208.2\\
46	3370.7\\
23	991.23\\
34	2086\\
200	44668\\
37	2388.7\\
16	461.99\\
50	3705.7\\
19	621.95\\
41	2513.9\\
25	1120.5\\
39	2604.5\\
45	2968.5\\
83	9881.1\\
27	1268.4\\
67	7175\\
46	3112.4\\
35	2143\\
44	3300.2\\
30	1615.9\\
29	1402.8\\
23	932.95\\
43	2922.5\\
28	1323.6\\
24	1061\\
56	5089.4\\
33	1872.8\\
13	305.09\\
41	2935.3\\
34	1925.6\\
14	421.81\\
83	9860.6\\
33	1650.9\\
35	2109\\
29	1494.5\\
54	3824.1\\
41	2736\\
19	696.87\\
34	1766.9\\
19	658.17\\
38	2509\\
43	2900.4\\
14	434.27\\
17	538.13\\
12	275.34\\
};
\addlegendentry{Dispersione esatta}

\addplot[only marks, mark=*, mark size=1.2748pt, color=mycolor3, fill=mycolor3, opacity=0.70, draw=none, mark options={draw=none,line width=0pt}] table[row sep=crcr]{%
x	y\\
34	2135.5\\
29	1342.2\\
96	12881\\
31	1551.9\\
27	1296.3\\
70	7487.3\\
24	1072\\
37	2707.5\\
42	2846.8\\
16	431.79\\
29	1471.7\\
59	5225.5\\
69	6778.8\\
28	1313\\
11	250.13\\
20	779.89\\
45	3310.4\\
27	1280.7\\
22	979.64\\
27	1278.3\\
44	3074.2\\
16	497.41\\
39	2743.9\\
19	731.07\\
33	1932.4\\
37	2263.7\\
33	1949.8\\
18	646.58\\
29	1397.2\\
22	913.85\\
33	1857.3\\
22	916.36\\
49	3869.7\\
25	1177.4\\
43	2873.5\\
26	1123.2\\
31	1552.8\\
26	1234.9\\
18	539.74\\
13	369.03\\
32	1749.9\\
38	2509.2\\
32	1667.5\\
29	1508.2\\
35	2028.2\\
35	1918.7\\
135	24100\\
27	1282\\
42	2856.7\\
39	2431.9\\
51	4022\\
86	10068\\
15	490.06\\
39	2473.4\\
42	2811.5\\
43	3389.5\\
58	5089.2\\
97	12304\\
50	4085.2\\
20	774.32\\
28	1460.2\\
22	837.8\\
47	3368.6\\
154	27794\\
30	1490.7\\
10	200.01\\
37	2316\\
31	1573.5\\
54	4657.7\\
76	8212.4\\
55	4633.9\\
26	1251.3\\
33	1994\\
34	1984.8\\
25	1071.2\\
31	1678.3\\
34	1980.4\\
26	1144.6\\
39	2758\\
27	1167.7\\
32	1903.2\\
26	1088.7\\
29	1559.3\\
23	1025.4\\
26	1158.4\\
14	425.08\\
33	1843.9\\
24	1057\\
20	711.58\\
51	3655.8\\
35	2163.5\\
29	1336.4\\
46	3606.3\\
44	3048\\
45	3116.4\\
29	1359.3\\
28	1371.2\\
77	7989.7\\
70	7233.8\\
70	7502.7\\
40	2445.5\\
90	10805\\
53	4494\\
62	5832.6\\
47	3798.6\\
34	2026.1\\
14	351.78\\
47	3486.6\\
43	3045.3\\
20	725.51\\
28	1399\\
80	9010\\
25	1187.3\\
36	2217.3\\
27	1231.9\\
82	9897.3\\
31	1618.5\\
39	2899.5\\
23	821.99\\
27	1226.8\\
37	2341.9\\
129	21861\\
63	5980.2\\
64	5525.2\\
91	11909\\
27	1186.8\\
26	1243.4\\
17	579.27\\
33	2015.8\\
30	1555.7\\
33	1714.6\\
182	42427\\
26	1198\\
48	3857.6\\
46	3271.2\\
10	214.41\\
25	1143.9\\
26	1195.4\\
206	49461\\
37	2356\\
43	2799\\
51	3998.8\\
57	5250.3\\
36	2049.7\\
38	2310.4\\
28	1480.1\\
19	677.65\\
36	2223.9\\
45	3143.7\\
53	4262.9\\
48	3906.7\\
64	5807.9\\
65	6115.7\\
45	3078.5\\
54	4177.6\\
16	495.92\\
24	982.13\\
127	20514\\
48	3669\\
64	6356.3\\
42	2840.7\\
42	2785.4\\
17	532.26\\
48	3649.1\\
42	2449.1\\
20	767.31\\
34	2024\\
47	3400.9\\
42	2914.1\\
50	3853.9\\
87	10973\\
64	6165.9\\
32	1581.6\\
39	2311.6\\
35	2115.6\\
18	640.13\\
28	1395\\
27	1314.7\\
37	1977.8\\
30	1508.4\\
89	11897\\
22	868.13\\
28	1436\\
25	1194\\
15	443.08\\
41	2669.8\\
32	1924.7\\
34	1853.5\\
16	571.46\\
88	10377\\
39	2187.3\\
140	23859\\
39	2718.4\\
49	3448.4\\
72	6859\\
37	2170.9\\
41	2654.9\\
283	84255\\
31	1594.6\\
99	13494\\
103	15522\\
39	2369.1\\
35	1998.8\\
98	12600\\
45	3318.8\\
73	7311.9\\
34	1914.1\\
56	4944.2\\
39	2639.5\\
37	2345.9\\
55	4357\\
25	1106.5\\
29	1498.3\\
42	2454.8\\
35	2051.9\\
26	1182.4\\
45	3179.4\\
56	4816.1\\
39	2328.7\\
62	6282.3\\
94	12206\\
112	16351\\
25	1139.3\\
57	4888.9\\
50	3925.4\\
47	3691.8\\
96	12936\\
54	4575.7\\
28	1545.5\\
38	2500\\
47	3751.2\\
35	2098.4\\
45	3455.9\\
50	3835.7\\
43	3023.9\\
29	1430.7\\
42	2835\\
71	7520.2\\
68	6848.1\\
149	26776\\
67	6901.1\\
51	3717.3\\
34	1926.8\\
120	19897\\
43	3000.1\\
117	18050\\
39	2637.4\\
72	7382.1\\
40	2514.7\\
43	3225.1\\
32	1900.9\\
49	3757.7\\
41	2869.6\\
98	12090\\
96	12557\\
45	3641.2\\
129	23130\\
43	2889.8\\
93	11218\\
41	2584.7\\
88	10506\\
73	7758\\
108	14978\\
52	4557.2\\
18	632.81\\
36	2082.3\\
50	4182.8\\
37	2169.4\\
83	10037\\
40	2538.1\\
26	1159.7\\
40	2687.8\\
49	4139.4\\
27	1396.9\\
41	2725.2\\
36	2137.8\\
52	4286.7\\
37	2151.3\\
68	6743.6\\
43	2741\\
132	23340\\
73	8146.9\\
46	3439.8\\
47	3400\\
44	3203\\
41	2568.3\\
30	1505\\
25	1023.8\\
24	1065.6\\
85	10803\\
44	3146.6\\
29	1454\\
34	1908.4\\
101	13076\\
18	668.45\\
21	731.34\\
111	15602\\
95	11531\\
23	824.24\\
31	1677.2\\
84	10359\\
23	1001.8\\
55	5332\\
32	1743.7\\
27	1213.9\\
48	3322.1\\
18	581.63\\
28	1330.2\\
40	2418.6\\
30	1610.4\\
13	324.58\\
33	2055.2\\
14	385.93\\
52	4220.3\\
65	6696.1\\
49	3730.8\\
35	2070.9\\
109	16476\\
26	1166.6\\
41	2728.9\\
39	2654.9\\
61	5632\\
40	2864.4\\
36	2108.4\\
21	789.92\\
80	9178.8\\
36	2283.6\\
36	2109.8\\
38	2294.8\\
42	2768.9\\
27	1159.4\\
46	3227.2\\
23	890.85\\
34	1956.6\\
200	44048\\
37	2282.4\\
16	453.34\\
50	3864.2\\
19	735.41\\
41	2977.6\\
25	1134.2\\
39	2378.7\\
45	3377\\
83	9862\\
27	1254.9\\
67	6476\\
46	3415\\
35	1931.4\\
44	3043.5\\
30	1600\\
29	1521.7\\
23	1082.8\\
43	2902\\
28	1384.6\\
24	992.03\\
56	4589.5\\
33	1832.8\\
13	306.02\\
41	2836.1\\
34	1873.4\\
14	392.34\\
83	9650.2\\
33	1945.1\\
35	1961.1\\
29	1434.7\\
54	4522.3\\
41	2762\\
19	690.3\\
34	1843.6\\
19	644.05\\
38	2458\\
43	2764.6\\
14	392.9\\
17	575.66\\
12	285.16\\
};
\addlegendentry{Dispersione appr.}

    
        %     \nextgroupplot[%
        %         xmode=log,ymode=log,
        %         % xmin=10,xmax=283,
        %         % ymin=100,ymax=1000000,
        %     ] % This file was created by matlab2tikz.
%
\definecolor{mycolor1}{rgb}{0.00000,0.44706,0.69804}%
\definecolor{mycolor2}{rgb}{0.00000,0.44700,0.74100}%
\definecolor{mycolor3}{rgb}{0.85000,0.32500,0.09800}%
%
\addplot[only marks, mark=*, mark size=1.3919pt, color=mycolor1, fill=mycolor1, opacity=0.60, draw=none, mark options={draw=none,line width=0pt}] table[row sep=crcr]{%
x	y\\
34	2792.9\\
29	1461.2\\
96	17740\\
31	1841\\
27	1284.2\\
70	7413.7\\
24	1296.8\\
37	2086.3\\
42	2822.2\\
16	1121.2\\
29	1448.3\\
59	3517.6\\
69	3084.8\\
28	1666.2\\
11	316.43\\
20	1049.1\\
45	3669.5\\
27	1456\\
22	926.17\\
27	1339.8\\
44	5307.2\\
16	1660.6\\
39	4632.6\\
19	1253.3\\
33	2123.8\\
37	2536.8\\
33	1427.6\\
18	559.44\\
29	2346.3\\
22	1056.1\\
33	1455.6\\
22	853.89\\
49	8060.2\\
25	1557.4\\
43	3728.7\\
26	2498.1\\
31	3183.8\\
26	1283.1\\
18	928.69\\
13	531.65\\
32	4009.8\\
38	2341.5\\
32	3526.7\\
29	1847.1\\
35	1522.8\\
35	2959.3\\
135	21037\\
27	4250\\
42	4002.8\\
39	4585.9\\
51	7450.7\\
86	10546\\
15	958.4\\
39	4667.8\\
42	3524.4\\
43	3228.3\\
58	5888.3\\
97	25809\\
50	2364.1\\
20	1551.5\\
28	1180\\
22	1123.4\\
47	3198.2\\
154	61690\\
30	1800.9\\
10	476.02\\
37	7865.5\\
31	1732.5\\
54	14568\\
76	12463\\
55	4780.6\\
26	2590.9\\
33	1632.8\\
34	2090.4\\
25	1707.6\\
31	3693.4\\
34	4483.4\\
26	2189.6\\
39	3281\\
27	2696.9\\
32	1911.2\\
26	2267.6\\
29	2191.6\\
23	3101.7\\
26	2424.1\\
14	749.54\\
33	1965.1\\
24	969.75\\
20	1584.6\\
51	2047.9\\
35	3238\\
29	1268.3\\
46	1012.5\\
44	3591.5\\
45	3179.7\\
29	876.9\\
28	1678.3\\
77	3058.1\\
70	3893.3\\
70	3527.1\\
40	2561.9\\
90	5380.7\\
53	3920.4\\
62	1729.1\\
47	3658.2\\
34	1287.7\\
14	1023.3\\
47	2152.9\\
43	1029.3\\
20	773.59\\
28	1035.6\\
80	3048.4\\
25	559.24\\
36	1782.7\\
27	2168\\
82	4324.6\\
31	1099.3\\
39	1321.7\\
23	2157.3\\
27	2729.2\\
37	2212.2\\
129	8349.8\\
63	4553.7\\
64	1883\\
91	9154.3\\
27	1235.3\\
26	1276.6\\
17	753.82\\
33	1498.6\\
30	3090.4\\
33	1373.3\\
182	13735\\
26	1143.7\\
48	3081.7\\
46	1721.4\\
10	341.85\\
25	1199.2\\
26	825.11\\
206	25437\\
37	1207\\
43	1644.7\\
51	1967.5\\
57	6553.8\\
36	1815.8\\
38	1496\\
28	1388.6\\
19	1119.3\\
36	1547.1\\
45	3206.7\\
53	3502.1\\
48	3538.3\\
64	3162.7\\
65	2906\\
45	1333.3\\
54	2937.4\\
16	447.02\\
24	1107.1\\
127	8840.3\\
48	1831.9\\
64	2648.7\\
42	2839.5\\
42	1277.9\\
17	548.47\\
48	2816.9\\
42	2354.5\\
20	822.18\\
34	1128.5\\
47	989.88\\
42	2987\\
50	2293.3\\
87	6890.7\\
64	4044.4\\
32	1848.4\\
39	1189.2\\
35	2259.7\\
18	1096.8\\
28	964.82\\
27	892.5\\
37	2182\\
30	2481.7\\
89	11365\\
22	1876.3\\
28	1709.8\\
25	1245.4\\
15	603.61\\
41	3353.8\\
32	965.73\\
34	1941.2\\
16	777.61\\
88	6155.1\\
39	971.74\\
140	28355\\
39	1419.8\\
49	1914.8\\
72	2341.1\\
37	1657.1\\
41	3235.4\\
283	1.4911e+05\\
31	3074.5\\
99	18404\\
103	25946\\
39	3634.2\\
35	1688.7\\
98	13675\\
45	4322.5\\
73	8846.5\\
34	2226\\
56	8632.5\\
39	3146.1\\
37	2833.3\\
55	2350.2\\
25	727.72\\
29	1057.7\\
42	1232.3\\
35	3687.2\\
26	704.59\\
45	8540.1\\
56	6196.3\\
39	1517.5\\
62	3177.4\\
94	11954\\
112	24603\\
25	763.44\\
57	2394.6\\
50	2472.1\\
47	6309.8\\
96	6803\\
54	6535.5\\
28	2492.1\\
38	3660.2\\
47	3766\\
35	1989.3\\
45	1582.7\\
50	3799.9\\
43	1022.3\\
29	2097.1\\
42	1715.3\\
71	21267\\
68	6977.2\\
149	56314\\
67	5440.5\\
51	2841.7\\
34	1911.6\\
120	15030\\
43	7027.6\\
117	11737\\
39	1344.2\\
72	8082.9\\
40	3640.7\\
43	2307.2\\
32	2692.1\\
49	8621.3\\
41	3736.3\\
98	4900.1\\
96	11706\\
45	1674.9\\
129	34122\\
43	1978.4\\
93	8298.9\\
41	3445.7\\
88	10586\\
73	5801.6\\
108	16371\\
52	7741.2\\
18	413.78\\
36	1080.4\\
50	5419.3\\
37	2342.3\\
83	14389\\
40	2113.4\\
26	2404.1\\
40	1716.3\\
49	4315.8\\
27	1348\\
41	1402.6\\
36	918.64\\
52	5235.2\\
37	1030.1\\
68	7059\\
43	2343.2\\
132	14126\\
73	2991.4\\
46	4674.1\\
47	1179\\
44	1658.9\\
41	4812.7\\
30	1525.1\\
25	2873.9\\
24	1863.3\\
85	7690.3\\
44	2933.9\\
29	1795.1\\
34	1587.8\\
101	14161\\
18	1641.5\\
21	1386.1\\
111	16407\\
95	4706.6\\
23	1086.5\\
31	631.56\\
84	4251.8\\
23	847.99\\
55	4700\\
32	1171\\
27	835.69\\
48	880.94\\
18	365.55\\
28	1830.6\\
40	1133.6\\
30	1237.5\\
13	364.11\\
33	1634.7\\
14	555.67\\
52	2371.1\\
65	7456.3\\
49	2469.2\\
35	1344.7\\
109	6966.6\\
26	793.34\\
41	1319.3\\
39	1253.4\\
61	5936.9\\
40	1510.6\\
36	2086.5\\
21	594.12\\
80	4147.5\\
36	1272.1\\
36	2002.8\\
38	1140.6\\
42	2199.9\\
27	767.59\\
46	2351.8\\
23	631.19\\
34	1761.8\\
200	41377\\
37	2540.5\\
16	606.82\\
50	2176.7\\
19	661.74\\
41	2790.1\\
25	823.13\\
39	2005.5\\
45	2965.9\\
83	7188\\
27	595.27\\
67	2573.5\\
46	3338\\
35	1627\\
44	2367.6\\
30	1746.4\\
29	1119.9\\
23	511.76\\
43	2186.9\\
28	798.17\\
24	1210.4\\
56	3217.9\\
33	988.27\\
13	430.76\\
41	2929.4\\
34	900.86\\
14	434\\
83	8213.4\\
33	1569.8\\
35	1624.4\\
29	1040.3\\
54	3609.5\\
41	1429.5\\
19	678.71\\
34	2139.5\\
19	723.93\\
38	2002.5\\
43	1751.1\\
14	620.02\\
17	763.96\\
12	451.63\\
};
\addlegendentry{Dispersione esatta}

\addplot[only marks, mark=*, mark size=1.3919pt, color=black, fill=black, opacity=0.60, draw=none, mark options={draw=none,line width=0pt}] table[row sep=crcr]{%
x	y\\
34	1787\\
29	2052\\
96	39026\\
31	939\\
27	839\\
70	9435\\
24	756\\
37	2292\\
42	3353\\
16	509\\
29	1198\\
59	4061\\
69	4632\\
28	987\\
11	377\\
20	825\\
45	6367\\
27	1370\\
22	689\\
27	1107\\
44	4679\\
16	622\\
39	5236\\
19	558\\
33	1997\\
37	1291\\
33	1077\\
18	485\\
29	1640\\
22	758\\
33	1231\\
22	643\\
49	9267\\
25	1115\\
43	11048\\
26	1875\\
31	2762\\
26	927\\
18	665\\
13	140\\
32	2698\\
38	2211\\
32	775\\
29	1173\\
35	1719\\
35	3078\\
135	41095\\
27	2591\\
42	3900\\
39	3847\\
51	5607\\
86	11830\\
15	918\\
39	3169\\
42	3772\\
43	2489\\
58	4861\\
97	21264\\
50	3266\\
20	736\\
28	686\\
22	1102\\
47	4024\\
154	1.2234e+05\\
30	1543\\
10	296\\
37	7252\\
31	1142\\
54	13398\\
76	13899\\
55	3344\\
26	1499\\
33	1121\\
34	1971\\
25	1709\\
31	2961\\
34	3625\\
26	2774\\
39	3551\\
27	1922\\
32	1860\\
26	1781\\
29	1942\\
23	2014\\
26	1636\\
14	530\\
33	1466\\
24	787\\
20	1114\\
51	1692\\
35	2940\\
29	1332\\
46	1054\\
44	3996\\
45	4071\\
29	803\\
28	557\\
77	3928\\
70	3625\\
70	2531\\
40	1723\\
90	8518\\
53	3650\\
62	3213\\
47	8035\\
34	817\\
14	526\\
47	2742\\
43	759\\
20	950\\
28	533\\
80	4654\\
25	1153\\
36	1716\\
27	2299\\
82	3021\\
31	1103\\
39	1587\\
23	810\\
27	2388\\
37	2269\\
129	3241\\
63	3568\\
64	2459\\
91	6356\\
27	696\\
26	1459\\
17	558\\
33	2465\\
30	2046\\
33	1791\\
182	11424\\
26	663\\
48	2633\\
46	2157\\
10	222\\
25	802\\
26	424\\
206	37527\\
37	1069\\
43	1448\\
51	2715\\
57	7724\\
36	1800\\
38	1162\\
28	539\\
19	746\\
36	1003\\
45	3212\\
53	4779\\
48	5264\\
64	2484\\
65	3084\\
45	1597\\
54	3279\\
16	299\\
24	1131\\
127	2601\\
48	1783\\
64	2544\\
42	2057\\
42	1206\\
17	254\\
48	2507\\
42	2016\\
20	816\\
34	1805\\
47	1076\\
42	2516\\
50	2233\\
87	10377\\
64	2072\\
32	1299\\
39	1215\\
35	3735\\
18	866\\
28	626\\
27	253\\
37	2518\\
30	2667\\
89	9128\\
22	1204\\
28	1732\\
25	1513\\
15	873\\
41	3761\\
32	1140\\
34	1426\\
16	352\\
88	7596\\
39	668\\
140	20491\\
39	1086\\
49	1025\\
72	1475\\
37	1233\\
41	3002\\
283	2.0424e+05\\
31	2681\\
99	16428\\
103	32887\\
39	6629\\
35	1076\\
98	6332\\
45	4113\\
73	7877\\
34	1444\\
56	6926\\
39	2045\\
37	3241\\
55	1793\\
25	444\\
29	1020\\
42	1480\\
35	2286\\
26	591\\
45	5458\\
56	7320\\
39	1112\\
62	3063\\
94	13380\\
112	30134\\
25	317\\
57	1896\\
50	2648\\
47	5982\\
96	4539\\
54	4348\\
28	1451\\
38	3579\\
47	3124\\
35	1834\\
45	921\\
50	3106\\
43	719\\
29	1516\\
42	1249\\
71	5868\\
68	5857\\
149	61636\\
67	5463\\
51	1708\\
34	1459\\
120	10119\\
43	6342\\
117	8499\\
39	1041\\
72	6468\\
40	4061\\
43	1854\\
32	2516\\
49	12313\\
41	4009\\
98	4503\\
96	5379\\
45	1383\\
129	23237\\
43	1506\\
93	4228\\
41	2160\\
88	9837\\
73	5327\\
108	12182\\
52	5524\\
18	189\\
36	878\\
50	4430\\
37	1383\\
83	13086\\
40	2202\\
26	1338\\
40	1198\\
49	4702\\
27	1189\\
41	1263\\
36	572\\
52	3610\\
37	656\\
68	6317\\
43	2092\\
132	14984\\
73	3147\\
46	3825\\
47	739\\
44	1624\\
41	5048\\
30	1539\\
25	1574\\
24	2598\\
85	7294\\
44	1753\\
29	1546\\
34	1160\\
101	9599\\
18	1226\\
21	992\\
111	7348\\
95	2640\\
23	617\\
31	310\\
84	1691\\
23	466\\
55	3785\\
32	1352\\
27	544\\
48	517\\
18	119\\
28	1203\\
40	962\\
30	685\\
13	188\\
33	1762\\
14	209\\
52	1836\\
65	8994\\
49	3401\\
35	1176\\
109	4663\\
26	550\\
41	1006\\
39	973\\
61	4953\\
40	1238\\
36	1694\\
21	551\\
80	5029\\
36	1037\\
36	1691\\
38	832\\
42	1192\\
27	590\\
46	1533\\
23	486\\
34	1267\\
200	30990\\
37	1240\\
16	413\\
50	2688\\
19	324\\
41	2143\\
25	934\\
39	3741\\
45	2869\\
83	3945\\
27	521\\
67	2908\\
46	2419\\
35	1912\\
44	2676\\
30	2051\\
29	674\\
23	204\\
43	936\\
28	859\\
24	467\\
56	2173\\
33	679\\
13	261\\
41	2629\\
34	588\\
14	216\\
83	10336\\
33	1005\\
35	1380\\
29	687\\
54	3377\\
41	993\\
19	351\\
34	1937\\
19	459\\
38	1126\\
43	1196\\
14	412\\
17	325\\
12	184\\
};
\addlegendentry{Dispersione reale}

    
        %     \nextgroupplot[%
        %         xmode=log,ymode=log,
        %         % xmin=10,xmax=283,
        %         % ymin=100,ymax=1000000,
        %     ] % This file was created by matlab2tikz.
%
\definecolor{mycolor1}{rgb}{0.83529,0.36863,0.00000}%
\definecolor{mycolor2}{rgb}{0.00000,0.44700,0.74100}%
\definecolor{mycolor3}{rgb}{0.85000,0.32500,0.09800}%
%
\addplot[only marks, mark=*, mark size=1.2748pt, color=mycolor1, fill=mycolor1, opacity=0.60, draw=none, mark options={draw=none,line width=0pt}] table[row sep=crcr]{%
x	y\\
34	1181.2\\
29	195.19\\
96	1176.9\\
31	1213.6\\
27	581.44\\
70	3.1994e-05\\
24	458.2\\
37	280.3\\
42	85.348\\
16	1368.2\\
29	73.346\\
59	896.27\\
69	3138.3\\
28	305.75\\
11	1680.2\\
20	630.02\\
45	1079.4\\
27	43.186\\
22	131.02\\
27	815.82\\
44	0.35379\\
16	1404.1\\
39	578.34\\
19	1956\\
33	308.89\\
37	501.48\\
33	175.8\\
18	177.57\\
29	823.04\\
22	3054.5\\
33	125.76\\
22	1363\\
49	109.54\\
25	1422\\
43	9.9387\\
26	215.41\\
31	415.4\\
26	810.41\\
18	459.63\\
13	1107.2\\
32	984.98\\
38	2178.1\\
32	532.51\\
29	12.558\\
35	103.58\\
35	118.63\\
135	87359\\
27	148.1\\
42	1381.3\\
39	36.64\\
51	18.392\\
86	133.31\\
15	1116.7\\
39	489.1\\
42	189.17\\
43	831.81\\
58	0.08224\\
97	0.07288\\
50	374.99\\
20	1652.1\\
28	226.97\\
22	61.147\\
47	178.16\\
154	86610\\
30	312.63\\
10	1134\\
37	113.98\\
31	541.06\\
54	3.0622\\
76	4.5027e-08\\
55	5026.2\\
26	752.17\\
33	116.44\\
34	4409\\
25	835.86\\
31	588.99\\
34	1.9694\\
26	823.56\\
39	782.2\\
27	396.59\\
32	1201\\
26	182.32\\
29	759.85\\
23	737.33\\
26	243.05\\
14	1311.3\\
33	1720.9\\
24	1467\\
20	1998.8\\
51	415.38\\
35	9.5976\\
29	125.4\\
46	58.071\\
44	3917.9\\
45	1.1202\\
29	368.5\\
28	416.25\\
77	13516\\
70	1.4534e-08\\
70	392.48\\
40	17.843\\
90	2456.3\\
53	2.4597\\
62	1464.2\\
47	1256.5\\
34	545.86\\
14	893.36\\
47	6.7683\\
43	60.621\\
20	1093.9\\
28	328.06\\
80	53.553\\
25	830.37\\
36	2192.7\\
27	1714.2\\
82	6876.4\\
31	2537.4\\
39	0.44564\\
23	1412.2\\
27	397.05\\
37	1152.6\\
129	71760\\
63	2343.9\\
64	1395.1\\
91	17880\\
27	137.9\\
26	228.12\\
17	1865.6\\
33	212.93\\
30	159.61\\
33	6.5491\\
182	34862\\
26	648.23\\
48	0.036691\\
46	21.586\\
10	2122.1\\
25	49.548\\
26	1245.6\\
206	1.4774e+05\\
37	324.38\\
43	403.07\\
51	3.3213\\
57	879.85\\
36	505.3\\
38	204.28\\
28	55.571\\
19	812.89\\
36	86.713\\
45	3637.2\\
53	1457.3\\
48	158.7\\
64	15055\\
65	2785.8\\
45	11.749\\
54	32.631\\
16	1245.5\\
24	2076.3\\
127	51116\\
48	2163.3\\
64	1052.9\\
42	176.19\\
42	425.97\\
17	768.53\\
48	150.67\\
42	492.37\\
20	1031.8\\
34	76.142\\
47	1891.8\\
42	525.39\\
50	265.14\\
87	14638\\
64	2.4447\\
32	132.02\\
39	989.09\\
35	2352.5\\
18	866.32\\
28	595.75\\
27	2768\\
37	162.16\\
30	322.08\\
89	227\\
22	301.09\\
28	769.06\\
25	744.28\\
15	787.04\\
41	0.69486\\
32	262.98\\
34	11.335\\
16	794.28\\
88	3975.2\\
39	7.3249\\
140	63102\\
39	456.3\\
49	59.401\\
72	8602.3\\
37	0.0108\\
41	704.35\\
283	2.7648e+05\\
31	138.99\\
99	3115.9\\
103	2301.6\\
39	3543.5\\
35	348.94\\
98	3374.2\\
45	0.091589\\
73	5.9721\\
34	41.234\\
56	13.169\\
39	1060.3\\
37	1121.3\\
55	1849.2\\
25	397.8\\
29	2163\\
42	632.75\\
35	1055.9\\
26	2011.1\\
45	1821.5\\
56	225\\
39	647.68\\
62	8510.8\\
94	16255\\
112	1058.7\\
25	151.36\\
57	1208.1\\
50	925.94\\
47	18.72\\
96	8007.6\\
54	1374\\
28	1322.3\\
38	222.96\\
47	3.1997\\
35	0.032899\\
45	755.58\\
50	3.0507e-06\\
43	1489.7\\
29	290.69\\
42	2782.7\\
71	5512.6\\
68	3.7236e-09\\
149	45416\\
67	68.103\\
51	6735.5\\
34	27.974\\
120	745.68\\
43	200.13\\
117	27500\\
39	0.013448\\
72	72.469\\
40	0.54523\\
43	669.77\\
32	152.04\\
49	225.83\\
41	622.71\\
98	14402\\
96	28670\\
45	326.76\\
129	4407.9\\
43	1935.4\\
93	3397.7\\
41	5624.1\\
88	551.54\\
73	831.05\\
108	1894.9\\
52	5197.3\\
18	698.64\\
36	648.35\\
50	104.79\\
37	13.378\\
83	12835\\
40	9.7425\\
26	79.957\\
40	49.923\\
49	6.2296\\
27	297.95\\
41	2364.4\\
36	70.463\\
52	477.05\\
37	40.655\\
68	81.061\\
43	1612\\
132	18.154\\
73	4189.6\\
46	175.32\\
47	42.367\\
44	4336.2\\
41	286.22\\
30	96.768\\
25	1024\\
24	635.43\\
85	14451\\
44	206.56\\
29	597.51\\
34	118.05\\
101	23703\\
18	1005.6\\
21	755.62\\
111	17084\\
95	14246\\
23	1300\\
31	1260.1\\
84	4433.8\\
23	370.56\\
55	4530.3\\
32	1653.9\\
27	660.23\\
48	49.592\\
18	1071.4\\
28	515\\
40	293.07\\
30	174.92\\
13	672.62\\
33	1596.4\\
14	971.92\\
52	39.254\\
65	1206.5\\
49	14.105\\
35	51.535\\
109	18387\\
26	37.288\\
41	332.69\\
39	543.1\\
61	0.00041025\\
40	379.36\\
36	1804\\
21	574.01\\
80	11350\\
36	499.34\\
36	1415\\
38	116.83\\
42	5406.2\\
27	1.3599\\
46	26.817\\
23	593.55\\
34	48.534\\
200	1.9122e+05\\
37	1010.1\\
16	1613.3\\
50	900.68\\
19	232.1\\
41	725.69\\
25	408.71\\
39	1908.8\\
45	671.98\\
83	159.06\\
27	398.22\\
67	1304.5\\
46	1557\\
35	12.362\\
44	471.02\\
30	262.69\\
29	832.8\\
23	190.38\\
43	317.99\\
28	2044\\
24	1702.2\\
56	5.2612\\
33	169.68\\
13	1622.4\\
41	566.78\\
34	386.21\\
14	1496.2\\
83	412.96\\
33	46.589\\
35	428.22\\
29	26.538\\
54	1659.2\\
41	170.62\\
19	1763.1\\
34	2.5268\\
19	588.44\\
38	1.595\\
43	100.76\\
14	1391.5\\
17	676.93\\
12	889.5\\
};
\addlegendentry{Dispersione appr.}

\addplot[only marks, mark=*, mark size=1.2748pt, color=black, fill=black, opacity=0.60, draw=none, mark options={draw=none,line width=0pt}] table[row sep=crcr]{%
x	y\\
34	1787\\
29	2052\\
96	39026\\
31	939\\
27	839\\
70	9435\\
24	756\\
37	2292\\
42	3353\\
16	509\\
29	1198\\
59	4061\\
69	4632\\
28	987\\
11	377\\
20	825\\
45	6367\\
27	1370\\
22	689\\
27	1107\\
44	4679\\
16	622\\
39	5236\\
19	558\\
33	1997\\
37	1291\\
33	1077\\
18	485\\
29	1640\\
22	758\\
33	1231\\
22	643\\
49	9267\\
25	1115\\
43	11048\\
26	1875\\
31	2762\\
26	927\\
18	665\\
13	140\\
32	2698\\
38	2211\\
32	775\\
29	1173\\
35	1719\\
35	3078\\
135	41095\\
27	2591\\
42	3900\\
39	3847\\
51	5607\\
86	11830\\
15	918\\
39	3169\\
42	3772\\
43	2489\\
58	4861\\
97	21264\\
50	3266\\
20	736\\
28	686\\
22	1102\\
47	4024\\
154	1.2234e+05\\
30	1543\\
10	296\\
37	7252\\
31	1142\\
54	13398\\
76	13899\\
55	3344\\
26	1499\\
33	1121\\
34	1971\\
25	1709\\
31	2961\\
34	3625\\
26	2774\\
39	3551\\
27	1922\\
32	1860\\
26	1781\\
29	1942\\
23	2014\\
26	1636\\
14	530\\
33	1466\\
24	787\\
20	1114\\
51	1692\\
35	2940\\
29	1332\\
46	1054\\
44	3996\\
45	4071\\
29	803\\
28	557\\
77	3928\\
70	3625\\
70	2531\\
40	1723\\
90	8518\\
53	3650\\
62	3213\\
47	8035\\
34	817\\
14	526\\
47	2742\\
43	759\\
20	950\\
28	533\\
80	4654\\
25	1153\\
36	1716\\
27	2299\\
82	3021\\
31	1103\\
39	1587\\
23	810\\
27	2388\\
37	2269\\
129	3241\\
63	3568\\
64	2459\\
91	6356\\
27	696\\
26	1459\\
17	558\\
33	2465\\
30	2046\\
33	1791\\
182	11424\\
26	663\\
48	2633\\
46	2157\\
10	222\\
25	802\\
26	424\\
206	37527\\
37	1069\\
43	1448\\
51	2715\\
57	7724\\
36	1800\\
38	1162\\
28	539\\
19	746\\
36	1003\\
45	3212\\
53	4779\\
48	5264\\
64	2484\\
65	3084\\
45	1597\\
54	3279\\
16	299\\
24	1131\\
127	2601\\
48	1783\\
64	2544\\
42	2057\\
42	1206\\
17	254\\
48	2507\\
42	2016\\
20	816\\
34	1805\\
47	1076\\
42	2516\\
50	2233\\
87	10377\\
64	2072\\
32	1299\\
39	1215\\
35	3735\\
18	866\\
28	626\\
27	253\\
37	2518\\
30	2667\\
89	9128\\
22	1204\\
28	1732\\
25	1513\\
15	873\\
41	3761\\
32	1140\\
34	1426\\
16	352\\
88	7596\\
39	668\\
140	20491\\
39	1086\\
49	1025\\
72	1475\\
37	1233\\
41	3002\\
283	2.0424e+05\\
31	2681\\
99	16428\\
103	32887\\
39	6629\\
35	1076\\
98	6332\\
45	4113\\
73	7877\\
34	1444\\
56	6926\\
39	2045\\
37	3241\\
55	1793\\
25	444\\
29	1020\\
42	1480\\
35	2286\\
26	591\\
45	5458\\
56	7320\\
39	1112\\
62	3063\\
94	13380\\
112	30134\\
25	317\\
57	1896\\
50	2648\\
47	5982\\
96	4539\\
54	4348\\
28	1451\\
38	3579\\
47	3124\\
35	1834\\
45	921\\
50	3106\\
43	719\\
29	1516\\
42	1249\\
71	5868\\
68	5857\\
149	61636\\
67	5463\\
51	1708\\
34	1459\\
120	10119\\
43	6342\\
117	8499\\
39	1041\\
72	6468\\
40	4061\\
43	1854\\
32	2516\\
49	12313\\
41	4009\\
98	4503\\
96	5379\\
45	1383\\
129	23237\\
43	1506\\
93	4228\\
41	2160\\
88	9837\\
73	5327\\
108	12182\\
52	5524\\
18	189\\
36	878\\
50	4430\\
37	1383\\
83	13086\\
40	2202\\
26	1338\\
40	1198\\
49	4702\\
27	1189\\
41	1263\\
36	572\\
52	3610\\
37	656\\
68	6317\\
43	2092\\
132	14984\\
73	3147\\
46	3825\\
47	739\\
44	1624\\
41	5048\\
30	1539\\
25	1574\\
24	2598\\
85	7294\\
44	1753\\
29	1546\\
34	1160\\
101	9599\\
18	1226\\
21	992\\
111	7348\\
95	2640\\
23	617\\
31	310\\
84	1691\\
23	466\\
55	3785\\
32	1352\\
27	544\\
48	517\\
18	119\\
28	1203\\
40	962\\
30	685\\
13	188\\
33	1762\\
14	209\\
52	1836\\
65	8994\\
49	3401\\
35	1176\\
109	4663\\
26	550\\
41	1006\\
39	973\\
61	4953\\
40	1238\\
36	1694\\
21	551\\
80	5029\\
36	1037\\
36	1691\\
38	832\\
42	1192\\
27	590\\
46	1533\\
23	486\\
34	1267\\
200	30990\\
37	1240\\
16	413\\
50	2688\\
19	324\\
41	2143\\
25	934\\
39	3741\\
45	2869\\
83	3945\\
27	521\\
67	2908\\
46	2419\\
35	1912\\
44	2676\\
30	2051\\
29	674\\
23	204\\
43	936\\
28	859\\
24	467\\
56	2173\\
33	679\\
13	261\\
41	2629\\
34	588\\
14	216\\
83	10336\\
33	1005\\
35	1380\\
29	687\\
54	3377\\
41	993\\
19	351\\
34	1937\\
19	459\\
38	1126\\
43	1196\\
14	412\\
17	325\\
12	184\\
};
\addlegendentry{Dispersione reale}

        % \end{groupplot}
    
        % \begin{groupplot}[
        %     group style={
        %         group name=ColR2,
        %         plotRightGroup
        %     },
        %     plotRightCol,
        %     plotBottom,
        %     plotLegendSW,
        %     %
        %     xmin=3637.5,xmax=204237,
        % ]
        %     \nextgroupplot[%
        %         at={($(ColL2 c1r1.south east)+(\xPlotSep em,0)$)},
        %         xmode=log,ymode=log,
        %         % xmin=3637.5,xmax=204237,
        %         % ymin=0.00413009047051692,ymax=1.0004869525303,
        %     ] % This file was created by matlab2tikz.
%
\definecolor{mycolor1}{rgb}{0.00000,0.44700,0.74100}%

\addplot[only marks, mark=*, mark size=1.3693pt, color=black, fill=black, opacity=0.60, draw=none, mark options={draw=none,line width=0pt}] table[row sep=crcr]{%
x	y\\
3650	0.99468085106383\\
3735	0.984042553191489\\
3741	0.973404255319149\\
3761	0.962765957446808\\
3772	0.952127659574468\\
3785	0.941489361702128\\
3825	0.930851063829787\\
3847	0.920212765957447\\
3900	0.909574468085106\\
3928	0.898936170212766\\
3945	0.888297872340426\\
3996	0.877659574468085\\
4009	0.867021276595745\\
4024	0.856382978723404\\
4061	0.845744680851064\\
4061	0.835106382978723\\
4071	0.824468085106383\\
4113	0.813829787234043\\
4228	0.803191489361702\\
4348	0.792553191489362\\
4430	0.781914893617021\\
4503	0.771276595744681\\
4539	0.76063829787234\\
4632	0.75\\
4654	0.73936170212766\\
4663	0.728723404255319\\
4679	0.718085106382979\\
4702	0.707446808510638\\
4779	0.696808510638298\\
4861	0.686170212765957\\
4953	0.675531914893617\\
5029	0.664893617021277\\
5048	0.654255319148936\\
5236	0.643617021276596\\
5264	0.632978723404255\\
5327	0.622340425531915\\
5379	0.611702127659574\\
5458	0.601063829787234\\
5463	0.590425531914894\\
5524	0.579787234042553\\
5607	0.569148936170213\\
5857	0.558510638297872\\
5868	0.547872340425532\\
5982	0.537234042553192\\
6317	0.526595744680851\\
6332	0.515957446808511\\
6342	0.50531914893617\\
6356	0.49468085106383\\
6367	0.484042553191489\\
6468	0.473404255319149\\
6629	0.462765957446808\\
6926	0.452127659574468\\
7252	0.441489361702128\\
7294	0.430851063829787\\
7320	0.420212765957447\\
7348	0.409574468085106\\
7596	0.398936170212766\\
7724	0.388297872340426\\
7877	0.377659574468085\\
8035	0.367021276595745\\
8499	0.356382978723404\\
8518	0.345744680851064\\
8994	0.335106382978723\\
9128	0.324468085106383\\
9267	0.313829787234043\\
9435	0.303191489361702\\
9599	0.292553191489362\\
9837	0.281914893617021\\
10119	0.271276595744681\\
10336	0.26063829787234\\
10377	0.25\\
11048	0.23936170212766\\
11424	0.228723404255319\\
11830	0.218085106382979\\
12182	0.207446808510638\\
12313	0.196808510638298\\
13086	0.186170212765957\\
13380	0.175531914893617\\
13398	0.164893617021277\\
13899	0.154255319148936\\
14984	0.143617021276596\\
16428	0.132978723404255\\
20491	0.122340425531915\\
21264	0.111702127659574\\
23237	0.101063829787234\\
30134	0.0904255319148936\\
30990	0.0797872340425532\\
32887	0.0691489361702128\\
37527	0.0585106382978723\\
39026	0.0478723404255319\\
41095	0.0372340425531915\\
61636	0.0265957446808511\\
122339	0.0159574468085106\\
204237	0.00531914893617021\\
};
\addlegendentry{FRC empirica reale}

\addplot [color=black, mark repeat=9, mark phase=1, mark=+, mark options={solid, black}]
  table[row sep=crcr]{%
3637.5	1\\
3788.54993935839	0.949537816810858\\
3945.87234172165	0.901622065553929\\
4109.72767586131	0.856124247714574\\
4280.38722671172	0.812922349093734\\
4458.13354451935	0.771900512595218\\
4643.26091264341	0.732948727524845\\
4836.07583478224	0.695962534568237\\
5036.89754243213	0.660842745656075\\
5246.05852341874	0.627495177965562\\
5463.90507237627	0.59583040134476\\
5690.79786408554	0.565763498482441\\
5927.11255062052	0.53721383718029\\
6173.24038329176	0.510104854116755\\
6429.58886041643	0.484363849522645\\
6696.58240198765	0.459921792217835\\
6974.66305235981	0.436713134486259\\
7264.29121211354	0.414675636292709\\
7565.94640031188	0.393750198370032\\
7880.12804840974	0.373880703729123\\
8207.35632713103	0.355013867166658\\
8548.17300768248	0.337099092367008\\
8903.14235873003	0.32008833621509\\
9272.85208062291	0.303935979956297\\
9657.9142784119	0.28859870685797\\
10058.9664752731	0.274035386044354\\
10476.6726680149	0.260206962193476\\
10911.7244264153	0.247076350800179\\
11364.8420382106	0.234608338724395\\
11836.7757016304	0.222769489757984\\
12328.3067674531	0.211528054956865\\
12840.2490326395	0.200853887497989\\
13373.4500876847	0.190718361832814\\
13928.7927199204	0.181094296920473\\
14507.19637509	0.171955883334763\\
15109.6186796172	0.163278614049474\\
15737.0570260872	0.15503921871644\\
16390.550224567	0.147215601260069\\
17071.1802224973	0.139786780620984\\
17780.0738960051	0.132732834489867\\
18518.4049156008	0.126034845880625\\
19287.3956893508	0.119674852399582\\
20088.3193867412	0.113635798074661\\
20922.5020465844	0.107901487615373\\
21791.3247724572	0.102456542980945\\
22696.2260193077	0.0972863621401141\\
23638.7039750138	0.0923770799119943\\
24620.3190408384	0.0877155307829971\\
25642.6964148873	0.0832892136000928\\
26707.5287828473	0.0790862580457252\\
27816.5791204588	0.0750953928044779\\
28971.6836123634	0.0713059153361178\\
30174.7546921595	0.0677076631739571\\
31427.7842086969	0.0642909866715642\\
32732.8467238564	0.061046723124733\\
34092.1029472698	0.0579661721993158\\
35507.8033136712	0.0550410725990206\\
36982.2917087997	0.0522635799106019\\
38518.0093500227	0.0496262455670326\\
40117.4988281057	0.0471219968722397\\
41783.4083168193	0.0447441180338345\\
43518.4959573534	0.0424862321529745\\
45325.6344247971	0.0403422841230547\\
47207.8156842464	0.0383065243913687\\
49168.1559444106	0.0363734935401921\\
51209.900816924	0.0345380076459378\\
53336.4306898988	0.0327951443771205\\
55551.2663246212	0.0311402297938478\\
57858.074684653	0.0295688258134387\\
60260.6750069927	0.0280767183085531\\
62763.0451253437	0.026659905805917\\
65369.3280559641	0.0253145887553335\\
68083.8388569957	0.024037159340204\\
70911.0717726344	0.022824191802232\\
73855.7076739664	0.0216724332543637\\
76922.6218087908	0.0205787949573275\\
80116.89187326	0.0195403440363791\\
83443.8064187003	0.0185542956160364\\
86908.8736075329	0.0176180053517145\\
90517.8303327903	0.016728962338229\\
94276.651716329	0.015884782376153\\
98191.5610014598	0.0150832015779679\\
102269.039856381	0.0143220702968617\\
106515.839105467	0.0135993473618937\\
110938.989906178	0.012913094604065\\
115545.815390112	0.012261471658616\\
120343.942787444	0.0116427310296104\\
125341.316054851	0.0110552134035723\\
130546.209027823	0.0104973431996062\\
135967.239119128	0.00996762434406833\\
141613.381586117	0.00946463625845741\\
147493.984390494	0.00898703004976452\\
153618.783675143	0.00853352489306699\\
159997.919883649	0.00810290459666392\\
166641.954549187	0.00769401434054292\\
173561.887780587	0.00730575757943056\\
180769.176474522	0.00693709310212186\\
188275.753283964	0.00658703223920245\\
196094.046374327	0.00625463621167503\\
204237	0.00593901361338004\\
};
\addlegendentry{Adatt. Pareto reale ($\mathit{ML}$)}

\addplot [color=black, mark repeat=9, mark phase=1, mark=x, mark options={solid, black}]
  table[row sep=crcr]{%
3637.5	1.0004869525303\\
3788.54993935839	0.953165189609478\\
3945.87234172165	0.907427082756923\\
4109.72767586131	0.863290892405921\\
4280.38722671172	0.820768189461985\\
4458.13354451935	0.77986406626895\\
4643.26091264341	0.740577387414794\\
4836.07583478224	0.702901076404548\\
5036.89754243213	0.666822433995031\\
5246.05852341874	0.632323483817515\\
5463.90507237627	0.59938134080958\\
5690.79786408554	0.567968597935033\\
5927.11255062052	0.538053726688371\\
6173.24038329176	0.509601486955019\\
6429.58886041643	0.482573341926024\\
6696.58240198765	0.456927873941986\\
6974.66305235981	0.432621197360071\\
7264.29121211354	0.409607364794705\\
7565.94640031188	0.387838763371018\\
7880.12804840974	0.367266497944131\\
8207.35632713103	0.347840758571035\\
8548.17300768248	0.329511169868802\\
8903.14235873003	0.312227120247536\\
9272.85208062291	0.295938069362992\\
9657.9142784119	0.280593832487039\\
10058.9664752731	0.266144840839093\\
10476.6726680149	0.252542377254078\\
10911.7244264153	0.239738786878372\\
11364.8420382106	0.227687662881319\\
11836.7757016304	0.21634400744349\\
12328.3067674531	0.205664368531773\\
12840.2490326395	0.195606953193889\\
13373.4500876847	0.186131718300219\\
13928.7927199204	0.177200439828105\\
14507.19637509	0.168776761923201\\
15109.6186796172	0.160826227084399\\
15737.0570260872	0.153316288904044\\
16390.550224567	0.146216308854908\\
17071.1802224973	0.139497538651068\\
17780.0738960051	0.133133089723205\\
18518.4049156008	0.127097891341835\\
19287.3956893508	0.121368638896574\\
20088.3193867412	0.115923733797975\\
20922.5020465844	0.110743216412846\\
21791.3247724572	0.105808693376563\\
22696.2260193077	0.101103260548696\\
23638.7039750138	0.096611422793629\\
24620.3190408384	0.0923190116774575\\
25642.6964148873	0.0882131020783547\\
26707.5287828473	0.0842819286114441\\
27816.5791204588	0.0805148026725548\\
28971.6836123634	0.0769020308094709\\
30174.7546921595	0.0734348350356383\\
31427.7842086969	0.0701052756108015\\
32732.8467238564	0.0669061767265721\\
34092.1029472698	0.063831055453225\\
35507.8033136712	0.0608740542275787\\
36982.2917087997	0.0580298770910952\\
38518.0093500227	0.0552937298225691\\
40117.4988281057	0.0526612640511152\\
41783.4083168193	0.0501285253826495\\
43518.4959573534	0.0476919055266147\\
45325.6344247971	0.0453480983691843\\
47207.8156842464	0.0430940599043728\\
49168.1559444106	0.0409269719051086\\
51209.900816924	0.0388442091920599\\
53336.4306898988	0.0368433103385172\\
55551.2663246212	0.0349219516345141\\
57858.074684653	0.0330779241222635\\
60260.6750069927	0.0313091135074718\\
62763.0451253437	0.0296134827468033\\
65369.3280559641	0.0279890571102966\\
68083.8388569957	0.0264339115185369\\
70911.0717726344	0.0249461599574891\\
73855.7076739664	0.0235239467787836\\
76922.6218087908	0.0221654396996028\\
80116.89187326	0.0208688243238502\\
83443.8064187003	0.0196323000147655\\
86908.8736075329	0.0184540769582996\\
90517.8303327903	0.0173323742662265\\
94276.651716329	0.0162654189779162\\
98191.5610014598	0.0152514458297972\\
102269.039856381	0.0142886976716394\\
106515.839105467	0.0133754264187857\\
110938.989906178	0.0125098944392443\\
115545.815390112	0.0116903762840562\\
120343.942787444	0.0109151606784948\\
125341.316054851	0.0101825527003985\\
130546.209027823	0.00949087608024179\\
135967.239119128	0.00883847556538394\\
141613.381586117	0.00822371929828939\\
147493.984390494	0.00764500116537325\\
153618.783675143	0.00710074307949289\\
159997.919883649	0.00658939716498238\\
166641.954549187	0.00610944781952394\\
173561.887780587	0.00565941363207689\\
180769.176474522	0.00523784914056129\\
188275.753283964	0.00484334641703407\\
196094.046374327	0.00447453647172286\\
204237	0.00413009047051692\\
};
\addlegendentry{Adatt. BLN reale ($\mathit{ML}$)}
    
        %     \nextgroupplot[%
        %         xmode=log,ymode=log,
        %         % xmin=3925.51066195196,xmax=181881.391659954,
        %         % ymin=0.000929039723010346,ymax=1.01784721007988,
        %     ] % This file was created by matlab2tikz.
%
\definecolor{mycolor1}{rgb}{0.00000,0.44706,0.69804}%
\definecolor{mycolor2}{rgb}{0.00000,0.44700,0.74100}%
%
\addplot[only marks, mark=*, mark size=1.2500pt, color=mycolor1, fill=mycolor1, opacity=0.60, draw=none, mark options={draw=none,line width=0pt}] table[row sep=crcr]{%
x	y\\
4009.8	0.99468\\
4042.8	0.98404\\
4083.2	0.9734\\
4118.9	0.96277\\
4158.9	0.95213\\
4199.9	0.94149\\
4241.8	0.93085\\
4293.5	0.92021\\
4346.1	0.90957\\
4392.3	0.89894\\
4440.5	0.8883\\
4485	0.87766\\
4529	0.86702\\
4574.2	0.85638\\
4618.5	0.84574\\
4670.1	0.83511\\
4730.3	0.82447\\
4787.1	0.81383\\
4841	0.80319\\
4903.8	0.79255\\
4960.2	0.78191\\
5017.2	0.77128\\
5087.5	0.76064\\
5155.4	0.75\\
5233.1	0.73936\\
5291.8	0.72872\\
5360.5	0.71809\\
5431.2	0.70745\\
5501.1	0.69681\\
5574	0.68617\\
5661.2	0.67553\\
5733.2	0.66489\\
5815.2	0.65426\\
5904.3	0.64362\\
5987.8	0.63298\\
6075.5	0.62234\\
6160.2	0.6117\\
6238.2	0.60106\\
6339.4	0.59043\\
6443.4	0.57979\\
6550.4	0.56915\\
6655.6	0.55851\\
6768.6	0.54787\\
6863	0.53723\\
6984.1	0.5266\\
7105.1	0.51596\\
7243	0.50532\\
7363.9	0.49468\\
7529.2	0.48404\\
7674.5	0.4734\\
7798.1	0.46277\\
7951.1	0.45213\\
8117.5	0.44149\\
8265.8	0.43085\\
8464.1	0.42021\\
8676.5	0.40957\\
8871.5	0.39894\\
9049.7	0.3883\\
9256	0.37766\\
9441.2	0.36702\\
9666.5	0.35638\\
9881.2	0.34574\\
10150	0.33511\\
10431	0.32447\\
10718	0.31383\\
11062	0.30319\\
11293	0.29255\\
11688	0.28191\\
12013	0.27128\\
12316	0.26064\\
12705	0.25\\
13150	0.23936\\
13536	0.22872\\
13995	0.21809\\
14547	0.20745\\
15079	0.19681\\
15757	0.18617\\
16494	0.17553\\
17200	0.16489\\
17972	0.15426\\
18737	0.14362\\
19593	0.13298\\
20816	0.12234\\
22198	0.1117\\
23437	0.10106\\
25016	0.090426\\
27001	0.079787\\
29590	0.069149\\
32563	0.058511\\
36426	0.047872\\
43570	0.037234\\
54044	0.026596\\
68767	0.015957\\
1.4931e+05	0.0053191\\
};
\addlegendentry{FRC empirica esatta}

\addplot [color=mycolor1, only marks, every error bar/.append style={opacity=0.30}, mark=*, mark size=0pt, draw=none, forget plot]
 plot [error bars/.cd, x dir=both, x explicit, error bar style={line width=1pt, color=mycolor1}, error mark options={mark=none,mark size=0pt}]
 table[row sep=crcr, x error plus index=2, x error minus index=3]{%
4009.8	0.99468	41.683	41.683\\
4042.8	0.98404	42.729	42.729\\
4083.2	0.9734	42.64	42.64\\
4118.9	0.96277	42.259	42.259\\
4158.9	0.95213	42.401	42.401\\
4199.9	0.94149	43.247	43.247\\
4241.8	0.93085	44.025	44.025\\
4293.5	0.92021	45.653	45.653\\
4346.1	0.90957	47.232	47.232\\
4392.3	0.89894	47.443	47.443\\
4440.5	0.8883	47.604	47.604\\
4485	0.87766	47.648	47.648\\
4529	0.86702	48.056	48.056\\
4574.2	0.85638	48.712	48.712\\
4618.5	0.84574	48.794	48.794\\
4670.1	0.83511	49.967	49.967\\
4730.3	0.82447	48.825	48.825\\
4787.1	0.81383	49.432	49.432\\
4841	0.80319	50.68	50.68\\
4903.8	0.79255	49.872	49.872\\
4960.2	0.78191	50.875	50.875\\
5017.2	0.77128	51.172	51.172\\
5087.5	0.76064	52.783	52.783\\
5155.4	0.75	53.548	53.548\\
5233.1	0.73936	54.7	54.7\\
5291.8	0.72872	56.659	56.659\\
5360.5	0.71809	57.557	57.557\\
5431.2	0.70745	56.553	56.553\\
5501.1	0.69681	55.696	55.696\\
5574	0.68617	59.561	59.561\\
5661.2	0.67553	62.198	62.198\\
5733.2	0.66489	60.853	60.853\\
5815.2	0.65426	65.634	65.634\\
5904.3	0.64362	63.226	63.226\\
5987.8	0.63298	65.228	65.228\\
6075.5	0.62234	68.996	68.996\\
6160.2	0.6117	69.293	69.293\\
6238.2	0.60106	72.946	72.946\\
6339.4	0.59043	75.51	75.51\\
6443.4	0.57979	74.961	74.961\\
6550.4	0.56915	74.817	74.817\\
6655.6	0.55851	79.725	79.725\\
6768.6	0.54787	82.852	82.852\\
6863	0.53723	81.443	81.443\\
6984.1	0.5266	81.923	81.923\\
7105.1	0.51596	88.89	88.89\\
7243	0.50532	93.199	93.199\\
7363.9	0.49468	97.499	97.499\\
7529.2	0.48404	96.758	96.758\\
7674.5	0.4734	95.477	95.477\\
7798.1	0.46277	93.925	93.925\\
7951.1	0.45213	98.413	98.413\\
8117.5	0.44149	97.624	97.624\\
8265.8	0.43085	101.94	101.94\\
8464.1	0.42021	108.66	108.66\\
8676.5	0.40957	112.81	112.81\\
8871.5	0.39894	110.39	110.39\\
9049.7	0.3883	117.1	117.1\\
9256	0.37766	118.93	118.93\\
9441.2	0.36702	117.23	117.23\\
9666.5	0.35638	114.85	114.85\\
9881.2	0.34574	119.48	119.48\\
10150	0.33511	123.96	123.96\\
10431	0.32447	124.98	124.98\\
10718	0.31383	133.42	133.42\\
11062	0.30319	143.38	143.38\\
11293	0.29255	150.62	150.62\\
11688	0.28191	166.57	166.57\\
12013	0.27128	171.22	171.22\\
12316	0.26064	181.72	181.72\\
12705	0.25	192.57	192.57\\
13150	0.23936	208.87	208.87\\
13536	0.22872	200.05	200.05\\
13995	0.21809	213.59	213.59\\
14547	0.20745	224.95	224.95\\
15079	0.19681	224.48	224.48\\
15757	0.18617	254.54	254.54\\
16494	0.17553	273.77	273.77\\
17200	0.16489	282.21	282.21\\
17972	0.15426	309.96	309.96\\
18737	0.14362	340.73	340.73\\
19593	0.13298	351.38	351.38\\
20816	0.12234	385	385\\
22198	0.1117	402.42	402.42\\
23437	0.10106	424.8	424.8\\
25016	0.090426	456.58	456.58\\
27001	0.079787	560.02	560.02\\
29590	0.069149	636.25	636.25\\
32563	0.058511	797.84	797.84\\
36426	0.047872	902.21	902.21\\
43570	0.037234	1186.4	1186.4\\
54044	0.026596	1844.3	1844.3\\
68767	0.015957	2565.5	2565.5\\
1.4931e+05	0.0053191	4008.6	4008.6\\
};
\addplot [color=mycolor1, mark repeat=9, mark phase=1, mark=+, mark options={solid, mycolor1}]
  table[row sep=crcr]{%
3997.8	0.97494\\
4168.3	0.94141\\
4346.1	0.89973\\
4531.5	0.85488\\
4724.9	0.81129\\
4926.4	0.76977\\
5136.6	0.73039\\
5355.7	0.69303\\
5584.2	0.65759\\
5822.4	0.62397\\
6070.8	0.59207\\
6329.8	0.56181\\
6599.9	0.5331\\
6881.4	0.50586\\
7175	0.48002\\
7481.1	0.4555\\
7800.2	0.43224\\
8133	0.41017\\
8480	0.38924\\
8841.7	0.36937\\
9218.9	0.35052\\
9612.2	0.33264\\
10022	0.31567\\
10450	0.29957\\
10896	0.2843\\
11360	0.2698\\
11845	0.25605\\
12350	0.243\\
12877	0.23062\\
13427	0.21887\\
14000	0.20772\\
14597	0.19714\\
15219	0.18711\\
15869	0.17758\\
16546	0.16854\\
17252	0.15997\\
17988	0.15183\\
18755	0.1441\\
19555	0.13677\\
20389	0.12982\\
21259	0.12322\\
22166	0.11696\\
23112	0.11101\\
24098	0.10537\\
25126	0.10002\\
26198	0.094937\\
27315	0.090115\\
28481	0.085539\\
29696	0.081196\\
30962	0.077075\\
32283	0.073163\\
33661	0.069451\\
35097	0.065927\\
36594	0.062583\\
38155	0.059409\\
39783	0.056397\\
41480	0.053538\\
43249	0.050824\\
45095	0.048248\\
47018	0.045804\\
49024	0.043483\\
51116	0.041281\\
53296	0.03919\\
55570	0.037206\\
57941	0.035322\\
60412	0.033534\\
62990	0.031837\\
65677	0.030226\\
68479	0.028697\\
71400	0.027246\\
74446	0.025868\\
77622	0.02456\\
80934	0.023318\\
84386	0.02214\\
87986	0.021021\\
91740	0.019959\\
95654	0.01895\\
99734	0.017993\\
1.0399e+05	0.017085\\
1.0843e+05	0.016222\\
1.1305e+05	0.015403\\
1.1787e+05	0.014626\\
1.229e+05	0.013888\\
1.2815e+05	0.013187\\
1.3361e+05	0.012522\\
1.3931e+05	0.01189\\
1.4526e+05	0.01129\\
1.5145e+05	0.010721\\
1.5791e+05	0.010181\\
1.6465e+05	0.0096675\\
1.7168e+05	0.0091803\\
1.79e+05	0.0087177\\
1.8664e+05	0.0082785\\
1.946e+05	0.0078615\\
2.029e+05	0.0074656\\
2.1156e+05	0.0070897\\
2.2058e+05	0.0067328\\
2.2999e+05	0.0063939\\
2.398e+05	0.0060721\\
2.5003e+05	0.0057666\\
};
\addlegendentry{Adatt. Pareto esatto ($\mathit{ML}$)}


\addplot[area legend, draw=none, fill=mycolor1, fill opacity=0.15, postaction={pattern={Lines[angle=90,distance={5pt},line width={0.5pt}]}, pattern color=mycolor1!60}, forget plot]
table[row sep=crcr] {%
x	y\\
3997.8	0.98138\\
4168.3	0.95033\\
4346.1	0.90976\\
4531.5	0.8645\\
4724.9	0.82008\\
4926.4	0.77772\\
5136.6	0.73756\\
5355.7	0.6995\\
5584.2	0.66342\\
5822.4	0.62922\\
6070.8	0.59681\\
6329.8	0.56608\\
6599.9	0.53697\\
6881.4	0.50937\\
7175	0.48323\\
7481.1	0.45845\\
7800.2	0.43497\\
8133	0.41272\\
8480	0.39164\\
8841.7	0.37166\\
9218.9	0.35273\\
9612.2	0.33478\\
10022	0.31777\\
10450	0.30164\\
10896	0.28635\\
11360	0.27184\\
11845	0.25808\\
12350	0.24503\\
12877	0.23264\\
13427	0.22089\\
14000	0.20974\\
14597	0.19915\\
15219	0.18911\\
15869	0.17957\\
16546	0.17052\\
17252	0.16193\\
17988	0.15377\\
18755	0.14603\\
19555	0.13868\\
20389	0.1317\\
21259	0.12507\\
22166	0.11878\\
23112	0.11281\\
24098	0.10713\\
25126	0.10175\\
26198	0.096634\\
27315	0.091778\\
28481	0.087168\\
29696	0.08279\\
30962	0.078632\\
32283	0.074684\\
33661	0.070936\\
35097	0.067376\\
36594	0.063995\\
38155	0.060784\\
39783	0.057736\\
41480	0.05484\\
43249	0.052091\\
45095	0.049479\\
47018	0.046999\\
49024	0.044644\\
51116	0.042407\\
53296	0.040283\\
55570	0.038265\\
57941	0.036349\\
60412	0.034529\\
62990	0.032801\\
65677	0.031159\\
68479	0.0296\\
71400	0.028119\\
74446	0.026712\\
77622	0.025376\\
80934	0.024107\\
84386	0.022902\\
87986	0.021757\\
91740	0.020669\\
95654	0.019637\\
99734	0.018655\\
1.0399e+05	0.017723\\
1.0843e+05	0.016838\\
1.1305e+05	0.015997\\
1.1787e+05	0.015199\\
1.229e+05	0.01444\\
1.2815e+05	0.013719\\
1.3361e+05	0.013034\\
1.3931e+05	0.012384\\
1.4526e+05	0.011766\\
1.5145e+05	0.011179\\
1.5791e+05	0.010622\\
1.6465e+05	0.010092\\
1.7168e+05	0.0095889\\
1.79e+05	0.0091109\\
1.8664e+05	0.0086568\\
1.946e+05	0.0082255\\
2.029e+05	0.0078156\\
2.1156e+05	0.0074263\\
2.2058e+05	0.0070564\\
2.2999e+05	0.006705\\
2.398e+05	0.0063712\\
2.5003e+05	0.006054\\
2.5003e+05	0.0054792\\
2.398e+05	0.0057731\\
2.2999e+05	0.0060828\\
2.2058e+05	0.0064092\\
2.1156e+05	0.0067531\\
2.029e+05	0.0071156\\
1.946e+05	0.0074976\\
1.8664e+05	0.0079002\\
1.79e+05	0.0083245\\
1.7168e+05	0.0087716\\
1.6465e+05	0.0092429\\
1.5791e+05	0.0097395\\
1.5145e+05	0.010263\\
1.4526e+05	0.010815\\
1.3931e+05	0.011396\\
1.3361e+05	0.012009\\
1.2815e+05	0.012655\\
1.229e+05	0.013335\\
1.1787e+05	0.014053\\
1.1305e+05	0.014809\\
1.0843e+05	0.015606\\
1.0399e+05	0.016446\\
99734	0.017331\\
95654	0.018264\\
91740	0.019248\\
87986	0.020285\\
84386	0.021378\\
80934	0.02253\\
77622	0.023744\\
74446	0.025024\\
71400	0.026373\\
68479	0.027795\\
65677	0.029294\\
62990	0.030874\\
60412	0.03254\\
57941	0.034295\\
55570	0.036146\\
53296	0.038097\\
51116	0.040154\\
49024	0.042322\\
47018	0.044608\\
45095	0.047018\\
43249	0.049558\\
41480	0.052235\\
39783	0.055058\\
38155	0.058034\\
36594	0.061172\\
35097	0.064479\\
33661	0.067966\\
32283	0.071642\\
30962	0.075517\\
29696	0.079603\\
28481	0.08391\\
27315	0.088452\\
26198	0.093239\\
25126	0.098286\\
24098	0.10361\\
23112	0.10922\\
22166	0.11513\\
21259	0.12137\\
20389	0.12794\\
19555	0.13487\\
18755	0.14218\\
17988	0.14988\\
17252	0.15801\\
16546	0.16657\\
15869	0.17559\\
15219	0.18511\\
14597	0.19514\\
14000	0.20571\\
13427	0.21685\\
12877	0.2286\\
12350	0.24097\\
11845	0.25402\\
11360	0.26776\\
10896	0.28225\\
10450	0.2975\\
10022	0.31357\\
9612.2	0.3305\\
9218.9	0.34832\\
8841.7	0.36708\\
8480	0.38683\\
8133	0.40763\\
7800.2	0.42952\\
7481.1	0.45256\\
7175	0.47681\\
6881.4	0.50235\\
6599.9	0.52923\\
6329.8	0.55753\\
6070.8	0.58733\\
5822.4	0.61872\\
5584.2	0.65176\\
5355.7	0.68657\\
5136.6	0.72322\\
4926.4	0.76183\\
4724.9	0.8025\\
4531.5	0.84527\\
4346.1	0.88969\\
4168.3	0.9325\\
3997.8	0.96849\\
}--cycle;
\addplot [color=mycolor1, mark repeat=9, mark phase=1, mark=x, mark options={solid, mycolor1}]
  table[row sep=crcr]{%
3997.8	1.0099\\
4168.3	0.96444\\
4346.1	0.92061\\
4531.5	0.87836\\
4724.9	0.83767\\
4926.4	0.79853\\
5136.6	0.76091\\
5355.7	0.72479\\
5584.2	0.69014\\
5822.4	0.65691\\
6070.8	0.62508\\
6329.8	0.5946\\
6599.9	0.56544\\
6881.4	0.53755\\
7175	0.51088\\
7481.1	0.4854\\
7800.2	0.46107\\
8133	0.43783\\
8480	0.41564\\
8841.7	0.39447\\
9218.9	0.37427\\
9612.2	0.35499\\
10022	0.33661\\
10450	0.31908\\
10896	0.30236\\
11360	0.28642\\
11845	0.27123\\
12350	0.25674\\
12877	0.24294\\
13427	0.22978\\
14000	0.21724\\
14597	0.20529\\
15219	0.19391\\
15869	0.18307\\
16546	0.17275\\
17252	0.16292\\
17988	0.15356\\
18755	0.14466\\
19555	0.13619\\
20389	0.12814\\
21259	0.12048\\
22166	0.11321\\
23112	0.1063\\
24098	0.099745\\
25126	0.093524\\
26198	0.087626\\
27315	0.082036\\
28481	0.076742\\
29696	0.071733\\
30962	0.066995\\
32283	0.062519\\
33661	0.058291\\
35097	0.054303\\
36594	0.050543\\
38155	0.047002\\
39783	0.043669\\
41480	0.040536\\
43249	0.037592\\
45095	0.03483\\
47018	0.032241\\
49024	0.029816\\
51116	0.027547\\
53296	0.025426\\
55570	0.023445\\
57941	0.021598\\
60412	0.019877\\
62990	0.018276\\
65677	0.016787\\
68479	0.015404\\
71400	0.014121\\
74446	0.012932\\
77622	0.011831\\
80934	0.010814\\
84386	0.0098738\\
87986	0.0090067\\
91740	0.0082077\\
95654	0.0074722\\
99734	0.0067959\\
1.0399e+05	0.0061748\\
1.0843e+05	0.005605\\
1.1305e+05	0.0050828\\
1.1787e+05	0.0046047\\
1.229e+05	0.0041676\\
1.2815e+05	0.0037683\\
1.3361e+05	0.0034041\\
1.3931e+05	0.0030721\\
1.4526e+05	0.0027698\\
1.5145e+05	0.0024949\\
1.5791e+05	0.0022452\\
1.6465e+05	0.0020186\\
1.7168e+05	0.0018132\\
1.79e+05	0.0016272\\
1.8664e+05	0.0014589\\
1.946e+05	0.0013068\\
2.029e+05	0.0011696\\
2.1156e+05	0.0010458\\
2.2058e+05	0.00093423\\
2.2999e+05	0.00083385\\
2.398e+05	0.00074361\\
2.5003e+05	0.00066255\\
};
\addlegendentry{Adatt. BLN esatto ($\mathit{ML}$)}


\addplot[area legend, draw=none, fill=mycolor1, fill opacity=0.15, postaction={pattern={Lines[angle=45,distance={5pt},line width={0.5pt}]}, pattern color=mycolor1!60}, forget plot]
table[row sep=crcr] {%
x	y\\
3997.8	1.0175\\
4168.3	0.97182\\
4346.1	0.92776\\
4531.5	0.88527\\
4724.9	0.84436\\
4926.4	0.805\\
5136.6	0.76718\\
5355.7	0.73085\\
5584.2	0.696\\
5822.4	0.66258\\
6070.8	0.63056\\
6329.8	0.59991\\
6599.9	0.57058\\
6881.4	0.54252\\
7175	0.5157\\
7481.1	0.49007\\
7800.2	0.46559\\
8133	0.44221\\
8480	0.41988\\
8841.7	0.39857\\
9218.9	0.37824\\
9612.2	0.35883\\
10022	0.34031\\
10450	0.32265\\
10896	0.3058\\
11360	0.28973\\
11845	0.27441\\
12350	0.25979\\
12877	0.24586\\
13427	0.23258\\
14000	0.21991\\
14597	0.20785\\
15219	0.19635\\
15869	0.18539\\
16546	0.17496\\
17252	0.16502\\
17988	0.15556\\
18755	0.14656\\
19555	0.138\\
20389	0.12986\\
21259	0.12212\\
22166	0.11477\\
23112	0.1078\\
24098	0.10117\\
25126	0.094891\\
26198	0.088937\\
27315	0.083297\\
28481	0.077958\\
29696	0.072906\\
30962	0.06813\\
32283	0.063618\\
33661	0.059358\\
35097	0.055339\\
36594	0.05155\\
38155	0.047981\\
39783	0.044622\\
41480	0.041462\\
43249	0.038494\\
45095	0.035706\\
47018	0.033091\\
49024	0.030641\\
51116	0.028346\\
53296	0.0262\\
55570	0.024193\\
57941	0.02232\\
60412	0.020573\\
62990	0.018945\\
65677	0.017429\\
68479	0.016019\\
71400	0.01471\\
74446	0.013495\\
77622	0.012368\\
80934	0.011325\\
84386	0.010359\\
87986	0.0094673\\
91740	0.0086439\\
95654	0.0078846\\
99734	0.0071852\\
1.0399e+05	0.0065416\\
1.0843e+05	0.00595\\
1.1305e+05	0.0054068\\
1.1787e+05	0.0049086\\
1.229e+05	0.0044521\\
1.2815e+05	0.0040343\\
1.3361e+05	0.0036523\\
1.3931e+05	0.0033034\\
1.4526e+05	0.0029851\\
1.5145e+05	0.002695\\
1.5791e+05	0.0024308\\
1.6465e+05	0.0021906\\
1.7168e+05	0.0019723\\
1.79e+05	0.0017742\\
1.8664e+05	0.0015946\\
1.946e+05	0.0014319\\
2.029e+05	0.0012846\\
2.1156e+05	0.0011515\\
2.2058e+05	0.0010313\\
2.2999e+05	0.00092285\\
2.398e+05	0.0008251\\
2.5003e+05	0.00073707\\
2.5003e+05	0.00058803\\
2.398e+05	0.00066212\\
2.2999e+05	0.00074485\\
2.2058e+05	0.00083715\\
2.1156e+05	0.00094\\
2.029e+05	0.0010545\\
1.946e+05	0.0011818\\
1.8664e+05	0.0013232\\
1.79e+05	0.0014801\\
1.7168e+05	0.0016541\\
1.6465e+05	0.0018466\\
1.5791e+05	0.0020596\\
1.5145e+05	0.0022949\\
1.4526e+05	0.0025545\\
1.3931e+05	0.0028407\\
1.3361e+05	0.0031559\\
1.2815e+05	0.0035024\\
1.229e+05	0.0038831\\
1.1787e+05	0.0043009\\
1.1305e+05	0.0047587\\
1.0843e+05	0.00526\\
1.0399e+05	0.0058081\\
99734	0.0064067\\
95654	0.0070599\\
91740	0.0077715\\
87986	0.0085461\\
84386	0.0093882\\
80934	0.010303\\
77622	0.011294\\
74446	0.012369\\
71400	0.013531\\
68479	0.014788\\
65677	0.016144\\
62990	0.017607\\
60412	0.019182\\
57941	0.020877\\
55570	0.022697\\
53296	0.024652\\
51116	0.026747\\
49024	0.02899\\
47018	0.03139\\
45095	0.033954\\
43249	0.036691\\
41480	0.039609\\
39783	0.042716\\
38155	0.046022\\
36594	0.049536\\
35097	0.053267\\
33661	0.057225\\
32283	0.061419\\
30962	0.065861\\
29696	0.07056\\
28481	0.075527\\
27315	0.080775\\
26198	0.086314\\
25126	0.092157\\
24098	0.098317\\
23112	0.10481\\
22166	0.11165\\
21259	0.11884\\
20389	0.12641\\
19555	0.13438\\
18755	0.14276\\
17988	0.15156\\
17252	0.16081\\
16546	0.17054\\
15869	0.18075\\
15219	0.19148\\
14597	0.20274\\
14000	0.21457\\
13427	0.22698\\
12877	0.24001\\
12350	0.25369\\
11845	0.26805\\
11360	0.28312\\
10896	0.29893\\
10450	0.31551\\
10022	0.33291\\
9612.2	0.35116\\
9218.9	0.3703\\
8841.7	0.39036\\
8480	0.4114\\
8133	0.43344\\
7800.2	0.45654\\
7481.1	0.48073\\
7175	0.50606\\
6881.4	0.53257\\
6599.9	0.5603\\
6329.8	0.5893\\
6070.8	0.6196\\
5822.4	0.65125\\
5584.2	0.68428\\
5355.7	0.71874\\
5136.6	0.75465\\
4926.4	0.79206\\
4724.9	0.83098\\
4531.5	0.87144\\
4346.1	0.91346\\
4168.3	0.95706\\
3997.8	1.0023\\
}--cycle;
    
        %     \nextgroupplot[%
        %         xmode=log,ymode=log,
        %         % xmin=3825.04007879312,xmax=181881.391659954,
        %         % ymin=0.000887211268178068,ymax=1.04178769427766,
        %     ] % This file was created by matlab2tikz.
%
\definecolor{mycolor1}{rgb}{0.83529,0.36863,0.00000}%
\definecolor{mycolor2}{rgb}{0.00000,0.44700,0.74100}%
%
\addplot[only marks, mark=*, mark size=1.2500pt, color=mycolor1, fill=mycolor1, opacity=0.60, draw=none, mark options={draw=none,line width=0pt}] table[row sep=crcr]{%
x	y\\
4026.3	0.99468\\
4058.6	0.98404\\
4099.8	0.9734\\
4144	0.96277\\
4185.7	0.95213\\
4222.6	0.94149\\
4258.8	0.93085\\
4294	0.92021\\
4338.8	0.90957\\
4387.5	0.89894\\
4434.4	0.8883\\
4477.8	0.87766\\
4531.1	0.86702\\
4583.2	0.85638\\
4627.9	0.84574\\
4689.2	0.83511\\
4733	0.82447\\
4787	0.81383\\
4843.5	0.80319\\
4896.6	0.79255\\
4951.8	0.78191\\
5010.6	0.77128\\
5072.2	0.76064\\
5133.1	0.75\\
5190.8	0.73936\\
5257.5	0.72872\\
5330.8	0.71809\\
5395.5	0.70745\\
5462.9	0.69681\\
5544.5	0.68617\\
5620.8	0.67553\\
5702.2	0.66489\\
5790.5	0.65426\\
5881.8	0.64362\\
5964	0.63298\\
6028.3	0.62234\\
6133.8	0.6117\\
6237	0.60106\\
6320	0.59043\\
6406.8	0.57979\\
6506.9	0.56915\\
6616.8	0.55851\\
6722.3	0.54787\\
6828.5	0.53723\\
6953.2	0.5266\\
7088.7	0.51596\\
7213.5	0.50532\\
7325.9	0.49468\\
7446.2	0.48404\\
7581.1	0.4734\\
7724.5	0.46277\\
7861.5	0.45213\\
8007	0.44149\\
8164.3	0.43085\\
8339.4	0.42021\\
8505.8	0.40957\\
8697.7	0.39894\\
8854.2	0.3883\\
9063	0.37766\\
9293.7	0.36702\\
9542.3	0.35638\\
9795.1	0.34574\\
10050	0.33511\\
10298	0.32447\\
10572	0.31383\\
10877	0.30319\\
11188	0.29255\\
11456	0.28191\\
11809	0.27128\\
12134	0.26064\\
12539	0.25\\
12911	0.23936\\
13427	0.22872\\
13974	0.21809\\
14460	0.20745\\
15013	0.19681\\
15718	0.18617\\
16449	0.17553\\
17281	0.16489\\
18072	0.15426\\
18837	0.14362\\
19847	0.13298\\
20821	0.12234\\
22193	0.1117\\
23455	0.10106\\
24972	0.090426\\
26751	0.079787\\
29026	0.069149\\
31699	0.058511\\
35825	0.047872\\
42313	0.037234\\
53000	0.026596\\
72389	0.015957\\
1.5403e+05	0.0053191\\
};
\addlegendentry{FRC empirica appr.}

\addplot [color=mycolor1, only marks, every error bar/.append style={opacity=0.30}, mark=*, mark size=0pt, draw=none, forget plot]
 plot [error bars/.cd, x dir=both, x explicit, error bar style={line width=1pt, color=mycolor1}, error mark options={mark=none,mark size=0pt}]
 table[row sep=crcr, x error plus index=2, x error minus index=3]{%
4026.3	0.99468	36.588	36.588\\
4058.6	0.98404	35.959	35.959\\
4099.8	0.9734	36.838	36.838\\
4144	0.96277	37.465	37.465\\
4185.7	0.95213	38.893	38.893\\
4222.6	0.94149	39.784	39.784\\
4258.8	0.93085	40.702	40.702\\
4294	0.92021	40.402	40.402\\
4338.8	0.90957	39.037	39.037\\
4387.5	0.89894	40.059	40.059\\
4434.4	0.8883	39.704	39.704\\
4477.8	0.87766	40.479	40.479\\
4531.1	0.86702	41.553	41.553\\
4583.2	0.85638	42.897	42.897\\
4627.9	0.84574	42.524	42.524\\
4689.2	0.83511	46.655	46.655\\
4733	0.82447	48.383	48.383\\
4787	0.81383	48.919	48.919\\
4843.5	0.80319	48.945	48.945\\
4896.6	0.79255	49.791	49.791\\
4951.8	0.78191	49.148	49.148\\
5010.6	0.77128	47.503	47.503\\
5072.2	0.76064	48.508	48.508\\
5133.1	0.75	50.07	50.07\\
5190.8	0.73936	49.206	49.206\\
5257.5	0.72872	50.721	50.721\\
5330.8	0.71809	50.264	50.264\\
5395.5	0.70745	50.129	50.129\\
5462.9	0.69681	50.129	50.129\\
5544.5	0.68617	52.551	52.551\\
5620.8	0.67553	53.404	53.404\\
5702.2	0.66489	54.765	54.765\\
5790.5	0.65426	55.651	55.651\\
5881.8	0.64362	58.065	58.065\\
5964	0.63298	61.673	61.673\\
6028.3	0.62234	61.598	61.598\\
6133.8	0.6117	62.325	62.325\\
6237	0.60106	65.704	65.704\\
6320	0.59043	68.439	68.439\\
6406.8	0.57979	67.477	67.477\\
6506.9	0.56915	74.772	74.772\\
6616.8	0.55851	75.741	75.741\\
6722.3	0.54787	78.334	78.334\\
6828.5	0.53723	80.276	80.276\\
6953.2	0.5266	85.433	85.433\\
7088.7	0.51596	85.646	85.646\\
7213.5	0.50532	87.296	87.296\\
7325.9	0.49468	90.363	90.363\\
7446.2	0.48404	89.834	89.834\\
7581.1	0.4734	87.181	87.181\\
7724.5	0.46277	85.579	85.579\\
7861.5	0.45213	91.022	91.022\\
8007	0.44149	89.981	89.981\\
8164.3	0.43085	95.299	95.299\\
8339.4	0.42021	90.558	90.558\\
8505.8	0.40957	97.967	97.967\\
8697.7	0.39894	97.569	97.569\\
8854.2	0.3883	97.36	97.36\\
9063	0.37766	101.82	101.82\\
9293.7	0.36702	108.12	108.12\\
9542.3	0.35638	116.55	116.55\\
9795.1	0.34574	116.34	116.34\\
10050	0.33511	122.98	122.98\\
10298	0.32447	127.97	127.97\\
10572	0.31383	139.11	139.11\\
10877	0.30319	147.36	147.36\\
11188	0.29255	147.95	147.95\\
11456	0.28191	158.48	158.48\\
11809	0.27128	170.86	170.86\\
12134	0.26064	180.39	180.39\\
12539	0.25	194.84	194.84\\
12911	0.23936	200.17	200.17\\
13427	0.22872	209.38	209.38\\
13974	0.21809	224.5	224.5\\
14460	0.20745	234.84	234.84\\
15013	0.19681	233.15	233.15\\
15718	0.18617	258.63	258.63\\
16449	0.17553	271.52	271.52\\
17281	0.16489	294.36	294.36\\
18072	0.15426	345.21	345.21\\
18837	0.14362	336.44	336.44\\
19847	0.13298	382.5	382.5\\
20821	0.12234	385.99	385.99\\
22193	0.1117	410.26	410.26\\
23455	0.10106	437.73	437.73\\
24972	0.090426	505.73	505.73\\
26751	0.079787	587.47	587.47\\
29026	0.069149	684.85	684.85\\
31699	0.058511	813.07	813.07\\
35825	0.047872	967.25	967.25\\
42313	0.037234	1232	1232\\
53000	0.026596	1956	1956\\
72389	0.015957	2969.7	2969.7\\
1.5403e+05	0.0053191	5124.6	5124.6\\
};
\addplot [color=mycolor1, mark repeat=9, mark phase=1, mark=+, mark options={solid, mycolor1}]
  table[row sep=crcr]{%
4007.8	0.97865\\
4178.6	0.94382\\
4356.8	0.90082\\
4542.5	0.85485\\
4736.2	0.81065\\
4938.1	0.76874\\
5148.7	0.72901\\
5368.2	0.69133\\
5597.1	0.65561\\
5835.7	0.62174\\
6084.5	0.58962\\
6343.9	0.55917\\
6614.4	0.53029\\
6896.4	0.50291\\
7190.4	0.47695\\
7497	0.45233\\
7816.6	0.42899\\
8149.9	0.40685\\
8497.3	0.38586\\
8859.6	0.36596\\
9237.4	0.34708\\
9631.2	0.32918\\
10042	0.31221\\
10470	0.29612\\
10916	0.28085\\
11382	0.26638\\
11867	0.25265\\
12373	0.23964\\
12900	0.22729\\
13450	0.21559\\
14024	0.20448\\
14622	0.19396\\
15245	0.18397\\
15895	0.1745\\
16573	0.16552\\
17280	0.157\\
18016	0.14893\\
18784	0.14127\\
19585	0.134\\
20420	0.12711\\
21291	0.12057\\
22199	0.11438\\
23145	0.1085\\
24132	0.10292\\
25161	0.097632\\
26233	0.092616\\
27352	0.087859\\
28518	0.083346\\
29734	0.079066\\
31002	0.075006\\
32323	0.071156\\
33701	0.067504\\
35138	0.064039\\
36636	0.060753\\
38198	0.057636\\
39827	0.05468\\
41525	0.051875\\
43295	0.049215\\
45141	0.046692\\
47066	0.044298\\
49072	0.042028\\
51165	0.039874\\
53346	0.037831\\
55621	0.035892\\
57992	0.034054\\
60464	0.03231\\
63042	0.030655\\
65730	0.029085\\
68532	0.027596\\
71454	0.026184\\
74501	0.024844\\
77677	0.023573\\
80989	0.022367\\
84442	0.021222\\
88042	0.020137\\
91796	0.019107\\
95709	0.01813\\
99790	0.017203\\
1.0404e+05	0.016324\\
1.0848e+05	0.01549\\
1.1311e+05	0.014698\\
1.1793e+05	0.013947\\
1.2296e+05	0.013235\\
1.282e+05	0.012559\\
1.3366e+05	0.011918\\
1.3936e+05	0.011309\\
1.453e+05	0.010732\\
1.515e+05	0.010184\\
1.5796e+05	0.0096642\\
1.6469e+05	0.009171\\
1.7171e+05	0.0087031\\
1.7904e+05	0.0082591\\
1.8667e+05	0.0078379\\
1.9463e+05	0.0074382\\
2.0292e+05	0.0070589\\
2.1158e+05	0.006699\\
2.206e+05	0.0063576\\
2.3e+05	0.0060336\\
2.3981e+05	0.0057261\\
2.5003e+05	0.0054343\\
};
\addlegendentry{Adatt. Pareto appr. ($\mathit{ML}$)}


\addplot[area legend, draw=none, fill=mycolor1, fill opacity=0.15, postaction={pattern={Lines[angle=90,distance={5pt},line width={0.5pt}]}, pattern color=mycolor1!60}, forget plot]
table[row sep=crcr] {%
x	y\\
4007.8	0.98443\\
4178.6	0.95203\\
4356.8	0.91008\\
4542.5	0.86353\\
4736.2	0.8185\\
4938.1	0.77583\\
5148.7	0.73541\\
5368.2	0.69711\\
5597.1	0.66082\\
5835.7	0.62645\\
6084.5	0.59388\\
6343.9	0.56303\\
6614.4	0.5338\\
6896.4	0.50612\\
7190.4	0.4799\\
7497	0.45506\\
7816.6	0.43154\\
8149.9	0.40926\\
8497.3	0.38816\\
8859.6	0.36816\\
9237.4	0.34922\\
9631.2	0.33128\\
10042	0.31427\\
10470	0.29815\\
10916	0.28287\\
11382	0.26838\\
11867	0.25465\\
12373	0.24163\\
12900	0.22927\\
13450	0.21756\\
14024	0.20645\\
14622	0.19591\\
15245	0.18591\\
15895	0.17643\\
16573	0.16743\\
17280	0.15889\\
18016	0.1508\\
18784	0.14311\\
19585	0.13582\\
20420	0.1289\\
21291	0.12234\\
22199	0.11611\\
23145	0.1102\\
24132	0.10459\\
25161	0.099274\\
26233	0.094224\\
27352	0.089432\\
28518	0.084885\\
29734	0.080569\\
31002	0.076474\\
32323	0.072588\\
33701	0.0689\\
35138	0.0654\\
36636	0.062078\\
38198	0.058925\\
39827	0.055933\\
41525	0.053094\\
43295	0.050399\\
45141	0.047841\\
47066	0.045414\\
49072	0.04311\\
51165	0.040923\\
53346	0.038848\\
55621	0.036878\\
57992	0.035008\\
60464	0.033233\\
63042	0.031549\\
65730	0.02995\\
68532	0.028433\\
71454	0.026992\\
74501	0.025625\\
77677	0.024327\\
80989	0.023096\\
84442	0.021926\\
88042	0.020816\\
91796	0.019763\\
95709	0.018763\\
99790	0.017814\\
1.0404e+05	0.016912\\
1.0848e+05	0.016057\\
1.1311e+05	0.015245\\
1.1793e+05	0.014474\\
1.2296e+05	0.013742\\
1.282e+05	0.013048\\
1.3366e+05	0.012388\\
1.3936e+05	0.011762\\
1.453e+05	0.011168\\
1.515e+05	0.010604\\
1.5796e+05	0.010068\\
1.6469e+05	0.0095597\\
1.7171e+05	0.009077\\
1.7904e+05	0.0086187\\
1.8667e+05	0.0081837\\
1.9463e+05	0.0077706\\
2.0292e+05	0.0073785\\
2.1158e+05	0.0070062\\
2.206e+05	0.0066528\\
2.3e+05	0.0063172\\
2.3981e+05	0.0059986\\
2.5003e+05	0.0056961\\
2.5003e+05	0.0051726\\
2.3981e+05	0.0054536\\
2.3e+05	0.0057499\\
2.206e+05	0.0060624\\
2.1158e+05	0.0063919\\
2.0292e+05	0.0067393\\
1.9463e+05	0.0071057\\
1.8667e+05	0.0074921\\
1.7904e+05	0.0078996\\
1.7171e+05	0.0083292\\
1.6469e+05	0.0087824\\
1.5796e+05	0.0092602\\
1.515e+05	0.0097641\\
1.453e+05	0.010295\\
1.3936e+05	0.010856\\
1.3366e+05	0.011447\\
1.282e+05	0.01207\\
1.2296e+05	0.012727\\
1.1793e+05	0.013421\\
1.1311e+05	0.014152\\
1.0848e+05	0.014923\\
1.0404e+05	0.015736\\
99790	0.016593\\
95709	0.017498\\
91796	0.018452\\
88042	0.019458\\
84442	0.020519\\
80989	0.021638\\
77677	0.022818\\
74501	0.024063\\
71454	0.025376\\
68532	0.02676\\
65730	0.028221\\
63042	0.029761\\
60464	0.031386\\
57992	0.033099\\
55621	0.034907\\
53346	0.036813\\
51165	0.038824\\
49072	0.040945\\
47066	0.043183\\
45141	0.045542\\
43295	0.048031\\
41525	0.050657\\
39827	0.053426\\
38198	0.056347\\
36636	0.059429\\
35138	0.062679\\
33701	0.066107\\
32323	0.069724\\
31002	0.073539\\
29734	0.077563\\
28518	0.081807\\
27352	0.086285\\
26233	0.091008\\
25161	0.09599\\
24132	0.10125\\
23145	0.10679\\
22199	0.11264\\
21291	0.11881\\
20420	0.12531\\
19585	0.13218\\
18784	0.13942\\
18016	0.14706\\
17280	0.15511\\
16573	0.16361\\
15895	0.17257\\
15245	0.18203\\
14622	0.192\\
14024	0.20252\\
13450	0.21361\\
12900	0.22531\\
12373	0.23765\\
11867	0.25066\\
11382	0.26437\\
10916	0.27884\\
10470	0.29408\\
10042	0.31015\\
9631.2	0.32709\\
9237.4	0.34494\\
8859.6	0.36375\\
8497.3	0.38357\\
8149.9	0.40444\\
7816.6	0.42643\\
7497	0.4496\\
7190.4	0.474\\
6896.4	0.49971\\
6614.4	0.52678\\
6343.9	0.55531\\
6084.5	0.58537\\
5835.7	0.61703\\
5597.1	0.6504\\
5368.2	0.68556\\
5148.7	0.72261\\
4938.1	0.76166\\
4736.2	0.80281\\
4542.5	0.84618\\
4356.8	0.89157\\
4178.6	0.93561\\
4007.8	0.97287\\
}--cycle;
\addplot [color=mycolor1, mark repeat=9, mark phase=1, mark=x, mark options={solid, mycolor1}]
  table[row sep=crcr]{%
4007.8	1.0058\\
4178.6	0.96036\\
4356.8	0.91653\\
4542.5	0.87427\\
4736.2	0.83356\\
4938.1	0.79438\\
5148.7	0.7567\\
5368.2	0.72052\\
5597.1	0.68578\\
5835.7	0.65247\\
6084.5	0.62055\\
6343.9	0.58998\\
6614.4	0.56072\\
6896.4	0.53274\\
7190.4	0.50598\\
7497	0.48043\\
7816.6	0.45602\\
8149.9	0.43272\\
8497.3	0.41049\\
8859.6	0.38928\\
9237.4	0.36906\\
9631.2	0.34978\\
10042	0.33141\\
10470	0.31391\\
10916	0.29724\\
11382	0.28136\\
11867	0.26624\\
12373	0.25184\\
12900	0.23814\\
13450	0.22509\\
14024	0.21268\\
14622	0.20088\\
15245	0.18964\\
15895	0.17896\\
16573	0.1688\\
17280	0.15915\\
18016	0.14997\\
18784	0.14125\\
19585	0.13296\\
20420	0.12509\\
21291	0.11762\\
22199	0.11053\\
23145	0.1038\\
24132	0.097427\\
25161	0.091382\\
26233	0.085656\\
27352	0.080233\\
28518	0.075101\\
29734	0.070248\\
31002	0.06566\\
32323	0.061326\\
33701	0.057235\\
35138	0.053375\\
36636	0.049737\\
38198	0.04631\\
39827	0.043084\\
41525	0.040051\\
43295	0.0372\\
45141	0.034523\\
47066	0.032012\\
49072	0.029659\\
51165	0.027455\\
53346	0.025393\\
55621	0.023466\\
57992	0.021666\\
60464	0.019987\\
63042	0.018422\\
65730	0.016964\\
68532	0.015609\\
71454	0.014349\\
74501	0.013179\\
77677	0.012094\\
80989	0.011088\\
84442	0.010157\\
88042	0.0092967\\
91796	0.0085014\\
95709	0.0077674\\
99790	0.0070907\\
1.0404e+05	0.0064673\\
1.0848e+05	0.0058936\\
1.1311e+05	0.0053662\\
1.1793e+05	0.0048818\\
1.2296e+05	0.0044373\\
1.282e+05	0.0040299\\
1.3366e+05	0.0036568\\
1.3936e+05	0.0033154\\
1.453e+05	0.0030034\\
1.515e+05	0.0027184\\
1.5796e+05	0.0024585\\
1.6469e+05	0.0022216\\
1.7171e+05	0.0020058\\
1.7904e+05	0.0018095\\
1.8667e+05	0.0016311\\
1.9463e+05	0.0014691\\
2.0292e+05	0.0013221\\
2.1158e+05	0.0011889\\
2.206e+05	0.0010682\\
2.3e+05	0.00095904\\
2.3981e+05	0.00086034\\
2.5003e+05	0.00077119\\
};
\addlegendentry{Adatt. BLN appr. ($\mathit{ML}$)}


\addplot[area legend, draw=none, fill=mycolor1, fill opacity=0.15, postaction={pattern={Lines[angle=45,distance={5pt},line width={0.5pt}]}, pattern color=mycolor1!60}, forget plot]
table[row sep=crcr] {%
x	y\\
4007.8	1.0119\\
4178.6	0.9662\\
4356.8	0.92213\\
4542.5	0.87965\\
4736.2	0.83875\\
4938.1	0.7994\\
5148.7	0.76158\\
5368.2	0.72527\\
5597.1	0.69043\\
5835.7	0.65702\\
6084.5	0.62501\\
6343.9	0.59437\\
6614.4	0.56504\\
6896.4	0.53699\\
7190.4	0.51017\\
7497	0.48455\\
7816.6	0.46007\\
8149.9	0.4367\\
8497.3	0.4144\\
8859.6	0.39311\\
9237.4	0.37281\\
9631.2	0.35345\\
10042	0.33499\\
10470	0.31739\\
10916	0.30062\\
11382	0.28465\\
11867	0.26943\\
12373	0.25493\\
12900	0.24113\\
13450	0.22799\\
14024	0.21548\\
14622	0.20357\\
15245	0.19224\\
15895	0.18146\\
16573	0.17121\\
17280	0.16146\\
18016	0.1522\\
18784	0.14339\\
19585	0.13502\\
20420	0.12707\\
21291	0.11953\\
22199	0.11237\\
23145	0.10557\\
24132	0.099129\\
25161	0.093021\\
26233	0.087234\\
27352	0.081755\\
28518	0.076569\\
29734	0.071664\\
31002	0.067027\\
32323	0.062646\\
33701	0.05851\\
35138	0.054608\\
36636	0.050929\\
38198	0.047462\\
39827	0.044198\\
41525	0.041128\\
43295	0.038241\\
45141	0.03553\\
47066	0.032985\\
49072	0.030598\\
51165	0.028361\\
53346	0.026267\\
55621	0.024308\\
57992	0.022477\\
60464	0.020767\\
63042	0.019171\\
65730	0.017683\\
68532	0.016297\\
71454	0.015007\\
74501	0.013808\\
77677	0.012694\\
80989	0.01166\\
84442	0.010701\\
88042	0.0098132\\
91796	0.0089912\\
95709	0.0082311\\
99790	0.007529\\
1.0404e+05	0.0068809\\
1.0848e+05	0.0062833\\
1.1311e+05	0.0057329\\
1.1793e+05	0.0052262\\
1.2296e+05	0.0047604\\
1.282e+05	0.0043325\\
1.3366e+05	0.0039397\\
1.3936e+05	0.0035796\\
1.453e+05	0.0032498\\
1.515e+05	0.0029479\\
1.5796e+05	0.0026719\\
1.6469e+05	0.0024198\\
1.7171e+05	0.0021897\\
1.7904e+05	0.0019799\\
1.8667e+05	0.0017888\\
1.9463e+05	0.0016148\\
2.0292e+05	0.0014567\\
2.1158e+05	0.001313\\
2.206e+05	0.0011825\\
2.3e+05	0.0010642\\
2.3981e+05	0.00095703\\
2.5003e+05	0.00085999\\
2.5003e+05	0.00068239\\
2.3981e+05	0.00076365\\
2.3e+05	0.00085386\\
2.206e+05	0.00095392\\
2.1158e+05	0.0010648\\
2.0292e+05	0.0011876\\
1.9463e+05	0.0013234\\
1.8667e+05	0.0014734\\
1.7904e+05	0.0016391\\
1.7171e+05	0.0018219\\
1.6469e+05	0.0020233\\
1.5796e+05	0.002245\\
1.515e+05	0.0024889\\
1.453e+05	0.0027569\\
1.3936e+05	0.0030511\\
1.3366e+05	0.0033738\\
1.282e+05	0.0037273\\
1.2296e+05	0.0041142\\
1.1793e+05	0.0045373\\
1.1311e+05	0.0049995\\
1.0848e+05	0.0055038\\
1.0404e+05	0.0060536\\
99790	0.0066524\\
95709	0.0073038\\
91796	0.0080116\\
88042	0.0087801\\
84442	0.0096135\\
80989	0.010516\\
77677	0.011493\\
74501	0.01255\\
71454	0.01369\\
68532	0.01492\\
65730	0.016246\\
63042	0.017673\\
60464	0.019207\\
57992	0.020855\\
55621	0.022624\\
53346	0.024519\\
51165	0.026549\\
49072	0.02872\\
47066	0.03104\\
45141	0.033517\\
43295	0.036159\\
41525	0.038974\\
39827	0.04197\\
38198	0.045158\\
36636	0.048545\\
35138	0.052142\\
33701	0.055959\\
32323	0.060006\\
31002	0.064293\\
29734	0.068831\\
28518	0.073633\\
27352	0.078711\\
26233	0.084077\\
25161	0.089744\\
24132	0.095726\\
23145	0.10204\\
22199	0.10869\\
21291	0.11571\\
20420	0.12311\\
19585	0.1309\\
18784	0.1391\\
18016	0.14774\\
17280	0.15683\\
16573	0.1664\\
15895	0.17646\\
15245	0.18705\\
14622	0.19818\\
14024	0.20989\\
13450	0.2222\\
12900	0.23515\\
12373	0.24875\\
11867	0.26304\\
11382	0.27807\\
10916	0.29385\\
10470	0.31043\\
10042	0.32784\\
9631.2	0.34612\\
9237.4	0.36531\\
8859.6	0.38545\\
8497.3	0.40658\\
8149.9	0.42873\\
7816.6	0.45196\\
7497	0.4763\\
7190.4	0.5018\\
6896.4	0.52848\\
6614.4	0.5564\\
6343.9	0.58559\\
6084.5	0.61609\\
5835.7	0.64792\\
5597.1	0.68114\\
5368.2	0.71576\\
5148.7	0.75183\\
4938.1	0.78935\\
4736.2	0.82836\\
4542.5	0.86889\\
4356.8	0.91094\\
4178.6	0.95453\\
4007.8	0.99967\\
}--cycle;
        % \end{groupplot}
    %\endgroup
%\endgroup
}
            {Studio della configurazione di riferimento della \ref{eqLeggeEmigrazioneTagliaForza} con parametri dalla \cref{tabParametriRegolaEmigrazioneTagliaForza}; per la spiegazione dei grafici si veda il \cref{secDefinizioneGrafici}.}
            {.575\linewidth}{figStudioConfigurazioneRiferimentoRETF}

        \subsection{Interpretazioni}

            In poche parole la \eqref{eqLeggeEmigrazioneTaglia} descrive la tendenza degl'individui di aggregarsi per i piú svariati motivi: lavoro, sicurezza, famiglia, eccetera.

            I due principali parametri del modello possono quindi essere cosí interpretati:

            \begin{itemize}[label=\(\diamond\)]
                \item \(\lambda\in(0,1)\) rappresenta l'attrattività dei poli, ossia la frazione che le città piú popolose riescono al massimo ad attrarre in un'interazione;
                \item \(\alpha\in\mathbb R^+\) indica la rapidità d'emigrazione e influenza quanto rapidamente il rapporto \(s_*/s\) raggiunge la massima attrattività \(\lambda\).
            \end{itemize}
        
    \section{Altri casi notevoli}

        \subsection{Varianti delle regole d'interazione}\label{secVariantiRegoleInterazione}

        \subsection{Il caso dell'Italia}\label{secCasoItalia}

        \subsection{Il caso della Valle d'Aosta}\label{secValleDAosta}
