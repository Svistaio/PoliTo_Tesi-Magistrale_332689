
% !TEX root = ../Testa/Principale.tex
% LTeX: language=it

\chapter{Simulazioni}\label{secSimulazioni}

    Quello che ho riscontrato è che i risultati ottenuti per la Sardegna sono di fatto gli stessi per tutte le regioni, inclusa l'Italia; questa è la ragione per cui in questa sezione ci si concentrerà solo su quella regione. Alla fine, però, si mostrano alcuni grafici anche per l'Italia intera.

    \section{Premesse}

        \subsection{Metodo Monte Carlo}\label{secMetodoMonteCarlo}

            \newcommand\ave[1]{\left\langle#1\right\rangle}
            \newcommand\bM{\mathbf M}
            \newcommand\cO{\mathcal O}
            \begin{algorithm}[!t]
                \caption{Algoritmo di Monte Carlo per equazioni di tipo su un grafo}
                \label{algMonteCarlo}
                \begin{algorithmic}[1]
                    \Require adjacency matrix \(\bM\); initial state \(V_0\in\cO^N\); time step \(\Delta{t}>0\); final time \(T>0\)
                    \State \(\tilde{V}\gets V_0\)
                    \State \(t\gets 1\)
                    \For{\(t<T\)}
                    \State \(\ave{\varphi}(t)\gets\frac{1}{N}\sum_{i=1}^N\varphi(\tilde{V}(i))\)
                    \State \(V\gets\tilde{V}\)
                    \State \(P\gets \text{random permutation of } \left\{1,\,\dots,\,N\right\}\)
                    \State \(p_1\gets (P(1),\,\dots,\,P(N/2))\)
                    \State \(p_2\gets (P(N/2 + 1),\,\dots,\,P(N))\)
                    \State \(i\gets 1\)
                    \For{\(i<N/2\)}
                    \State \(\Theta\sim\operatorname{Bernoulli}\!\left(B(p_1(i),p_2(i))\Delta{t}\right)\)
                    \State \(\tilde{V}(p_1(i))\gets V(p_1(i))(1-\Theta)+\Psi(V(p_1(i)),V(p_2(i)))\Theta\)
                    \State \(\tilde{V}(p_2(i))\gets V(p_2(i))(1-\Theta)+\Psi_\ast(V(p_2(i)),V(p_1(i)))\Theta\)
                    \State \(i\gets i+1\)
                    \EndFor
                    \State \(t\gets t+\Delta t\)
                    \EndFor
                \end{algorithmic}
            \end{algorithm}

        \subsection{Fluttuazioni}

        \subsection{Grafici}

    \section{Regole d'emigrazione}

        La scelta di \(E(s,s_*)\) dipende da come si vuole modellizzare il fenomeno dell'immigrazione.

        \subsection{Popolazione}

            Una prima possibilità consiste nella
            %
            \begin{equation}
                \label{eqLeggeEmigrazione}
                E(s,s_*)\equiv\lambda\frac{(s_*/s)^\alpha}{1+(s_*/s)^\alpha},
            \end{equation}
            %
            che in essenza è la \cite[(2.2), § 2, p. 223]{Gualandi2019SDC} modificata mediante la \cite[(4.5), § 4, p. 228]{Gualandi2019SDC}, ossia è una funzione di Hill di ordine \(\alpha\), in cui v'è un tasso di emigrazione maggiore verso città con popolazione relativa, data dal rapporto \(s_*/s\), maggiore; gli unici due parametri presenti, invece, presentano il seguente significato:

            \begin{itemize}[label=\(\diamond\)]
                \item \(\lambda\in(0,1)\) rappresenta l'attrattività dei poli, ossia la frazione che le città piú popolose riescono al massimo ad attrarre in un'interazione;
                \item \(\alpha\in\mathbb R^+\) indica la rapidità d'emigrazione e influenza quanto rapidamente il rapporto \(s_*/s\) raggiunge la massima attrattività \(\lambda\).
            \end{itemize}

            In poche parole la \eqref{eqLeggeEmigrazione} descrive la tendenza degl'individui di aggregarsi per i piú svariati motivi: lavoro, sicurezza, famiglia, eccetera.

            In questo caso si hanno quindi interazioni non simmetriche, poiché dalla \eqref{eqLegameLineareInterazioneCittà} \(p\neq q\) e \(q\neq p\), e non lineari, a causa della \eqref{eqLeggeEmigrazione}.
            
            % Mediante la conservazione di popolazione ho rifatto molto velocemente i conti della formula prima del Teorema 5.1 a p. 16 di [1], confermando che d⟨s⟩/dt=0, come ci si aspetta da questo tipo d'interazione.


        \subsection{Popolazione-Connettività}

        \subsection{Popolazione-Connettività frazionata}

        \subsection{Popolazione-Forza}
        
    \section{Interpretazione}
