
% !TEX root = ../Testa/Principale.tex
% LTeX: language=it

\chapter{Introduzione}\label{secIntroduzione}

    Lo studio della distribuzione della popolazione tra le città è stato un fenomeno già studiato [sporadicamente] in passato. Il primo fu Auerbach \cite{Auerbach1913} che all'inizio del 19-esimo secolo notò una caratteristica poi formalizzata successivamente da Zipf \cite{Zipf2016} quasi cinquantanni dopo: nella sua teoria ha introdotto la legge che oggi prende il suo nome.\daRivedere{}

    Aggiungere le formule sul rango\daRivedere{}.

    D'altra parte, nella letteratura piú recente, perlomeno quella a conoscenza dell'autore, la maggior parte degli articoli su tale argomento o considerano le città come luoghi ove vivono gli agenti, creando cosí un modello di equazioni di Boltzmann elevate e accoppiate \daRivedere{}, o non considerano una naturale struttura da grafo \daRivedere{} oltre ad astrarre eccessivamente le città e i loro dintorni \daRivedere{}.
   
    In questo elaborato le città verranno invece considerate come agenti, caratterizzati dalla loro popolazione, che interagiscono su mediante la struttura di un grafo spaziale; tale descrizione, salvo il grafo, non è affatto dissimile a quella classica delle molecole di un gas caratterizzate dalla loro velocità e posizione.

    La suddivisione è come segue: nel secondo capitolo si affronteranno brevemente e [superficialmente] alcuni concetti della teoria dei grafi, soprattutto quelli pertinenti alla topologia interurbana; nel terzo la teoria cinetica dei sistemi multiagente è approfondita prima da un punto di vista classico, poi mediata dai grafi sia esattamente che approssimatamente; infine nel quarto si illustreranno i principali risultati mentre nel quinto si conclude con delle note finali e potenziali sviluppi futuri.
    
    Per il lettore interessato sarà anche presente un'appendice ove il codice in Python è spiegato a grandi linee.