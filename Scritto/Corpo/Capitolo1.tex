
% !TEX root = ../Testa/Principale.tex
% LTeX: language=it

\chapter{Introduzione}\label{secIntroduzione}

    Lo studio della distribuzione della popolazione tra le città è stato un fenomeno già analizzato [sporadicamente] in passato. Il primo fu Auerbach \cite{Auerbach1913} che all'inizio del 19-esimo secolo notò una caratteristica poi formalizzata successivamente da Zipf\footnotemark{} \cite{Zipf2016} quasi cinquantanni dopo, seppure inizialmente in un contesto linguistico; difatti la legge di Zipf è una legge empirica di forma
    \footnotetext{In realtà, come Zipf stesso ammette \cite[p. 546]{Zipf2016}, non fu il primo a scoprire tale legge; in ogni caso la popolarizzò a tal punto che oggi è conosciuta col suo nome.}
    %
    \begin{equation}
        \label{eqLeggeZipf}
        f\propto\frac1 r\quad\text{ove}
        \left\{
        \begin{split}
            &f\text{ è la frequenza della parola,}\\
            &r\text{ è il suo rango nella classifica;}
        \end{split}
        \right.
        % https://en.wikipedia.org/wiki/Zipf%27s_law#cite_note-7
        % https://archive.org/details/in.ernet.dli.2015.90211/page/n393/mode/1up
    \end{equation}
    %
    per dare un esempio concreto, se la 10 parola piú frequente compare 2000 volte allora esiste una costante \(C\) tale che \(2000\approx C/10\). La stessa legge descritta in \eqref{eqLeggeZipf} si può ritrovare in molt'altri contesti, tra i quali figura la distribuzione della popolazione tra le città se si in luogo di \(f\) la popolazione \(t\) \cite[Fig. 9-2, p. 375]{Zipf2016} (\(t\) sta per \textit{taglia}); tuttavia ciò è vero solo qualora \(t\gg1\), ossia nella coda della distribuzione, contrariamente al contesto linguistico originale di Zipf ove \eqref{eqLeggeZipf} vale sempre.
  
    Piú in generale la legge [empirica] di Zipf può essere descritta da una distribuzione di Pareto: sia \(T\) la variabile aleatoria legata alla taglia delle città e sia \(f_T(t)\) la sua funzione di densità di probabilità\footnotemark{}, per la quale \(f_T(t)\de t\) indica il numero di città con \(t\in(t,t+dt)\), allora si dice che \(T\) segue una distribuzione di Pareto se soddisfa
    \footnotetext{Affinché \(f_T(t)\) esista bisogna rigorosamente anche supporre che sia assolutamente continua ma in questa introduzione non è necessario entrare nei dettagli, per i quali si rimanda alla § \ref{secTCSM}.}
    %
    \begin{equation}
        \label{eqDistribuzionePareto}
        R(t)\approx\frac1{t^p}
        \quad\text{se }t\gg1,
        \quad\quad\text{ove }
        R(t)=\int_{t}^{+\infty}f_T(\tau)\de\tau,
        % https://en.wikipedia.org/wiki/Pareto_distribution#Relation_to_Zipf's_law
    \end{equation}
    %
    che altro non è che la funzione di ripartizione complementare di \(T\), la quale rappresenta matematicamente il numero di città con popolazione maggiore o uguale a \(t\); si indica con \(R(t)\) perché è analoga al rango nella \eqref{eqLeggeZipf} una volta che s'impone l'indice di Pareto \(p\) unitario.

    Piú recentemente \cite{Eeckhout2004,Gualandi2019HBLD} si è scoperto che al di fuori della coda la \(f_T(t)\) è ben fittata dalla distribuzione lognormale con funzione di distribuzione di probabilità
    %
    \begin{equation}
        \label{eqDistribuzioneLognormale}
        f_X(x)=\frac1{x\sigma\sqrt{2\pi}}\exp\left( -\frac{(\log x-\mu)^2}{2\sigma^2}\right)
        \quad\text{ove }X\sim\DstrLognormale(\mu,\sigma).
    \end{equation}
    %
    Come suggerisce il nome, la peculiarità di tale distribuzione è che vale \(\log X\sim\mathcal N(\mu,\sigma)\) (\cref{figEsempioFunzioneLognormale}) in cui \(\mathcal N(\mu,\sigma)\) indica la distribuzione gaussiana.

    \tikzfigure{H}{.75\linewidth}
        {
% !TEX root = ../../../Esperimenti/.tex/MEF.tex
% LTeX: language=it

\begin{tikzpicture}%
    % La lognormale è «f(x)=(1/(x*sigma*sqrt(2*pi)))*exp(-(ln(x)-mu)^2/(2*sigma^2))»

    \newcommand\muPrm{0}    % Media della normale sottostante
    \newcommand\sigmaPrm{1} % Deviazione standard della normale sottostante

    % Definizione della funzione matematica pgf
    \pgfmathdeclarefunction{lognormalpdf}{1}{%
        \pgfmathparse{1/(#1*\sigmaPrm*sqrt(2*pi))*exp(-((ln(#1)-\muPrm)^2)/(2*\sigmaPrm*\sigmaPrm))}%
    }

    \newcommand\captionSize{12}
    \newcommand\labelSize{10}
    \newcommand\tickSize{10}

    \pgfmathsetmacro\hsep{1cm}
    % \pgfmathsetmacro\widthPlot{(\linewidth-\hsep)/2}
    \pgfmathsetmacro\widthPlot{7cm}
    \pgfmathsetmacro\heightPlot{.75*\widthPlot}

    \begin{groupplot}[
        group style={
            group name=lgnrml,
            group size=2 by 1,
            horizontal sep=\hsep,
            xlabels at=edge bottom,
            ylabels at=edge left,
            x descriptions at=edge bottom,
            y descriptions at=edge left
        },
        width=\widthPlot,
        height=\heightPlot,
        axis x line*=bottom,
        axis y line*=left,
        xlabel={$x$},
        ylabel={$f_X(x)$},
        xlabel style={font=\color{white!15!black}\dimTesto{\labelSize}},
        ylabel style={font=\color{white!15!black}\dimTesto{\labelSize}},
        xticklabel style={font=\dimTesto{\tickSize}},
        yticklabel style={font=\dimTesto{\tickSize}},
        grid=major,
        grid style={dashed},
        every axis plot/.append style={
            % ultra thick,
            thick,
            smooth
        },
        ymin=0,
    ]
        % Scala lineare
        \nextgroupplot[
            xmin=0,
            xmax=10, %5
            minor tick num=1,
        ]
        \addplot[
            % color=blue,
            domain=0.01:20,
            samples=400
        ] {lognormalpdf(x)};

        % Scala semilogaritmica
        \nextgroupplot[
            xmode=log,
            xmin=0.01,
            xmax=20,
            % log ticks with fixed point, % Mostra i numeri (0.1,1,10) invece di (10^-1,10^0,10^1)
        ]
        % Nota: il dominio è molto più ampio qui per mostrare le code
        \addplot[
            % color=red,
            domain=0.005:25,
            samples=500
        ] {lognormalpdf(x)};
    \end{groupplot}

    \tikzset{SubCaption/.style={
        text width=\widthPlot,
        anchor=north,
        align=center,
        yshift=-2.5em
    }}
    \node[SubCaption] at (lgnrml c1r1.south)
        {\dimTesto{\captionSize}\sottoDidascalia{Scala lineare.\label{figLognormaleScalaLineare}}};
    \node[SubCaption] at (lgnrml c2r1.south)
        {\dimTesto{\captionSize}\sottoDidascalia{Scala semilogaritmica.\label{figLognormaleScalaSemiLogaritmica}}};

    \redefineTikZbounds{lgnrml c1r1}{lgnrml c2r1}
\end{tikzpicture}%}
        {Funzione \(\DstrLognormale(0,1)\).}
        {\linewidth}{figEsempioFunzioneLognormale}

    Perdipiú da Gualandi \etal{} \cite{Gualandi2019HBLD,Gualandi2019SDC} ci si è poi resi conto che per fittare l'intera distribuzione è sufficiente considerare una lognormale bimodale, vale a dire una combinazione convessa della \eqref{eqDistribuzioneLognormale}
    %
    \begin{equation}
        \label{eqDistribuzioneLognormaleBimodale}
        f_T(t)=zf^1_T(t)+(1-z)f^2_T(t),
    \end{equation}
    %
    ove \(z\in[0,1]\) è un parametro da fittare tanto quanto le medie e le varianze associate a \(f^1_T(t)\) e \(f^2_T(t)\) per un totale di 5 parametri.

    Pertanto l'obbiettivo di questa tesi è il seguente: si vuole simulare attraverso la Teoria Cinetica dei Sistemi MultiAgente (TCSMA) le interazioni tra città su un grafo spaziale, tentando di riprodurre una distribuzione della popolazione con caratteristiche analoghe a quelle realmente misurate (principalmente corpo lognormale e coda di Pareto); una volta raggiunto questo scopo si potrà poi meditare sulle implicazioni della legge d'interazione riguardo al fenomeno dell'emigrazione, che è alla base delle interazioni tra città.

    Si sottolinea che qui le città verranno invece considerate come agenti, caratterizzati dalla loro popolazione, che interagiscono attraverso la struttura sottostante di un grafo [spaziale]; tale descrizione, salvo il grafo, non è affatto dissimile a quella classica delle molecole di un gas caratterizzate dalla loro velocità e posizione.
   
    Difatti, in letteratura, perlomeno quella a conoscenza dell'autore, non esistono articoli che trattano contemporaneamente le teorie cinetiche, la distribuzione della città e i grafi, nonostante la rappresentazione di queste su un grafo sia del tutto naturale; piú nel dettaglio gli articoli in letteratura
    
    \begin{itemize}[
        label=\(\diamond\),
        topsep=0.5em,
        parsep=0em,
        itemsep=0.25em,
        leftmargin=2em,
        rightmargin=1.5em,
        % \leftmargin + \itemindent = \labelindent + \labelwidth + \labelsep
        %itemindent=!,
        labelindent=30pt,
        %labelwidth=!,
        %labelsep=!,
    ]
        \item o applicano la TCSMA alle città senza considerare una naturale topologia sottostante \cite{Gualandi2019SDC}\footnotemark{} o, in senso opposto, considera la struttura da grafo senza applicare la TCSMA\cite{Berliant2018};
        \footnotetext{In particolare \cite{Gualandi2019SDC}, nonostante sia un articolo molto interessante e che giunge a conclusioni altrettanto importanti, astrae eccessivamente le interazioni tra città oltre a non definire con sufficiente chiarezza che cosa s'intende per \textit{dintorni}.}
        \item altri vedono i nodi piú come luoghi ove vivono e interagiscono gli agenti, creando cosí un modello descritto da un sistema di equazioni di Boltzmann elevate e accoppiate\cite{Loy2021};
        \item infine l'articolo che piú si avvicina all'obbiettivo di questa tesi, applica tuttavia la sua teoria nel contesto delle reti sociali\cite{Nurisso2024}.
    \end{itemize}

    Questa lacuna è probabilmente attribuile al fatto che gli sviluppi della teoria dei grafi applicata alla TCSMA sono stati piuttosto recenti e principalmente concentrati sulla prospettiva nodo-luogo anziché nodo-agente. Dunque l'aspetto piú innovativo di questo lavoro è mostrare che la prospettiva nodo-agente è altrettanto valida quanto quella piú comune e recente del nodo-luogo; per il resto questa tesi dev'essere considerata come un'esplorazione formale applicativa con pochi approfondimenti analitici.

    Lo scritto sarà cosí suddiviso: nel secondo capitolo si affronteranno brevemente e [superficialmente] alcuni concetti della teoria dei grafi, soprattutto quelli pertinenti alla topologia interurbana; quindi nel terzo la TCSMA è approfondita prima da un punto di vista classico, e poi mediata dai grafi sia esattamente che approssimatamente; infine negli ultimi due si illustreranno i principali risultati concludendo con delle note finali e potenziali sviluppi futuri.
    
    Per il lettore interessato sarà anche presente un'appendice ove il codice di Python dell'implementazione è spiegato a grandi linee.