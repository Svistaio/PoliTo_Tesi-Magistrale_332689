
% !TEX root = ../Testa/Principale.tex

%\begingroup Tipografia
    \newcommand\dimTesto[1]{%
        \pgfmathparse{#1*1.2}%
        \fontsize{#1 pt}{\pgfmathresult pt}\selectfont%
    }

    \newcommand\daRivedere{{\color[HTML]{DC143C}\{§\}}}
    \newcommand\dubbio{{\color[HTML]{DC143C}\{?\}}}

    \newcommand\tref[2]{\hyperref[#1]{#2}}

    % Comando per definire citazioni con note separate da un punto e virgola: [citazione1, nota1; citazione2, nota2; ecc.]
    \DeclareMultiCiteCommand{\bracketcitesemi}[\mkbibbrackets]{\cite}{\addsemicolon\space}

    \crefrangeformat{equation}{(#3#1#4--#5#2#6)}
    \crefmultiformat{equation}{(#2#1#3}{, #2#1#3)}{, #2#1#3}{, #2#1#3)}
    \crefrangemultiformat{equation}{(#3#1#4--#5#2#6}{, #3#1#4--#5#2#6)}{, #3#1#4--#5#2#6}{, #3#1#4--#5#2#6)}
%\endgroup

%\begingroup TikZ
    \newcommand\didascalia[2]{%
        \captionsetup{width=#2} % aboveskip=-.2cm,
        % \addcontentsline{lof}{figure}{\protect\numberline{\thefigure}#1}%
        % \caption*{\figurename~\thefigure: #1}
        \captionof{figure}{#1}
    }
    
    \newcommand\sottoDidascalia[1]{%
        \refstepcounter{subfigure}%
        (\thesubfigure) #1%
    }
    
    \makeatletter
    \newcommand\includetikzgroupplotfigure
        {\@ifstar{\includetikzgroupplotfigure@star}
        {\includetikzgroupplotfigure@nostar}}
    
    \newcommand{\includetikzgroupplotfigure@nostar}[6]{%
        \begin{figure}[#1]%
            \centering%
            \refstepcounter{figure}%
            \resizebox{#2}{!}{#3}%
            \didascalia{#4}{#5}%
            \label{#6}%
        \end{figure}%
    }
    
    \newcommand\includetikzgroupplotfigure@star[6]{%
        \newgeometry{
            left=1cm,
            right=1cm,
            top=1cm,
            bottom=1cm
        }
        \begin{center}
            \vspace{#1}%
            % \refstepcounter{figure}%
            \resizebox{#2}{!}{#3}%
            \didascalia{#4}{#5}%
            \label{#6}%
        \end{center}
        % \begin{figure*}[#1]%
        %     \vspace{#2}%
        %     \centering%
        %     % \refstepcounter{figure}%
        %     \resizebox{#3}{!}{#4}%
        %     \didascalia{#5}{#6}%
        %     \label{#7}%
        % \end{figure*}%
        \restoregeometry
    }
    \makeatother
%\endgroup

%\begingroup Liste
    \setlist[enumerate]{
        topsep=0.5em,
        parsep=0em,
        itemsep=0.25em,
        leftmargin=2em,
        rightmargin=1.5em,
        % \leftmargin + \itemindent = \labelindent + \labelwidth + \labelsep
        %itemindent=!,
        %labelindent=30pt,
        %labelwidth=!,
        %labelsep=!,
    }

    \setlist[itemize]{
        topsep=0.5em,
        parsep=0em,
        itemsep=0.25em,
        leftmargin=2em,
        rightmargin=1.5em,
        % \leftmargin + \itemindent = \labelindent + \labelwidth + \labelsep
        %itemindent=!,
        labelindent=30pt,
        %labelwidth=!,
        %labelsep=!,
    }
%\endgroup

%\begingroup Matematica
    \newcommand\SpazMate[3]{ % Spaz[iatura] Mate[matica]
        \thinmuskip = #1mu\relax
        \medmuskip = #2mu\relax
        \thickmuskip = #3mu\relax
    }
    \SpazMate123

    \newcommand\de[1][]{\mathrm{d}#1}
    \newcommand\derS[3][d]{\frac{\mathrm{#1}#2}{\mathrm{#1}#3}}
    
    \DeclareMathOperator\DstrBernoulli{Bernoulli}
    \DeclareMathOperator\DstrGamma{Gamma}

    \theoremstyle{definition}
    \newtheorem{dfn}{Definizione}
    \newtheorem{dfnn}{Definizioni}

    \theoremstyle{plain}
    \newtheorem{ipo}{Ipotesi}
    \newtheorem{thm}{Teorema}
    \newtheorem{lmm}{Lemma}
    \newtheorem{crl}{Corollario}
%\endgroup

%\begingroup Frontespizio
    \newcommand\lineaPuntallata[1][4cm]{\makebox[#1]{\dotfill}}
    \newcommand\Frontespizio{
        \begin{titlepage}
            \newgeometry{
                left=1cm,
                right=1cm,
                top=3cm,
                bottom=3.5cm
            }
            \centering
            %
            {\Huge\scshape Politecnico di Torino}\\[1cm]
            {\Large Corso di Laurea\\in Ingegneria Matematica}
            %
            \vfill
            %
            {\Large Tesi di Laurea Magistrale}\\[0.5cm]
            \parbox{.9\textwidth}{
                \centering\huge\bfseries%
                Modellizzazione della distribuzione della popolazione tra città su reti spaziali mediante la teoria cinetica dei sistemi multiagente
            }\\[1.5cm]
            \includegraphics[width=0.25\textwidth]{../../Figure/Scelte/logo_PoliTO_2021.jpg}
            %
            \vfill
            %
            \begin{minipage}[t]{0.85\textwidth}
                \begin{flushleft}\large
                    \textbf{Relatori} \hfill \textbf{Candidato}\\
                    prof. Andrea Tosin \hfill Valerio Taralli\\
                    prof. Nome Cognome\\[0.5cm]
                    \textit{firma dei relatori} \hfill \textit{firma del candidato}\\[0.35cm]
                    \lineaPuntallata\ \hfill \\
                    \lineaPuntallata\ \hfill \lineaPuntallata
                \end{flushleft}
            \end{minipage}
            %
            \vfill
            %
            {\large Anno Accademico 2025-2026\par}
            %
            \restoregeometry
        \end{titlepage}
    }
%\endgroup

%\begingroup Dedica
    \newcommand\Dedica{
        \thispagestyle{empty}
        \phantomsection
        \pdfbookmark[1]{Dedica}{Dedica}
    
        \vspace*{3cm}
    
        \begin{center}
            \Large
            Ai miei genitori, \\
            \emph{Elisabetta} e \emph{Marco}
        \end{center}
    }
%\endgroup
