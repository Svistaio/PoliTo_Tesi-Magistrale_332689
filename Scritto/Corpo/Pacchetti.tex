
% !TEX root = ../Testa/Principale.tex

\newcommand\bozza{true}
% \newcommand\bozza{false}

\usepackage[
    pass,
    showframe=\bozza
]{geometry}

\usepackage[T1]{fontenc}
% \usepackage[utf8]{inputenc}
\usepackage[utf8]{inputenx}
\usepackage{microtype}

\usepackage[italian]{babel}

\usepackage{csquotes}
\usepackage[
    backend=biber,
    style=numeric-comp
]{biblatex}
\DeclareDelimFormat{finalnamedelim}{\addspace\&\space}
\addbibresource{../Corpo/Bibliografia.bib}
\nocite{*}

\usepackage[
    drafting=\bozza,
    eulerchapternumbers,
    % subfig,
    beramono,
    eulermath,
    pdfspacing,
    dottedtoc=true
]{classicthesis}
\usepackage{arsclassica}
\usepackage{scrhack}

\usepackage{tikz}
\usetikzlibrary{
    % external,
    positioning,
    shapes.geometric,
    plotmarks,
    arrows.meta,
    calc,
    decorations.pathreplacing,
    backgrounds,
    patterns.meta
} 
% \immediate\write18{if not exist "../../Figure/.fnl/TikZpdf" mkdir "../../Figure/.fnl/TikZpdf"}
% \tikzexternalize[prefix={../../Figure/.fnl/TikZpdf/}]
\usepackage{pgfplots}
\pgfplotsset{compat=newest}
\usepgfplotslibrary{
    patchplots,
    groupplots,
    colorbrewer
}
\usepackage{grffile}
\usepackage{nicematrix}

\usepackage{zref-savepos}
\newcounter{figposcount} % Crea un contatore unico per tracciare la posizione
\usepackage{xstring}
    
\usepackage{accents}
\usepackage{
    amsmath,
    amsfonts,
    amsthm,
    % amssymb
    bbold
}

\usepackage{
    graphicx,
    caption,
    subcaption,
    float,
    stfloats
}

\usepackage[
    italiano,
    italianokw,
    ruled,
    vlined,
    linesnumbered
]{algorithm2e}
% \usepackage{listings}

\usepackage[left]{lineno}
\usepackage[
    section,
    newfloat
]{minted}
\usepackage{newunicodechar}

\usepackage{enumitem}
\usepackage{ragged2e} 

% \usepackage{hyperref} % Già caricato da «classicthesis» colle impostazioni
\PassOptionsToPackage{hyphens}{url}
\hypersetup{
    colorlinks=true,
    linktocpage=true,
    breaklinks=true,
    urlcolor=CTurl,
    linkcolor=CTlink,
    citecolor=CTcitation,
    % destlabel=true % Migliora la precisione dei nessi ipertestuali
} % Si v. la linea 732 di «classicthesis.sty»
% \hypersetup{
%     colorlinks=false,
%     linkbordercolor=CTlink,
%     citebordercolor=CTcitation,
%     urlbordercolor=CTurl
%     linktocpage=true,
%     breaklinks=true,
%     pdfborder={1 1 1},
% }
% \usepackage[all]{hypcap}
\usepackage{xurl}

\usepackage[italian]{cleveref}

\usepackage{siunitx}
\sisetup{
    exponent-mode=threshold,
    exponent-thresholds=-3:3,
    round-mode=figures,
    round-precision=3,
    round-pad=false,
    separate-uncertainty=true,
    print-unity-mantissa=false,
    tight-spacing=true,
}

\usepackage[normalem]{ulem}
\usepackage{circledsteps}
