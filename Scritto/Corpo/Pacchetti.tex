
% !TEX root = ../Testa/Principale.tex

\usepackage[
    pass,
    showframe
]{geometry}

\usepackage[T1]{fontenc}
\usepackage[utf8]{inputenc}
\usepackage{microtype}

\usepackage[italian]{babel}

\usepackage[
    backend=biber,
    style=numeric-comp
]{biblatex}
\usepackage{csquotes}
\addbibresource{../Corpo/Bibliografia.bib}
\nocite{*}

\usepackage[
    drafting=true,
    eulerchapternumbers,
    % subfig,
    beramono,
    eulermath,
    pdfspacing,
    dottedtoc=true
]{classicthesis}
\usepackage{arsclassica}

\usepackage{tikz}
\usetikzlibrary{
    positioning,
    shapes.geometric,
    plotmarks,
    arrows.meta
}
\usepackage{pgfplots}
\pgfplotsset{compat=newest}
\usepgfplotslibrary{
    patchplots,
    groupplots
}
\usepackage{grffile}
\usepackage{nicematrix}

\usepackage{zref-savepos}
\newcounter{figposcount} % Crea un contatore unico per tracciare la posizione
\usepackage{xstring}
    
\usepackage{
    amsmath,
    amsfonts,
    amsthm,
    % amssymb
    bbold
}

\usepackage{
    graphicx,
    caption,
    subcaption,
    float,
    stfloats
}

% \usepackage[italiano,ruled]{algorithm2e}
\usepackage{
    algorithm,
    algpseudocode
}

\usepackage{enumitem}

% \usepackage{hyperref} % Già caricato da «classicthesis» colle impostazioni
\hypersetup{
    colorlinks=true,
    linktocpage=true,
    breaklinks=true,
    urlcolor=CTurl,
    linkcolor=CTlink,
    citecolor=CTcitation
} % Si v. la linea 732 di «classicthesis.sty»
% \hypersetup{
%     colorlinks=false,
%     linkbordercolor=CTlink,
%     citebordercolor=CTcitation,
%     urlbordercolor=CTurl
%     linktocpage=true,
%     breaklinks=true,
%     pdfborder={1 1 1},
% }

\usepackage[italian]{cleveref}

\usepackage[left]{lineno}

\usepackage[normalem]{ulem}
