
% !TEX root = ../Testa/Principale.tex

\usepackage[
    pass,
    showframe
]{geometry}

\usepackage[T1]{fontenc}
\usepackage[utf8]{inputenc}
\usepackage{microtype}

\usepackage[italian]{babel}

\usepackage[
    backend=biber,
    style=numeric-comp
]{biblatex}
\usepackage{csquotes}
\addbibresource{../Corpo/Bibliografia.bib}
\nocite{*}

\usepackage[
    drafting=true,
    eulerchapternumbers,
    subfig,
    beramono,
    eulermath,
    pdfspacing,
    dottedtoc=true
]{classicthesis}
\usepackage{arsclassica}

\usepackage{tikz}

\usepackage{amsmath,
            amsfonts,
            amsthm,
            amssymb}

\usepackage{graphicx}
\usepackage{subfig}

\usepackage{enumitem}

% \usepackage{hyperref} % Già caricato da «classicthesis» colle impostazioni
\hypersetup{
    colorlinks=true,
    linktocpage=true,
    breaklinks=true,
    urlcolor=CTurl,
    linkcolor=CTlink,
    citecolor=CTcitation
} % Si v. la linea 732 di «classicthesis.sty»
% \hypersetup{
%     colorlinks=false,
%     linkbordercolor=CTlink,
%     citebordercolor=CTcitation,
%     urlbordercolor=CTurl
%     linktocpage=true,
%     breaklinks=true,
%     pdfborder={1 1 1},
% }
\usepackage{cleveref}

\usepackage[left]{lineno}
